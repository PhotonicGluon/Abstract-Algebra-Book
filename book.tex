\documentclass[
    b5paper,
    pagesize,
    10pt,
    bibtotoc,
    normalheadings,
    twoside,
    openany,
    chapterprefix,
    DIV=9
]{scrbook}

%=============== Preamble ================
\setpartpreamble[u][\textwidth]{
    \quoteattr{
        Aus dem Paradies, das Cantor uns geschaffen, soll uns niemand vertreiben k\"{o}nnen.\\
        \textnormal{(No one shall expel us from the Paradise that Cantor has created.)}
    }
    {
        David Hilbert, 1926
    }
    {
        \cite[p.~170]{hilbert_1926}
    }

    A firm foundation is required to build the core of abstract algebra. Without such a foundation, the claims put forth in future parts would be unwarranted and unjustified. The proofs of those claims would also seem baffling to the uninitiated. This part provides these tools and techniques to understand the subject matter in later parts.

    We start with the absolute fundamentals: sets, logic, and proof writing. Sets are a fundamental object in mathematics and will be used countless times in abstract algebra. Then, in logic, introduce first-order logic notation and its meaning as an overview of how statements are formed in later parts. We also explore the properties of such statements and discuss numerous ways to prove them. We spend a lot of time on proof writing as we will employ various types of proofs in abstract algebra, and it is critical to understand why they are valid proofs and how they work.

    Although this book assumes an understanding of high-school algebra, it is helpful to recapitulate the symbols, terminology, and results used there. Granted, the use of such algebraic skills will be limited in group theory, but they will be critical when we move on to later parts where polynomials and algebraic manipulation are involved.

    Afterwards, we look at relations and functions. Although we do not go too much into detail into relations, we highlight the important properties of equivalence relations and link functions to relations. We note functions/maps is so widespread in abstract algebra that they deserve their own chapter.

    Elementary number theory and modular arithmetic will also often come up, as these fields are highly integrated within abstract algebra. Readers are advised to look at the notation of divisibility, Euclid's division lemma, and congruence modulo $n$ in particular. Other results are important but are not used as often.

    Unlike the other parts, not all results will have a proof. This is because most of these results are outside the scope of abstract algebra, and that their proofs can be easily found online. We only prove some results there, especially if the result is hard to prove or hard to find a proof online. Other parts will have full proofs for all results.
}
\part{Preliminaries}


%=========== Global Variables ============
\newcommand{\version}{0.19}

%=========== Paths to images =============
\graphicspath{
    {part0/images}
    {part1/images}
    {part2/images}
    {part3/images}
}

%========= Front Matter Pages ============
% Quote page
\newcommand{\quotepagetext}{
    TODO: ADD
}
\newcommand{\quotepageattribution}{NONE}
\newcommand{\quotepagecitation}{NONE}

% Suggestions of use
\newcommand{\interdependencenotes}{
    TODO: ADD
}

%==== Include only relevant chapters =====
\IfFileExists{\jobname.run.xml}
{
    \includeonly{
        front-matter,
        %
        % Part 0
        part0/algebra,
        part0/sets,
        part0/functions,
        part0/proofs,
        part0/number-theory,
        part0/modular-arithmetic,
        %
        % Part 1
        part1/intro-to-groups,
        part1/basics-of-groups,
        part1/subgroups,
        part1/homomorphisms-and-isomorphisms,
        part1/symmetry-groups,
        part1/products,
        part1/further-homomorphisms,
        part1/more-groups,
        part1/group-actions,
        part1/sylow-theorems,
        part1/composition-series,
        part1/simple-groups,
        %
        % Part 2
        part2/intro-to-rings,
        part2/basics-of-rings,
        part2/integral-domains,
        part2/ideals-and-quotient-rings,
        part2/ring-homomorphisms,
        part2/polynomial-rings,
        %
        % Appendices
        exercise-solutions,
        problem-solutions,
        image-acknowledgements
    }
}
{
    % Do a full document generation initially to generate all the aux files
}

%=========================================
\begin{document}

\frontmatter  % Use lowercase roman numerals for page numbers
% Define roman numerals for front matter
\makeatletter\newcommand*{\rom}[1]{\Ifstr{#1}{0}{0}{\expandafter\@slowromancap\romannumeral #1@}}\makeatother

% Half title page
\thispagestyle{empty}
\null\vspace{4cm}
\begin{raggedleft}
    {\fontsize{24pt}{0pt}\selectfont \textbf{A Complete Introduction To Abstract Algebra}}\\
\end{raggedleft}

% Title page
\begin{titlepage}
    \null\vspace{4cm}
    \begin{raggedleft}
        {\fontsize{20pt}{0pt}\selectfont \textbf{A Complete}\\\textbf{Introduction To}}\\

        \vspace{0.4cm}
        {\fontsize{48pt}{0pt}\selectfont \textbf{ABSTRACT}}\\
        \vspace{0.15cm}
        {\fontsize{48pt}{0pt}\selectfont \textbf{ALGEBRA}}\\

        \vspace{0.5cm}
        {\fontsize{16pt}{0pt}\selectfont Version \version}\\
        
        \vspace{1.25cm}
        {\fontsize{20pt}{0pt}\selectfont Kan Onn Kit}\\
    \end{raggedleft}
    \vspace*{\fill}
\end{titlepage}

\newpage{}

% Edition notice
\clearpage\null\vfill
\thispagestyle{empty}
\begin{minipage}[b]{0.9\textwidth}
    \footnotesize\raggedright
    \setlength{\parskip}{0.5\baselineskip}

    Published by Kan Onn Kit\\
    Singapore
    \vspace{5cm}

    \textbf{A Complete Introduction To Abstract Algebra}\par
    Version \version
    \vspace{0.3cm}

    Copyright \copyright \ 2022 -- \the\year\ by Kan Onn Kit\par
    This work is licensed under a
    Creative Commons Attribution-NonCommercial-ShareAlike 4.0 International Licence.\par
    \pdfteximg{2.5cm}{images/CC_BY-NC-SA_4.0.pdf_tex}\\
    The full licence text is available at \url{http://creativecommons.org/licenses/by-nc-sa/4.0/}.\par    
    The source files for the project are available at \url{https://github.com/PhotonicGluon/Abstract-Algebra-Book}.
    \vspace{0.3cm}

    Typeset in 10pt \TeX~Gyre Pagella using PDF\LaTeX.
\end{minipage}

\vspace*{2\baselineskip}
\cleardoublepage

% "Quote" page
\thispagestyle{empty}
\vspace*{2cm}

\begin{center}
    \Large{\parbox{10cm}{
        \begin{raggedright}
            \Large
            Symmetry is a vast subject, significant in art and nature. Mathematics lies at its root, and it would be hard to find a better one on which to demonstrate the working of the mathematical intellect.
            \vspace{0.3cm}
            
            \hfill
            --- Hermann Weyl, 1952\\
            \vspace{-0.25cm}
            
            \hfill
            \normalsize
            ({\cite[p.~145]{weyl_1952}})
        \end{raggedright}
    }
}
\end{center}

\newpage

% Table of contents
\createtoc

% Acknowledgements
\chapter{Acknowledgements}
Undertaking such a monumental project is new to me, and I am indebted to the people who accompanied me on this journey.

I am eternally grateful to my parents, who have spent countless hours and an ungodly amount of effort to raise me into who I am today. Their omnipresent kindness, patience, and love for me are something I certainly do not deserve, and I thank them for taking care of me.

I would like to thank my tutor Leong Chong Ming, who got me interested in abstract algebra in the first place. His enthusiasm and eagerness to share his knowledge on the subject is the driving force behind my decision to write these books.

I am grateful for the help of my friend Low Ji Yuan, who has assisted me with countless revisions of the content in these books and given me another pair of eyes in the vetting of content.

I also sincerely appreciate the continued support from my mathematics tutors, Loke Weng Heng, Siow Yun Jie, and Teng Yen Ping, who have been there through my junior college years inspiring me with the wonders of mathematics. I am indebted to them for allowing me to excel in my final examinations.

My close friends, Aidan Tay, Gabriel Fong, and Low Ji Yuan, have accompanied me through two years of schooling (and math jokes). I offer infinite thanks to them for sticking with me and for encouraging this math nerd to pursue his wacky projects.

A thousand thanks go out to my teachers at the School of Science and Technology, Singapore, and specifically my form teacher Lee Tsi Yew Samuel, who instilled important character values into me so I can excel in my future endeavours.

% Preface
\chapter{Preface}
Although algebra has a long history, it has undergone quite striking changes in the past few decades. Abstract (or modern) algebra is widely recognised as an essential element of higher mathematical education. However, the results that it showcases are often hard to grasp and understand without prerequisite knowledge or a heavy background in mathematics. Most books on this subject are crafted for undergraduates at universities. They are not for a general mathematics enthusiast or one who seeks to understand more about the inner structure of algebra that mathematicians encounter frequently.

The exploration of such structures is fundamental to the current underpinning of scientific inquiries. For example, groups are important as they describe the symmetries which the laws of physics seem to obey. Finite fields are also used in coding theory and combinatorics. I hope this series of books will inspire more people to learn more about abstract algebra, beyond the simple introduction presented here.

In addition, I find that, in most textbooks, important details are left for the reader to figure out independently, without providing any additional guidance or help along the way. Exercises and problems are provided for topics taught in abstract algebra, but only a handful have written solutions provided, leaving readers unsure of the correctness of their answers. I believe that the completeness of a textbook is essential; no claim made should be without justification (unless absolutely necessary). These books offer a more complete picture of abstract algebra by providing full-worked solutions to all exercises and problems posed.

This series of books serves to achieve several goals.
\begin{itemize}
    \item Provide a step-by-step explanation of core results from abstract algebra, without ambiguity of the results discussed.
    \item Demystify the core steps that many textbooks gloss over when proving results or when writing the solutions to problems/exercises.
    \item Ensure that results from abstract algebra are as accessible, as approachable, and as understandable for as many people as possible.
\end{itemize}
I hope that these books can accomplish these goals and let readers enjoy the wonders of abstract algebra.

\hfill{\textit{24 September, 2023}}

% Suggestions on the use of this book
\chapter{Suggestions on the Use of This Book}
\section*{General Information}
\begin{itemize}
    \item We include both exercises and problems.
    \begin{itemize}
        \item An exercise can be thought of as a simple ``self-review'' question. Exercises ensure that the content of a particular section is understood and should not be too hard to answer.
        \item A problem is a more holistic version of an exercise. Generally, solutions to problems require a thorough understanding of the current chapter and may require results from other chapters.
    \end{itemize}
    \item A consistent labelling system for all the results within and between parts is necessary for a project as long as this one.
    \begin{itemize}
        \item All definitions, axioms, examples, lemmas, theorems, propositions, and corollaries are consecutively numbered, using the format
        \begin{quote}
            \code{[CHAPTER].[SECTION].[NUMBER]}
        \end{quote}
        For example, the fourth statement in chapter 2, section 3 is labelled \textbf{2.3.4}.
        \item Exercises and problems are also numbered consecutively, using the format
        \begin{quote}
            \code{[CHAPTER].[NUMBER]}
        \end{quote}
        For example, the third exercise in chapter 2 is labelled \textbf{2.3}. Likewise, the fourth exercise in chapter 3 is labelled \textbf{3.4}.
    \end{itemize}
    \item The symbol ``$\qedsymbol$'' marks the end of a proof.
\end{itemize}

\section*{Chapter Interdependence}
The diagram on the next page shows chapter interdependence. It should be used in conjunction with the table of contents and notes listed.

\newpage
\begin{center}
    \pdfteximg{\linewidth}{images/interdependence.pdf_tex}
\end{center}

\newpage

\textbf{Notes}:
\begin{itemize}
    \item Each part is largely independent from the other parts. However, part 0 is required reading for future parts of the book.
    \item Part 1 on group theory can be thought of as the fundamentals of abstract algebra.
    \begin{itemize}
        \item Chapter 8 is essentially independent from the rest of the other chapters. It provides motivation for the axioms of groups, but readers who want to skip this introduction can move straight to chapter 9.
        \item Chapters 9, 10, and 11 are considered to be the essentials of group theory.
        \item Chapter 11 is required reading for chapters 12, 13, 14, and 16.
        \item Chapter 14 only requires knowledge of the subgroup product from chapter 13 (specifically \myref{definition-subgroup-product} and \myref{prop-subgroup-product-is-subgroup}); otherwise these two chapters are relatively independent.
        \item Chapter 15 assumes knowledge of chapter 14, and knowledge of permutations and the symmetric group from chapter 12.
        \item Usually, group actions (chapter 16) would be read after the essentials of group theory; therefore chapter 16 could be read after chapter 11.
        \item Chapter 17 only requires chapters 14 and 16.
        \item Chapter 18 only require results from chapter 14, except for \myref{problem-S4-composition-series} which uses the alternating group introduced in chapter 15.
        \item Chapter 19 assumes full knowledge of chapter 14; minor results from chapter 15 (specifically, the alternating group), chapter 17 (the Third Sylow Theorem, \myref{thrm-sylow-3}), and chapter 18 (\myref{problem-S4-composition-series}) are required.
    \end{itemize}
    \item Part 2 on ring theory builds on ideas of group theory.
    \begin{itemize}
        \item Chapters 9 and 10 from group theory are required reading before continuing with ring theory.
        \item TODO: Add ring theory interdependence.
    \end{itemize}
\end{itemize}


\setcounter{part}{-1}  % So that the first part starts with 0
\mainmatter  % Now use arabic numerals for page numbers

%=========================================
\setpartpreamble[u][\textwidth]{
    \quoteattr{
        Aus dem Paradies, das Cantor uns geschaffen, soll uns niemand vertreiben k\"{o}nnen.\\
        \textnormal{(No one shall expel us from the Paradise that Cantor has created.)}
    }
    {
        David Hilbert, 1926
    }
    {
        \cite[p.~170]{hilbert_1926}
    }

    To understand the subject material covered in the later parts, the preliminaries and fundamentals must first be understood. This part gives one sufficient knowledge to dive into the meat of abstract algebra. We cover important facts about algebra, basic set theory, functions and maps, mathematical logic and proof writing, elementary number theory, and simple modular arithmetic, which should be plenty for one to understand the content covered in future parts.
}
\part{Preliminaries}
\chapter{Algebra}
Competent algebraic manipulation in the real numbers is assumed. We collect a few important and useful concepts and results when we manipulate algebraic objects.

\section{Addition and Multiplication}
Addition and multiplication in the real numbers are fundamental operations when we want to perform algebraic manipulations. We formally state some of their properties here. We note that these properties are really hard to prove directly since they are so ingrained in calculation; although a proof exists, it is too hard to reproduce them here. Thus we leave these properties as axioms: statements that are assumed to be true.

We first look at two important properties of addition over the real numbers.
\begin{axiom}\label{axiom-addition-is-commutative}\index{axiom!of commutative addition}
    Addition is commutative. That is, for any real numbers $x$ and $y$ we have $x + y = y + x$.
\end{axiom}
\begin{example}
    Clearly adding 3 to 4 (i.e., $3 + 4$) is the same as adding 4 to 3 (i.e., $4 + 3$) through our everyday intuition of adding numbers together.
\end{example}

\begin{axiom}\label{axiom-addition-is-associative}\index{axiom!of associative addition}
    Addition is associative. That is, for any real numbers $x$, $y$, and $z$ we have $x+(y+z) = (x+y)+z$.
\end{axiom}
\begin{example}
    We may evaluate the expression $5+(6+7)$ in two ways.
    \begin{itemize}
        \item We may first evaluate $6+7 = 13$, and then sum 5 to it to get $5 + 13 = 18$.
        \item Alternatively, we may use \myref{axiom-addition-is-associative} to rewrite $5+(6+7)$ to be $(5+6)+7$. Then we first evaluate $5+6 = 11$ and then add 7 to it, yielding $11 + 7 = 18$.
    \end{itemize}
\end{example}

Combining the two axioms for addition gives us multiple ways to evaluate expressions.
\begin{example}
    Consider the expression $(3 + 4) + (1 + 2)$.
    \begin{itemize}
        \item We could directly compute the values $3 + 4 = 7$, $1 + 2 = 3$, and then taking their sum to be $7 + 3 = 10$.
        \item Alternatively, via commutativity of addition (\myref{axiom-addition-is-commutative}) we can rewrite $(3 + 4) + (1 + 2) = (1+2)+(3+4)$ and then use associativity (\myref{axiom-addition-is-associative}) to further see that $(1+2)+(3+4) = ((1+2)+3)+4$. We can then compute the sum inside out, $1 + 2 = 3$, $3 + 3 = 6$, and finally $6 + 4 = 10$.
    \end{itemize}
\end{example}
\begin{example}
    Consider the expression $(x^4 + x^2) + (x^3 + x + 1)$. The two axioms above allow us to rewrite this as a nicer expression of $x^4 + x^3 + x^2 + x + 1$.
\end{example}

We now look at two properties of multiplication over the real numbers.
\begin{axiom}\label{axiom-multiplication-is-commutative}\index{axiom!of commutative multiplication}
    Multiplication is commutative. That is, for any real numbers $x$ and $y$ we have $x \times y = y \times x$.
\end{axiom}
\begin{remark}
    This also applies to subsets of the real numbers, such as the integers.
\end{remark}
\begin{remark}
    We may also denote multiplication using the centred dot ($\cdot$), or suppress the operator. That is, $x\times y = x\cdot y = xy$.
\end{remark}
\begin{example}
    Clearly multiplying 3 by 4 (i.e., $3 \times 4$) is the same as multiplying 4 by 3 (i.e., $4 \times 3$) through our everyday intuition of adding numbers together.
\end{example}

\begin{axiom}\label{axiom-multiplication-is-associative}\index{axiom!of associative multiplication}
    Multiplication is associative. That is, for any real numbers $x$, $y$, and $z$ we have $x\times(y\times z) = (x\times y)\times z$.
\end{axiom}
\begin{example}
    We may evaluate the expression $5\times(6\times7)$ in two ways.
    \begin{itemize}
        \item We may first evaluate $6\times7 = 42$, and then multiply 5 to it to get $5 \times 42 = 210$.
        \item Alternatively, we may use \myref{axiom-multiplication-is-associative} to rewrite $5\times(6\times7)$ to be $(5\times6)\times7$. Then we first evaluate $5\times6 = 30$ and then multiply 7 to it, yielding $30\times7 = 210$.
    \end{itemize}
\end{example}

Again, using the two axioms for multiplication gives us several ways to evaluate expressions.
\begin{example}
    Consider the expression $(3\times4)(1\times2)$.
    \begin{itemize}
        \item We could directly compute the values $3 \times 4 = 12$, $1 \times 2 = 2$, and then taking their product to be $12\times2 = 24$.
        \item Alternatively, via commutativity of multiplication (\myref{axiom-multiplication-is-commutative}) we can rewrite $(3\times4)(1\times2)=(1\times2)\times(3\times4)$ and then use associativity (\myref{axiom-multiplication-is-associative}) to further see that $(1\times2)\times(3\times4) = ((1\times2)\times3)\times4$. We can then compute the sum inside out, $1 \times 2 = 2$, $2 \times 3 = 6$, and finally $6 \times 4 = 24$.
    \end{itemize}
\end{example}
\begin{example}
    Consider the expression $(x^4\times x^2)(x^3 \times x \times 1)$. The two axioms above allow us to rewrite this as a nicer expression of $x^4 \times x^3 \times x^2 \times x \times 1 = x^{10}$.
\end{example}

We also note one property of combining addition and multiplication together.
\begin{axiom}\label{axiom-distributivity}\index{axiom!of distributivity}
    Multiplication distributes over addition. That is, for any real numbers $x$, $y$, and $z$, we have
    \begin{itemize}
        \item $x\times(y+z) = (x\times y) + (x\times z)$; and
        \item $(x+y)\times z = (x\times z) + (y\times z)$.
    \end{itemize}
\end{axiom}

\begin{example}
    Consider the expression $4\times(5+6)$.
    \begin{itemize}
        \item We could first evaluate $5+6=11$ and then multiply 4 by that to yield $4\times11=44$.
        \item Alternatively, we can use \myref{axiom-distributivity} to write $4\times(5+6) = (4\times5)+(4\times6)$. Then we can evaluate each part separately, seeing that $4\times5 = 20$, $4\times6 = 24$, so $(4\times5)+(4\times6) = 20+24=44$.
    \end{itemize}
\end{example}
\begin{example}
    Consider the expression $(x+1)(x+2)$. We use the axioms to expand the expression.
    \begin{align*}
        (x+1)(x+2) &= ((x+1)x) + ((x+1)\times2) & (\myref{axiom-distributivity})\\
        &= ((x\times x) + (1\times x)) + ((x\times2) + (1\times2)) & (\myref{axiom-distributivity})\\
        &= (x^2+x) + (2x+2)\\
        &= x^2+(x + (2x+2)) & (\myref{axiom-addition-is-associative})\\
        &= x^2+((x + 2x)+2) & (\myref{axiom-addition-is-associative})\\
        &= x^2 + (3x+2)\\
        &= x^2 + 3x + 2.
    \end{align*}
\end{example}

Clearly, these axioms have been ingrained into us whenever we did algebra. We simply state these axioms here whenever we need to use them to prove that a certain structure has a certain property that we need.

\section{Summation}
When we are summing many similar terms, we use the capital Greek letter sigma ($\sum$) to represent the sum of similar terms. 
\begin{definition}\index{summation}
    Define
    \[
        \sum_{i=m}^{n}a_i = a_m + a_{m+1} + a_{m+2} + \cdots + a_{n-1} + a_n,
    \]
    where $i$ is the \textbf{index of summation}\index{summation!index}, $a_i$ represents each \textbf{summand}\index{summation!summand} of the sum, $m$ is the \textbf{lower bound of summation}\index{summation!lower bound}, and $n$ is the \textbf{upper bound of summation}\index{summation!upper bound}.
\end{definition}
\begin{remark}
    $\displaystyle \sum_{i=m}^{n}a_i$ is read as ``the sum of $a_i$ from $i=m$ to $n$''.
\end{remark}
\begin{remark}
    The ``$i = m$'' at the bottom of the summation means that the index of summation starts at $m$, increments by 1 for each successive term, and continues until $i = n$.
\end{remark}

\begin{example}
    The sum of the first 10 positive integers can be written as
    \[
        \sum_{i=1}^{10}i = 1 + 2 + \cdots + 9 + 10
    \]
    and is evaluated to 55.
\end{example}

\begin{example}
    The sum of the squares from 3 to 9 can be written as
    \[
        \sum_{i=3}^{9}i^2 = 3^2 + 4^2 + \cdots + 8^2 + 9^2.
    \]
    There is no reason to just use ``$i$'' for the index; we may also use other letters, such as $j$:
    \[
        \sum_{j=3}^{9}j^2 = 3^2 + 4^2 + \cdots + 8^2 + 9^2.
    \]
\end{example}

\begin{example}
    The case where the summation has one summand is just equal to the summand itself. That is,
    \[
        \sum_{i=k}^{k}a_i = a_k.
    \]
    For instance,
    \[
        \sum_{i=4}^4(i+2)^2 = (4+2)^2 = 36.
    \]
\end{example}

There is one degenerate case for summation, which we note as a definition.
\begin{definition}
    Let $k$ and $n$ be integers such that $k > n$. Define
    \[
        \sum_{i=k}^{n}a_i = 0.
    \]
\end{definition}
\begin{example}
    We see
    \[
        \sum_{i=5}^{4}i = \sum_{i=3}^{2}1 = \sum_{i=12}^{3}(i^4 + 2i^3 + 3i^2 + 4i + 5) = 0
    \]
    by virtue of the above definition.
\end{example}
\begin{exercise}
    Evaluate the following sums.
    \begin{multicols}{3}
        \begin{partquestions}{\alph*}
            \item $\displaystyle \sum_{i=3}^{5}(7ix+11)$
            \item $\displaystyle \sum_{x=3}^{5}(7ix+11)$
            \item $\displaystyle \sum_{i=-3}^{-5}(-7ix-11)$
            \item $\displaystyle \sum_{i=4}^{8}ijk$
            \item $\displaystyle \sum_{j=4}^{8}ijk$
            \item $\displaystyle \sum_{n=4}^{8}ijk$
            \item $\displaystyle \sum_{i=3}^{7}13$
            \item $\displaystyle \sum_{i=1}^{3}i^2 + \sum_{j=4}^{6}j^2$
            \item $\displaystyle \sum_{i=1}^{3}\left(\sum_{j=5}^{7}(i+j)\right)$
        \end{partquestions}
    \end{multicols}
\end{exercise}

\newpage

We note some properties of summation.
\begin{itemize}
    \item $\displaystyle \sum_{i=m}^na_i = \sum_{j=m}^na_j$\hfill(Index renaming)
    \item $\displaystyle \sum_{i=m}^n(Ca_i) = C\sum_{i=m}^na_i$\hfill(Factor constants out of a sum)
    \item $\displaystyle \left(\sum_{i=m}^na_i\right) \pm \left(\sum_{i=m}^nb_i\right) = \sum_{i=m}^n(a_i \pm b_i)$\hfill(Break sum across sum or difference)
    \item $\displaystyle \sum_{i=m}^ka_i = \left(\sum_{i=m}^na_i\right) + \left(\sum_{j={n+1}}^ka_j\right)$\hfill(Splitting the sum)
    \item $\displaystyle \sum_{i=m}^na_i = \sum_{i=m+k}^{n+k}a_{i-k}$\hfill(Index shift)
\end{itemize}

\begin{exercise}
    Given $\displaystyle \sum_{i=1}^5a_i^2 = 100$, $\displaystyle \sum_{i=2}^5a_{i-1} = 10$, and $\displaystyle \sum_{i=1}^5(a_i+2)^2 = 200$, what is $a_5$?
\end{exercise}

\section{The Binomial Theorem}
Sometimes we are forced to expand expressions such as $(x-1)^5$, $(3x^2 + 5)^4$, and $(7x - 3)^9$. These expressions can be readily expanded using the \textbf{binomial theorem}\index{Bionomial Theorem}.
\begin{theorem}[Binomial Theorem]\label{thrm-binomial}
    Let $n$ be a non-negative integer. Then
    \[
        (x+y)^n = \sum_{k=0}^n {n \choose k}x^ky^{n-k} = \sum_{k=0}^n {n \choose k}x^{n-k}y^k
    \]
    where
    \[
        {n \choose k} = \frac{n!}{k!(n-k)!}.
    \]
\end{theorem}
We note that ${n \choose k}$ is read as ``$n$ choose $k$'' and is known as the \textbf{binomial coefficient}\index{binomial coefficient}.

\begin{example}
    $(x+1)^4 = x^4 + 4x^3 + 6x^2 + 4x + 1$.
\end{example}
\begin{example}
    $(x-1)^5 = x^5 - 5x^4 + 10x^3 - 10x^2 + 5x - 1$.
\end{example}
\begin{example}
    $(3x^2 + 5)^4 = 81x^8 + 540x^6 + 1350x^4 + 1500x^2 + 625$.
\end{example}
\begin{exercise}
    Find the coefficient of $x^6$ in $(7x-3)^9$.
\end{exercise}

We note two facts about the binomial coefficient here.
\begin{proposition}
    ${n\choose k} = {n\choose {k-n}}$ for all integers $0 \leq k \leq n$.
\end{proposition}
% \begin{proof}
%     One sees clearly that
%     \begin{align*}
%         {n\choose k} &= \frac{n!}{k!(n-k)!}\\
%         &= \frac{n!}{(n-k)!k!}\\
%         &= \frac{n!}{(n-k)!(n-(n-k))!}\\
%         &= {n\choose {k-n}}
%     \end{align*}
%     which proves this proposition.
% \end{proof}
\begin{proposition}\label{prop-binomial-coefficient-multiple-of-n}
    $n\choose k$ is a multiple of $n$ for all integers $1 \leq k \leq n - 1$.
\end{proposition}

\chapter{Sets}
Sets may be used to describe all of mathematics. All different kinds of mathematical structures may be described and explained using the notion of sets.

\section{What Are Sets?}
\begin{definition}
    A \textbf{set}\index{set} is a collection of things called \textbf{elements}\index{element} of the set.
\end{definition}
If $x$ is an element of the set $S$, we write $x \in S$. This is read as ``$x$ is an element of the set $S$'' or ``$x$ is in $S$''. Otherwise, we write $x \notin S$. For convenience, if $x \in S$ and $y \in S$, we may write $x, y \in S$. If $z \in S$ also, we may write $x, y, z \in S$. The same applies for ``$\notin$''.

A set is often described by listing elements separated by commas, or by a characterizing property of its elements, within braces $\{ \ \}$. 
\begin{example}
    The collection $\{2, 3, 4, 5\}$ is a set with 4 elements, namely 2, 3, 4, and 5.
\end{example}
\begin{example}
    The collection $\{\{1, 2, 3\}, \{4, 5\}, \{6, \{7\}\}\}$ is a set with 3 elements, which are also sets. Namely, it contains the set $\{1, 2, 3\}$, containing the elements 1, 2, and 3, the set $\{4, 5\}$, containing the elements 4 and 5, and the set $\{6, \{7\}\}$, containing the number 6 and the set $\{7\}$ which contains a single element 7.
\end{example}

Some sets may have infinitely many elements.
\begin{example}
    The integers $\{\dots, -3, -2, -1, 0, 1, 2, 3, \dots\}$ is a set with infinitely many elements. The dots indicate that the pattern of integers goes on forever in both directions.
\end{example}

\begin{definition}
    A set with a finite number of elements is called a \textbf{finite set}\index{set!finite}. A set with an infinite number of elements is called an \textbf{infinite set}\index{set!infinte}.
\end{definition}

\begin{definition}\index{set!equality}
    Two sets are equal if and only if they contain the same elements.
\end{definition}
\begin{example}
    The sets $A = \{1, 2, 3, 4\}$, $B = \{4, 3, 2, 1\}$, and $C = \{3, 4, 1, 2\}$ are equal to each other even though their elements are listed in a different order.
\end{example}
\begin{example}
    $\{1, 2, 3, 4\} \neq \{1, 2, 3, 5\}$ since the elements in the two sets differ.
\end{example}

What if a set has no elements?
\begin{definition}
    The \textbf{empty set}\index{set!empty} is the set $\{\}$ and is denoted by $\emptyset$. That is, $\emptyset = \{\}$.
\end{definition}

\newpage

We introduce the idea of subsets.
\begin{definition}
    Let $A$ and $B$ be sets.
    \begin{itemize}
        \item $A$ is a \textbf{subset}\index{subset} of $B$ if and only if all elements of $A$ are elements of $B$. This is denoted by $A \subseteq B$.
        \item $A$ is a \textbf{proper subset}\index{subset!proper} if and only if $A$ is a subset of $B$ and $B$ contains at least one element not in $A$. In this case, we write $A \subset B$.
    \end{itemize}
\end{definition}

\begin{example}
    Let $A = \{1, 2\}$, $B = \{1, 2, 3\}$, $C = \{1, 4\}$, and $S = \{1, 2, 3\}$. Then $A \subseteq S$ since the elements of $A$, namely 1 and 2, also appear in $S$. Also $B \subseteq S$ since all elements of $B$ appear in $S$. However $C \not\subseteq S$ since 4 is not an element of $S$.

    We note that $B \not\subset S$ since $S$ does not contain an element that is not in $B$. But $A \subset S$ since 3 is not in $A$.
\end{example}

\begin{exercise}
    Let $S$ be a non-empty set. Determine whether the following statements are true or false.

    \begin{multicols}{2}
        \begin{partquestions}{\alph*}
            \item $\{1, 2\} \subset \{1, 2, 3, 4\}$
            \item $\{1, 2, 3\} \subseteq \{1, 2, 4\}$
            \item $\emptyset \subseteq \emptyset$
            \item $S \subset S$
            \item $S \in \{S, \emptyset\}$
            \item $\{S\} \notin \{S, \emptyset\}$
            \item $S \subseteq \{S, \emptyset\}$
            \item $\{S\} \subseteq \{S, \emptyset\}$
        \end{partquestions}
    \end{multicols}
\end{exercise}

There are some special sets that are so common that we have given special names and symbols.
\begin{definition}
    The set of positive integers\index{set!of positive integers} is denoted $\mathbb{N}$.
\end{definition}
\begin{remark}
    Some authors denote the set containing the positive integers and 0 by $\mathbb{N}$. We use $\mathbb{N}$ to exclusively denote the positive integers here.
\end{remark}
\begin{definition}
    The set of integers \index{set!of integers} is denoted $\Z$.
\end{definition}
\begin{remark}
    The use of the blackboard ``Z'' (i.e., $\Z$) to denote the set of integers comes from the German ``Z\"{a}hlen'' which means ``numbers'' and is attributed to David Hilbert.
\end{remark}
\begin{definition}
    The set of rational numbers\index{set!of rational numbers} is denoted $\Q$.
\end{definition}
\begin{definition}
    The set of real numbers\index{set!of real numbers} is denoted $\R$.
\end{definition}

\newpage

\section{Set Operations}
Certain operations could be performed on sets. We list the most commonly used ones here.

\begin{definition}
    The \textbf{union}\index{set!union} of two sets is the set of all objects that are a member of $A$, or $B$, or both. It is denoted $A \cup B$.
\end{definition}
\begin{example}
    $\{1, 2, 3\} \cup \{2, 3, 4\} = \{1, 2, 3, 4\}$.
\end{example}

\begin{definition}
    The \textbf{intersection}\index{set!intersection} of two sets is the set of all objects that are a member of \textit{both} the sets $A$ and $B$. It is denoted $A \cap B$.
\end{definition}
\begin{example}
    $\{1, 2, 3\} \cap \{2, 3, 4\} = \{2, 3\}$
\end{example}

\begin{definition}
    The \textbf{set difference}\index{set!difference} of $S$ and $A$, denoted $S \setminus A$, is the set of all members of $S$ that are not members of $A$.
\end{definition}
\begin{remark}
    Some authors will use the minus sign to denote the set difference, i.e. $A - B$ denotes the difference of $A$ and $B$ and is the same as $A \setminus B$ in this book.
\end{remark}
\begin{example}
    Let $A = \{1, 2, 3\}$ and $B = \{2, 3, 4\}$. Then $A \setminus B = \{1\}$ and $B \setminus A = \{4\}$.
\end{example}

\begin{definition}
    The \textbf{Cartesian Product}\index{Cartesian Product} of $A$ and $B$, denoted $A \times B$, is the set whose members are all possible ordered pairs $(a, b)$, where $a$ is an element of $A$ and $b$ is an element of $B$.
\end{definition}
\begin{example}
    Let $A = \{1, 2, 3\}$ and $B = \{4, 5\}$. Then
    \[
        A \times B = \{(1, 4), (1, 5), (2, 4), (2, 5), (3, 4), (3, 5)\}.
    \]
\end{example}
\begin{remark}
    In particular, if $A$ is a set, the Cartesian product $A \times A = A^2$, $A\times A \times A = A^3$, and so on.
\end{remark}

\begin{exercise}
    Let $S = \{1, 2, 3, 4\}$, $T = \{2, 3, 5\}$, $U = \{(2, 2), (3, 3), (5, 5)\}$. Determine whether the following statements are true or false.
    \begin{multicols}{2}
        \begin{partquestions}{\alph*}
            \item $S \cup T = \{1, 2, 3, 4, 5\}$
            \item $S \cup U = \{1, 2, 3, (5, 5)\}$
            \item $S \cap T = \{2, 3\}$
            \item $T \cap U = \emptyset$
            \item $S \setminus T = \{1, 4\}$
            \item $S \setminus \{1, 4\} = T$
            \item $T^2 = U$
            \item $U \subset (S \cup T)^2$
        \end{partquestions}
    \end{multicols}
\end{exercise}

\newpage

\section{Set-Builder Notation}
Sometimes, some sets are too big or complex to list between the braces. In these cases, we use set-builder notation to describe the sets.
\begin{definition}
    A set $S$ with \textbf{set-builder notation}\index{set!builder notation} has the syntax
    \[
        S = \{\mathrm{expression} \ | \ \mathrm{rule}\},
    \]
    where the elements of $S$ are all values of the expression that satisfy the rule.
\end{definition}

\begin{example}
    Consider the set of even integers, $E = \{\dots, -6, -4, -2, 0, 2, 4, 6, \dots\}$. It is often written as
    \[
        E = \{2n \vert n \in \mathbb{Z}\}.
    \]
    In this case, we may read the above expression as ``$E$ is the set of all things of form $2n$, where $n$ is an element of $\mathbb{Z}$''.

    There are equivalent forms of $E$:
    \[
        E = \{n \vert n \textrm{ is an even integer}\} = \{n \vert n = 2k, k \in \mathbb{Z}\}.
    \]

    Another common way of writing $E$ is
    \[
        E = \{n \in \mathbb{Z} \vert n \textrm{ is even}\}
    \]
    where it could be read as ``$E$ is the set of all $n$ in $\mathbb{Z}$ such that $n$ is even''.
\end{example}
\begin{remark}
    Some authors use the colon instead of a vertical line, for example they may write $S = \{\mathrm{expression} \ : \ \mathrm{rule}\}$.
\end{remark}

\begin{example}
    The set $S = \{x \in \mathbb{R} \ | \ x \geq 0 \}$ is the set of all non-negative real numbers.
\end{example}

We introduce the notation for intervals.
\begin{definition}
    An \textbf{interval}\index{interval} is a subset of the real numbers. In particular, given real numbers $a$ and $b$ where $a \leq b$:
    \begin{align*}
        (a,b) &= \{x \in \mathbb{R} \vert a < x < b\} & [a,b] &= \{x \in \mathbb{R} \vert a \leq x \leq b\}\\
        (a,b] &= \{x \in \mathbb{R} \vert a < x \leq b\} & [a,b) &= \{x \in \mathbb{R} \vert a \leq x < b\}.
    \end{align*}

    Infinity ($\infty$) can also be used as one of the bounds to denote an \textbf{unbounded interval}. In particular, for any real number $r$:
    \begin{align*}
        (r, \infty) &= \{x \in \mathbb{R} \vert x > r\} & [r, \infty) &= \{x \in \mathbb{R} \vert x \geq r\}\\
        (-\infty,r) &= \{x \in \mathbb{R} \vert x < r\} & (-\infty,r] &= \{x \in \mathbb{R} \vert x \leq r\}.
    \end{align*}
\end{definition}
\begin{example}
    The interval $I = [2, 5)$ is the set $\{x \in \mathbb{R} \vert 2 \leq x < 5\}$. Note $1 \notin I$, $2 \in I$, $3 \in I$, $4 \in I$, $5 \notin I$, and $6 \notin I$.
\end{example}
\begin{example}
    The interval $I = (2, \infty)$ is the set $\{x \in \mathbb{R} \vert x > 2\}$. Note $1 \notin I$, $2 \notin I$, $3 \in I$, and $\pi \in I$.
\end{example}

\section{Cardinality}
\begin{definition}
    The \textbf{cardinality}\index{cardinality} of a set $S$, denoted by $|S|$, is a measure of the number of elements of the set.
    \begin{itemize}
        \item If $S$ is finite, then $|S|$ is the number of elements in $S$.
        \item If $S$ is infinite, then we write $|S| = \infty$.
    \end{itemize}
\end{definition}
\begin{remark}
    Of course, the notion that $|S| = \infty$ is poorly defined in other contexts. However, for this book, this definition would be sufficient for most of the things we wish to accomplish.
\end{remark}
\begin{example}
    The set $A = \{1, 2, 3\}$ has cardinality 3, i.e. $|A| = 3$.
\end{example}
\begin{example}
    The empty set $\emptyset$ has cardinality 0 since it has no elements, i.e. $|\emptyset| = 0$.
\end{example}
\begin{example}
    The set of integers has infinite elements, so we write $|\mathbb{Z}| = \infty$.
\end{example}

\begin{exercise}
    Let the sets
    \begin{align*}
        S &= \{x \in \mathbb{Q} \vert x \in (-\infty, 0]\}\\
        T &= \{y \in \mathbb{Z} \vert y \in [-2, 10] \text{ and } y \text{ is an even number} \}
    \end{align*}
    List the elements in the set $S \cap T$.
\end{exercise}

\newpage

\section{Problems}
\begin{problem}
    Let $A$ and $B$ be finite sets with cardinality $n$. For what value(s) of $n$ can we be sure that $A = B$?
\end{problem}

\begin{problem}
    Let the sets
    \begin{align*}
        A &= \{x \in \R \vert x^2 - x - 2 \leq 0\},\\
        B &= \{x \in [0, \infty) \vert 12 - x - x^2 > 0\}.
    \end{align*}
    \begin{partquestions}{\alph*}
        \item Express $A \cap B$ in interval notation.
        \item Express $A \cup B$ in interval notation.
        \item Express $B \setminus A$ in interval notation.
        \item Is $A \setminus B \subset [-1, 0)$? Explain.
    \end{partquestions}
\end{problem}

\begin{problem}
    Let the sets
    \begin{align*}
        A &= \{(x,y) \in \Z^2 \vert 5x+2y+3=0\},\\
        B &= \{(x,y) \in [0,1]^2 \vert 2x^2+5x+2y+1=0\}.
    \end{align*}
    Find the cardinality of the following sets. If the cardinality is finite, list all elements of the set.
    \begin{partquestions}{\alph*}
        \item $A \cap B$
        \item $A \cup B$
    \end{partquestions}
\end{problem}

\begin{problem}
    Show that $A \cap (B \setminus C) = (A \cap B) \setminus C$ for any sets $A$, $B$, and $C$.
\end{problem}

\chapter{Functions / Maps}
Functions (or maps) play a fundamental role in mathematics. Functions compare and relate different kinds of mathematical structures to each other, and provide a way to relate elements from one structure to another.

\section{What is a Function/Map?}
\begin{definition}
    A \textbf{function}\index{function} (or a \textbf{map}\index{map}) $f$ from a set $X$ to a set $Y$ assigns each value in $X$ to exactly one element in $Y$, and is denoted by $f: X \to Y$.
\end{definition}
\begin{definition}
    For a function $f: X \to Y$, the set $X$ is called the \textbf{domain}\index{domain} of the function and the set $Y$ is called the \textbf{codomain}\index{codomain} of the function.
\end{definition}
\begin{example}
    Consider the simple function $f: \mathbb{Z} \to \mathbb{Q}$ where $f(n) = \frac1n$. In this case, $f$ has a domain of $\mathbb{Z}$, i.e. the integers, and a codomain of $\mathbb{Q}$, i.e. the rational numbers.

    We may evaluate the function $f$ at $2 \in \mathbb{Z}$ to get the resulting value of $f(2) = \frac12$.
\end{example}

\textbf{Arrow notation}\index{function!arrow notation} can also be used to define the rule of a function. There is no good way of defining arrow notation, but some examples should help illustrate the basics.
\begin{example}
    Consider $f: \mathbb{N} \to \mathbb{Q}$ where $f(n) = \frac1n$. We may write this more succinctly as $f: \mathbb{N} \to \mathbb{Q}, n \mapsto \frac1n$. Specifically, $n \mapsto \frac1n$ is read as ``$n$ maps to $\frac1n$''.
    
    It is important to note that $\to$ indicates the domain and codomain, and that $\mapsto$ indicates how an element in the domain is `transformed' into an element in the codomain.
\end{example}
\begin{example}
    The function $g: \mathbb{R} \to \mathbb{R}$ where $g(x) = x^2 - 2x + 1$ can be more succinctly written as $g: \mathbb{R} \to \mathbb{R}, x \mapsto x^2 - 2x + 1$.
\end{example}
\begin{example}
    Let $h: \mathbb{Z} \to \mathbb{R}$ where $h(x^2) = x$. This can be written succinctly as either $h: \mathbb{Z} \to \mathbb{R}, x^2 \mapsto x$ or $h: \mathbb{Z} \to \mathbb{R}, n \mapsto \sqrt n$.
\end{example}

We now look at the definition of the image and range.
\begin{definition}
    Let $f: X \to Y$ be a function, and $x \in X$.
    \begin{itemize}
        \item The \textbf{image}\index{function!image} of an element $x \in X$ under the function $f$ is denoted $f(x)$ and is defined to be the value after applying $f$ to $x$.
        \item The \textbf{image} or \textbf{range}\index{function!range} of $f$ is denoted by either $\im f$ or $f(X)$ and is the set of the images of all elements in the domain.
    \end{itemize}
\end{definition}

\newpage

\begin{example}
    Consider the function $f: \mathbb{Z} \to \mathbb{Z}, n \mapsto 1$.
    \begin{itemize}
        \item The image of 0 under $f$ is the \textit{element} 1.
        \item The range/image of $f$ is the \textit{set} $\{1\}$, i.e. $\im f = f(\mathbb{Z}) = \{1\}$.
    \end{itemize}
    It is important to note that the image of an element is a single element, while the image of the function is a set.
\end{example}
\begin{example}
    Consider the function $g: \mathbb{Z} \to \mathbb{Z}, n \mapsto |n|$, where $|n|$ denotes the absolute value of $n$.
    \begin{itemize}
        \item The image of 2 under $g$ is $|2| = 2$.
        \item The image of -3 under $g$ is $|-3| = 3$.
        \item The image of 0 under $g$ is $|0| = 0$.
    \end{itemize}
    The range of the function $g$ is the set of non-negative integers, i.e. $\im g = g(\mathbb{Z}) = \mathbb{N} \cup \{0\}$.
\end{example}

\begin{exercise}
    Let the function $f: \{1, 2, 3\} \to \{1, 4, 9, 16, 25\}$ be such that $f(x) = x^2$.
    \begin{partquestions}{\roman*}
        \item Use arrow notation to write a definition for $f$.
        \item State the domain, codomain, and range of $f$.
        \item What is the image of 2 under $f$?
        \item Is the function $g: \{1, 2, 3\} \to \{1, 8\}, x \mapsto x^3$ \textit{valid}?
    \end{partquestions}
\end{exercise}

We end this section with defining equality of two functions.
\begin{definition}
    Let $f: A \to B$ and $g: C \to D$ be functions. Then $f$ and $g$ are \textbf{equal}\index{function!equality} if and only if
    \begin{itemize}
        \item $A = C$ and $B = D$; and
        \item for all $x \in A = C$, we have $f(x) = g(x)$.
    \end{itemize}
    We denote $f = g$ if the two functions are equal.
\end{definition}
In other words, two functions $f$ and $g$ are equal if their domain and codomain sets are the same and their output values agree on the whole domain.
\begin{example}
    Consider the functions $f: \mathbb{Z} \to \mathbb{Z}, x \mapsto (x-1)^2$ and $g: \mathbb{Z} \to \mathbb{Z}, x \mapsto x^2 - 2x + 1$. Since the two functions' domains and codomains are the same, and because $(x-1)^2 = x^2 - 2x + 1$, thus $f = g$.
\end{example}
\begin{example}
    The functions $f: \mathbb{Z} \to \mathbb{R}, x \mapsto (x-1)^2$ and $g: \mathbb{Z} \to \mathbb{Q}, x \mapsto x^2 - 2x + 1$ are not equal because their codomains differ.
\end{example}

\newpage

\section{Well-Defined Functions}
\begin{definition}
    A function $f: X \to Y$ is \textbf{well-defined} if and only if for each $x \in X$ there is a unique $y \in Y$ such that $f(x) = y$.
\end{definition}
Informally, \textit{identical inputs} produce \textit{identical outputs} for a well-defined function.
\begin{remark}
    Functions that are not well-defined are called `ambiguous' or `ill-defined' functions, and is not a valid function.
\end{remark}

\begin{example}
    Let $S_1$ and $S_2$ be sets, and let $S = S_1 \cup S_2$. Let $f: S \to \{1, 2\}$, such that
    \[
        f(x) = \begin{cases}
            1 & \textrm{ if } x \in S_1\\
            2 & \textrm{ if } x \in S_2
        \end{cases}
    \]
    Then $f$ is well-defined if $S_1 \cap S_2 = \emptyset$. For example, if $S_1 = \{1, 2\}$ and $S_2 = \{3, 4\}$, then $f$ is well-defined.
    
    On the other hand, if $S_1 \cap S_2 \neq \emptyset$, then $f$ is not well-defined. For example, if $S_1 = \{1, 2\}$ and $S_2 = \{2, 3\}$, then $f(2) = 1$ and $f(2) = 2$ simultaneously.
\end{example}

\begin{exercise}
    Is $f: \mathbb{Q} \to \mathbb{Z},\;\frac pq \mapsto p + q$ a well-defined function?
\end{exercise}

\section{Function Composition}
\begin{definition}
    Let $f: X \to Y$ and $g: Y \to Z$ be functions. Then \textbf{composing $f$ with $g$}\index{function!composition} produces a function $h: X \to Z$ where $h(x) = f(g(x))$. We denote $h = f \circ g$ where $\circ$ is the function composition operator.
\end{definition}
\begin{remark}
    We may also alternatively write $fg$ in place of $f \circ g$.
\end{remark}

It is important to note the following about function composition.
\begin{itemize}
    \item Function composition is associative\index{function!composition!associative}. That is, if $f$, $g$, and $h$ are composable, then $f \circ (g \circ h) = (f \circ g) \circ h$. As parentheses do not change the result, they are usually omitted.
    \item The composition $f \circ g$ is only meaningful if the codomain of $g$ is a subset of the domain of $f$. That is, if $f: A \to B$ and $g: C \to D$, then $f \circ g$ is only meaningful if $\im g \subseteq A$.
\end{itemize}

\begin{exercise}
    Let $f: \mathbb{R} \to \mathbb{R}$ and $g: \mathbb{R} \to \mathbb{R}$. Write down the rule of the function $fg$ if $f(x) = x^2 - x + 1$ and $g(y) = \frac1{y^2+1}$.
\end{exercise}

\section{Injective, Surjective, and Bijective Functions}
\begin{definition}
    A function $f: X \to Y$ is \textbf{injective}\index{function!injective} (or \textbf{one-to-one}\index{function!one-to-one}) if $f(x_1) = f(x_2)$ implies $x_1 = x_2$.
\end{definition}
\begin{remark}
    Equivalently, if $x_1 \neq x_2$ then $f(x_1) \neq f(x_2)$ for all $x_1$ and $x_2$ in $X$.
\end{remark}
\begin{example}
    Consider $f: \mathbb{N} \to \mathbb{N}, n \mapsto n^2$. We show that $f$ is injective.
    
    Note that if $n_1, n_2 \in \mathbb{N}$ are such that $f(n_1) = n_1^2 = f(n_2) = n_2^2$ then $n_1 = n_2$ (since $n_1, n_2 > 0$ so taking the square root is okay). Thus $f$ is injective.
\end{example}
\begin{example}
    Consider instead $g: \mathbb{Z} \to \mathbb{Z}, n \mapsto n^2$. Then $g$ is not injective since $g(-2) = g(2) = 4$.
\end{example}

\begin{definition}
    A function $f: X \to Y$ is \textbf{surjective}\index{function!surjective} (or \textbf{onto}\index{function!onto}) if for every $y \in Y$, there exists an $x \in X$ (called the \textbf{pre-image}\index{function!pre-image} of $y$) such that $f(x) = y$.
\end{definition}
\begin{remark}
    Equivalently, the image of $f$ is equal to its codomain, i.e. $\im f = Y$.
\end{remark}
\begin{example}
    Let $S$ denote the set of non-negative real numbers, i.e. $S = \{x\in\mathbb{R} | x \geq 0\}$. Consider the function $f: \mathbb{R} \to S, x \mapsto x^2$. We show that $f$ is surjective.

    Let $y \in S$. Note that $\sqrt{y} \in \mathbb{R}$ since $y$ is a non-negative real number. Observe that $f(\sqrt{y}) = (\sqrt y)^2 = y$. Thus any $y \in Y$ has a pre-image $\sqrt y \in X$. Thus $f$ is surjective.
\end{example}
\begin{example}
    Consider instead the function $g: \mathbb{R} \to \mathbb{R}, x \mapsto x^2$. Then $g$ is not surjective because there is no real number $x \in \mathbb{R}$ such that $g(x) = -1$.
\end{example}

\begin{definition}
    A function is \textbf{bijective}\index{function!bijective} (or a \textbf{bijection}\index{bijection} or a \textbf{one-to-one correspondence}\index{function!one-to-one!correspondence}) if the function is both injective and surjective.
\end{definition}
\begin{example}
    Consider the function $f: \mathbb{R} \to \mathbb{R}, x \mapsto x^3$. We show that $f$ is bijective.
    \begin{itemize}
        \item \textbf{Injective}: Let $a, b \in \mathbb{R}$ such that $f(a) = f(b)$, i.e. $a^3 = b^3$. Clearly we may take the cube root on both sides to yield $a = b$, so $f$ is injective.
        \item \textbf{Surjective}: Let $y \in \mathbb{R}$. Set $x=y^{\frac13}$. Note $x \in \mathbb{R}$ and observe that $f(x) = \left(y^{\frac13}\right)^3 = y$. Thus $y$ has a pre-image of $y^{\frac13}$ in $\mathbb{R}$ and so $f$ is surjective.
    \end{itemize}
    Since $f$ is both injective and surjective it is thus bijective.
\end{example}

\newpage

\begin{definition}
    Let $A$ and $B$ be sets. Then $A$ and $B$ are \textbf{equinumerous}\index{set!equinumerous} if there exists a bijective function $f: A \to B$. In this case, $A$ and $B$ have the same cardinality, i.e., $|A| = |B|$.
\end{definition}
\begin{remark}
    It should be noted that bijectivity is also implied if the function $f: X \to Y$ is injective and the sets $X$ and $Y$ are equinumerous.  % TODO: Add proof
\end{remark}

\begin{exercise}
    Define the function $f: \mathbb{N} \to \mathbb{Z}$ such that
    \[
        f(x) = \begin{cases}
            \frac{x}{2} & \text{ if } x \text{ is even}\\
            \frac{1-x}{2} & \text{ if } x \text{ is odd} 
        \end{cases}
    \]
    By considering $f$, prove that $|\mathbb{N}| = |\mathbb{Z}|$.
\end{exercise}

\chapter{Mathematical Logic and Proof Writing}
The heart of mathematics is in its logical statements. These statements are used to produce a series of logical steps to prove a claim. We explore the basics of logic and proof writing here.

\section{Statements}
\begin{definition}
    A \textbf{statement}\index{statement} (or \textbf{proposition}\index{proposition}) is a sentence that is definitely true or definitely false, but not both.
\end{definition}
\begin{remark}
    A statement can be written in English, or using mathematical notation.
\end{remark}
\begin{example}
    ``Every square with length $x$ has area $x^2$'' is a true statement.
\end{example}
\begin{example}
    `Every circle with radius $x$ has area $x^2$'' is a false statement.
\end{example}
\begin{example}
    ``$12 \in \mathbb{Z}$'' is a true statement.
\end{example}
\begin{example}
    ``$\sqrt2 \in \mathbb{Z}$'' is a false statement.
\end{example}

We may name statements using variables like $P$, $Q$, $R$, etc.
\begin{example}
    If
    \begin{align*}
        P: &\ \text{Every odd number is one more than an even number}\\
        Q: &\ \text{Every triangle has sides of equal length}\\
        R: &\ \frac12 \in \mathbb{Q}
    \end{align*}
    then $P$ is true, $Q$ is false, and $R$ is true.
\end{example}

There are a few operations\index{logical operation} that may be carried out on statements. For the following, assume $P$ and $Q$ are statements.
\begin{itemize}
    \item \textbf{Logical NOT}\index{logical operation!NOT}/\textbf{Negation}\index{logical operation!negation} ($\lnot$): The statement $\lnot P$ is read as ``not $P$''.
    \item \textbf{Logical AND}\index{logical operation!AND}/\textbf{Conjunction}\index{logical operation!conjunction} ($\land$): The statement $P\land Q$ is read as ``$P$ and $Q$''.
    \item \textbf{Logical OR}\index{logical operation!OR}/\textbf{Disjunction}\index{logical operation!disjunction} ($\lor$): The statement $P\lor Q$ is read as ``$P$ or $Q$''.
    
    \newpage

    \item \textbf{Conditional}\index{logical operation!conditional}/\textbf{Implication}\index{logical operation!implication} ($\implies$): The statement $P \implies Q$ can be read many different ways. We list a few:
    \begin{itemize}
        \item $P$ implies $Q$;
        \item if $P$ then $Q$;
        \item $Q$ if $P$;
        \item $P$ only if $Q$;
        \item $P$ is a sufficient condition for $Q$; and
        \item $Q$ is a necessary condition for $P$.
    \end{itemize}
\end{itemize}
\begin{example}
    If $P$ is the statement ``3 is an odd number'' and $Q$ is the statement ``4 is an odd number'', then
    \begin{itemize}
        \item $P\land Q$ is ``3 is an odd number \textbf{and} 4 is an odd number'', which is false;
        \item $P\lor Q$ is ``3 is an odd number \textbf{or} 4 is an odd number'', which is true; and
        \item $\lnot Q$ is ``4 is \textbf{not} an odd number'', which is true.
    \end{itemize}
\end{example}
\begin{exercise}
    Let $P$ be ``1 is a positive number'', $Q$ be ``$-1 > 0$'', and $R$ be ``1 is an odd number''. Is the statement ``$\lnot((P\lor Q)\land R)$ is false'' true?
\end{exercise}

We use \textbf{truth tables}\index{truth table} to explore the relationships between statements and operators. They list all possibilities of the truth or falsity of the statements $P$ and $Q$, and then write the truth for each of the combinations with operations. In a truth table, we denote true statements by ``T'' and false statements by ``F''. For example, the truth table for the logical AND operator is:
\begin{table}[h]
    \centering
    \begin{tabular}{|l|l||l|}
        \hline
        $P$ & $Q$ & $P\land Q$ \\ \hline
        F   & F   & F          \\ \hline
        F   & T   & F          \\ \hline
        T   & F   & F          \\ \hline
        T   & T   & T          \\ \hline
    \end{tabular}
\end{table}

The truth table for the logical OR operator is:
\begin{table}[h]
    \centering
    \begin{tabular}{|l|l||l|}
        \hline
        $P$ & $Q$ & $P\lor Q$ \\ \hline
        F   & F   & F         \\ \hline
        F   & T   & T         \\ \hline
        T   & F   & T         \\ \hline
        T   & T   & T         \\ \hline
    \end{tabular}
\end{table}

\newpage

The truth table for the logical NOT operator is:
\begin{table}[h]
    \centering
    \begin{tabular}{|l||l|}
        \hline
        $P$ & $\lnot P$ \\ \hline
        F   & T         \\ \hline
        T   & F         \\ \hline
    \end{tabular}
\end{table}

We motivate the truth table for the conditional by considering the statements ``you pass the exam'' and ``you pass the course'', which we will denote by $P$ and $Q$ respectively. So $P \implies Q$ would be ``if you pass the exam then you pass the course''.
\begin{itemize}
    \item If $P$ and $Q$ are true, then that means that you passed the exam and passed the course. Hence, ``if you pass the exam then you pass the course'' is \textbf{true}, meaning $P \implies Q$ is true.
    \item If $P$ is true and $Q$ is false, then that means that you passed the exam but failed the course. Hence, the promise that ``if you pass the exam then you pass the course'' is broken, meaning $P \implies Q$ is false.
    \item Now consider the third case when $P$ is false but $Q$ is true. This means that you failed the exam but passed the course. This does not mean that the promise was broken; you could have passed the course through other means. The only promise was that if you pass the exam then you pass the course; the promise was not that passing the exam was the \textit{only} way of passing the course. Since the promise was not broken, thus the promise was kept, so $P \implies Q$ is true.
    \item Finally we consider the case when $P$ and $Q$ are both false: you failed the exam and failed the course. The promise certainly was not broken in this case, so $P \implies Q$ is true.
\end{itemize}
\begin{remark}
    A conditional statement that is true by the virtue of the fact that $P$ is false is called \textbf{vacuously true}\index{vacuously true}.
\end{remark}

In summary, the truth table for the conditional is:
\begin{table}[h]
    \centering
    \begin{tabular}{|l|l||l|}
        \hline
        $P$ & $Q$ & $P\implies Q$ \\ \hline
        F   & F   & T             \\ \hline
        F   & T   & T             \\ \hline
        T   & F   & F             \\ \hline
        T   & T   & T             \\ \hline
    \end{tabular}
\end{table}

\begin{exercise}
    Let $P$ and $Q$ be statements. Draw the truth table for $P \land (\lnot Q)$.
\end{exercise}

\newpage

We now introduce the idea of the \textbf{biconditional}\index{logical operation!biconditional}.
\begin{definition}
    Let $P$ and $Q$ be mathematical statements. If both $(P \implies Q)$ and $(Q \implies P)$ are true, then we write $(P \iff Q)$. In other words, $(P \iff Q) \equiv ((P \implies Q) \land (Q \implies P))$.
\end{definition}
\begin{remark}
    The statement $(P \iff Q)$ can be written in several ways in English:
    \begin{itemize}
        \item $P$ if and only if $Q$;
        \item $P$ is a necessary and sufficient condition for $Q$; or
        \item $P$ is equivalent to $Q$.
    \end{itemize}
\end{remark}

The truth table for the biconditional is:
\begin{table}[h]
    \centering
    \begin{tabular}{|l|l||l|}
        \hline
        $P$ & $Q$ & $P\iff Q$ \\ \hline
        F   & F   & T         \\ \hline
        F   & T   & F         \\ \hline
        T   & F   & F         \\ \hline
        T   & T   & T         \\ \hline
    \end{tabular}
\end{table}

\begin{exercise}
    Suppose $n$ is an integer. Let $P$ be the statement ``$n$ is a multiple of 5'' and $Q$ be the statement ``the last digit of $n$ is 0 or 5''. Let the statement $R = (P \iff Q)$.
    \begin{partquestions}{\roman*}
        \item Write the statement $R$ in English.
        \item Is the statement $R$ true? Justify your answer.
    \end{partquestions}
\end{exercise}

We end this section by noting the order of operations of the logical operators.
\begin{itemize}
    \item $\lnot$ is performed first.
    \item $\land$ and $\lor$ are performed second, and are coequal in their order of operation.
    \item $\implies$ and $\iff$ are performed last, and are coequal in their order of operation.
\end{itemize}
To disambiguate the order of operations, we use parentheses (i.e., brackets).

\newpage

\section{Logical Equivalence and Properties of Logical Operators}
\begin{definition}
    Two statements $P$ and $Q$ are \textbf{logically equivalent}\index{logically equivalent} if and only if they have identical truth values for each possible statement that $P$ and $Q$ may take. If $P$ and $Q$ are logically equivalent, we write $P \equiv Q$.
\end{definition}
\begin{example}
    We show that $(P \implies Q) \equiv (\lnot P) \lor Q$ by considering the truth table for $(\lnot P) \lor Q$.
    \begin{table}[h]
        \centering
        \begin{tabular}{|l|l||l||l|}
            \hline
            $P$ & $Q$ & $\lnot P$ & $(\lnot P) \lor Q$ \\ \hline
            F   & F   & T         & T                  \\ \hline
            F   & T   & T         & T                  \\ \hline
            T   & F   & F         & F                  \\ \hline
            T   & T   & F         & T                  \\ \hline
        \end{tabular}
    \end{table}

    By inspection, we see that $(\lnot P) \lor Q$ has the same truth table as $P \implies Q$. Thus $(P \implies Q) \equiv (\lnot P) \lor Q$.
\end{example}
\begin{remark}
    We separate the intermediate value(s) (e.g. $\lnot P$ in the above example) from the rest by drawing a double line to the sides of the intermediate values.
\end{remark}

\begin{example}
    We show that $(P \iff Q) \equiv (P \land Q) \lor ((\lnot P) \land (\lnot Q))$. For brevity, let $R = (\lnot P) \land (\lnot Q)$.
    \begin{table}[h]
        \centering
        \begin{tabular}{|l|l||l|l|l|l||l|}
            \hline
            $P$ & $Q$ & $\lnot P$ & $\lnot Q$ & $P \land Q$ & $R$ & $(P \land Q) \lor R$ \\ \hline
            F   & F   & T         & T         & F           & T   & T                    \\ \hline
            F   & T   & T         & F         & F           & F   & F                    \\ \hline
            T   & F   & F         & T         & F           & F   & F                    \\ \hline
            T   & T   & F         & F         & T           & F   & T                    \\ \hline
        \end{tabular}
    \end{table}

    By inspection of the truth table we establish the required result.
\end{example}

\begin{exercise}
    Show that
    \[
        ((\lnot P) \iff Q) \equiv (P \implies (\lnot Q)) \land ((\lnot Q) \implies P)
    \]
    by drawing a truth table.
\end{exercise}

\newpage

We note some important properties of logical operators.
\begin{itemize}
    \item \textbf{Contrapositive}\index{contrapositive}: $(P \implies Q) \equiv ((\lnot Q) \implies (\lnot P))$
    \item \textbf{De Morgan's Laws}: \begin{itemize}
        \item $(\lnot (P \land Q)) \equiv ((\lnot P) \lor (\lnot Q))$
        \item $(\lnot (P \lor Q)) \equiv ((\lnot P) \land (\lnot Q))$
    \end{itemize}
    \item \textbf{Commutativity of AND and OR}: \begin{itemize}
        \item $P \land Q \equiv Q \land P$
        \item $P \lor Q \equiv Q \lor P$
    \end{itemize}
    \item \textbf{Associativity of AND and OR}: \begin{itemize}
        \item $P \land (Q \land R) \equiv (P \land Q) \land R$
        \item $P \lor (Q \lor R) \equiv (P \lor Q) \lor R$
    \end{itemize}
    \item \textbf{Distributive Rules}: \begin{itemize}
        \item $P \land (Q \lor R) \equiv (P \land Q) \lor (P \land R)$
        \item $P \lor (Q \land R) \equiv (P \lor Q) \land (P \lor R)$
    \end{itemize}
\end{itemize}
\begin{remark}
    The most important one of these properties would arguably be the contrapositive. We will use this result several times later and in later volumes.
\end{remark}
\begin{exercise}
    Simplify the statement
    \[
        ((P \lor \lnot Q) \land \lnot R) \lor ((P \lor \lnot Q) \land (P \lor R) \land (P \lor \lnot R))
    \]
    into a statement that uses only \textbf{three} operators in total.
\end{exercise}

\section{Predicates and Quantifiers}
\begin{definition}
    A \textbf{predicate}\index{predicate} is a sentence that contains a finite number of variables and becomes a statement when specific values from the \textbf{domain}\index{predicate!domain} are substituted for the variables.
\end{definition}
\begin{example}
    Let $P(n)$ be the predicate ``$n$ is a multiple of 3'', with the domain being the positive integers. Then $P(1)$ is a false statement, $P(2)$ is a false statement, $P(3)$ is a true statement, and so on.
\end{example}
\begin{example}
    Let $Q(n)$ be the predicate ``$n$ is a factor of 9''.
    \begin{itemize}
        \item If the domain of $n$ is the positive integers, then $Q(1)$, $Q(3)$, and $Q(9)$ are true statements, and every other $Q(n)$ is false.
        \item If the domain of $n$ is the integers, then $Q(1)$, $Q(3)$, $Q(9)$, $Q(-1)$, $Q(-3)$, and $Q(-9)$ are all true statements.
    \end{itemize}
\end{example}

There are other ways to convert predicates into statements. One way is to use quantifiers. Quantifiers are words that refer to quantities such as ``some'' or ``all'' to tell people for how many elements make a predicate true.
\begin{definition}
    The \textbf{universal quantifier}\index{quantifier!universal} is $\forall$ and is read as ``for all''. A statement $Q$ of the form $\forall x \in D, P(x)$, where $P(x)$ is the predicate and $D$ is the domain, is called a \textbf{universal statement}\index{statement!universal}.
    \begin{itemize}
        \item $Q$ is true if and only if $P(x)$ is true for every $x$ in $D$.
        \item $Q$ is false if and only if $P(x)$ is false for at least one $x$ in $D$.
    \end{itemize}
    Values $x \in D$ for which $P(x)$ is false is called a \textbf{counterexample}\index{counterexample}.
\end{definition}
\begin{example}
    The true statement ``for every integer $n$, the integer $2n$ is an even number'' can be written as ``$\forall n \in \mathbb{Z}, 2n$ is even''.
\end{example}

\begin{definition}
    The \textbf{existential quantifier}\index{quantifier!existential} is $\exists$ and is read as ``there exists''. A statement $Q$ of the form $\exists x \in D \textrm{ such that } P(x)$, where $P(x)$ is the predicate and $D$ is the domain, is called a \textbf{existential statement}\index{statement!existential}.
    \begin{itemize}
        \item $Q$ is true if and only if $P(x)$ is true for at least one $x$ in $D$.
        \item $Q$ is false if and only if $P(x)$ is false for all $x$ in $D$.
    \end{itemize}
\end{definition}
\begin{example}
    The statement ``there exists a subset of $\mathbb{Z}$ which has 10 elements'' is equivalent to ``$\exists S \subseteq \mathbb{Z} \text{ such that } |S| = 10$''.
\end{example}
\begin{remark}
    We usually shorten the ``such that'' as ``s.t.'', so the above statement is written symbolically as ``$\exists S \subseteq \mathbb{Z} \text{ s.t. } |S| = 10$''.
\end{remark}

We can also combine quantifiers and logical operators together.
\begin{example}
    The statement ``every $\epsilon > 0$ has a $\delta > 0$ such that $|x^2 - 4| < \epsilon$ if $0 < |x - 2| < \delta$'' can be written as ``$\forall \epsilon > 0,\;\exists \delta > 0 \text{ s.t. } (0 < |x - 2| < \delta \implies |x^2 - 4| < \epsilon)$''.
\end{example}
\begin{example}
    Fermat's Last Theorem, which states that for all integers $n\geq 3$ the equation $x^n + y^n = z^n$ has no solution for $x, y, z \in \mathbb{N}$, can be written as
    \[
        \left((n \in \mathbb{Z}) \land (n \geq 3)\right) \implies \left(\forall x, y, z \in \mathbb{N}, x^n + y^n \neq z^n\right).
    \]
    It should be noted that when translating from English to \textit{symbolic writing}, we may need to change some things around.
\end{example}

\begin{exercise}
    Convert the statement
    \begin{quote}
        for all positive integers $n > 2$ there exist integers $a$ and $b$ such that $a^3 + b^4 = n^5$
    \end{quote}
    into symbolic notation using quantifiers and logical operators.
\end{exercise}

\newpage

We now look at negating quantifiers\index{quantifier!negation}. Note that
\begin{itemize}
    \item $\lnot(\forall x, P(x)) \equiv \exists x \text{ s.t. } \lnot P(x)$; and
    \item $\lnot(\exists x \text{ s.t. } P(x)) \equiv \forall x, \lnot P(x)$.
\end{itemize}
\begin{example}
    Consider the statement ``for every integer $n$, the integer $2n$ is even''. One may write that using quantifiers as
    \[
        \forall n \in\mathbb{Z}, 2n \text{ is even}.
    \]
    Negating the above statement would make it
    \[
        \exists n \in \mathbb{Z}, 2n \text{ is not even},
    \]
    i.e., ``there exists an integer $n$ such that $2n$ is odd''.
\end{example}

We also note that $\lnot(P \implies Q) \equiv P \land \lnot Q$. We leave verifying this identity as an exercise to the reader.

\begin{example}
    Consider the statement ``$n^3$ is odd if $n$ is odd for all integers $n$''. We may write this as
    \[
        \forall n \in \mathbb{Z}, (n \text{ is odd}) \implies (n^3 \text{ is odd}).
    \]
    The negation of such a statement would hence be
    \[
        \exists n \in \mathbb{Z} \text{ s.t. }((n \text{ is odd}) \land \lnot(n^3 \text{ is odd})),
    \]
    which in english would be ``there exists an integer $n$ such that $n$ is odd and $n^3$ is even''.
\end{example}

\begin{exercise}
    Consider the statement ``if $x$ is a non-zero real number, then there exists a real number $y$ such that $xy = 1$''.
    \begin{partquestions}{\roman*}
        \item Write down two statements $P$ and $Q$ in symbols such that the above statement is $P \implies Q$.
        \item Negate the above statement, writing your answer in symbolic notation.
    \end{partquestions}
\end{exercise}

\section{Hierarchy of Mathematical Results}
Mathematical statements often have a certain `tier' attached to them. We look at the hierarchy of some of these `tiers'.
\begin{itemize}
    \item \textbf{Proposition}\index{proposition}: A proposition can be thought of as a general proven result. Equivalent names for a proposition are ``\textbf{Claim}'' or ``\textbf{Observation}''.
    \item \textbf{Lemma}\index{lemma}: A lemma can be thought of as a `small' proven result that can help build other mathematical results. For example, Euclid's lemma that ``if a prime $p$ divides the product $ab$ of two integers $a$ and $b$, then $p$ must divide at least one of those integers $a$ or $b$'' is used in the proof of the Fundamental Theorem of Arithmetic.
    \item \textbf{Theorem}\index{theorem}: A theorem can be thought of as a `big' proven result. For example, the Fundamental Theorem of Arithmetic is core to arithmetic as it describes how integers can be uniquely decomposed into its prime factors.
    \item \textbf{Corollary}\index{corollary}: A corollary is said to be a `follow-up result' from a theorem. These results usually follow `very quickly' from a theorem or a proposition. For example, the AM-GM inequality is a corollary of the Pythagorean theorem.
    \item \textbf{Conjecture}\index{conjecture}: A conjecture is a statement whose truth or falsity is unknown.
\end{itemize}

\section{Mathematical Proof Techniques}
\subsection{Direct Proof}
In essence, a direct proof\index{proof!direct} for the statement ``if $P$ then $Q$'' would begin by assuming $P$ is true and showing that this forces $Q$ to be true. We don't need to worry about the case of $P$ being false, since $P \implies Q$ is vacuously true in the case when $P$ is false.
\begin{example}
    We prove that ``if $n$ is an odd integer then $n^2$ is odd'' using direct proof.
    \begin{proof}
        Suppose $n\in\mathbb{Z}$ is an odd integer. Then $n$ can be written in the form $n = 2k + 1$ where $k$ is an integer. Hence
        \[
            n^2 = (2k+1)^2 = 4k^2 + 4k + 1 = 2(2k^2 + 2k) + 1
        \]
        so $n^2$ is one more than a multiple of 2. Thus $n^2 = 2(2k^2 + 2k) + 1$ is odd.
    \end{proof}
\end{example}

\begin{example}
    We look at a proof of the statement ``$1 + (-1)^n(2n-1)$ is a multiple of 4 if $n$ is an integer''.
    \begin{proof}[Proof (see {\cite[p.~124]{hammack_2018}})]
        Suppose $n$ is an integer. Then, $n$ is either odd or even. We look at two cases.
        \begin{itemize}
            \item If $n$ is odd, then $(-1)^n = -1$ and $n = 2k+1$ for some integer $k$. Thus
            \[
                1 + (-1)^n(2n-1) = 1 - (2(2k+1)-1) = -4k
            \]
            which is a multiple of 4.
            \item If $n$ is even, then $(-1)^n = 1$ and $n = 2k$ for some integer $k$. Thus
            \[
                1 + (-1)^n(2n-1) = 1 + (2(2k)-1) = 4k
            \]
            which is a multiple of 4.
        \end{itemize}
        Hence, in both cases, $1 + (-1)^n(2n-1)$ is a multiple of 4.
    \end{proof}
\end{example}

\begin{exercise}
    Prove that $m + n$ is even if the integers $m$ and $n$ have the same parity (i.e., both odd or both even). 
\end{exercise}

\subsection{Contrapositive Proof}
We now look at a \textbf{contrapositive proof}\index{proof!contrapositive}. Recall that $(P \implies Q)$ is logically equivalent to $(\lnot Q \implies \lnot P)$. Thus, a contrapositive proof for ``if $P$ then $Q$'' would begin by assuming $\lnot Q$ is true and deducing that this means that $\lnot P$ is true.

Generally, we would want to prove in the direction from simplicity to complexity. So if $P$ is more complex than $Q$, we may consider using a contrapositive proof.

\begin{example}\label{example-if-(n-1)(n-5)-is-even-then-n-is-odd}
    Suppose $n$ is an integer. We prove the statement ``if $n^2 - 6n + 5$ is even then $n$ is odd''. We note that a direct proof would be tedious and problematic. Using a contrapositive proof would be easier.
    
    We first note that the contrapositive statement that we want to prove is ``if $n$ is \textbf{not} odd, then $n^2 - 6n + 5$ is \textbf{not} even'', that is, ``if $n$ is even, then $n^2 - 6n + 5$ is odd''.
    \begin{proof}[Proof (see {\cite[p.~130]{hammack_2018}})]
        We consider a proof by contrapositive.
        
        Suppose $n$ is even. Then $n = 2k$ where $k$ is an integer. Note
        \begin{align*}
            n^2 - 6n + 5 &= (2k)^2 - 6(2k) + 5\\
            &= 4k^2 - 12k + 5\\
            &= (4k^2 - 12k + 4) + 1\\
            &= 2(2k^2 - 6k + 2) + 1
        \end{align*}
        which means that $n^2 - 6n + 5$ is one more than a multiple of 2, which hence means $n^2 - 6n + 5$ is odd.
    \end{proof}    
\end{example}

\begin{example}
    Suppose $x$ and $y$ are real numbers. We prove the statement ``$x \leq y$ if $x^3 + xy^2 \leq x^2y + y^3$'' using a contrapositive proof.
    
    We first note that the contrapositive statement that we want to prove is ``if $x > y$ then $x^3 + xy^2 > x^2y + y^3$''.

    \begin{proof}[Proof (cf. {\cite[p.~130]{hammack_2018}})]
        We consider a proof by contrapositive.
        
        Assume $x > y$. Then $x - y > 0$. Also, since $x > y$, thus $x$ and $y$ are not both zero. Hence $x^2 + y^2 > 0$.
        Observe
        \[
            (x-y)(x^2+y^2) > 0 \times (x^2+y^2) = 0        
        \]
        so $(x-y)(x^2+y^2) = x^3 + xy^2 - x^2y - y^3 > 0$. Therefore $x^3 + xy^2 > x^2y + y^3$.
    \end{proof}
\end{example}

\begin{exercise}
    Suppose that $a$ and $b$ are integers. Prove that either $a$ is even or $b$ is odd if $a(b^2 + 5)$ is even.
\end{exercise}

\subsection{Proof by Contradiction}
The third proof technique is called a \textbf{proof by contradiction}\index{proof!contradiction}. This method can be used to prove any kind of statement. The basic idea is to assume that the statement we want to prove is false, and then show that this assumption leads to a contradiction. A proof by for the statement ``$P$'' (yes, just $P$) would start by assuming that $P$ is false, and then showing that this assumption would lead to a contradiction, which means that $P$ is \textit{not} false, i.e. $P$ is true.
\begin{remark}
    In fact, what we are showing is that the statement ``$\lnot P \implies \textbf{false}$'' is true.
\end{remark}

Usually, when writing a proof by contradiction, we would like to inform the reader that a proof by contradiction is being employed. Language such as ``by way of contradiction'', ``towards a contradiction'', ``suppose for the sake of contradiction'' etc. may be used to signpost the use of a proof by contradiction.
\begin{remark}
    Some authors would also signal the use of contradiction by using the initialism ``BWOC'' (by way of contradiction). 
\end{remark}

\begin{example}\label{example-sqrt2-is-irrational}
    We prove the classic result that ``$\sqrt 2$ is irrational'' via a proof by contradiction.
    \begin{proof}
        By way of contradiction, assume that $\sqrt2 = \frac ab$ for some integers $a$ and $b$. Furthermore let this fraction be fully reduced; in particular, this means that $a$ and $b$ are not both even. Squaring both sides yields $2 = \frac{a^2}{b^2}$, meaning  $a^2 = 2b^2$. Hence $a^2$ is even, so write $a = 2c$ where $c$ is an integer. This leads to $2b^2 = (2c)^2 = 4c^2$ which implies $b^2 = 2c^2$. Hence $b$ is even, which contradicts the fact that $a$ and $b$ are not both even.
        
        Hence, $\sqrt 2$ is irrational.
    \end{proof}
\end{example}
\begin{remark}
    It is not necessary to have the final statement that ``$\sqrt 2$ is irrational'' (or, more generally, ``$P$ is true'') as it is implied from the proof by contradiction.
\end{remark}

\begin{example}
    We prove the statement that ``for every positive rational number $x$, there exists a positive rational number $y$ such that $y < x$'' by way of contradiction.
    
    We note that the negation of the above statement is ``there exists a rational number $x$ such that for every positive rational number $y$ we have $y \geq x$''.
    \begin{proof}
        Suppose for the sake of contradiction that there exists a rational number $x$ such that for every positive rational number $y$ we have $y \geq x$. Write $x = \frac pq$ where $p$ and $q$ are positive integers.
        
        Now consider the rational number $\frac{p-1}{q}$. Clearly $\frac{p-1}{q} < \frac pq = x$. By assumption, every positive rational number $y$ satisfies $y \geq x$. Hence, $\frac{p-1}{q}$ is non-positive, meaning $\frac{p-1}{q} \leq 0$. Since $q$ is positive, hence $p - 1 \leq 0$ which means $p \leq 1$. But as $p$ is a positive integer, we conclude $p = 1$. Hence $x = \frac 1q$.
        
        We now consider the rational number $\frac{1}{q+1}$. Clearly $\frac{1}{q+1} < \frac{1}{q} = x$. By assumption we must conclude that $\frac{1}{q+1}$ is non-positive. However, $1 > 0$ and $q + 1 > 0$, so $\frac{1}{q+1}$ is positive. Hence we have the fact that $\frac{1}{q+1}$ is positive and non-positive simultaneously, leading to a contradiction.
    \end{proof}
\end{example}
\begin{remark}
    The statement above is one where a direct proof would be easier. We provide a direct proof of it below.
    \begin{proof}
        Since $x$ is a positive rational number write $x = \frac pq$ where $p$ and $q$ are positive integers. Then set $y = \frac{p}{q+1}$. Clearly $\frac{p}{q+1} < \frac{p}{q} = x$ and $\frac{p}{q+1}$ is positive, hence we have found a $y$ such that $y < x$.
    \end{proof}
\end{remark}

\begin{exercise}
    Prove that there exist no integers $a$ and $b$ such that $2a + 4b = 1$.
\end{exercise}

We now look at a proof by contradiction for conditional statements. Recall that $\lnot(P \implies Q) \equiv P \land \lnot Q$. Hence, to prove the statement ``if $P$ then $Q$'' via contradiction, we would start by assuming that $P \land \lnot Q$, and then showing that this assumption would lead to a contradiction, which means that $P \implies Q$ is \textit{not} false, i.e. $P\implies Q$ is true.

\begin{example}
    Suppose $a$ and $b$ are real numbers. We prove the statement ``if $a$ is rational and $ab$ is irrational then $b$ is irrational'' using a proof by contradiction.
    
    We note that the statement we want to contradict is ``$a$ is rational and $ab$ is irrational \textbf{and} $b$ is \textbf{not} irrational'', i.e. ``$a$ is rational and $b$ is rational and $ab$ is irrational''.
    \begin{proof}
        By way of contradiction assume $a$ is rational, $b$ is rational, and $ab$ is irrational. We may then write $a = \frac mn$ and $b = \frac pq$ where $m, n, p, q \in \mathbb{Z}$. Hence $ab = \left(\frac mn\right)\left(\frac pq\right) = \frac{mp}{nq}$ which is clearly rational. Therefore we have that $ab$ is irrational (by assumption) and $ab$ is rational, a contradiction.
    \end{proof}
\end{example}

\begin{example}
    Suppose $a$, $b$, and $c$ are integers. We prove the statement that ``if $a^2 + b^2 = c^2$ then at least one of $a$ or $b$ is even'' using a proof by contradiction.
    
    We note that the statement we want to contradict is ``$a^2 + b^2 = c^2$ \textbf{and not} (at least one of $a$ or $b$ is even)'', i.e. ``$a^2 + b^2 = c^2$ \textbf{and} both $a$ and $b$ are odd''.
    \begin{proof}
        Seeking a contradiction, assume that $a^2 + b^2 = c^2$ and both $a$ and $b$ are odd. Thus we may write $a = 2m + 1$ and $b = 2n + 1$ where $m$ and $n$ are integers. Hence
        \begin{align*}
            a^2 + b^2 &= (2m+1)^2 + (2n+1)^2\\
            &= (4m^2+4m+1) + (4n^2+4n+1)\\
            &= 4m^2 + 4n^2 + 4m + 4n + 2\\
            &= 2(2m^2 + 2n^2 + 2m + 2n +1)
        \end{align*}
        which means that $c^2 = a^2 + b^2$ is even. Hence $c$ is even, which means we may write $c = 2k$ where $k$ is an integer. This leads to
        \[
            c^2 = 4k^2 = 2(2m^2 + 2n^2 + 2m + 2n + 1) = a^2 + b^2.
        \]
        Clearly $4k^2$ is a multiple of 4, while $2(2m^2 + 2n^2 + 2m + 2n + 1)$ is not. Yet, they are equal to each other, a contradiction.
    \end{proof}
\end{example}

\begin{exercise}
    Prove that $\frac{a+b}{2} \geq \sqrt{ab}$ if $a$ and $b$ are positive real numbers by way of contradiction.
\end{exercise}

Despite the power of proof by contradiction, it's best to use it only when the direct and contrapositive approaches do not seem to work.
\begin{example}
    Suppose $n$ is an integer. We prove the statement ``if $n^2 - 6n + 5$ is even then $n$ is odd'' using a proof by contradiction.
    \begin{proof}
        Working towards a contradiction, assume $n^2 - 6n + 5$ is even and $n$ is \textbf{not} odd, i.e. $n$ is even. Then $n = 2k$ for some integer $k$. Note that
        \begin{align*}
            n^2 - 6n + 5 &= (2k)^2 - 6(2k) + 5\\
            &= 4k^2 - 12k + 5\\
            &= (4k^2 - 12k + 4) + 1\\
            &= 2(2k^2 - 5k + 2) + 1
        \end{align*}
        which means that $n^2 - 6n + 5$ is odd. Hence, $n^2 - 6n + 5$ is even (by assumption) and $n^2 - 6n + 5$ is odd (as above), a contradiction.
    \end{proof}
    While there is nothing wrong with this proof, notice that part of it assumes that $n$ is even and concludes that  $n^2 - 6n + 5$ is odd, which is the contrapositive approach done in \myref{example-if-(n-1)(n-5)-is-even-then-n-is-odd}.
\end{example}

\subsection{Proof by Mathematical Induction}
Mathematical induction\index{proof!induction} is a method for proving that a predicate $P(n)$ is true for every positive integer $n$, that is, that the infinitely many statements $P(1), P(2), P(3), \dots$ all hold.

A proof by induction consists of two steps.
\begin{itemize}
    \item The first, the \textbf{base case}\index{proof!induction!base case}, proves the statement for $n = 1$ without assuming any knowledge of other cases. In other words, the base case proves the statement $P(1)$. 
    \item The second, the \textbf{induction step}\index{proof!induction!induction step}, proves that if the statement holds for any given case $n = k$, then it must also hold for the next case $n = k + 1$. In other words, $\forall k \in \mathbb{N}, (P(k) \implies P(k+1))$. The assumption that $P(k)$ being true is called the \textbf{induction hypothesis}.
\end{itemize}
These two steps establish that the predicate $P(n)$ holds for all positive integers $n$.

The base case does not necessarily need to begin with $n = 1$. Sometimes we may begin with $n = 0$, and possibly with any fixed natural number $n = N$, establishing the truth of the statement for all natural numbers $n \geq N$.

In summary, mathematical induction involves two steps:
\begin{itemize}
    \item \textbf{Base Case}: Prove the statement for the initial value.
    \item \textbf{Induction Step}: Prove that for every $n$, if the statement holds for $n$, then it holds for $n + 1$.
\end{itemize}

\begin{example}
    We prove the famous identity
    \[
        1 + 2 + 3 + \cdots + n = \frac{n(n+1)}2
    \]
    using mathematical induction.
    \begin{proof}
        When $n = 1$, the left hand side is 1; the right hand side is $\frac{1(1+1)}{2} = 1$. Thus the initial case is true.

        Now assume that the statement holds for some positive integer $k$, meaning
        \[
            1 + \cdots + k = \frac{k(k+1)}2.
        \]
        We need to show that the statement holds for $k+1$, meaning
        \[
            1 + \cdots + k + (k+1) = \frac{(k+1)(k+2)}2.
        \]
        We work slowly.
        \begin{align*}
            1 + \cdots + k + (k+1) &= (1 + \cdots + k) + (k+1)\\
            &= \frac{k(k+1)}{2} + (k+1) & (\text{by hypothesis})\\
            &= \frac{k(k+1)}2 + \frac{2(k+1)}{2}\\
            &= \frac{k(k+1) + 2(k+1)}2\\
            &= \frac{(k+1)(k+2)}2
        \end{align*}
        which proves the case for $k + 1$. Hence $1 + 2 + 3 + \cdots + n = \frac{n(n+1)}2$.
    \end{proof}
\end{example}

\begin{example}
    Suppose $x > -1$. We will prove that $(1+x)^n \geq 1+nx$ if $n$ is a positive integer.
    \begin{proof}[Proof (cf. {\cite[p.~186]{hammack_2018}})]
        When $n = 1$, the left hand side is $(1+x)^1 = 1+x$ which is exactly the right hand side. Thus the base case is true.
        
        Assume that the statement holds for some positive integer $k$, i.e. $(1+x)^k \geq 1+kx$. We show that the statement holds for $k+1$, i.e. $(1+x)^{k+1} \geq 1+(k+1)x$.
        
        We first note that since $x>-1$, thus $1+x > 0$. We start with our induction hypothesis.
        \begin{align*}
            (1+x)^k &\geq 1+kx\\
            (1+x)^k(1+x) &\geq (1+kx)(1+x) & (\text{since }1+x > 0)\\
            (1+x)^{k+1} &\geq 1 + x + kx + kx^2\\
            &= 1+(k+1)x + kx^2\\
            &> 1+(k+1)x
        \end{align*}
        Hence we see $(1+x)^{k+1} \geq 1+(k+1)x$, meaning that the statement is true for $k+1$.
        
        Therefore $(1+x)^n \geq 1+nx$ if $n$ is a positive integer.
    \end{proof}
\end{example}

\begin{exercise}
    Prove by induction that $a^2 - 1$ is a multiple of 8 for all positive odd integers $a$.
\end{exercise}

We now look at another form of mathematical induction, called \textbf{strong induction}\index{proof!induction!strong}. Unlike regular induction, strong induction assumes that all preceding cases are true, and proves the truth of the next case.

Strong mathematical induction involves two steps:
\begin{itemize}
    \item \textbf{Base Cases}: Prove the statement for the initial values.
    \item \textbf{Induction Step}: Prove that for every $n$, if the statement holds for all (positive) integers $m$ that are at most $n$, then it holds for $n + 1$.
\end{itemize}

\begin{example}
    We prove that every integer $n \geq 8$ can be expressed in the form $3a + 5b$ where $a$ and $b$ are non-negative integers.
    \begin{proof}
        We use strong induction on $n$.
        
        We show the base cases of 8, 9, and 10 hold:
        \begin{itemize}
            \item When $n = 8$, we have $8 = 3 + 5$.
            \item When $n = 9$, we have $9 = 3 \times 3 + 5 \times 0$.
            \item When $n = 10$, we have $10 = 3 \times 0 + 5 \times 2$.
        \end{itemize}
        
        Now assume that for some positive integer $k \geq 8$, every integer $m$ satisfying $8 \leq m \leq k$ results in the statement being true, i.e. $m$ can be written in the form $3a + 5b$. We are to show that the statement for $k+1$ is true, i.e. $k+1$ can be expressed in the form $3a + 5b$.
        
        By hypothesis, $k - 2$ can be expressed in the form $3a+5b$. Hence $k+1 = (k-2) + 3 = 3(a+1) + 5b$, proving the statement for $k+1$.
        
        Therefore by mathematical induction, every integer $n \geq 8$ can be expressed in the form $3a + 5b$ where $a$ and $b$ are non-negative integers.
    \end{proof}
\end{example}

\begin{example}\label{example-strong-induction-on-function}
    Consider the function $f: \left(\mathbb{N}\right)^2\to\mathbb{N}$ where
    \[
        f(m, n) =
        \begin{cases}
            n & \text{if } m = 1, \\
            m & \text{if } n = 1, \\
            f\left(n-1,f(n-1,m-1)\right) & \text{otherwise.}
        \end{cases}
    \]
    We will prove the non-obvious fact that $f(n+1, n) = 2$ for all positive integers $n$.
    \begin{proof}
        We show the base cases of 1 and 2 hold:
        \begin{itemize}
            \item When $n = 1$, we have $f(2, 1) = 2$, so the first case is true.
            \item When $n = 2$, we have
            \[
                f(3,2) = f(1, f(1, 2)) = f(1, 2) = 2
            \]
            so the second case is true.
        \end{itemize}

        Now suppose for some positive integer $k$, every integer $1 \leq m \leq k$ results in the statement being true, i.e. $f(m+1,m) = 2$. We want to show that the case for $k+1$ is true, i.e. $f(k+2, k+1) = 2$.
        \begin{align*}
            f(k+2, k+1) &= f(k, f(k, k+1))\\
            &= f(k, f(k, f(k, k-1)))\\
            &= f(k, f(k, 2)) & (\text{hypothesis on } k-1)\\
            &= f(k, f(1, f(1, k-1)))\\
            &= f(k, f(1, k-1))\\
            &= f(k, k-1) \\
            &= 2 & (\text{hypothesis on } k-1)
        \end{align*}
        which proves that the statement for $k+1$ holds. Hence by mathematical induction, $f(n+1, n) = 2$.
    \end{proof}
\end{example}

\begin{exercise}
    Let the function $f: \mathbb{N} \to \mathbb{Z}$ be defined such that $f(1) = 0$, $f(2) = 1$, and $f(n+2) = 3f(n+1) - 2f(n) + 1$ for all positive integers $n$. Prove that $f(n) = 2^n - n - 1$ for all positive integers $n$.
\end{exercise}

\section{Proving Non-Conditional Statements}
\subsection{Biconditional Statements}\index{proof!biconditional}
Recall that a biconditional statement is a statement like ``$P \iff Q$'', i.e., ``$P$ if and only if $Q$''. We prove such a statement by proving that $P \implies Q$ (which we call the `forward direction') and $Q \implies P$ (which we call the `reverse direction'). Each of these statements may be proved using any of the proof techniques that we covered.

\begin{example}
    We will prove the biconditional statement ``the integer $n$ is even if and only if $n^2$ is even''.
    \begin{proof}
        We prove the forward direction ($n$ is even implies $n^2$ is even) first by using direct proof. Assume that $n$ is even. Then we may write $n = 2k$ where $k$ is an integer. Hence $n^2 = (2k)^2 = 4k^2 = 2(2k^2)$ which is even.

        We now prove the reverse direction ($n^2$ is even implies $n$ is even) via a proof by contrapositive. Suppose $n$ is \textbf{not} even, meaning $n$ is odd. Hence $n = 2k + 1$ where $k$ is an integer. Observe $n^2 = (2k+1)^2 = 4k^2 + 4k + 1 = 2(2k^2 + 2k) + 1$ which is odd.
    \end{proof}
\end{example}

\begin{example}
    Suppose $n$ is an integer. We will prove ``$n$ is a multiple of 6 if and only if $n$ is a multiple of 2 and 3''.
    \begin{proof}
        We prove the forward direction first by using direct proof. Assume $n$ is a multiple of 6, meaning $n = 6k$ for some integer $k$. Clearly $6k = 2(3k)$ and $6k = 3(2k)$, so $n$ is both a multiple of 2 and 3.
        
        We now prove the reverse direction, again using direct proof. Assume $n$ is a multiple of 2 and 3, so we may write $n = 2a$ and $n = 3b$ for some integers $a$ and $b$. Then $2a = 3b$. Hence $a = \frac 32 b$ and $b = \frac 23 a$. Since $a$ and $b$ are integers, hence we conclude $b$ is a multiple of 2 and $a$ is a multiple of 3. Write $a = 3p$ and $b = 2q$ where $p$ and $q$ are integers. Hence $n = 2(3p) = 6p$ and $n = 3(2q) = 6q$. In both cases we see $n$ is a multiple of 6.
    \end{proof}
\end{example}

\begin{exercise}
    Let $n$ be an integer. Prove that $n$ is one more than a multiple of 5 if and only if $n$ is of the form $5k - 4$ where $k$ is an integer.
\end{exercise}

\subsection{Existence Statements}
Some statements only assert the existence of something. These are called \textbf{existence statements} and one only has to provide a particular example that shows it is true.\index{proof!existence proof} 
\begin{example}
    The statement ``there exists an even prime number'' is readily proven by noticing that 2 is an even prime number.
\end{example}

\newpage

\begin{example}
    The statement ``an integer that can be expressed as the sum of two perfect cubes in two different ways exists'' is proven by giving the example 1729, since $1729 = 1^3 + 12^3 = 9^3 + 10^3$.
\end{example}
Note that while an example suffices to prove an existence statement, a single example does not prove a conditional statement.
\begin{exercise}
    Prove that there is a positive integer that is one less than a perfect cube and two less than a perfect square.
\end{exercise}

Existence proofs fall into two categories: \textbf{constructive}\index{proof!constructive} and \textbf{non-constructive}\index{proof!non-constructive} proofs.
\begin{itemize}
    \item Constructive proofs provide an explicit example that proves the statement.
    \item Non-constructive proofs prove that an example exists without providing it.
\end{itemize}
We have only seen constructive proofs so far, so let's look at an example of a non-constructive proof.

\begin{example}
    We prove the classic statement that ``there exist irrational numbers $x$ and $y$ such that $x^y$ is rational'' using a non-constructive proof.
    \begin{proof}
        Let $x = \sqrt2^{\sqrt2}$ and $y = \sqrt2$. We know that $\sqrt2$ is irrational from \myref{example-sqrt2-is-irrational}. Now consider two cases.
        \begin{itemize}
            \item If $x$ is rational, then we have found two irrational numbers (in particular, $x = \sqrt 2$ and $y = \sqrt 2$) such that their exponentiation (i.e.,  $x = \sqrt2^{\sqrt2}$) is rational, proving the claim.
            \item If $x$ is irrational, then \[x^y = \left(\sqrt2^{\sqrt2}\right)^{\sqrt2} = (\sqrt2)^{\sqrt2 \times \sqrt2} = (\sqrt2)^2 = 2\]
            is rational.
        \end{itemize}
        Hence, in both cases, we have an irrational number to an irrational power that results in a rational number.
    \end{proof}

    Notice that we did not explicitly prove whether $\sqrt2^{\sqrt2}$ is rational or irrational; we just showed that either case leads to a case where two irrational numbers, when exponentiated, results in a rational number.
\end{example}

\begin{exercise}
    Let $x = \sqrt2$ and $y = 2\log_2{3}$. It may be assumed that $\sqrt2$ is irrational.
    \begin{partquestions}{\roman*}
        \item Prove that $y$ is irrational.
        \item Produce a constructive proof that there exist irrational $x$ and $y$ such that $x^y$ is rational.
    \end{partquestions}
\end{exercise}

\chapter{Elementary Number Theory}
Number theory is an important part of abstract algebra, as we will use some of its more famous results in proofs. In this chapter, we explore the essentials of number theory.

\section{Divisibility}
\begin{definition}
    Let $a$ and $b$ be integers. Then \textbf{$a$ divides $b$}\index{divides} (or that $a$ is a \textbf{divisor}\index{divisor} of $b$) if there is an integer $k$ such that $ak = b$. This is denoted $a\vert b$.
\end{definition}
\begin{remark}
    A consequence of this definition is that every number divides zero since $a \times 0 = 0$ for every integer $a$.
\end{remark}
\begin{example}
    $7\vert 63$ since $7 \times 9 = 63$
\end{example}
\begin{example}
    8 does not divide 63 since there is not an integer $k$ such that $8k = 63$, so we write $8 \nmid 63$.
\end{example}

\begin{definition}
    An integer $a$ is a \textbf{multiple}\index{multiple} of an integer $b$ if and only if $b$ divides $a$.
\end{definition}
\begin{example}
    63 is a multiple of 7 since 7 divides 63.
\end{example}
\begin{example}
    63 is not a multiple of 8 since 8 does not divide 63.
\end{example}

We list some basic facts about divisibility that are not difficult to prove.
\begin{itemize}
    \item If $a\vert b$ then $a\vert bc$ for all integers $c$.
    \item If $a\vert b$ and $b\vert c$ then $a\vert c$.
    \item If $a\vert b$ and $a\vert c$ then $a\vert sb+tc$ for all integers $s$ and $t$.
    \item If $c \neq 0$, then $a\vert b$ if and only if $ac\vert bc$.
    \item If $a \vert b$ and $b \vert a$ then $a = b$.
\end{itemize}

\begin{definition}
    An integer $p > 1$ with no positive divisors other than 1 and itself is called \textbf{prime}\index{prime}. Every other number greater than 1 is called \textbf{composite}\index{composite}.
\end{definition}
\begin{example}
    The integers 2, 3, 5, 7, 11, and 13 are all prime, but 4, 6, 8, and 9 are composite.
\end{example}
\begin{remark}
    The number 1 is considered neither prime nor composite.
\end{remark}

\newpage

Primes are useful since they can construct any positive integer $n>1$ uniquely.

\begin{theorem}[Fundamental Theorem of Arithmetic]\label{thrm-fundamental-theorem-of-arithmetic}
    Any integer $n > 1$ can be expressed as the product of one or more prime numbers, uniquely up to the order in which they appear.
\end{theorem}

\begin{exercise}
    Express 44100 as a product of primes.
\end{exercise}

\section{Euclid's Division Lemma}
\begin{lemma}[Euclid's Division Lemma]\label{lemma-euclid-division}\index{Euclid's Division Lemma}
    Given two integers $n$ and $d$, with $d \neq 0$ being the \textbf{divisor}\index{divisor}, there exist unique integers $q$ and $r$ such that $n = qd + r$ and $0 \leq r < |d|$, where $|d|$ denotes the absolute value of $d$.

    Here, $n$ is called the \textbf{dividend}\index{dividend}, $q$ is called the \textbf{quotient}\index{quotient}, and $r$ is called the \textbf{remainder}\index{remainder}.
\end{lemma}
\begin{remark}
    This is also known as \textbf{the division algorithm}\index{division algorithm} (e.g. \cite[p.~4]{dummit_foote_2004}) or \textbf{the division theorem}\index{division theorem} (e.g. \cite[\S 21]{clark_1984}).
\end{remark}

\begin{example}
    Using $n = 63$ and $d = 8$, we will have $63 = 7\times8 + 7$.
\end{example}
\begin{example}
    Using $n = 14$ and $d = -3$, we will have $13 = -5\times3 + 1$.
\end{example}

\begin{exercise}
    Express $-210$ in the form $a-13b$, where $a$ and $b$ are positive integers with $0 \leq a \leq 12$.
\end{exercise}

\section{Greatest Common Divisor (GCD) and Lowest Common Multiple (LCM)}
In number theory, the idea of a greatest common divisor and the least common multiple are omnipresent.

\begin{definition}
    Let $m$ and $n$ be two non-zero integers. Then an integer $d$ is said to be the \textbf{greatest common divisor}\index{greatest common divisor} (GCD)\index{GCD} of $m$ and $n$ if $m = pd$ and $n = qd$ for some integers $p$ and $q$, and that $d$ is the largest possible integer that achieves this.

    The GCD of $m$ and $n$ is denoted by $\gcd(m, n)$.
\end{definition}

\begin{example}
    $\gcd(2, 8) = 2$ since $2 = 1 \times 2$ and $8 = 4 \times 2$.
\end{example}
\begin{example}
    $\gcd(42, 231) = 21$ since $42 = 2 \times 21$ and $231 = 11 \times 21$.
\end{example}
\begin{example}
    $\gcd(-10, 25) = 5$ since $-10 = -2 \times 5$ and $25 = 5 \times 5$.
\end{example}
\begin{exercise}
    Find $\gcd(-112, -35)$.
\end{exercise}

\begin{remark}
    If $\gcd(m, n) = 1$ for non-zero integers $m$ and $n$, then $m$ and $n$ are said to be \textbf{coprime}\index{coprime} to each other.
\end{remark}

We now look at the lowest common multiple of two integers.
\begin{definition}
    Let $m$ and $n$ be two non-zero integers. Then an integer $l$ is said to be the \textbf{lowest common multiple}\index{lowest common multiple} (LCM)\index{LCM} of $m$ and $n$ if $l = pm = qn$ for some integers $p$ and $q$, and that $l$ is the smallest possible \textbf{positive} integer that achieves this.

    The LCM of $m$ and $n$ is denoted by $\lcm(m,n)$.
\end{definition}

\begin{example}
    $\lcm(2, 8) = 8$ since $8 = 4 \times 2$ and $8 = 1 \times 8$.
\end{example}
\begin{example}
    $\lcm(42, 231) = 462$ since $462 = 11 \times 42$ and $462 = 2 \times 231$.
\end{example}
\begin{example}
    $\lcm(-10, 25) = 50$ since $50 = 5 \times -10$ and $50 = 2 \times 25$.
\end{example}
\begin{exercise}
    Find $\lcm(-112, -35)$.
\end{exercise}

We note some important results regarding the GCD and LCM.
\begin{lemma}[B\'{e}zout]\label{lemma-bezout}\index{B\'{e}zout's Lemma}
    Let $m$ and $n$ be non-zero integers such that $\gcd(m, n) = d$. Then there exist integers $\lambda$ and $\mu$ such that $\lambda m + \mu n = d$. Moreover, the integers of the form $am + bn$ (where $a$ and $b$ are integers) are multiples of $d$.
\end{lemma}
\begin{proposition}\label{prop-product-of-gcd-and-lcm}
    Let $m$ and $n$ be non-zero integers. Then
    \[
        |mn| = \gcd(m,n) \times \lcm(m,n).
    \]
\end{proposition}
\begin{proposition}\label{prop-gcd-divides-common-divisor}
    Let $a$, $b$, and $d$ are integers. If $d \vert a$ and $d \vert b$ then $d \vert \gcd(a, b)$.
\end{proposition}
\begin{proof}
    We know $\gcd(a,b) \vert a$ and $\gcd(a,b) \vert b$ by definition of GCD. By B\'{e}zout's Lemma (\myref{lemma-bezout}) there exist integers $\lambda$ and $\mu$ such that $\lambda a + \lambda b = \gcd(a,b)$. Now since $d \vert a$ and $d \vert b$ we must have $d \vert \lambda a + \mu b$ by properties of division. Therefore $d \vert \gcd(a,b)$.
\end{proof}

\begin{exercise}
    Suppose $m = 42$ and $n = 70$.
    \begin{partquestions}{\roman*}
        \item Let $d = \gcd(m,n)$. Find $d$.
        \item Hence find $\lcm(m,n)$.
        \item Find a pair of integers $x$ and $y$ such that $mx + ny = d$.
    \end{partquestions}
\end{exercise}

\newpage

\section{Problems}
\begin{problem}
    Find the positive integer $a$ such that $\gcd(a, 50) = 5$ and $\lcm(a, 50) = 150$.
\end{problem}

\begin{problem}
    Let $a$ and $b$ be positive integers such that $\lcm(a, b) = a^2$. What does this imply about $b$?
\end{problem}

\begin{problem}
    Show $n^2 \vert (1+n)^n - 1$ for all positive integers $n$.
\end{problem}

\begin{problem}
    Let $n$ be an integer.
    \begin{partquestions}{\roman*}
        \item Prove that if $n$ is positive, then $6 \vert 2n^3 + 3n^2 + n$.
        \item Prove that $12 \vert n^4 - n^2$.
    \end{partquestions}
\end{problem}

\begin{problem}
    Let $a$ and $b$ be non-negative integers.
    \begin{partquestions}{\roman*}
        \item Prove that $\gcd(a^2, b^2) = \gcd(a,b)^2$.
        \item Find a similar expression for $\lcm(a^2, b^2)$.
    \end{partquestions}
\end{problem}

\chapter{Modular Arithmetic}
One may think of modular arithmetic as a number system where numbers \textit{wrap around} after reaching a certain value. We state useful properties and results from this field here.

\section{Modulo and Modular Congruence}
\begin{definition}
    Given an integer $n>1$, called a \textbf{modulus}\index{modulus}, two integers $a$ and $b$ are said to be \textbf{congruent modulo $n$}\index{congruence} if $n$ is a divisor of $a - b$. It is denoted $a \equiv b \pmod{n}$.
\end{definition}
\begin{remark}
    Equivalently, $a \equiv b \pmod n$ means that $a = kn + b$ for some integer $k$.
\end{remark}
\begin{remark}
    The parentheses mean that ``$\pmod{n}$'' applies to the entire equation, not just to the right-hand side (here, $b$). This notation is not to be confused with the notation ``$b \mod n$'' (without parentheses), which refers to the modulo operation that returns the remainder upon division by $n$.
\end{remark}
\begin{example}
    We see that $38 \equiv 14 \pmod{12}$ since $38 - 14 = 24 = 2 \times 12$. Another way to express this is to say that both 38 and 14 have the same remainder (i.e., 2) when divided by 12.
\end{example}

\begin{exercise}
    Let $m = 5$ and $n = 3$.
    \begin{partquestions}{\alph*}
        \item State the value of $17 \mod m$.
        \item Find an $x$ where $0 \leq x < m$ and $19 \equiv x \pmod m$.
        \item If $A = 1234n + 5$, what is $A \mod n$?
    \end{partquestions}
\end{exercise}

The definition of congruence also applies to negative values.
\begin{example}
    $-3 \equiv 2 \pmod5$ since $-3 = -1\times5 + 2$.
\end{example}
\begin{example}
    $-8 \equiv 7 \pmod5$ since $-8 = -3\times5 + 7$. Furthermore, $7 \equiv 2 \pmod5$ since $7 = 1\times5 + 2$.
\end{example}
\begin{example}
    $-1 \equiv n-1 \pmod{n}$ since $-1 = 1\times n + (n-1)$.
\end{example}

\begin{exercise}
    Explain why $-n \equiv n \pmod{2n}$.
\end{exercise}

The operation of congruence modulo $n$ has a few properties\index{congruence!properties} which we state without proof. Let $k$ be an integer, $a \equiv b \pmod n$, $a_1 \equiv b_1 \pmod n$ and $a_2 \equiv b_2 \pmod n$. Then
\begin{itemize}
    \item $a + k \equiv b + k \pmod n$;
    \item if $a+k \equiv b+k \pmod n$ then $a \equiv b \pmod n$;
    \item $ka \equiv kb \pmod n$;
    \item $ka \equiv kb \pmod {kn}$;
    \item $a_1 \pm a_2 \equiv b_1 \pm b_2 \pmod n$;
    \item $a_1a_2 \equiv b_1b_2 \pmod n$;
    \item $a^k \equiv b^k \pmod n$ if $k \geq 0$;
    \item if $ka \equiv kb \pmod n$ and $\gcd(k, n) = 1$, then $a \equiv b \pmod n$; and
    \item if $ka \equiv kb \pmod{kn}$ where $k \neq 0$ then $a \equiv b \pmod n$.
\end{itemize}

\begin{exercise}
    Find the last two digits of $778899^{112233}$.
\end{exercise}

\section{Modular Multiplicative Inverse}
\begin{definition}
    Let $m$ be a positive integer, and let $a$ be an integer. An integer $x$ such that $ax \equiv 1 \pmod m$ is said to be the \textbf{multiplicative inverse of $a$ modulo $m$}\index{multiplicative inverse modulo $n$}.
\end{definition}
\begin{example}
    4 is the multiplicative inverse of 7 modulo 9 since $4 \times 7 = 28 = 3 \times 9 + 1 \equiv 1 \pmod 9$.
\end{example}

\begin{proposition}\label{prop-multiplicative-inverse-exists-iff-coprime}
    $a$ has a multiplicative inverse modulo $m$ if and only if $\gcd(a,m) = 1$.
    % A multiplicative inverse of $a$ modulo $m$ exists if and only if $\gcd(a,m) = 1$.
\end{proposition}
\begin{proof}
    We first work forwards and suppose $k$ is the multiplicative inverse of $a$ modulo $m$. Then $ka \equiv 1 \pmod m$. Hence $ka - 1 \equiv 0 \pmod m$, so $m$ divides $ka - 1$. This means that $ka - 1$ is a multiple of $m$, so $ka - 1 = pm$ for some integer $p$. Therefore $ka + pm = 1$ By B\'{e}zout's Lemma (\myref{lemma-bezout}) this means that $\gcd(a, m) = 1$.
    
    Now, working in the reverse direction, suppose $\gcd(a, m) = 1$. B\'{e}zout's Lemma tells us that integers $k$ and $p$ exist such that $ka + pm = 1$, meaning $ka - 1 = pm$. So $m$ divides $ka - 1$, meaning $ka - 1 \equiv 0 \pmod m$ which the result quickly yields.
\end{proof}

\begin{example}
    The number 20 has a multiplicative inverse modulo 31 since $\gcd(20, 31) = 1$. One can verify that 14 is the multiplicative inverse of 20 modulo 31.
\end{example}

\begin{exercise}
    Find the modular multiplicative inverse of 123 modulo 5.
\end{exercise}


%=========================================
\setpartpreamble[u][\textwidth]{
    \quoteattr{
        [The] axioms for a group are short and natural... [yet] somehow hidden behind these axioms is the monster simple group, a huge and extraordinary mathematical object, which appears to rely on numerous bizarre coincidences to exist. The axioms for groups give no obvious hint that anything like this exists.
    }
    {
        Richard Borcherds, 2009
    }
    {
        \cite{cook_borcherds_2009}
    }

    Groups are one of the most fundamental structures in abstract algebra. They underpin the ideas of symmetry and allow us to explore the relationships between symmetrical objects. An analysis of all the different ways an object can be symmetric is also possible with groups. It would be an understatement to say that groups are important in abstract algebra; without them, there can be no further and more in-depth exploration of the other structures.

    Part I is a simple introduction to the world of groups. As with most books on this topic, we concentrate on abstract groups, and, in particular, on finite groups. We also discuss and explore some crucial results about the structure of groups. The content covered in this part should be ample for one to understand the fundamentals of group theory, and appreciate the wonders of groups and symmetry.
}
\part{Group Theory}
\chapter{Introduction To Groups}
In this chapter, we motivate the definition and use of groups in mathematics.

\section{The Study of Symmetry}
A group is a \textit{collection of symmetries of something}. A symmetry is a \textit{mapping} from something to itself that \textit{preserves structure}. This is, of course, not the formal definition of a group, but it gives an intuition of \textit{why} mathematicians care about groups.

\begin{wrapfigure}{r}{0.35\textwidth}
    \centering
    \pdfteximgframed{0.3\textwidth}{part1/images/intro-to-groups/triangle-reflections.pdf_tex}
\end{wrapfigure}

For example, one may consider the collection of symmetries of an equilateral triangle. What actions could one perform to make the triangle ``look the same'' as before applying the action? Well, we could do nothing. That action is called the \textit{identity action}. We could also reflect the triangle about the line $S_0$ and observe that the triangle ``looks the same as before''. We may also reflect the triangle about the lines $S_1$ and $S_2$, and the triangle will still ``look the same as before''. One may also consider rotating the triangle $120^\circ$ or $240^\circ$ about the centre in a clockwise manner (note that rotating the triangle $360^\circ$ is the same as the identity action, so we do not count it here).

So we count 6 distinct actions in total: 1 identity action, 2 rotation actions, and 3 reflection actions. We can say that the \textit{group of symmetries of the equilateral triangle} has 6 actions (or elements) in total.

\begin{wrapfigure}{r}{0.35\textwidth}
    \centering
    \pdfteximgframed{0.3\textwidth}{part1/images/intro-to-groups/circular-crossings.pdf_tex}
\end{wrapfigure}

Another category of groups that we can consider is \textit{groups of rotation}. For example, consider the image on the right. There are 4 actions that we can do to this image that makes it ``look the same as before''.
\begin{itemize}
    \item Do nothing (the identity action).
    \item Rotate the circle $90^\circ$ clockwise.
    \item Rotate the circle $180^\circ$ clockwise.
    \item Rotate the circle $270^\circ$ clockwise.
\end{itemize}
Note that the image has \textbf{no} lines of symmetry; due to the unique braiding on the knots, this image only has \textit{rotational symmetry} and no \textit{mirror} (or \textit{reflective}) \textit{symmetry}. Thus, this group has only 4 actions, all of which are rotations. One could say that this group is a \textit{cyclic group} and that it has \textit{order} 4. We'll formally define what these terms mean in later chapters.

\begin{wrapfigure}{r}{0.35\textwidth}
    \centering
    \pdfteximgframed{0.3\textwidth}{part1/images/intro-to-groups/permutations.pdf_tex}
\end{wrapfigure}

Let's look at a more technical example. Consider a set of points in a pane. We can consider the \textit{group of symmetries of a finite collection of points}. What are the symmetries of the points? Well, in this case, a \textit{symmetry} is a way to move one of these points to another point while making it ``look the same as before''.

\begin{itemize}
    \item One possible symmetry is given by the solid arrows. In such a symmetry, we have a a point (1) mapping to itself, a cycle of 2 points (5 and 6) on the left, and a cycle of 3 points (2, 3, and 4) on the right. Since the points ``look the same as before'', this is a valid symmetry.
    \item Another possible symmetry is given by the dotted arrows. In this case, all the points are shifted clockwise in a circle. Since the points ``look the same as before'', this is valid.
\end{itemize}

One may notice that what each of the symmetries is doing is \textit{permuting} the points around. In this case, this is exactly what each of the symmetries is doing: generating a possible permutation of points and ensuring that their locations stay the same. The collection of symmetries is thus called the \textit{symmetric group of degree 6}, and its group actions compose of \textit{bijections from the set to itself}.

\begin{exercise}
    How many symmetries are there in the symmetric group of degree 6? In other words, what is the order of the group above?\newline
    (\textit{Hint: Consider the number of permutations in the group.})
\end{exercise}

\section{What Constitutes a Group?}
Now that we have taken a look at some examples of groups, we ask: what properties do all groups satisfy? These properties are called the \textbf{group axioms}. We motivate the `discovery' of each axiom with examples.

\begin{wrapfigure}{r}{0.325\textwidth}
    \centering
    \pdfteximgframed{0.275\textwidth}{part1/images/intro-to-groups/triangle-reflections.pdf_tex}
\end{wrapfigure}

Consider the group of symmetries of an equilateral triangle. One condition that that group must satisfy is that performing group actions one after another should not make the underlying object ``non-symmetric''. We should not be able to form a group action that results in the triangle being ``non-symmetric''. For example, we do not include rotating the triangle $90^\circ$ clockwise about $S_0$ (into 3D space) as this immediately makes the triangle ``different to how it began''.

\newpage

This property can be called the \textbf{axiom of closure} and can be written like this:
\begin{quote}
    A group $(G, \ast)$ is a set $G$ together with a binary operation $\ast$ that ensures closure. That is, if $a$ and $b$ are in $G$, then $a \ast b$ is also in $G$.
\end{quote}
(Of course, this is not the actual definition, but we'll get into it later.)

The set $G$ in the previous example is the set of actions on the equilateral triangle that preserves symmetry. The binary operation $\ast$ can be informally called the ``followed by'' operator. Thus
\begin{quote}
    (reflect about $S_0$) $\ast$ (rotate the triangle 120° clockwise)
\end{quote}
means
\begin{quote}
    rotate the triangle 120° clockwise, \textit{followed by} reflection about $S_0$
\end{quote}
in standard English. It should be noted that we read actions \textbf{from right to left}, as seen in the example above. In the following chapters, such actions will be replaced with symbols.

Another property that a group must have is that it must contain an \textit{identity element}. We emphasised this property numerous times in the previous examples. This is called the \textbf{axiom of identity} and is phrased as follows.
\begin{quote}
    A group $(G, \ast)$ has an element $e$ where, for any element $x$ in the group, it satisfies $e \ast x = x \ast e = x$.
\end{quote}
This means that the identity action should do nothing. Applying the action before or after another action should just perform the action.

We would also like, for every action, to have an action that \textit{undoes} the previous action. For example, for rotation, we would like to have an action that undoes the rotation. This action is called the \textit{inverse} of the action, and the \textbf{axiom of inverse} guarantees that every action in a group has an inverse.

\begin{quote}
    For every element $x$ in the group $(G, \ast)$, there exists an action in the group, called the inverse of $x$ and denoted by $x^{-1}$, such that $x \ast x^{-1} = x^{-1} \ast x = e$.
\end{quote}

The last axiom is hard to discover naturally and hard to motivate, but is absolutely necessary for groups. Let's say we want to perform 3 rotations (say, $r_1$, $r_2$ and $r_3$) in that sequence on a braided circle. We would not want to distinguish between performing ``$r_1$, then $r_2$ and $r_3$'' (i.e., $r_1 \ast (r_2 \ast r_3)$), and ``$r_1$ then $r_2$, then $r_3$'' (i.e. $(r_1 \ast r_2) \ast r_3$). We are only concerned about the \textit{sequence} of the rotations. This is called the \textbf{axiom of associativity}:
\begin{quote}
    Let $x, y$, and $z$ be elements in $(G, \ast)$. Then $(x \ast y) \ast z = x \ast (y \ast z)$.
\end{quote}

\newpage

So what is a group?
\begin{definition}
    A \textbf{group}\index{group} is a set $G$ together with a operation on $G$, here denoted by $\ast$, satisfying the following axioms.
    \begin{enumerate}
        \item \textbf{Closure}\index{axiom!group!closure}: For all elements $a$ and $b$ in $G$, $a \ast b$ is also in $G$.
        \item \textbf{Associativity}\index{axiom!group!associativity}: For all elements $a, b$, and $c$ in $G$, we have $a \ast (b \ast c) = (a \ast b) \ast c$.
        \item \textbf{Identity}\index{axiom!group!identity}: There exists an element $e$ in $G$ such that for any element $x$ in $G$ we have $e \ast x = x \ast e = x$.
        \item \textbf{Inverse}\index{axiom!group!inverse}: For every element $x$ in $G$, there exists an element $x^{-1}$ in $G$ such that $x \ast x^{-1} = x^{-1} \ast x = e$.
    \end{enumerate}
\end{definition}

Usually, for the brevity of notation, we will write $a \ast b$ as $ab$. We will look at more properties of groups in later chapters. We would usually suppress the operation $\ast$ when defining a group, so instead of saying that the group is $(G, \ast)$, we just say that the group is $G$.

We will look at examples of groups in the next chapter.

\newpage

\section{Problems}
\begin{problem}
Determine whether the following are groups. If they are, prove it. If not, explain why they are not groups.
\begin{partquestions}{\alph*}
    \item $(\Z, +)$.
    \item $(\Z \setminus \{0\}, \times)$ where $\times$ denotes regular multiplication.
    \item $(\R \setminus \{0\}, \times)$ where $\times$ denotes regular multiplication.
    \item $(\{0\}, \times)$ where $\times$ denotes regular multiplication.
    \item $(\{1\}, +)$ where $+$ denotes regular addition.
    \item $(\{1\}, \times)$ where $\times$ denotes regular multiplication.
\end{partquestions}
\end{problem}

\begin{problem}
    Show that the \textbf{trivial group}\index{trivial group} $(\{e\}, *)$ where $e \ast e = e$ is indeed a group.
\end{problem}

\chapter{Basics of Groups}
With a basic intuition of the definition of groups, we look at the basic properties satisfied by groups and introduce two simple types of groups.

\section{Basic Examples of Groups}
Recall that a group satisfies four axioms.
\begin{enumerate}
    \item \textbf{Closure}: For all elements $a$ and $b$ in $G$, $a \ast b$ is also in $G$.
    \item \textbf{Associativity}: For all elements $a, b$, and $c$ in $G$, we have $a \ast (b \ast c) = (a \ast b) \ast c$.
    \item \textbf{Identity}: There exists an element $e$ in $G$, called the identity element\index{group!identity}, such that, for any element $x$ in $G$, we have $e \ast x = x \ast e = x$.
    \item \textbf{Inverse}: For every element $x$ in $G$, there exists an element $x^{-1}$ in $G$ such that $x \ast x^{-1} = x^{-1} \ast x = e$.
\end{enumerate}

Recall also we write $a \ast b$ as $ab$, and denote the group $(G, \ast)$ by $G$ only.
\begin{remark}
    If the group operation is additive\index{group!additive}, we write $a + b$ instead of $ab$.
\end{remark}

Let's look at some examples of groups.
\begin{example}
    Let $\Z$ be the set of integers and let $+$ denote regular addition. Then $(\Z, +)$ forms a group.
    \begin{enumerate}
        \item \textbf{Closure}: For all integers $a$ and $b$, we know $a + b$ is an integer.
        \item \textbf{Associativity}: By \myref{axiom-addition-is-associative} we know for all integers $a$, $b$, and $c$ that $a + (b + c) = (a + b) + c$.
        \item \textbf{Identity}: The identity is 0, since $0 + x = x + 0 = x$.
        \item \textbf{Inverse}: The inverse of the integer $x$ is $-x$, since $x + (-x) = (-x) + x = 0$.
    \end{enumerate}
\end{example}

An important thing to note about addition is that it is commutative (by \myref{axiom-addition-is-commutative}). A group with a commutative operation is called a \textbf{commutative group}\index{group!commutative}. However, it is more often called an \textbf{abelian group}\index{group!abelian}, named after Norwegian mathematician Niels Henrik Abel.

We now look at the notion of $\Z_n$.
\begin{definition}
    Define $\Z_n$ to be the set $\{0, 1, 2, \dots, n-1\}$ where $n$ is a non-negative integer.
\end{definition}
\begin{definition}
    The operation $\oplus_n$ denotes addition modulo $n$. That is, $a \oplus_n b = (a + b) \mod{n}$ for any integers $a$ and $b$.
\end{definition}
\begin{proposition}\label{prop-Zn-is-abelian-group}
    The set $\Z_n$ with the operation $\oplus_n$ forms an abelian group.
\end{proposition}
\begin{proof}
    We prove this by first showing that the group axioms hold.
    \begin{enumerate}
        \item \textbf{Closure}: For $a$ and $b$ in $\Z_n$, $a \oplus_n b = (a + b) \mod{n}$ is a integer between 0 and $n - 1$. Thus $a \oplus b$ is inside $\Z_n$.
        \item \textbf{Associativity}: For $a$, $b$, and $c$ in $\Z_n$, since addition is associative (\myref{axiom-addition-is-associative}), thus $a \oplus_n (b \oplus_n c) = (a + (b + c)) \mod{n} = ((a + b) + c) \mod{n} = (a \oplus_n b) \oplus_n c$.
        \item \textbf{Identity}: The identity is $0$ since for every $x$ in $\Z_n$, $0 \oplus_n x = (0 + x) \mod{n} = (x + 0) \mod{n} = x \oplus_n 0 = x$.
        \item \textbf{Inverse}:
        \begin{itemize}
            \item $0$ is its own inverse since $0 \oplus_n 0 = 0$ which is the identity.
            \item For any other integer $x$ in $\Z_n$, the inverse is $n - x$. Since $1 \leq x \leq n - 1$, thus $1 \leq n - x \leq n - 1$ so $n - x$ is indeed in $\Z_n$. Also, $x \oplus_n (n - x) = (x + (n - x)) \mod{n} = n \mod{n} = 0$ and $(n - x) \oplus_n x = ((n-x) + x)\mod{n} = n \mod{n} = 0$.
        \end{itemize}
    \end{enumerate}
    Since the four group axioms are satisfied, this is a group. Furthermore, as addition is commutative (\myref{axiom-addition-is-commutative}), thus addition modulo $n$ is commutative. Therefore $(\Z_n, \oplus_n)$ is a commutative group.
\end{proof}
\begin{remark}
    Some sources (e.g. {\cite[\S 33]{clark_1984}} and {\cite[Proposition 2.31]{humphreys_1996}}) define $\Z_n$ as a set of congruence classes modulo $n$, i.e. $\Z/n\Z$. We will show that these two definitions are equivalent in a future chapter.
\end{remark}

One could use a \textbf{Cayley table}\index{cayley table} (or \textbf{group table}\index{group table}) to show that a structure is a group.
\begin{example}
    We draw the Cayley table of $(\Z_6, \oplus_6)$ to show that it is a group.
    \begin{table}[h]
        \centering
        \begin{tabular}{|l|l|l|l|l|l|l|}
        \hline
        $\boldsymbol{\oplus_6}$ & $\boldsymbol{0}$ & $\boldsymbol{1}$ & $\boldsymbol{2}$ & $\boldsymbol{3}$ & $\boldsymbol{4}$ & $\boldsymbol{5}$ \\ \hline
        $\boldsymbol{0}$          & 0          & 1          & 2          & 3          & 4          & 5          \\ \hline
        $\boldsymbol{1}$          & 1          & 2          & 3          & 4          & 5          & 0          \\ \hline
        $\boldsymbol{2}$          & 2          & 3          & 4          & 5          & 0          & 1          \\ \hline
        $\boldsymbol{3}$          & 3          & 4          & 5          & 0          & 1          & 2          \\ \hline
        $\boldsymbol{4}$          & 4          & 5          & 0          & 1          & 2          & 3          \\ \hline
        $\boldsymbol{5}$          & 5          & 0          & 1          & 2          & 3          & 4          \\ \hline
        \end{tabular}
    \end{table}

    One observes from the Cayley table that
    \begin{enumerate}
        \item for all $x$ and $y$ in $\Z_6$, $x \oplus_6 y$ is in $\Z_6$;
        \item for all $x, y$, and $z$ in $\Z_6$, $x \oplus_6 (y \oplus_6 z) = (x \oplus_6 y) \oplus_6 z$;
        \item 0 is the identity since adding anything to it returns the original number; and
        \item every row has an integer that, when added, gives 0.
    \end{enumerate}
    Thus $(\Z_6, \oplus_6)$ is a group.
\end{example}

It should be noted that, in this book, we use the convention of reading the \textbf{row before the column} in a group table. However, since $(\Z_6, \oplus_6)$ is an abelian group, the order does not matter. We will look at Cayley tables of non-abelian groups later.

\begin{definition}
    The operation $\otimes_n$ denotes multiplication modulo $n$. That is, $a \otimes_n b = (a \times b) \mod{n}$ for any integers $a$ and $b$.
\end{definition}
\begin{example}
    Let the set $A = \{1, 2, 3, 4\}$. We draw the Cayley table of $(A, \otimes_5)$ to show that it is a group.
    \begin{table}[h]
        \centering
        \begin{tabular}{|l|l|l|l|l|l|l|}
        \hline
        $\boldsymbol{\otimes_5}$ & $\boldsymbol{1}$ & $\boldsymbol{2}$ & $\boldsymbol{3}$ & $\boldsymbol{4}$ \\ \hline
        $\boldsymbol{1}$          & 1          & 2          & 3          & 4          \\ \hline
        $\boldsymbol{2}$          & 2          & 4          & 1          & 3          \\ \hline
        $\boldsymbol{3}$          & 3          & 1          & 4          & 2          \\ \hline
        $\boldsymbol{4}$          & 4          & 3          & 2          & 1          \\ \hline
        \end{tabular}
    \end{table}

    One observes from the Cayley table that
    \begin{enumerate}
        \item for all $x$ and $y$ in $A$, $x \otimes_5 y$ is in $A$;
        \item for all $x, y$, and $z$ in $A$, $x \otimes_5 (y \otimes_5 z) = (x \otimes_5 y) \otimes_5 z$;
        \item 1 is the identity since multiplying anything to it returns the original number; and
        \item every row has an integer that, when multiplied, gives 1.
    \end{enumerate}
    Thus $(A, \otimes_5)$ is a group.
\end{example}

\begin{exercise}
    By using a Cayley table, show that $(\Z_6, \otimes_6)$ does \textbf{not} form a group.
\end{exercise}

\section{General Properties of Groups}
\begin{proposition}
    The identity of a group $G$ is unique.
\end{proposition}
\begin{proof}
    Suppose $e_1$ and $e_2$ are identities of the group $G$. Then, for all $x$ in $G$, we have
    \[
        e_1x = xe_1 = x \text{ and } e_2x = xe_2 = x,
    \]
    since they are identities. Thus,
    \begin{align*}
        e_1 &= e_1e_2 & (e_2 \text{ is an identity, so } xe_2 = x)\\
        &= e_2 & (e_1 \text{ is an identity, so } xe_1 = x)
    \end{align*}
    Hence $e_1 = e_2$, meaning that the identity of a group is unique.
\end{proof}

\begin{proposition}
    The inverse of an element $x$ of a group $G$ is unique.
\end{proposition}
\begin{proof}
    Suppose that $a$ and $b$ are inverses of $x$. Then by definition, $ax = xa = e \text{ and } bx = xb = e$. So,
    \begin{align*}
        a &= ae & (e \text{ is the identity})\\
        &= a(xb) & (b \text{ is an inverse, so } xb = e)\\
        &= (ax)b & (\text{associativity})\\
        &= eb & (a \text{ is an inverse, so } ax = e)\\
        &= b & (e \text{ is the identity})
    \end{align*}
    Therefore $a = b$. Thus the inverse of $x$ is unique.
\end{proof}

\begin{proposition}\label{prop-inverse-of-identity-is-identity}
    $e^{-1} = e$.
\end{proposition}
\begin{proof}
    Clearly $e = ee^{-1}$ since $xx^{-1} = e$ for all elements $x$ in $G$, including $x = e$. Thus, by left multiplying $e^{-1}$ on both sides, we see $e = e^{-1}$.
\end{proof}

\begin{proposition}
    $\left(x^{-1}\right)^{-1} = x$ for all $x$ in $G$.
\end{proposition}
\begin{proof}
    See \myref{exercise-inverse-of-inverse-is-element} (later).
\end{proof}

\begin{proposition}[Shoes and Socks]\index{Shoes and Socks}
    For all elements $a$ and $b$ in $G$, $(ab)^{-1} = b^{-1}a^{-1}$.
\end{proposition}
\begin{proof}
    Recall that $g^{-1}$ is the inverse of $g$ if and only if $gg^{-1} = g^{-1}g = e$. We show that $(ab)(b^{-1}a^{-1}) = e$ and $(b^{-1}a^{-1})(ab) = e$.
    \begin{align*}
        (ab)(b^{-1}a^{-1}) &= a(bb^{-1})a^{-1} & (\text{associativity})\\
        &= a(e)a^{-1} & (bb^{-1} = e)\\
        &= aa^{-1} & (e \text{ is the identity})\\
        &= e, & (aa^{-1} = e)\\
        (b^{-1}a^{-1})(ab) &= b^{-1}(a^{-1}a)b & (\text{associativity})\\
        &= b^{-1}(e)b & (a^{-1}a = e)\\
        &= b^{-1}b & (e \text{ is the identity})\\
        &= e & ( b^{-1}b = e).
    \end{align*}
    Thus, $b^{-1}a^{-1}$ is the inverse of $ab$, i.e. $(ab)^{-1} = b^{-1}a^{-1}$.
\end{proof}

\begin{exercise}\label{exercise-inverse-of-inverse-is-element}
Prove that for all $x$ in $G$, $\left(x^{-1}\right)^{-1} = x$.
\end{exercise}

\newpage

We now prove an important property of groups: the \textbf{cancellation law}.
\begin{proposition}[Cancellation Law]\index{group!cancellation law}
    Let $g$, $x$, and $y$ be elements in the group $G$. Then the following statements are equivalent.
    \begin{enumerate}[label=(\arabic*)]
        \item $x = y$
        \item $gx = gy$
        \item $xg = yg$
    \end{enumerate}
\end{proposition}

\begin{proof}
    We prove the statements in order.
    \begin{itemize}
        \item $\boxed{(1) \implies (2)}$ Given $x = y$, applying $g$ on the left on both sides yields $gx = gy$.
        
        \item $\boxed{(2) \implies (3)}$ Given $gx = gy$. Applying 
        $g^{-1}$ on the left on both sides yields $g^{-1}(gx) = g^{-1}(gy)$, meaning $(g^{-1}g)x = (g^{-1}g)y$ by associativity. Since $g^{-1}g = e$ by definition of $g^{-1}$, so $x = y$. Applying $g$ on the right on both sides yields $xg = yg$.
        
        \item $\boxed{(3) \implies (1)}$ Given $xg = yg$ we may apply $g^{-1}$ on the right on both sides to obtain $(xg)g^{-1} = (yg)g^{-1}$. By associativity we have $x(gg^{-1}) = y(gg^{-1})$. Now $gg^{-1} = e$ by definition of $g^{-1}$, so $x = y$.
    \end{itemize}
    This completes the proof.
\end{proof}

To end this section, we introduce notation for the repeated application of $\ast$ on a single element $x$ in the group $G$.
\[
    x^n =
    \begin{cases}
        \underbrace{x\ast x\ast \cdots \ast x}_{n \text{ copies of } x} & \text{ if } n > 0\\
        e & \text{ if } n=0 \\
        \underbrace{x^{-1}\ast x^{-1}\ast \cdots \ast x^{-1}}_{|n| \text{ copies of } x^{-1}} & \text{ if } n<0
    \end{cases}
\]
In the case of an additive group operation, $x^n$ is written as $nx$, where $n$ is an integer.

Note that some laws of exponents apply to the above operation.
\begin{proposition}
    Let $G$ be a group, $x$ be an element in that group, and let $m$ and $n$ be non-negative integers. Then
    \begin{enumerate}
        \item $x^m \ast x^n = x^{m+n}$;
        \item $\left(x^m\right)^n = x^{mn}$; and
        \item $\left(x^{-1}\right)^n = \left(x^n\right)^{-1}$.
    \end{enumerate}
\end{proposition}

\newpage

\begin{proof}
    We prove each statement individually.
    \begin{enumerate}
        \item We consider a proof by induction by inducting on $n$.
        
        When $n = 0$, one sees
        \begin{align*}
            x^m \ast x^0 &= x^m \ast e & (\text{definition of }x^0)\\
            &= x^m & (\text{definition of }e)\\
            &= x^{m+0}
        \end{align*}
        so the case where $n=0$ is true.

        Now assume that for some non-negative integer $k$, the statement holds for all integers $0 \leq i \leq k$. We are to show that the statement holds for $k+1$ as well.

        Note
        \begin{align*}
            x^m\ast x^{k+1} &= x^m\ast (x^k\ast x) & (\text{base case})\\
            &= (x^m \ast x^k) \ast x & (\text{associativity})\\
            &= x^{m+k}\ast x & (\text{by the }k^{\text{th}}\text{ case})\\
            &= x^{(m+k)+1} & (\text{base case})\\
            &= x^{m+(k+1)}
        \end{align*}
        so the statement also holds for $k+1$.

        Hence, by mathematical induction, we have $x^m \ast x^n = x^{m+n}$ for all non-negative integers $m$ and $n$.
        
        \item We again use a proof by induction by inducting on $n$.
        
        When $n = 0$, one sees
        \begin{align*}
            \left(x^m\right)^0 &= e & (\text{definition of }g^0 \text{ for any }g\in G)\\
            &= x^0 & (\text{definition of }x^0)\\
            &= x^{m \times 0}
        \end{align*}
        so the case when $n = 0$ is true.

        Now assume that the statement holds for some non-negative integer $k$, i.e. $\left(x^m\right)^k = x^{mk}$ for all non-negative integers $m$. We are to show that the statement holds for $k+1$, i.e. $\left(x^m\right)^{k+1} = x^{m(k+1)}$.

        We see that
        \begin{align*}
            \left(x^m\right)^{k+1} &= \left(x^m\right)^k\ast x^m & (\text{by statement 1})\\
            &= x^{mk} \ast x^m & (\text{by hypothesis})\\
            &= x^{mk+k} & (\text{by statement 1})\\
            &= x^{m(k+1)}
        \end{align*}
        so the statement holds for $k+1$.

        Hence, by mathematical induction, we have $\left(x^m\right)^n = x^{mn}$ for all non-negative integers $m$ and $n$.

        \item Left as \myref{exercise-swap-inverse-with-power} (later).
    \end{enumerate}
    This proves the proposition.
\end{proof}
\begin{exercise}\label{exercise-swap-inverse-with-power}
    Prove that $(x^{-1})^n = (x^n)^{-1}$ for all non-negative integers $n$.
\end{exercise}

\section{Order of a Group and Order of an Element}
We look at the notion of the \textit{order}\index{order} of a group and the order of an element of a group.

\begin{definition}
    Let $G$ be a group. The order of a group\index{order!group}, denoted by $|G|$, is the cardinality of the set $G$.
\end{definition}

If $|G| = n$ where $n$ is finite, we say that $G$ is a \textbf{finite group}. On the other hand, if $|G| = \infty$, then we say that $G$ is an \textbf{infinite group}.

\begin{example}
    The group $(\Z_4, \oplus_4)$ has order 4 since it has four elements, namely 0, 1, 2, and 3.
\end{example}

\begin{example}
    The group $(\R, +)$ is an infinite group since $\R$ has an uncountably infinite number of elements.
\end{example}

Let's now look at the order of an element.
\begin{definition}
    Let $g$ be an element of the group $G$. Then the order of $g$\index{order!element}, denoted by $|g|$, is the least positive integer $n$ such that $g^n = e$.
\end{definition}
Note that if $n$ is infinite, we say that the order of $g$ is infinite (or that $g$ has infinite order).

\begin{example}
    Consider the group $G = (\Z_4, \oplus_4)$ and its 4 elements 0, 1, 2, and 3.
    \begin{itemize}
        \item The element $0$ has order 1 since $0^1 = 0$, which is the identity. Thus $|0| = 1$ in $G$.
        \item The element $1$ has order 4 since $1 \oplus_4 1 \oplus_4 1 \oplus_4 1 = 0$ and no smaller $n$ than 4 exists. Thus $|1| = 4$ in $G$.
        \item The element $2$ has order 2 since $2 \oplus_4 2 = 0$ and no smaller $n$ than 2 exists. Thus $|2| = 2$ in $G$.
        \item The element $3$ has order 4 since $3 \oplus_4 3 \oplus_4 3 \oplus_4 3 = 0$ and no smaller $n$ than 4 exists. Thus $|3| = 4$ in $G$.
    \end{itemize}
\end{example}

\newpage

We note a few things about the order of elements in a group.
\begin{itemize}
    \item The identity element has order 1.
    \item A group where every element has finite order is a \textbf{periodic group}\index{group!periodic}.
    \item A finite group is always periodic since all elements in it has finite order.
    \item The order of any element in a group divides the order of the group.
\end{itemize}
The last point is actually a consequence of Lagrange's theorem (\myref{thrm-lagrange}). We will look into its proof in the next chapter.

\begin{exercise}
    Let $i$ be a number such that $i^2 = -1$. Let $\mathcal{S} = \{1, -1, i, -i\}$.
    \begin{partquestions}{\roman*}
        \item Find the identity of the group $(\mathcal{S}, \times)$ where $\times$ denotes regular multiplication.
        \item Find the orders of the elements of the above group.
    \end{partquestions}
\end{exercise}

\section{Cyclic Groups}\index{cyclic group}
Now that we have gotten some basic terminology and properties of groups out of the way, let's introduce a very simple type of group: the cyclic groups.

\begin{definition}
    Let $G$ be a group and $g$ be in $G$. If
    \[
        G = \{g, g^2, g^3, \dots, g^n\}
    \]
    for some positive integer $n$, then $G$ is a \textbf{cyclic group of order $n$}\index{cyclic group!of order $n$} and has a \textbf{generator $g$}\index{cyclic group!generator}, and is written as $G = \langle g \rangle$.
\end{definition}

\begin{example}
    Let $i$ be the imaginary unit, i.e. $i^2 = -1$. Let $\mathcal{S} = \{1, -1, i, -i\}$. Notice the group $(\mathcal{S}, \times)$ is completely generated by the element $i$ since
    \[
    i^1 = i,\; i^2 = -1,\; i^3 = -i, \text{ and } i^4 = 1.
    \]
    Thus, $\mathcal{S} = \{i, i^2, i^3, i^4\} = \langle i \rangle$.
\end{example}

\begin{exercise}
    Using the set $\mathcal{S}$ from the above example, find the other generator of the group $(\mathcal{S}, \times)$.
\end{exercise}

It should be noted that not every element in a cyclic group is a generator, and that a cyclic group may have more than 1 generator.

Cyclic groups may also be of \textbf{infinite order}. Such cyclic groups are called \textbf{cyclic groups of infinite order} or \textbf{infinite cyclic groups}\index{cyclic group!infinite}.
\begin{example}
    The group $(\Z, +)$ is an infinite cyclic group with generators 1 and -1.
\end{example}

\newpage

We now look at two results involving cyclic groups.
\begin{proposition}\label{prop-cyclic-group-is-abelian}
    Every cyclic group is abelian.
\end{proposition}
\begin{proof}
    Let $G$ be the cyclic group with a generator $g$. Suppose $x$ and $y$ are elements in $g$. Then $x = g^m$ and $y = g^n$ for some positive integers $m$ and $n$.

    Thus,
    \begin{align*}
        xy &= (g^m)(g^n)\\
        &= g^mg^n\\
        &= g^{m+n}\\
        &= g^{n+m} & (\text{+ is commutative})\\
        &= g^ng^m\\
        &= (g^n)(g^m)\\
        &= yx
    \end{align*}
    so $xy = yx$. Therefore $G$ is abelian.
\end{proof}

\begin{theorem}\label{thrm-cyclic-group-has-element-with-same-order}
    A finite group $G$ is cyclic if and only if there exists an element $g$ in $G$ with the same order as the group.
\end{theorem}
\begin{proof}
    We first prove the forward direction: suppose $G$ is cyclic and $|G| = n$. Then, by definition, there exists an element $g$ in $G$ such that
    \[
        G = \langle g \rangle = \{g^k \vert 1 \leq k \leq n, k \in \Z\},
    \]
    i.e. $g$ is a generator of $G$. We just need to show $|g| = n$. Suppose on the contrary there exists an integer $1 \leq m < n$ where $g^m = e$. Then $\langle g \rangle = \{g, g^2, \dots, g^m\}$. Thus $|\langle g \rangle| = m < n = |G|$. But by the hypothesis of the forward direction, $G = \langle g \rangle$ so $n = |G| = |\langle g \rangle| = m$. This is a contradiction, i.e. there does \textbf{not} exist an integer $1 \leq m < n$ where $g^m = e$. Therefore $g^n = e$, i.e. $|g| = n$.

    We now prove the reverse direction: suppose there is an element $g$ in $G$ with order $n$.
    
    We claim that $g, g^2, \dots, g^n$ are all distinct. Suppose on the contrary that there exist integers $i$ and $j$ where $1 \leq i < j \leq n$ such that $g^i = g^j$. Then $g^i = g^ig^{j-i}$, which means $g^{j-i} = e$ by the cancellation law. Note that $1 \leq j - i < n$. Thus, since $g^{j-i} = e$, therefore $|g| = j - i < n$ which contradicts $|g| = n$. Hence, $g, g^2, \dots, g^n$ are all distinct. Therefore, $\langle g \rangle = \{g, g^2, \dots, g^n\}$ contain distinct elements of $G$. But there are only $n$ elements in $G$ and $\langle g \rangle$ contains $n$ distinct elements. Therefore, $G = \langle g \rangle$ which means that $G$ is cyclic with generator $g$.

    This completes the proof.
\end{proof}

There are more interesting properties of cyclic groups, but we will get to them when we develop more tools to explain and prove these properties.

\section{Dihedral Groups}
We motivate the definition of the dihedral groups by discussing the symmetries of an equilateral triangle.

\begin{figure}[h]
    \centering
    \fbox{\includegraphics[width=0.4\textwidth]{basics-of-groups/d3.jpg}}
    \caption{Symmetries of an Equilateral Triangle}
\end{figure}

What actions could we perform in order to maintain symmetry on an equilateral triangle? Well, we could rotate the triangle in $120^\circ$ anti-clockwise increments about the center of the triangle. We denote this action by the symbol $r$. Another thing we could do is reflect the triangle about the line going through one of the vertices and the center, like we discussed in an earlier chapter. This action is denoted by $s$.

Now, suppose we define $r$ to be the $120^\circ$ anti-clockwise rotation about the center and $s$ be the reflection of the triangle about the line going through vertex 1 and the center, like shown in the diagram. How do we obtain a $240^\circ$ anti-clockwise rotation? Well, we apply two $120^\circ$ anticlockwise rotations one after another. In other words, if $\ast$ means ``action composition'', then a $240^\circ$ rotation would be represented by $r^2$. Note that $r^3$, which represents a $360^\circ$ anti-clockwise rotation, is the same as doing nothing. So $r^3 = e$. Similarly, applying the reflection $s$ twice in a row (i.e., $s^2$) is the same as doing nothing, so $s^2 = e$. Thus, we have
\[
    r^3 = s^2 = e
\]
for the case of an equilateral triangle.

There's another relationship governing $r$ and $s$. Consider this: how do we obtain a reflection about the line through vertex 3 and the center? Well, we apply $r$ first, followed by $s$. This means that a reflection about the line through vertex 3 and the center is given by $rs$. Notice that this is the same thing as reflecting first and then applying $r$ twice, i.e. $sr^2$. Thus, we have the second relationship:
\[
    rs = sr^2
\]
for the case of an equilateral triangle.

The group of symmetries of an equilateral triangle is called the \textbf{dihedral group of order 6} (or the \textbf{dihedral group of degree 3}) and is denoted by $D_3$. In general, the \textbf{dihedral group of degree $n$}\index{dihedral group!of degree $n$} (or the \textbf{dihedral group of order $2n$}\index{dihedral group!of order $2n$}) is denoted by $D_n$ and can be thought of as the symmetries of a regular polygon of $n$ sides (a regular $n$-gon).

\newpage

\begin{example}
    The symmetries of the square is given by the group $D_4$.
\end{example} 
\begin{figure}[h]
    \centering
    \fbox{\includegraphics[width=0.4\textwidth]{basics-of-groups/d4.jpg}}
    \caption{Symmetries of a Square}
\end{figure}

Thus, in general, the set $D_n$ consists of the following elements.
\[
    D_n = \{e, r, r^2, \dots, r^{n-1}, s, rs, r^2s, \dots, r^{n-1}s\}
\]
with the relationship between $r$ and $s$ given by $r^n = s^2 = e$ and $rs = sr^{n-1}$.

\begin{remark}
    Some authors (e.g. {\cite[p.~25]{dummit_foote_2004}}) will write the reflections of $D_n$ with $s$ leading $r$, i.e. $s, sr, sr^2, sr^3, \dots, sr^{n-1}$. The underlying definition, however, remains the same in either case. This fact will become evident with the proof of \myref{prop-Dn-cannonical-form} (later).
\end{remark}

These relationships are succinctly given by the following \textit{presentation}\index{presentation}, and is the actual definition of the dihedral group.
\begin{definition}
    The \textbf{dihedral group of degree $n$}\index{dihedral group!of degree $n$} (or the \textbf{dihedral group of order $2n$}\index{dihedral group!of order $2n$}) is
    \[
        D_n = \langle r, s \vert r^n = s^2 = e,\;rs = sr^{n-1} \rangle.
    \]
\end{definition}
\begin{remark}
    In the above definition, $r$ and $s$ can be thought of as `generators'\index{presentation!generators} and the conditions\index{presentation!conditions} are given on the right side of the pipe ($|$).
\end{remark}

\begin{example}\label{example-presentation-of-D3}
    The group $D_3$ has presentation
    \[
        D_3 = \langle r, s \vert r^3 = s^2 = e,\;rs = sr^2 \rangle.
    \]
    The Cayley table of $D_3$ is given below.

    \begin{table}[h]
        \centering
        \begin{tabular}{|l|l|l|l|l|l|l|}
        \hline
        $\boldsymbol{\ast}$ & $\boldsymbol{e}$ & $\boldsymbol{r}$ & $\boldsymbol{r^2}$ & $\boldsymbol{s}$ & $\boldsymbol{rs}$ & $\boldsymbol{r^2s}$ \\ \hline
        $\boldsymbol{e}$    & $e$    & $r$    & $r^2$  & $s$    & $rs$   & $r^2s$ \\ \hline
        $\boldsymbol{r}$    & $r$    & $r^2$  & $e$    & $rs$   & $r^2s$ & $s$    \\ \hline
        $\boldsymbol{r^2}$  & $r^2$  & $e$    & $r$    & $r^2s$ & $s$    & $rs$   \\ \hline
        $\boldsymbol{s}$    & $s$    & $r^2s$ & $rs$   & $e$    & $r^2$  & $r$    \\ \hline
        $\boldsymbol{rs}$   & $rs$   & $s$    & $r^2s$ & $r$    & $e$    & $r^2$  \\ \hline
        $\boldsymbol{r^2s}$ & $r^2s$ & $rs$   & $s$    & $r^2$  & $r$    & $e$    \\ \hline
        \end{tabular}
    \end{table}

    We use the convention of reading the \textbf{row before the column}, so the action $rs \ast r^2$ (which is usually written as $rsr^2$) is given by the row of $rs$ and the column of $r^2$, which is $r^2s$.
\end{example}

Such a definition for the dihedral groups removes the need to associate the group $D_n$ with any idea of an $n$-gon, and focuses solely on the relationship between $r$ and $s$. In fact, the removal of reliance on $n$-gons help define the non-intuitive groups $D_1$ and $D_2$.
\begin{example}
    The group $D_1$ has presentation
    \[
        D_1 = \langle r, s \vert r = s^2 = e, rs = s \rangle,
    \]
    which quickly means that $D_1$ has only two elements, $e = r$ and $s$. Thus $D_1$ is, in fact, just the cyclic group of order 2.
\end{example}

\begin{example}
    The group $D_2$ has presentation
    \[
        D_2 = \langle r, s \vert r^2 = s^2 = e, rs = sr \rangle.
    \]
    So we see that $D_2 = \{e, r, s, rs\}$.
    
    \begin{table}[h]
        \centering
        \begin{tabular}{|l|l|l|l|l|}
            \hline
            $\boldsymbol{\ast}$ & $\boldsymbol{e}$ & $\boldsymbol{r}$ & $\boldsymbol{s}$ & $\boldsymbol{rs}$ \\ \hline
            $\boldsymbol{e}$    & $e$              & $r$              & $s$              & $rs$              \\ \hline
            $\boldsymbol{r}$    & $r$              & $e$              & $rs$             & $s$               \\ \hline
            $\boldsymbol{s}$    & $s$              & $rs$             & $e$              & $r$               \\ \hline
            $\boldsymbol{rs}$   & $rs$             & $s$              & $r$              & $e$               \\ \hline
        \end{tabular}
    \end{table}
    
    In fact, we can see from the above Cayley table that $D_2$ is indeed abelian.
\end{example}

The \textit{canonical form}\index{dihedral group!canonical form} of an element in a dihedral group is $r^ms^n$, where $m$ and $n$ are non-negative integers. So how do we find the canonical form of elements like $sr$, $srs$, or $rsr^3$? We have this useful proposition to help.
\begin{proposition}\label{prop-Dn-cannonical-form}
    In the group $D_n$, we have $r^ms = sr^{n-m}$ for all integers $1 \leq m < n$.
\end{proposition}
\begin{proof}
    We induct on $m$. When $m = 1$, $rs = sr^{n-1}$ by the definition of $D_n$. Assume now that for some integer $1 \leq k < n$, we have $r^ks = sr^{n-k}$. We consider two cases.
    \begin{itemize}
        \item If $k = n - 1$, then $k + 1 = n$. Thus, $r^{k+1}s = r^ns = s$ since $r^n = e$. Note that $sr^{(k+1)-n} = sr^{n-n} = sr^0 = s$. Therefore $r^{k+1}s = sr^{n-(k+1)}$.
        \item The other case is if $1 \leq k \leq n - 2$. Then we have
        \begin{align*}
            r^{k+1}s &= r^k(rs)\\
            &= r^k(sr^{n-1}) & (\text{base case})\\
            &= (r^ks)r^{n-1} & (\text{associativity})\\
            &= (sr^{n-k})r^{n-1} & (\text{by induction hypothesis})\\
            &= sr^{2n - k - 1}\\
            &= sr^nr^{n-k-1}\\
            &= sr^{n-(k+1)} & (\text{since } r^n = e)
        \end{align*}
        which means $r^{k+1}s = sr^{n-(k+1)}$.
    \end{itemize}
    In either case, the statement is true for $k+1$.

    Therefore $r^ms = sr^{n-m}$ for all integers $1 \leq m < n$.
\end{proof}

\begin{exercise}
    Simplify $rsr^4sr^3$ in the group $D_6$.
\end{exercise}

\newpage

\section{Problems}
\begin{problem}
    Draw the Cayley table for $D_4$, the dihedral group of order 8, representing the symmetries of a square.\newline
    By referring to the Cayley table,
    \begin{partquestions}{\alph*}
        \item explain why $D_4$ is \textbf{not} abelian;
        \item simplify $r^3srsr^3sr^3sr^2$.
    \end{partquestions}
\end{problem}

\begin{problem}
    If every element in a group $G$ is its own inverse, show that $G$ is abelian.
\end{problem}

\begin{problem}\label{problem-element-to-power-of-multiple-of-order-is-identity}
    Let $G$ be a group with identity $e$. Suppose an element $x$ in $G$ has finite order $n$. Prove that a positive integer $m$ is a multiple of $n$ if and only if $x^m = e$.\newline
    (\textit{Hint: consider Euclid's division lemma (\myref{lemma-euclid-division}) to prove one direction of the claim.})
\end{problem}

\begin{problem}
    Let $G$ be a group.
    \begin{partquestions}{\alph*}
        \item Suppose $(gh)^2 = g^2h^2$ for all elements $g$ and $h$ in $G$. Prove that $G$ is abelian.
        \item Suppose $G$ is abelian. Prove that $(gh)^n = g^nh^n$ for all elements $g$ and $h$ in $G$ and for all positive integers $n$.
    \end{partquestions}
\end{problem}

\begin{problem}
    Let the group $G = (\Z_n, \oplus_n)$. Show that $G$ is cyclic with order $n$.
\end{problem}

\begin{problem}
    Let the set $S = \R^2$, that is,
    \[
        S = \{(x, y) \vert x, y \in \R\}.
    \]
    Let the transformation $T: S \to S$ be defined by
    \[
        T(x, y) = (-y, x+y).
    \]
    Define the set $A = \{T^r \vert r \in \Z \text{ and } r \geq 1\}$. Show that $A$ is a group under function composition ($\circ$), and state the order of this group.
\end{problem}

\section{Subgroups}
\begin{questions}
    \item We note that $G$ contains $\{e, r, r^2, r^3, s, rs, r^2s, r^3s\}$.
    \begin{partquestions}{\alph*}
        \item Yes, this is the trivial subgroup.
        \item No, it is not closed. ($rs$ can be generated by $r \ast s$ but is not in the set)
        \item No, the identity $e$ is missing.
        \item Yes, $\{r, r^3, r^4, r^6\} = \{r, r^3, e, r^2\} = \langle r \rangle$.
    \end{partquestions}

    \item \begin{partquestions}{\alph*}
        \item Clearly $e \in K$ since $e^2 = e \in H$.
        
        Let $x, y \in K$, so $x^2 \in H$ and $y^2 \in H$. We note $y^{-1} \in K$ since $(y^{-1})^2 = (y^2)^{-1} \in H$. Therefore $(xy^{-1})^2 = xy^{-1}xy^{-1} \in H$, so $xy^{-1} \in K$. Hence $K \leq G$ by subgroup test.

        \item Note that the identity of $K$, $e$, is also the identity of $G$. Since $H \leq G$, thus $e \in H$. Since $H$ is a subgroup of $G$, thus for any $x$ and $y$ in $H$ we have $xy^{-1} \in H$. Hence, by subgroup test, $H \leq K$.
    \end{partquestions}

    \item \begin{partquestions}{\alph*}
        \item We have proved that $\Z{G} \leq G$ so we only prove normality. Let $g$ and $z$ be arbitrary elements from $G$ and $\mathrm{Z}(G)$ respectively. Then
        \begin{align*}
            gzg^{-1} &= g(zg^{-1})\\
            &= g(g^{-1}z) & (\text{since }z \in \mathrm{Z}(G))\\
            &= (gg^{-1})z \\
            &= z\\
            &\in \mathrm{Z}(G)
        \end{align*}
        which proves that $\mathrm{Z}(G) \unlhd G$.

        \item We first work in the forward direction by assuming $G = \Z{G}$. Then for all $z \in \Z{G} = G$ we have $gz = zg$ for any $g \in G$ by definition, which means that $G$ is abelian.

        We now work in the reverse direction by assuming that $G$ is abelian. Note $\mathrm{Z}(G) = \{z \in G \vert gz = zg \text{ for all } g \in G\}$. But since $G$ is abelian, $gh = hg$ for all $g$ and $h$ in $G$. Thus every element in $G$ satisfies the condition to be in the center of $G$, meaning $\mathrm{Z}(G) = G$.

        \item We note that $D_4 = \{e, r, r^2, r^3, s, rs, r^2s, r^3s\}$. Since $\mathrm{Z}(D_4)$ is a subgroup of $D_4$ it has a maximum order of 2, by Lagrange's theorem (\myref{thrm-lagrange}). Since 2 is prime the subgroups must be cyclic. Thus the non-trivial subgroups of $D_4$ are $\{e, r^2\}$ and $\{e, s\}$ (since $|r^2| = |s| = 2$). Now like how we proved that $\langle s \rangle = \{e, s\}$ is not a normal subgroup in $D_3$ in \myref{example-normal-subgroups-of-d3}, $\{e, s\}$ is not a normal subgroup of $D_4$. One verifies easily that $\{e, r^2\} = \langle r^2 \rangle$ is a normal subgroup of $D_4$. Thus $\mathrm{Z}(D_4) = \langle r^2 \rangle$ since $\mathrm{Z}(D_4)$ must be a normal subgroup of $D_4$ with order not exceeding 2.
    \end{partquestions}

    \item \begin{partquestions}{\alph*}
        \item We will prove this statement.

        Clearly $e \in H \cap K$ since $e \in H$ and $e \in K$ as both are subgroups of $G$.

        Let $x$ and $y$ be in $H \cap K$, meaning that $x, y \in H$ and $x, y \in K$. Thus $xy^{-1} \in H$ and $xy^{-1} \in K$ as both are subgroups of $G$. Hence $xy^{-1} \in H \cap K$.
        By subgroup test, $H \cap K \leq G$.

        \item We will prove this statement. One sees that $H \cap K \subseteq H$. Since $H \cap K \leq G$, it is thus a group. Hence $H \cap K \leq H$ by definition of a subgroup.

        \item We will disprove this statement. Consider:
        \begin{align*}
            &G = \mathbb{Z}_6 \text{ under }\oplus_6,\\
            &H = \{0, 2, 4\},\text{ and}\\
            &K = \{0, 3\}.
        \end{align*}
        Clearly $H \leq G$ and $K \leq G$. Note $H \cup K = \{0, 2, 3, 4\}$. But $H \cup K$ is not closed since $2 \oplus_6 3 = 5 \not \in H \cup K$. Hence $H \cup K \not\leq G$.

        \item We will disprove this statement. Since $H \cup K$ is not closed it is not a group, meaning it cannot be a subgroup.
    \end{partquestions}

    \item By Lagrange's Theorem (\myref{thrm-lagrange}), the order of a subgroup must divide the order of the group. Since $H \leq G$ is non-trivial, and since $1024 = 2^{10}$, the largest order that $H$ can be is $512$ with $[G:H] = 2$. An example is $G = \mathbb{Z}_{1024}$ and $H = \langle 2 \rangle$, since $|H| = |2| = 512$ as $2 \times 512 = 1024 \equiv 0 \pmod{1024}$.

    \item Let $|G| = 2n$. The identity is its own inverse, leaving an odd number of non-identity elements.
    
    Suppose $x$ is an element of $G$ with $|x| > 2$; we cannot have $x^{-1} = x$ (otherwise $x^2 = e$). Thus $x^{-1}$ and $x$ are distinct. Pair every one of these $x$'s with its inverse $x^{-1}$.

    Remember that there is an odd number of non-identity elements. Hence, there must be at least one element which has not been paired off with any of the others, which is therefore its own self inverse.

    Since this element is not the identity, thus it has to have order 2 (as $g^{-1} = g$ implies $g^2 = e$).

    \item Suppose $G = \langle g \rangle$ and $H \leq G$. Then any element in $H$ is of the form $g^a$ where $a$ is an integer. Suppose $m$ is the smallest positive integer $m$ such that $g^m \in H$. Suppose now $g^n \in H$ for some $n$. By Euclid's division lemma (\myref{lemma-euclid-division}), $n = mq + r$ where $q$ and $r$ are non-negative integers such that $0 \leq r < m$. Hence,
    \[
        g^n = g^{mq}g^r = (g^m)^q g^r.
    \]
    Now, $m$ is the smallest positive integer such that $g^m \in H$. This means that if $r \neq 0$, $g^r \not\in H$ as $0 \leq r < m$. Hence, $r = 0$, which means
    \[
        g^n = (g^m)^q.
    \]
    Thus, every element in the subgroup $H$ can be formed by applying $g^m$ a certain number of times, meaning $H$ is cyclic with generator $g^m$.

    \item \begin{partquestions}{\roman*}
        \item Let $xH$ be a coset in $G$. Since cosets partition $G$, either $xH = H$ or $xH = G \setminus H$ (since there are only two distinct cosets).
    \begin{itemize}
            \item If $xH = H$, then $x \in H$, meaning $xH = H = Hx$.
            \item If $xH \neq H$, then $x \in G \setminus H$. Hence $xH = G \setminus H = Hx$.
    \end{itemize}
    Therefore $H$ is a normal subgroup of $G$, i.e. $H \lhd G$.
        \item Note that $G/H$ is a group since $H \lhd G$. Also since $[G:H] = 2$ thus $|G/H| = 2$.
        
        If $x \in G$ then $x^2H = (xH)^2 = H$ since the order of $G/H$ is 2, meaning that any non-identity element inside it (like $xH$) has an order of at most 2. Since $x^2H = H$, therefore by Coset Equality (\myref{lemma-coset-equality}), statements 1 and 4, we must have $x^2 \in H$.
        
        \item Suppose $x$ has odd order. Write $|x| = 2k - 1$ where $k$ is a positive integer. Hence $x^{2k-1} = e \in H$ since $H < G$. Therefore
        \[
            x = x^{2k} = \left(x^k\right)^2 \in H        
        \]
        which means $x \in H$.
    \end{partquestions}

    \item \begin{partquestions}{\alph*}
        \item Suppose we have an element $x \in H \cap K$, meaning that $x \in H$ and $x \in K$. By a corollary of Lagrange's Theorem (\myref{corollary-order-of-group-multiple-of-order-of-element}), the order of $x$ must divide the order of its group. Hence, $|x|$ divides $|H|$ and $|x|$ divides $|K|$ simultaneously, meaning that $|x| = \gcd(|H|, |K|)$. But the GCD of the orders of both subgroups is 1. Hence, $|x| = 1$, meaning the only element in the intersection $H \cap K$ is the identity $e$.
        
        \item Consider $hkh^{-1}k^{-1}$.
        \begin{itemize}
            \item On one hand, note that $hkh^{-1}k^{-1} = h(kh^{-1}k^{-1})$. Clearly $h \in H$ and $kh^{-1}k^{-1} \in H$ by normality of $H$. Therefore $hkh^{-1}k^{-1} \in H$.
            \item On another hand, $hkh^{-1}k^{-1} = (hkh^{-1})k^{-1}$. Note $hkh^{-1} \in K$ by normality of $K$ and $k^{-1} \in K$, so $hkh^{-1}k^{-1} \in K$.
        \end{itemize}
        Therefore $hkh^{-1}k^{-1} \in H \cap K$. But by \textbf{(a)}, the only element in $H \cap K$ is the identity. Thus, $hkh^{-1}k^{-1} = e$ which the result follows quickly.
    \end{partquestions}
    
    \item \begin{partquestions}{\alph*}
        \item $m = 6$.
        \item We first prove that all groups of order less than 6 are abelian, and then find a non-abelian group of order 6.

        We note that a group of order 1 is the trivial group which is abelian. The groups of order 2, 3, and 5 are groups of prime order, meaning that they are cyclic and hence abelian. We are left with a group of order 4.

        We note that the order of an element of a group of order 4 must divide 4 (\myref{corollary-order-of-group-multiple-of-order-of-element}). Hence the possible orders of an element in such a group is 1, 2, or 4. An element of order 1 is the identity. If an element with order 4 exists, then the group is cyclic and hence abelian. So we assume that all elements are either order 1 or order 2 (in fact, the orders must be 1, 2, 2, 2). This is precisely the group
        \[
            D_2 = \langle r, s \vert r^2 = s^2 = e, rs = sr\rangle
        \]
        which is abelian. Hence all groups of order 4 are abelian.

        We now show that a group of order 6 can be non-abelian. We note that the group
        \[
            D_3 =  \langle r, s \vert r^3 = s^2 = e, rs = sr^2\rangle
        \]
        has order 6 and because $rs = sr^2 \neq sr$, thus $D_3$ is non-abelian. Hence $m = 6$.

        \item For all even $n \geq 6$, the group $D_{\frac n2}$ has $n$ elements and $rs = sr^{\frac n2 - 1} \neq sr$, so $D_{\frac n2}$ is non-abelian.
    \end{partquestions}

    \item Suppose $G / \Z{G}$ is cyclic. Then by definition, $G / \Z{G} = \langle g\Z{G}\rangle$ for some $g \in G$, and any element in $G/\Z{G}$ is of the form $g^n\Z{G}$.

    Now take $x, y \in G$. By \myref{lemma-left-coset-partition}, left cosets partition the group, so we may assume $x \in g^m\Z{G}$ and $y \in g^n\Z{G}$, meaning $x = g^mz_1$ and $y = g^nz_2$ for some $z_1, z_2 \in \Z{G}$. We note
    \begin{align*}
        xy &= (g^mz_1)(g^nz_2)\\
        &= g^m(z_1g^n)z_2\\
        &= g^m(g^nz_1)z_2 & (\text{since }z_1 \in \Z{G})\\
        &= (g^mg^n)(z_1z_2)\\
        &= g^{m+n}z_1z_2\\
        &= g^{n+m}z_2z_1\\
        &= g^ng^mz_2z_1\\
        &= g^n(g^mz_2)z_1\\
        &= g^n(z_2g^m)z_1\\
        &= (g^nz_2)(g^mz_1)\\
        &= yx
    \end{align*}
    which means that $xy = yx$ for any $x, y \in G$. Hence $G$ is abelian.
\end{questions}

\section{Homomorphisms and Isomorphisms}
\begin{questions}
    \item We will prove that $f$ is a homomorphism, is injective, and is surjective.
    \begin{itemize}
        \item \textbf{Homomorphism}: Let $x, y \in G$. Then
        \begin{align*}
            f(xy) &= g(xy)g^{-1}\\
            &= (gxg^{-1})(gyg^{-1})\\
            &= f(x)f(y)
        \end{align*}
        which means that $f$ is a homomorphism.
        \item \textbf{Injective}: Let $x, y \in G$ be such that $f(x) = f(y)$. Then $gxg^{-1} = gyg^{-1}$. By cancellation law, $x = y$.
        \item \textbf{Surjective}: Suppose $y \in G$. Set $x = g^{-1}yg$. Since $G$ is closed, thus $x \in G$. Note $f(x) = g(g^{-1}yg)g^{-1} = y$. Hence $y$ has a pre-image of $x = g^{-1}yg$ in $G$.
    \end{itemize}
    Therefore $f$ is an isomorphism.

    \item Suppose on the contrary there exists an isomorphism $\phi: G \to H$. Since $\phi$ is an isomorphism, it is surjective. Hence, there must exists a rational number $r \in G$ such that $\phi(r) = 2$. As $r$ is rational, so is $\frac r2$.

    Now consider $\phi\left(\frac r2 + \frac r2\right)$. On one hand, $\phi\left(\frac r2 + \frac r2\right) = \phi(r) = 2$. On another hand, $\phi(\frac r2 + \frac r2) = \left(\phi\left(\frac r2\right)\right)^2$ as $\phi$ is a homomorphism. Therefore, $\left(\phi\left(\frac r2\right)\right)^2 = 2$ which quickly implies $\phi\left(\frac r2\right) = \sqrt 2$ since $\phi\left(\frac r2\right)$ must be positive. However, $\sqrt 2 \notin H$ while $\phi\left(\frac r2\right) \in H$, a contradiction.

    Hence, $G \not\cong H$.

    \item \begin{partquestions}{\alph*}
        \item Let $m, n \in G$. Then
        \[
            \phi(m + n) = 2(m + n) = 2m + 2n = \phi(m) + \phi(n)
        \]
        which means $\phi$ is a homomorphism.

        \item Suppose $m, n \in G$ such that $\phi(m) = \phi(n)$. Then $2m = 2n$. Clearly this means that $m = n$. Thus $\phi$ is injective.

        \item Suppose on the contrary there existed a homomorphism $\psi: H \to G$ such that $\psi(\phi(n)) = n$. Then $\psi(2n) = n$ by definition of $\phi$. Note that
        \[
            \psi(2n) = \psi(n + n) = \psi(n) + \psi(n) = 2\psi(n)
        \]
        since $\psi$ is a homomorphism. Hence $2\psi(n) = n$ which implies that $\psi(n) = \frac n2$. But for the case of $n = 1$, $\psi(1) = \frac 12 \notin G$. Hence $\psi$ does not exist.
    \end{partquestions}

    \item We prove the forward direction first: assume that $G$ is abelian. Then $f$ is a homomorphism since
    \[
        f(gh) = (gh)^{-1} = h^{-1}g^{-1} = g^{-1}h^{-1} = f(g)h(g).
    \]

    We now prove the reverse direction: assume that $f$ is a homomorphism, meaning $f(gh) = f(g)f(h) = g^{-1}h^{-1}$. But $f(gh) = (gh)^{-1} = h^{-1}g^{-1}$. Therefore we have $g^{-1}h^{-1} = h^{-1}g^{-1}$ which clearly shows that the group is abelian.

    \item Suppose $\phi: G \to H$ is a surjective homomorphism and $G$ is abelian. Since $\phi$ is surjective, thus $\im \phi = H$. Let $g_1, g_2 \in G$ and $h_1, h_2 \in H$ such that $\phi(g_1) = h_1$ and $\phi(g_2) = h_2$. Consider $\phi(g_1g_2)$.
    \begin{itemize}
        \item On one hand, $\phi(g_1g_2) = \phi(g_1)\phi(g_2) = h_1h_2$.
        \item On another hand, $\phi(g_1g_2) = \phi(g_2g_1) = \phi(g_2)\phi(g_1) = h_2h_1$.
    \end{itemize}
    Hence $h_1h_2 = h_2h_1$ which means that $H$ is abelian.

    \item We first prove $\phi(N)$ is a subgroup of $H$ by using subgroup test before proving normality.

    Note that $e_H \in \phi(N)$ since $e_G \in N$ and $\phi(e_G) = e_H$. Now let $x, y \in \phi(N)$. As $\phi$ is surjective, we know that there exists $n_x, n_y \in N$ where $\phi(n_x) = x$ and $\phi(n_y) = y$. Note that $\phi(n_y^{-1}) = y^{-1}$ and $n_xn_y^{-1} \in N$. Hence, $xy^{-1} = \phi(n_xn_y^{-1}) \in \phi(N)$. By subgroup test, $\phi(N) \leq H$.

    We now show that $\phi(N)$ is a normal subgroup of $H$. Take $g \in G$, $h \in H$, $n \in N$, and $x \in \phi(N)$, such that $\phi(g) = h$ and $\phi(n) = x$. Note that since $N \unlhd G$, thus $gng^{-1} \in N$. Therefore,
    \begin{align*}
        hxh^{-1} &= \phi(g)\phi(n)\phi(g^{-1})\\
        &= \phi(\underbrace{gng^{-1}}_{\text{In }N})\\
        &\in \phi(N)
    \end{align*}
    which means that $\phi(N) \unlhd H$.

    \item Consider the map $\phi: G \to H, a \mapsto a + n\mathbb{Z}$. We show that $\phi$ is an isomorphism:
    \begin{itemize}
        \item \textbf{Homomorphism}: Let $a$ and $b$ be in $G$. Then
        \begin{align*}
            \phi(a\oplus_n b) &= (a\oplus_n b) + n\mathbb{Z}\\
            &= \{(a \oplus_n b) + pn \vert p \in \mathbb{Z}\}\\
            &= \{a+b + pn \vert p \in \mathbb{Z}\}\\
            &= \{a+b + pn + qn\vert p, q \in \mathbb{Z}\}\\
            &= a+b+n\mathbb{Z} + n\mathbb{Z}\\
            &= (a+n\mathbb{Z}) + (b + n\mathbb{Z})\\
            &= \phi(a) + \phi(b).
        \end{align*}
        \item \textbf{Injective}: Let $a$ and $b$ be in $G$ such that $\phi(a) = \phi(b)$. Thus
        \[
            \{a + pn \vert p \in \mathbb{Z} \} = \ \{b + qn \vert q \in \mathbb{Z} \}
        \]
        by definition of $\phi$. Hence $a \equiv b \pmod n$. But since $0 \leq a, b < n$, we must have $a = b$.
        \item \textbf{Surjective}: Let $x + n\mathbb{Z} \in H$. We use Euclid's division lemma (\myref{lemma-euclid-division}) on $x$ to yield
        \[
            x = qn + r, \text{ where } 0 \leq r < n.
        \]
        Note that
        \begin{align*}
            x + n\mathbb{Z} &= \{x + kn \vert k \in \mathbb{Z}\}\\
            &= \{(qn + r) + kn \vert k \in \mathbb{Z}\}\\
            &= \{r + n(\underbrace{q + k}_{\text{In }\mathbb{Z}}) \vertalt k \in \mathbb{Z} \}\\
            &= r + n\mathbb{Z}
        \end{align*}
        with $0 \leq r < n$, meaning $r \in G$. Now observe $\phi(r) = r+n\mathbb{Z} = x+n\mathbb{Z}$ which means that there is a pre-image for every element in $H$, hence proving that $\phi$ is surjective.
    \end{itemize}
    Therefore $\phi$ is an isomorphism, proving $G \cong H$.
    
    \item Consider the map $\phi: G \to G/N$ such that $g \mapsto gN$. We note that $\phi$ is a homomorphism as
    \[
        \phi(gh) = (gh)N = (gN)(hN) = \phi(g)\phi(H).
    \]
    We note by \myref{prop-homomorphism-inverse-is-subgroup} that $A = \phi^{-1}(B) \leq G$. Thus
    \begin{align*}
        \phi^{-1}(N) &= \{g \in G \vert \phi(g) = N\}\\
        &= \{g \in G \vert gN = N\}\\
        &= \{g \in G \vert g \in N\}\\
        &= G \cap N\\
        &= N\\
        &\subseteq A
    \end{align*}
    by assumption. Since $N$ is a group, we know $N \leq A$. Furthermore $N \leq A \leq G$ and $N \unlhd G$, meaning $N \unlhd A$ (since $gN = Ng$ for all $g \in G$, including those in $A$). Hence $A/N$ is a group.
    
    Now clearly $\phi$ is surjective (since for any $gN \in G/N$ we know $\phi(g) = gN$), which means that $\phi(\phi^{-1}(B)) = B$. Since $\phi^{-1}(B) = A$, so $\phi(A) = B$. Finally,
    \begin{align*}
        \phi(A) &= \{\phi(a) \vert a \in A\}\\
        &= \{aN \vert a \in A\}\\
        &= A/N
    \end{align*}
    which means $B = A/N$.
\end{questions}

\chapter{Symmetry Groups}
This chapter is central to the relevance and analysis of group theory. The core result of this chapter, Cayley's theorem, links our ideas of symmetry with the idea of groups, and how groups are a form of \textit{generalized symmetry}. It answers why group theory is oft called ``the study of symmetry'', and highlights the importance of bijections in the study of groups.

\section{Permutations}
A bijective function is too abstract an object. Such functions can take many forms. Thus, it is worth asking: what properties must a bijective function satisfy?

A bijective function is a function that maps all elements from one set to another set exactly. There are no leftovers (surjective), and each output has exactly one input that produces it (injective). In a sense, a bijective function \textit{rearranges} the elements in a set; it renames elements and shuffles them around, without destroying the relative relationships between the elements.

For bijections between finite sets, each set has the same number of elements, so it is reasonable to talk about the rearrangement and enumeration of elements in such sets.
\begin{itemize}
    \item What we mean by \textbf{rearrange} is to rename elements. We can give elements a new name and place it in the codomain.
    \item What we mean by \textbf{enumerate} is to assign each element in each finite set a unique `index number', per se. Each element can have a unique number identifying its original \textit{position} in the set, and its final position in the destination set.
\end{itemize}

Such bijections between finite sets are called \textbf{permutations}\index{permutations}, since they simply permute the `index number' of the elements in the sets.

\begin{example}
    Consider the set $S = \{1, 2, 3, 4, 5\}$. A bijection $f: S \to S$ could perform the following mapping:
    \begin{itemize}
        \item $1 \mapsto 2$;
        \item $2 \mapsto 4$;
        \item $3 \mapsto 3$;
        \item $4 \mapsto 5$; and
        \item $5 \mapsto 1$.
    \end{itemize}
    In this case, the function $f$ is said to be a permutation because the ordered list $[2, 4, 3, 5, 1]$ is one rearrangement of the items in the set $\{1, 2, 3, 4, 5\}$.
\end{example}

\begin{remark}
    It is certainly confusing that the operation of rearranging the items is also called \textit{permuting} the items in the set, and one such rearrangement is called a permutation. For groups, treat a ``permutation'' as a bijective function between finite groups.
\end{remark}

Permutations come in many different forms, but the core thing that they do is to rearrange items. From the above example, one could form a `cycle'\index{cycle} of how each item is mapped to another:
\begin{itemize}
    \item $1 \mapsto 2 \mapsto 4 \mapsto 5 \mapsto 1$; and
    \item $3 \mapsto 3$.
\end{itemize}
We can describe a permutation based on how it cycles elements. Consider this mapping performed by the map $\phi: S \to S$:
\begin{itemize}
    \item $1 \mapsto 2$;
    \item $2 \mapsto 4$;
    \item $3 \mapsto 5$;
    \item $4 \mapsto 1$; and
    \item $5 \mapsto 3$.
\end{itemize}
How $\phi$ operates on an element can be described in \textbf{cycle notation}\index{cycle!notation}. The cycle notation of a permutation may also be called its \textbf{cycle decomposition}\index{cycle!decomposition}. Here's how to describe a permutation in cycle notation.
\begin{enumerate}
    \item Start by opening a bracket: ``(''.
    \item Write the first element that has not appeared yet in the cycle notation.
    \begin{itemize}
        \item Initially, we write the number 1, so it currently is ``(1''.
    \end{itemize}
    \item Find out where that element is mapped to.
    \begin{itemize}
        \item For the case of the element 1, it is mapped to 2.
    \end{itemize}
    \item Write the mapped element next to the previous element.
    \begin{itemize}
        \item In this case, we will write ``(1 2''
    \end{itemize}
    \item Repeat steps 3 and 4 with the mapped element, until reaching an element that has already appeared in the cycle notation.
    \begin{itemize}
        \item Since 2 maps to 4, we write ``(1 2 4''
        \item Since 4 maps to 1, we terminate this process.
    \end{itemize}
    \item Close the bracket.
    \begin{itemize}
        \item So our first cycle looks like ``(1 2 4)''
    \end{itemize}
    \item Repeat steps 1 to 6 until all elements are used.
    \begin{itemize}
        \item So our final cycle notation for $g$ is ``(1 2 4)(3 5)''
    \end{itemize}
\end{enumerate}

We note some important things about this process.
\begin{itemize}
    \item Omit any elements that maps to itself. For example, if $1 \mapsto 3$, $2 \mapsto 6$, $3 \mapsto 4$, $4 \mapsto 1$, $5 \mapsto 5$, $6 \mapsto 2$, and $7 \mapsto 7$, then we ignore the 5 and 7; the corresponding cycle notation is ``(1 3 4)(2 6)''.
    \item If the permutation is the identity permutation, then it is denoted by $\id$.
\end{itemize}

\begin{example}
    Let the permutation $\alpha$ have cycle decomposition $\begin{pmatrix}1 & 3 & 5 & 2\end{pmatrix}$. Then
    \begin{multicols}{3}
        \begin{itemize}
            \item $\alpha(1) = 3$;
            \item $\alpha(2) = 1$;
            \item $\alpha(3) = 5$;
            \item $\alpha(4) = 4$;
            \item $\alpha(5) = 2$; and
            \item $\alpha(n) = n$ for $n \geq 6$.
        \end{itemize}
    \end{multicols}
\end{example}

\begin{example}
    Let the permutation $\beta$ have cycle notation $\begin{pmatrix}1 & 6 & 2 & 9 & 7 & 4\end{pmatrix}$. Then
    \begin{multicols}{3}
        \begin{itemize}
            \item $\beta(1) = 6$;
            \item $\beta(2) = 9$;
            \item $\beta(3) = 3$;
            \item $\beta(4) = 1$;
            \item $\beta(5) = 5$;
            \item $\beta(6) = 2$;
            \item $\beta(7) = 4$;
            \item $\beta(8) = 8$;
            \item $\beta(9) = 7$; and
            \item $\beta(n) = n$ for $n \geq 10$.
        \end{itemize}
    \end{multicols}
\end{example}

\begin{exercise}
    Find the cycle decomposition of the following permutations.
    \begin{partquestions}{\alph*}
        \item $1 \mapsto 2$, $2 \mapsto 3$, $3 \mapsto 1$
        \item $1 \mapsto 3$, $2 \mapsto 2$, $3 \mapsto 1$
        \item $1 \mapsto 3$, $2 \mapsto 4$, $3 \mapsto 1$, $4 \mapsto 5$, $5 \mapsto 2$
    \end{partquestions}
\end{exercise}

We now look at composing permutations\index{permutation!composing}.
\begin{example}
    Let $f$ and $g$ be permutations. Let $f$ have cycle notation $\begin{pmatrix}1 & 3 & 5 & 2\end{pmatrix}$ and $g$ have cycle notation $\begin{pmatrix}2 & 4 & 3\end{pmatrix}$. Then $h = fg$ is a permutation with
    \begin{itemize}
        \item $h(1) = f(g(1)) = f(1) = 3$;
        \item $h(2) = f(g(2)) = f(4) = 4$;
        \item $h(3) = f(g(3)) = f(2) = 1$;
        \item $h(4) = f(g(4)) = f(3) = 5$; and
        \item $h(5) = f(g(5)) = f(5) = 2$.
    \end{itemize}

    So $h$ maps 1 to 3, 2 to 4, 3 to 1, 4 to 5, and 5 to 2, meaning $h$ has cycle notation
    \[
        \begin{pmatrix}1 & 3\end{pmatrix}\begin{pmatrix}2 & 4 & 5\end{pmatrix}.
    \]
\end{example}

\begin{example}
    We have
    \[
        \begin{pmatrix}2 & 9 & 7 & 4\end{pmatrix}\begin{pmatrix}1 & 6 & 4\end{pmatrix} = \begin{pmatrix}1 & 6 & 2 & 9 & 7 & 4\end{pmatrix}.
    \]
\end{example}

We now describe how to find the inverse of a permutation\index{permutation!inverse}. Given a cycle decomposition for the permutation $f$, simply read the cycle notation backwards, ensuring that the smallest element remains at the front.

\begin{example}
    Consider the permutation $\begin{pmatrix}1 & 8 & 4 & 2\end{pmatrix}$. Its inverse is $\begin{pmatrix}2 & 4 & 8 & 1\end{pmatrix} = \begin{pmatrix}1 & 2 & 4 & 8\end{pmatrix}$.
\end{example}

\begin{example}
    $\begin{pmatrix}1 & 7 & 5 & 3 & 9\end{pmatrix}^{-1} = \begin{pmatrix}1 & 9 & 3 & 5 & 7\end{pmatrix}$.
\end{example}

\begin{exercise}
    Find the inverse of the permutation $\pi$, which has cycle notation
    \[
        \begin{pmatrix}1 & 5 & 2\end{pmatrix}\begin{pmatrix}2 & 5 & 3 & 4\end{pmatrix}.
    \]
\end{exercise}

\section{The Symmetric Group of a Set}
With the definition of permutations out of the way, we can finally introduce a very important type of group: the \textbf{symmetric group} of a set $X$.

\begin{definition}
    Let $X$ be a set. Define the \textbf{symmetric group of $X$}\index{symmetric group} by
    \[
        \Sym{X} = \{f: X \to X \vert f \text{ is a bijection}\}.
    \]
\end{definition}
\begin{proposition}
    $(\Sym{X}, \circ)$ is a group, where $\circ$ is the function composition operator.
\end{proposition}
\begin{proof}
    We prove the 4 group axioms.
    \begin{enumerate}
        \item \textbf{Closure}: Let $f$ and $g$ be functions in $\Sym{X}$, so $f: X\to X$ and $g:X \to X$ are bijective functions. Define $h:X \to X$ where $h = f\circ g$. By \myref{exercise-composition-of-isomorphisms-is-isomorphisms} we know $h$ is bijective. Thus $\Sym{X}$ is closed under $\circ$.
        
        \item \textbf{Associativity}: Function composition is associative.
        
        \item \textbf{Identity}: Clearly the identity map, $\id$, is in $\Sym{X}$. Also $\id$ is a bijection by \myref{exercise-identity-map-is-isomorphism}. Now we show that $\id$ is indeed the identity in $\Sym{X}$. Let $x$ be an arbitrary element of $X$, and $f$ be any function in $\Sym{X}$. Then
        \[
            (\id \circ f)(x) = \id(f(x)) = f(x)
        \]
        and
        \[
            (f \circ \id)(x) = f(\id(x)) = f(x)
        \]
        so $\id$ is the identity in $\Sym{X}$.
        
        \item \textbf{Inverse}: For all functions $f$ in $\Sym{X}$, $f^{-1}$ exists since $f$ is a bijection. Furthermore, $f^{-1}$ is a bijection from $X$ to $X$, so $f^{-1}$ is in $\Sym{X}$. By definition of $f^{-1}$,
        \[
            f \circ f^{-1} = f^{-1} \circ f = \id
        \]
        so $f^{-1}$ is indeed the inverse of $f$ in $\Sym{X}$.
    \end{enumerate}
    Therefore $(\Sym{X}, \circ)$ is a group.
\end{proof}

The group $(\Sym{X}, \circ)$ is called the \textbf{symmetric group} of $X$\index{symmetric group!of a set $X$}. We usually suppress the function composition operator and call $\Sym{X}$ the symmetric group of $X$.

The most relevant type of symmetric group we encounter when working with finite groups is the \textbf{symmetric group of degree $n$}\index{symmetric group!of degree $n$} (or \textbf{symmetric group of $n$ letters}\index{symmetric group!of $n$ letters}).
\begin{definition}
    The \textbf{symmetric group of degree $n$} is denoted by $\Sn{n}$ and is given by the group $\Sym{\{1, 2, 3, \dots, n\}}$.
\end{definition}
\begin{remark}
    Elements of $\Sn{n}$ are called permutations.
\end{remark}

\begin{example}\label{example-symmetric-group-of-degree-3}
    Consider the symmetric group of degree 3, $\Sn{3}$. We show all function mappings of $\Sn{3}$.

    \begin{figure}[h]
        \centering
        \fbox{\includegraphics[width=5cm]{symmetry-groups/s3.jpg}}
        \caption{All Mappings of $\Sn{3}$}
    \end{figure}

    Thus, $|\Sn{3}| = 6$.
\end{example}
\begin{exercise}\label{exercise-order-of-Sn}
    Explain why $|\Sn{n}| = n!$.
\end{exercise}

It should also be noted that subgroups of $\Sym{X}$ are called \textbf{permutation groups}\index{permutation groups}, primarily because they contain permutations. Since a group is its own subgroup, the symmetric group may sometimes be called \textit{the} permutation group.

Finally, we prove an important result regarding symmetric groups of equal order.
\begin{theorem}\label{thrm-symmetric-groups-of-same-order-are-isomorphic}
    Let $S_1$ and $S_2$ be two finite sets with cardinality $n$. Then
    \[
        \Sym{S_1} \cong \Sym{S_2}.
    \]
\end{theorem}
\begin{proof}[Proof (see {\cite[Proof 2]{proofwiki_symmetric-group-of-same-order-are-isomorphic}})]
    Since the two sets $S_1$ and $S_2$ have the same cardinality, they have the same number of elements. Hence there exists a bijection $f: S_1 \to S_2$.

    Let $\phi: \Sym{S_1} \to \Sym{S_2}$ where $\sigma \mapsto f\sigma f^{-1}$. We show that $\phi$ is an isomorphism.
    \begin{itemize}
        \item \textbf{Homomorphism}: Let $\sigma, \pi \in \Sym{S_1}$. Then
        \begin{align*}
            \phi(\sigma\pi) &= f(\sigma\pi)f^{-1}\\
            &= f\sigma(f^{-1}f)\pi f^{-1}\\
            &= (f\sigma f^{-1})(f\pi f^{-1})\\
            &= \phi(\sigma)\phi(\pi)
        \end{align*}
        which shows that $\phi$ is a homomorphism.

        \item \textbf{Injective}: Let $\sigma, \pi \in \Sym{S_1}$ such that $\phi(\sigma) = \phi(\pi)$. Thus $f\sigma f^{-1} = f\pi f^{-1}$ which quickly implies $\sigma = \pi$ by cancellation law. Therefore $\phi$ is injective.
        
        \item \textbf{Surjective}: Let $\pi \in \Sym{S_2}$. Let $\sigma = f^{-1}\pi f$. We note that $\sigma: S_1 \to S_1$ is a bijection, so $\sigma \in \Sym{S_1}$. Clearly
        \[
            \phi(\sigma) = \phi(f^{-1}\pi f) = f(f^{-1}\pi f)f^{-1} = \pi
        \]
        so $\sigma$ is the pre-image of $\pi$. Hence $\sigma$ is the pre-image of $\pi$.
    \end{itemize}
    Therefore $\phi$ is an isomorphism, which means that $\Sym{S_1} \cong \Sym{S_2}$.
\end{proof}

\begin{corollary}\label{corollary-symmetric-group-of-finite-order}
    If the set $A$ is finite with cardinality $n$ then $\Sym{A} \cong \Sn{n}$.
\end{corollary}
\begin{proof}
    Let $X = \{1, 2, 3, \dots, n\}$. We note that $\Sn{n} = \Sym{X}$ by definition, and that $|X| = n$. We also note that $|A| = n$ so $A$ and $X$ are two finite sets with equal cardinality. Result follows from \myref{thrm-symmetric-groups-of-same-order-are-isomorphic}.
\end{proof}

\section{Cayley's Theorem}
We now have sufficient background to state and prove Cayley's theorem.

\begin{theorem}[Cayley]\label{thrm-cayley}\index{Cayley's Theorem}
    Every group is isomorphic to a permutation group.
\end{theorem}

The statement of the theorem, although simple, is the reason \textit{why} group theorists study group theory: to explore all the ways that a group can be symmetric.

The proof of this theorem is involved and technical, but we'll try and simplify its proof as much as possible.

\begin{proof}[Proof (cf. {\cite[Proof 2]{proofwiki_cayleys-theorem}})]
    Let $G$ be any group. We want to prove that there exists a group of bijective functions from $G$ to $G$ that is isomorphic to $G$ (i.e., a permutation group).

    For any $g$ in $G$ define the map $\lambda_g: G \to G$ such that $x \mapsto gx$. We claim that $\lambda_g$ is a bijection.
    \begin{itemize}
        \item \textbf{Injective}: Let $x$ and $y$ be elements of $G$ such that $\lambda_g(x) = \lambda_g(y)$. Then $gx = gy$ by definition of $\lambda_g$, which immediately means $x = y$ by cancellation law. Thus $\lambda_g(x) = \lambda_g(y)$ implies $x = y$, meaning $\lambda_g$ is injective.
        
        \item \textbf{Surjective}: Let $y$ be an element of $G$. Note that $g^{-1}y$ is an element of $G$ (since $G$ is closed), and that $\lambda_g(g^{-1}y) = g(g^{-1}y) = y$. Thus, a preimage of $y$ is $g^{-1}y$ and it exists in the domain $G$, meaning $\lambda_g$ is surjective.
    \end{itemize}
    Since $\lambda_g$ is both injective and surjective it is thus bijective.

    Now let $H = \{\lambda_g \vert g \in G\}$. Since $\lambda_g$ are all bijections from $G$ to $G$, thus $H \subseteq \Sym{G}$. We show that $H \leq \Sym{G}$ via the subgroup test.

    We first note that $\lambda_e = \id$ since
    \[
        \lambda_e(x) = ex = x = \id(x)
    \]
    for all $x$ in $G$, meaning $\id = \lambda_e \in H$. Also, the inverse of any function $\lambda_g$ is $\lambda_{g^{-1}}$ since
    \[
        (\lambda_g \circ \lambda_{g^{-1}})(x) = gg^{-1}x = x = \lambda_e(x)
    \]
    and
    \[
        (\lambda_{g^{-1}} \circ \lambda_g)(x) = g^{-1}gx = x = \lambda_e(x).
    \]
    Now suppose $\lambda_{g_1}, \lambda_{g_2} \in H$ (where $g_1, g_2 \in G$). Note $g_1g_2^{-1}$ is an element of $G$ since $G$ is closed. Therefore, for all $x$ in $G$,
    \begin{align*}
        \left(\lambda_{g_1} \circ \left(\lambda_{g_2}\right)^{-1}\right)(x) &= \lambda_{g_1}\circ\lambda_{g_2^{-1}}(x)\\
        &= g_1g_2^{-1}x\\
        &= \lambda_{g_1g_2^{-1}}(x).
    \end{align*}
    Note $\lambda_{g_1g_2^{-1}}$ is clearly an element of $H$. Thus if $\lambda_{g_1}$ and $\lambda_{g_2}$ are functions in $H$, then $\left(\lambda_{g_1} \circ \left(\lambda_{g_2}\right)^{-1}\right)$ is also a function in $H$. Therefore by the subgroup test we have $H \leq \Sym{G}$.

    We finally show that $G \cong H$ by considering the map $\phi: G\to H$ where $g \mapsto \lambda_g$. We need to show that $\phi$ is an isomorphism.
    \begin{itemize}
        \item \textbf{Homomorphism}: For any $x$ in $G$,
            \begin{align*}
                \phi(gh)(x) &= \lambda_{gh}(x)\\
                &= ghx\\
                &= g(hx)\\
                &= \lambda_g\left(\lambda_h(x)\right)\\
                &= \lambda_g\circ\lambda_h(x)\\
                &= (\phi(g)\phi(h))(x).
            \end{align*}
            Thus, $\phi(gh) = \phi(g)\phi(h)$.
        
        \item \textbf{Injective}: Let $g_1, g_2 \in G$ such that $\phi(g_1) = \phi(g_2)$. Then $\lambda_{g_1} = \lambda_{g_2}$. Therefore, $\lambda_{g_1}(x) = \lambda_{g_2}(x)$ for all $x$ in $G$, which means that $\lambda_{g_1}(e) = \lambda_{g_2}(e)$ when $x = e$. By definition of $\lambda_g$, we have $eg_1 = eg_2$ which ultimately means that $g_1=g_2$. Thus if $\phi(g_1) = \phi(g_2)$ then $g_1=g_2$.
        
        \item \textbf{Surjective}: Let $\lambda_g \in H$. Clearly $\phi(g) = \lambda_g$, which means $\lambda_g$ has a pre-image of $g$. Thus $\phi$ is surjective.
    \end{itemize}
    Therefore we have proven that $\phi$ is an isomorphism, which means that
    \[
        G \cong H \leq \Sym{G},
    \]
    that is, any group $G$ is isomorphic to a subgroup of the symmetric group of $G$ (i.e., a permutation group).
\end{proof}

We note one corollary of this theorem.
\begin{corollary}
    Let $G$ be a finite group of order $n$. Then there exists a group $H \leq \Sn{n}$ such that $G \cong H$.
\end{corollary}
\begin{proof}
    By Cayley's theorem (\myref{thrm-cayley}), there exists a group $H \leq \Sym{G}$ such that $G \cong H$. Now since $G$ is finite with order $n$, thus by \myref{corollary-symmetric-group-of-finite-order}, $\Sym{G} \cong \Sn{n}$. Thus, $H \leq \Sn{n}$, and $G \cong H$.
\end{proof}

One might ask what the use of Cayley's Theorem is in group theory. To put it simply, it is a sanity check on the definition of a group. Before anyone had the idea of writing down the axioms for groups, people studied collections of bijections of sets closed under composition and inverses. Cayley's Theorem tells us that every abstract group is a type of the above collection, so the axioms of group theory capture the concrete phenomenon that groups were designed to capture.

\newpage

\section{Problems}
\begin{problem}
    Let the permutations
    \begin{align*}
        &\alpha = \begin{pmatrix}1 & 5 & 2 & 3\end{pmatrix},\\
        &\beta  = \begin{pmatrix}1 & 5 & 2\end{pmatrix}\begin{pmatrix}3 & 4\end{pmatrix},\\
        &\gamma = \begin{pmatrix}1 & 2 & 5\end{pmatrix}\begin{pmatrix}3 & 4\end{pmatrix}, \text{ and}\\
        &\delta = \begin{pmatrix}1 & 3 & 2 & 5\end{pmatrix}.
    \end{align*}
    What is the cycle decomposition of $\alpha\beta\gamma\delta$?
\end{problem}

\begin{problem}
    Prove that the symmetric group of degree 3, $\Sn{3}$, is isomorphic to the dihedral group of order 6, $D_3$.
\end{problem}

\begin{problem}
    State the number of elements in $\Sn{4}$.
    \begin{partquestions}{\alph*}
        \item Let $G$ be the cyclic group of order 4. Cayley's Theorem says that it is isomorphic to a subgroup of $\Sn{4}$. Find one such subgroup of $\Sn{4}$ and prove that it is, indeed, isomorphic to $G$.
        \item Let $G$ be the group with presentation
        \[
            \langle a, b \vert a^2 = b^2 = (ab)^2 = e \rangle.
        \]
        Cayley's Theorem says that it is isomorphic to a subgroup of $\Sn{4}$. Find one such subgroup of $\Sn{4}$ and prove that it is, indeed, isomorphic to $G$.
    \end{partquestions}
\end{problem}

\section{Direct Products of Groups}
\begin{questions}
    \item Let $g_1, g_2 \in G$ and $h_1, h_2 \in H$. Then for $(g_1, h_1), (g_2, h_2) \in G\times H$ we see that
    \begin{align*}
        (g_1, h_1)(g_2, h_2) &= (g_1g_2, h_1h_2)\\
        &= (g_2g_1, h_2h_1)\\
        &= (g_2,h_2)(g_1,h_1)
    \end{align*}
    which means that $G \times H$ is abelian.

    \item Let the map $\phi: G\times H \to H \times G, (g, h) \mapsto (h, g)$. We prove that $\phi$ is an isomorphism:
    \begin{itemize}
        \item \textbf{Homomorphism}: Let $(g_1, h_1), (g_2, h_2) \in G \times H$. We note that
        \begin{align*}
            \phi((g_1, h_1)(g_2, h_2)) &= \phi((g_1g_2, h_1h_2))\\
            &= (h_1h_2, g_1g_2)\\
            &= (h_1, g_1)(h_2, g_2)\\
            &= \phi((g_1, h_1))\phi((g_2, h_2))
        \end{align*}
        which proves that $\phi$ is a homomorphism.
        \item \textbf{Injective}: Suppose there exists $(g_1, h_1), (g_2, h_2) \in G \times H$ such that $\phi((g_1, h_1)) = \phi((g_2, h_2))$. Then by definition of $\phi$ we have $(h_1, g_1) = (h_2, g_2)$. Clearly by comparing component parts of each ordered pair, we have $g_1 = g_2$ and $h_1 = h_2$, meaning $(g_1, h_1) = (g_2, h_2)$. Hence $\phi$ is injective.
        \item \textbf{Surjective}: Let $(h, g) \in H \times G$. Clearly $(g, h) \in G \times H$ and $\phi((g, h)) = (h, g)$, meaning that $(h, g)$ has a pre-image of $(g, h)$. Therefore $\phi$ is surjective.
    \end{itemize}
    Therefore $\phi$ is an isomorphism, meaning $G \times H \cong H \times G$.

    \item We claim that $G$ is the internal direct product of $H$ and $K$. We need to check 3 things.
    \begin{itemize}
        \item $\boxed{G = HK}$ We note that
        \begin{align*}
            HK &= \{h \oplus_6 k \vert h \in H, k \in K\}\\
            &= \{0 \oplus_6 0, 0 \oplus_6 3, 2 \oplus_6 0, 2 \oplus_6 3, 4 \oplus_3 0, 4 \oplus_3 3\}\\
            &= \{0, 3, 2, 5, 4, 1\}\\
            &= \mathbb{Z}_6\\
            &= G
        \end{align*}
        so in fact $G = HK$.

        \item $\boxed{H \cap K = \{e\}}$ Clearly $H \cap K = \{0\}$.

        \item $\boxed{hk = kh}$ Since $\oplus_6$ is commutative, thus $h \oplus_6 k = k \oplus_6$.
    \end{itemize}
    Thus $G$ is the internal direct product of $H$ and $K$.

    \item Define the subgroups $H = \{e, a\}$ and $K = \{e, b\}$. We show the $\mathrm{V}$ is the internal direct product of $H$ and $K$.
    \begin{itemize}
        \item $\boxed{\mathrm{V} = HK}$ Observe that
        \begin{align*}
            HK &= \{hk \vert h \in H, k \in K\}\\
            &= \{ee, eb, ae, ab\}\\
            &= \{e, b, a, ab\}\\
            &= \mathrm{V}
        \end{align*}
        so in fact $\mathrm{V} = HK$.

        \item $\boxed{H \cap K = \{e\}}$ Clearly $H \cap K = \{e\}$.

        \item $\boxed{hk = kh}$ Clearly if one of the elements is the identity then result follows. So assume that $h$ and $k$ are both non-identity elements, so $h = a$ and $k = b$. Note
        \begin{align*}
            kh &= ba\\
            &= (ba)\left((ab)(ab)\right) & (\text{since }(ab)^2 = e)\\
            &= (ba ab)(ab)\\
            &= (bb)(ab) & (\text{since }a^2 = e)\\
            &= ab & (\text{since }b^2 = e)\\
            &= hk
        \end{align*}
        so in fact $hk = kh$ for all $h \in H$, $k \in K$.
    \end{itemize}
    Therefore $\mathrm{V}$ is the internal direct product of $H$ and $K$. 
    
    We note $H = \langle a\rangle \cong \mathbb{Z}_2$ and $K = \langle b \rangle \cong \mathbb{Z}_2$. By direct product equivalence (\myref{thrm-direct-product-equivilance}) we know $\mathrm{V} \cong H \times K \cong \mathbb{Z}_2 \times \mathbb{Z}_2 = (\mathbb{Z}_2)^2$.
\end{questions}

\chapter{Further Properties of Homomorphisms}
Earlier in this book, we introduced homomorphisms and isomorphisms, special types of maps that transform elements of one group to another. We look at more properties of such maps in this chapter and describe the uses of these new properties.

\section{Image of a Homomorphism}
As a homomorphism is a mapping between two groups, it is worthy to look at the \textbf{image} of the homomorphism.
\begin{definition}
    The \textbf{image}\index{homomorphism!image} (or \textbf{range}\index{homomorphism!range}) of a homomorphism $\phi: G \to H$ is the set
    \[
        \im\phi = \{\phi(g) \vert g \in G\}.
    \]
\end{definition}
\begin{remark}
    Some authors (e.g. {\cite[Definition 4.2.0]{libretexts_im-and-ker}}) will use the notation $\phi(G)$ for the image of $\phi$. The alternate notation $\mathrm{Im}\;\phi$ may also be used (e.g. by {\cite[Definition I.2.2]{hungerford_1980}} and \cite[\S 66]{clark_1984}).
\end{remark}

\begin{example}
    Consider the homomorphism $f: \Z \to \Z, x \mapsto 0$. Clearly, all possible values of $x$ maps to 0, so $\im f = \{0\}$.
\end{example}
\begin{example}
    The homomorphism $f: \R \to \R$ where $f(x) = |x|$ has an image of $\{x \in \R \vert x \geq 0\}$, i.e. all non-negative real numbers.
\end{example}

\begin{proposition}\label{prop-image-is-subgroup-of-codomain}
    Let $\phi: G \to H$ be a homomorphism. Then $\im\phi \leq H$.
\end{proposition}
\begin{proof}
    Note that $\phi(e_G) = e_H \in \im\phi$, where $e_G$ and $e_H$ are the identities of $G$ and $H$ respectively.
    
    Now suppose $h_1$ and $h_2$ are in the image of $\phi$, meaning that there exists $g_1$ and $g_2$ such that $\phi(g_1) = h_1$ and $\phi(g_2) = h_2$. Note that $\phi(g_2^{-1}) = h_2^{-1}$ by homomorphism property. Hence $\phi(g_1g_2^{-1}) = h_1h_2^{-1} \in \im\phi$.

    Therefore, by subgroup test, $\im\phi \leq H$.
\end{proof}

\begin{exercise}
    Consider the map $\phi: \Z_3 \to \Z_6, n \mapsto 2n$. Determine whether $\phi$ is a homomorphism and, if so, find its image.
\end{exercise}

\section{Kernel of a Homomorphism}
\begin{definition}
    The \textbf{kernel}\index{homomorphism!kernel} of a homomorphism $\phi: G \to H$ is the set of elements in the group $G$ which map to the identity in the group $H$. That is, if the identity of $H$ is $e_H$, then the kernel of $\phi$ is the set
    \[
        \ker\phi = \{x \in G \vert \phi(x) = e_H\}.
    \]
\end{definition}


\begin{remark}
    Some authors (e.g. {\cite[Definition 4.2.0]{libretexts_im-and-ker}}) will use the notation $\phi^{-1}(e_H)$ for the kernel of $\phi$. The alternate notation $\mathrm{Ker}\;\phi$ may also be used by some authors (e.g. by {\cite[Definition I.2.2]{hungerford_1980}} and \cite[\S 65]{clark_1984}).
\end{remark}

\begin{example}
    Let the groups $G = (\Z^2, (+, +))$ and $H = (\Z, +)$. Let the map $\phi: G \to H, (a, b) \mapsto a+b$. Then, $(a, b) \in \ker\phi$ if $\phi((a,b)) = 0$. This means that $a+b = 0$, implying $ b = -a$. Hence the kernel of $\phi$ is $\{(a, -a) \vert a \in \Z\}$.
\end{example}

\begin{exercise}
    Let $i$ be the imaginary unit, that is $i^2 = -1$. Let the group $G$ be the integers under addition and $H = \langle i \rangle$ be under multiplication. Let the map $\phi: G \to H, n \mapsto i^n$.  Show that $\phi$ is a homomorphism and hence find $\ker\phi$.
\end{exercise}

\begin{proposition}\label{prop-kernel-is-normal-subgroup-of-domain}
    Let $\phi: G \to H$ be a homomorphism. Then $\ker\phi \unlhd G$.
\end{proposition}
\begin{proof}
    We will first show $\ker\phi\leq G$. Clearly $e_G \in \ker\phi$ since $\phi(e_G) = e_H$, so $\ker\phi$ is non-empty. Now let $x, y \in \ker\phi$. This means that $\phi(x) = \phi(y) = e_H$. Note
    \begin{align*}
        \phi(xy^{-1}) &= \phi(x)\left(\phi(y)\right)^{-1}\\
        &= e_H(e_H)^{-1}\\
        &= e_H
    \end{align*}
    which means that $xy^{-1}\in\ker\phi$. By subgroup test, $\ker\phi\leq G$.

    Now we prove normality. Let $x \in G$ and $n \in \ker\phi$. We need to show that $xnx^{-1}\in\ker\phi$ to prove normality. Observe that
    \begin{align*}
        \phi(xnx^{-1}) &= \phi(x)\phi(n)\phi(x^{-1})\\
        &= \phi(x)e_H\phi(x)^{-1} & (n \in \ker\phi)\\
        &= \phi(x)\phi(x)^{-1}\\
        &= e_H,
    \end{align*}
    which means that $xnx^{-1} \in \ker\phi$. Hence, $\ker\phi \unlhd G$.
\end{proof}

\begin{exercise}\label{exercise-trivial-kernel-means-injective}
    Prove that a homomorphism $\phi:G\to H$ is injective if and only if $\ker \phi$ is trivial, i.e. $\ker \phi = \{e_G\}$.
\end{exercise}

\section{The Fundamental Homomorphism Theorem}
We are now ready to tackle the three most important theorems regarding homomorphisms. We first state the \textbf{Fundamental Homomorphism Theorem}, which is also sometimes called the \textbf{First Isomorphism Theorem} (e.g. in {\cite[p.~251, Theorem 3]{cohn_1982}}).
\begin{theorem}[Fundamental Homomorphism Theorem]\label{thrm-isomorphism-1}\index{Fundamental Homomorphism Theorem}\index{Isomorphism Theorem!First}
    Let $G$ and $H$ be groups. Let $\phi: G \to H$ be a homomorphism, and let $\pi: G \to G/\ker\phi$ where $g\mapsto g\ker\phi$ be the natural surjective homomorphism. Then there exists a unique isomorphism $\psi: G/\ker\phi \to \im\phi$ such that $\psi\pi = \phi$.
\end{theorem}
\begin{remark}
    Equivalently, the Fundamental Homomorphism Theorem states that
    \[
        G/\ker\phi \cong \im\phi
    \]
    for any homomorphism $\phi$.
\end{remark}

We include the commutativity diagram of the homomorphisms stated to aid clarity:

\begin{figure}[h]
    \centering
    \fbox{\includegraphics[width=0.35\textwidth]{further-homomorphisms/iso-1-comm-diagram.png}}
    \caption{Commutativity Diagram for \myreffigures{thrm-isomorphism-1}}
\end{figure}

In the diagram, $\phi$ sends elements from $G$ to $\im\phi$ and $\pi$ sends elements from $G$ to $G/\ker\phi$. Then the map $\psi$ is a unique map that sends elements from $G/\ker\phi$ to the image of $\phi$.

\begin{proof}
    We know by \myref{prop-image-is-subgroup-of-codomain} that $\im\phi \leq H$. Let $\psi: G/\ker\phi \to \im\phi$ such that $\psi(x\ker\phi) = \phi(x)$. We need to check that $\psi$ is a well-defined isomorphism.
    \begin{itemize}
        \item \textbf{Well-defined}: Suppose $x\ker\phi = y\ker\phi$ where $x, y \in G$. Then $xy^{-1} \in \ker\phi$ by Coset Equality (\myref{lemma-coset-equality}), statements 1 and 5. This means that $\phi(xy^{-1}) = e_H$ by definition of the kernel. Note $\phi(xy^{-1}) = \phi(x)\left(\phi(y)\right)^{-1}$, so $\phi(x)\left(\phi(y)\right)^{-1} = e_H$. Hence $\phi(x) = \phi(y)$. Thus,
        \[
            \psi(x\ker\phi) = \phi(x) = \phi(y) = \psi(y\ker\phi)
        \]
        so $\psi$ is well-defined.

        \item \textbf{Homomorphism}: Note that
        \begin{align*}
            \psi((x\ker\phi)(y\ker\phi)) &= \psi((xy)\ker\phi)\\
            &= \phi(xy)\\
            &= \phi(x)\phi(y)\\
            &= \psi(x\ker\phi)\psi(y\ker\phi)
        \end{align*}
        so $\psi$ is a homomorphism.
        \item \textbf{Injective}: By \myref{exercise-trivial-kernel-means-injective}, we check that $\psi$ is injective by showing that $\ker\psi$ is trivial, i.e. $\ker\psi = \{\ker\phi\}$.

        Suppose $x\ker\phi\in\ker\psi$. Then $\psi(x\ker\phi) = e_H$ by definition of kernel. Hence $\phi(x) = e_H$ by definition of $\psi$, which means $x \in \ker\phi$ by definition of kernel. Thus $x\ker\phi = \ker\phi$ by Element in Coset (\myref{corollary-equivalence-of-element-in-coset}). Therefore $\psi$ is injective.

        \item \textbf{Surjective}: Suppose $y$ is in the image of $\phi$, meaning there exists a $x \in G$ such that $\phi(x) = y$. Note that $\psi(x\ker\phi) = \phi(x) = y$. Thus $\psi$ is surjective.
    \end{itemize}
    Thus $\psi$ is a well-defined isomorphism.

    We now check that $\psi$ satisfies the requirement that $\psi\pi = \phi$. Let $x \in G$. Note that $\pi(x) = x\ker\phi$, and
    \[
        \psi\pi(x) = \psi(x\ker\phi) = \phi(x)
    \]
    for all $x \in G$, so $\psi\pi = \phi$.

    Finally we show that $\psi$ is unique. Suppose $f: G/\ker\phi \to \im\phi$ is an isomorphism satisfying $f\pi=\phi$. Take $x\ker\phi \in G/\ker\phi$. Note that
    \begin{align*}
        f(x\ker\phi) &= f(\pi(x))\\
        &= (f\pi)(x)\\
        &= \phi(x)\\
        &= (\psi\pi)(x)\\
        &= \psi(\pi(x))\\
        &= \psi(x\ker\phi)
    \end{align*}
    for all $x \in G$, meaning that $f = \psi$. Therefore $\psi$ is unique.

    Hence, $\psi$ is a unique isomorphism satisfying $\psi\pi = \phi$.
\end{proof}

\begin{example}
    Let $R = \{x \in \R \vert x > 0\}$, $G = \{x \in \R \vert x \neq 0\}$, and $H = \{1, -1\}$ be groups under multiplication. We show $G / H \cong R$.

    Consider the map $\phi: G \to R$ where $x \mapsto |x|$. We show that $\phi$ is a homomorphism, then find the image of $\phi$, and finally find its kernel.

    \begin{itemize}
        \item \textbf{Homomorphism}: $\phi$ is a homomorphism since $\phi(xy) = |xy| = |x||y| = \phi(x)\phi(y)$.
        \item \textbf{Image}: We find the image of $\phi$.
        \begin{align*}
            \im\phi &= \{\phi(x) \vert x \in G\}\\
            &= \{|x| \vert x \neq 0\}\\
            &= \{x \in \R \vert x > 0\} & (\text{by definition of } |x|)\\
            &= R
        \end{align*}
        which actually means that $\phi$ is surjective.
        \item \textbf{Kernel}: We find the kernel of $\phi$.
        \begin{align*}
            \ker\phi &= \{x \in G \vert \phi(x) = 1\} & (1 \text{ is the identity in } R)\\
            &= \{x \in G \vert |x| = 1\}\\
            &= \{1, -1\}\\
            &= H
        \end{align*}
    \end{itemize}
    Thus $G/H \cong R$ by the Fundamental Homomorphism Theorem (\myref{thrm-isomorphism-1}).
\end{example}

\begin{exercise}
    Let $\phi: G \to H$ be a homomorphism between finite groups $G$ and $H$. Prove that
    \[
        |G| = |\im \phi|\times|\ker \phi|.
    \]
\end{exercise}

\section{The Diamond Isomorphism Theorem}
We now look at the next theorem, called the \textbf{Diamond Isomorphism Theorem} (e.g. in {\cite[Theorem 3.18]{dummit_foote_2004}}) or the \textbf{Second Isomorphism Theorem} (e.g. in {\cite[\S 69]{clark_1984}}).
\begin{theorem}[Diamond Isomorphism Theorem]\label{thrm-isomorphism-2}\index{Diamond Isomorphism Theorem}\index{Isomorphism Theorem!Second}
    Let $G$ be a group and let $H$ and $K$ be subgroups of $G$. Then
    \begin{enumerate}
        \item $H \cap K \leq H$; and
        \item $H \leq HK$.
    \end{enumerate}
    Furthermore, if $N \unlhd G$, then
    \begin{enumerate}[start=3]
        \item $HN \leq G$;
        \item $H \cap N \unlhd H$;
        \item $N \unlhd HN$; and
        \item $H / (H\cap N) \cong HN / N$.
    \end{enumerate}
\end{theorem}

\newpage

We can capture the overall relationships of the subgroups of $G$ using a \textbf{subgroup lattice}.
\begin{figure}[h]
    \centering
    \fbox{\includegraphics[width=0.25\textwidth]{further-homomorphisms/iso-2-subgroup-diagram.png}}
    \caption{Subgroup Lattice for \myreffigures{thrm-isomorphism-2}}
\end{figure}

We only show subgroups that we care about in the diagram. The group $G$ has a (direct) subgroup $HK$; $HK$ has subgroups $H$ and $K$; and $H$ and $K$ has a common subgroup $H\cap K$. The dotted quotient groups are isomorphic to each other if $H \unlhd G$.

\begin{proof}
    We prove each statement in sequence.

    \begin{enumerate}
        \item Clearly $e_G \in H$ and $e_G \in K$ so $e_G \in H \cap K$. Now take $x, y \in H \cap K$, meaning $x, y \in H$ and $x, y \in K$. Since $H, K \leq G$ so $xy^{-1} \in H$ and $xy^{-1} \in K$. Thus $xy^{-1} \in H \cap K$. By the subgroup test, this means that $H \cap K \leq H$.
        
        \item Note that $H = \{he_G \vert h \in H\} \subseteq \{hk \vert h \in H, k \in K\} = HK$, and $H$ is a group (as $H$ is a subgroup). Therefore $H \leq HK$.
        
        \item We note that, because $N$ is normal, hence $hN = Nh$ for all $h \in H \subseteq G$, meaning that $HN = NH$. Therefore by \myref{prop-subgroup-product-is-subgroup}, we have $HN \leq G$.

        \item We know $H \cap N \leq H$ by statement 1, so we only prove normality. Take $x \in H \cap N$. Since $H \leq G$, thus $x \in H \cap N \subseteq H$, meaning for all $g \in H$, $gxg^{-1} \in H$ (where we think of $g$ and $x$ as being in $H$). But since $x \in H \cap N \subseteq N$ and $N \unlhd G$, thus $gxg^{-1} \in N$ (where we think of $g \in H$ and $x \in N$). Therefore $H \cap N \unlhd H$.

        \item We know $N \leq HN$ by statement 2, so we only prove normality. Take $n \in N$ and $x \in HN$ such that $x = h_xn_x$. Then
        \begin{align*}
            xnx^{-1} &= (h_xn_x)n(h_xn_x)^{-1}\\
            &= (h_xn_x)n(n_x^{-1}h_x^{-1}) & (\text{Shoes and Socks})\\
            &= \underbrace{h_x}_{\text{In }G}\underbrace{n_xnn_x^{-1}}_{\text{In }N}\underbrace{h_x^{-1}}_{\text{In G}}\\
            &\in N
        \end{align*}
        since $N \unlhd G$. This proves that $N \unlhd HN$.

        \item This is the main result of this theorem. We define $\phi: H \to HN/N, h \mapsto hN$. We show that $\phi$ is a homomorphism and then find its image and kernel.
        \begin{itemize}
            \item \textbf{Homomorphism}:
            \[
                \phi(xy) = (xy)N = (xN)(yN) = \phi(x)\phi(y)    
            \]

            \item \textbf{Image}: We show that $\phi$ is surjective to show that $\im\phi = HN/N$. Suppose $x \in HN$, meaning $x = hn$ where $h \in H$ and $n \in N$. Thus $xN \in HN/N$, so
            \[
                xN = (hn)N = h(nN) = hN
            \]
            meaning $\phi(h) = hN = xN$. Hence we have found a pre-image of the coset $xN$, meaning $\phi$ is surjective. Thus $\im \phi = HN/N$.

            \item \textbf{Kernel}: We claim that $\ker\phi = H \cap N$.
            
            Note that $\ker\phi = \{h \in H \vert \phi(h) = eN = N\}$ by definition of kernel. This means that if $h \in \ker\phi$ then $\phi(h) = N$. Hence $\phi(h) = hN = N$, which means $h \in N$ by Element in Coset (\myref{corollary-equivalence-of-element-in-coset}). Thus, $h \in H$ and $h \in N$, meaning $h \in H \cap N$. Therefore $\ker \phi \subseteq H \cap N$.
    
            Now suppose $x \in H \cap N$. This means that $x \in N$ necessarily, implying $xN = N$. Thus $\phi(x) = N$ which quickly implies $x \in \ker\phi$. Therefore $H \cap N \subseteq \ker\phi$.
    
            Since $\ker \phi \subseteq H \cap N$ and  $H \cap N \subseteq \ker\phi$ therefore $\ker\phi = H\cap N$.
        \end{itemize}

        By the Fundamental Homomorphism Theorem (\myref{thrm-isomorphism-1}),
        \[
            H / \ker\phi \cong \im \phi,
        \]
        which means
        \[
            H/(H\cap N) \cong HN/N.
        \]
    \end{enumerate}
    This completes the proof of the theorem.
\end{proof}

\begin{corollary}\label{corollary-subgroup-product-is-normal-subgroup-if-subgroups-are-normal}
    Let $G$ be a group with proper subgroups $H$ and $K$. Then $HK \unlhd G$ if $H$ and $K$ are normal subgroups of $G$.
\end{corollary}
\begin{proof}
    Assume that $H, K \lhd G$. By the Diamond Isomorphism Theorem (\myref{thrm-isomorphism-2}), statement 3, we know that $HK \leq G$ since $H \lhd G$. We just need to prove normality. Suppose $hk \in HK$ and take $g \in G$. Then
    \begin{align*}
        g(hk)g^{-1} &= (gh)(kg^{-1})\\
        &= (hg)(g^{-1}k) & (\text{as } H, K \lhd G)\\
        &= h(gg^{-1})k\\
        &= hk \in HK
    \end{align*}
    which means that $HK \unlhd G$.
\end{proof}

We look at two examples using the Diamond Isomorphism Theorem.
\begin{example}
    We say a group $G$ is \textbf{metabelian}\index{metabelian} if and only if there exists $A \unlhd G$ such that $A$ and $G/A$ are both abelian. We will prove that any subgroup of a metabelian group is also metabelian.

    Let $H \leq G$. Then, by the Diamond Isomorphism Theorem (\myref{thrm-isomorphism-2}), we know $H \cap A \unlhd H$ (statement 4) and $H/(H \cap A) \cong HA / A$ (statement 6). We just need to prove that both $H \cap A$ and $H/(H \cap A)$ are abelian.
    \begin{itemize}
        \item Consider any two elements from $H \cap A$, say $x$ and $y$. Then $x \in A$ and $y \in A$, so $xy = yx$ as $A$ is abelian. Hence, elements from $H \cap A$ commute, meaning that $H \cap A$ is abelian.
        \item Consider $H/(H\cap A) \cong HA / A$. Note that $HA \leq G$ since $H \leq G$ and $A \leq G$. Thus, $HA / A \leq G / A$. Note that $G/A$ is abelian by definition of metabelian group. Hence, $H/(H \cap A)$ is also abelian.
    \end{itemize}
    Therefore, we have found a subgroup of $H$ (in particular $H \cap A$) such that both $H \cap A$ and $H/(H\cap A)$ are both abelian. Hence, $H$ is metabelian.
\end{example}

We look at another application of the Diamond Isomorphism Theorem, which has use in Number Theory.
\begin{example}
    We will prove that $\lcm(m,n)\times\gcd(m,n) = mn$ (\myref{prop-product-of-gcd-and-lcm}) by considering the Diamond Isomorphism Theorem. For brevity, let $d = \gcd(m,n)$ and $l = \lcm(m,n)$.

    Consider the groups $G = \Z$, $H = m\Z$, and $N = n\Z$ under addition. By Diamond Isomorphism Theorem (\myref{thrm-isomorphism-2}),
    \[
        m\Z/(m\Z \cap n\Z) \cong (m\Z + n\Z)/(n\Z).
    \]

    Now $m\Z \cap n\Z$ is the set of integers that are both a multiple of $m$ and $n$. Hence, $m\Z \cap n\Z = \lcm(m,n)\Z = l\Z$. On the other hand, $m\Z + n\Z$ is the set of all integers of the form $mx+ny$ where $x$ and $y$ are integers. B\'{e}zout's lemma (\myref{lemma-bezout}) tells us that this set consists of the multiples of $\gcd(m,n)$, i.e. $m\Z + n\Z = \gcd(m,n)\Z = d\Z$. Hence,
    \[
        m\Z/(l\Z) \cong d\Z/(n\Z).
    \]

    We claim that $m\Z / (l\Z) \cong \Z_{\frac lm} \text{ and } d\Z / (n\Z) \cong \Z_{\frac nd}$. This is a specific case of \myref{problem-mZ/nZ-isomorphic-to-Zn/m} which we have left as a problem for later. Hence,
    \[
    \Z_{\frac lm} \cong m\Z/(l\Z) \cong d\Z/(n\Z) \cong \Z_{\frac nd},
    \]
    which means that $\Z_{\frac lm} \cong \Z_{\frac nd}$. We can now finally take orders on both sides:
    \[
        \frac{l}{m} = \frac{n}{d},
    \]
    which means that $ld = mn$. Hence, $\lcm(m,n)\times\gcd(m,n) = mn$.
\end{example}

\begin{exercise}\label{exercise-order-of-subgroup-product}
    Let $G$ be a finite group, $H \leq G$, and $N \lhd G$. Prove that
    \[
        |HN| = \frac{|H||N|}{|H \cap N|}.
    \]
\end{exercise}

\section{The Third Isomorphism Theorem}
We look at the last important theorem regarding homomorphisms and isomorphisms. This is often called the \textbf{Third Isomorphism Theorem} (e.g. in {\cite[Corollary I.5.10]{hungerford_1980}} and {\cite[pp.~253--254, Theorem 5]{cohn_1982}}).

It should be noted that there is no consistency with the numbering of these theorems in books (cf. {\cite[\S 68]{clark_1984}} as ``First Isomorphism Theorem'', {\cite[Theorem 8.16]{humphreys_1996}} as ``Second Isomorphism Theorem''), but the name ``Third Isomorphism Theorem'' is the easiest to research. Hence, we use that name here.

\begin{theorem}[Third Isomorphism Theorem]\label{thrm-isomorphism-3}\index{Isomorphism Theorem!Third}
    Let $G$ be a group. Let $H \unlhd G$ and $N \unlhd G$. Suppose $N \subseteq H$. Then
    \begin{enumerate}
        \item $N \unlhd H$;
        \item $H/N \unlhd G/N$; and
        \item $\frac{G/N}{H/N} \cong G/H$.
    \end{enumerate}
\end{theorem}
\begin{proof}
    We prove the statements in sequence.

    \begin{enumerate}
        \item We note that since $N \subseteq H$ and $N$ is a group (since $N$ is a normal subgroup of $G$) thus $N \leq H$. We just need to prove normality. Since $H$ and $N$ are normal subgroups of $G$, thus for all $g \in G$,
        \[
            gH = Hg \text{ and } gN = Ng.
        \]
        Now since $N \subseteq H \subseteq G$, thus for all $n$ in $N$, $nH = Hn$ (since $n \in G$). This means that $N \unlhd H$.

        \item We first prove that it is a subgroup before proving normality.

        Clearly $N = eN \in H/N$. Let $x$ and $y$ be in $H/N$. Then $x=h_xN$ and $y=h_yN$ for some $h_x, h_y \in H$. Note that $y^{-1} = (h_y^{-1})N$ by group operator on cosets. Hence,
        \begin{align*}
            xy^{-1} &= (h_xN)(h_y^{-1}N)\\
            &= (\underbrace{h_xh_y^{-1}}_{\text{In }H})N\\
            &\in H/N
        \end{align*}
        Hence, by subgroup test, $H/N \leq G/N$.

        Now let $gN \in G/N$ and $hN \in H/N$. We need to show that $(gN)(hN)(gN)^{-1} \in H/N$. Note $(gN)(hN)(gN)^{-1} = (ghg^{-1})N$. Since $H \unlhd G$, thus $ghg^{-1} \in H$ which means that $(ghg^{-1})N \in H/N$.

        Therefore $H/N \unlhd G/N$.

        \item This is the main result of the theorem.

        Define $\phi: G/N \to G/H, gN \mapsto gH$. We check that $\phi$ is a well-defined homomorphism and find its image and kernel.
        \begin{itemize}
            \item \textbf{Well-defined}: Suppose $gN = g'N$. Then $g(g')^{-1} \in N$ by Coset Equality (\myref{lemma-coset-equality}), statements 1 and 5. Since $N \subseteq H$, thus $g(g')^{-1} \in H$ which implies $gH = g'H$, again by Coset Equality, statements 1 and 5. Hence
            \[
                \phi(gN) = gH = g'H = \phi(g'N)
            \]
            so $\phi$ is well-defined.

            \item \textbf{Homomorphism}: Take $gN, g'N \in G/N$. Then
            \begin{align*}
                \phi((gN)(g'N)) &= \phi((gg')N)\\
                &= (gg')H\\
                &= (gH)(g'H)\\
                &= \phi(gN)\phi(g'N)
            \end{align*}
            which means that $\phi$ is a homomorphism.
            
            \item \textbf{Image}: We show $\phi$ is surjective to prove that $\im\phi = G/H$. Suppose $gH \in G/H$. Clearly $\phi(gN) = gH$. Thus $gN$ is a pre-image of $gH$, meaning that $\phi$ is surjective. Hence $\im\phi = G/H$.
            
            \item \textbf{Kernel}: Suppose $gN \in \ker\phi = \{gN \vert \phi(gN) = eH = H\}$. Thus $\phi(gN) = gH = H$, which means $g \in H$. Hence $gN \in H/N$, so $\ker\phi \subseteq H/N$.

            Now assume $hN \in H/N$. Since $H\subseteq G$ (as $H \unlhd G$), thus $h \in G$. Therefore $hN \in G/N$, so $\phi(hN) = hH = H$. Hence $hN \in \ker\phi$ which means $H/N \subseteq \ker\phi$.

            Since $\ker\phi \subseteq H/N$ and $H/N \subseteq \ker\phi$, we must have $\ker\phi = H/N$.
        \end{itemize}

        By the Fundamental Homomorphism Theorem (\myref{thrm-isomorphism-1}), we have $\frac{G/N}{\ker\phi} \cong \im\phi$, which means
        \[
            \frac{G/N}{H/N} \cong G/H,
        \]
        proving statement 3.
    \end{enumerate}
    This proves the theorem.
\end{proof}

\begin{example}
    Take $G = \Z$, $H = m\Z$, and $N = mn\Z$. Note that clearly $H, N \leq G$, and since $G$ is abelian, we must also have $H \unlhd G$ and $N \unlhd G$. By the Third Isomorphism Theorem (\myref{thrm-isomorphism-3}), statement 3,
    \[
        \frac{G/N}{H/N} \cong G/H.
    \]
    Note $G/H = \Z/(m\Z) \cong \Z_m$ by \myref{problem-Zn-isomorphic-to-Z-by-nZ}. Note also
    \[
        \frac{G/N}{H/N} = \frac{\Z/(mn\Z)}{m\Z/(mn\Z)}.
    \]
    Hence we see that
    \[
        \frac{\Z/(mn\Z)}{m\Z/(mn\Z)} \cong \Z/(m\Z) \cong \Z_m.
    \]
\end{example}

\begin{exercise}
    Suppose $x$ and $y$ are positive integers such that $y = mx$ for some integer $m$. Let $H = x\Z$ and $N = y\Z$ be groups under addition.
    \begin{partquestions}{\roman*}
        \item Explain why $N \subseteq H$.
        \item Find a group $G$ such that $H \lhd G$ and $N \lhd G$.
        \item Hence find the order of $H/N$.
    \end{partquestions}
\end{exercise}

\newpage

\section{Problems}
\begin{problem}
    Let $G$ be a group. Prove that $G/G$ is isomorphic to the trivial group.
\end{problem}

\begin{problem}
    Let $R = (\R, +)$. Also let $G = R^2$ and $H = \left\{(r\sqrt2, r\sqrt3) \vert r\in R\right\}$ be groups under component-wise addition. Prove that $G/H \cong R$.
\end{problem}

\begin{problem}\label{problem-subgroup-product-equal-to-subgroup-if-one-is-subgroup-of-another}
    Let $G$ be a group. Let $H$ and $K$ be subgroups of $G$ such that $K \subseteq H$. Prove that $HK = H$.
\end{problem}

\begin{problem}\label{problem-cartesian-product-of-group-by-group-isomorphic-to-group}
    Let $G$ be an abelian group with operation $\ast$. Let $I = \{(g, g^{-1}) \vert g \in G\}$ be a group under component-wise application of $\ast$.
    \begin{partquestions}{\roman*}
        \item Show that $I \cong G$.
        \item Hence prove $G^2/G \cong G$ by considering a suitable homomorphism.
    \end{partquestions}
\end{problem}

\begin{problem}\label{problem-mZ/nZ-isomorphic-to-Zn/m}
    Let $G = m\Z$ and $H = n\Z$ be groups under addition, where $m\vert n$ and $m \neq n$. Let the map $\phi: G \to \Z/({\frac nm}\Z)$ be defined such that
    \[
        \phi(am) = a + \frac nm \Z.
    \]
    Prove that $G/H \cong \Z_{\frac nm}$.
\end{problem}

\chapter{More Types of Groups}
We previously introduced some types of groups, like the cyclic groups, the dihedral groups, and the Klein-four group. We now introduce more types of groups to expand our knowledge of the types of groups that are involved in group theory.

\section{More About Cyclic Groups}
We previously covered several properties of cyclic groups.
\begin{itemize}
    \item Every cyclic group is abelian. (\myref{prop-cyclic-group-is-abelian})
    \item A finite group $G$ is cyclic if and only if there exists an element $g$ in the group $G$ with the same order as the group. (\myref{thrm-cyclic-group-has-element-with-same-order})
    \item If $G$ is cyclic and $H \leq G$ then $G/H$ is cyclic. (\myref{exercise-quotient-group-of-cyclic-group-is-cyclic})
    \item Any subgroup of a cyclic group is cyclic. (\myref{problem-subgroup-of-cyclic-group-is-cyclic})
    \item $\Z / (n\Z) \cong \Z_n$. (\myref{problem-Zn-isomorphic-to-Z-by-nZ})
    \item $\Z_m \times \Z_n \cong \Z_{mn}$ if and only if $\gcd(m,n) = 1$. (\myref{thrm-Zm-cross-Zn-isomorphic-to-Zmn-condition})
    \item $m\Z / n\Z \cong \Z_{\frac nm}$. (\myref{problem-mZ/nZ-isomorphic-to-Zn/m})
\end{itemize}

\begin{exercise}\label{exercise-Zmn-mod-Zn-cong-Zn}
    Prove that
    \[
        \Z_{mn} / \Z_m \cong \Z_n.
    \]
    for any positive integers $m$ and $n$.
\end{exercise}

Note that we may denote the finite cyclic group of order $n$ by $\Cn{n}$ instead of $\Z_n$. This may be sometimes used to distinguish (finite) cyclic subgroups of a group from the integers modulo $n$.

We prove an essential theorem on cyclic groups. Before that, we require the following lemma.
\begin{lemma}\label{lemma-order-of-an-element-that-is-equivalent-to-identity}
    Let $\Cn{n}$ have generator $g$. Then $g^k = e$ if and only if $k$ is a multiple of $n$.\newline
    (That is, if $g^k = e$ then $n$ divides $k$, i.e. $n\vert k$.)
\end{lemma}
\begin{proof}
    We note that the order of $g$ is $n$, since $g$ is a generator of $G$. Thus by \myref{problem-element-to-power-of-multiple-of-order-is-identity} the lemma is proven.
\end{proof}

\newpage

\begin{theorem}\label{thrm-order-of-element-in-cyclic-group}
    Let $\Cn{n}$ have generator $g$. Let $x = g^k$ for some integer $k$. Then
    \[
        |x| = \frac{n}{\gcd(k,n)}.
    \]
\end{theorem}
\begin{proof}
    For brevity let $m = |x|$ and suppose $k = \lambda \gcd(k, n)$ for some positive integer $\lambda$.

    Let $\mathcal{X} = \langle x \rangle = \{x, x^2, x^3, \dots, x^m\}$. Note that $\mathcal{X}$ is a cyclic group with order $m$ and generator $x$.

    Observe that
    \[
        x^m = \left(g^k\right)^m = g^{km}
    \]
    and since $|x| = m$, this means that $x^m = e$. Hence $g^{km} = e$. By \myref{lemma-order-of-an-element-that-is-equivalent-to-identity} on $\Cn{n}$, $n \vert kn$. Dividing both sides by $\gcd(k,n)$ yields
    \[
        \frac{n}{\gcd(k,n)} \vert \frac{km}{\gcd(k,m)} = \lambda m.
    \]
    Note that $\gcd\left(\lambda, \frac{n}{\gcd(k,n)}\right) = \gcd\left(\frac{k}{\gcd(k,n)}, \frac{n}{\gcd(k,n)}\right) = 1$. Thus $\frac{n}{\gcd(k,n)}$ does not divide $\lambda$, further meaning $\frac{n}{\gcd(k,n)} \vert m$ (\myref{theorem-n-divides-ab-and-n-coprime-with-a-implies-n-divides-b}).

    Also note that
    \begin{align*}
        x^{\frac{n}{\gcd(k,n)}} &= \left(g^k\right)^{\frac{n}{\gcd(k,n)}}\\
        &= \left(g^n\right)^{\frac{k}{\gcd(k,n)}}\\
        &= \left(g^n\right)^\lambda\\
        &= e^\lambda & (\text{since } |g| = n)\\
        &= e,
    \end{align*}
    implying $x^{\frac{n}{\gcd(k,n)}} = e$. By \myref{lemma-order-of-an-element-that-is-equivalent-to-identity} on group $\mathcal{X}$, we have $m \vert \frac{n}{\gcd(k,n)}$.

    Since $\frac{n}{\gcd(k,n)} \vert m$ and $m \vert \frac{n}{\gcd(k,n)}$ simultaneously, thus $m = \frac{n}{\gcd(k,n)}$, that is,
    \[
        |x| = \frac{n}{\gcd(k,n)}.\qedhere
    \]
\end{proof}

\begin{exercise}
    The number 12 is equivalent to 0 in the group $\Z_n$. What are the possible value(s) of $n$?
\end{exercise}
\begin{exercise}
    In the group $\Z_{210}$, find the order of 10, 42, 75, and 140.
\end{exercise}

\begin{corollary}\label{corollary-element-in-cyclic-group-is-generator-iff-gcd-is-1}
    Let $\Cn{n}$ have generator $g$. Then $g^m$ is also a generator if and only if $\gcd(m, n) = 1$.
\end{corollary}
\begin{proof}
    We prove the forward direction first. Suppose that $\Cn{n}$ has a generator of $g^m$. On one hand, $|g^m| = n$ since a generator of a group necessarily has to have an order equal to that of the group. On another hand, by \myref{thrm-order-of-element-in-cyclic-group}, $|g^m| = \frac{n}{\gcd(m, n)}$. Hence, $n = \frac{n}{\gcd(m, n)}$ which quickly implies $\gcd(m, n) = 1$.

    Now we prove the reverse direction. Suppose $\gcd(m,n) = 1$. Then $|g^m| = \frac{n}{\gcd(m,n)} = \frac{n}{1} = n$ by \myref{thrm-order-of-element-in-cyclic-group}. Hence $g^m$ is a generator of $\Cn{n}$ by \myref{thrm-cyclic-group-has-element-with-same-order}.
\end{proof}

\begin{exercise}
    Find all the generators of the following groups.
    \begin{partquestions}{\alph*}
        \item $\Z_{10}$
        \item $\Z_{101}$
    \end{partquestions}
\end{exercise}

\section{Quaternion Group}
We look at an interesting group that has use in computer graphics: the \textbf{quaternion group}. We present one definition here.
\begin{definition}\label{definition-quaternion-group}
    The quaternion group\index{quaternion group} is
    \[
            \mathrm{Q} = \{1, -1, i, -i, j, -j, k, -k\}
    \]
    where
    \begin{multicols}{2}
        \begin{itemize}
            \item 1 is the identity;
            \item $(-1)^2 = 1$;
            \item $i^2 = j^2 = k^2 = -1$;
            \item $ij = k$ and $ji = -k$;
            \item $jk = i$ and $kj = -i$; and
            \item $ki = j$ and $ik = -j$.
        \end{itemize}
    \end{multicols}
\end{definition}

The quaternion group has 4 proper non-trivial subgroups, namely $\langle -1 \rangle = \{\pm1\}$, $\langle i \rangle = \{\pm1, \pm i\}$, $\langle j \rangle = \{\pm1, \pm j\}$, and $\langle k \rangle = \{\pm1, \pm k\}$.

With these subgroups, we can write a presentation for $\mathrm{Q}$.
\begin{proposition}
    $\mathrm{Q} = \langle i, j \rangle$.
\end{proposition}
\begin{proof}
    For brevity let $G = \langle i, j \rangle$. We note that $i^0 = 1$, $i^4 = j^4 = 1$ and $ji = -k = -ij = i^3j$. Hence,
    \begin{align*}
        G &= \{1, i, i^2, i^3, j, ij, i^2j, i^3j\}\\
        &= \{1, i, -1, -i, j, ij, -j, -ij\}\\
        &= \{1, -1, i, -i, j, -j, ij, -ij\} & (\text{reordering})\\
        &= \{1, -1, i, -i, j, -j, k, -k\} & (\text{since } ij = k)\\
        &= \mathrm{Q}
    \end{align*}
    so $\mathrm{Q} = \langle i, j \rangle$.
\end{proof}

Thus, an alternate definition of $\mathrm{Q}$ is
\[
    \mathrm{Q} = \langle \alpha, \beta \vert \alpha^4 = e,\; \alpha^2 = \beta^2, \text{ and } \beta\alpha = \alpha^3\beta \rangle.
\]
One sees clearly that $\alpha = i$ and $\beta = j$ in this definition.

\begin{exercise}\label{exercise-normal-subgroups-of-quarternion-group}
    Find all the normal subgroups of the quaternion group $\mathrm{Q}$.\newline
    (\textit{Hint: consider \myref{problem-subgroup-of-index-2}.})
\end{exercise}

\section{Alternating Group}
The alternating group is a very important group in the field of group theory. However, before we can properly define it, we need to introduce the idea of \textbf{transpositions}.

\subsection{Transpositions}
\begin{definition}
    A \textbf{transposition}\index{transposition} is a 2-cycle. That is, a transposition $\tau$ is represented as $(a\quad b)$ in cycle notation.
\end{definition}
For example, we see that (1 5), (4 7), (3 6) etc. are transpositions, while (1 4 5), (3 4 6 5 9), (1 3 4 5) etc. are not. One sees clearly that every transposition is its own inverse.

We call transpositions which has cycle decomposition $\begin{pmatrix}i&i+1\end{pmatrix}$ \textbf{adjacent transpositions}\index{transposition!adjacent}, and we may denote them by $\alpha$.

We look at one lemma which can be said to help `break up' transpositions into a composition of adjacent transpositions. Ignore line breaks in the following lemma.
\begin{lemma}\label{lemma-decompose-transposition}
    Let a transposition $\tau = (i\quad i+d)$ where $d$ is a positive integer. Then $\tau$ is equal to the permutation
    \begin{align*}
        & (i\quad i+1)(i+1\quad i+2)\cdots(i+d-2\quad i+d-1)(i+d-1\quad i+d)\\
        &\quad (i+d-2\quad i+d-1)\cdots(i+1\quad i+2)(i\quad i+1)
    \end{align*}
\end{lemma}

\begin{proof}
    We induct on $d$.

    When $d = 1$, note $\tau = (i\quad i+1)$ so the lemma is true for $d=1$.

    Assume that the given representation of $\tau$ holds for some positive integer $d$. Note this means that it holds for all $i$, including $i+1$. We are to show that it works for $d + 1$.
    \begin{align*}
        (i\quad i+(d+1)) &= (i\quad i+1)(i+1\quad i+d+1)(i\quad i+1)\\
        &= (i\quad i+1)\underbrace{((i+1)\quad (i+d)+1)}_{\text{use induction hypothesis here}}(i\quad i+1)\\
        &= (i\quad i+1)((i+1)\quad(i+1)+1)\cdots((i+1)\quad(i+1)+1)(i\quad i+1)\\
        &= (i\quad i+1)(i+1\quad i+2)\cdots(i+1\quad i+2)(i\quad i+1)
    \end{align*}
    which shows that $d+1$ works as well.
\end{proof}
\begin{remark}
    The above representation of $\tau$ uses $2d-1$ compositions.
\end{remark}
\begin{example}
    An adjacent transposition decomposition of $(4\quad9)$ is
    \[
        (4\quad 5)(5\quad 6)(6\quad 7)(7\quad 8)(8\quad 9)(7\quad 8)(6\quad 7)(5\quad 6)(4\quad 5).
    \]
\end{example}
\begin{exercise}
    `Decompose' the transposition (2 6) into a composition of adjacent transpositions.
\end{exercise}

\subsection{Links with Permutations}
As mentioned before, a transposition is a 2-cycle permutation. We note that transpositions are important as seen in the following lemma.
\begin{lemma}\label{lemma-permutations-as-product-of-transpositions}
    Every permutation can be expressed as a product of transpositions.
\end{lemma}
\begin{proof}[Proof (see {\cite[\S 80 Corollary]{clark_1984}})]
    For the identity $\id$ it can be expressed as $(a\quad b)(a \quad b)$. For any permutation with length $k \geq 2$, say $\sigma = \begin{pmatrix}a_1 & a_2 & a_3 & \cdots & a_k\end{pmatrix}$, write
    \[
        \sigma = \begin{pmatrix}a_1 & a_k\end{pmatrix}\begin{pmatrix}a_1 & a_{k-1}\end{pmatrix}\begin{pmatrix}a_1 & a_{k-2}\end{pmatrix}\cdots\begin{pmatrix}a_1 & a_3\end{pmatrix}\begin{pmatrix}a_1 & a_2\end{pmatrix}
    \]
    and since every permutation is a product of cycles, thus every permutation is a product of transpositions.
\end{proof}

We now look at the idea of \textbf{inversions} inside permutations.
\begin{definition}
    Let $\sigma$ be a permutation. An \textbf{inversion}\index{permutation!inversion} of $\sigma$ between $i$ and $j$ exists if $i > j$ and $\sigma(i) < \sigma(j)$.
\end{definition}
An inversion between the elements $i$ and $j$ is denoted by either $(i, j)$ or $(\sigma(i), \sigma(j))$.
\begin{example}
    In the permutation $\sigma = \begin{pmatrix}1 & 3 & 2 & 4\end{pmatrix}$, we see that $3 > 2$ but $\sigma(3) = 2 < 4 = \sigma(2)$. Thus there is an inversion (3, 2) = (2, 4).
\end{example}

We now define the idea of \textbf{even} permutations and \textbf{odd} permutations.
\begin{definition}
    A permutation $\sigma$ is said to be \textbf{even}\index{permutation!even} if there are an even number of inversions in $\sigma$. If $\sigma$ is not even then it is said to be \textbf{odd}\index{permutation!odd}. The evenness or oddness of a permutation $\sigma$ is called the \textbf{parity of $\sigma$}\index{permutation!parity}.
\end{definition}

Counting the number of inversions in a permutation may be hard to do, so we have an alternative definition for the parity of a permutation.

\begin{theorem}\label{thrm-parity-of-permutation}
    Let $\sigma$ be a permutation. Suppose $\sigma$ can be expressed as a product of $n$ transpositions. Then $\sigma$ is even if and only if $n$ is even.
\end{theorem}
\begin{proof}
    We only need to show that the parity of $n$ and the parity of the permutation is the same to prove this.

    By \myref{lemma-permutations-as-product-of-transpositions}, all permutations can be produced by a sequence of transpositions, say $\sigma = \tau_1\tau_2\tau_3\cdots\tau_k$. By \myref{lemma-decompose-transposition}, every transposition can be written as a product of $2d - 1$ adjacent transpositions. Expressing each $\tau_i$ as a product of adjacent transpositions yields
    \[
        \sigma = \alpha_1\alpha_2\cdots\alpha_m
    \]
    where $\alpha_i$ is an adjacent transposition for $1 \leq i \leq m$. Note that the parity of $m$ is the same as that of $k$.

    It is clear that for any permutation $\pi$ and adjacent permutation $\alpha$, the permutation $\alpha\pi$ has either one more or one less inversion than $\pi$. Thus the parity of the number of inversions of a permutation is switched when composed with adjacent transpositions.

    We note that the identity permutation $\id$ is an even permutation. So $\alpha_1$ is odd, $\alpha_1\alpha_2$ is even, $\alpha_1\alpha_2\alpha_3$ is odd, etc., so $\alpha_1\alpha_2\cdots\alpha_m$ has the parity of $m$. Therefore the parity of the number of inversions of $\sigma$ is the parity of $k$ (since $m$ and $k$ have the same parity).
\end{proof}

\begin{corollary}\label{corollary-permutation-and-inverse-have-same-parity}
    $\sigma$ and $\sigma^{-1}$ have the same parity for any permutation $\sigma$.
\end{corollary}
\begin{proof}
    One observes clearly that $\sigma\sigma^{-1} = \id$. Since $\id$ is even, thus $\sigma\sigma^{-1}$ must be composed of an even number of transpositions (\myref{thrm-parity-of-permutation}).
    \begin{itemize}
        \item If $\sigma$ is even, then $\sigma$ can be expressed as a product of an even number of transpositions. Hence $\sigma^{-1}$ must also be a product of an even number of transpositions in order for $\sigma\sigma^{-1} = \id$ to be even.
        \item If $\sigma$ is odd, then $\sigma$ can be expressed as a product of an odd number of transpositions. Hence $\sigma^{-1}$ must also be a product of an odd number of transpositions in order for $\sigma\sigma^{-1} = \id$ to be even.
    \end{itemize}
    This proves the claim.
\end{proof}

We look at one final useful construct: the sign of a permutation.
\begin{definition}
    The \textbf{sign of a permutation}\index{permutation!sign} is $+1$ if the permutation is even and $-1$ if the permutation is odd.
\end{definition}
If $\sigma$ is the permutation, $N(\sigma)$ is the number of inversions in $\sigma$, and $m$ is the number of transpositions in the decomposition of $\sigma$, then the sign of $\sigma$ is given by the \textbf{signum} function:
\[
    \sgn(\sigma) = (-1)^{N(\sigma)} = (-1)^m.
\]
\begin{exercise}
    Find $\sgn\left(\begin{pmatrix}1&3&2&5&4\end{pmatrix}\right)$.
\end{exercise}

One observes that the number of inversions in $\sigma\tau$ is the same as the sum of inversions in $\sigma$ and $\tau$ separately, which means
\[
    N(\sigma\tau) = N(\sigma) + N(\tau)
\]
so
\begin{align*}
    \sgn(\sigma\tau) &= (-1)^{N(\sigma\tau)}\\
    &= (-1)^{N(\sigma) + N(\tau)}\\
    &= (-1)^{N(\sigma)}(-1)^{N(\tau)}\\
    &= \sgn(\sigma)\sgn(\tau),
\end{align*}
meaning that $\sgn$ is a \textit{multiplicative map}. One can then quickly verify \myref{corollary-permutation-and-inverse-have-same-parity} by noting
\[
    1 = \sgn(\id) = \sgn(\sigma\sigma^{-1}) = \sgn(\sigma)\sgn(\sigma^{-1})
\]
which implies $\sgn(\sigma)$ and $\sgn(\sigma^{-1})$ have the same parity.

\subsection{The Alternating Group}
We are now ready to look at the alternating group.
\begin{definition}
    The \textbf{alternating group of degree $n$}\index{alternating group!of degree $n$}, denoted by $\An{n}$, is given by
    \[
        \An{n} = \left\{\sigma \in \Sn{n} \vert \sigma \text{ is even}\right\}
    \]
    for $n \geq 2$.
\end{definition}

\begin{proposition}\label{prop-An-normal-subgroup-of-Sn}
    $\An{n} \lhd \Sn{n}$ where $n \geq 2$.
\end{proposition}
\begin{proof}
    Note that the identity function $\id \in \An{n}$ since $\id \in \Sn{n}$ and the identity is even.

    Suppose now that $\mu$ and $\sigma$ are in $\An{n}$. We show that $\mu\sigma^{-1} \in \An{n}$. By \myref{lemma-permutations-as-product-of-transpositions}, we may write
    \[
        \mu = \tau_1\tau_2\cdots\tau_{2k} \text{ and } \sigma = \tau_1'\tau_2'\cdots\tau_{2m}'.
    \]
    We note that any transposition is its own inverse, that is $\tau = \tau^{-1}$. Hence $\sigma^{-1} = \tau_{2m}'\tau_{2m-1}'\cdots\tau_2'\tau_1'$, so
    \[
        \mu\sigma^{-1} = \underbrace{\tau_1\cdots\tau_{2k}\tau_{2m}'\cdots\tau_1'}_{2(k+m) \text{ transpositions}}
    \]
    is an even permutation, meaning $\mu\sigma^{-1} \in \An{n}$. By subgroup test, $\An{n} \leq \Sn{n}$. But as $\An{n}$ contains only even permutations, thus $\An{n} < \Sn{n}$.
    
    Now take $\sigma \in \Sn{n}$ and $\mu \in \An{n}$. Note that
    \begin{align*}
        \sgn(\sigma\mu\sigma^{-1}) &= \sgn(\sigma)\sgn(\mu)\sgn(\sigma^{-1})\\
        &= \sgn(\sigma)\sgn(\sigma^{-1})\sgn(\mu)\\
        &= \sgn(\sigma\sigma^{-1})\sgn(\mu)\\
        &= \sgn(\id)\sgn(\mu)\\
        &= \sgn(\mu).
    \end{align*}
    \[
        \sgn(\sigma\mu\sigma^{-1}) = \sgn(\sigma)\sgn(\mu)\sgn(\sigma^{-1}).
    \]
    As $\mu$ is even, so $\sgn(\mu) = 1$. Hence $\sgn(\sigma\mu\sigma^{-1}) = 1$ which means that $\sigma\mu\sigma^{-1}$ is an even permutation. Therefore $\sigma\mu\sigma^{-1} \in \An{n}$, meaning that $\An{n} \lhd \Sn{n}$.
\end{proof}

\begin{proposition}\label{prop-order-of-An}
    The order of $\An{n}$ is $\frac{n!}{2}$ for $n \geq 2$.
\end{proposition}
\begin{proof}
    For brevity, let
    \[
        \mathrm{O}_n = \left\{\sigma \in \Sn{n} \vert \sigma \text{ is odd}\right\} = \Sn{n} \setminus \An{n}.
    \]
    Clearly $A_n \cup \mathrm{O}_n = \Sn{n}$ and $A_n \cap \mathrm{O}_n = \emptyset$.

    Define a map $f: \An{n} \to \mathrm{O}_n$ such that $\sigma \mapsto \begin{pmatrix}1 & 2\end{pmatrix}\sigma$. We will show that this is a bijection.
    \begin{itemize}
        \item \textbf{Injective}: Let $\mu$ and $\sigma$ be in $\An{n}$ such that $f(\mu) = f(\sigma)$. This means that $\begin{pmatrix}1 & 2\end{pmatrix}\mu = \begin{pmatrix}1 & 2\end{pmatrix}\sigma$. Now left-applying $\begin{pmatrix}1 & 2\end{pmatrix}$ on both sides yields $\mu = \sigma$ (since transpositions are their own self-inverse).
        
        \item \textbf{Surjective}: Take $\mu \in \mathrm{O}_n$, say $\mu = \tau_1\tau_2\cdots\tau_{2k-1}$ where $\tau_i$ is a transposition. Clearly,
        \[
            \mu = \underbrace{\begin{pmatrix}1 & 2\end{pmatrix}\begin{pmatrix}1 & 2\end{pmatrix}}_{\id}\tau_1\tau_2\cdots\tau_{2k-1}.
        \]
        Consider $\sigma = \begin{pmatrix}1 & 2\end{pmatrix}\tau_1\tau_2\cdots\tau_{2k-1}$, which is an even permutation and thus is in $\An{n}$. Observe that
        \begin{align*}
            f(\sigma) &= \begin{pmatrix}1 & 2\end{pmatrix}\sigma\\
            &= \begin{pmatrix}1 & 2\end{pmatrix}\left(\begin{pmatrix}1 & 2\end{pmatrix}\tau_1\tau_2\cdots\tau_{2k-1}\right)\\
            &= \tau_1\tau_2\cdots\tau_{2k-1}\\
            &= \mu
        \end{align*}
        which means that $\mu \in \mathrm{O}_n$ has a pre-image $\sigma$ in $\An{n}$.
    \end{itemize}
    This proves that $f$ is a bijection, so $|\An{n}| = |\mathrm{O}_n|$.
    
    Since $A_n \cup \mathrm{O}_n = \Sn{n}$ and $A_n \cap \mathrm{O}_n = \emptyset$, thus $|\Sn{n}| = |\An{n}| + |\mathrm{O}_n| = 2|\An{n}|$. Now because $|\Sn{n}| = n!$ by \myref{exercise-order-of-Sn}, therefore $|\An{n}| = \frac{n!}2$.
\end{proof}

\begin{exercise}
    List all elements of $\An{3}$.
\end{exercise}

\section{Group of Units Modulo \texorpdfstring{$n$}{n}}
We look at a useful group in Number Theory: the \textbf{group of units modulo $n$}.

\begin{definition}
    For a positive integer $n \geq 2$, the \textbf{group of units modulo $n$}\index{group of units modulo $n$}, denoted by $\Un{n}$, is the set
    \[
        \Un{n} = \{m \in \Z \vert 1 \leq m < n \text{ and } \gcd(m, n) = 1\}
    \]
    together with the operation $\otimes_n$ (multiplication modulo $n$).
\end{definition}

\begin{proposition}
    $\Un{n}$ is an abelian group.
\end{proposition}
\begin{proof}
    We first prove that $\Un{n}$ satisfies the four group axioms to show that $\Un{n}$ is a group.
    \begin{enumerate}
        \item \textbf{Closure}: Let $x, y \in \Un{n}$. Then $\gcd(x, n) = \gcd(y, n) = 1$. Hence $\gcd(xy, n) = 1$ which means that $\gcd(x\otimes_ny,n)=1$. Thus $x\otimes_ny \in \Un{n}$.
        
        \item \textbf{Associativity}: Multiplication is associative (\myref{axiom-multiplication-is-associative}), so $\otimes_n$ is associative.
        
        \item \textbf{Identity}: Note that 1 is the identity in $\Un{n}$ since $1 \otimes_n x = x$.
        
        \item \textbf{Inverse}: Let $x$ be in $\Un{n}$, meaning $\gcd(x, n) = 1$. Then \myref{prop-multiplicative-inverse-exists-iff-coprime} tells us that there exists an $m$ such that $mx \equiv 1 \pmod n$. Therefore $m \otimes_n x = 1$, which means $m$ is the inverse of $x$.
    \end{enumerate}
    Now multiplication is commutative (\myref{axiom-multiplication-is-commutative}), so multiplication modulo $n$ is also commutative. Thus $\Un{n}$ is an abelian group under $\otimes_n$.
\end{proof}

\begin{exercise}
    List the elements of $\Un{10}$.
\end{exercise}

There is another representation of $\Un{n}$, but we leave it to later in this section.

A useful Number Theory function that will occur frequently in this section is \textbf{Euler's totient function}\index{Euler's totient function}.

\begin{definition}
    \textbf{Euler's totient function} $\totient$ gives the number of positive integers that are smaller than and coprime to a positive integer $x$. That is,
    \[
        \totient(x) = \left|\{n \in \Z \vert 1 \leq n < x \text{ and } \gcd(n, x) = 1\}\right|.
    \]
\end{definition}
In particular, if the positive integer $x = p_1^{n_1}p_2^{n_2}\cdots p_k^{n_k}$, where $p_1, p_2, \dots, p_k$ are distinct primes and $n_1,n_2,\dots,n_k$ are positive integers, then
\[
    \totient(x) = x \left(1 - \frac1{p_1}\right)\left(1 - \frac1{p_2}\right)\cdots\left(1 - \frac1{p_k}\right).
\]
Note $|\Un{n}| = \totient(n)$ by definition of $\Un{n}$.

\begin{exercise}\label{exercise-order-of-a-divides-phi-a}
    Let $a \in \Un{n}$. Prove that $|a|$ divides $\totient(n)$.
\end{exercise}

We look at the specific case where $|a| = \totient(n)$.
\begin{definition}
    Suppose $\gcd(a, n) = 1$. Then $a$ is a \textbf{primitive root modulo $n$}\index{primitive root} if $|a| = \totient(n)$ in $\Un{n}$.
\end{definition}

We note that we have a way to determine whether a primitive root modulo $n$ exists. However, the proof is way too complex to note down here, so we leave it as an axiom.
\begin{axiom}\label{axiom-primitive-root-modulo-p}
    There is a primitive root modulo $n$ if and only if $n$ is 1, 2, 4, $p^k$, or $2p^k$ where $p$ is an odd prime and $k$ is a positive integer.
\end{axiom}

We now note the condition for $\Un{n}$ to be cyclic.
\begin{proposition}\label{prop-Un-cyclic-only-if-exists-primitive-root}
    $\Un{n}$ is cyclic if and only if there is a primitive root modulo $n$.
\end{proposition}
\begin{proof}
    We first prove the forward direction. Let $\Un{n}$ be cyclic. Then there exists an element $r \in \Un{n}$ such that $|r| = |\Un{n}| = \totient(n)$, so $r$ is a primitive root modulo $n$ by definition of a primitive root.

    We now prove the reverse direction. Let $r \in \Un{n}$ be a primitive root modulo $n$. Then $\langle r \rangle \cong \Z_{\totient(n)}$. Note that since $|\langle r \rangle| = \totient(n) = |\Un{n}|$, therefore $r$ is a generator of $\Un{n}$, meaning $\Un{n}$ is cyclic.
\end{proof}
\begin{remark}
    In fact, what \myref{prop-Un-cyclic-only-if-exists-primitive-root} shows is that $\Un{n} \cong \Z_{\totient(n)}$ if there exists a primitive root modulo $n$.
\end{remark}

We end this section by looking at an alternate representation of $\Un{n}$. 
\begin{definition}
    Define the group
    \[
        \left(\Z/(n\Z)\right)^{\times} = \left\{m + n\Z \ \vert \ m,n \in \Z,\; 1 \leq m < n,\; \gcd(m, n)=1\right\}
    \]
    under the operation $\ast$ where $(a+n\Z)\ast(b+n\Z) = (a\otimes_n b) + n\Z$ for $n \geq 2$.
\end{definition}
\begin{proposition}
    $\Un{n} \cong \left(\Z/(n\Z)\right)^{\times}$.
\end{proposition}
\begin{proof}
    Define the map $\phi: \Un{n} \to \left(\Z/(n\Z)\right)^{\times}$ such that $m \mapsto m + n\Z$. We show that $\phi$ is an isomorphism.

    \begin{itemize}
        \item \textbf{Homomorphism}: Let $x, y \in \Un{n}$. Then $\phi$ is a homomorphism since
        \begin{align*}
            \phi(x \otimes_n y) &= (x \otimes_n y) + n\Z\\
            &= (x + n\Z) \ast (y + n\Z)\\
            &= \phi(x) \ast \phi(y).
        \end{align*}

        \item \textbf{Injective}: Suppose we have $x, y \in \Un{n}$ such that $\phi(x) = \phi(y)$. Then this means that $x + n\Z = y + n\Z$. Thus
        \[
            \{x + pn \vert p \in \Z \} = \ \{y + qn \vert q \in \Z \}.
        \]
        Hence we conclude that $x \equiv y \pmod{n}$. But since $1 \leq x, y < n$, we must have $x = y$. Therefore $\phi(x) = \phi(y)$ implies $x = y$, meaning $\phi$ is injective.

        \item \textbf{Surjective}: Let $x + n\Z \in (\Z/(n\Z))^\times$, so $\gcd(x,n) = 1$. Using Euclid's division lemma (\myref{lemma-euclid-division}) on $x$ yields
        \[
            x = qn + r, \text{ where } 0 \leq r < n.
        \]
        Note that
        \begin{align*}
            x + n\Z &= \{x + kn \vert k \in \Z\}\\
            &= \{(qn + r) + kn \vert k \in \Z\}\\
            &= \{r + n(\underbrace{q+k}_{\text{In } \Z}) \vertalt k \in \Z\}\\
            &= r + n\Z
        \end{align*}
        with $0 \leq r < n$. Note that if $r = 0$, this means that $x = qn$, which means that $\gcd(x, n) = \gcd(qn, n) = n \neq 1$. Thus, $r \neq 0$, meaning $1 \leq r < n$ so $r \in \Un{n}$.

        Observing that $\phi(r) = r + n\Z = x + n\Z$ shows that $x + n\Z$ has a pre-image $r$ in $\Un{n}$, which means that $\phi$ is surjective.
    \end{itemize}

    Thus $\phi$ is an isomorphism, meaning $\Un{n} \cong \left(\Z/(n\Z)\right)^{\times}$.
\end{proof}

\section{Groups of Matrices}
\subsection{Introduction to Matrices}\label{subsection-intro-to-matrices}
Before we can introduce the groups of matrices, we need to understand what they are, and to learn some operations that can be applied to matrices.

A matrix\index{matrix} is a rectangular array of numbers, symbols, or expressions, arranged in rows and columns. They are used to represent mathematical objects or properties of objects.

For example,
\[
    \textbf{M} = \begin{pmatrix}
    1 & 2\\
    3 & 4\\
    5 & 6
    \end{pmatrix}
\]
is a matrix. In this section, we consider only square matrices\index{matrix!square}, which have the same number of rows as columns. For example,
\[
    \textbf{A} = \begin{pmatrix}
    1 & 2 & 3\\
    4 & 5 & 6\\
    7 & 8 & 9
    \end{pmatrix}, \textbf{B} = \begin{pmatrix}
    -1 & 0 & 1 & 1\\
    1 & 0 & -1 & 1\\
    1 & 1 & 1 & 1\\
    2 & 3 & 3 & 3
    \end{pmatrix}, \textrm{ and } \textbf{C} = \begin{pmatrix}
    x & x^2\\
    x^3 & x^4\\
    \end{pmatrix}
\]
are square matrices. For brevity, for a matrix \textbf{M}, we denote the element in the $i$th row and the $j$th column by $m_{i,j}$ (where $1 \leq i, j \leq n$ with $n$ being the number of rows and columns in \textbf{M}). For example, using the above matrices, $a_{2,3} = 6$, $b_{2,4}=1$, and $c_{1,2} = x^2$.

We now introduce the idea of \textbf{matrix multiplication}\index{matrix!multiplication}. Consider two square matrices \textbf{A} and \textbf{B} with the same number of rows and columns (say, $n$ rows and columns). Let their product, denoted \textbf{AB}, be the matrix \textbf{C}. Then
\[
    c_{i,j} = \sum_{k=1}^n a_{i,k}b_{k,j}
\]
for any $1 \leq i, j \leq n$. For example, if $\textbf{A} = \begin{pmatrix}1 & 2\\3 & 4\end{pmatrix}$ and $\textbf{B} = \begin{pmatrix}5 & 6\\7 & 8\end{pmatrix}$ then
\[
    \textbf{AB} = \begin{pmatrix}1\times5+2\times7 & 1\times6+2\times8\\3\times5+4\times7 & 3\times6+4\times8\end{pmatrix}
    = \begin{pmatrix}19 & 22\\43 & 50\end{pmatrix}.
\]
Note that matrix multiplication is \textbf{not} commutative, as
\[
    \textbf{AB} = \begin{pmatrix}19 & 22\\43 & 50\end{pmatrix} \text{ but } \textbf{BA} = \begin{pmatrix}23 & 34\\31 & 46\end{pmatrix}.
\]

\begin{exercise}
    Find the matrix given by the product
    \[
        \begin{pmatrix}1&1&0\\0&1&0\\0&1&1\end{pmatrix}\begin{pmatrix}1&1&1\\1&0&1\\1&1&1\end{pmatrix}.
    \]
\end{exercise}

Matrices can also be `multiplied' by real numbers (known as \textbf{scalar multiplication}\index{scalar multiplication}). For example,
\[
    1.23\begin{pmatrix}1 & 2\\3 & 4\end{pmatrix} = \begin{pmatrix}1.23 & 2.46\\3.69 & 4.92\end{pmatrix}.
\]

We now look at a special kind of square matrix: the \textbf{identity matrix of order $n$}\index{matrix!identity}. It is denoted $\textbf{I}_n$ and it is a matrix with $n$ rows and columns with 1s on the main diagonal and 0s everywhere else. For example,
\[
    \textbf{I}_2 = \begin{pmatrix}1 & 0\\0 & 1\end{pmatrix},
\textbf{I}_3 = \begin{pmatrix}1 & 0 & 0\\0 & 1 & 0\\0 & 0 & 1\end{pmatrix}, \text{ and }
\textbf{I}_4 = \begin{pmatrix}1 & 0 & 0 & 0\\0 & 1 & 0 & 0\\0 & 0 & 1 & 0\\0 & 0 & 0 & 1\end{pmatrix}.
\]
We note that for any matrix \textbf{M} with $n$ rows and columns,
\[
    \textbf{MI}_n = \textbf{I}_n\textbf{M} = \textbf{M}.
\]

A square matrix may have an \textbf{inverse}\index{matrix!inverse}. Consider a square matrix \textbf{A} with $n$ rows and columns. Then \textbf{B} is an inverse of \textbf{A} if
\[
    \textbf{AB} = \textbf{BA} = \textbf{I}_n.
\]
For example, consider the matrices
\[
    \textbf{A} = \begin{pmatrix}1&1&0\\ 0&1&0\\ 0&1&1\end{pmatrix} \text{ and } \textbf{B} = \begin{pmatrix}1&-1&0\\0&1&0\\0&-1&1\end{pmatrix}.
\]
Note that
\[
    \textbf{AB} = \textbf{BA} = \textbf{I}_3
\]
so \textbf{B} is the inverse of \textbf{A} (and \textbf{A} is the inverse of \textbf{B}). We denote the inverse of a square matrix \textbf{M} by $\textbf{M}^{-1}$. We note that $\textbf{I}_n^{-1} = \textbf{I}_n$ but we do not prove it here.

One last thing we introduce here is the idea of a \textbf{matrix determinant}\index{matrix!determinant} (or simply the \textbf{determinant}). The determinant is only well defined for square matrices (say $\textbf{A}$), and is denoted by $\det(\textbf{A})$ or $\det \textbf{A}$. The rule for the determinant changes as we increase the number of rows and columns in the square matrix, so we only look at small cases.
\begin{itemize}
    \item If the square matrix only has one row, then its determinant is the only element in the matrix. Thus, if $\textbf{A} = (a_{1,1})$ then $\det \textbf{A} = a_{1,1}$.
    \item If the square matrix has two rows, such as the matrix $\begin{pmatrix}a & b\\c & d\end{pmatrix}$, then its determinant is $ad-bc$.
    \item If the square matrix has three rows, like $\begin{pmatrix}a & b & c \\ d & e & f \\ g & h & i\end{pmatrix}$, then its determinant is $aei+bfg+cdh-ceg-bdi-afh$.
\end{itemize}
An important property of the determinant is that it is a \textit{multiplicative map}: for two square matrices \textbf{A} and \textbf{B},
\[
    \det (\textbf{AB}) = \det(\textbf{A}) \times \det(\textbf{B}).
\]
Finally, not all square matrices has an inverse. The necessary and sufficient condition that determines whether a square matrix \textbf{M} has an inverse is whether $\det \textbf{M} \neq 0$. That is,
\[
    \textbf{M}^{-1} \text{ exists if and only if } \det \textbf{M} \neq 0.
\]

We conclude this subsection with a few properties of the determinant that we state but not prove:
\begin{itemize}
    \item $\det(\textbf{I}_n) = 1$;
    \item $\det(\textbf{M}^{-1}) = \left(\det \textbf{M}\right)^{-1}$; and
    \item $\det(k\textbf{M}) = k^n \det\textbf{M}$ for a matrix \textbf{M} with $n$ rows and columns.
\end{itemize}

\subsection{General Linear Group over the Real Numbers}\label{subsection-GLR-matrix-group}
With an introduction of matrices out of the way, we can introduce the first of two important matrix groups: the \textbf{General Linear Group of degree $n$} over the real numbers.
\begin{definition}
    The \textbf{General Linear Group of degree $n$}\index{general linear group} over the real numbers is denoted by $\GL{n}{\R}$ and is the group with set
    \[
        \left\{\textbf{M} \vert \text{\textbf{M} is a matrix with } n \text{ rows and columns, and } \det \textbf{M} \neq 0\right\}
    \]
    under the operation of matrix multiplication.
\end{definition}
In other words, $\GL{n}{\R}$ is the group of real-valued matrices with $n$ rows and columns that has an inverse. We show that $\GL{n}{\R}$ is a group under matrix multiplication.
\begin{proof}
    We need to prove the four group axioms.
    \begin{enumerate}
        \item \textbf{Closure}: Consider two matrices \textbf{A} and \textbf{B} in $\GL{n}{\R}$. Then that means that $\det \textbf{A} \neq 0$ and $\det \textbf{B} \neq 0$. Since $\det(\textbf{AB}) = (\det \textbf{A})(\det \textbf{B})$, thus $\det(\textbf{AB}) \neq 0$. Note also that $\textbf{AB}$ has $n$ rows and columns. Therefore $\textbf{AB}$ is also in $\GL{n}{\R}$, meaning that it is closed under matrix multiplication.
        
        \item \textbf{Associativity}: Consider three matrices \textbf{A}, \textbf{B}, and \textbf{C} in $\GL{n}{\R}$.
        \begin{itemize}
            \item Consider $(\textbf{AB})\textbf{C}$. Let $\textbf{R} = \textbf{AB}$ and $\textbf{S} = (\textbf{AB})\textbf{C}$. Then
            \[
                r_{i,k} = \sum_{l=1}^n a_{i,l}b_{l,k} \text{ and } s_{i,j} = \sum_{k=1}^n r_{i,k}c_{k,j}
            \]
            which means that
            \[
                s_{i,j} = \sum_{k=1}^n \left(\sum_{l=1}^n a_{i,l}b_{l,k}\right)c_{k,j} = \sum_{k=1}^n \sum_{l=1}^n (a_{i,l}b_{l,k})c_{k,j}.
            \]
            \item Now consider $\textbf{A}(\textbf{BC})$. Let $\textbf{R} = \textbf{BC}$ and $\textbf{S} = \textbf{A}(\textbf{BC})$. Then
            \[
                r_{l,j} = \sum_{k=1}^nb_{l,k}c_{k,j} \text{ and } s_{i,j} = \sum_{l=1}^n a_{i,l}r_{l,j}
            \]
            which means that
            \[
                s_{i,j} = \sum_{l=1}^n a_{i,l}\left(\sum_{k=1}^nb_{l,k}c_{k,j}\right) = \sum_{l=1}^n\sum_{k=1}^n a_{i,l}(b_{l,k}c_{k,j}).
            \]
        \end{itemize}
        Now, multiplication is associative (\myref{axiom-multiplication-is-associative}). So
        \[
                (a_{i,l}b_{l,k})c_{k,j} = a_{i,l}(b_{l,k}c_{k,j})
        \]
        which means
        \[
            \sum_{k=1}^n \sum_{l=1}^n (a_{i,l}b_{l,k})c_{k,j} = \sum_{l=1}^n\sum_{k=1}^n a_{i,l}(b_{l,k}c_{k,j}),
        \]
        thereby proving that matrix multiplication is associative.
        
        \item \textbf{Identity}: We note that $\det \textbf{I}_n = 1 \neq 0$, so $\textbf{I}_n$ is in $\GL{n}{\R}$. Since $\textbf{MI}_n = \textbf{I}_n\textbf{M} = \textbf{M}$ for any matrix $\textbf{M} \in \GL{n}{\R}$, thus $\textbf{I}_n$ is the identity of $\GL{n}{\R}$.
        
        \item \textbf{Inverse}: Let \textbf{M} be a matrix in $\GL{n}{\R}$. As $\det \textbf{M} \neq 0$, thus $\textbf{M}^{-1}$ exists. By properties of the determinant, we know that $\det \left(\textbf{M}^{-1}\right) = \left(\det \textbf{M}\right)^{-1}$, and since $\det \textbf{M} \neq 0$ thus $\det \textbf{M}^{-1} \neq 0$. Hence $\textbf{M}^{-1} \in \GL{n}{\R}$. Now because $\textbf{MM}^{-1} = \textbf{M}^{-1}\textbf{M} = \textbf{I}_n$, thus $\textbf{M}^{-1}$ is the inverse of \textbf{M} in $\GL{n}{\R}$.
    \end{enumerate}
    Thus $\GL{n}{\R}$ is a group.
\end{proof}

\subsection{Special Linear Group over the Real Numbers}
We look at another group of matrices: the \textbf{Special Linear Group of degree $n$} over the real numbers.
\begin{definition}
    The \textbf{Special Linear Group of degree $n$}\index{special linear group} over the real numbers is denoted by $\SL{n}{\R}$ and is the group with set
    \[
        \left\{\textbf{M} \vert \text{\textbf{M} is a matrix with } n \text{ rows and columns, and } \det \textbf{M} = 1\right\}
    \]
    under matrix multiplication.
\end{definition}
One sees clearly that $\SL{n}{\R}$ is a sub\textit{set} of $\GL{n}{\R}$: the set of $\GL{n}{\R}$ requires non-zero determinant while the set of $\SL{n}{\R}$ requires the determinant to be 1, which satisfies the non-zero determinant requirement. What we want to prove here is that $\SL{n}{\R}$ is a sub\textit{group} of $\GL{n}{\R}$. In fact,
\begin{proposition}
    $\SL{n}{\R} \lhd \GL{n}{\R}$.
\end{proposition}
\begin{proof}
    We consider the subgroup test. Clearly $\textbf{I}_n$ is in $\SL{n}{\R}$ since its determinant is 1.

    Let \textbf{A} and \textbf{B} be matrices in the set $\SL{n}{\R}$. This means that $\det \textbf{A} = 1$ and $\det \textbf{B} = 1$. Note that $\det(\textbf{B}^{-1}) = (\det \textbf{B})^{-1} = 1^{-1} = 1$. Thus, $\det \textbf{AB}^{-1} = (\det \textbf{A})(\det(\textbf{B}^{-1})) = 1 \times 1  = 1$ which means that $\textbf{AB}^{-1}$ is also in $\SL{n}{\R}$. Hence by the subgroup test, $\SL{n}{\R} < \GL{n}{\R}$.
    
    Now let $\textbf{M}$ be a matrix in $\GL{n}{\R}$ and $\textbf{N}$ be a matrix in $\SL{n}{\R}$. We are to show that $\textbf{MNM}^{-1}$ is in $\SL{n}{\R}$. Since $\textbf{M} \in \GL{n}{\R}$ thus $\det \textbf{M} \neq 0$, meaning $\det \textbf{M}^{-1} = (\det \textbf{M})^{-1} \neq 0$. Also, $\textbf{N} \in \SL{n}{\R}$ implies $\det \textbf{N} = 1$. Hence,
    \[
        \det \textbf{MNM}^{-1} = (\det \textbf{M})(\det \textbf{N})(\det \textbf{M})^{-1} = \det \textbf{N} = 1
    \]
    which means that $\textbf{MNM}^{-1}$ is in $\SL{n}{\R}$.

    Therefore $\SL{n}{\R} \lhd \GL{n}{\R}$.
\end{proof}

\subsection{A Consequence of the Fundamental Homomorphism Theorem}
We end this section with a consequence of the Fundamental Homomorphism Theorem on these groups of matrices. For brevity, let $\R^\times$ be the group of non-zero real numbers under multiplication.

\begin{proposition}
    $\GL{n}{\R} / \SL{n}{\R} \cong \R^\times$.
\end{proposition}
\begin{proof}
    Define the map $\phi: \GL{n}{\R} \to \R^\times$ where $\textbf{M} \mapsto \det\textbf{M}$.
	\begin{itemize}
	    \item \textbf{Homomorphism}: Take $\textbf{M}, \textbf{N} \in \GL{n}{\R}$. Then
	    \[
	        \phi(\textbf{MN}) = \det \textbf{MN} = \det(\textbf{M})\det(\textbf{N}) = \phi(\textbf{M})\phi(\textbf{N})
	    \]
	    which means that $\phi$ is a homomorphism.

	    \item \textbf{Image}: We prove that $\phi$ is surjective to show that $\im \phi = \R^\times$. Suppose $r \in \R^\times$. Then $r^{\frac1n} \in \R^\times$, and a matrix with $r^{\frac1n}$ on its main diagonal (written as $r^{\frac1n}\textbf{I}_n$ where $\textbf{I}_n$ is the identity matrix with $n$ rows and columns) is in $\GL{n}{\R}$. Note that $\det\left(r^{\frac1n}\textbf{I}_n\right) = \left(r^{\frac1n}\right)^n\det(\textbf{I}_n) = r$. Thus there is a pre-image of $r$ inside the codomain $\R^\times$, meaning that $\phi$ is surjective. Hence, $\im \phi = \R^\times$.

	    \item \textbf{Kernel}: Note that 1 is the identity in $\R^\times$. Thus
	    \begin{align*}
	        \ker\phi &= \{\textbf{M} \in \GL{n}{\R} \vert \phi(\textbf{M}) = 1\} \\
	        &= \{\textbf{M} \in \GL{n}{\R} \vert \det(\textbf{M}) = 1\}\\
	        &= \SL{n}{\R}.
	    \end{align*}
	\end{itemize}
	The Fundamental Homomorphism Theorem (\myref{thrm-isomorphism-1}) tells us that
	\[
	    \GL{n}{\R} / \SL{n}{\R} \cong \R^\times
	\]
	which proves the claim.
\end{proof}

\section{Automorphism Groups}
\subsection{Group of Automorphisms of \texorpdfstring{$G$}{G}}
We look at automorphisms, an important type of map in abstract algebra.
\begin{definition}
    An \textbf{automorphism}\index{automorphism} of a group $G$ is an isomorphism from a group $G$ to itself. That is, $\phi: G \to G$ is an automorphism if $\phi$ is an isomorphism.
\end{definition}
Clearly, from this definition, the identity function $\id$ is an automorphism.

We now look at the group of automorphisms of a group $G$.
\begin{definition}
    Let $G$ be a group. The \textbf{group of automorphisms of $G$}\index{automorphism group} is denoted $\Aut{G}$ and given by
    \[
        \Aut{G} = \{\phi: G \to G \vert \phi \text{ is an isomorphism}\}
    \]
    under function composition (denoted by $\circ$).
\end{definition}
We prove that $\Aut{G}$ is indeed a group.

\begin{proof}
    We look at the four group axioms.
    \begin{itemize}
        \item \textbf{Closure}: If $f, g \in \Aut{G}$, and $h = fg$, then $h: G \to G$ is a bijection. Furthermore $h$ is a homomorphism since for any $x, y \in G$ we have
        \begin{align*}
            h(xy) &= f(g(xy))\\
            &= f(g(x)g(y)) & (g \text{ is isomorphism})\\
            &= f(g(x))f(g(y)) & (f \text{ is isomorphism})\\
            &= h(x)h(y)
        \end{align*}
        so $h$ is an isomorphism, meaning $h = fg \in \Aut{G}$.

        \item \textbf{Associativity}: Function composition is associative.

        \item \textbf{Identity}: As mentioned above, the identity isomorphism $\id: G \to G, g \mapsto g$ is in $\Aut{G}$. By definition of $\id$, $f \circ \id = f$ and $\id \circ f = f$, so $\id$ is indeed the identity in $\Aut{G}$.

        \item \textbf{Inverse}: Suppose $f \in \Aut{G}$. Then $f$ is an isomorphism. By \myref{thrm-isomorphism-consequences}, $f^{-1}: G \to G$ is also an isomorphism, so $f^{-1} \in \Aut{G}$. Recall also that
        \[
            f\circ f^{-1} = \id \text{ and } f^{-1}\circ f = \id
        \]
        so $f^{-1}$ is indeed the identity of $f$.
    \end{itemize}
    Since the four group axioms are satisfied, thus $\Aut{G}$ is a group under function composition.
\end{proof}

\subsection{Group of Inner Automorphisms of \texorpdfstring{$G$}{G}}
We now look at inner automorphisms and its group.
\begin{definition}
    An \textbf{inner automorphism}\index{automorphism!inner} of a group $G$ is an automorphism $\iota: G \to G$ such that $\iota(x) = gxg^{-1}$ for some fixed $g \in G$.
\end{definition}
\begin{definition}
    Let $G$ be a group. The \textbf{group of inner automorphisms of $G$}\index{automorphism group!inner}, $\Inn{G}$, is given by
    \[
        \Inn{G} = \{\iota_g: G \to G \vert \iota_g(x) = gxg^{-1},\; g \in G\}
    \]
    under function composition (denoted by $\circ$).
\end{definition}

\begin{proposition}
    $\Inn{G} \leq \Aut{G}$.
\end{proposition}
\begin{proof}
    Clearly $\id \in \Inn{G}$ since $\id = \iota_e$ and 
    \[
        \id(x) = \iota_e(x) = exe^{-1} = x.
    \]
    for any $x \in G$. Hence $\Inn{G}$ is non-empty and, furthermore, $\Inn{G} \subseteq \Aut{G}$.

    Now take $\iota_x, \iota_y \in \Inn{G}$. Note that $\left(\iota_y\right)^{-1} = \iota_{y^{-1}}$ since
    \[
        \left(\iota_y\right)\left(\iota_{y^{-1}}\right)(g) = \left(\iota_y\right)\left(y^{-1}gy\right) = y\left(y^{-1}gy\right)y^{-1} = g
    \]
    which means $\left(\iota_y\right)\left(\iota_{y^{-1}}\right) = \id$. Therefore $\iota_x\left(\iota_y\right)^{-1} = \iota_{xy^{-1}}$ since for any $g \in G$ we have
    \begin{align*}
        \iota_x\left(\iota_y\right)^{-1}(g) &= \iota_x\iota_{y^{-1}}(g)\\
        &= \iota_x\left(y^{-1}gy\right)\\
        &= xy^{-1}gyx^{-1}\\
        &= \left(xy^{-1}\right) g \left(xy^{-1}\right)^{-1}\\
        &= \iota_{xy^{-1}}(g),
    \end{align*}
    which means that $\iota_x\left(\iota_y\right)^{-1} = \iota_{xy^{-1}} \in \Inn{G}$.

    Hence, by subgroup test, $\Inn{G} \leq \Aut{G}$.
\end{proof}

\begin{exercise}
    Let $G$ be a group. Prove that $\Inn{G} \unlhd \Aut{G}$.
\end{exercise}

\subsection{A Consequence of the Fundamental Homomorphism Theorem}
Before we can state the consequence, we revisit the idea of the center of a group as introduced in \myref{example-center-of-group}.
\begin{quote}
    The center of a group $G$ is the normal subgroup
    \[
        \CenterGrp{G} = \{z \in G \vert z = gzg^{-1} \text{ for all } g \in G\}.
    \]
\end{quote}

\begin{proposition}
    $G/\CenterGrp{G} \cong \Inn{G}$.
\end{proposition}
\begin{proof}
    We define $\phi: G \to \Inn{G}, g \mapsto \iota_g$ where $\iota_g(x) = gxg^{-1}$.
    \begin{itemize}
        \item \textbf{Homomorphism}: Let $g, h \in G$. Then for any $x \in G$,
        \begin{align*}
            (\phi(gh))(x) = \iota_{gh}(x) &= (gh)x(gh)^{-1}\\
            &= gh x h^{-1}g^{-1}\\
            &= g(hxh^{-1})g^{-1}\\
            &= g(\iota_h(x))g^{-1}\\
            &=\iota_g(\iota_h(x))\\
            &=(\iota_g\circ\iota_h)(x)\\
            &=(\phi(g)\phi(h))(x)
        \end{align*}
        which means that $\phi$ is a homomorphism.

        \item \textbf{Image}: We show that $\phi$ is surjective to prove that $\im \phi = \CenterGrp{G}$. Suppose $\iota_g \in \Inn{G}$. Clearly $\phi(g) = \iota_g$ which means that $\phi$ is surjective. Hence $\im\phi = \CenterGrp{G}$.

        \item \textbf{Kernel}: Note that $\ker\phi = \{g \in G \vert \phi(g) = \id\}$. So, if $g \in \ker\phi$ then $(\phi(g))(x) = \iota_g(x) = \id(x) = x$ for all $x \in G$. This means that $gxg^{-1} = x$, meaning $gx = xg$ for all $x \in G$, so $g \in \CenterGrp{G}$. Hence $\ker\phi = \CenterGrp{G}$.
    \end{itemize}
    Thus, one sees that by the Fundamental Homomorphism Theorem (\myref{thrm-isomorphism-1}),
    \[
        G/\ker\phi \cong \im\phi,
    \]
    implying that
    \[
        G/\CenterGrp{G} \cong \Inn{G}    
    \]
    which proves the result.
\end{proof}

\newpage

\section{Problems}
\begin{problem}
    By considering the group $\Z_{10101}$, find the smallest positive integers $a$ and $b$ such that
    \begin{partquestions}{\alph*}
        \item $1870a$ is a multiple of 10101.
        \item $3774b$ is a multiple of 10101.
    \end{partquestions}
\end{problem}

\begin{problem}
    Find the largest integer $n$ such that $\An{n}$ is an abelian group, proving your claim. Hence find all integers $k$ such that $\An{k}$ is cyclic.
\end{problem}

\begin{problem}
    Suppose $r$ is an odd primitive root modulo $p^k$, where $p$ is an odd prime and $k \geq 1$. Prove that $r$ is also a primitive root modulo $2p^k$.
\end{problem}

\begin{problem}
    Let $G = \Cn{n}$ with generator $g$.
    \begin{partquestions}{\roman*}
        \item Suppose $f_1: G \to G$ and $f_2: G \to G$ are homomorphisms. Prove that $f_1 = f_2$ if and only if $f_1(g) = f_2(g)$.
        
        \item Let $f: G \to G$ be a homomorphism. Explain why $f(g) = g^{m_f}$ for some $m_f \in \Z_n$.
        
        \item Show that the $m_f$ obtained in \textbf{(ii)} is unique to $f$.
        
        \item Suppose $f_1: G \to G$ and $f_2: G \to G$ are homomorphisms. Prove that
        \[
            m_{f_1\circ f_2} = m_{f_1} \otimes_n m_{f_2},
        \]
        where $\circ$ denotes function composition and $\otimes_n$ denotes multiplication modulo $n$.
        
        \item Let $f: G \to G$ be a homomorphism. Prove that $f$ is an automorphism if and only if $m_f$ has a multiplicative inverse modulo $n$. That is, there exists $k \in \Un{n}$ such that $m_fk \equiv 1 \pmod n$ if and only if $f$ is an automorphism.\newline
        (\textit{Hint: consider \myref{prop-multiplicative-inverse-exists-iff-coprime}, where we have $ab \equiv 1 \pmod m$ if and only if $\gcd(a, m) = 1$.})
        
        \item Hence, by considering the map $\phi: \Aut{G} \to \Un{n}$ where $\phi(f) = m_f$, prove that
        \[
            \Aut{G} \cong \Un{n}.
        \]
    \end{partquestions}
\end{problem}

\section{Group Actions}
\begin{questions}
    \item We prove the two group action axioms.
    \begin{itemize}
        \item \textbf{Identity}: $\alpha(e, x) = exe^{-1} = x$.
        \item \textbf{Compatibility}: Note
        \begin{align*}
            \alpha(g, \alpha(h, x)) &= \alpha(g, hxh^{-1})\\
            &= gh x h^{-1}g^{-1}\\
            &= (gh)x(gh)^{-1}\\
            &= \alpha(gh, x).
        \end{align*}
    \end{itemize}
    Therefore $\alpha$ is a group action of $G$ on $G$.

    \item Recall there are 6 elements in $\Sn{3}$: $\id$, $\begin{pmatrix}1 & 2 & 3\end{pmatrix}$, $\begin{pmatrix}1 & 3 & 2\end{pmatrix}$, $\begin{pmatrix}1 & 2\end{pmatrix}$, $\begin{pmatrix}1 & 3\end{pmatrix}$, and $\begin{pmatrix}2 & 3\end{pmatrix}$. Clearly the identity has all elements of $X$ as fixed points. It is also clear that $\begin{pmatrix}1 & 2 & 3\end{pmatrix}$ and $\begin{pmatrix}1 & 3 & 2\end{pmatrix}$ have no fixed points since they permute all elements. For the rest, the fixed points are the missing element from the cycle notation, i.e. $\begin{pmatrix}1 & 2\end{pmatrix}$ has fixed point 3, $\begin{pmatrix}1 & 3\end{pmatrix}$ has fixed point 2, and $\begin{pmatrix}2 & 3\end{pmatrix}$ has fixed point 1.

    \item For 1, it is $\{\id, \begin{pmatrix}2 & 3\end{pmatrix}\}$. For 2, it is $\{\id, \begin{pmatrix}1 & 3\end{pmatrix}\}$. For 3, it is $\{\id, \begin{pmatrix}1 & 2\end{pmatrix}\}$.

    \item We work from the statement forwards. Note that each of these statements are ``if and only if'' statements.
    \begin{align*}
	    g \cdot x = h \cdot x &\iff g^{-1} \cdot (g \cdot x) = g^{-1} \cdot (h \cdot x)\\
	    &\iff (g^{-1}g) \cdot x = (g^{-1}h) \cdot x\\
	    &\iff e \cdot x = (g^{-1}h) \cdot x\\
	    &\iff x = (g^{-1}h) \cdot x\\
	    &\iff (g^{-1}h) \cdot x = x\\
	    &\iff g^{-1}h \in \Stab{G}{x}
	\end{align*}

	\item \begin{partquestions}{\alph*}
		\item An orbit takes the form $\Orb{G}{x}$. Clearly $e \cdot x = x$ so $x \in \Orb{G}{x}$ and thus $\Orb{G}{x}$ is non-empty.
	    \item Let $x \in X$. Since $e \cdot x = x$, so $x \in \Orb{G}{x}$.
	    \item Suppose $x \in \Orb{G}{x_1} \cap \Orb{G}{x_2}$ (as their intersection is non-empty). Then there exists $g_1, g_2 \in G$ such that $g_1\cdot x_1 = x = g_2\cdot x_2$. Thus,
	    \begin{align*}
	        x_1 &= e \cdot x_1\\
	        &= (g_1^{-1}g_1)\cdot x_1\\
	        &= g_1^{-1} \cdot (g_1 \cdot x_1)\\
	        &= g_1^{-1} \cdot (g_2 \cdot x_2)\\
	        &= (g_1^{-1}g_2) \cdot x_2.
	    \end{align*}
	    Now suppose $y \in \Orb{G}{x_1}$. Then $y = g\cdot x_1$ for some $g \in G$. Hence,
	    \begin{align*}
	        y &= g\cdot x_1 \\
	        &= g \cdot \left((g_1^{-1}g_2) \cdot x_2\right)\\
	        &= (\underbrace{gg_1^{-1}g_2}_{\text{In } G})\cdot x_2\\
	        &\in \Orb{G}{x_2}
	    \end{align*}
	    which means any element in $\Orb{G}{x_1}$ is also in $\Orb{G}{x_2}$. Hence, $\Orb{G}{x_1}$ is a subset of $\Orb{G}{x_2}$. A similar argument can be used to show that $\Orb{G}{x_2}$ is a subset of $\Orb{G}{x_1}$. Hence $\Orb{G}{x_1} = \Orb{G}{x_2}$.
	\end{partquestions}

	\item We prove the forward direction first: suppose the action is transitive. Then there exists $x \in X$ such that $\Orb{G}{x} = X$. Now consider any other element $y \in X$. Since the action is transitive, this means that there exists a $\hat{g} \in G$ such that $\hat{g} \cdot x = y$. Note that $\Orb{G}{y} = \Orb{G}{\hat{g} \cdot x}$, and that $\Orb{G}{x} = \{g \cdot x \vert g \in G\}$. Hence,
	\[
        \Orb{G}{\hat{g} \cdot x} = \{g\cdot (\hat{g} \cdot x) \vert g \in G\} = \{(g\hat{g}) \cdot x \vert g \in G\}.
	\]
	Since $G$ is a group, $g\hat{g} \in G$. In particular, we may pick $g = g'\hat{g}^{-1}$ to obtain any arbitrary element $g' \in G$. Thus, this means that
	\[
        	\{(g\hat{g}) \cdot x \vert g \in G\} = \{g' \cdot x \vert g' \in G \} = \Orb{G}{x} = X.
	\]
	Hence, for any element $y \in X$, $\Orb{G}{y} = \Orb{G}{g \cdot x} = X$.

	The reverse direction is trivial: suppose $\Orb{G}{x} = X$ for all $x \in X$. Then certainly there exists an element $x \in X$ such that $\Orb{G}{x} = X$, meaning that the group action is transitive.

	\item \begin{partquestions}{\alph*}
	    \item Consider $x = n$. The orbit of $n$ is all of $X$. Consider the permutation $\sigma = \begin{pmatrix}k & n\end{pmatrix}$ where $1 \leq k \leq n$. Clearly $\sigma \in \Sn{n}$. Note that $\sigma \cdot n = \sigma(n) = k$. Thus, $\Orb{G}{n} = X$, meaning that the group action ``$\cdot$'' given by $g \cdot x \mapsto g(x)$ is transitive.
	    \item Note that $|X| = n$ and $|\Sn{n}| = n!$. By Orbit-Stabilizer theorem (\myref{thrm-orbit-stabilizer}), the stabilizer of $x$ by $G$ must have order $\frac{n!}{n} = (n-1)!$.
	\end{partquestions}

	\item By the Orbit-Stabilizer theorem (\myref{thrm-orbit-stabilizer}),
	\[
        |\Orb{G}{x}| = \frac{|G|}{|\Stab{G}{x}|} = [G : \Stab{G}{x}]	.
	\]
	Under the group action of conjugation, $\Orb{G}{x} = \Cl{x}$ and $\Stab{G}{x} = \Centralizer{G}{x}$. Hence, $|\Cl{x}| = [G : \Centralizer{G}{x}]$ as required.

	\item \begin{partquestions}{\alph*}
	    \item One sees that $\Z{D_3} = \{e\}$ based on the group table of $D_3$.
	    \item Recall that every element in $D_3$ can be expressed in the form $r^as^b$ where $a \in \{0, 1, 2\}$ and $b \in \{0, 1\}$. One finds that $\Cl{r} = \{r, r^2\}$ and $\Cl{s} = \{s, rs, rs^2\}$.
	    \item The class equation is $6 = 1 + 2 + 3$.
	\end{partquestions}

	\item By Cauchy's Theorem (\myref{thrm-cauchy}) there exists an element (say $x$) with order $p$. Consider $H = \langle x \rangle$. Note that $|H| = p$ and $H \leq G$. Hence we found a subgroup of $G$ of order $p$.
\end{questions}

\chapter{Sylow Theorems}
The Sylow theorems are a collection of theorems named after the Norwegian mathematician Peter Ludwig Sylow that give detailed information about the number of subgroups of fixed order that a given finite group contains. The Sylow theorems form a fundamental part of finite group theory and have very important applications in the classification of finite simple groups.

\section{First Sylow Theorem}
Before we can state the First Sylow Theorem, we introduce some terminology.

Recall that a group with order $p^k$ for some $k \geq 0$ is called a $p$-group.

\begin{definition}\label{definition-sylow-p-subgroup}
    Let $G$ be a finite group. Write the order of the group $G$ as $p^k m$ where $p$ is prime, $k \geq 0$, and $p \nmid m$. Then a subgroup $H$ with order $p^k$ is called a \textbf{Sylow $p$-subgroup}\index{Sylow $p$-subgroup} of $G$.
\end{definition}
We denote the set of all Sylow $p$-subgroups of the group $G$ for a given prime $p$ by $\Syl{p}{G}$.

\begin{exercise}
    Find the Sylow 2-subgroup of $\mathbb{Z}_{12}$.
\end{exercise}

We are now ready to state and prove the \textbf{First Sylow Theorem}.
\begin{theorem}[Sylow I]\label{thrm-sylow-1}\index{Sylow Theorem!First}
    Let $G$ be a finite group with order $p^k m$ where $p$ is prime, $k \geq 0$, and $p \nmid m$. Then $\Syl{p}{G} \neq \emptyset$.
\end{theorem}
\begin{remark}
    Equivalently, for a finite group $G$, for every prime factor $p$ of its order, there is a Sylow $p$-subgroup of $G$.
\end{remark}
\begin{proof}[Proof (see {\cite[pp.~1--3]{mann_2011}})]
    We use strong induction on the order of $G$.

    When the order of $G$ is 1, $|G| = p^0$ for any prime $p$, so $G$ is clearly a Sylow $p$-subgroup of itself.

    We assume now that the theorem holds for all groups of order strictly less than $n$, meaning that for any group $H$ with order $p^k m < n$, $\Syl{p}{H} \neq \emptyset$. We need to prove the case where the group has order $n$.
    
    Write $n = p^k m$ where $p$ is prime, $k \geq 1$, and $p \nmid m$. We split the argument into two separate cases.

    The first case is when order of the center of $G$ is a multiple of $p$, i.e. $|\CenterGrp{G}| = ap$ for some positive integer $a$. By Cauchy's Theorem (\myref{thrm-cauchy}), $\CenterGrp{G}$ contains a subgroup of order $p$, say $N$. We note that $N \unlhd G$ since for all $n \in N$ and $g \in G$,
    \begin{align*}
        gng^{-1} &= (gn)g^{-1}\\
        &= (ng)g^{-1} & (\text{since n} \in N \subseteq \CenterGrp{G})\\
        &= n(gg^{-1})\\
        &= n\\
        &\in N
    \end{align*}
    Hence $G/N$ is a quotient group, and it has order $\frac np = \frac{p^km}{p} = p^{k-1}m$. Since $p^{k-1}m < p^km = n$, thus by the inductive hypothesis, $G/N$ has a Sylow $p$-subgroup. Let that Sylow $p$-subgroup be $\bar{S}$, which means $\bar{S}$ has order $p^{k-1}$.

    We now construct $S = \{g \in G \vert gN \in \bar{S}\}$; we show that $S$ is a subgroup of $G$ under the group operation of $G$. Clearly $S \subseteq G$, and $e \in S$ as $eN = N \in \bar{S}$. Also for any $g_1, g_2 \in S$ (which means $g_1N, g_2N \in \bar{S}$), we see that
    \[
        (g_1g_2^{-1})N = (g_1N)(g_2N)^{-1} \in \bar{S},
    \]
    which implies that $g_1g_2^{-1} \in S$. By the subgroup test, $S \leq G$. Also, $N \leq S$ since for all $n \in N$, $nN = N \in \bar{S}$.

    We construct the homomorphism $\phi: S \to \bar{S}$ such that $g \mapsto gN$. Clearly any $gN \in \bar{S}$ has a pre-image of $g$ under $\phi$, so $\phi$ is surjective, meaning that $\im \phi = \bar{S}$. Furthermore,
    \begin{align*}
        \ker\phi &= \{s \in S \vert \phi(s) = N\}\\
        &= \{s \in S \vert sN = N \}\\
        &= \{s \in S \vert s \in N \}\\
        &= S \cap N\\
        &= N & (\text{since } N \subseteq S).
    \end{align*}
    Now we know that $S/N \cong \bar{S}$ by the Fundamental Homomorphism Theorem (\myref{thrm-isomorphism-1}), meaning that $p^{k-1} = |\bar{S}| = \frac{|S|}{|N|} = \frac{|S|}{p}$, which quickly implies that $|S| = p^k$. Hence $S$ is a subgroup of $G$ with order $p^k$, meaning that $S$ is a Sylow $p$-subgroup of $G$.

    We can now start on the second case where the order of the center of $G$ is not a multiple of $p$. Recall the class equation (\myref{thrm-class-equation})
    \[
        |G| = |\CenterGrp{G}| + \sum_{i=1}^l [G:\Centralizer{G}{g_i}]
    \]
    where $g_1, g_2, \dots, g_l$ are representatives of the $l$ distinct conjugacy classes with more than one element. We note that $p$ divides $|G|$. Since $p$ does not divide $|\CenterGrp{G}|$ in this case, thus
    \[
        \sum_{i=1}^l [G:\Centralizer{G}{g_i}]
    \]
    must also not divide $p$ in order for their sum to be divisible by $p$. Hence, there is at least one conjugacy class with more than one element such that $p$ does not divide $[G:\Centralizer{G}{g_i}]$. We note that $[G:\Centralizer{G}{g_i}] = \frac{|G|}{|\Centralizer{G}{g_i}|}$ by Lagrange's Theorem (\myref{thrm-lagrange}). Therefore, if $p$ does not divide $\frac{|G|}{|\Centralizer{G}{g_i}|} = \frac{p^k m}{|\Centralizer{G}{g_i}|}$, then $|\Centralizer{G}{g_i}| = p^ka$ for some positive integer $a$.

    We now argue that $a < m$ (recalling that $|G| = p^km$). Clearly if $a > m$ then $\frac{p^k m}{|\Centralizer{G}{g_i}|}$ is not an integer, so $a \leq m$. If instead $a = m$ then $|\Centralizer{G}{g_i}| = p^km = |G|$, and since $\Centralizer{G}{g_i} \leq G$ with them having equal orders, we conclude $G = \Centralizer{G}{g_i}$. This means that every element in $G$ commutes with $g_i$, which quickly implies $g_i \in \CenterGrp{G}$. But an element is in $\CenterGrp{G}$ if its conjugacy class has only one element, which is not the case. Hence $a \neq m$, meaning that $a < m$.

    In summary, $\Centralizer{G}{g_i} < G$ with $|\Centralizer{G}{g_i}| = p^ka < p^km = n$. Therefore we apply the induction hypothesis on $\Centralizer{G}{g_i}$ to say that $\Centralizer{G}{g_i}$ has a Sylow $p$-subgroup of order $p^k$. Clearly a subgroup of $\Centralizer{G}{g_i}$ has to also be a subgroup of $G$, meaning that $G$ has a Sylow $p$-subgroup.

    Therefore, any finite group $G$ with order written as $p^k m$ has a Sylow $p$-subgroup, meaning that $\Syl{p}{G} \neq \emptyset$.
\end{proof}

\begin{exercise}
    Find all primes $p$ such that $\Syl{p}{\Sn{5}}$ is non-empty.
\end{exercise}

\section{Conjugate Subgroup}
Before we look at the next Sylow theorem, we need to introduce two more things. The first is the \textbf{conjugate subgroup}.
\begin{definition}
    Let $G$ be a group, $H \leq G$, and $g \in G$. Then the \textbf{conjugate subgroup of $H$ by $g$}\index{conjugate subgroup} is
    \[
        gHg^{-1} = \{ghg^{-1} \vert h \in H\}
    \]
    under the group operation of $G$.
\end{definition}
We proved that $gHg^{-1}$ is a subgroup of $G$ in \myref{exercise-conjugate-subgroup}.

\begin{exercise}\label{exercise-conjugate-subgroup-isomorphic-to-subgroup}
    Let $G$ be a group and $H \leq G$. Prove that $gHg^{-1} \cong H$ for any $g \in G$.
\end{exercise}

We prove some results regarding the conjugate subgroup here.
\begin{proposition}\label{prop-power-of-conjugate-equals-conjugate-of-power}
    Let $G$ be a group, and let $a$ and $x$ be elements in $G$. Then $(axa^{-1})^n = ax^na^{-1}$ for all integers $n$.
\end{proposition}
\begin{proof}
    Trivially, when $n = 0$, $(axa^{-1})^0 = e = x^0 = ax^0a^{-1}$.
    
    We consider a proof by induction for positive integers $n$ and then prove the case when $n$ is negative.

    When $n = 1$, $(axa^{-1})^1 = axa^{-1}$ is given. Now assume $n = k$ holds true for some positive integer $k$, meaning that $(axa^{-1})^k = ax^ka^{-1}$. Then
    \begin{align*}
        (axa^{-1})^{k+1} &= (axa^{-1})(axa^{-1})^k\\
        &= (axa^{-1})(ax^ka^{-1}) & (\text{by induction hypothesis})\\
        &= axa^{-1}ax^ka^{-1}\\
        &= axx^ka^{-1}\\
        &= ax^{k+1}a^{-1}
    \end{align*}
    which completes the induction for positive integers $n$.

    Now suppose $n$ is a non-negative integer. Then
    \begin{align*}
        (axa^{-1})^{-n} &= ((axa^{-1})^n)^{-1}\\
        &= (ax^na^{-1})^{-1} & (\text{by above result})\\
        &= a(x^n)^{-1}a^{-1}\\
        &= ax^{-n}a^{-1}
    \end{align*}
    which completes the proof for all integers.
\end{proof}

\begin{proposition}\label{prop-order-of-conjugate-element-equals-order-of-element}
    Let $G$ be a group. Then for all $g, x \in G$, we have $|gxg^{-1}| = |x|$.
\end{proposition}
\begin{proof}
    The proposition trivially holds true for $g = e$ so we assume $g \neq e$.

    Suppose $|x| = n$, meaning $x^n = e$ and $x^k \neq e$ for all $1 \leq k < n$. Note we have
    \begin{align*}
        (gxg^{-1})^n  &= gx^ng^{-1} & (\myref{prop-power-of-conjugate-equals-conjugate-of-power})\\
        &= geg^{-1} & (\text{since } |x| = n)\\
        &= e
    \end{align*}
    which means that $|gxg^{-1}| \leq n = |x|$.

    Now let $k < n$. We show that if $x^k \neq e$ then $(gxg^{-1})^k \neq e$ using contrapositive proof. Suppose $(gxg^{-1})^k = e$. Thus $gx^kg^{-1} = e$ by \myref{prop-power-of-conjugate-equals-conjugate-of-power}. This implies that $gx^k = g$ which quickly means $x^k = e$. Therefore if $x^k \neq e$ then $(gxg^{-1})^k \neq e$, which shows that $|x| \leq |gxg^{-1}|$.

    As $|gxg^{-1}| \leq |x|$ and $|x| \leq |gxg^{-1}|$, so $|gxg^{-1}| = |x|$.
\end{proof}

\begin{exercise}
    Prove that for any group $G$, $|gh| = |hg|$ for any elements $g$ and $h$ in $G$.
\end{exercise}

\newpage

We note an important result with regards to the conjugate subgroup.
\begin{theorem}\label{thrm-unique-subgroup-of-given-order-is-normal}
    Let $G$ be a group. Suppose $H$ is the only subgroup of $G$ with a given order. Then $H \unlhd G$.
\end{theorem}
\begin{proof}
    Suppose $g \in G$. By \myref{exercise-conjugate-subgroup-isomorphic-to-subgroup} we know $gHg^{-1} \cong H$ which means that $|gHg^{-1}| = |H|$. Furthermore, by \myref{exercise-conjugate-subgroup} we know that $gHg^{-1} \leq G$. Since $H$ is the only subgroup of that order (by assumption), we conclude that $gHg^{-1} = H$, which quickly means that $H \unlhd G$ by definition of a normal subgroup.
\end{proof}

\section{The Normalizer}
We now look at the definition of the \textbf{normalizer}.
\begin{definition}
    Let $G$ be a group and $S$ be a subset of $G$. The \textbf{normalizer of $S$ in $G$}\index{normalizer} is given by
    \[
        \N{G}{S} = \{g \in G \vert gS = Sg \}.
    \]
    Equivalently, $\N{G}{S} = \{g \in G \vert gSg^{-1} = S \}$.
\end{definition}
\begin{exercise}\label{exercise-normalizer-is-subgroup-of-main-group}
    Let $G$ be a group and $S$ be a subset of $G$. Prove that $\N{G}{S} \leq G$.
\end{exercise}

We prove some properties of the normalizer here.
\begin{proposition}\label{prop-subgroup-is-a-normal-subgroup-of-normalizer}
    Let $G$ be a group, and $H \leq G$. Then $H \unlhd \N{G}{H}$.
\end{proposition}
\begin{proof}
    We first prove $H \leq \N{G}{H}$ and then prove normality.

    We know that both $H$ and $\N{G}{H}$ are subgroups of $G$, so both are groups. We just need to check that $H \subseteq \N{G}{H}$ to prove that $H \leq \N{G}{H}$.

    Consider any $h \in H$. We note that $hH = H$ and $Hh^{-1} = H$. Thus, if $h \in H$, then
    \[
        hHh^{-1} = h(Hh^{-1}) = hH = H
    \]
    which means that $h \in \N{G}{H}$ by definition of the normalizer. Hence any element in $H$ is also an element of $\N{G}{H}$, meaning $H \subseteq \N{G}{H}$. It follows then that $H \leq \N{G}{H}$ since $H$ is a group.

    Now we prove normality. Consider any $n \in \N{G}{H}$, which means that $nHn^{-1} = H$. This quickly implies that $H \unlhd \N{G}{H}$.
\end{proof}
\begin{remark}
    Combining the results from both \myref{exercise-normalizer-is-subgroup-of-main-group} and \myref{prop-subgroup-is-a-normal-subgroup-of-normalizer} yields $H \unlhd \N{G}{H} \leq G$.
\end{remark}

\begin{proposition}\label{prop-normalizer-of-subgroup-is-largest-subgroup-containing-that-subgroup-as-a-normal-subgroup}
    Let $G$ be a group, and $H \leq G$. Then $\N{G}{H}$ is the largest subgroup of $G$ containing $H$ as a normal subgroup.
\end{proposition}
\begin{remark}
    What we mean by ``largest'' here is that if there was another subgroup of $G$, say $K$, that permits $H \unlhd K$, then it must be the case that $K \subseteq \N{G}{H}$.
\end{remark}
\begin{proof}
    By \myref{prop-subgroup-is-a-normal-subgroup-of-normalizer} we know that $H \unlhd \N{G}{H}$. We just need to prove that any subgroup in which $H$ is normal inside it must be a subset of $\N{G}{H}$.

    Consider any subgroup $N \leq G$ such that $H \unlhd N \leq G$. Then for any $n \in N$ we have $nHn^{-1} = H$ by definition of normality, which immediately means that $n \in \N{G}{H}$ by definition of the normalizer of $H$ in $G$. Hence any element in $N$ also belongs in $\N{G}{H}$, meaning $N \subseteq \N{G}{H}$.

    This completes the proof that $\N{G}{H}$ is the largest subgroup of $G$ that contains $H$ as a normal subgroup.
\end{proof}

\begin{proposition}\label{prop-normalizer-of-sylow-p-subgroup}
    Let $G$ be a finite group. Let $P$ be a Sylow $p$-subgroup of $G$, and $Q$ be a $p$-subgroup of $\N{G}{P}$. Then $Q \subseteq P$. In particular, if $Q$ is a Sylow $p$-subgroup of $\N{G}{P}$, then $P = Q$.
\end{proposition}
\begin{proof}[Proof (see {\cite[Proposition 11.9]{humphreys_1996}})]
    For brevity, write the order of $G$ as $xp^n$ where $p \nmid x$, $|P| = p^n$, and $|Q| = p^m$.

    By the Diamond Isomorphism Theorem (\myref{thrm-isomorphism-2}), statement 3, we have $PQ \leq \N{G}{P}$ since $P \unlhd \N{G}{P}$ by \myref{prop-subgroup-is-a-normal-subgroup-of-normalizer}. As $\N{G}{P} \leq G$ thus $PQ \leq G$. In addition, by \myref{exercise-order-of-subgroup-product},
    \[
        |PQ| = \frac{|P||Q|}{|P \cap Q|} = p^{n+m-s}
    \]
    where we let $|P \cap Q| = p^s$. By Lagrange's Theorem (\myref{thrm-lagrange}) we know that $|G| = a|PQ|$ for some integer $a$, so $p^n = ap^{n+m-s}$ which implies $1 = ap^{m-s}$. Thus $p^{m-s} = \frac 1a \leq 1$ which means $m \leq s$.

    We note by \myref{problem-intersection-of-subgroups}, $P \cap Q \leq Q$. Therefore $|Q| = b|P\cap Q|$ for some positive integer $b$ by Lagrange's Theorem. Thus $p^m = bp^s$ which implies $p^{m-s} = b \geq 1$ so $m \geq s$.

    As $m \leq s$ and $m \geq s$, thus $m = s$ which means $|P \cap Q| = |Q|$, i.e. $P \cap Q = Q$. Therefore $Q \subseteq P$ which is the first part of the proposition proved.

    Now suppose $Q$ is, in particular, a Sylow $p$-subgroup of $\N{G}{P}$. Since $\N{G}{P} \leq G$ by \myref{exercise-normalizer-is-subgroup-of-main-group}, one concludes that $|G| = y|\N{G}{P}|$ for some positive integer $y$ by Lagrange's Theorem. Hence $xp^n = y|\N{G}{P}|$ which implies $|\N{G}{P}| = \frac{x}{y} p^n$. Set $z = \frac xy$ and note that $p \nmid z$. Therefore, if $Q$ is a Sylow $p$-subgroup, then $|Q| = p^n$. But since $|P| = p^n$ and $Q \subseteq P$, thus $P = Q$.
\end{proof}

\newpage

\section{Second Sylow Theorem}
We can now look at the \textbf{Second Sylow Theorem}.
\begin{theorem}[Sylow II]\label{thrm-sylow-2}\index{Sylow Theorem!Second}
    Let $G$ be a finite group and $p$ be a prime number. Suppose $H$ and $K$ are both Sylow $p$-subgroups of $G$. Then there exists an element $g \in G$ such that $gHg^{-1} = K$.
\end{theorem}
\begin{proof}[Proof (see {\cite[Theorem 11.10]{humphreys_1996}})]
    Define the set
    \[
        \mathcal{X} = \{gHg^{-1} \vert g \in G\}
    \]
    and denote an element from $\mathcal{X}$ by $X$.

    Let the Sylow $p$-subgroup $H$ act on $\mathcal{X}$ by conjugation, meaning $h \cdot X = hXh^{-1}$. We prove that this is a group action.
    
    \begin{itemize}
        \item \textbf{Closure}: We have to prove closure as it is not implicit in the action's definition.

        Let $X = gHg^{-1}$ for some $g \in G$. Then
        \[
            h\cdot X = hXh^{-1} = h(gHg^{-1})h^{-1} = (hg)H(hg)^{-1}.
        \]
        Since $h \in H \subseteq G$, then $hg \in G$, which thus means that $h \cdot X = (hg)H(hg)^{-1} \in \mathcal{X}$.
        \item \textbf{Identity}: $e \cdot X = eXe^{-1} = X$.
        \item \textbf{Compatibility}: Let $h_1, h_2 \in H$. Then
        \begin{align*}
            h_1 \cdot (h_2 \cdot X) &= h_1 \cdot (h_2Xh_2^{-1})\\
            &= h_1h_2Xh_2^{-1}h_1^{-1}\\
            &= (h_1h_2)X(h_1h_2)^{-1}\\
            &= (h_1h_2) \cdot X.
        \end{align*}
    \end{itemize}

    We consider orbits of this group action. In particular, we find element(s) with orbit(s) of only one element. Suppose $X \in \mathcal{X}$ has one element in its orbit, so for all $h \in H$ we have
    \[
        h\cdot X = hXh^{-1} = X.    
    \]
    Since $X = gHg^{-1}$ for some $g \in G$, thus
    \begin{align*}
        &h(gHg^{-1})h^{-1} = gHg^{-1}\\
        \iff&hgHg^{-1} = gHg^{-1}h\\
        \iff&hgH = gHg^{-1}hg\\
        \iff&(g^{-1}hg)H = H(g^{-1}hg)\\
        \iff&g^{-1}hg \in \N{G}{H}\\
        \iff&g^{-1}Hg \subseteq \N{G}{H}.
    \end{align*}
    Since $|g^{-1}hg| = |h|$ by \myref{prop-order-of-conjugate-element-equals-order-of-element}, and because $H$ is a $p$-subgroup, therefore $|g^{-1}hg| = p^r$ where $r$ is some positive integer. Note that $g^{-1}Hg$ is a (sub)group by \myref{exercise-conjugate-subgroup}. Therefore $g^{-1}Hg$ is a $p$-subgroup of $\N{G}{H}$.
    
    Furthermore $g^{-1}Hg \cong H$ by \myref{exercise-conjugate-subgroup-isomorphic-to-subgroup} so $|H| = |g^{-1}Hg|$. Therefore $g^{-1}Hg$ is a Sylow $p$-subgroup of $\N{G}{H}$. Hence, by \myref{prop-normalizer-of-sylow-p-subgroup}, we see $H = g^{-1}Hg$, implying $gHg^{-1} = H$. But $X = gHg^{-1}$, so $X = H$.
    
    Hence $H$ is the only element of $\mathcal{X}$ with $|\Orb{H}{X}| = 1$. Therefore for any $g \notin H$, $|\Orb{H}{gHg^{-1}}| > 1$. In fact, by Orbit-Stabilizer Theorem (\myref{thrm-orbit-stabilizer}), one sees
    \[
        |\Stab{H}{gHg^{-1}}| = \frac{|H|}{|\Orb{H}{gHg^{-1}}|}
    \]
    and since $|H|$ is a power of $p$ and $|\Stab{H}{gHg^{-1}}|$ is an integer, thus $|\Orb{H}{gHg^{-1}}|$ has to be a power of $p$, meaning $|\Orb{H}{gHg^{-1}}| \equiv 0 \pmod p$. Because distinct orbits partition the set $\mathcal{X}$ (\myref{exercise-distinct-orbits-partition-set}), thus $|\mathcal{X}| \equiv 1 \pmod p$.

    Now let the Sylow $p$-subgroup $K$ act on $\mathcal{X}$ by conjugation, meaning $k \star X = kXk^{-1}$. This is a group action as proven before, and the above conclusion means that there is at least one orbit of length 1. Hence there exists a $g \in G$ such that for all $k \in K$, we have $k(gHg^{-1})k^{-1} = gHg^{-1}$. Thus $g^{-1}kg \in \N{G}{H}$ which means $g^{-1}Kg \subseteq \N{G}{H}$. Therefore, since $|K| = |g^{-1}Kg|$ and by \myref{prop-normalizer-of-sylow-p-subgroup}, $g^{-1}Kg \subseteq H$, meaning $K \subseteq gHg^{-1}$. But because $|H| = |K|$ as they are both Sylow $p$-subgroups, therefore $K = gHg^{-1}$.
\end{proof}
\begin{remark}
    What part of this proof shows is that if $P$ is a Sylow $p$-subgroup but $Q$ is only a $p$-subgroup, then $Q \subseteq gPg^{-1}$ for some $g \in G$.
\end{remark}

We note one important corollary of the Second Sylow Theorem.
\begin{corollary}\label{corollary-sylow-subgroup-is-normal-if-it-is-unique}
    Let $G$ be a finite group and $P$ be a Sylow $p$-subgroup for some prime $p$. Then $P$ is a normal subgroup of $G$ if and only if $P$ is the only Sylow $p$-subgroup of $G$.
\end{corollary}
\begin{proof}
    The reverse direction is easy to prove. Since $P$ is the only Sylow $p$-subgroup of $G$, this means that $P$ is the only subgroup of order $p^k$. By \myref{thrm-unique-subgroup-of-given-order-is-normal}, this means that $P \unlhd G$.

    We work on the forward direction now and suppose $P$ is a normal subgroup of $G$. Let $\hat{P}$ be a normal Sylow $p$-subgroup. By the Second Sylow Theorem (\myref{thrm-sylow-2}), there exists $g \in G$ such that $g\hat{P}g^{-1} = P$. But since $\hat{P}$ is normal, thus $g\hat{P}g^{-1} = \hat{P}$ by definition of normality. Hence, $P = \hat{P}$, meaning that there is only one Sylow $p$-subgroup.
\end{proof}

\begin{exercise}
    Let $G$ be a finite group, $p$ be a prime number, and $H$ and $K$ be distinct Sylow $p$-subgroups of $G$. Prove that $H \cong K$.
\end{exercise}

\section{Third Sylow Theorem}
\begin{theorem}[Sylow III]\label{thrm-sylow-3}\index{Sylow Theorem!Third}
    Let $G$ be a finite group with order $p^k m$ where $p$ is prime, $k \geq 1$, and $p \nmid m$. Let $n_p$ denote the number of Sylow $p$-subgroups in $G$, i.e. $n_p = |\Syl{p}{G}|$. Then,
    \begin{enumerate}
        \item $n_p = [G:\N{G}{P}]$, where $P$ is a Sylow $p$-subgroup of $G$;
        \item $n_p$ divides $m$; and
        \item $n_p \equiv 1 \pmod p$.
    \end{enumerate}
\end{theorem}

\begin{proof}[Proof (see \cite{wielandt_1959})]
    We prove the three statements in order.
    \begin{enumerate}
        \item Let $G$ act on $\Syl{p}{G}$ by conjugation, meaning that for any $g \in G$ and $P \in \Syl{p}{G}$, $g\cdot P = gPg^{-1}$. By the Second Sylow Theorem (\myref{thrm-sylow-2}), all Sylow $p$-subgroups are conjugates of each other, so the orbit of any $P \in \Syl{p}{G}$ is the set of all Sylow $p$-subgroups, meaning $|\Orb{G}{P}| = |\Syl{p}{G}| = n_p$. Now consider $\Stab{G}{P}$; note $\Stab{G}{P} = \{g \in G \vert gPg^{-1} = P\} = \N{G}{P}$ by definition of the normalizer. Therefore
        \[
            n_p = |\Orb{G}{P}| = \frac{|G|}{|\Stab{G}{P}|} = \frac{|G|}{|\N{G}{P}|} = [G : \N{G}{P}],
        \]
        by the Orbit-Stabilizer Theorem (\myref{thrm-orbit-stabilizer}), proving the first statement.

        \item Let $P$ be a Sylow $p$-subgroup. We recall that $P \unlhd \N{G}{P} \leq G$. Note that by Lagrange's theorem (\myref{thrm-lagrange}) we know $|\N{G}{P}| = [\N{G}{P} : P]|P| = ap^k$ where $a \leq m$ (since $\N{G}{P} \leq G$). Furthermore
        \[
            mp^k = |G| = [G: \N{G}{P}]|\N{G}{P}| = n_p \times ap^k
        \]
        which means $m = an_p$. In other words, $n_p$ divides $m$.

        \item Let $H$ be a Sylow $p$-subgroup and let it act on $\Syl{p}{G}$ by conjugation, meaning that for any $h \in H$ and $P \in \Syl{p}{G}$, $h \cdot P = hPh^{-1}$. Let $\Omega$ denote the set of fixed points of $\Syl{p}{G}$ under this action.

        Suppose $Q \in \Omega$, which means that $hQh^{-1} = Q$ for all $h \in H$. Thus $H \subseteq \N{G}{Q}$ as $\N{G}{Q} = \{g \in G \vert gQg^{-1} = Q\}$. In fact, since $H$ is a Sylow $p$-subgroup, $H \leq \N{G}{Q}$. We note $Q \unlhd \N{G}{Q}$ by \myref{prop-subgroup-is-a-normal-subgroup-of-normalizer}. Hence $H$ and $Q$ are Sylow $p$-subgroups of $\N{G}{Q}$, which means there exists $n \in \N{G}{Q}$ such that $nQn^{-1} = H$ by the Second Sylow Theorem (\myref{thrm-sylow-2}). Furthermore $nQn^{-1} = Q$ since $Q \unlhd \N{G}{Q}$. Hence $Q = H$, which means that the only element in $\Omega$ is $H$.

        By a similar argument posed in the proof of the Second Sylow Theorem, for any $Q \in \Syl{p}{G}$ where $Q \neq H$ we have $|\Orb{H}{Q}| \equiv 0 \pmod p$. Note $|\Orb{H}{H}| = 1$. As distinct orbits partition $\Syl{p}{G}$ we must have $n_p = |\Syl{p}{G}| \equiv 1 \pmod p$.
    \end{enumerate}
    This completes the proof.
\end{proof}

\begin{example}
    We show that any group with order 4225 is abelian.

    Note that $4225 = 5^2 \times 13^2$. Let $G$ be a group of order 4225. By the Third Sylow Theorem (\myref{thrm-sylow-3}), we know that
    \begin{itemize}
        \item $n_5 \mid 13^2 = 169$ and $n_{13} \mid 5^2 = 25$, which means $n_5 \in \{1, 13, 169\}$ and $n_{13} \in \{1, 5, 25\}$; and
        \item $n_5 \equiv 1 \pmod 5$ and $n_{13} \equiv 1 \pmod{13}$, which means $n_5 \in \{1, 6, 11, \dots\}$ and $n_{13} \in \{1, 14, 27, \dots\}$.
    \end{itemize}
    Hence, $n_5 = 1$ and $n_{13} = 1$, meaning that there is only one Sylow 5-subgroup and one Sylow 13-subgroup.

    Let $P$ be the Sylow 5-subgroup and $Q$ be the Sylow 13-subgroup. By \myref{corollary-sylow-subgroup-is-normal-if-it-is-unique}, $P \lhd G$ and $Q \lhd G$, so $pq = qp$ for any $p \in P$ and $q \in Q$. Furthermore, by \myref{problem-intersection-of-coprime-subgroups}, $P \cap Q = \{e\}$. Finally, notice that
    \[
        |PQ| = \frac{|P||Q|}{|P\cap Q|} = |P||Q| = 5^2\times13^2 = |G|
    \]
    which means that $PQ$ and $G$ have the same number of elements. As $PQ \leq G$ by the Diamond Isomorphism Theorem (\myref{thrm-isomorphism-2}), statement 3, thus $G = PQ$. Hence, $G$ is the internal direct product of $P$ and $Q$. We note that
    \[
        G = PQ \cong P\times Q
    \]
    by direct product equivalence (\myref{thrm-direct-product-equivilance}). In addition, since $P$ and $Q$ are groups of prime-squared order, they are abelian (\myref{problem-group-of-order-prime-squared-is-abelian}), meaning that their external direct product $P\times Q$ is also abelian (\myref{problem-external-direct-product-of-abelian-groups-is-abelian}). Hence $G$ is abelian.
\end{example}

\begin{exercise}
    Let $G$ be a group of order 784, and let $P$ be a Sylow 7-subgroup that is \textbf{not} a normal subgroup of $G$. Find the order of $\N{G}{P}$.
\end{exercise}

\section{Testing Non-Simplicity Of Groups}
To end this chapter, we look at the idea of \textbf{simple groups} and describe ways to prove the non-simplicity of groups.
\begin{definition}
    Let $G$ be a group where $|G| \geq 2$. Then $G$ is \textbf{simple}\index{group!simple} if the only normal subgroups of $G$ are the trivial subgroup and $G$ itself. Equivalently, $G$ is simple if $G$ has no non-trivial proper normal subgroups.
\end{definition}

We may use part of the Third Sylow Theorem to create a test for non-simplicity.
\begin{theorem}[Sylow's Test]\index{Sylow's Test}
    Let $n$ be a non-prime integer and let $p$ be a prime divisor of $n$. If 1 is the only divisor of $n$ that is congruent 1 modulo $p$, then there does not exist a simple group of order $n$.
\end{theorem}
\begin{proof}
    We consider two cases, namely the case where $n=p^k$ where $k>1$ and the case where $n \neq p^k$.
    
    We first suppose $n = p^k$. Let $G$ be a group of order $n$. We consider two subcases.
    \begin{itemize}
        \item Suppose $G$ is abelian. By writing $p^k$ as $p \times p^{k-1}$, a corollary of Cauchy's Theorem (\myref{exercise-group-of-order-multiple-of-prime-has-subgroup-of-prime-order}) tells us that there exists a subgroup of order $p$. Since every subgroup of an abelian group is normal (\myref{prop-subgroup-of-abelian-group-is-normal}), thus there exists a non-trivial proper normal subgroup of $G$, meaning $G$ is non-simple.
        \item Now suppose $G$ is non-abelian. By \myref{example-group-with-prime-power-order-has-non-trivial-center}, $G$ has a non-trivial center. Furthermore, $G \neq \CenterGrp{G}$ because $G$ is non-abelian (\myref{problem-center-of-G}). Since the center is a normal subgroup of $G$, it is thus a non-trivial proper normal subgroup of $G$, meaning $G$ is non-simple.
    \end{itemize}
    Hence any group of order $p^k$ where $k > 1$ is non-simple.

    Now suppose $n$ is not a prime power. By the Third Sylow Theorem (\myref{thrm-sylow-3}), the number of Sylow $p$-subgroups, $n_p$, is congruent to 1 modulo $p$ and divides $n$. Since 1 is the only such number by our assumption, thus $n_p = 1$, meaning that there is only one Sylow $p$-subgroup. By \myref{corollary-sylow-subgroup-is-normal-if-it-is-unique} this means that that Sylow $p$-subgroup (which is proper) is normal, which thus means that any group of order $n$ is non-simple.
\end{proof}
\begin{example}
    Consider a group with order 15. Note $15 = 3 \times 5$, and consider $p = 5$. The divisors of 15 are 1, 3, 5, and 15, and clearly only 1 is congruent to 1 modulo 5. Hence by Sylow's Test we know that a group of order 15 cannot be simple.
\end{example}

For some groups of orders that do not satisfy the condition in Sylow's Test, we can still can prove that they cannot be simple.

\begin{example}
    We will show that a group of order 2552 is non-simple.

    We note first that $2552 = 2^3 \times 11 \times 29$. By the Third Sylow Theorem (\myref{thrm-sylow-3}), we know $n_p \mid m$ and $n_p \equiv 1 \pmod p$.

    The divisors of $m$ given the following primes are listed below.
    \begin{itemize}
        \item $p = 2$: $m = 319$ and so divisors are $\{1, 11, 29, 319\}$.
        \item $p = 11$: $m = 232$; divisors are $\{1, 2, 4, 8, 29, 58, 116, 232\}$.
        \item $p = 29$: $m = 88$; divisors are $\{1, 2, 4, 8, 11, 22, 44, 88\}$.
    \end{itemize}
    Thus, since $n_p \equiv 1 \pmod p$, we must have $n_2 \in \{1, 11, 29, 319\}$, $n_{11} \in \{1, 232\}$, and $n_{29} \in \{1, 88\}$.

    Seeking a contradiction, suppose that $n_2$, $n_{11}$, and $n_{29}$ are all not 1. Then $n_{11} = 232$ and $n_{29} = 88$. Recall that one element in a Sylow $p$-subgroup has order 1 (i.e., the identity). Thus, the number of elements of order 11 is $232 \times (11 - 1) = 2320$ and the number of elements of order 29 is $88 \times (29 - 1) = 2464$. Hence, the total number of elements in the group of order 2552 must be at least $2320 + 2464 = 4764$, a contradiction.

    Hence, we conclude that at least one of $n_2, n_{11}, n_{29}$ must be 1, meaning that there is a proper normal subgroup, which therefore means that any group of order 2552 is non-simple.
\end{example}

\begin{example}\label{example-using-kernel-to-show-non-simple}
    We show that any group of order 36 is non-simple by considering the kernel of a homomorphism.

    We note $36 = 2^2 \times 3^2$. Let $G$ be a group of order 36 and let $P$ be a Sylow 3-subgroup of the group of order 36. Let $G$ act on the set of cosets $G/P$ by left multiplication, meaning $g \cdot xP = (gx)P$. Then by \myref{thrm-group-action-definition-equivalence} this induces a homomorphism $\phi: G \to \Sn{4}$ since there are 4 cosets in $G/P$. We note $\phi(g) = \sigma_g$ where $\sigma_g(xP) = g\cdot xP = (gx)P$.

    We consider the kernel of $\phi$.
    \begin{align*}
        \ker\phi &= \{g \in G \vert \phi(g) = \id\}\\
        &= \{g \in G \vert \sigma_g = \id\}\\
        &= \{g \in G \vert \sigma_g(xP) = xP \text{ for all } x \in G\}\\
        &= \{g \in G \vert (gx)P = xP \text{ for all } x \in G\}\\
        &= \{g \in G \vert x^{-1}gx \in P \text{ for all } x \in G\} & (\text{Coset Equality})\\
        &= \{g \in G \vert g \in xPx^{-1} \text{ for all } x \in G\}\\
        &= \bigcap_{x \in G} xPx^{-1}.
    \end{align*}
    We note that $\ker\phi \neq \{e\}$ since that would imply that $\phi$ is injective (\myref{exercise-trivial-kernel-means-injective}), which would then mean $36 = |G| \leq |\Sn{4}| = 4! = 24$, a contradiction. We also note $\ker\phi \neq G$, otherwise
    \[
        36 = |G| = |\ker\phi| = \left|\bigcap_{x \in G} xPx^{-1}\right| \leq |xPx^{-1}| = |P| = 9,
    \]
    a contradiction. Hence $\ker\phi$ is a non-trivial proper subgroup of $G$. We note that $\ker\phi \lhd G$, so we have found a non-trivial proper normal subgroup of $G$, meaning that $G$ is non-simple.
\end{example}

\begin{exercise}
    Show that any group of order $130$ is non-simple.
\end{exercise}

\newpage

\section{Problems}
\begin{problem}
    Show that a group of order 200 has a normal Sylow 5-subgroup.
\end{problem}

\begin{problem}
    Show that any Sylow $p$-subgroup of a group of order 33 must be normal.
\end{problem}

\begin{problem}
    A perfect number is a positive integer that is equal to the sum of its positive divisors, excluding the number itself. All even perfect numbers are of the form $2^{p-1}\left(2^p-1\right)$ where both $p$ and $2^p-1$ are primes. Prove that any group with an even perfect number order is not simple.
\end{problem}

\begin{problem}\label{problem-group-of-order-pq-has-normal-subgroup-of-order-q}
    Let $p$ and $q$ be primes such that $p < q$. Let $G$ be a group of order $pq$.
    \begin{partquestions}{\roman*}
        \item Prove that there is only one subgroup $H$ of $G$ of order $q$. Deduce that $H \lhd G$.
        \item Prove also that if $q \not\equiv 1 \pmod p$ then $G$ is cyclic.
    \end{partquestions}
\end{problem}

\begin{problem}\label{problem-normal-subgroup-of-G-contains-all-sylow-p-subgroups}
    Let $G$ be a finite group, and write the order of $G$ as $p^km$ where $k \geq 0$ and $p \nmid m$. Let $N \lhd G$ such that $p$ does not divide the index of $N$ in $G$.
    \begin{partquestions}{\roman*}
        \item Prove that any Sylow $p$-subgroup of $N$ is also in $G$.
        \item Prove that any Sylow $p$-subgroup of $G$ is also in $N$.
    \end{partquestions}
    (That is, prove that $N$ contains all Sylow $p$-subgroups of $G$ and vice versa.)
\end{problem}

\begin{problem}
    Show that any group of order 3325 is abelian.
\end{problem}

\begin{problem}\label{problem-if-m!<|G|-then-G-is-simple}
    Let $G$ be a finite group such that $|G| = p^km$ where $k \geq 1$, $m > 1$, and $p \nmid m$. Prove that if $m! < |G|$ then $G$ is non-simple.
\end{problem}

\begin{problem}\label{problem-group-of-order-30-has-normal-subgroup-of-order-5}
    Prove that a group of order 30 has a normal subgroup of order 5.
\end{problem}

\begin{problem}\label{problem-group-of-order-pqr-is-non-simple}
    Let $p, q$, and $r$ be distinct primes such that $p < q < r$. Let $G$ be a group of order $pqr$. Prove that $G$ is non-simple.\newline
    (\textit{Hint: show that a normal subgroup of order $p$, $q$, or $r$ must exist.})
\end{problem}

\section{Composition Series}
\subsection*{Exercises}
\begin{questions}
    \item \begin{partquestions}{\roman*}
        \item One sees clearly that $\{0, 2\}$ is the only non-trivial proper normal subgroup of $G$, so the subnormal series of length 2 is $1 \lhd \{0, 2\} \lhd G$.
        \item There are 2 factor groups of the above subnormal series. The first is $\{0, 2\} / 1 \cong \Z_2$ and the second is
        \begin{align*}
            G / \{0, 2\} &= \{g \oplus_4 \{0, 2\} \vert g \in G\}\\
            &= \{\{0, 2\}, \{1, 3\}, \{2, 0\}, \{3, 1\}\}\\
            &= \{\{0, 2\}, \{1, 3\}\}\\
            &= \langle \{1, 3\} \rangle\\
            &\cong \Z_2.
        \end{align*}
        \item Since $1 \lhd G$ and $\{0, 2\} \lhd G$ thus the subnormal series in \textbf{(i)} is also a normal series of $G$.
    \end{partquestions}

    \item By Lagrange's theorem (\myref{thrm-lagrange}) we know that the order of a subgroup must divide the order of the group. Furthermore $\Z_{120}$ is abelian, so any subgroup of it is normal. Now the subgroup $N = \{0, 2, 4, \dots, 118\}$ has 60 elements which is the maximum possible guaranteed by Lagrange. Hence $N$ is the maximal normal subgroup of $\Z_{120}$, which has order 60.

    \item $\Cn{6}$ has
    \begin{align*}
        &1 \lhd \Cn{2} \lhd \Cn{6} \text{ and }\\
        &1 \lhd \Cn{3} \lhd \Cn{6}
    \end{align*}
    as composition series up to isomorphism. In both cases, their composition length is 2. Their respective composition factors are
    \begin{itemize}
        \item $\Cn{2} / 1 \cong \Cn{2}$ and $\Cn{6} / \Cn{2} \cong \Cn{3}$ by \myref{exercise-Zmn-mod-Zn-cong-Zn}; and
        \item $\Cn{3} / 1 \cong \Cn{3}$ and $\Cn{6} / \Cn{3} \cong \Cn{2}$ by \myref{exercise-Zmn-mod-Zn-cong-Zn},
    \end{itemize}
    up to isomorphism.

    \item Let the group in question be $G$. We know by Cauchy's theorem (\myref{thrm-cauchy}) and \myref{exercise-group-of-order-multiple-of-prime-has-subgroup-of-prime-order}, and by writing $p^2$ as $p \times p$, that $G$ has a subgroup of order $p$. Let this subgroup be $H$.

    Lagrange's theorem (\myref{thrm-lagrange}) tells us that the possible orders of the subgroups of $G$ are 1, $p$, and $p^2$. These subgroups are $\{e\}$, $H$, and $G$ respectively. Furthermore, by \myref{problem-group-of-order-prime-squared-is-abelian}, $G$ must be abelian, therefore its subgroups are all normal (\myref{prop-subgroup-of-abelian-group-is-normal}). Finally, a corollary of Lagrange's theorem (\myref{corollary-group-with-prime-order-subgroups}) says that the only subgroups of $H$ are the trivial group and the group itself. Hence, $G$ has only one composition series, namely $1 \lhd H \lhd G$.
\end{questions}

\subsection*{Problems}
\begin{questions}
    \item \begin{partquestions}{\roman*}
        \item We note $\mathrm{V}$ has order 4. As $4 = 2 \times 2$, thus we know that $\mathrm{V}$ has a subgroup of order 2 (which is cyclic) by Cauchy's theorem (\myref{thrm-cauchy}). Now $\mathrm{V}$ is abelian (\myref{problem-group-of-order-prime-squared-is-abelian}) which means that the subgroup of order 2 is normal (\myref{prop-subgroup-of-abelian-group-is-normal}). Finally, the only possible order for a non-trivial proper subgroup of $\mathrm{V}$ is 2 by Lagrange's theorem (\myref{thrm-lagrange}). Hence, the only composition series for $\mathrm{V}$ is $1 \lhd \Cn{2} \lhd \mathrm{V}$ up to isomorphism.\newline
        (Note that this analysis applies for any group of order 4.)

        \item Recall that $\mathrm{Q} = \langle \alpha, \beta \vert \alpha^4 = e, \alpha^2 = \beta^2, \text{ and } \beta\alpha = \alpha^3\beta \rangle$. From the solution of \myref{exercise-normal-subgroups-of-quarternion-group}, the maximal subgroups of $\mathrm{Q}$ are $G_1 = \langle \alpha \rangle$, $G_2 = \langle \beta \rangle$, and $G_3 = \langle \alpha\beta \rangle$ (by setting $\alpha = i$ and $\beta = j$). We note the following.
        \begin{itemize}
            \item $G_1 = \{e, \alpha, \alpha^2, \alpha^3\} \cong \Cn{4}$.
            \item $G_2 = \{e, \beta, \beta^2, \beta^3\} = \{e, \beta, \alpha^2, \alpha^2\beta\} \cong \mathrm{V}$ where $a = \alpha^2$ and $b = \beta$.
            \item $G_3 = \{e, \alpha\beta, (\alpha\beta)^2, (\alpha\beta)^3\} = \{e, \alpha\beta, \alpha^2, \alpha^3\beta\} \cong \mathrm{V}$ with $a = \alpha\beta$ and $b = \alpha^2$.
        \end{itemize}
        Also, note that $\Cn{2} \cong \langle \alpha^2 \rangle \lhd G_1$, $\Cn{2} \cong \langle \beta^2 \rangle \lhd G_2$, and $\Cn{2} \cong \langle (\alpha\beta)^2 \rangle \lhd G_3$. Hence, the two series up to isomorphism are
        \begin{align*}
            1 \lhd \Cn{2} \lhd \Cn{4} \lhd \mathrm{Q} & \text{ and }\\
            1 \lhd \Cn{2} \lhd \mathrm{V} \lhd \mathrm{Q}.
        \end{align*}

        \item By the Jordan-H\"older theorem (\myref{thrm-jordan-holder}), the composition factors are isomorphic to each other. We note
        \begin{itemize}
            \item $\Cn{2} / 1 \cong \Cn{2}$;
            \item $\Cn{4} / \Cn{2} \cong \Cn{2}$ by \myref{exercise-Zmn-mod-Zn-cong-Zn}; and
            \item $\mathrm{V} / \Cn{2} \cong (\Cn{2})^2 / \Cn{2} \cong \Cn{2}$ by \myref{problem-cartesian-product-of-group-by-group-isomorphic-to-group}.
        \end{itemize}
        The only unaccounted set of factors is $\mathrm{Q}/\mathrm{V}$ and $\mathrm{Q}/\Cn{4}$. So, either $\mathrm{Q}/\mathrm{V} \cong \Cn{2}$ and $\mathrm{Q}/\Cn{4} \cong \Cn{2}$, or $\mathrm{Q}/\mathrm{V} \cong \mathrm{Q}/\Cn{4}$. Hence $\mathrm{Q}/H \cong \mathrm{Q}/K$.
    \end{partquestions}

    \item We know that $\An{4} \lhd \Sn{4}$ by \myref{prop-An-normal-subgroup-of-Sn}. Note $\An{4}$ is a maximal normal subgroup since $|\An{4}| = \frac{4!}2 = 12$ by \myref{prop-order-of-An}, and a subgroup's order must divide the order of the group by Lagrange's theorem (\myref{thrm-lagrange}).

    Now applying that theorem on $\An{4}$, we see that the possible orders of a subgroup of $\An{4}$ are 6, 4, 3, 2, and 1. We claim that a subgroup of order 6 does not exist. Note that $\An{4}$ contains
    \begin{itemize}
        \item 1 element of order 1;
        \item 3 elements of order 2; and
        \item 8 elements of order 3.
    \end{itemize}
    If a subgroup of order 6 exists (say, $H$), then its index would be $\frac{12}{6} = 2$ by Lagrange, meaning $H$ contains all odd order elements (\myref{problem-subgroup-of-index-2}). However, there are $1 + 8 = 9$ odd order elements, meaning that $H$ has an order of at least 9, a contradiction. Hence a subgroup with $\An{4}$ of order 6 is impossible.

    Now we note that a subgroup of order $4 = 2^2$ exists by a corollary of the First Sylow Theorem (\myref{corollary-sylow-p-subgroup-exists}) as it is a Sylow 2-subgroup. The Third Sylow Theorem (\myref{thrm-sylow-3}) tells us how many Sylow 2-subgroups there are, in particular
    \begin{itemize}
        \item $n_2 \vert 3$, so $n_2$ is 1 or 3; and
        \item $n_2 \equiv 1 \pmod2$, so $n_2 \in \{1, 3, 5, \dots\}$.
    \end{itemize}
    Hence $n_2 = 1$ or $n_2 = 3$. Now if $n_2 = 3$, then the number of elements of order of 1, 2, or 4 is
    \[
        3 \times (4 - 1) + 1 = 10,
    \]
    where the 3 is $n_2$, the $4-1$ is the number of non-identity elements in each Sylow 2-subgroup, and the $+1$ is to add the identity element. However, as noted above, there are only 4 elements of order 1, 2, or 4, a contradiction. Hence $n_2 = 1$, meaning the Sylow 2-subgroup (which is a subgroup of order 4) is normal (\myref{corollary-sylow-subgroup-is-normal-if-it-is-unique}). Therefore the subgroup of order 4 is the maximal normal subgroup of $\An{4}$.

    We note that the subgroup of order 4 of $\An{4}$ is not $\Cn{4}$ (as this would imply that $\An{4}$ has an element of order 4, which it does not). Hence, from \myref{problem-smallest-nonabelian-group}, the subgroup of order 4 must be isomorphic to the Klein-4 group, $\mathrm{V}$.

    Note that a group of order 4 has a subgroup of order 2 by Cauchy's theorem (\myref{thrm-cauchy}). Clearly such a subgroup is cyclic (since 2 is prime), and has index $\frac42 = 2$, meaning that it is normal in the group of order 4. Furthermore the trivial group is always a subgroup of any group.

    Hence, the composition series for $\Sn{4}$, up to isomorphism, is
    \[
        1 \lhd \Cn{2} \lhd \mathrm{V} \lhd \An{4} \lhd \Sn{4}.
    \]
    \begin{remark}
        We list the actual subgroups that are isomorphic to the above terms in the composition series here.
        \begin{itemize}
            \item $\Cn{2}$: $\{e, \begin{pmatrix}1&2\end{pmatrix}\begin{pmatrix}3&4\end{pmatrix}\}$
            \item V: $\{e, \begin{pmatrix}1&2\end{pmatrix}\begin{pmatrix}3&4\end{pmatrix}, \begin{pmatrix}1&3\end{pmatrix}\begin{pmatrix}2&4\end{pmatrix}, \begin{pmatrix}1&4\end{pmatrix}\begin{pmatrix}2&3\end{pmatrix}\}$
            \item $\An{4}$ is an actual subgroup of $\Sn{4}$
        \end{itemize}
    \end{remark}
\end{questions}

\chapter{Simple Groups}
Simple groups can be thought of as the `building blocks' of all (finite) groups. The finite simple groups have been completely classified; each belongs to one of 18 infinite families, or is one of 26 sporadic groups that do not follow a specific pattern. We look at the classification of the families of these simple groups here.

\section{Cyclic Groups of Prime Order}
The first infinite family of simple groups we will look at is the family of Cyclic Groups of Prime Order\index{cyclic group!of prime order}.

\begin{lemma}\label{lemma-cyclic-group-simple-iff-order-is-prime}
    $\Cn{n}$ is simple if and only if $n$ is prime.
\end{lemma}
\begin{proof}
    We first prove the forward direction. Suppose $\Cn{n}$ is simple with generator $g$. Then the only normal subgroups of $\Cn{n}$ are the trivial group and the group itself. Seeking a contradiction, assume $n$ is not prime; write $n = ab$ where $a$ and $b$ are positive integers that are both smaller than $n$. Then clearly $\langle g^a\rangle$ is a proper subgroup of $\Cn{n}$. Now $\Cn{n}$ is abelian (\myref{prop-cyclic-group-is-abelian}) which means all subgroups are normal (\myref{prop-subgroup-of-abelian-group-is-normal}). Hence we have found a non-trivial proper normal subgroup of $\Cn{n}$, namely $\langle g^a \rangle$, contradicting that $\Cn{n}$ has no non-trivial proper normal subgroups. Therefore $n$ is prime.

    We now prove the reverse direction. Suppose $n$ is a prime. Then by a corollary of Lagrange's theorem (\myref{corollary-group-with-prime-order-subgroups}), $\Cn{n}$ has no non-trivial proper subgroups. So the only subgroup with order smaller than $n$ is the trivial group, $\{e\}$. Clearly $\Cn{n}$ is normal in itself, and the trivial group is always a normal subgroup. Hence, as the only normal subgroups of $\Cn{n}$ are the trivial group and itself, thus $\Cn{n}$ is simple. Therefore, if $n$ is prime then $\Cn{n}$ is simple.
\end{proof}

In fact, we have a much stronger result which we prove here.

\begin{theorem}\label{thrm-abelian-group-simple-iff-cylic-group-of-prime-order}
    An abelian group is simple if and only if it has prime order.
\end{theorem}
Note that we do not assume that the abelian group is finite; we will show that the group is finite in the proof below.
\begin{proof}
    The reverse direction follows immediately from \myref{lemma-cyclic-group-simple-iff-order-is-prime}, so we prove the forward direction only.

    Suppose $G$ is a simple abelian group; we show that $G$ is finite. Let $g$ a non-identity element of $G$. Then $H = \langle g \rangle$ is a subgroup of $G$. In fact, since $G$ is abelian, $H \unlhd G$ (\myref{prop-subgroup-of-abelian-group-is-normal}). As $G$ is simple, therefore $H = G$, meaning that $g$ is a generator of $G$. Now if $G$ is an infinite group, then one also sees that $\langle g^2 \rangle < G$ which implies $\langle g^2 \rangle \lhd G$, contradicting the fact that $G$ is simple. Hence $G$ is a finite abelian group with generator $g$, meaning $G$ is cyclic. Result follows directly from \myref{lemma-cyclic-group-simple-iff-order-is-prime}.
\end{proof}

From this, we conclude that the only family of simple abelian groups is the family of cyclic groups of prime order.

\section{Alternating Group With Degree $>4$}
The other family of simple groups that is relatively easy to find (and define) is the family of alternating groups with degree above 4\index{alternating group!of degree $>4$}. However, to prove this claim, we need several preliminary results.

\begin{theorem}\label{thrm-group-of-order-60-with->1-sylow-5-subgroup-is-simple}
    Let $G$ be a group of order 60. If $G$ has more than one Sylow 5-subgroup then $G$ is simple.
\end{theorem}
\begin{proof}[Proof (see {\cite[Proposition 4.21]{dummit_foote_2004}})]
    By way of contradiction assume $G$ is a group of order 60 with more than one Sylow 5-subgroup, but has a non-trivial proper normal subgroup $H$. Note $60 = 5 \times 12$, so by the Third Sylow Theorem (\myref{thrm-sylow-3}),
    \begin{itemize}
        \item $12 \vert n_5$, so $n_5 \in \{1, 2, 3, 4, 6, 12\}$; and
        \item $n_5 \equiv 1 \pmod 5$, so $n_5 \in \{1, 6, 11, 16, \dots\}$.
    \end{itemize}
    Therefore $n_5 = 6$ as $n_5 > 1$ (given), i.e. there are 6 Sylow 5-subgroups.

    We note by Lagrange's theorem (\myref{thrm-lagrange}) that the order of $H$ belongs in the set $\{1, 2, 3, 4, 5, 6, 10, 12, 15, 20, 30, 60\}$. As $H$ is a non-trivial proper subgroup of $G$, thus $|H| \neq 1$ and $|H| \neq 60$. That leaves 4 cases which we will deal with separately.
    \begin{enumerate}
        \item $|H| = 6$. Note $6 = 2 \times 3$, so \myref{problem-group-of-order-pq-has-normal-subgroup-of-order-q} tells us that there exists a $N \lhd H$ with $|N| = 3$. Note also $[G:H] = 10$ which is not a multiple of 3, so \myref{problem-normal-subgroup-of-G-contains-all-sylow-p-subgroups} tells us that all Sylow 3-subgroups of $G$ are in $H$. But $N \lhd H$ means that $N$ is the unique Sylow 3-subgroup of $H$ and $G$ (\myref{corollary-sylow-subgroup-is-normal-if-it-is-unique}), so $N \lhd G$ (by the same corollary). Proceed to case 3, using $N$ in place of $H$.

        \item $|H| = 12$. Note $12 = 2^2 \times 3$. Now \myref{exercise-group-of-order-12-has-normal-subgroup-of-3-or-4} (later) tells us that there exists a normal subgroup of $H$ with order 3 or 4. Call that subgroup $N$. If $|N| = 3$ then it is a Sylow 3-subgroup; if $|N| = 4 = 2^2$ it is a Sylow 2-subgroup. As $N \lhd H$, thus $N$ is the unique Sylow 2- or 3- subgroup (\myref{corollary-sylow-subgroup-is-normal-if-it-is-unique}). Since $H \lhd G$, thus $H$ contains all Sylow 2- and 3-subgroups of $G$ (\myref{problem-normal-subgroup-of-G-contains-all-sylow-p-subgroups}), meaning $G$ has only one Sylow 2-subgroup or one Sylow 3-subgroup (or both), in particular $N$. Hence, $N \lhd G$ since a Sylow $p$-subgroup is unique if and only if it is normal (\myref{corollary-sylow-subgroup-is-normal-if-it-is-unique}). Proceed with case 3, using $N$ instead of $H$.

        \item $|H| \in \{2, 3, 4\}$. Since $H \lhd G$, thus $G/H$ is a group. Note $|G/H| \in \{15, 20, 30\}$. We claim that each of these cases produces a new normal subgroup of $G/H$ (call it $\bar{P}$) with order 5. This is proven for the case where $|G/H| = 30$ in \myref{problem-group-of-order-30-has-normal-subgroup-of-order-5}; the other two cases are for \myref{exercise-group-of-order-15-or-20-has-normal-subgroup-of-order-5} (later).

        Now \myref{problem-subgroup-of-quotient-group-is-quotient-group} tells us that $\bar{P}$ has the form $K/H$ where $K < G$ and $H \subseteq K$. Since $\bar{P} = K/H \lhd G/H$, thus for any $g \in G$ and $kH \in \bar{P}$ we have
        \[
            (gH)(kH)(g^{-1}H) = (gkg^{-1})H \in K/H,
        \]
        which means $gkg^{-1} \in K$. Therefore $K \lhd G$ by definition of normality.

        Observe that this means that
        \[
            |K| = |K/H||H| = |\bar{P}||H| = 5|H|,
        \]
        meaning $K$ is a normal subgroup of $G$ with an order that is a multiple of 5. Proceed to case 4, using $K$ in place of $H$.

        \item $|H|$ is a multiple of 5, meaning $H$ has a Sylow 5-subgroup. Note that there are $5-1=4$ non-identity elements in each Sylow 5-subgroup; therefore
        \[
            |H| \geq n_5(5-1) = 24
        \]
        which means that $|H| = 30$. By \myref{problem-group-of-order-30-has-normal-subgroup-of-order-5} again, such a group has only a unique Sylow 5-subgroup.  Note $5 \nmid [G:H]$, so \myref{problem-normal-subgroup-of-G-contains-all-sylow-p-subgroups} implies all Sylow 5-subgroups of $G$ are in $H$. However, right at the start, we concluded that there are 6 Sylow 5-subgroups in $G$, so $H$ must have 6 Sylow 5-subgroups, a contradiction.
    \end{enumerate}
    Hence, $H$ does not exist, and so $G$ is simple.
\end{proof}

\begin{exercise}\label{exercise-group-of-order-12-has-normal-subgroup-of-3-or-4}
    Prove that a group of order 12 either has a normal subgroup of order 3, or a normal subgroup of order 4, or both.
\end{exercise}

\begin{exercise}\label{exercise-group-of-order-15-or-20-has-normal-subgroup-of-order-5}
    Prove that a group of each of the following orders has a normal subgroup of order 5.
    \begin{partquestions}{\alph*}
        \item 15
        \item 20
    \end{partquestions}
\end{exercise}

\begin{corollary}\label{corollary-A5-is-simple}
    The group $\An5$ is simple.
\end{corollary}
\begin{proof}
    \myref{exercise-A5-has-two-distinct-subgroups-of-order-5} (later) gives two distinct subgroups of order 5. Since $|\An{5}| = 60 = 2^2 \times 3 \times 5$, thus subgroups of order 5 are Sylow 5-subgroups. Therefore $\An5$ is simple by \myref{thrm-group-of-order-60-with->1-sylow-5-subgroup-is-simple}.
\end{proof}
\begin{exercise}\label{exercise-A5-has-two-distinct-subgroups-of-order-5}
    Consider the permutation $\sigma = \begin{pmatrix}1&3&2&4&5\end{pmatrix}$.
    \begin{partquestions}{\roman*}
        \item Explain why $\sigma \in \An{5}$.
        \item Find the order of the subgroup $\langle \sigma \rangle$.
        \item Find another subgroup of $\An{5}$ with order 5.
    \end{partquestions}
\end{exercise}

We also state and prove a fairly obvious proposition.

\begin{proposition}\label{prop-An-stabilizer-of-i-is-isomorphic-to-A(n-1)}
    Let the integer $n \geq 3$. Let the set $\{1, 2, 3, \dots, n\}$ be denoted by $\mathcal{N}_n$. Suppose $\An{n}$ acts on $\mathcal{N}_n$ naturally. Then $\Stab{\An{n}}{r} \cong \An{n-1}$.
\end{proposition}
\begin{proof}
    We note that elements of $\Stab{\An{n}}{r}$ are permutations that fix $r$, thereby permuting the $n - 1$ other elements. Therefore the elements of $\Stab{\An{n}}{r}$ are even permutations on $n - 1$ elements, i.e. $\Stab{\An{n}}{r} \cong \An{n-1}$.
\end{proof}

With these results, we are ready to prove the main result of this section.

\begin{theorem}\label{thrm-An-is-simple-for-n>=5}
    The group $\An{n}$ is simple if $n \geq 5$.
\end{theorem}
\begin{proof}[Proof (see {\cite[Theorem 4.24]{dummit_foote_2004}})]
    We induct on $n$. For brevity, let $\mathcal{N}_n = \{1, 2, 3, \dots, n\}$. The base case of $n = 5$ is covered by \myref{corollary-A5-is-simple}. Assume that $\An{k-1}$ is simple for some $k \geq 6$; we will prove that $\An{k}$ is also simple.

    Let $G = \An{k}$ and, seeking a contradiction, assume that $G$ has a non-trivial proper normal subgroup $H$. Let $G$ act on $\mathcal{N}_{k}$ naturally; thus we see that $\Stab{G}{i} \leq G$ with $\Stab{G}{i} \cong \An{k-1}$ (\myref{prop-An-stabilizer-of-i-is-isomorphic-to-A(n-1)}) for any $i \in \mathcal{N}_k$. Note $\An{k-1}$ is simple by the induction hypothesis, so $\Stab{G}{i}$ is simple for each $i \in \mathcal{N}_{k}$.

    Suppose first that there is some non-identity $\pi \in H$ such that $\pi(i) = i$ for some $i \in \mathcal{N}_{k}$. This means that $\pi$ fixes $i$; thus $\pi \in H \cap \Stab{G}{i}$. Note that since $H \lhd G$ and $\Stab{G}{i} \leq G$ thus $H \cap \Stab{G}{i} \lhd \Stab{G}{i}$ by the Second Isomorphism Theorem (\myref{thrm-isomorphism-2}), statement 4. But as $\Stab{G}{i}$ is simple (and non-trivial) we must have $H \cap \Stab{G}{i} = \Stab{G}{i}$. Therefore $\Stab{G}{i} \subseteq H$ which means $\Stab{G}{i} \leq H$. Now by \myref{exercise-conjugate-of-stabilizer} (later), for any $\sigma \in G$, we know that $\sigma\Stab{G}{i}\sigma^{-1} = \Stab{G}{\sigma(i)}$. Therefore we see
    \[
        \sigma\Stab{G}{i}\sigma^{-1} \leq \sigma H\sigma^{-1} = H
    \]
    since $H \lhd G$. Thus, for any $j \in \mathcal{N}_{k+1}$, there exists $\sigma \in G$ where $\sigma(i) = j$ such that
    \[
        \sigma\Stab{G}{i}\sigma^{-1} = \Stab{G}{j} \leq H.
    \]

    Note that any $\lambda \in G$ may be written as a product of an even number of transpositions (\myref{thrm-parity-of-permutation}), say $2t$ transpositions. Thus, we may write $\lambda = \lambda_1\lambda_2\cdots\lambda_t$ where each $\lambda_i$ is a product of two transpositions. Now as $k \geq 5$, each $\lambda_i$ (which could at most consist of two disjoint cycles of 4 elements) must fix at least one element in $\mathcal{N}_{k}$, say $j$. That is, $\lambda_i \in \Stab{G}{j}$ for some $j \in \mathcal{N}_{k}$. Since $\lambda_i \in \Stab{G}{j} \leq H$ for some $j \in \mathcal{N}_{k}$, thus $\lambda_i \in H$. Hence $\lambda = \lambda_1\lambda_2\cdots\lambda_t \in H$. Therefore, any element in $G$ is also in $H$, meaning $G \subseteq H$, contradicting the fact that $H \lhd G$.

    We conclude that for any $\pi \in H$, if $\pi \neq \id$ then $\pi(i) \neq i$ for all $i \in \mathcal{N}_k$. The contrapositive of this statement is that if $\pi(i) = i$ for some $i \in \mathcal{N}_k$ then $\pi = \id$. Now suppose $\pi_1, \pi_2 \in H$ and $\pi_1(i) = \pi_2(i)$ for some $i \in \mathcal{N}_{k}$. Then $\pi_2^{-1}\pi_1(i) = i$, which implies $\pi_2^{-1}\pi_1 = \id$. Hence $\pi_1 = \pi_2$. Therefore, if $\pi_1, \pi_2 \in H$ and $\pi_1(i) = \pi_2(i)$ for some $i \in \mathcal{N}_{k}$, then $\pi_1 = \pi_2$.

    Now suppose a non-identity $\pi_1 \in H$ exists such that the cycle decomposition of $\pi_1$ contains a cycle of length of at least 3, say
    \[
        \pi_1 = \begin{pmatrix}a_1&a_2&a_3&\cdots\end{pmatrix} \begin{pmatrix}b_1&b_2&\cdots\end{pmatrix}\cdots
    \]
    where $a_1$, $a_2$, $a_3$, $b_1$, $b_2$, etc. are distinct (which is possible since $k \geq 5$). We note an element $\sigma \in G$ exists such that $\sigma(a_1) = a_1$, $\sigma(a_2) = a_2$, but $\sigma(a_3) \neq a_3$ because $k \geq 4$ (for example, the permutation $\begin{pmatrix}a_3 & a_4\end{pmatrix}$). Then \myref{exercise-conjugation-of-permutation-by-another} (later) tells us that
    \[
        \sigma\pi_1\sigma^{-1} = \begin{pmatrix}a_1&a_2&\sigma(a_3)&\cdots\end{pmatrix} \begin{pmatrix}\sigma(b_1)&\sigma(b_2)&\cdots\end{pmatrix}\cdots.
    \]
    Set $\pi_2 = \sigma\pi_1\sigma^{-1}$, which is clearly distinct from $\pi_1$. Then we see $\pi_1(a_1) = \pi_2(a_1) = a_2$, contrary to the above observation that $\pi_1(i) = \pi_2(i)$ for any $i \in \mathcal{N}_k$ implies $\pi_1 = \pi_2$. Therefore only 2-cycles can appear in the cycle decomposition of non-identity elements of $H$.

    Let $\pi_1 \in H$ be a non-identity element, so that
    \[
        \pi_1 = \begin{pmatrix}a_1&a_2\end{pmatrix} \begin{pmatrix}a_3&a_4\end{pmatrix} \begin{pmatrix}a_5&a_6\end{pmatrix}\cdots
    \]
    where each $a_i$ is distinct (note that $k \geq 6$ is used above). Consider the permutation $\sigma = \begin{pmatrix}a_1&a_2\end{pmatrix} \begin{pmatrix}a_3&a_5\end{pmatrix}$, which is in $G$ since it is made up of 2 transpositions. Then \myref{exercise-conjugation-of-permutation-by-another} again gives
    \[
        \sigma\pi_1\sigma^{-1} = \begin{pmatrix}a_1&a_2\end{pmatrix} \begin{pmatrix}a_5&a_4\end{pmatrix} \begin{pmatrix}a_3&a_6\end{pmatrix}\cdots.
    \]
    Setting $\pi_2 = \sigma\pi_1\sigma^{-1}$ again gives two distinct permutations $\pi_1$ and $\pi_2$ where $\pi_1(a_1) = \pi_2(a_1) = a_2$, again contrary to the above observation.

    We conclude that such a non-trivial proper normal subgroup $H$ of $\An{k}$ cannot exist. Thus, $\An{k-1}$ being simple implies that $\An{k}$ is also simple.

    By mathematical induction, $\An{n}$ is simple for all $n \geq 5$.
\end{proof}

\begin{exercise}\label{exercise-conjugate-of-stabilizer}
    Let $S$ be a non-empty set and let $G \leq \Sym{S}$ act on $S$. Show that $\sigma\Stab{G}{x}\sigma^{-1} = \Stab{G}{\sigma(x)}$ for any $\sigma \in G$ and $x \in S$.
\end{exercise}

\begin{exercise}\label{exercise-conjugation-of-permutation-by-another}
    Let $\sigma, \pi \in \Sn{n}$. Suppose $\sigma$ has cycle decomposition
    \[
        \begin{pmatrix}a_1&a_2&\cdots&a_{k_1}\end{pmatrix} \begin{pmatrix}b_1&b_2&\cdots&b_{k_2}\end{pmatrix}\cdots,
    \]
    where $a_1, a_2, \dots, a_{k_1}, b_1, b_2, \dots, b_{k_2}, \dots$ are all distinct. Show that
    \[
        \pi\sigma\pi^{-1} = \begin{pmatrix}\pi(a_1)&\cdots&\pi(a_{k_1})\end{pmatrix} \begin{pmatrix}\pi(b_1)&\cdots&\pi(b_{k_2})\end{pmatrix}\cdots,
    \]
    that is, $\pi\sigma\pi^{-1}$ is obtained from $\sigma$ by replacing each entry $i$ by $\pi(i)$.
\end{exercise}

\begin{corollary}
    The group $\An{n}$ is simple for $n \geq 3$ and $n \neq 4$.
\end{corollary}
\begin{proof}
    We note $\An3$ has order $\frac{3!}{2} = 3$ which is prime, so $\An3 \cong \Cn3$ which is simple. Also $\An{n}$ is simple for $n \geq 5$ by \myref{thrm-An-is-simple-for-n>=5}.
\end{proof}

We note that $\An4$ is non-simple by the solution of \myref{problem-S4-composition-series}, in which we found that $\An4$ has a unique composition series of
\[
    1 \lhd \Cn2 \lhd \mathrm{V} \lhd \An4
\]
up to isomorphism.

\section{Groups of Lie Type}
We briefly mention groups of Lie type; we will not prove any significant results here.

Groups of Lie (pronounced ``lee'') type\index{groups of Lie type} usually refers to finite groups that are closely related to the group of rational points of a reductive linear algebraic group with values in a finite field. We will cover finite fields in part III. We briefly mention these groups here.

The list below, taken from \cite{wikipedia_list-of-simple-groups}, is a list of the families of simple groups of Lie type. In what follows, $n$ is a positive integer and $q$ is a positive power of a prime number $p$.
\begin{itemize}
    \item \term{Classical Chevalley groups}\index{Chevalley groups!classical}: there are 4 families of simple groups.
    \begin{itemize}
        \item $A_n(q)$, except for $A_1(2)$ and $A_1(3)$. There are several duplicates, which are
        \begin{itemize}
            \item $A_1(4) \cong A_1(5) \cong \An{5}$;
            \item $A_1(7) \cong A_2(2)$;
            \item $A_1(9) \cong \An{6}$; and
            \item $A_3(2) \cong \An{8}$.
        \end{itemize}
        We note that $\An{n}$ is not the same as $A_n(q)$. We distinguish between the alternating group of degree $n$ ($\An{n}$) and the groups of Lie type $A_n(q)$ by letting the latter be in italics and the former be in `normal' font.

        \item $B_n(q)$ for $n > 1$, except for $B_2(2)$. There are several duplicates, which are
        \begin{itemize}
            \item $B_n(2^m) \cong C_n(2^m)$; and
            \item $B_2(3) \cong {^2A_3(2^2)} = {^2A_3(4)}$, where ${^2A_3(4)}$ is a classical Steinberg group.
        \end{itemize}
        \item $C_n(q)$ for $n > 2$. The only duplicate is $C_n(2^m) \cong B_n(2^m)$ mentioned earlier.
        \item $D_n(q)$ for $n > 3$.
    \end{itemize}

    \item \term{Exceptional Chevalley groups}\index{Chevalley groups!exceptional}: there are 5 families of such groups.
    \begin{itemize}
        \item $E_6(q)$;
        \item $E_7(q)$;
        \item $E_8(q)$;
        \item $F_4(q)$; and
        \item $G_2(q)$, except for $G_2(2)$.
    \end{itemize}

    \item \term{Classical Steinberg groups}\index{Steinberg groups!classical}: there are 2 families of simple groups.
    \begin{itemize}
        \item ${^2A_n(q^2)}$ for $n > 1$, except for ${^2A_2(2^2)} = {^2A_2(4)}$. The only duplicate is ${^2A_3(2^2)} \cong B_2(3)$ mentioned earlier.
        \item ${^2D_n(q^2)}$ for $n > 3$.
    \end{itemize}

    \item \term{Exceptional Steinberg groups}\index{Steinberg groups!exceptional}: there are 2 families of simple groups.
    \begin{itemize}
        \item ${^2E_6(q^2)}$; and
        \item ${^3D_4(q^3)}$.
    \end{itemize}

    \item \term{Suzuki groups}\index{Suzuki groups}: there is 1 family of simple groups, which is ${^2B_2(q)}$ where $q = 2^{2n+1}$ and $n \geq 1$. Such a group has order $q^2(q^2+1)(q-1)$.

    \item \term{Ree groups}\index{Ree groups}: there are 2 families of simple groups.
    \begin{itemize}
        \item $^2F_4(q)$ where $q = 2^{2n+1}$ and $n \geq 1$. The order of such a group is $q^{12}(q^6+1)(q^4-1)(q^3+1)(q-1)$.
        \item $^2G_2(q)$ where $q = 3^{2n+1}$ and $n \geq 1$. The order of such a group is $q^3(q^3+1)(q-1)$.
    \end{itemize}
\end{itemize}

There is also the \term{Tits group}\index{Tits group}, $^2F_4(2)'$, with an order of $17,971,200$. It is the commutator subgroup of $^2F_4(2)$, which is a Lie group but not a simple group. The fact that $^2F_4(2)'$ is linked to Ree groups makes most authors consider it not a sporadic group (see below).

\section{The Sporadic Groups}
Along with the 18 infinite families of simple groups, there are also 26 sporadic simple groups\index{sporadic group} that do not fall within the families (27 if the Tits group is considered a sporadic group).

We first list four categories of sporadic groups.
\begin{table}[h]
    \centering
    \begin{tabular}{|l|l|l|}
        \hline
        \textbf{Name} & \textbf{Symbol} & \textbf{Order} \\ \hline
        \multirow{5}{*}{\term{Mathieu Groups}\index{Mathieu groups}} & $\mathrm{M}_{11}$ & 7,290 \\ \cline{2-3}
        & $\mathrm{M}_{12}$ & 95,040 \\ \cline{2-3}
        & $\mathrm{M}_{22}$ & 443,520 \\ \cline{2-3}
        & $\mathrm{M}_{23}$ & 10,200,960 \\ \cline{2-3}
        & $\mathrm{M}_{24}$ & 244,823,040 \\ \hline
        \multirow{4}{*}{\term{Janko Groups}\index{Janko groups}} & $\mathrm{J}_1$ & 175,560 \\ \cline{2-3}
        & $\mathrm{J}_2$ & 604,800 \\ \cline{2-3}
        & $\mathrm{J}_3$ & 50,232,960 \\ \cline{2-3}
        & $\mathrm{J}_4$ & 86,775,571,046,077,562,880 \\ \hline
        \multirow{3}{*}{\term{Conway Groups}\index{Conway groups}} & $\mathrm{Co}_3$ & 495,766,656,000 \\ \cline{2-3}
        & $\mathrm{Co}_2$ & 42,305,421,312,000 \\ \cline{2-3}
        & $\mathrm{Co}_1$ & 4,157,776,806,543,360,000 \\ \hline
        \multirow{3}{*}{\term{Fischer Groups}\index{Fischer groups}} & $\mathrm{Fi}_{22}$ & 64,561,751,654,400 \\ \cline{2-3}
        & $\mathrm{Fi}_{23}$ & 4,089,470,473,293,004,800 \\ \cline{2-3}
        & $\mathrm{Fi}_{24}$ & 1,255,205,709,190,661,721,292,800 \\ \hline
    \end{tabular}
\end{table}



More sporadic groups are listed below.
\begin{table}[h]
    \centering
    \begin{tabular}{|l|l|l|}
        \hline
        \textbf{Group}        & \textbf{Symbol} & \textbf{Order}  \\ \hline
        \term{Higman-Sims group}\index{Higman-Sims group}     & HS              & 44,352,000      \\ \hline
        \term{McLaughlin group}\index{McLaughlin group}      & McL             & 898,128,000     \\ \hline
        \term{Held group}\index{Held group}            & He              & 4,030,387,200   \\ \hline
        \term{Rudvalis group}\index{Rudvalis group}        & Ru              & 145,926,144,000 \\ \hline
        \term{Suzuki sporadic group}\index{Suzuki sporadic group} & Suz             & 448,345,497,600 \\ \hline
        \term{O'Nan group}\index{O'Nan group}           & $\mathrm{O'N}$  & 460,815,505,920 \\ \hline
        \term{Harada-Norton group}\index{Harada-Norton group}   & HN              & 273,030,912,000,000    \\ \hline
        \term{Lyons group}\index{Lyons group}           & Ly              & 51,765,179,004,000,000 \\ \hline
        \term{Thompson group}\index{Thompson group}        & Th              & 90,745,943,887,872,000 \\ \hline
    \end{tabular}
\end{table}

The remaining 2 sporadic groups are special in that they have extremely large order.
\begin{itemize}
    \item The \term{Baby Monster group}\index{Baby Monster group}, usually denoted $\mathrm{B}$, has order
    \begin{align*}
        &2^{41} \times 3^{13} \times 5^6 \times 7^2 \times 11 \times 13 \times 17 \times 19 \times 23 \times 31 \times 47\\
        &= 4,154,781,481,226,426,191,177,580,544,000,000.
    \end{align*}
    \item The \term{Monster group}\index{Monster group}, usually denoted $\mathrm{M}$, has order $2^{46} \times 3^{20} \times 5^9 \times 7^6 \times 11^{2} \times 13^3 \times 17 \times 19 \times 23 \times 29 \times 31 \times 41 \times 47 \times 59 \times 71$ which equals 808,017,424,794,512,875,886,459,904,961,710,757,005,754,368,\linebreak000,000,000. It is the largest sporadic group.
\end{itemize}

\section{The Classification Theorem of Finite Simple Groups}
One might rightly wonder what the importance of listing out all of these different types of simple groups are. It turns out that, amazingly, that these results provide a complete classification of what a finite simple group can really be. This is captured in the Classification Theorem of Finite Simple Groups\index{Classification Theorem of Finite Simple Groups}, which is sometimes called the enormous theorem\index{Enormous Theorem}.

\begin{theorem}[Classification Theorem]
    Every finite simple group is isomorphic to either
    \begin{itemize}
        \item a cyclic group of prime order;
        \item an alternating group with degree of at least 5;
        \item a group in the 16 infinite families of groups of Lie type, or the Tits group; or
        \item one of 26 sporadic groups.
    \end{itemize}
\end{theorem}

The proof of this theorem required tens of thousands of pages in hundreds of articles, written by a large number of authors that were  published mostly between 1955 to 2004. The longest paper, and the last paper needed to fill in the gap for quasithin groups, was published in 2004 by Aschbacher and Smith and spanned in a 1221 pages. But after all that work, mathematicians had a complete classification of all finite simple groups.


%=========================================
\setpartpreamble[u][\textwidth]{
    \quoteattr{
        [Some] of the major discoveries in ring theory have helped shape the course of development of modern abstract algebra... A course in ring theory is an indispensable part of the education of any fledgling algebraist.
    }
    {
        Tsit-Yuen Lam, 2001
    }
    {
        \cite{lam_2001}
    }

    The investigation into the properties of rings is primarily motivated by rings’ connections to other fields in mathematics, such as algebraic geometry, algebraic number theory and commutative algebra. The discovery of results relating to rings reveals deeper connections in related algebraic structures, such as fields. Results relating to rings are often technical but are necessary as they provide the underpinnings of more useful results in other areas.
    
    %TODO: Add
}
\part{Ring Theory}
\chapter{Introduction to Rings}
In part I, we looked exclusively at groups and their operations. We discussed how groups are a generalisation of symmetry and looked at results related to groups. In this part, we look at rings.

\section{Extending Number Systems}
The entire field fo ring theory was kickstarted to generalize number systems.

We are intimately familiar with number systems in our daily lives. The simplest number system that was developed involved only the positive integers (which we denote by $\mathbb{N}$). What properties do we have in the positive integers?

Let's start with addition. Adding two positive numbers together still results in a positive integer. For example, we see $2 + 3 = 5$, $3 + 4 = 7$, $58 + 95 = 153$, and so on. It is impossible to add two positive integers and end up with a non-positive integer. This means that the set of positive integers is closed under addition. Furthermore, we want addition to be associative. Take for example the expression $1 + 2 + 3$. We want $1 + (2 + 3) = (1 + 2) + 3 = 6$, as we really don't care about the order of addition within brackets. Finally addition is commutative. As an example, the sum $4 + 6$ is equal to $6 + 4$, and we don't care what order we perform the addition in.

Let's now turn our attention to multiplication. One sees that multiplying two positive integers together still results in a positive integer, and that we don't care about the order of evaluating multiplication of three positive integers.

Finally, we see that multiplication distributes over addition. To see what this means, consider the expression $5(6+7)$. How could we evaluate this? Well, we could sum $6+7 = 13$ first, then multiply it by 5 to get $5\times13 = 65$. Alternatively, we could notice that $5(6+7) = 5 \times 6 + 5 \times 7$ and evaluate it that way. This is called left distribution; right distribution is defined similarly.

In summary, we may call the set of positive integers, along with addition and multiplication, a \textbf{semiring} (see \cite{mathworld_semiring-definition}).

\newpage

\begin{definition}
    A set $S$ together with two operations $+$ and $\cdot$ is called a \textbf{semiring}\index{semiring} if it satisfies the following properties.
    \begin{itemize}
        \item \textbf{Additive Closure}\index{axiom!semiring!additive closure}: For any two elements $a$ and $b$ in $S$, $a + b$ is also in $S$.
        \item \textbf{Additive Associativity}\index{axiom!semiring!additive associativity}: For any three elements $a$, $b$, and $c$ in $S$, $a+(b+c) = (a+b)+c$.
        \item \textbf{Additive Commutativity}\index{axiom!semiring!additive commutativity}: For any two elements $a$ and $b$ in $S$, $a + b = b + a$.
        \item \textbf{Multiplicative Closure}\index{axiom!semiring!multiplicative closure}: For any two elements $a$ and $b$ in $S$, $a \cdot b$ is also in $S$.
        \item \textbf{Multiplicative Associativity}\index{axiom!semiring!multiplicative associativity}: For any three elements $a$, $b$, and $c$ in $S$, $a\cdot(b\cdot c) = (a\cdot b)\cdot c$.
        \item \textbf{Left and Right Distributivity}\index{axiom!semiring!distributivity}: For any three elements $a$, $b$, and $c$ in $S$, $a\cdot(b + c) = (a \cdot b) + (a \cdot c)$ and $(a + b) \cdot c = (a \cdot c) + (b \cdot c)$.
    \end{itemize}
\end{definition}

However, one may notice that $\mathbb{N}$ isn't particularly fun to work in. We only have these trivial properties to work with, and these do not give us enough to generate results about the positive integers. So ancient civilizations decided to include the notion of a ``additive identity'', which is called zero (0). Now, adding zero to any positive integer results in the positive integer itself.

We also notice that \textit{multiplying} any number by 0 results in 0: this is obvious, especially for us who work with integers. This set of non-negative integers (which, for brevity, we denote by $N$ in this chapter only) with addition and multiplication is called a \textbf{rig} (see \cite{proofwiki_rig-definition}).
\begin{definition}
    A set $S$ together with two operations $+$ and $\cdot$ is called a \textbf{rig}\index{rig} if it is a semiring and satisfies these additional properties.
    \begin{itemize}
        \item \textbf{Additive Identity}\index{axiom!rig!additive identity}: There is an element $0_S$ in $S$ such that for any $x$ in $S$ we have $0_S + x = x + 0_S = x$.
        \item \textbf{Multiplication by Zero}\index{axiom!rig!multiplication by zero}: For any element $x$ in $S$ we have $0_S \cdot x = x \cdot 0_S = 0_S$.
    \end{itemize}
\end{definition}
\begin{remark}
    A rig is a ri\textit{\textbf{n}}g without \textit{\textbf{n}}egative elements.
\end{remark}

Now, for any element $n$ in $N$, it would be nice to have a `corresponding' element in that sums to the additive identity 0. Clearly $N$ doesn't have such an element, but the integers ($\Z$) does. For example, the integer 3 has a `corresponding' element -3 that results in a sum of 0, i.e. $3 + (-3) = 0$. In general, for any integer $n$, summing $n$ with $-n$ results in 0. This `corresponding' element is called the additive inverse of $n$. With additive inverses, we finally have a construction for a \textbf{ring}. Note that we follow \cite[p.~223]{dummit_foote_2004}, \cite[p.~115, Definition 1.1]{hungerford_1980}, and \cite{proofwiki_ring-definition} for the definition of a ring.
\begin{definition}
    A set $S$ together with two operations $+$ and $\cdot$ is called a \textbf{ring}\index{ring} if it is a rig and satisfies the \textbf{Additive Inverse} property, where for every element $x$ in $S$, there exists an element $-x$ in $S$ such that $x + (-x) = (-x) + x = 0_S$.
\end{definition}

\section{Rings as Algebraic Structures}
With an intuition of what rings are, we more concretely define what a ring is using algebraic structures.

In the previous part, we explored the concept of groups. We weaken the conditions required for that structure to form a \textbf{semigroup}.
\begin{definition}
    A \textbf{semigroup}\index{semigroup} is a set $S$ together with an operation $\ast$ satisfying the \textbf{semigroup axioms}\index{axiom!semigroup}.
    \begin{itemize}
        \item \textbf{Closure}\index{axiom!semigroup!closure}: For any two elements $a$ and $b$ in $S$, the element $a\ast b$ is also in $S$.
        \item \textbf{Associativity}\index{axiom!semigroup!associativity}: For any $a$, $b$, and $c$ in $S$, we have $a \ast (b \ast c) = (a \ast b) \ast c$.
    \end{itemize}
\end{definition}
\begin{example}
    Consider the set $S = \{1, 2, 3, 4\}$ with the operation $\ast$ such that $(S, \ast)$ has the Cayley table as shown below.
    \begin{table}[h]
        \centering
        \begin{tabular}{|l|l|l|l|l|}
            \hline
            $\ast$     & \textbf{1} & \textbf{2} & \textbf{3} & \textbf{4} \\ \hline
            \textbf{1} & 1          & 1          & 1          & 1          \\ \hline
            \textbf{2} & 2          & 2          & 2          & 2          \\ \hline
            \textbf{3} & 3          & 3          & 3          & 3          \\ \hline
            \textbf{4} & 4          & 4          & 4          & 4          \\ \hline
        \end{tabular}
    \end{table}
    
    One sees that $(S, \ast)$ is closed under $\ast$. In addition, $\ast$ is associative. Hence $(S, \ast)$ is a semigroup.
\end{example}

With all that set up, we are ready to rigorously define what a ring is.
\begin{definition}
    A \textbf{ring}\index{ring} is a set $R$ with two binary operations $+$ and $\cdot$ satisfying the following axioms.
    \begin{itemize}
        \item \textbf{Addition-Abelian}\index{axiom!ring!addition-abelian}: $(R, +)$ is an abelian group.
        \item \textbf{Multiplication-Semigroup}\index{axiom!ring!multiplication-semigroup}: $(R, \cdot)$ is a semigroup.
        \item \textbf{Distributive}\index{axiom!ring!distributive}: $\cdot$ is distributive over $+$. That is,
        \begin{itemize}
            \item $a \cdot (b + c) = (a \cdot b) + (b \cdot c)$; and
            \item $(a + b) \cdot c = (a \cdot c) + (b \cdot c)$.
        \end{itemize}
    \end{itemize}
    We denote such a ring by $(R, +, \cdot)$.
\end{definition}
\begin{remark}
    We do not need to define \textbf{Multiplication by Zero} as an axiom here because it is implied by the three ring axioms above. We prove this in the next chapter.
\end{remark}

We note two important types of rings here.
\begin{definition}
    A \textbf{ring with identity}\index{ring!with identity} is a ring $(R, +, \cdot)$ with an element $1_R$ such that for any $x \in R$ we have $1_R \cdot x = x \cdot 1_R = x$.
\end{definition}
\begin{remark}
    Other authors (e.g. \cite[p.~136]{cohn_1982}, \cite[pp.~145--146]{clark_1984}) define a ring as a ring with identity.
\end{remark}
\begin{example}
    We introduced the integers ($\mathbb{Z}$) in the previous section. One sees clearly that 1 is the multiplicative identity in the integers since $1n = n$ for any integer $n$, os $\mathbb{Z}$ is a ring with identity.
\end{example}

\begin{definition}
    A ring where $a \cdot b = b \cdot a$ for all $a$ and $b$ in $R$ is called a \textbf{commutative ring}\index{ring!commutative}.
\end{definition}
\begin{example}
    Considering the integers again, we see that $mn = nm$ for any two integers $m$ and $n$. Thus $\mathbb{Z}$ is a commutative ring.
\end{example}

We end this chapter by introducing the \textbf{trivial ring}.
\begin{definition}
    The \textbf{trivial ring}\index{trivial ring} (or \textbf{zero ring}\index{zero ring}), denoted $\textbf{0}$, is the ring $(\{0\}, +, \cdot)$ where
    \[
        0 + 0 = 0 \text{ and } 0 \cdot 0 = 0.    
    \]
\end{definition}
\begin{exercise}
    Prove that the trivial ring is a commutative ring with identity.
\end{exercise}

\chapter{Basics of Rings}
With an intuition and definition of rings out of the way, we are now ready to tackle the basics in this chapter.

\section{Obvious Rings}
Before we introduce some examples of rings, we make some remarks for the notation that is used in Ring Theory.
\begin{itemize}
    \item The multiplication symbol $\cdot$ is usually omitted, so $x \cdot y$ is written as $xy$.
    \item The additive identity of $R$ will always be denoted by 0 and the multiplicative identity of $R$ (if it exists) will always be denoted by 1.
    \item The additive inverse of the element $x$ will be denoted by $-x$ and the multiplicative inverse of $x$ (if it exists) will be denoted by $x^{-1}$.
    \item $n$ applications of $+$ on an element $x$ will be denoted $nx$ (and will be denoted $-nx$ if the element is $-x$), while $n$ applications of $\cdot$ on an element $x$ will be denoted $x^n$ (and will be denoted $x^{-n}$ if the element is $x^{-1}$ and if it exists).
\end{itemize}

Let's look at some examples of rings.
\begin{definition}
    The \textbf{ring of integers}\index{ring!of integers} is the set $\Z$ together with integer addition and multiplication.
\end{definition}
\begin{remark}
    We denote the ring of integers by $\Z$.
\end{remark}
\begin{proposition}
    $\Z$ is a commutative ring with identity.
\end{proposition}
\begin{proof}
    \myref{exercise-ring-of-integers-is-a-ring} (later) shows that $\Z$ is a ring. In addition, multiplication is commutative (\myref{axiom-multiplication-is-commutative}), and 1 is the multiplicative identity. Thus $\Z$ is a commutative ring with identity.
\end{proof}

\begin{definition}
    Let the integer $n > 2$. The \textbf{ring of integers modulo $n$}\index{ring!of integers!modulo $n$} is $(\Z_n, \oplus_n, \otimes_n)$, where $\oplus_n$ and $\otimes_n$ denote addition and multiplication modulo $n$ respectively.
\end{definition}
\begin{remark}
    We denote the ring of integers modulo $n$ by $\Z_n$.
\end{remark}
\begin{proposition}
    $\Z_n$ is a commutative ring with identity.
\end{proposition}
\begin{proof}
    We first prove the ring axioms before showing that it is commutative with a multiplicative identity.
    \begin{itemize}
        \item \textbf{Addition-Abelian}: We know $(\Z_n, \oplus_n)$ is an abelian group by \myref{prop-Zn-is-abelian-group}
        \item \textbf{Multiplication-Semigroup}: We can see that $(\Z_n, \otimes_n)$ is a semigroup as
        \begin{itemize}
            \item $\Z_n$ is closed under $\otimes_n$ because $a \otimes_n b \in \{0, 1, 2, \dots, n-1\} = \Z_n$; and
            \item multiplication is associative (\myref{axiom-multiplication-is-associative}), so multiplication modulo $n$ is associative.
        \end{itemize}
        \item \textbf{Distributive}: Since multiplication distributes over addition (\myref{axiom-distributivity}), thus multiplication modulo $n$ (i.e. $\otimes_n$) distributes over addition modulo $n$ (i.e. $\oplus_n$).
    \end{itemize}
    Hence $(\Z_n, \oplus_n, \otimes_n)$ is a ring.
    
    Furthermore, multiplication is commutative (\myref{axiom-multiplication-is-commutative}), so $\otimes_n$ is commutative. Also $\otimes_n$ has an identity of 1. Therefore $(\Z_n, \oplus_n, \otimes_n)$ is a commutative ring with identity.
\end{proof}

\begin{definition}
    The \textbf{ring of rational numbers}\index{ring!of rational numbers} is $(\Q, +, \times)$, where $+$ and $\times$ denote normal addition and multiplication.
\end{definition}
\begin{remark}
    We denote the ring of rational numbers by $(\Q, +, \times)$.
\end{remark}
\begin{proposition}
    $\Q$ is a commutative ring with identity.
\end{proposition}
\begin{proof}
    We first show that $\Q$ satisfies the ring axioms.
    \begin{itemize}
        \item \textbf{Addition-Abelian}: We know that $(\Q, +)$ is an abelian group from \myref{problem-Q-is-abelian-group-under-addition}.
        \item \textbf{Multiplication-Semigroup}: We note that $(\Q, \times)$ is a semigroup as
        \begin{itemize}
            \item $\Q$ is closed under $\times$ because multiplying two rational numbers together produce a rational number; and
            \item multiplication is associative (\myref{axiom-multiplication-is-associative}).
        \end{itemize}
        \item \textbf{Distributive}: Multiplication distributes over addition by \myref{axiom-distributivity}.
    \end{itemize}
    Hence $\Q$ is a ring. Furthermore, $\times$ has an identity of 1 and is commutative (\myref{axiom-multiplication-is-commutative}). So $\Q$ is a commutative ring with identity.
\end{proof}

\begin{definition}
    The \textbf{ring of real numbers}\index{ring!of real numbers} is the ring $(\R, +, \times)$ where $+$ and $\times$ denotes regular addition and multiplication respectively.
\end{definition}
\begin{remark}
    We denote the ring of real numbers by $(\R, +, \times)$.
\end{remark}
\begin{proposition}
    $\R$ is a commutative ring with identity.
\end{proposition}
\begin{proof}
    Replace $(\Q, +)$ with $(\R, +)$ and $(\Q, \times)$ with $(\R, \times)$ in the previous proof.
\end{proof}

We end this section by looking at the ring of complex numbers.
\begin{definition}
    Let the set of \textbf{complex numbers}\index{complex numbers}
    \[
        \C = \{a + bi \vert a, b \in \R\}
    \]
    where $i = \sqrt{-1}$ is known as the \textbf{imaginary unit}\index{imaginary unit}, where $i^2 = -1$. Define complex addition and multiplication by
    \begin{align*}
        (a+bi) + (c+di) &= (a+c) + (b+d)i,\\
        (a+bi) \cdot (c+di) &= (ac-bd) + (ad+bc)i.
    \end{align*}
    Then $\C$ under complex addition and multiplication is the \textbf{ring of complex numbers}\index{ring!of complex numbers}.
\end{definition}
\begin{remark}
    We denote the ring of complex numbers by $\C$.
\end{remark}
\begin{proposition}
    $\C$ is a commutative ring with identity.
\end{proposition}
\begin{proof}
    We first show that $\C$ satisfies the ring axioms.
    \begin{itemize}
        \item \textbf{Addition-Abelian}: We show that $(\C, +)$ satisfies the group axioms, and then show that $(\C, +)$ is commutative.
        \begin{itemize}
            \item \textbf{Closure}: Clearly for all real numbers $a$, $b$, $c$, and $d$ we have $a + c \in \R$ and $b+d \in \R$. Thus $(a+bi) + (c+di) = (a+c) + (b+d)i \in \C$, meaning $\C$ is closed under complex addition.
            
            \item \textbf{Associativity}: Let $a+bi, c+di, e+fi \in \C$. Then note that
            \begin{align*}
                &(a+bi) + ((c+di) + (e+fi))\\
                &= (a+bi) + ((c+e) + (d+f)i)\\
                &= (a+(c+e)) + (b+(d+f))i\\
                &= ((a+c)+e) + ((b+d)+f)i & (+ \text{ is associative, }\myref{axiom-addition-is-associative})\\
                &= ((a+c) + (b+d)i) + (e+fi)\\
                &= ((a+bi) + (c+di)) + (e+fi)
            \end{align*}
            so complex addition is associative.
            
            \item \textbf{Identity}: The identity in $\C$ is $0 + 0i = 0$ since
            \[
                (0+0i) + (a+bi) = (0+a) + (0+b)i = a+bi
            \]
            and complex addition is commutative (to be proved later), so $(\C,+)$ has an additive identity.
            
            \item \textbf{Inverse}: Let $a+bi \in \C$. Clearly $-a, -b \in \R$ and that
            \[
                (a+bi) + (-a+(-b)i)= (a+(-a)) + (b+(-b))i = 0
            \]
            and complex addition is commutative (to be proved later), so any $a+bi\in C$ has an additive inverse of $-a-bi \in \C$.

            \item \textbf{Commutative}: Let $a+bi, c+di \in \C$. Then
            \begin{align*}
                (a+bi) + (c+di) &= (a+c) + (b+d)i\\
                &= (c+a) + (d+b)i & (+\text{ is commutative, } \myref{axiom-addition-is-commutative})\\
                &= (c+di) + (a+bi)
            \end{align*}
            so complex addition is commutative.
        \end{itemize}
        
        \item \textbf{Multiplication-Semigroup}: We show that $(\C, \times)$ is a semigroup.
        \begin{itemize}
            \item \textbf{Closure}: Clearly for all real numbers $a$, $b$, $c$, and $d$ we have $ac, bd, ad, bc \in \R$, so $ac - bd, ad + bc \in \R$. Therefore
            \[
                (a+bi)(c+di) = (ac-bd) + (ad+bc)i \in \C
            \]
            which means $\C$ is closed under multiplication.

            \item \textbf{Associativity}: Let $a+bi, c+di, e+fi \in \C$. Note that
            \begin{align*}
                &(a+bi)((c+di)(e+fi))\\
                &= (a+bi)((ce-df)+(cf+de)i)\\
                &= (a(ce-df) - b(cf+de)) + (a(cf+de) + b(ce-df))i\\
                &= (ace - adf - bcf - bde) + (acf + ade + bce - bdf)i\\
                &= (ace - bde - adf - bcf) + (acf - bdf + ade + bce)i\\
                &= ((ac-bd)e - (ad+bc)f) + ((ac-bd)f + (ad+bc)e)i\\
                &= ((ac-bd)+(ad+bc)i)(e+fi)\\
                &= ((a+bi)(c+di))(e+fi)
            \end{align*}
            so complex multiplication is associative.
        \end{itemize}
        
        \item \textbf{Distributive}: We only prove left distributivity because we will show that complex multiplication is commutative later. Let $a+bi, c+di, e+fi \in \C$. Note that
        \begin{align*}
            &(a+bi)((c+di) + (e+fi))\\
            &= (a+bi)((c+e) + (d+f)i)\\
            &= (a(c+e)-b(d+f)) + (a(d+f) + b(c+e))i\\
            &= (ac+ae-bd-bf) + (ad+af+bc+be)i\\
            &= (ac-bd+ae-bf) + (ad+bc+af+be)i & (+ \text{ is associative})\\
            &= ((ac-bd) + (ad+bc)i) + ((ae - bf) + (af + be)i)\\
            &= (a+bi)(c+di) + (a+bi)(e+fi)
        \end{align*}
        so complex multiplication distributes over complex addition.
    \end{itemize}
    Hence $\C$ is a ring.
    
    \newpage
     
    We now show that complex multiplication is commutative. Let $a+bi, c+di \in C$. Then we see
    \begin{align*}
        (a+bi)(c+di) &= (ac-bd) + (ad+bc)i\\
        &= (ca-db) + (da+cb)i & (\times\text{ is commutative, } \myref{axiom-multiplication-is-commutative})\\
        &= (c+di)(a+bi)
    \end{align*}
    so complex multiplication is commutative.

    Finally we show that complex multiplication has an identity. Consider $1 + 0i \in \C$. Note that
    \[
        (1+0i)(a+bi) = (1a-0b) + (1b+0a)i = a+bi,
    \]
    and since complex multiplication is commutative, therefore $1+0i$ is the multiplicative identity in $\C$.
    
    Therefore $\C$ is a commutative ring with identity.
\end{proof}

These are just some examples of rings; we explore more later in this chapter.
\begin{exercise}\label{exercise-ring-of-integers-is-a-ring}
    Prove that $\Z$ is a ring under regular addition and multiplication.\newline
    (\textit{You do \textbf{not} need to prove the \textbf{Distributive} axiom.})
\end{exercise}

\section{General Properties of Rings}
We list some properties of rings here. For each of the propositions, assume $R$ is a ring.

\begin{proposition}\label{prop-multiplying-by-zero-is-zero}
    $0x = x0 = 0$ for all $x \in R$.
\end{proposition}
\begin{proof}
    We note that
    \begin{align*}
        0x &= (0 + 0)x & (0 \text{ is additive inverse})\\
        &= 0x + 0x & (\text{by \textbf{Distributive} axiom})
    \end{align*}
    so by `subtracting' $0x$ on both sides (i.e., adding $-0x$ on both sides) we see $0 = 0x$.
    
    Also
    \begin{align*}
        x0 &= x(0 + 0) & (0 \text{ is additive inverse})\\
        &= x0 + x0 & (\text{by \textbf{Distributive} axiom})
    \end{align*}
    so by `subtracting' $x0$ on both sides we see $0 = x0$.
    
    Therefore $0x = x0 = 0$ for all $x \in R$.
\end{proof}

\begin{proposition}\label{prop-product-of-element-and-additive-inverse-is-additive-inverse-of-product}
    $(-a)b = a(-b) = -(ab)$ for any $a$ and $b$ in $R$.
\end{proposition}
\begin{proof}
    We show that $(-a)b = -(ab)$ and $a(-b) = -(ab)$ to complete the proof.
    \begin{itemize}
        \item Note $(-a)b + ab = (-a + a)b = 0b = 0$ by \textbf{Distributive} axiom. Hence by subtracting $ab$ on both sides we see $(-a)b = -(ab)$.
        \item Note also $a(-b) + ab = a(-b + b) = a0 = 0$ by \textbf{Distributive} axiom. Hence by subtracting $ab$ on both sides we see $a(-b) = -(ab)$.
    \end{itemize}
    Result follows.
\end{proof}

\begin{proposition}
    $(-a)(-b) = ab$ for any $a$ and $b$ in $R$.
\end{proposition}
\begin{proof}
    See \myref{exercise-product-of-additive-inverses} (later).
\end{proof}

\begin{proposition}
    If $R$ has an identity, it is unique.
\end{proposition}
\begin{proof}
    Suppose 1 and $1'$ are identities, and consider the sum $1 + 1'$. Then
    \begin{align*}
        1 + 1' &= 1\times(1+1') & (\text{multiplying by identity }1)\\
        &= 1\times1 + 1\times1' & (\text{by \textbf{Distributive} axiom})\\
        &= 1 + 1. & (1 \text{ and } 1' \text{ are identities})
    \end{align*}
    Subtracting 1 on both sides yields $1 = 1'$, meaning that the identity is unique.
\end{proof}

\begin{exercise}\label{exercise-product-of-additive-inverses}
    Show that $(-a)(-b) = ab$ for any $a$ and $b$ in $R$.
\end{exercise}

% \section{Some Non-Obvious Rings}
\section{Matrix Rings}
The rings that we explored in previous sections can be thought of as the `obvious' rings, since they are number systems. As rings were made to generalize number systems, they should clearly be rings. However, there are less obvious rings.

% \subsection{Matrix Rings}
We looked at matrices in the context of the General/Special Linear Group of matrices. Here we see that matrices in fact form rings, known as matrix rings. Before that though, we need to define the operations within that ring.

\begin{definition}[Matrix Addition]\index{matrix addition}
    For any two matrices $\textbf{A}$ and $\textbf{B}$ with $n$ rows and columns and entries in the ring $(R, \oplus, \otimes)$, their sum is the matrix $\textbf{C} = \textbf{A} + \textbf{B}$ with $n$ rows and columns such that
    \[
        c_{i,j} = a_{i,j} \oplus b_{i,j}
    \]
    for all $i,j \in \{1, 2, \dots, n\}$.
\end{definition}
\begin{definition}[Matrix Multiplication]\index{matrix multiplication}
    For any two matrices $\textbf{A}$ and $\textbf{B}$ with $n$ rows and columns and entries in the ring $(R, \oplus, \otimes)$, their product is the matrix $\textbf{C} = \textbf{AB}$ with $n$ rows and columns such that, for all $i,j \in \{1, 2, \dots, n\}$, we have
    \begin{align*}
        c_{i,j} &= (a_{i,1}\otimes b_{1,j}) \oplus (a_{i,2}\otimes b_{2,j}) \oplus \cdots \oplus (a_{i,n}\otimes b_{n,j})\\
        &= \bigoplus_{k=1}^n (a_{i,k}\otimes b_{k,j}).
    \end{align*}
\end{definition}

We also define two matrices that will become useful when we work with matrix rings.
\begin{definition}
    The \textbf{zero matrix}\index{zero matrix} with $n$ rows and columns is
    \[
        \ZeroM{n} = 
        \begin{pmatrix}
            0 & 0 & 0 & \cdots & 0 \\
            0 & 0 & 0 & \cdots & 0 \\
            0 & 0 & 0 & \cdots & 0 \\
            \vdots & \vdots & \vdots & \ddots & \vdots \\
            0 & 0 & 0 & \cdots & 0 \\
        \end{pmatrix}
    \]
    where 0 is the additive identity (i.e. zero) in the ring $(R, \oplus, \otimes)$.
\end{definition}
\begin{definition}
    The \textbf{identity matrix}\index{identity matrix} with $n$ rows and columns is
    \[
        \IdentityM{n} = 
        \begin{pmatrix}
            1 & 0 & 0 & \cdots & 0 \\
            0 & 1 & 0 & \cdots & 0 \\
            0 & 0 & 1 & \cdots & 0 \\
            \vdots & \vdots & \vdots & \ddots & \vdots \\
            0 & 0 & 0 & \cdots & 1 \\
        \end{pmatrix}
    \]
    where 0 and 1 are the additive and multiplicative identities (i.e. zero and one) in the ring $(R, \oplus, \otimes)$ respectively. That is, the identity matrix is the matrix with 1s in the leading diagonal.
\end{definition}

We can now define what is a matrix ring.
\begin{definition}
    Let $(R, \oplus, \otimes)$ be a ring and $n$ be a positive integer. Then $\Mn{n}{R}$ under matrix addition and multiplication is a ring, known as the \textbf{matrix ring}\index{matrix ring} with elements in $(R, \oplus, \otimes)$.
\end{definition}
\begin{proposition}
    $\Mn{n}{R}$ is a ring with identity.
\end{proposition}
\begin{proof}
    We need to prove that the ring axioms hold.
    \begin{itemize}
        \item \textbf{Addition-Abelian}: We first prove that $(\Mn{n}{R}, +)$ is indeed an abelian group.
        \begin{itemize}
            \item \textbf{Closure}: Clearly the sum of any two matrices in $\Mn{n}{R}$ is also a square matrix with $n$ rows with elements inside $R$, meaning that $\Mn{n}{R}$ is closed under matrix addition.

            \item \textbf{Associativity}: Let the matrices $\textbf{A}$, $\textbf{B}$, and $\textbf{C}$ belong inside $\Mn{n}{R}$. Let $\textbf{P} = \textbf{A} + (\textbf{B} + \textbf{C})$ and $\textbf{Q} = (\textbf{A} + \textbf{B}) + \textbf{C}$. We note that $\textbf{P} = \textbf{Q}$ as
            \[
                p_{i,j} = a_{i,j} \oplus (b_{i,j} \oplus c_{i,j}) = (a_{i,j} \oplus b_{i,j}) \oplus c_{i,j} = q_{i,j}
            \]
            by associativity of $\oplus$, which proves that matrix addition is associative.
    
            \item \textbf{Identity}: We show that $\ZeroM{n}$ is the additive identity in $\Mn{n}{R}$. Let $\textbf{M} \in \Mn{n}{R}$; let $\textbf{N} = \textbf{M} + \ZeroM{n}$. Note that $n_{i,j} = m_{i,j} \oplus 0 = m_{i,j}$ so $\textbf{M} + \ZeroM{n} = \textbf{M}$. Therefore $\textbf{M} + \ZeroM{n} = \textbf{M}$ for any matrix in $\Mn{n}{R}$.
            
            \item \textbf{Inverse}: Let $\textbf{A} \in \Mn{n}{R}$. Define the matrix $\textbf{B} = -\textbf{A}$ such that $b_{i,j} = -a_{i,j}$. That is, $b_{i,j}$ contains the additive inverse of $a_{i,j}$ in the ring $R$. Then one sees that $\textbf{A} + \textbf{B} = \ZeroM{n}$. (We denote the additive inverse of a matrix $\textbf{M}$ by $-\textbf{M}$).

            \item \textbf{Commutative}: Let $\textbf{A}, \textbf{B} \in \Mn{n}{R}$. Set $\textbf{C} = \textbf{A} + \textbf{B}$ and $\textbf{D} = \textbf{B} + \textbf{C}$. Consider $c_{i,j} = a_{i,j} \oplus b_{i,j}$. Since $\oplus$ is commutative, thus $a_{i,j} \oplus b_{i,j} = b_{i,j} \oplus a_{i,j}$. But $d_{i,j} = b_{i,j} \oplus a_{i,j}$, so we have $c_{i,j} = d_{i,j}$. Therefore $\textbf{C} = \textbf{D}$.
        \end{itemize}

        \item \textbf{Multiplication-Semigroup}: We show that $(\Mn{n}{R}, \cdot)$ is a semigroup.
        \begin{itemize}
            \item \textbf{Closure}: In \myref{subsection-intro-to-matrices} we showed that matrix multiplication produces another $n \times n$ matrix. Furthermore the entries of the new matrix are elements of $R$. Hence $\Mn{n}{R}$ is closed under matrix multiplication.
        
            \item \textbf{Associativity}: We proved matrix multiplication is associative in \myref{subsection-GLR-matrix-group}.
        \end{itemize}
        
        \item \textbf{Distributive}: We prove only $\textbf{A}(\textbf{B} + \textbf{C}) = (\textbf{AB}) + (\textbf{AC})$ as the other case is proven similarly. Let $\textbf{R} = \textbf{A}(\textbf{B} + \textbf{C})$, $\textbf{G} = \textbf{AB}$, and $\textbf{H} = \textbf{AC}$. We note
        \begin{align*}
            r_{i,j} &= \bigoplus_{k=1}^n \left(a_{i,k} \otimes \left(b_{k,j} \oplus c_{k,j}\right)\right)\\
            &= \bigoplus_{k=1}^n \left((a_{i,k} \otimes b_{k,j}) \oplus (a_{i,k} \otimes c_{k,j})\right)\\
            &= \left(\bigoplus_{k=1}^n (a_{i,k} \otimes b_{k,j})\right) \oplus \left(\bigoplus_{k=1}^n (a_{i,k} \otimes c_{k,j})\right)\\
            &= g_{i,j}\oplus h_{i,j}
        \end{align*}
        which means $\textbf{R} = \textbf{G} + \textbf{H}$.
    \end{itemize}
    As all the ring axioms are satisfied, thus $\Mn{n}{R}$ is a ring.

    We now show that $\Mn{n}{R}$ has a multiplicative identity, namely the identity matrix $\IdentityM{n}$. Let $\textbf{A} \in \Mn{n}{R}$ and let $\textbf{B} = \IdentityM{n}$. Note that $b_{i,j} = 1$ if and only if $i = j$.
    \begin{itemize}
        \item Let $\textbf{C} = \textbf{AB}$ and we see
        \begin{align*}
            &c_{i,j}\\
            &= \bigoplus_{k=1}^n(a_{i,k}\otimes b_{k,j})\\
            &= (a_{i,1}\otimes b_{1,j}) \oplus \cdots \oplus (a_{i,j-1}\otimes b_{j-1,j}) \oplus (a_{i,j}\otimes b_{j,j})\\
            &\quad\quad\oplus (a_{i,{j+1}}\otimes b_{j+1,j}) \oplus \cdots \oplus (a_{i,n}\otimes b_{n,j})\\
            &= (a_{i,1}\otimes 0) \oplus \cdots \oplus (a_{i,{j-1}}\otimes 0)\oplus (a_{i,j}\otimes 1) \oplus (a_{i,{j+1}}\otimes 0) \oplus \cdots \oplus (a_{i,n}\otimes 0)\\
            &= 0 \oplus \cdots \oplus 0 \oplus a_{i,j} \oplus 0 \oplus \cdots \oplus 0\\
            &= a_{i,j}
        \end{align*}
        so $\textbf{A}\IdentityM{n} = \textbf{A}$.

        \item Now let $\textbf{D} = \textbf{BA}$ and we also see
        \begin{align*}
            &d_{i,j}\\
            &= \bigoplus_{k=1}^n(b_{i,k}\otimes a_{k,j})\\
            &= (b_{i,1}\otimes a_{1,j}) \oplus \cdots \oplus (b_{i,i-1}\otimes b_{i-1,j}) \oplus (b_{i,i}\otimes a_{i,j})\\
            &\quad\quad\oplus (b_{i,{i+1}}\otimes a_{i+1,j}) \oplus \cdots \oplus (b_{i,n}\otimes a_{n,j})\\
            &= (0 \otimes a_{1,j}) \oplus \cdots \oplus (0\otimes a_{i-1,j})\oplus (1\otimes a_{i,j}) \oplus (0\otimes a_{i+1,j}) \oplus \cdots \oplus (0\otimes a_{n,j})\\
            &= 0 \oplus \cdots \oplus 0 \oplus a_{i,j} \oplus 0 \oplus \cdots \oplus 0\\
            &= a_{i,j}
        \end{align*}
        so $\IdentityM{n}\textbf{A} = \textbf{A}$.
    \end{itemize}
    Therefore the identity matrix $\IdentityM{n}$ is the multiplicative identity.

    Hence $\Mn{n}{R}$ is a ring with identity.
\end{proof}

% \subsection{Hamilton's Quaternions}
% The quaternions is a way to extend the complex numbers into 4 dimensions. Irish mathematician William Rowan Hamilton was looking for a way represent points in 3-dimensional space using numbers. He knew how to add and subtract triples of numbers, but had difficulty in defining a way to multiply and divide these numbers, just like in the ring of complex numbers. The breakthrough in quaternions came in 1843 when he thought of using 4-dimensional numbers instead of 3-dimensional numbers.

% We define what quaternions are, and what the quaternion ring is.
% \begin{definition}
%     A \textbf{quaternion}\index{quaternion} is an expression of the form
%     \[
%         a + b\qi + c\qj + d\qk
%     \]
%     where $a,b,c,d \in \R$ and $\qi$, $\qj$, and $\qk$ are quantities such that
%     \begin{itemize}
%         \item $\qi\qj = -\qj\qi = \qk$;
%         \item $\qj\qk = -\qk\qj = \qi$;
%         \item $\qk\qi = -\qi\qk = \qj$; and
%         \item $\qi^2 = \qk^2 = \qk^2 = \qi\qj\qk = -1$.
%     \end{itemize}
% \end{definition}
% \begin{definition}
%     The \textbf{quaternion ring}\index{quaternion ring} is
%     \[
%         \H = \{a + b\qi + c\qj + d\qk \vert a,b,c,d \in \R\}
%     \]
%     where, for two quaternions $q_1 = a_1+b_1\qi+c_1\qj+d_1\qk$ and $q_2 = a_2+b_2\qi+c_2\qj+d_2\qk$, addition is
%     \[
%         q_1 + q_2 = (a_1+a_2) + (b_1+b_2)\qi + (c_1+c_2)\qj + (d_1+d_2)\qk
%     \]
%     and multiplication is
%     \begin{align*}
%         q_1q_2 &= (a_1a_2 - b_1b_2 - c_1c_2 - d_1d_2)\\
%         &+(a_1b_2 + b_1a_2 + c_1d_2 - d_1c_2)\qi\\
%         &+(a_1c_2 - b_1d_2 + c_1a_2 + d_1b_2)\qj\\
%         &+(a_1d_2 + b_1c_2 - c_1b_2 + d_1a_2)\qk.
%     \end{align*}
% \end{definition}

\section{More Definitions}
Suppose $R$ is a ring.
\begin{definition}
    We say that $a \neq 0$ is a \textbf{zero divisor}\index{zero divisor} in $R$ if there exists $b \neq 0$ such that $ab = 0$.
\end{definition}
\begin{example}
    Consider the ring $\Z_{12}$. Clearly 4 and 6 are in $\Z_{12}$, and their product is $24 = 2 \times 12 = 0$ in $\Z_{12}$. Hence 4 and 6 are zero divisors in $\Z_{12}$.
\end{example}
\begin{example}
    Let $R$ be the ring of functions with domain and codomain $[0, 1]$. We claim that $R$ has zero divisors. Consider the functions
    \begin{align*}
        f:[0,1]\to[0,1], x &\mapsto x\\
        g:[0,1]\to[0,1], x &\mapsto \begin{cases}
            0 & \text{ if } x \neq 0\\
            1 & \text{ if } x = 0
        \end{cases}
    \end{align*}
    Clearly neither of them are the zero function. However, consider $f(x)g(x)$.
    \begin{itemize}
        \item If $x \neq 0$, then $g(x) = 0$ which means $f(x)g(x) = 0$.
        \item If $x = 0$, then $f(x) = 0$ which means $f(x)g(x) = 0$.
    \end{itemize}
    Hence their product is the zero function, meaning that $R$ has zero divisors $f$ and $g$.
\end{example}
\begin{exercise}
    Does the ring $\Mn{2}{\mathbb{R}}$ have zero divisors?
\end{exercise}
We note one property about zero divisors, which will be used in future chapters.
\begin{proposition}\label{prop-zero-divisors-have-no-inverses}
    Zero divisors do not have inverses.
\end{proposition}
\begin{proof}
    Assume $a \neq 0$ and $b \neq 0$ are zero divisors in the ring $R$, so $ab = 0$. Seeking a contradiction, assume $a$ has an inverse, so
    \[
        b = (a^{-1}a)b = a^{-1}(ab) = a^{-1}0 = 0    
    \]
    which contradicts $b \neq 0$. Hence a zero divisor has no inverse.
\end{proof}

\begin{definition}
    Suppose $R$ is a ring with identity such that $0 \neq 1$. An element $u \in R$ is called a \textbf{unit}\index{unit} if there exists a $v \in R$ such that $uv=vu=1$. Equivalently, $u$ is a unit if it has a multiplicative inverse.
\end{definition}
\begin{example}
    3 and 7 are units in $\Z_{10}$ since $3 \times 7 = 7 \times 3 = 21 = 1$ in $\Z_{10}$.
\end{example}

\begin{definition}
    Suppose $R$ is a ring with identity such that $0 \neq 1$. If every non-zero element $x \in R$ is a unit, then $R$ is said to be a \textbf{division ring}\index{division ring}.
\end{definition}

\begin{definition}
    A commutative division ring is called a \textbf{field}\index{field}.
\end{definition}

\begin{example}
    Earlier, we shown that $\R$ is a commutative ring. We now show that $\R$ is actually a field by noting that every non-zero $x \in \R$ has a reciprocal $\frac1x$ that is a real number (\myref{axiom-reciprocal}) such that $x\left(\frac1x\right) = \left(\frac1x\right)x = 1$. Thus every non-zero $x$ in $\R$ is a unit, meaning that $\R$ is a division ring. Coupled with the fact that $\R$ is a commutative ring means that $\R$ is a field.
\end{example}
\begin{example}
    We also shown earlier that $\C$ is a commutative ring. We note that any non-zero complex number $z = a + bi$ has a multiplicative inverse given by
    \[
        w = \frac{a}{a^2+b^2} - \frac{b}{a^2+b^2}i
    \]
    since
    \begin{align*}
        zw &= (a+bi)\left(\frac{a}{a^2+b^2} - \frac{b}{a^2+b^2}i\right)\\
        &= \frac{(a+bi)(a-bi)}{a^2+b^2}\\
        &= \frac{a^2 - b^2i^2}{a^2+b^2}\\
        &= \frac{a^2+b^2}{a^2+b^2} & (i^2 = -1)\\
        &= 1.
    \end{align*}
    Thus any non-zero complex number is a unit, meaning that $\C$ is a division ring. As $\C$ is also a commutative ring, this means that $\C$ is a field.
\end{example}

We look at fields in more detail in part III.

\begin{exercise}
    Which of the following rings, if any, are fields?
    \begin{partquestions}{\alph*}
        \item $\Z$
        \item $\Q$
    \end{partquestions}
\end{exercise}

\section{Subrings}
We end this chapter off with an exploration about subrings.

\begin{definition}
    Let $R$ be a ring and $S$ be a subset of $R$. Then $S$ is a \textbf{subring}\index{subring} of $R$ if
    \begin{itemize}
        \item $(S, +) \leq (R, +)$, that is, the subset $S$ under addition is a subgroup of $R$ under addition; and
        \item for all $a$ and $b$ in $S$ we have $ab \in S$, i.e. $S$ is closed under multiplication.
    \end{itemize}
\end{definition}
\begin{remark}
    Alternatively, one may show that $S$ is a ring to prove that $S$ is a subring of $R$.
\end{remark}

\begin{example}
    We know that $\Z$ and $\Q$ are rings, and clearly $\Z \subseteq \Q$. Hence $\Z$ is a subring of $\Q$. Similarly, since $\Q \subseteq \R$ and $\R \subseteq \C$, thus $\Q$ is a subring of $\R$ and $\R$ is a subring of $\C$.
\end{example}

\begin{example}
    Consider the set of \textbf{gaussian integers}\index{gaussian integers}
    \[
        \Z[i] = \{a + bi \vert a,b\in\mathbb{Z}\},
    \]
    read as ``$\Z$ adjoin $i$''. We first show that $\Z[i]$ is a subring of $\C$.
    \begin{proof}
        Clearly $\Z[i] \subseteq \C$. We show that $(\Z[i], +) \leq (\C, +)$.
        \begin{itemize}
            \item Clearly the identity of $(\C, +)$, which is 0, is inside $(\Z[i], +)$ as $0 = 0 + 0i$.
            \item For any $a + bi, c+di \in \Z[i]$, we have
            \[
                (a+bi) + (-(c+di)) = (a-c) + (b-d)i \in \Z[i].
            \]
        \end{itemize}
        Thus the subgroup test (\myref{thrm-subgroup-test}) tells us that $(\Z[i], +) \leq (\C, +)$.

        Now we show that $\Z[i]$ is closed under multiplication. Let $a + bi, c+di \in \Z[i]$. Note that $ac, bd, ad, bc \in \Z$; the product of the two gaussian integers is
        \begin{align*}
            (a+bi)(c+di) &= (ac-bd) + (ad+bc)i\\
            &\in \Z[i]
        \end{align*}
        so $\Z[i]$ is closed under multiplication.
        
        Therefore $\Z[i]$ is a subring of $\C$.
    \end{proof}
\end{example}
\begin{exercise}
    Show that
    \[
        R = \left\{\begin{pmatrix}a&a\\a&a\end{pmatrix} \vert a \in \R\right\}
    \]
    is a subring of $\Mn{2}{\mathbb{R}}$.
\end{exercise}

\newpage

\section{Problems}
\begin{problem}
    Let $R$ be a ring. Prove that if $u \in R$ is a unit then so is $-u$.
\end{problem}

\begin{problem}
    Prove that the trivial ring is the unique ring with identity in which $0 = 1$.
\end{problem}

\begin{problem}
    Let $R$ be a set with an operation $\ast$ such that for all elements $x$ and $y$ in $R$ we have $x \ast y \in R$. If $(R, \ast, \ast)$ is a ring, describe the elements in $R$.
\end{problem}

\begin{problem}
    Prove that it is impossible for an element of a ring $R$ to be both a zero divisor and a unit.
\end{problem}

\begin{problem}
    Let $R$ be a ring with identity 1, and let $x$ be an element from that ring.
    \begin{partquestions}{\roman*}
        \item Find \textbf{four} closed forms for the geometric series $1 + x + x^2 + x^3 + \cdots + x^n$.
        \item What are the condition(s) such that the closed forms are valid?
        \item Evaluate 112 in the ring $\Z_{37}$.
        \item Hence, using the result(s) above, evaluate
        \[
            1 + 2^3 + 2^6 + 2^9 + \cdots + 2^{72}
        \]
        in the ring $\Z_{37}$.
    \end{partquestions}
\end{problem}

\begin{problem}
    Show that
    \[
        \Q[\sqrt2] = \{a + b\sqrt2 \vert a,b \in \Q\}
    \]
    is a ring. Hence show it is a field.
\end{problem}

\begin{problem}
    Let
    \[
        R = \left\{\begin{pmatrix}a&b\\0&0\end{pmatrix} \vert a,b \in \R\right\}    
    \]
    be a ring under matrix addition and multiplication.
    \begin{partquestions}{\roman*}
        \item Show that $R$ has no identity.
        \item Show that $R$ contains a non-trivial subring $S$ with identity.
    \end{partquestions}
\end{problem}

\begin{problem}
    A ring $R$ is called a \textbf{Boolean ring}\index{Boolean ring} if $r^2 = r$ for all $r \in R$.
    \begin{partquestions}{\roman*}
        \item Show that $r = -r$ for all $r \in R$.
        \item Prove that every Boolean ring is commutative.
    \end{partquestions}
\end{problem}

\begin{problem}
    Let $R$ be a commutative ring with identity. We say that an element $x$ in $R$ is \textbf{nilpotent}\index{nilpotent} if there exists a positive integer $n$ such that $x^n = 0$.
    \begin{partquestions}{\roman*}
        \item Show that the product of two units is a unit.
        \item Let $u \in R$ be a unit and $x \in R$ be nilpotent. Show that $ux$ is nilpotent.
        \item Show that $u - x$ is a unit.
    \end{partquestions}
\end{problem}

\chapter{Integral Domains}
We explored the absolute basics of rings in the previous chapter. When working in abstract algebra, we usually prefer rings with slightly more structure as they allow us to generate more useful and applicable results. Integral domains are an important category of rings, which we explore in this chapter.

\section{What is an Integral Domain?}
\begin{definition}
    An \textbf{integral domain}\index{integral domain}, generally denoted $D$, is a commutative ring with identity and contains no zero divisors.
\end{definition}
\begin{remark}
    Recall that an element $a \neq 0$ is a zero divisor if there exists a $b \neq 0$ such that $ab = 0$. So, equivalently, if $a \neq 0$ and $b \neq 0$ then $ab \neq 0$ for any $a$ and $b$ in the integral domain.
\end{remark}
\begin{remark}
    What this property grants us is a ``cancellation law'', similar to that for groups. We will prove this in \myref{prop-integral-domain-cancellation-law}. However, there may not be inverses when performing multiplication. This is similar to how the integers do not have multiplicative inverses in general, hence the name of ``integral domain''.
\end{remark}

\begin{example}\label{example-integers-is-integral-domain}
    We show that the integers, $\Z$, the namesake for integral domains, is indeed an integral domain. We note that $\Z$ is a commutative ring as multiplication is commutative. Furthermore, 1 is the multiplicative identity in $\Z$. All that remains is to show that there does not exist any zero divisors in $\Z$.

    Suppose $a$ and $b$ are non-zero elements in $\Z$. We show that $ab \neq 0$ to prove that there does not exist any zero divisors. Let's first consider the case when $b > 0$. We may write
    \[
        ab = \underbrace{a + a + \cdots + a}_{b \text{ times}}
    \]
    which clearly is not zero since $a \neq 0$. Now if $b < 0$ then note
    \[
        ab = (-a)(-b) = \underbrace{(-a) + (-a) + \cdots + (-a)}_{(-b) \text{times}}
    \]
    which is also not zero as $-a \neq 0$. Hence there are no zero divisors in $\Z$.

    Therefore $\Z$ is a commutative ring with identity without any zero divisors, meaning that $\Z$ is an integral domain.
\end{example}

\begin{example}
    We show that the ring $\Z[i]$, the gaussian integers, is an integral domain. Similar to $\Z$, we note multiplication is commutative with identity 1.

    Suppose $w$ and $z$ are non-zero elements in $\Z[i]$; write $w = a+bi$ and $z = c+di$. Note that $a^2+b^2 \neq 0$ and $c^2 + d^2 \neq 0$. By definition of complex multiplication we have
    \[
        wz = (a+bi)(c+di) = (ac-bd) + (ad+bc)i.
    \]
    We just need to show that $ac-bd \neq 0$ and $ad+bc \neq 0$. Note that for any real numbers $x$ and $y$, we have $x^2 + y^2 \neq 0$ if and only if both $x$ and $y$ are non-zero. So we see
    \begin{align*}
        (ac-bd)^2 + (ad+bc)^2 &= (a^2c^2 - 2abcd + b^2d^2) + (a^2d^2 + 2abcd + b^2c^2)\\
        &= a^2c^2 + a^2d^2 + b^2c^2 + b^2d^2\\
        &= (a^2 + b^2)(c^2 + d^2)\\
        &\neq 0
    \end{align*}
    which hence means that $ac-bd \neq 0$ and $ad+bc \neq 0$. Therefore $wz \neq 0$, meaning that $\Z[i]$ has no zero divisors. Thus $\Z[i]$ is indeed an integral domain.
\end{example}

We note an interesting result regarding fields here.
\begin{proposition}\label{prop-field-is-integral-domain}
    Any field is an integral domain.
\end{proposition}
\begin{proof}
    Any non-zero element in a field is a unit, i.e. has an inverse. Now by \myref{prop-zero-divisors-have-no-inverses}, any zero divisor must not have an inverse; the converse being that any element that has an inverse cannot be a zero divisor. Hence, there are no zero divisors in a field. Now a field is commutative, therefore meaning that the field is an integral domain.
\end{proof}

Let's look at some non-examples of integral domains.
\begin{example}
    The ring $\Mn{n}{R}$ where $R$ is any ring is not an integral domain since $\Mn{n}{R}$ is not commutative.
\end{example}
\begin{example}
    The ring $2\Z$ is not an integral domain as it does not contain a multiplicative identity (namely, 1).
\end{example}
\begin{example}
    The ring $\Z \times \Z$ under pairwise addition and multiplication is not an integral domain as $(0,1) \neq (0, 0)$ and $(1, 0) \neq (0, 0)$ but $(0,1)(1,0) = (0, 0)$, meaning that $\Z \times \Z$ contains at least one zero divisor.
\end{example}

\begin{exercise}
    Prove that the ring
    \[
        \Z[\sqrt2] = \left\{a + b\sqrt2 \vert a,b \in \Z\right\}
    \]
    is an integral domain.
\end{exercise}
\begin{exercise}
    Explain why all $\Z_n$ where $n$ is a positive composite number are not integral domains.
\end{exercise}

\section{General Results}
With an introduction of integral domains out of the way, we introduce some general results applicable to integral domains.
\begin{proposition}[Cancellation Law for Integral Domains]\index{integral domain!cancellation law}\label{prop-integral-domain-cancellation-law}
    Let $D$ be an integral domain, $r, x, y \in D$, and $r \neq 0$. Then the following statements are equivalent.
    \begin{enumerate}[label=(\arabic*)]
        \item $x = y$
        \item $rx = ry$
        \item $xr = yr$
    \end{enumerate}
\end{proposition}
\begin{proof}
    We prove the statements in order.
    \begin{itemize}
        \item $\boxed{(1) \implies (2)}$ Given $x = y$, this means $x - y = 0$. Multiplying $r$ on the left on both sides yields $r(x-y) = r0 = 0$. Distributing yields $rx - ry = 0$ which hence means $rx = ry$.
        
        \item $\boxed{(2) \implies (3)}$ Given $rx = ry$. Multiplying $r$ on the right on both sides yields $rxr = ryr$, thereby meaning $rxr - ryr = 0$. Factoring yields $r(xr - yr) = 0$. As $R$ is an integral domain, the only way for this to occur is if $r = 0$ (which is impossible) or $xr - yr = 0$. Therefore $xr = yr$.
        
        \item $\boxed{(3) \implies (1)}$ Given $xr = yr$ we may write $xr - yr = (x-y)r = 0$. Since $R$ is an integral domain we must have $x - y = 0$ or $r = 0$ (impossible as $r \neq 0$). Hence $x = y$.
    \end{itemize}
    This proves the proposition.
\end{proof}

\begin{theorem}\label{thrm-finite-integral-domain-is-field}
    Every finite integral domain is a field.
\end{theorem}
\begin{proof}
    A finite integral domain is a commutative ring with identity; all that remains is to prove that every non-zero element has an inverse.

    Suppose $D$ is a finite integral domain and take a non-zero element $r$ in $D$. Consider
    \begin{align*}
        A &= \{r^k \vert k \in \Z,\;k\geq 0\}\\
        &= \{1, r, r^2, r^3, \dots\}\\
        &\subseteq D.
    \end{align*}
    Note that since $D$ is finite, $A$ must be finite as well. Therefore there must exist positive integers $m$ and $n$ with $m > n$ such that $r^m = r^n$, meaning $r^m - r^n = 0$. As $m > n$, thus $r^n\left(r^{m-n}-1\right) = 0$. Hence, as there are no zero divisors in $R$, either $r^n = 0$ or $r^{m-n} - 1 = 0$.

    Now if $r^n = 0$ then $n > 1$ (since $r^1 = r \neq 0$ by assumption). Note $r^n = rr^{n-1} = 0$, which hence means $r^{n-1} = 0$ as $r \neq 0$ and $D$ has no zero divisors. But we may repeat this argument on $n - 1$ to eventually terminate at $r = 0$, which is a contradiction. Hence we must conclude that $r^{m-n} - 1 = 0$. We may write $r^{m-n}-1 = 0$ as $rr^{m-n-1} = 1$ which means that $r$ is a unit (as $r^{m-n-1}$ is its inverse). Therefore every non-zero element has an inverse.

    Hence $D$ is a field.
\end{proof}

\begin{remark}
    We note that finite fields are also known as \textbf{Galois fields}\index{Galois Field}, so-named in honour of \'Evariste Galois. We explore more properties of fields and Galois fields in part III.
\end{remark}

\begin{example}\label{example-Zp-is-field}
    Let $p$ be a prime and consider the ring $\Z_p$. Clearly multiplication is commutative with identity 1. Also, as the only way to factor $p$ is $1 \times p$ and since $p \notin \Z_p$, thus there are no zero divisors in $\Z_p$. Hence $\Z_p$ is an integral domain. Now $\Z_p$ only has $p$ elements (namely the integers 0 to $p - 1$), so it is finite. By \myref{thrm-finite-integral-domain-is-field} we know that $\Z_p$ is a field.
\end{example}

\begin{example}
    We show that $\Z_3[i] = \{a+bi \vert a,b \in \Z_3\}$ is a finite integral domain in order to show that it is a field. Clearly multiplication is commutative with identity 1, so we just need to show that there are no zero divisors in $\Z_3[i]$.
    \begin{table}[h]
        \centering
        \resizebox{\textwidth}{!}{
            \begin{tabular}{|l|l|l|l|l|l|l|l|l|}
            \hline
            $\boldsymbol{\times}$ & $\boldsymbol{1}$ & $\boldsymbol{2}$ & $\boldsymbol{i}$ & $\boldsymbol{1+i}$ & $\boldsymbol{2+i}$ & $\boldsymbol{2i}$ & $\boldsymbol{1+2i}$ & $\boldsymbol{2+2i}$ \\ \hline
            $\boldsymbol{1}$    & 1      & 2      & $i$    & $1+i$  & $2+i$  & $2i$   & $1+2i$ & $2+2i$ \\ \hline
            $\boldsymbol{2}$    & 2      & 1      & $2i$   & $2+2i$ & $1+2i$ & $i$    & $2+i$  & $1+i$  \\ \hline
            $\boldsymbol{i}$    & $i$    & $2i$   & 2      & $2+i$  & $2+2i$ & 1      & $1+i$  & $1+2i$ \\ \hline
            $\boldsymbol{1+i}$  & $1+i$  & $2+2i$ & $2+i$  & $2i$   & 1      & $1+2i$ & 2      & $i$    \\ \hline
            $\boldsymbol{2+i}$  & $2+i$  & $1+2i$ & $2+2i$ & 1      & $i$    & $1+i$  & $2i$   & 2      \\ \hline
            $\boldsymbol{2i}$   & $2i$   & $i$    & 1      & $1+2i$ & $1+i$  & 2      & $2+2i$ & $2+i$  \\ \hline
            $\boldsymbol{1+2i}$ & $1+2i$ & $2+i$  & $1+i$  & 2      & $2i$   & $2+2i$ & $i$    & 1      \\ \hline
            $\boldsymbol{2+2i}$ & $2+2i$ & $1+i$  & $1+2i$ & $i$    & 2      & $2+i$  & 1      & $2i$   \\ \hline
            \end{tabular}
        }
    \end{table}
    
    From the table, we have shown that no two non-zero elements in $\Z_3[i]$ multiply to zero. Therefore $\Z_3[i]$ has no zero divisors, which means $\Z_3[i]$ is an integral domain. Furthermore, as $\Z_3[i]$ is finite, therefore we know that it is a finite field by \myref{thrm-finite-integral-domain-is-field}.
\end{example}

\begin{exercise}\label{exercise-Zn2[alpha]}
    Is the ring
    \[
        \Z_2[\alpha] = \{a+b\alpha \vert a,b\in\Z_2\}
    \]
    where $\alpha^2 = 1 + \alpha$ under regular addition and multiplication a field?
\end{exercise}

\section{Characteristic of a Ring}
Before we introduce the characteristic of a ring, we clarify the meaning of the order of an element in the ring.
\begin{definition}
    Let $R$ be a ring and $x$ an element of the ring $R$.
    \begin{itemize}
        \item The \textbf{additive order}\index{order!additive} of $x$ is the order of $x$ in the group $(R, +)$. We denote the additive order of $x$ by $|x|_+$.
        \item If $R$ is a group with identity 1, the \textbf{multiplicative order}\index{order!multiplicative} of $x$ is the smallest positive integer $n$ such that $x^n = 1$ and is denoted $|x|_\times$, if it exists.
    \end{itemize}
\end{definition}

\begin{definition}
    Let $R$ be a ring. We say that \textbf{characteristic}\index{characteristic} of $R$ is $n$ if $n$ is the smallest positive integer such that
    \[
        nr = \underbrace{r+r+\cdots+r}_{n \text{ times}} = 0
    \]
    for all $r \in R$. We then write $\Char{R} = n$. If no such $n$ exists we say $\Char{R} = 0$.
\end{definition}

We look at two properties about the characteristic.
\begin{proposition}
    If $R$ is a ring with identity and $\Char{R} \neq 0$, then $\Char{R} = |1|_+$.
\end{proposition}
\begin{proof}
    Let $\Char{R} = n$ and $|1|_+ = m$. Our goal is to show that $m = n$.

    On one hand, note that
    \[
        n1 = \underbrace{1+1+1+\cdots+1}_{n \text{ times}} = 0
    \]
    by definition of the characteristic of a ring. Note also that $m1 = \overbrace{1+1+1+\cdots+1}^{m \text{ times}} = 0$ by definition of the order of an element in the group $(R, +)$. By \myref{problem-element-to-power-of-multiple-of-order-is-identity}, this means that $n$ is a multiple of $m$, which thus means $m \leq n$.

    On another hand, for any $r \in R$, we know that
    \begin{align*}
        \underbrace{r + r + r + \cdots + r}_{m \text{ times}} &= r(\underbrace{1+1+1+\cdots+1}_{m \text{ times}})\\
        &= r0 & (\text{since } |1|_+ = m)\\
        &= 0.
    \end{align*}
    The minimality of the characteristic $n$ means that $m$ has to be at least $n$, i.e. $m \geq n$.

    Since $m \leq n$ and $m \geq n$, thus $m = n$.
\end{proof}

\begin{proposition}\label{prop-zero-of-prime-characteristic-if-integral-domain}
    If $D$ is an integral domain, then either $\Char{D} = 0$ or $\Char{D} = p$ where $p$ is a prime.
\end{proposition}
\begin{proof}
    If $\Char{D} = 0$ we are done, so assume that $\Char{D} = n \neq 0$. Furthermore assume $n = ab$ with $1 \leq a,b \leq n$.

    Note that
    \[
        0 = \underbrace{1 + 1 + \cdots + 1}_{n \text{ times}} = (\underbrace{1+1+\cdots+1}_{a \text{ times}})(\underbrace{1+1+\cdots+1}_{b \text{ times}}).
    \]
    As $R$ is an integral domain, this means that either
    \[
        \underbrace{1+1+\cdots+1}_{a \text{ times}} = 0 \text{ or } \underbrace{1+1+\cdots+1}_{b \text{ times}} = 0.
    \]
    Without loss of generality, assume that $\underbrace{1+1+\cdots+1}_{a \text{ times}} = 0$. By minimality of $\Char{R} = n$, this means that $a \geq n$. But we assumed that $a \leq n$, so we must have $a = n$. Therefore $a = n$ and $b = 1$ which is exactly what we need to show that $n$ is a prime.
\end{proof}

\begin{exercise}\label{exercise-trivial-ring-is-not-an-integral-domain}
    Is the trivial ring an integral domain?
\end{exercise}
\begin{exercise}
    What is the characteristic of the ring $\Z_2[\alpha]$ (as defined in \myref{exercise-Zn2[alpha]})?
\end{exercise} 

\newpage

\section{Problems}
\begin{problem}
    Find two zero divisors in the ring
    \[
        \Z_5[i] = \{a + bi \vert a,b\in\Z_5\}.
    \]
\end{problem}

\begin{problem}
    Show that the ring of gaussian integers, $\Z[i]$, is an integral domain.
\end{problem}

\begin{problem}
    Let the integer $n$ be such that $\sqrt{|n|}$ is not an integer. Let the ring
    \[
        R = \Z[\sqrt{n}] = \{a + b\sqrt{n} \vert a,b\in \Z\}.
    \]
    \begin{partquestions}{\alph*}
        \item Show that $R$ is an integral domain.
        \item Is $R$ a field for all integers $n$?
    \end{partquestions}
\end{problem}

\begin{problem}
    Show that
    \[
        R = \left\{\begin{pmatrix}0&0\\0&0\end{pmatrix},\begin{pmatrix}1&0\\0&1\end{pmatrix},\begin{pmatrix}1&1\\1&0\end{pmatrix},\begin{pmatrix}0&1\\1&1\end{pmatrix}\right\}
    \]
    with entries in $\Z_2$ is
    \begin{partquestions}{\roman*}
        \item a subring of $\Mn{2}{\Z_2}$; and
        \item a field.
    \end{partquestions}
\end{problem}

\chapter{Ideals and Quotient Rings}
Ideals are a special type of subring. They help us define the idea of quotient rings in ring theory, similar to how quotient ring are defined in group theory. Ideals form a fundamental part of ring theory, and we will see them appearing repeatedly in the following chapters.

\section{History Behind Ideals}
The term ``ideal'' originated from Ernst Kummer, who was looking to study factorization of numbers. For instance, we may factor $45$ in two ``different'' ways, $45 = 5 \times 9$ and $45 = 3 \times 15$. However, these numbers have not been `factored enough', as we can reduce the factorization down to the primes, $45 = 3^3 \times 5$. This is a unique factorization. However, over algebraic rings, such as in $\Z[\sqrt{-3}]$, this idea of unique factorization down to the irreducible factors fails, since
\[
    4 = 2 \times 2 = (1+\sqrt{-3})(1-\sqrt{-3})
\]
and all the factors here are irreducible.

Kummer's idea was that these irreducible factors were not `reduced enough'; that there were better, ``ideal'' factors for which the ideal factorization of numbers hold. However, such a construction requires that one prove the existence of such ideals, and then show that they are indeed ideals -- too tedious for practical use. 

Richard Dedekind came up with an alternate definition for ``ideals'', defining not the numbers themselves, but the set of numbers that they divide. So instead of talking about the ``ideal number'' 2, we talk about the set of numbers that are divided by 2, namely $\{\dots, -4, -2, 0, 2, 4 \dots\}$, which is what we now call the principal ideal generated by 2. This way, sums and products of ideals become easier to handle, and more results can be created from their use.

The modern formulation of ideals are much weaker than the original ideals proposed by Dedekind; their motivation in modern times stems from the desire to create ``quotient rings'', similar to that of ``quotient groups'' in group theory. Ideals are the ``normal subgroups'' of ring theory. In the coming sections, we look at ideals, before working our way back to Dedekind's original principal ideal definition.

\newpage

\section{What is an Ideal?}
Rings have two operations defined on them (addition and multiplication). Therefore, we need to define which operation for which a coset will apply to.
\begin{definition}
    Given a ring $R$ and a subring $S$ of $R$, the \textbf{coset}\index{coset!for rings} of $S$ in $R$ with \textbf{representative}\index{coset!representative} $r \in R$ is
    \[
        r + S = \{r+s \vert s \in S\}.
    \]
\end{definition}
\begin{remark}
    As $(R,+)$ is an abelian group, thus, as groups, $(S,+)$ is a normal subgroup of $(R,+)$ and so $(R/S,+)$ is well-defined.
\end{remark}

At this point, we only know that $(R/S,+)$ is a subgroup of $(R,+)$. What conditions do we need on $S$ such that $R/S$ is a ring? Well, we need a ``well-defined'' multiplication operation on the cosets, specifically, for any two elements $a$ and $b$ in $R$, we want
\[
    (a+S)(b+s) = ab + S.
\]
We now try and find the condition(s) required for this operation to be well-defined.

Suppose $a+S = c+S$ and $b+S = d+S$ for some other elements $c$ and $d$ in $R$; our goal is to show $ab+S = cd+S$. Now by Coset Equality (\myref{lemma-coset-equality}), statements 1 and 5, we have $a-c \in S$ and $b-d \in S$. Set $a-c = s_1 \in S$ and $b-d = s_2 \in S$, so $a = c+s_1$ and $b = d+s_2$. Hence,
\[
    ab = (c+s_1)(d+s_2) = cd + cs_2 + s_1d + s_1s_2.
\]
Now for $ab + S = cd+S$ we need to have $ab-cd \in S$, again by Coset Equality statements 1 and 5. Therefore, we need to have $ab-cd = cs_2+s_1d+s_1s_2 \in S$. We note that $s_1s_2 \in S$ (as $S$ is a subring), so we just need both $cs_2$ and $s_1d$ to be in $S$ for $R/S$ to be a ring.

We are now ready to present the definition of an ideal.
\begin{definition}
    Let $R$ be a ring and a subset $I$ of $R$. Suppose $(I,+) \leq (R,+)$. Then $I$ is an \textbf{ideal}\index{ideal} (or \textbf{two-sided ideal}\index{ideal!two-sided}) if, for every $r \in R$ and $i \in R$, both $ri$ and $ir$ are in $I$.
\end{definition}
\begin{exercise}\label{exercise-ideal-is-a-subring}
    Let $I$ be an ideal of a ring $R$. Show that $I$ is a subring of $R$.
\end{exercise}
\begin{example}\label{example-nZ-ideal-of-Z}
    We show that the subring
    \[
        n\Z = \{\dots, -2n, -n, 0, n, 2n, \dots\}
    \]
    is an ideal of $\Z$.
    
    Suppose $m \in \Z$ and $an \in n\Z$.  Note that
    \[
        (m)(an) = (ma)n \in n\Z
    \]
    and
    \[
        (an)(m) = a(nm) = a(mn) = (am)n \in n\Z,
    \]
    with swapping of $m$ and $n$ possible since $\Z$ is a commutative ring. Therefore $n\Z$ is an ideal of $\Z$.
\end{example}

\begin{example}
    The subset $\{0\}$ is an ideal of any ring, called the \textbf{trivial ideal}\index{ideal!trivial} (or \textbf{zero ideal}\index{ideal!zero}). An ideal that is not $\{0\}$ is called a \textbf{non-trivial ideal}\index{ideal!non-trivial} (or \textbf{non-zero ideal}\index{ideal!non-zero}).
\end{example}
\begin{example}
    For any ring $R$, the ring itself is an ideal of $R$. An ideal $I$ that is a proper subset of the ring $R$ is a \textbf{proper ideal}\index{ideal!proper} of $R$.
\end{example}

We are almost ready to define the quotient ring; we just need to define the operations within it.
\begin{definition}[Coset Addition]
    The sum of the cosets $r+I$ and $s+I$ is $(r+s)+I$.
\end{definition}
\begin{definition}[Coset Multiplication]
    The product of $r+I$ and $s+I$ is $rs+I$.
\end{definition}

We can now define the quotient ring.
\begin{definition}
    Given a ring $R$ and an ideal $I$ of $R$, the \textbf{quotient ring}\index{quotient ring} of $R$ by $I$ is
    \[
        R/I = \{r + I \vert r \in R\},
    \]
    under coset addition and multiplication.
\end{definition}
\begin{remark}
    Some authors (e.g. \cite[p.~243]{dummit_foote_2004}) may choose to represent $r + I$ as $\overline{r}$. With this notation, addition and multiplication in the quotient ring becomes $\overline{r}+\overline{s} = \overline{r+s}$ and $\overline{r}\,\overline{s} = \overline{rs}$ respectively.
\end{remark}
\begin{remark}
    Like in group theory, $R/I$ is read as ``$R$ mod $I$''.
\end{remark}

\begin{proposition}
    If $R$ is a ring with ideal $I$, then $R/I$ is indeed a ring under coset addition and multiplication.
\end{proposition}
\begin{proof}
    We prove this using the ring axioms.
    \begin{itemize}
        \item \textbf{Addition-Abelian}: We show that $(R/I,+)$ is an abelian group.
        \begin{itemize}
            \item \textbf{Group}: As $(R,+)$ is an abelian group and $(I,+)$ is a normal subgroup of $(R,+)$, therefore $(R/I,+)$ is a well-defined quotient group.
            \item \textbf{Commutative}: All that remains is to show that coset addition is commutative. Let $r+I$ and $s+I$ both be in $R/I$. Then
            \begin{align*}
                (r+I) + (s+I) &= (r+s)+I\\
                &= (s+r) + I & (\text{since } + \text{ is commutative})\\
                &= (s+I) + (r+I)
            \end{align*}
            so coset addition is commutative.
        \end{itemize}
        \item \textbf{Multiplication-Semigroup}: We need to show that $R/I$ is a semigroup under coset multiplication.
        \begin{itemize}
            \item \textbf{Closure}: Let $r+I$ and $s+I$ be both cosets in $R/I$. Note that $rs \in R$ by closure of multiplication. Therefore
            \[
                (r+I)(s+I) = rs+I \in R/I
            \]
            which means that $R/I$ is closed under coset multiplication.

            \item \textbf{Associativity}: Let $r+I$, $s+I$ and $t+I$ be in $R/I$. Then
            \begin{align*}
                (r+I)((s+I)(t+I)) &= (r+I)(st+I)\\
                &= (r(st))+I\\
                &= ((rs)t) + I & (\text{since } \times \text{ is associative})\\
                &= ((rs)+I)(t+I)\\
                &= ((r+I)(s+I))(t+I)
            \end{align*}
            which means that coset multiplication is associative.

        \end{itemize}
        \item \textbf{Distributive}: Lastly, we need to show that coset multiplication distributes over coset addition. For the following, let the cosets $r+I$, $s+I$ and $t+I$ be in $R/I$.
        \begin{itemize}
            \item We first show left distribution.
            \begin{align*}
                (r+I)((s+I)+(t+I)) &= (r+I)((s+t)+I)\\
                &= (r(s+t))+I\\
                &= (rs+rt)+I\\
                &= (rs+I) + (rt+I)\\
                &=(r+I)(s+I) + (r+I)(t+I).
            \end{align*}
            Therefore the left distributive property has been shown.

            \item We now show right distribution.
            \begin{align*}
                ((r+I)+(s+I))(t+I) &= ((r+s)+I)(t+I)\\
                &= ((r+s)t)+I\\
                &= (rt+st)+I\\
                &= (rt+I) + (st+I)\\
                &= (r+I)(t+I) + (s+I)(t+I).
            \end{align*}
            Therefore the right distributive property has been shown.
        \end{itemize}
    \end{itemize}
    Therefore $R/I$ is a ring.
\end{proof}

\begin{example}
    Consider the subring $6\Z$. From \myref{example-nZ-ideal-of-Z} we know that $6\Z$ is an ideal of $\Z$. Thus $\Z/6\Z$ is a quotient ring; it is given by
    \[
        \Z/6\Z = \{0 + 6\Z, 1+6\Z, 2+6\Z, 3+6\Z, 4+6\Z, 5+6\Z\}.
    \]
    
    We explore two possible products in this quotient ring.
    \begin{itemize}
        \item $(3+6\Z)(4+6\Z) = 12+6\Z = 2\times 6 + 6\Z = 0 + 6\Z$. Thus we have found a pair of zero divisors in $\Z/6\Z$, namely $(3+6\Z)$ and $(4+6\Z)$.
        \item $(5+6\Z)^2 = 25+6\Z = (1 + 6\times4) + 6\Z = 1 + 6\Z$.
    \end{itemize}
\end{example}

\begin{exercise}
    Let the sets
    \begin{align*}
        R &= \left\{\begin{pmatrix}x&y\\0&z\end{pmatrix} \vert x,y,z \in \R\right\},\\
        I &= \left\{\begin{pmatrix}x&y\\0&0\end{pmatrix} \vert x,y \in \R\right\}.
    \end{align*}
    It is given that $R$ is a subring of $\Mn{2}{\R}$.
    \begin{partquestions}{\roman*}
        \item Show that $I$ is a subring of $R$.
        \item Show that $I$ is an ideal of $R$.
        \item Simplify
        \[
            \left(\begin{pmatrix}1&2\\0&1\end{pmatrix} + I\right)\left(\begin{pmatrix}1&-2\\0&1\end{pmatrix} + I\right)
        \]
        in $R/I$.
    \end{partquestions}
\end{exercise}

To save us hassle of testing whether a subset of a ring is an ideal, we introduce the test for ideal\index{test for ideal}.
\begin{theorem}[Test for Ideal]\label{thrm-test-for-ideal}
    Let $R$ be a ring and $I$ be a non-empty subset of $R$. Then $I$ is an ideal of $R$ if and only if
    \begin{enumerate}
        \item $x - y \in I$ for all $x$ and $y$ in $I$; and
        \item $ri$ and $ir$ are in $I$ for all $i$ in $I$ and $r$ in $R$.
    \end{enumerate}
\end{theorem}
\begin{proof}
    The forward direction is trivial to prove; if $I$ is an ideal then $I$ is a subring by \myref{exercise-ideal-is-a-subring}, meaning statement 1 holds. Statement 2 holds because that is the definition of an ideal.

    We now work in the reverse direction by assuming the two statements hold. First and foremost we know that $(I,+) \leq (R,+)$ by virtue of statement 1 and by applying the subgroup test (\myref{thrm-subgroup-test}). Now statement 2 tells us that $ri \in I$ for any $r \in R$, in particular we may choose an $r \in I$ so that $ri \in I$. Therefore $I$ is closed under multiplication, and so $I$ is a subring of $R$. Finally, statement 2 is exactly the condition for $I$ to be an ideal.
\end{proof}
\begin{remark}
    Like with the subgroup test (\myref{thrm-subgroup-test}), we can check whether $I$ is non-empty by seeing if the additive identity 0 is in $I$.
\end{remark}

We end this section by noting one result.
\begin{proposition}\label{prop-ideal-contains-unit-iff-ideal-is-whole-ring}
    Let $R$ be a ring with identity and $I$ an ideal of $R$. Then $I$ contains a unit if and only if $I = R$.
\end{proposition}
\begin{proof}
    See \myref{exercise-ideal-contains-unit-iff-ideal-is-whole-ring} (later).
\end{proof}

\begin{exercise}\label{exercise-ideal-contains-unit-iff-ideal-is-whole-ring}
    Let $R$ be a ring with identity 1, and let $I$ be an ideal of $R$.
    \begin{partquestions}{\roman*}
        \item If $1 \in I$, what does this imply about $I$?
        \item Prove $I$ contains a unit if and only if $I = R$.
    \end{partquestions}
\end{exercise}

\section{Ideal Operations}
We first look at the sum of ideals.
\begin{definition}\index{ideal!sum}
    Let $R$ be a ring and let $\ideal{a}$ and $\ideal{b}$ be ideals of $R$. Then the sum of the ideals $\ideal{a}$ and $\ideal{b}$ is
    \[
        \ideal{a} + \ideal{b} = \{a + b \vert a\in\ideal{a},\;b\in\ideal{b}\}.
    \]
\end{definition}
\begin{example}
    Consider the ring $\Z$ and the ideals $I = \{2n \vert n \in \Z\}$ and $J = \{3n \vert n \in \Z\}$. Then their sum is
    \[
        I + J = \{2a + 3b \vert a,b \in \Z\}.
    \]
\end{example}
\begin{proposition}\label{prop-sum-of-ideals-is-ideal}
    Let $R$ be a ring and let $\ideal{a}$ and $\ideal{b}$ be ideals of $R$. Then $\ideal{a} + \ideal{b}$ is an ideal of $R$.
\end{proposition}
\begin{proof}
    We note that since $\ideal{a}$ and $\ideal{b}$ are ideals, they are thus non-empty and so $\ideal{a}+\ideal{b}$ is non-empty.

    Let $a_1, a_2 \in \ideal{a}$ and $b_1, b_2 \in \ideal{b}$. Then we know $a_1 - a_2 \in \ideal{a}$ and $b_1 - b_2 \in \ideal{b}$ by the test for ideal. Therefore $(a_1 - a_2) + (b_1 - b_2) = (a_1 + b_1) - (a_2 + b_2) \in \ideal{a} + \ideal{b}$, satisfying the first statement.

    Now take any $r \in R$ and $a+b \in \ideal{a}+\ideal{b}$.
    \begin{itemize}
        \item $r(a+b) = ra + rb \in \ideal{a}+\ideal{b}$ since $ra \in \ideal{a}$ (because $\ideal{a}$ is an ideal) and $rb \in \ideal{b}$ (because $\ideal{b}$ is an ideal).
        \item $(a+b)r = ar + br \in \ideal{a}+\ideal{b}$ since $ar \in \ideal{a}$ (because $\ideal{a}$ is an ideal) and $br \in \ideal{b}$ (because $\ideal{b}$ is an ideal).
    \end{itemize}
    Therefore $\ideal{a} + \ideal{b}$ is an ideal by the test for ideal (\myref{thrm-test-for-ideal}).
\end{proof}

We now look at the product of ideals.
\begin{definition}\index{ideal!product}
    Let $R$ be a ring and let $\ideal{a}$ and $\ideal{b}$ be ideals of $R$. Then the product of the ideals $\ideal{a}$ and $\ideal{b}$ is
    \[
        \ideal{ab} = \{a_1b_1 + a_2b_2 + \cdots + a_nb_n \vert a_i\in\ideal{a},\;b_i\in\ideal{b}\}.
    \]
\end{definition}

\begin{example}
    Consider the ring $\Z$ and the ideals $I = \{2n \vert n \in \Z\}$ and $J = \{3n \vert n \in \Z\}$. Then their product is
    \begin{align*}
        IJ &= \{(2a_1)(3b_1) + (2a_2)(3b_2) + \cdots + (2a_n)(3b_n) \vert a_i,b_i \in \Z\}\\
        &= \{6(a_1b_1 + a_2b_2 + \cdots + a_nb_n) \vert a_i,b_i \in \Z\}\\
        &= \{6k \vert k \in \Z\}.
    \end{align*}
\end{example}

\begin{proposition}\label{prop-product-of-ideals-is-ideal}
    Let $R$ be a ring and let $\ideal{a}$ and $\ideal{b}$ be ideals of $R$. Then $\ideal{ab}$ is an ideal of $R$.
\end{proposition}
\begin{proof}
    Since $\ideal{a}$ and $\ideal{b}$ are non-empty as they are ideals, therefore $\ideal{ab}$ is non-empty.

    Let $a_1,a_2 \in \ideal{a}$ and $b_1,b_2 \in \ideal{b}$. Since $\ideal{a}$ is an ideal and is hence a subring, so $-a_2 \in \ideal{a}$. Thus $a_1b_1, (-a_2)(b_2) \in \ideal{ab}$. Clearly $a_1b_1 + (-a_2)(b_2) = a_1b_1 - a_2b_2 \in \ideal{ab}$ so this satisfies the first statement.

    Now take any $r \in R$ and let $a_1b_1 + \cdots + a_nb_n \in \ideal{ab}$.
    \begin{itemize}
        \item $r(a_1b_1 + \cdots + a_nb_n) = (ra_1)b_1 + \cdots + (ra_n)b_n$. Note $ra_i \in \ideal{a}$ since $\ideal{a}$ is an ideal, and $b_i \in \ideal{b}$. Hence $(ra_1)b_1 + \cdots + (ra_n)b_n \in \ideal{ab}$.
        \item $(a_1b_1 + \cdots + a_nb_n)r = a_1(b_1r) + \cdots + a_n(b_nr)$. Note $b_ir \in \ideal{b}$ since $\ideal{b}$ is an ideal, and $a_i \in \ideal{a}$. Hence $a_1(b_1r) + \cdots + a_n(b_nr) \in \ideal{ab}$.
    \end{itemize}

    Therefore $\ideal{ab}$ is an ideal by the test for ideal (\myref{thrm-test-for-ideal}).
\end{proof}

\begin{exercise}
    Let $R$ be a ring and let $\ideal{a}$ and $\ideal{b}$ be ideals of $R$. Prove that $\ideal{a} \cap \ideal{b}$ is an ideal of $R$.
\end{exercise}

\section{Principal Ideals}
We return to the original definition for ideals by Dedekind, which is what we now call principal ideals. They can be thought of as ideals that are `generated' by one element from the ring.
\begin{definition}
    Let $R$ be a commutative ring with identity and let $a$ be an element from $R$. Then the \textbf{principal ideal generated by $a$}\index{ideal!principal} is
    \[
        \princ{a} = aR = \{ar \vert r \in R\}.
    \]
\end{definition}
\begin{remark}
    Most authors (e.g. \cite[p.~123, Definition III.2.4]{hungerford_1980}, \cite[\S 158]{clark_1984}, \cite[p.~251]{dummit_foote_2004}) choose to denote the principal ideal generated by $a$ by $(a)$. However, to avoid ambiguity with normal parentheses, we follow \cite[p.~250, Example 3]{gallian_2016} and choose to denote it by $\princ{a}$ instead. Although one may be concerned with this being confused with the cyclic (sub)group generated by $a$, the meaning of this notation should be clear from context.
\end{remark}
\begin{remark}[see {\cite[p.~251]{dummit_foote_2004}}]
    If $R$ is a non-commutative ring, or if $R$ does not contain an identity, then the situation becomes more complicated. In particular,
    \begin{align*}
        \princ{a} &= \text{Smallest ideal of } R \text{ containing }a\\
        &= \bigcap_{\substack{\text{all ideals }I \\ \text{with } a \in I}}I
    \end{align*}
    
    As such, we restrict the discussion of principal ideals to commutative rings with identity.
\end{remark}

\begin{proposition}
    All principal ideals are ideals.
\end{proposition}
\begin{proof}
    See \myref{exercise-principal-ideal-is-ideal} (later).
\end{proof}

\begin{proposition}\label{prop-principal-ideals-equal-iff-associates}
    Let $D$ be an integral domain, and let $a,b\in D$. Then $\princ{a} = \princ{b}$ if and only if $a = bu$ for some unit $u$ in $D$.
\end{proposition}
\begin{proof}
    See \myref{problem-principal-ideals-equal-iff-associates} (later).
\end{proof}

\begin{example}
    The ideal $n\Z$ is a principal ideal of $\Z$ since
    \[
        n\Z = \{\dots, -2n, -n, 0, n, 2n, \dots\} = \princ{n}.
    \]
\end{example}
\begin{remark}
    In the context of $\Z$, we usually write the principal ideal $\princ{n}$ as $n\Z$.
\end{remark}

\begin{exercise}\label{exercise-principal-ideal-is-ideal}
    Show that all principal ideals are ideals.
\end{exercise}

\begin{exercise}\label{exercise-trivial-ideal-and-whole-ring-are-principal-ideals}
    Let $R$ be a commutative ring with identity. Show that the following ideals are principal in $R$.
    \begin{partquestions}{\alph*}
        \item The trivial ideal.
        \item The ring itself.
    \end{partquestions}
\end{exercise}

We look at two related definitions to the principal ideal before we look at an interesting proposition.
\begin{definition}
    A commutative ring where every ideal is principal is called a \textbf{principal ideal ring}\index{principal ideal ring}.
\end{definition}
\begin{definition}
    A principal ideal ring that is an integral domain is called a \textbf{principal ideal domain}\index{principal ideal domain}, or \textbf{PID}\index{PID}.
\end{definition}

We look at one example of a PID.
\begin{proposition}\label{prop-Z-is-PID}
    $\Z$ is a PID.
\end{proposition}
\begin{proof}
    We note that $\Z$ is an integral domain, so all that remains to be proven is that every ideal of $\Z$ is principal.

    Let $I$ be an ideal of $\Z$, and suppose $I$ is not the trivial ideal. This means that $I$ must contain both positive and negative numbers.
    
    Set $n = \min(I \cap \mathbb{N})$, i.e. $n$ is the smallest positive integer that is in $I$. Observe that since $n \in I$ we have $mn \in I$ for all $m \in \Z$. Therefore
    \[
        \princ{n} = \{mn \vert m \in \Z\} = n\Z \subseteq I.
    \]

    Now we want to show that $I \subseteq \princ{n}$. Suppose $a \in I$. By Euclid's division lemma (\myref{lemma-euclid-division}), we see that
    \[
        a = nq + r \text{ with } q,r\in\Z \text{ and } 0 \leq r < n,
    \]
    meaning that $r = a - nq \in I$. Note that $a$ and $n$ are in $I$, so $nq \in I$ and therefore $a - nq = r \in I$. If $r$ is positive, then we have found a smaller positive integer than $n$ in $I$, a contradiction since $n$ is the smallest positive integer in $I$. Hence $r = 0$, which means $a = nq \in \princ{n}$. Therefore $I \subseteq \princ{n} = n\Z$.

    As $n\Z \subseteq I$ and $I \subseteq n\Z$, therefore $I = n\Z$, meaning that any arbitrary ideal in $\Z$ is principal. Hence $\Z$ is a PID.
\end{proof}

\section{Prime and Maximal Ideals}
Let's first look at the definition of a prime ideal.
\begin{definition}
    Let $R$ be a commutative ring with identity. An proper ideal $P$ of $R$ is called a \textbf{prime ideal}\index{ideal!prime}\index{prime!ideal} if whenever $ab \in P$ we have $a \in P$ or $b \in P$.
\end{definition}
This definition may seem weird, but is completely natural when we consider the properties of primes in the positive integers. Recall that Euclid's Lemma (\myref{corollary-euclid}) tells us that if a prime $p$ divides $ab$, then either $p$ divides $a$ or $p$ divides $b$. Similarly, if the product $ab$ belongs within the prime ideal $P$, then either $a$ belongs in $P$ or $b$ belongs in $P$. However, prime ideals does not necessarily imply unique `factorization' like what occurs in the integers via the Fundamental Theorem of Arithmetic (\myref{thrm-fundamental-theorem-of-arithmetic}); we will explore the uniqueness criteria in a later chapter.

We look at the connection between this definition of a prime ideal and primes in the integers.
\begin{proposition}\label{prop-ideals-of-Z}
    The prime ideals of $\Z$ are the trivial ideal and $p\Z$, where $p$ is a prime number.
\end{proposition}
\begin{proof}
    First let's assume the ideal in question is the trivial ideal. Suppose $ab \in \{0\}$, meaning $ab = 0$. As $\Z$ is an integral domain, this means either $a = 0$ or $b = 0$, which therefore means either $a \in \{0\}$ or $b \in \{0\}$. Hence the trivial ideal is a prime ideal in $\Z$.

    Now suppose we have a non-trivial prime ideal $P$ of $\Z$. As $\Z$ is a PID, we may write $P = n\Z$ where $n \geq 2$ (note that if $n = 1$ we will get $P = \Z$). Furthermore write $n = ab$ where $1 \leq a,b \leq n$. Since $ab = n \in P$, therefore $a \in P$ or $b \in P$. Without loss of generality assume $a \in P$. We note $P = n\Z = \{\dots, -2n, -n, 0, n, 2n, \dots\}$ and $1 \leq a \leq n$, so we must have $a = n$. Therefore $a = n$ and $b = 1$, which shows that $n$ is prime. Hence, the prime ideal $P = n\Z$ where $n$ is prime.
\end{proof}
\begin{exercise}
    Is the principal ideal $\princ{2} = \{0, 2, 4, 6\}$ prime in $\Z_8$?
\end{exercise}

We now look at the definition of a maximal ideal.
\begin{definition}
    Let $R$ be a commutative ring with identity. An ideal $M \subset R$ is called a \textbf{maximal ideal}\index{ideal!maximal} if whenever an ideal $I$ is such that $M \subseteq I \subseteq R$ we have $I = M$ or $I = R$.
\end{definition}
\begin{remark}
    This means that the only ideal that properly contains a maximal ideal is the entire ring.
\end{remark}
\begin{remark}
    To show maximality of an ideal, we usually assume that $I \neq M$ and show that $I$ has to be equal to $R$.
\end{remark}
\begin{example}
    We show that $M = \princ{2} = \{0, 2, 4, \dots, 32, 34\}$ is a maximal ideal of $\Z_{36}$. Suppose we have an ideal $I$ of $\Z_{36}$ such that $M \subset I \subseteq \Z_{36}$. We show that $I = \Z_{36}$.

    Take an $n \in I \setminus M$. Then Euclid's division lemma (\myref{lemma-euclid-division}) tells us that we may write
    \[
        n = 2q + r \text{ with } 0 \leq r < 2
    \]
    Now since $n \notin M$, therefore $n$ is not a multiple of 2. Thus there must be a remainder, i.e. $1 \leq r < 2$, meaning that $r = 1$. Furthermore, one sees that $r = 1 = n - 2q$, and because $n \in I$ and $2q \in I$ (since $2q \in M$ and $M \subset I$), therefore $1 \in I$. Hence, by \myref{prop-ideal-contains-unit-iff-ideal-is-whole-ring}, $I = \Z_{36}$, which means that $\princ{2}$ is a maximal ideal of $\Z_{36}$.
\end{example}
\begin{exercise}
    What conditions must be placed on the positive integer $n$ so that $n\Z$ is a maximal ideal in $\Z$?
\end{exercise}

We note that there are much easier ways to determine whether an ideal is prime, maximal, or both. We state two important results here.

\begin{theorem}\label{thrm-prime-ideal-iff-quotient-ring-is-integral-domain}
    Let $R$ be a commutative ring with identity, and let $P$ be an ideal of $R$. Then $P$ is prime if and only if $R/P$ is an integral domain.
\end{theorem}
\begin{proof}
    We first prove the forward direction. Suppose $P$ is prime and $a+P, b+P \in R/P$ such that $(a+P)(b+P) = 0+P$. This means that $ab + P = 0 + P$. By Coset Equality (\myref{lemma-coset-equality}), this therefore means that $ab \in P$. Now as $P$ is prime, this thus means that either $a \in P$ or $b \in P$. Hence, $a + P = 0 + P$ or $b + P = 0 + P$. So we have shown that if two elements in $R/P$ multiply to zero, then one of the elements must be zero, meaning that $R/P$ has no zero divisors.
    
    Now clearly $(a+P)(b+P) = ab + P = ba + P = (b+P)(a+P)$ since $R$ is commutative, so $R/P$ is commutative. Furthermore one sees that $1 + P$ is the identity in $R/P$. Therefore, $R/P$ is an integral domain.

    Now we prove the reverse direction. Suppose $R/P$ is an integral domain. Take $a,b \in R$ such that $ab \in P$. This means that $ab + P = (a+P)(b+P) = 0 + P$. Therefore $a+P = 0 + P$ or $b + P = 0 + P$ since $R/P$ is an integral domain with no zero divisors. Hence $a \in P$ or $b \in P$ by Coset Equality.
    
    Now if $P = R$ then $R/P = \{0 + P\}$ which is the trivial quotient ring. But by \myref{exercise-trivial-ring-is-not-an-integral-domain} it is not an integral domain. Therefore $P \neq R$, meaning that $P$ is a prime ideal.
\end{proof}
\begin{exercise}
    Is $\princ{3 - i}$ a prime ideal in the Gaussian integers?
\end{exercise}

\begin{theorem}\label{thrm-maximal-ideal-iff-quotient-ring-is-field}
    Let $R$ be a commutative ring with identity, and let $M$ be an ideal of $R$. Then $M$ is maximal if and only if $R/M$ is a field.
\end{theorem}
\begin{proof}
    We first work in the forward direction. Suppose $M$ is a maximal ideal, and take $a + M \in R/M$ such that $a \neq 0$ and $a \notin M$ (which means $a + M$ is non-zero). Observe
    \[
        M \subset \princ{a} + M \subseteq R,
    \]
    with strict subset achieved because $a \notin M$, and the sum of ideals is an ideal. As $M$ is maximal, this means that $\princ{a} + M = R$. Therefore $1 \in \princ{a} + M$; write $1 = ar + m$ for some $r \in R$ and $m \in M$. Note $1-ar = m \in M$. Coset Equality (\myref{lemma-coset-equality}) therefore tells us that $ar + M = 1 + M$. Note $ar + M = (a+M)(r+M) = 1 + M$, so $(a+M)^{-1} = r+M$, meaning that any non-zero element in $R/M$ has an inverse. A similar argument to what is shown in \myref{thrm-prime-ideal-iff-quotient-ring-is-integral-domain} shows that $R/M$ is commutative with identity $1 + M$, so $R/M$ is a field.

    Now we work in the reverse direction; assume that $R/M$ is a field. Suppose $I$ is an ideal such that $M \subset I \subseteq R$. We want to show that $I = R$.

    Take $a \in I \setminus M$ with $a \neq 0$, so $a + M \in R/M$ is non-zero. As $R/M$ is a field, therefore there exists a $b + M \in R/M$ such that $(a+M)(b+M) = 1 + M$, meaning $ab+M = 1+M$. Then Coset Equality tells us that $ab - 1 \in M \subset I$. Since $ab \in I$ and $ab - 1 \in I$, the only way for this to happen is if $1 \in I$. By \myref{prop-ideal-contains-unit-iff-ideal-is-whole-ring} this means $I = R$.

    Finally, if $M = R$, then $R/M = \{0 + M\}$, the trivial ring. But in the trivial ring, the additive and multiplicative inverses is the same element, so $R/M$ is not a field by definition. Hence $M \neq R$, so $M \subset R$, meaning $M$ is maximal.
\end{proof}

\begin{corollary}\label{corollary-all-maximal-ideals-are-prime-ideals}
    All maximal ideals are prime ideals.
\end{corollary}
\begin{proof}
    If $M$ is a maximal ideal in a commutative ring with identity $R$, then $R/M$ is a field by \myref{thrm-maximal-ideal-iff-quotient-ring-is-field}. Note that any field is an integral domain by \myref{prop-field-is-integral-domain}. Therefore $R/M$ is an integral domain, meaning $M$ is prime by \myref{thrm-prime-ideal-iff-quotient-ring-is-integral-domain}.
\end{proof}

\begin{corollary}\label{corollary-prime-ideal-is-maximal-in-finite-commutative-ring-with-identity}
    All prime ideals in a finite commutative ring with identity are maximal.
\end{corollary}
\begin{proof}
    See \myref{exercise-prime-ideal-is-maximal-in-finite-commutative-ring-with-identity} (later).
\end{proof}
\begin{exercise}\label{exercise-prime-ideal-is-maximal-in-finite-commutative-ring-with-identity}
    Let $R$ be a finite commutative ring with identity, and let $P$ be a prime ideal in $R$. Show that $P$ is maximal in $R$.
\end{exercise}

\section{The Annihilator and Radical}
To end this chapter, we look at two constructs relating to a ring.
\begin{definition}
    Let $R$ be a commutative ring and $A$ a non-empty subset of $R$. The \textbf{annihilator}\index{annihilator} of $A$ is
    \[
        \Ann{R}{A} = \{r \in R \vert ra = 0 \text{ for all } a \in A\}.
    \]
\end{definition}
\begin{example}
    $\Ann{\Z}{\Z} = \{0\}$.
\end{example}
\begin{proposition}
    Let $R$ be a commutative ring and $A$ a non-empty subset of $R$. Then $\Ann{R}{A}$ is an ideal of $R$.
\end{proposition}
\begin{proof}
    See \myref{exercise-annihilator-is-an-ideal} (later).
\end{proof}

\begin{definition}
    Let $R$ be a commutative ring and $I$ an ideal of $R$. The \textbf{radical}\index{radical} of $I$ is
    \[
        \sqrt I = \{r \in R \vert r^n \in I \text{ for some } n \in \mathbb{N}\}.
    \]
\end{definition}
\begin{definition}
    The radical of the trivial ideal is called the \textbf{nilradical}\index{nilradical} and is given by
    \[
        \Nilr{R} = \sqrt{\{0\}} = \{r \in R \vert r^n = 0 \text{ for some } n \in \mathbb{N}\}.
    \]
    That is, the nilradical of a commutative ring $R$ is the set of all nilpotents in $R$.
\end{definition}
\begin{example}
    We find the nilradical of the ring $\Z_{12}$.
    \begin{align*}
        \Nilr{\Z_{12}} &= \{r \in \Z_{12} \vert r^n = 0 \text{ for some } n \in \mathbb{N}\}\\
        &= \{0, 6\}.
    \end{align*}
\end{example}
\begin{example}
    In the integers, we show that $\sqrt{4\Z} = 2\Z$.

    Suppose $r \in \sqrt{4\Z}$, so $r^n = 4m$ for some $m \in \Z$ and $n \in \mathbb{N}$. Clearly $r^n = 4m = 2(2m) \in 2\Z$, so $r \in 2\Z$. Thus this means that $\sqrt{4\Z} \subseteq 2\Z$.

    On another hand, suppose $r \in 2\Z$, meaning $r = 2m$ for some $m \in \Z$. Note that $r^2 = (2m)^2 = 4m^2 \in 4\Z$, so $r \in \sqrt{4\Z}$. Thus $2\Z \subseteq \sqrt{4\Z}$.

    Therefore $\sqrt{4\Z} = 2\Z$.
\end{example}

\newpage

\begin{proposition}
    Let $R$ be a commutative ring and $I$ be an ideal of $R$. Then $\sqrt{I}$ is an ideal of $R$.
\end{proposition}
\begin{proof}
    We again consider the test for ideal (\myref{thrm-test-for-ideal}) to prove this. Note that $0 \in \sqrt{I}$ since $0^1 = 0 \in I$, so $\sqrt{I}$ is non-empty.

    First let $r, s \in \sqrt{I}$, meaning $r^m \in I$ and $s^n \in I$ for some positive integers $m$ and $n$. Without loss of generality, assume $m \geq n$. Consider $(r-s)^{mn}$. The Binomial Theorem (\myref{thrm-binomial}) tells us that
    \[
        (r-s)^{mn} = \sum_{k=0}^{mn}(-1)^k{mn \choose k}r^{mn-k}s^k.
    \]
    Observe that at any one point, either $mn - k \geq m$ or $k \geq m$, so at any point either $r^{mn-k} \in I$ or $s^k \in I$, meaning that $(-1)^k{mn \choose k}r^{mn-k} \in I$. Therefore, $(r-s)^{mn} \in I$, which in turn means $r-s \in \sqrt{I}$.

    Next suppose $x \in R$ and $r \in \sqrt{I}$, meaning $r^n \in I$ for some positive integer $n$. Note that $(rx)^n = (xr)^n = x^nr^n \in I$ since $R$ is commutative and $r^n \in I$. Therefore $rx, xr \in \sqrt{I}$.

    By the test for ideal, we have $\sqrt{I}$ is an ideal of $R$.
\end{proof}

\begin{exercise}\label{exercise-annihilator-is-an-ideal}
    Let $R$ be a commutative ring and $A$ a non-empty subset of $R$. Show that $\Ann{R}{A}$ is an ideal of $R$.
\end{exercise}

\newpage

\section{Problems}
\begin{problem}
    Find $\Ann{\Z_{36}}{\{15\}}$.
\end{problem}

\begin{problem}
    Let $S = \{a + 2bi \vert a, b \in \Z\}$. Show that $S$ is a subring of $\Z[i]$ but not an ideal of $\Z[i]$.
\end{problem}

\begin{problem}
    Consider
    \[
        I = \left\{\begin{pmatrix}2a&2b\\2c&2d\end{pmatrix} \vert a,b,c,d \in \Z\right\}.
    \]
    Show that $I$ is an ideal of $\Mn{2}{\Z}$.
\end{problem}

\begin{problem}\label{problem-ring-is-field-iff-no-proper-ideals}
    Let $R$ be a commutative ring with identity and $I$ be an ideal of $R$.
    \begin{partquestions}{\alph*}
        \item Prove that if $R$ is a field then $I$ is either the trivial ring or $R$ (i.e., $R$ has no proper ideals). Hence prove that any field is a PID.
        \item Prove that if $R$ has no proper ideals, then $R$ is a field.
    \end{partquestions}
\end{problem}

\begin{problem}
    Let $R$ be a commutative ring and let $I$ and $J$ be ideals in $R$. Prove the following statements.
    \begin{partquestions}{\alph*}
        \item $\sqrt{\sqrt{I}} = \sqrt{I}$
        \item $\sqrt{I \cap J} = \sqrt{I} \cap \sqrt{J}$
    \end{partquestions}
\end{problem}

\begin{problem}
    Let $m$ and $n$ be positive integers, and let $d = \gcd(m,n)$ and $l = \lcm(m,n)$. Prove the following.
    \begin{partquestions}{\alph*}
        \item $m\Z \cap n\Z = l\Z$
        \item $m\Z + n\Z = d\Z$
    \end{partquestions}
\end{problem}

\begin{problem}
    Let $R$ be a commutative ring. Prove that $R / \Nilr{R}$ has no non-zero nilpotents.
\end{problem}

\begin{problem}
    Prove that every non-trivial prime ideal is a maximal ideal in a PID.
\end{problem}

\begin{problem}\label{problem-principal-ideals-equal-iff-associates}
    Prove \myref{prop-principal-ideals-equal-iff-associates}.
\end{problem}

\chapter{Ring Homomorphisms and Isomorphisms}
Just like with groups, rings too have homomorphisms and isomorphisms, although they are defined slightly differently than in groups. Like how group homomorphisms preserve some structure between the two groups, ring homomorphisms and isomorphisms also preserve structure between rings.

\section{Ring Homomorphisms, Endomorphisms, and Isomorphisms}
\begin{definition}
    Let $(R_1, +, \cdot)$ and $(R_2, \oplus, \otimes)$ be rings. A function $\phi: R_1 \to R_2$ is a \textbf{ring homomorphism}\index{ring homomorphism} if for all $a, b \in R_1$,
    \begin{align*}
        \phi(a+b) &= \phi(a) \oplus \phi(b), \text{ and}\\
        \phi(a\cdot b) &= \phi(a)\otimes\phi(b).
    \end{align*}
\end{definition}
\begin{remark}
    Like with group homomorphisms, we usually suppress the multiplication operation and use ``$+$'' for both addition operations. That is, the ring homomorphism conditions become
    \begin{align*}
        \phi(a+b) &= \phi(a) + \phi(b) \text{ and}\\
        \phi(ab) &= \phi(a)\phi(b).
    \end{align*}
\end{remark}

\begin{example}
    We show that the map $\phi: \Z \to \Z/n\Z, x \mapsto x + n\Z$ is a ring homomorphism.
     
    \begin{proof}
        Let $a, b \in \Z$. Note
        \begin{align*}
            \phi(a+b) &= (a+b) + n\Z\\
            &= (a + n\Z) + (b + n\Z) & (\text{Definition of coset addition})\\
            &=\phi(a)+\phi(b)
        \end{align*}
        and
        \begin{align*}
            \phi(ab) &= (ab) + n\Z\\
            &= (a + n\Z)(b + n\Z) & (\text{Definition of coset multiplication})\\
            &= \phi(a)\phi(b)
        \end{align*}
        so $\phi$ is a homomorphism.
    \end{proof}
\end{example}

\begin{example}
    Consider
    \[
        R = \left\{\begin{pmatrix}a&b\\0&c\end{pmatrix}\vert a,b,c\in\Z\right\}.
    \]
    Let $\phi: R \to \Z^2, \begin{pmatrix}a&b\\0&c\end{pmatrix} \mapsto (a,c)$. We show that $\phi$ is a ring homomorphism.

    \begin{proof}
        We see that
        \begin{align*}
            \phi\left(\begin{pmatrix}a&b\\0&c\end{pmatrix} + \begin{pmatrix}x&y\\0&z\end{pmatrix}\right) &= \phi\left(\begin{pmatrix}a+x&b+y\\0&c+z\end{pmatrix}\right)\\
            &= (a+x,c+z)\\
            &= (a,c) + (x,z)\\
            &= \phi\left(\begin{pmatrix}a&b\\0&c\end{pmatrix}\right) + \phi\left(\begin{pmatrix}x&y\\0&z\end{pmatrix}\right)
        \end{align*}
        and also
        \begin{align*}
            \phi\left(\begin{pmatrix}a&b\\0&c\end{pmatrix}\begin{pmatrix}x&y\\0&z\end{pmatrix}\right) &= \phi\left(\begin{pmatrix}ax&ay+bz\\0&cz\end{pmatrix}\right)\\
            &= (ax, cz)\\
            &= (a,c)(x,z)\\
            &= \phi\left(\begin{pmatrix}a&b\\0&c\end{pmatrix}\right)\phi\left(\begin{pmatrix}x&y\\0&z\end{pmatrix}\right)
        \end{align*}
        so $\phi$ is a homomorphism.
    \end{proof}
\end{example}
\begin{exercise}
    Let $\phi: \Mn{2}{\Z} \to \Z$ be such that
    \[
        \phi\left(\begin{pmatrix}a&b\\c&d\end{pmatrix}\right) = a+d.
    \]
    Is $\phi$ a ring homomorphism?
\end{exercise}
\begin{exercise}
    Let $R$ and $S$ be rings with additive identities $0_R$ and $0_S$ respectively. Show that the \textbf{trivial homomorphism}\index{trivial homomorphism} $\phi: R \to S, r \mapsto 0_S$ is, indeed, a ring homomorphism.
\end{exercise}

An endomorphism is a specific type of homomorphism.
\begin{definition}
    A \textbf{ring endomorphism}\index{ring endomorphism} of a ring $R$ is a homomorphism $\phi: R \to R$.
\end{definition}
\begin{example}
    Let $R$ be a commutative ring with prime characteristic $p$. The \textbf{Frobenius endomorphism}\index{Frobenius endomorphism} $\phi: R \to R$ is such that $\phi(r) = r^p$. We show that $\phi$ is a ring endomorphism.
    
    \begin{proof}
        Note that
        \begin{align*}
            \phi(a+b) &= (a+b)^p\\
            &= a^p + pa^{p-1}b + {p \choose 2}a^{p-2}b^2 + \cdots + pab^{p-1} + b^p.
        \end{align*}
        Note that the binomial coefficients ${p \choose k}$ where $1 \leq k \leq p-1$ are all multiples of $p$ (\myref{prop-binomial-coefficient-multiple-of-p}). As the characteristic of the ring $R$ is $p$, thus $px = 0$ for any $x \in R$. Therefore,
        \begin{align*}
            \phi(a+b) = &a^p + pa^{p-1}b + {p \choose 2}a^{p-2}b^2 + \cdots + pab^{p-1} + b^p\\
            &= a^p + 0 + 0 + \cdots + 0 + b^p\\
            &= a^p + b^p\\
            &=\phi(a) + \phi(b)
        \end{align*}

        Also,
        \[
            \phi(ab) = (ab)^p = a^pb^p = \phi(a)\phi(b).
        \]
        Therefore $\phi$ is a ring endomorphism.
    \end{proof}
\end{example}

\begin{exercise}
    Let $R$ be a ring. Show that the \textbf{identity homomorphism}\index{identity homomorphism} $\id: R \to R, r \mapsto r$ is a ring endomorphism.
\end{exercise}

Similar to group isomorphisms, rings too have an analogous `isomorphism' definition.
\begin{definition}
    A \textbf{ring isomorphism}\index{ring isomorphism} is a bijective ring homomorphism.
\end{definition}
Similar to groups, if two rings $R$ and $R'$ are isomorphic, then we write $R \cong R'$.

\begin{example}
    We show that $\Z_n \cong \Z/n\Z$.

    \begin{proof}
        Consider the map $\phi:\Z_n \to \Z/n\Z$ where $m \mapsto m + n\Z$. We show that $\phi$ is an isomorphism.
        \begin{itemize}
            \item \textbf{Homomorphism}:
            \[
                \phi(a+b) = (a+b) + n\Z = (a + n\Z) + (b + n\Z) = \phi(a) + \phi(b)
            \]
            and
            \[
                \phi(ab) = (ab) + n\Z = (a+n\Z)(b+n\Z) = \phi(a)\phi(b)
            \]

            \item \textbf{Injective}: Suppose $a, b \in \Z_n$ such that $\phi(a) = \phi(b)$. This means $a + n\Z = b + n\Z$. Now note that $0 \leq a,b < n$ so we have $a = b$.
            
            \item \textbf{Surjective}: Suppose $m + n\Z \in \Z/n\Z$. Applying Euclid's division lemma (\myref{lemma-euclid-division}) on $m$ we have
            \[
                m = nq + r
            \]
            with $0 \leq r < n$. One sees that
            \begin{align*}
                \phi(r) &= r + n\Z\\
                &= r + (nq + n\Z)\\
                &= (r + nq) + n\Z\\
                &= m + n\Z
            \end{align*}
            so $m + n\Z$ has a pre-image of $r$ in $\Z_n$.
        \end{itemize}
        Since $\phi$ is a bijective ring homomorphism, this $\phi$ is an isomorphism, meaning $\Z_n \cong \Z/n\Z$ as rings.
    \end{proof}
\end{example}
\begin{example}
    We show that $\Z \not\cong 2\Z$ as rings.

    \begin{proof}
        Suppose $\phi: \Z \to 2\Z$ is a ring isomorphism.

        Set $a = \phi(1) = 2\Z$. Note that
        \[
            a = \phi(1) = \phi(1\times1) = (\phi(1))^2 = a^2
        \]
        so $a^2 = a$, which means $a = 0$ or $a = 1$. But as $a \in 2\Z$, thus $a \neq 1$ which means $a = 0$.

        Now notice for any $n \in \Z$ we have
        \begin{align*}
            \phi(n) &= \phi(n1)\\
            &= \phi(n)\phi(1)\\
            &= \phi(n) \times 0\\
            &= 0.
        \end{align*}
        Thus one sees that $\phi(0) = \phi(1) = 0$ which means $\phi$ is not injective, a contradiction.
    \end{proof}
\end{example}
\begin{exercise}\label{exercise-identity-homomorphism-is-an-isomorphism}
    Show that the identity homomorphism is an isomorphism.
\end{exercise}

\section{Properties of Ring Homomorphisms}
For the following, let $R_1$ and $R_2$ be rings with additive identities $0_1$ and $0_2$ respectively. Also let $\phi: R_1 \to R_2$ be a ring homomorphism.

\begin{proposition}\label{prop-ring-image-of-additive-identity-is-additive-identity}
    $\phi(0_1) = 0_2$.
\end{proposition}
\begin{proof}
    See \myref{exercise-ring-image-of-identity-is-identity} (later).
\end{proof}

\begin{proposition}
    If $R_1$ and $R_2$ are division rings, then $\phi(1_1) = 1_2$ where $1_1$ and $1_2$ are the multiplicative identities of $R_1$ and $R_2$ respectively.
\end{proposition}
\begin{proof}
    See \myref{exercise-ring-image-of-identity-is-identity} (later).
\end{proof}

\begin{proposition}
    $\phi(-x) = -\phi(x)$ for all $x \in R_1$.
\end{proposition}
\begin{proof}
    See \myref{exercise-ring-image-of-inverse-is-inverse} (later).
\end{proof}

\begin{proposition}
    If $R_1$ and $R_2$ are division rings, then $\phi(x^{-1}) = (\phi(x))^{-1}$ for all $x \in R_1$. 
\end{proposition}
\begin{proof}
    See \myref{exercise-ring-image-of-inverse-is-inverse} (later).
\end{proof}

\begin{proposition}\label{prop-homomorphism-on-subring-is-subring}
    If $S$ is a subring of $R_1$, then
    \[
        \phi(S) = \{\phi(s) | s \in S\}
    \]
    is a subring of $R_2$.
\end{proposition}
\begin{proof}
    Let $S$ be a subring of $R_1$. Take $a, b \in \phi(S)$, which means that there exist $s_a$ and $s_b$ such that $\phi(s_a) = a$ and $\phi(s_b) = b$.
    \begin{itemize}
        \item We show that $(\phi(S), +) \leq (R_2, +)$.
        \begin{itemize}
            \item $\phi(S) \neq \emptyset$ since $\phi(0_1) = 0_2 \in \phi(S)$.
            \item $a - b = \phi(s_a) - \phi(s_b) = \phi(s_a-s_b) \in \phi(S)$.
        \end{itemize}

        \item Now we show $ab \in \phi(S)$.
        \[
            ab = \phi(s_a)\phi(s_b) = \phi(s_as_b) \in \phi(S).
        \]
    \end{itemize}
    Therefore $\phi(S)$ is a subring of $R_2$.
\end{proof}

\begin{proposition}
    If $\phi$ is surjective and $I$ is an ideal of $R_1$, then $\phi(I)$ is an ideal of $R_2$.
\end{proposition}
\begin{proof}
    From previous proposition $\phi(I)$ is a subring of $R_2$. We just need to show that $\phi(I)$ is an ideal of $R_2$.

    Take $a \in \phi(I)$ and $r_2 \in R_2$. As $\phi$ is surjective, we can find a $r_1 \in R_1$ such that $\phi(r_1) = r_2$. Also, let $a = \phi(i)$ for an $i \in I$.

    Note
    \begin{align*}
        ar_2 = \phi(i)\phi(r_1) = \phi(\underbrace{ir_1}_{\text{In }I}) \in \phi(I)\\
        r_2a = \phi(r_1)\phi(i) = \phi(\underbrace{r_1i}_{\text{In }I}) \in \phi(I)
    \end{align*}
    so $\phi(I)$ is an ideal of $R_2$.
\end{proof}

\begin{proposition}\label{prop-inverse-homomorphism-on-ideal-is-ideal}
    Let $J$ be an ideal of $R_2$. Then
    \[
        \phi^{-1}(J) = \{r \in R_1 \vert \phi(r) \in J\}
    \]
    is an ideal of $R_1$.
\end{proposition}
\begin{proof}
    Suppose $J$ is an ideal of $R_2$. We consider the test for ideal (\myref{thrm-test-for-ideal}) to show $\phi^{-1}(J)$ is an ideal of $R_1$.
    
    One sees that $\phi^{-1}(J) \neq \emptyset$ since $\phi(0_1) = 0_2 \in J$, so $0_1 \in \phi^{-1}(J)$.

    Let $a, b \in \phi^{-1}(J)$, so $\phi(a), \phi(b) \in J$. Note that
    \[
        \phi(a-b) = \phi(a) - \phi(b) \in J
    \]
    so $a-b \in \phi^{-1}(J)$ for all $a,b \in J$.

    Let $r \in R_1$ and $a \in \phi^{-1}(J)$. Note that $\phi(a) \in J$ and $\phi(r) \in R_2$, so
    \begin{align*}
        \underbrace{\phi(a)}_{\text{In }J}\underbrace{\phi(r)}_{\text{In }R_2} &\in J & (J\text{ is an ideal, so } rj \in J)\\
        \phi(r)\phi(a) &\in J & (\text{since } jr \in J).
    \end{align*}
    Note $\phi(a)\phi(r) = \phi(ar) \in J$, so $ar \in \phi^{-1}(J)$, and similarly we have $\phi(r)\phi(a) = \phi(ra) \in J$, so $ra \in \phi^{-1}(J)$.

    Therefore, by the test for ideal, is an ideal of $R_1$.
\end{proof}

\begin{exercise}\label{exercise-ring-image-of-identity-is-identity}
    Let $R_1$ and $R_2$ be rings, and $\phi: R_1 \to R_2$ be a ring homomorphism.
    \begin{partquestions}{\alph*}
        \item Show $\phi(0_1) = 0_2$, where $0_1$ and $0_2$ is the additive identity of $R_1$ and $R_2$ respectively.
        \item If $R_1$ and $R_2$ are division rings, then show $\phi(1_1) = 1_2$, where $1_1$ and $1_2$ is the multiplicative identity of $R_1$ and $R_2$ respectively.
    \end{partquestions}
\end{exercise}

\begin{exercise}\label{exercise-ring-image-of-inverse-is-inverse}
    Let $R_1$ and $R_2$ be rings, $x \in R_1$, and $\phi: R_1 \to R_2$ be a ring homomorphism.
    \begin{partquestions}{\alph*}
        \item Show that $\phi(-x) = -\phi(x)$.
        \item If $R_1$ and $R_2$ are division rings, show that $\phi(x^{-1}) = (\phi(x))^{-1}$.
    \end{partquestions}
\end{exercise}

\section{Image and Kernel}
Similar to groups, ring homomorphisms too have a image and kernel.
\begin{definition}
    The \textbf{image}\index{image} of a ring homomorphism $\phi: R_1 \to R_2$ is
    \[
        \im\phi = \{\phi(r) \vert r \in R_1\}.
    \]
\end{definition}
\begin{definition}
    The \textbf{kernel}\index{kernel} of a ring homomorphism $\phi:R_1 \to R_2$ is
    \[
        \ker\phi = \{r \in R_1 \vert \phi(r) = 0\}.
    \]
\end{definition}

\begin{example}\label{example-homomorphism-on-upper-triangle-matrices}
    Consider the ring
    \[
        R = \left\{\begin{pmatrix}a&b\\0&c\end{pmatrix}\vert a,b,c\in\Z\right\}.
    \]
    and the homomorphism $\phi: R \to \Z^2, \begin{pmatrix}a&b\\0&c\end{pmatrix} \mapsto (a,c)$.

    We note that $\phi$ is surjective; for any $(x,y)\in\Z^2$, we note that $\phi\left(\begin{pmatrix}x&0\\0&y\end{pmatrix}\right) = (x,y)$ so $(x,y)$ has a pre-image in $R$. Therefore $\im \phi = \Z^2$.

    We now find the kernel of $\phi$.
    \begin{align*}
        \ker\phi &= \left\{\begin{pmatrix}a&b\\0&c\end{pmatrix} \in R \vert \phi\left(\begin{pmatrix}a&b\\0&c\end{pmatrix}\right) = (0,0)\right\}\\
        &= \left\{\begin{pmatrix}a&b\\0&c\end{pmatrix} \in R \vert (a,c) = (0,0)\right\}\\
        &= \left\{\begin{pmatrix}0&n\\0&0\end{pmatrix} \vert n \in \Z\right\}.
    \end{align*}
\end{example}

We look at some results regarding the image and kernel of a ring homomorphism. These results may look familiar to those who read part I.
\begin{proposition}\label{prop-image-is-a-subring}
    Let $R_1$ and $R_2$ be rings, and let $\phi: R_1 \to R_2$ be a ring homomorphism. Then $\im\phi$ is a subring of $R_2$.
\end{proposition}
\begin{proof}
    Note that $\im\phi = \phi(R_1)$ is a subring of $R_2$ by \myref{prop-homomorphism-on-subring-is-subring}.
\end{proof}

\begin{proposition}\label{prop-kernel-is-an-ideal}
    Let $R_1$ and $R_2$ be rings, and let $\phi: R_1 \to R_2$ be a ring homomorphism. Then $\ker\phi$ is an ideal of $R_1$.
\end{proposition}
\begin{proof}
    See \myref{exercise-kernel-is-an-ideal} (later).
\end{proof}

\begin{proposition}
    Let $R_1$ and $R_2$ be rings, and let $\phi: R_1 \to R_2$ be a ring homomorphism. Then $\phi$ is injective if and only if $\ker\phi = \{0_1\}$.
\end{proposition}
\begin{proof}
    We first prove the forward direction; suppose $\phi$ is injective and let $a \in \ker\phi$. By definition of the kernel we have $\phi(a) = 0_2$. But by \myref{prop-ring-image-of-additive-identity-is-additive-identity}, we have $\phi(0_1) = 0_2$. Since $\phi$ is injective, therefore $a = 0_1$, meaning $\ker\phi = \{0_1\}$.

    We now prove the reverse direction; suppose $\ker\phi = \{0_1\}$. Now let $a,b \in R_1$ such that $\phi(a) = \phi(b)$. Therefore $\phi(a) - \phi(b) = \phi(a-b) = 0_2$. Therefore $a-b \in \ker\phi$ by definition of the kernel. However $\ker\phi = \{0_1\}$ which means that $a - b = 0_1$. Therefore $a = b$, meaning $\phi$ is injective.
\end{proof}

\begin{exercise}\label{exercise-kernel-is-an-ideal}
    Let $R_1$ and $R_2$ be rings, and let $\phi: R_1 \to R_2$ be a ring homomorphism. Prove that $\ker\phi$ is an ideal of $R_1$.
\end{exercise}

\newpage

\section{The Ring Isomorphism Theorems}
Similar to group theory, there are three main ring isomorphism theorems. However, we will only explicitly prove the first ring isomorphism theorem; the other two will be left as problems.

\begin{theorem}[First Ring Isomorphism Theorem (FRIT)]\label{thrm-ring-isomorphism-1}\index{ring isomorphism theorem!first}\index{FRIT}
    Let $R$ and $R'$ be rings. Let $\phi: R \to R'$ be a ring homomorphism, and let $\pi: R \to R/\ker\phi$ where $r\mapsto r + \ker\phi$ be the natural surjective homomorphism. Then there exists a unique ring isomorphism $\psi: R / \ker\phi \to \im\phi$ such that $\psi\pi = \phi$.
\end{theorem}
\begin{remark}
    Equivalently, the FRIT states that
    \[
        R / \ker\phi \cong \im\phi
    \]
    for any ring homomorphism $\phi$.
\end{remark}

We include the commutativity diagram of the maps stated to aid clarity:
\begin{figure}[h]
    \centering
    \pdfteximgframed[12pt]{0.25\textwidth}{part2/images/ring-homomorphisms/ring-iso-1-commutativity.pdf_tex}
    \caption{Commutativity Diagram for \myreffigures{thrm-ring-isomorphism-1}}
\end{figure}

In the diagram, $\phi$ sends elements from $R$ to $\im\phi$ and $\pi$ sends elements from $R$ to $R/\ker\phi$. Then the map $\psi$ is a unique map that sends elements from $R/\ker\phi$ to the image of $\phi$.

\begin{proof}[Proof (cf. {\cite[p.~302, Factor Theorem For Rings]{cohn_1982}})]
    Let the map $\phi$ be defined such that $\psi(r + \ker\phi) = \phi(r)$. We first show that $\psi$ is a well-defined ring isomorphism.
    \begin{itemize}
        \item \textbf{Well-Defined}: Suppose $a + \ker\phi$ and $b + \ker\phi$ are in $R/\ker\phi$ such that $a + \ker\phi = b+\ker\phi$. This means that $a - b \in \ker\phi$ by Coset Equality (\myref{lemma-coset-equality}). Thus $\phi(a-b) = 0$ by definition of the kernel. Hence $\phi(a) - \phi(b) = 0$ which means $\phi(a) = \phi(b)$. Therefore, we see that
        \[
            \psi(a + \ker\phi) = \phi(a) = \phi(b) = \psi(b + \ker\phi)
        \]
        which means $\psi$ is well-defined.

        \item \textbf{Homomorphism}: We first show that $\psi$ is a ring homomorphism.
        \begin{itemize}
            \item $\boxed{+}$: If $a + \ker\phi$ and $b + \ker\phi$ are in $R/\ker\phi$,
            \begin{align*}
                &\psi((a + \ker\phi)+(b+\ker\phi))\\
                &= \psi((a+b)+\ker\phi)\\
                &= \phi(a+b)\\
                &= \phi(a) + \phi(b)\\
                &= \psi(a + \ker\phi) + \psi(b + \ker\phi).
            \end{align*}
            \item $\boxed{\times}$: If $a + \ker\phi$ and $b + \ker\phi$ are in $R/\ker\phi$,
            \begin{align*}
                \psi((a + \ker\phi)(b+\ker\phi)) &= \psi((ab)+\ker\phi)\\
                &= \phi(ab)\\
                &= \phi(a)\phi(b)\\
                &= \psi(a + \ker\phi)\psi(b + \ker\phi).
            \end{align*}
        \end{itemize}
        Therefore $\psi$ is a ring homomorphism.

        \item \textbf{Injective}: Suppose $\psi(a+\ker\phi) = \psi(b+\ker\phi)$.
        \begin{align*}
            \phi(a) &= \phi(b) & (\text{definition of }\psi)\\
            \phi(a) - \phi(b) &= 0\\
            \phi(a-b) &= 0 & (\phi \text{ is a ring homomorphism})\\
            a - b &\in \ker\phi & (\text{definition of kernel})\\
            a + \ker\phi &= b + \ker\phi & (\text{Coset Equality})
        \end{align*}
        Therefore if $\psi(a+\ker\phi) = \psi(b+\ker\phi)$ then $a+\ker\phi = b+\ker\phi$, which means $\psi$ is injective.

        \item \textbf{Surjective}: Suppose $s \in \im\phi$, so there is an $r \in R$ such that $s = \phi(r)$. Clearly $\psi(r + \ker\phi) = \phi(r) = s$, so $s$ has a pre-image of $r + \ker\phi$, i.e. $\psi$ is surjective.
    \end{itemize}
    Therefore, $\psi$ is a well-defined bijective ring homomorphism, i.e. $\psi$ is a well-defined ring isomorphism.

    We now check that $\psi$ satisfies the requirement that $\psi\pi = \phi$. Let $x \in R$. Note that $\pi(x) = x + \ker\phi$, and
    \[
        \psi\pi(x) = \psi(x + \ker\phi) = \phi(x)
    \]
    for all $x \in R$, so $\psi\pi = \phi$.

    Finally we show that $\psi$ is unique. Suppose $f: R/\ker\phi \to \im\phi$ is an isomorphism satisfying $f\pi=\phi$. Take $x + \ker\phi \in R/\ker\phi$. Note that
    \begin{align*}
        f(x + \ker\phi) &= f(\pi(x))\\
        &= (f\pi)(x)\\
        &= \phi(x)\\
        &= (\psi\pi)(x)\\
        &= \psi(\pi(x))\\
        &= \psi(x + \ker\phi)
    \end{align*}
    for all $x \in R$, meaning that $f = \psi$. Therefore $\psi$ is unique.

    Hence, $\psi$ is a unique ring isomorphism satisfying $\psi\pi = \phi$.
\end{proof}

\begin{example}
    Consider the ring
    \[
        R = \left\{\begin{pmatrix}a&b\\0&c\end{pmatrix}\vert a,b,c\in\Z\right\}
    \]
    and the homomorphism $\phi: R \to \Z^2, \begin{pmatrix}a&b\\0&c\end{pmatrix} \mapsto (a,c)$. We found in \myref{example-homomorphism-on-upper-triangle-matrices} that $\phi$ is surjective (i.e., $\im\phi = \Z^2$) with kernel
    \[
        \left\{\begin{pmatrix}0&n\\0&0\end{pmatrix} \vert n \in \Z\right\}
    \]
    which, for brevity, we shall denote by $I$. Thus we see via the FRIT (\myref{thrm-ring-isomorphism-1}) that
    \[
        R/I \cong \Z^2.
    \]
\end{example}
\begin{exercise}
    Show that $\Z_n \cong \Z/n\Z$ via the ring homomorphism
    \[
        \phi: \Z \to \Z_n, m \mapsto m \mod n
    \]
    and by using the FRIT.
\end{exercise}

We briefly mention the other two main ring isomorphism theorems, although the proof of them will be left as problems. They are much less used than the FRIT, so we make only a passing mention of them.
\begin{theorem}[Second Ring Isomorphism Theorem]\label{thrm-ring-isomorphism-2}\index{ring isomorphism theorem!second}
    Let $R$ be a ring with subring $R$ and ideal $I$. Then
    \begin{enumerate}
        \item $S+I = \{s+i \vert s\in S,\;i\in I\}$ is a subring of $R$;
        \item $S \cap I$ is an ideal of $S$; and
        \item $(S+I)/I \cong S/(S\cap I)$.
    \end{enumerate}
\end{theorem}
\begin{proof}
    See \myref{problem-ring-isomorphism-2} (later).
\end{proof}

\begin{theorem}[Third Ring Isomorphism Theorem]\label{thrm-ring-isomorphism-3}\index{ring isomorphism theorem!third}
    Let $R$ be a ring with ideals $I$ and $J$ such that $I$ is a subset of $J$. Then
    \begin{enumerate}
        \item $J/I$ is an ideal of $R/I$; and
        \item $\frac{R/I}{J/I} \cong R/J$.
    \end{enumerate}
\end{theorem}
\begin{proof}
    See \myref{problem-ring-isomorphism-3} (later).
\end{proof}

\section{How Restrictive are Ring Homomorphisms?}
Although ring homomorphisms appear to be quite general, we explore how restricted they really are when dealing with certain rings.

\begin{example}\label{example-endomorphisms-of-Z}
    We find all ring endomorphisms of $\Z$.

    Let $\phi:\Z\to\Z$ be a ring endomorphism. Set $a = \phi(1)$. Note that
    \[
        a = \phi(1) = \phi(1\times1) = \phi(1)\phi(1) = a^2
    \]
    so $a^2 = a$. Thus $a = 0$ or $a = 1$ in $\Z$.

    If $a = 0$, then for any $n \in \Z$ we have
    \[
        \phi(n) = \phi(1n) = \phi(1)\phi(n) = 0\phi(n) = 0
    \]
    so $\phi(n) = 0$ for all $n \in \Z$, which is the trivial homomorphism.

    Now consider the case that $a = 1$. We claim that $\phi(n) = n$ for all $n \in \Z$.

    We leave the proof that $\phi(n) = n$ for all \textit{positive} integers $n$ for \myref{exercise-homomorphism-maps-n-to-n-if-n-is-positive} (later). Furthermore $\phi(0) = 0$ by the properties of ring homomorphism (specifically \myref{prop-ring-image-of-additive-identity-is-additive-identity}). Finally, note that for any non-negative integer $n$,
    \begin{align*}
        0 = \phi(0) &= \phi(n - n)\\
        &= \phi(n) + \phi(-n)\\
        &= n + \phi(-n)
    \end{align*}
    which means $\phi(-n) = -n$. Therefore $\phi(n) = n$ for all integers $n$.

    Therefore, the only ring endomorphisms $\phi:\Z\to\Z$ are $\phi(n) = 0$ or $\phi(n) = n$ for all integers $n$.
\end{example}
\begin{exercise}\label{exercise-homomorphism-maps-n-to-n-if-n-is-positive}
    In the above example, show that $\phi(n) = n$ for all positive integers $n$.
\end{exercise}
Note that in \myref{example-endomorphisms-of-Z} we started the entire computation with the observation that $\phi(1) = \phi(1)^2$. This means that $\phi(1)$ is an idempotent.
\begin{definition}
    Let $R$ be a ring. Then an element $x$ in $R$ is an \textbf{idempotent}\index{idempotent} if and only if $x^2 = x$.
\end{definition}
\begin{proposition}\label{prop-homomorphism-on-multiplicative-identity-is-idempotent}
    Let $R$ and $R'$ be rings, and let $\phi: R \to R'$ be a ring homomorphism. If $R$ is a ring with identity, then $\phi(1)$ is an idempotent.
\end{proposition}
\begin{proof}
    Note
    \[
        \phi(1) = \phi(1 \times 1) = \phi(1) \times \phi(1) = \left(\phi(1)\right)^2
    \]
    which means $\phi(1)$ is an idempotent.
\end{proof}

In \myref{example-endomorphisms-of-Z} we used the fact that the only idempotents of $\Z$ are 0 and 1. However, this is not true for all rings.

\begin{example}\label{example-homomorphisms-from-Z12-to-Z28}
    We find all ring homomorphisms $\phi: \Z_{12} \to \Z_{28}$. Note that $\Z_{12}$ is a ring with identity.

    By \myref{prop-homomorphism-on-multiplicative-identity-is-idempotent}, $a = \phi(1)$ is an idempotent. However, we cannot just assume that 0 and 1 are the \textit{only} idempotents in $\Z_{28}$; we need to check for them exhaustively.

    By exhaustion, we see that
    \begin{itemize}
        \item $0^2 = 0$;
        \item $1^2 = 1$;
        \item $8^2 = 64 = 2 \times 28 + 8 = 8$; and
        \item $21^2 = 441 = 15 \times 28 + 21 = 21$.
    \end{itemize}
    So the idempotents in $\Z_{28}$ are 0, 1, 8, and 21. This is not enough to narrow down the possible values of $\phi(1)$, so we need to invoke more facts.

    Recall from part I that $|\phi(1)|_+$ divides $|1|_+$ by \myref{exercise-order-of-homomorphism-divides-order}. Therefore $|\phi(1)|_+$ divides 12. Furthermore, \myref{thrm-order-of-element-in-cyclic-group} tells us that the additive order of an element $k$ in the group $(\Z_n, +)$ is $\frac{n}{\gcd(k,n)}$. So we must now exhaust all idempotents in $\Z_{28}$ to check whether it is a valid value of $\phi(1)$.
    \begin{itemize}
        \item $|0|_+ = 1$ which clearly divides 12, so it is a valid value of $\phi(1)$.
        \item $|1|_+ = 28$ which does not divide 12, so it is not a valid value of $\phi(1)$. This is one way that differs from the previous example, where 1 \textit{was} a possible value of $\phi(1)$.
        \item $|8|_+ = \frac{28}{\gcd(8,28)} = \frac{28}4 = 7$ which does not divide 12, so it is not a valid value of $\phi(1)$.
        \item $|21|_+ = \frac{28}{\gcd(21,28)} = \frac{28}7 = 4$ which divides 12, so it is a valid value of $\phi(1)$.
    \end{itemize}
    Hence $a \in \{0, 21\}$, i.e. $\phi(1) = 0$ or $\phi(1) = 21$.

    If $\phi(1) = 0$, then for any $n \in \Z_{12}$,
    \begin{align*}
        \phi(n) &= \phi(\underbrace{1 + 1 + \cdot + 1}_{n \text{ times}})\\
        &= \underbrace{\phi(1) + \phi(1) + \cdot + \phi(1)}_{n \text{ times}}\\
        &= \underbrace{0 + 0 + \cdots + 0}_{n \text{ times}}\\
        &= 0
    \end{align*}
    which means that $\phi(n) = 0$ for all $n \in \Z_{12}$, i.e. $\phi$ is trivial.

    If instead $\phi(1) = 21$, then
    \begin{align*}
        \phi(n) &= \underbrace{\phi(1) + \phi(1) + \cdot + \phi(1)}_{n \text{ times}}\\
        &= \underbrace{21 + 21 + \cdots + 21}_{n \text{ times}}\\
        &= 21n
    \end{align*}
    which means $\phi(n) = 21n$ for all $n \in \Z_{12}$.

    Thus the only homomorphisms $\phi: \Z_{12} \to \Z_{28}$ are $\phi(n) = 0$ and $\phi(n) = 21n$ for all $n \in \Z_{12}$.
\end{example}

\begin{exercise}\label{exercise-homomorphism-over-Q-fixes-elements-of-Q}
    Suppose $R$ and $R'$ are rings such that $\Q$ is a subring of both $R$ and $R'$. Let $\phi: R \to R'$ be a ring homomorphism such that $\phi(1) = 1$. Show that for any $q \in Q$ we have $\phi(q) = q$.
\end{exercise}

\newpage

\section{Problems}
\begin{problem}
    Let $R$ be a ring.
    \begin{partquestions}{\roman*}
        \item Show that $R/\{0\} \cong R$.
        \item Prove that $R$ is an integral domain if and only if $\{0\}$ is a prime ideal.
        \item Prove that $R$ is a field if and only if $\{0\}$ is a maximal ideal.
    \end{partquestions}
\end{problem}

\begin{problem}
    Find all ring endomorphisms of $\Q$.
\end{problem}

\begin{problem}
    Show $\Z^2 \not\cong \Q$.
\end{problem}

\begin{problem}
    Show $\Q[\sqrt2] \not\cong \Q[\sqrt3]$.
\end{problem}

\begin{problem}
    Let
    \[
        R = \left\{\begin{pmatrix}a&0\\0&b\end{pmatrix}\vert a,b \in \Z\right\},
    \]
    which is a subring of $\Mn{2}{\Z}$. Show that $R \cong \Z^2$.
\end{problem}

\begin{problem}
    Let $R$ and $R'$ be rings, and let $\phi: R \to R'$ be a ring isomorphism. Prove or disprove the following statements.
    \begin{partquestions}{\alph*}
        \item $\phi^{-1}: R' \to R$ is a ring isomorphism.
        \item If $R$ has a subring with $n$ elements, then so does $R'$.
        \item If $R$ has an ideal, then so does $R'$.
    \end{partquestions}
\end{problem}

\begin{problem}
    Find all ring endomorphisms of $\Z_{10}$.\newline
    Hence find all ring isomorphisms $\psi: \Z_{10} \to \Z_{10}$.
\end{problem}

\begin{problem}
    Find all endomorphisms over $\Q[\sqrt3]$.\newline
    Hence find all ring isomorphisms $\psi: \Q[\sqrt3] \to \Q[\sqrt3]$.
\end{problem}

\begin{problem}
    Let $R$ and $R'$ be commutative rings, $I$ be an ideal of $R$, and $\phi: R\to R'$ be a ring homomorphism.
    \begin{partquestions}{\roman*}
        \item Show that $\phi(\sqrt I) \subseteq \sqrt{\phi(I)}$.
        \item If $\phi$ is surjective with $\ker\phi \subseteq I$, prove that $\phi(\sqrt I) = \sqrt{\phi(I)}$.
    \end{partquestions}
\end{problem}

\begin{problem}\label{problem-ring-isomorphism-2}
    Let $R$ be a ring with subring $S$ and ideal $I$.
    \begin{partquestions}{\roman*}
        \item Prove $S+I$ is a subring of $R$.
        \item Prove $S \cap I$ is an ideal of $S$.
        \item Prove $S/(S\cap I)\cong (S+I)/I$.
    \end{partquestions}
\end{problem}

\newpage

\begin{problem}\label{problem-ring-isomorphism-3}
    Let $R$ be a ring with ideals $I$ and $J$ such that $I$ is a subset of $J$.
    \begin{partquestions}{\roman*}
        \item Prove that $J/I$ is an ideal of $R/I$.
        \item Prove that $\frac{R/I}{J/I} \cong R/J$.\newline
        (\textit{Note: remember to prove that the map is well-defined.})
    \end{partquestions}
\end{problem}

\chapter{Polynomial Rings and Division}
Polynomial rings are an important part of algebra and in ring theory, since polynomials are ubiquitous in modern algebra. We explore polynomials and polynomial rings in this chapter.

\section{What is a Polynomial Ring?}
We first define polynomials.
\begin{definition}
    A \textbf{polynomial}\index{polynomial} is an expression consisting of \textbf{variables}\index{polynomial!variable} (or \textbf{indeterminates}\index{polynomial!indeterminate}) and coefficients, that involves only the operations of addition, subtraction, multiplication, and positive-integer powers of variables.
\end{definition}
\begin{definition}
    Polynomials in a single variable are called \textbf{univariate polynomials}\index{polynomial!univariate} and takes the form
    \[
        a_0+a_1x+a_2x^2+\cdots+a_nx^n = \sum_{i=0}^n a_ix^i,
    \]
    where $a_0, a_1, a_2, \dots, a_n$ are called the \textbf{coefficients}\index{polynomial!coefficient} of the polynomial.
\end{definition}

Note that in the above definition, we did not explicitly state where the coefficients originate from; we do so now in the definition for a polynomial ring.
\begin{definition}
    Let $R$ be a commutative ring with identity, where $1 \neq 0$. Then the \textbf{polynomial ring}\index{polynomial ring} in \textbf{indeterminate}\index{indeterminate} (or \textbf{variable}\index{variable}) $x$ and coefficients in $R$ is
    \[
        R[x] = \{a_0 + a_1x + \cdots + a_nx^n \vert n \in \mathbb{N} \cup \{0\}, a_i \in R\},
    \]
    where for $f(x) = a_0 + \cdots + a_mx^m, g(x) = b_0 + \cdots + b_nx^n \in R[x]$, and assuming $m \leq n$, we define addition\index{polynomial!addition} and multiplication\index{polynomial!multiplication} by
    \begin{align*}
        f(x) + g(x) &= \sum_{i=0}^n\left((a_i+b_i)x^i\right)\\
        f(x)\times g(x) &= \sum_{k=0}^{m+n}\left(\left(\sum_{i=0}^k a_ib_{k-i}\right)x^k\right)
    \end{align*}
    respectively, where $a_l = 0$ for all $l > m$.
\end{definition}
\begin{proposition}
    The polynomial ring $R[x]$ is a commutative ring.
\end{proposition}
\begin{proof}
    We first prove that $R[x]$ is indeed a ring, before proving commutativity of multiplication of polynomials. For brevity, let $f(x), g(x), h(x) \in R[x]$ where
    \begin{align*}
        f(x) &= a_0 + a_1x + a_2x^2 + \cdots + a_mx^m,\\
        g(x) &= b_0 + b_1x + b_2x^2 + \cdots + b_nx^n,\\
        h(x) &= c_0 + c_1x + c_2x^2 + \cdots + c_lx^l,
    \end{align*}
    each $a_i$, $b_j$, and $c_k$ are elements from $R$, the integers $m$, $n$, and $l$ are all non-negative, and $a_m$, $b_n$, and $c_l$ are all non-zero. Without loss of generality, assume $m \geq n \geq l$, and `pad' the polynomials $g(x)$ and $h(x)$ with extra terms so that the highest power in $x$ is $m$.
    \begin{itemize}
        \item \textbf{Addition-Abelian}: We show that $(R[x], +)$ is an abelian group.
        \begin{itemize}
            \item \textbf{Closure}: We see
            \[
                f(x) + g(x) = \sum_{i=0}^n\left((a_i+b_i)x^i\right)
            \]
            and since $R$ is a ring, thus $a_i+b_i \in R$, meaning $f(x) + g(x)$ is another polynomial in $R[x]$. Therefore $R[x]$ is closed under addition.
            
            \item \textbf{Associativity}: Note
            \begin{align*}
                f(x) + (g(x) + h(x)) &= f(x) + \sum_{i=0}^m\left((b_i+c_i)x^i\right)\\
                &= \sum_{i=0}^m\left((a_i + (b_i + c_i))x^i\right)\\
                &= \sum_{i=0}^m\left(((a_i + b_i) + c_i)x^i\right) & (+ \text{ is associative in }R)\\
                &= \sum_{i=0}^m\left((a_i+b_i)x^i\right) + h(x)\\
                &= (f(x) + g(x)) + h(x)
            \end{align*}
            so addition of functions is associative.
            
            \item \textbf{Identity}: Note that $0 \in R$ is also the identity in $R[x]$, since
            \[
                0 + f(x) = \sum_{i=0}^m\left((0+a_i)x^i\right) = \sum_{i=0}^m\left(a_ix^i\right) = f(x).
            \]
            
            \item \textbf{Inverse}: For the polynomial $f(x)$, construct the polynomial $-f(x)$ where the coefficient of $x^i$ in $-f(x)$ is $-a_i$. Then
            \[
                f(x) + (-f(x)) = \sum_{i=0}^m\left((a_i+(-a_i))x^i\right) \sum_{i=0}^m\left(0x^i\right) = 0
           \]
           so $-f(x)$ is indeed the additive inverse of $f(x)$.
            
            \item \textbf{Commutativity}: One sees clearly that
            \[
                f(x) + g(x) = \sum_{i=0}^m\left((a_i+b_i)x^i\right) = \sum_{i=0}^m\left((b_i + a_i)x^i\right) = g(x) + f(x)
            \]
            since addition in $R$ is commutative. Therefore addition in $R[x]$ is also commutative.
        \end{itemize}

        \item \textbf{Multiplication-Subgroup}: We show that $(R[x], \times)$ is a subgroup.
        \begin{itemize}
            \item \textbf{Closure}: We note that
            \[
                f(x)\times g(x) = \sum_{k=0}^{m+n}\left(\underbrace{\left(\sum_{i=0}^k a_ib_{k-i}\right)}_{\text{In }R}x^k\right)
            \]
            so $f(x) \times g(x)$ is another polynomial in $R$.
            
            \item \textbf{Associativity}: \myref{exercise-polynomial-multiplication-is-associative} (later) proves that polynomial multiplication is associative.
        \end{itemize}

        \item \textbf{Distributive}: We finally show that $\times$ distributes over $+$. We only show that $f(x)(g(x) + h(x)) = f(x)g(x) + f(x)h(x)$ as we will later prove that $R[x]$ is commutative. Note
        \begin{align*}
            f(x)(g(x) + h(x)) &= \sum_{i=0}^m\left(\left(\sum_{j=0}^ia_j(b_{i-j}+c_{i-j})\right)x^i\right)\\
            &= \sum_{i=0}^m\left(\left(\sum_{j=0}^i(a_jb_{i-j}+a_jc_{i-j})\right)x^i\right) & (\text{Distribute in }R)\\
            &= \sum_{i=0}^m\left(\sum_{j=0}^i(a_jb_{i-j}x^i+a_jc_{i-j}x^i)\right)\\
            &= \sum_{i=0}^m\left(\sum_{j=0}^ia_jb_{i-j}x^i + \sum_{j=0}^ia_jc_{i-j}x^i\right)\\
            &= \sum_{i=0}^m\left(\sum_{j=0}^ia_jb_{i-j}x^i\right) + \sum_{i=0}^m\left(\sum_{j=0}^ia_jc_{i-j}x^i\right)\\
            &= f(x)g(x) + f(x)h(x)
        \end{align*}
        which is what was needed to be shown.
    \end{itemize}
    Therefore $R[x]$ is a ring.

    Now we prove commutativity of multiplication. Let $f(x) = a_0 + \cdots + a_mx^m, g(x) = b_0 + \cdots + b_nx^n \in R[x]$, and $m \leq n$. Then
    \begin{align*}
        f(x)g(x) &= \sum_{k=0}^{m+n}\left(\left(\sum_{i=0}^k a_ib_{k-i}\right)x^k\right)\\
        &= \sum_{k=0}^{m+n}\left(\left(\sum_{i=0}^k b_{k-i}a_i\right)x^k\right) & (R\text{ is commutative})\\
        &= \sum_{k=0}^{m+n}\left(\left(\sum_{i=0}^k b_i a_{k-i}\right)x^k\right)\\
        &= g(x)f(x)
    \end{align*}
    which therefore means that $R[x]$ is a commutative ring.
\end{proof}

Let's look at some examples of polynomial rings.
\begin{example}
    The polynomial ring $\R[x]$ is the most familiar for most of us, as this is the `standard' ring of polynomials. Examples of polynomials in this ring include $1+x$, $\sqrt2x^{10} - 5x^3 + \pi x$, and $1+x+x^2+\cdots+x^n$. However, infinite polynomials such as $1+x+x^2+\cdots$ do not belong in $\R[x]$.
\end{example}
\begin{example}
    Another commonly used polynomial ring is $\Q[x]$. Examples of polynomials in this ring are $1+x$, $\frac23x^5 - \frac7{11}x^{13}$, and $2x^2-5x-3$. However polynomials like $\sqrt2$, $\pi x + 1$, and $1+ex$ do not belong in $\Q[x]$.
\end{example}
\begin{example}
    We look at the polynomial ring $\Mn{2}{\R}[x]$. One example of a polynomial in this ring is
    \[
        \begin{pmatrix}2&1\\-2&0\end{pmatrix} + \begin{pmatrix}\sqrt2&1\\1&1\end{pmatrix}x + \begin{pmatrix}5&4\\e&2\end{pmatrix}x^2 + \begin{pmatrix}0&1\\0&0\end{pmatrix}x^5.
    \]
\end{example}

\begin{exercise}\label{exercise-polynomial-multiplication-is-associative}
    Prove that polynomial multiplication is associative.\newline
    (\textit{Hint: $\displaystyle \sum_{i=0}^k a_ib_{k-i} = \sum_{i+j=k} a_ib_j$. The second sum is the sum over all non-negative integers $i$ and $j$ with the property that $i+j=k$.})
\end{exercise}
\begin{exercise}
    Let $R$ be a ring. The \textbf{evaluation homomorphism}\index{evaluation homomorphism} is $\phi_a: R[x] \to R$ where $\phi_a(p(x)) = p(a)$ and $a \in R$. Prove that $\phi_a$ is indeed a ring homomorphism.
\end{exercise}
\begin{exercise}
    Let $I$ be a principal ideal of $\Z[x]$ generated by the polynomial $x^2 + 3x - 1$. Simplify $\left((x + 3) + I\right)\left((2x^2 + 3x - 1) + I\right)$ in the quotient ring $\Z[x]/I$.
\end{exercise}

We end this section by noting what form $R[x]$ takes when $x$ is \textit{not} a variable.
\begin{example}
    Recall that $\Q[\sqrt2] = \{a + b\sqrt2 \vert a,b \in \Q\}$. If we use the definition of $\Q[x]$, we see that
    \begin{align*}
        &\Q[\sqrt2] \\
        &= \{a_0 + a_1\sqrt2 + a_2(\sqrt2)^2 + a_3(\sqrt2)^3 + \cdots + a_n(\sqrt2)^n \vert a_i \in \Q \}\\
        &= \{a_0 + a_1\sqrt2 + a_2(2) + a_3(2\sqrt2) + \cdots + a_n(\sqrt2)^n \vert a_i \in \Q \}\\
        &= \{(a_0 + 2a_2 + 4a_4 + \cdots) + \sqrt2 (a_1 + 2a_3 + 4a_5 + \cdots) \vert a_i \in \Q\}\\
        &= \{a + b\sqrt2 \vert a,b \in Q\}
    \end{align*}
    which agrees with the previous definition of $\Q[\sqrt2]$.
\end{example}
\begin{example}
    Recall that $\Z[i]$, the gaussian integers, is the set $\{a+bi \vert a,b \in \Z\}$. If we use the definition of $\Z[x]$, we see that
    \begin{align*}
        \Z[i] &= \{a_0 + a_1i + a_2i^2 + a_3i^3 + \cdots + a_ni^n \vert a_i \in \Z\}\\
        &= \{a_0 + a_1i + a_2(-1) + a_3(-i) + \cdots + a_ni^n \vert a_i \in \Z\}\\
        &= \{(a_0 - a_2 + a_4 - \cdots) + i(a_1 - a_3 + a_5 - \cdots) \vert a_i \in \Z\}\\
        &= \{a + bi \vert a,b \in \Z\}
    \end{align*}
    which agrees with the previous definition of $\Z[i]$.
\end{example}

\section{Basic Terminology in Polynomial Rings}
We define the degree of a polynomial.
\begin{definition}
    Let $R[x]$ be a polynomial ring. The \textbf{degree}\index{degree} of a polynomial $f(x) \in R[x]$, denoted $\deg f(x)$, is the largest integer $k$ such that the coefficient of $x^k$ of $f(x)$ is non-zero.
\end{definition}
\begin{remark}
    For the zero polynomial (0), the degree is undefined.
\end{remark}
\begin{example}
    The degree of the polynomial $1+x+5x^2$ in $\Z[x]$ is 2.
\end{example}
\begin{example}
    The degree of the polynomial
    \[
        \begin{pmatrix}0&1\\3&0\end{pmatrix}x^5 + \begin{pmatrix}3&6\\7&2\end{pmatrix}x^4 + \begin{pmatrix}3&4\\9&4\end{pmatrix}
    \]
    in $\Mn{2}{\Z}[x]$ is 5.
\end{example}
\begin{exercise}
    Give an example of a degree 5 polynomial in the ring $\Z_2[x]$.
\end{exercise}

\begin{definition}
    Let $f(x) = a_0 + a_1x + a_2x^2 + \cdots + a_nx^n$ be a polynomial in the polynomial ring $R[x]$.
    \begin{itemize}
        \item The \textbf{constant term}\index{constant term} is $a_0$.
        \item The \textbf{leading term}\index{leading term} is the term $a_nx^n$.
        \item The \textbf{leading coefficient}\index{leading coefficient} is $a_n$.
    \end{itemize}
\end{definition}
\begin{remark}
    For the zero polynomial, the constant term is 0, the leading term is undefined, and the leading coefficient is undefined.
\end{remark}
\begin{example}
    Consider the polynomial
    \[
        \begin{pmatrix}0&1\\3&0\end{pmatrix}x^5 + \begin{pmatrix}3&6\\7&2\end{pmatrix}x^4 + \begin{pmatrix}3&4\\9&4\end{pmatrix}
    \]
    in $\Mn{2}{\Z}[x]$. Then
    \begin{itemize}
        \item the constant term is $\begin{pmatrix}3&4\\9&4\end{pmatrix}$;
        \item the leading term is $\begin{pmatrix}0&1\\3&0\end{pmatrix}x^5$; and
        \item the leading coefficient is $\begin{pmatrix}0&1\\3&0\end{pmatrix}$.
    \end{itemize}
\end{example}

\begin{definition}
    A \textbf{constant polynomial}\index{constant polynomial} is either the zero polynomial of a polynomial of degree 0.
\end{definition}
\begin{remark}
    This definition immediately implies that any constant polynomial in the polynomial ring $R[x]$ is an element of $R$.
\end{remark}

\section{Properties of Polynomials and Polynomial Rings}
We can now state the theorem that produces a condition for a ring to be an integral domain.
\begin{theorem}\label{thrm-integral-domain-iff-polynomial-ring-is-also}
    Let $R$ be a ring. Then $R$ is an integral domain if and only if $R[x]$ is an integral domain.
\end{theorem}
\begin{proof}
    We first need to show that $R$ is a commutative ring with identity if and only if $R[x]$ is a commutative ring with identity. We leave this for \myref{exercise-commutative-ring-with-identity-iff-polynomial-ring-is-also} (later). We only prove that $R$ has no zero divisors if and only if $R[x]$ has no zero divisors using a contrapositive proof.

    For the forward direction, take non-zero $a$ and $b$ in $R$ such that $ab = 0$. We may view both $a$ and $b$ as degree 0 polynomials in $R[x]$. Clearly these two multiply together to form the zero polynomial in $R[x]$, meaning that they are zero divisors in $R[x]$.

    For the reverse direction, take non-zero polynomials $f(x)$ and $g(x)$ in $R[x]$ such that $f(x)g(x) = 0$. Write
    \begin{align*}
        f(x) = a_0+a_1x+a_2x^2+\cdots+a_mx^m \text{ with } a_m \neq 0\\
        g(x) = b_0+b_1x+b_2x^2+\cdots+b_nx^n \text{ with } b_n \neq 0
    \end{align*}
    where all coefficients are in $R$. Multiplying them together yields something like
    \[
        a_mb_nx^{m+n} + (\text{A polynomial with degree less than }m+n) = 0
    \]
    which hence means that all coefficients must be zero. Therefore $a_mb_n = 0$. This means that we have found non-zero elements $a_m$ and $b_n$ in $R$ such that their product is zero, meaning that they are zero divisors.

    This completes the proof.
\end{proof}
\begin{exercise}\label{exercise-commutative-ring-with-identity-iff-polynomial-ring-is-also}
    Let $R$ be a ring.
    \begin{partquestions}{\alph*}
        \item Prove that $R$ is a ring with identity if and only if $R[x]$ is a ring with identity.
        \item Prove that $R$ is a commutative ring if and only if $R[x]$ is a commutative ring.
    \end{partquestions}
\end{exercise}

\newpage

\section{Problems}
\begin{problem}
    Show that $\princ{x}$ is a prime ideal in $\Z[x]$.
\end{problem}
\begin{problem}
    Let $I = \{f(x) \in \Z[x] \vert f(-2) = 0\}$ be a subset of $\Z[x]$, and let the map $\phi:\Z[x]\to\Z, f(x) \mapsto f(-2)$.
    \begin{partquestions}{\roman*}
        \item Show that $\phi$ is a ring homomorphism.
        \item Show that $I$ is an ideal of $\Z[x]$.
        \item Hence determine if the ideal $I$ is prime, maximal, or both.
    \end{partquestions}
\end{problem}
\begin{problem}
    Prove that $\Z[x] / \princ{x} \cong \Z$.
\end{problem}


%=========================================
\setpartpreamble[u][\textwidth]{
    \quoteattr{
        I became convinced that studying the algebraic relationship of numbers is most conveniently based on a concept that is directly connected with the simplest arithmetic properties. I had originally used the term ``rational domain'', which I later changed to ``field''.
    }
    {
        Richard Dedekind, 1871
    }
    {
        \cite[p.~66]{kleiner_2007}
    }

    We cover field theory essentials in part III. %TODO: Add
}
\part{Field Theory}

%=========================================
\appendix
\unnumberedpart{Appendices}
\section{Galois Theory}
\begin{questions}
    \item \begin{partquestions}{\roman*}
        \item We note that $[\C:\R] = 2$ since $\C = \R(i)$ and $i$ is a zero of the irreducible polynomial $x^2 + 1$ over $\R$. Therefore, by \myref{thrm-order-of-galois-group-is-degree-of-field-extension}, we see $|\Gal{\C/\R}| = [\C:\R] = 2$.
        
        \item Certainly $\id \in \Gal{\C/\R}$. We claim that the other element in $\Gal{\C/\R}$ is $\phi: \C \to \C$ where $\phi(a+bi) = a-bi$ for all $a+bi \in \C$. We first need to check that $\phi$ is an automorphism.
        \begin{itemize}
            \item \textbf{Homomorphism}: One sees clearly for any $a+bi, c+di \in \C$ that
            \begin{align*}
                \phi((a+bi)+(c+di)) &= \phi((a+c)+(b+d)i)\\
                &= (a+c)-(b+d)i\\
                &= (a-bi) + (c-di)\\
                &= \phi(a+bi) + \phi(c+di)
            \end{align*}
            and
            \begin{align*}
                \phi((a+bi)(c+di)) &= \phi((ac-bd) + (ad+bc)i)\\
                &= (ac-bd) - (ad+bc)i\\
                &= (a-bi)(c-di)\\
                &= \phi(a+bi)\phi(c+di)
            \end{align*}
            which proves that $\phi$ is indeed a homomorphism.

            \item \textbf{Injective}: Since $\phi$ is non-trivial it is thus injective by \myref{thrm-homomorphism-from-field-is-injective-or-trivial}.
            
            \item \textbf{Surjective}: For any $a + bi \in \C$ we note that $a - bi \in \C$ and that $\phi(a - bi) = a - (-b)i = a+bi$, proving that $\phi$ is surjective.
        \end{itemize}
        Therefore $\phi$ is a bijective homomorphism from $\C$ to $\C$, i.e. an automorphism. One also sees that $\phi(r) = r$ for all $r \in \R$, so $\phi$ fixes $\R$. Thus $\phi \in \Gal{\C/\R}$. But as $\Gal{\C/\R}$ has order 2, thus $\id$ and $\phi$ are the only two elements in $\Gal{\C/\R}$.
    \end{partquestions}
\end{questions}


\chapter{Problem Solutions}
\section{Introduction to Rings}
\subsection*{Exercises}
\begin{questions}
    \item We prove the ring axioms.
    \begin{itemize}
        \item \textbf{Addition-Abelian}: $(\{0\}, +)$ is an abelian group since this is just the trivial group.
        \item \textbf{Multiplication-Semigroup}: $(\{0\}, \cdot)$ is an abelian group since this is, again, just the trivial group. So $(\{0\}, \cdot)$ is a semigroup.
        \item \textbf{Distributive}: We know $+$ and $\cdot$ distribute.
    \end{itemize}
    Hence $(\{0\}, +, \cdot)$ is a commutative ring with identity.
\end{questions}

\subsection*{Problems}
No problems.

\section{Basics of Rings}
\begin{questions}
    \item We note the following.
    \begin{itemize}
        \item \textbf{Addition-Abelian}: $(\Z, +)$ is an abelian group.
        \item \textbf{Multiplication-Semigroup}: $(\Z, \times)$ is a semigroup since
        \begin{itemize}
            \item multiplying two integers always results in an integer, so $\Z$ is closed under $\times$; and
            \item $\times$ is associative.
        \end{itemize}
        \item \textbf{Distributive}: We know $+$ and $\times$ distribute.
    \end{itemize}
    Hence $(\Z, +, \times)$ is a ring.

    \item Consider $(-a)(-b) + (-ab)$ and note
    \begin{align*}
        &(-a)(-b) + (-ab)\\
        &= (-a)(-b) + (-a)b & (\text{\myref{prop-product-of-element-and-additive-inverse-is-additive-inverse-of-product}})\\
        &= (-a)(-b + b) & (\text{by \textbf{Distributive} axiom})\\
        &= (-a)0\\
        &= 0 & (\myref{prop-multiplying-by-zero-is-zero})
    \end{align*}
    which means $(-a)(-b) = -(-ab) = ab$ as required.

    \item The ring $\Mn{2}{\mathbb{R}}$ indeed has zero divisors, as $\begin{pmatrix}0&1\\0&0\end{pmatrix} \neq \begin{pmatrix}0&0\\0&0\end{pmatrix}$ but $\begin{pmatrix}0&1\\0&0\end{pmatrix}^2 = \begin{pmatrix}0&0\\0&0\end{pmatrix}$ which means that $\begin{pmatrix}0&1\\0&0\end{pmatrix}$ is a zero divisor.

    \item \begin{partquestions}{\alph*}
        \item $\Z$ is not a field. Note that the multiplicative inverse of 2 is $\frac12$ which is not an integer. Hence not all non-zero elements in $\Z$ has a multiplicative inverse, meaning that not all non-zero elements are units.

        \item $\Q$ is a field. Note for any rational number $\frac ab$ (where $b \neq 0$) it has an inverse of $\frac ba$. Thus any non-zero rational number is a unit, which means $\Q$ is a division ring. Since $\Q$ is also a commutative ring, therefore $\Q$ is a field.
    \end{partquestions}

    \item Let $u$ and $v$ be units, meaning that $u^{-1}$ and $v^{-1}$ exist. Then one sees that $(uv)(v^{-1}u^{-1}) = (v^{-1}u^{-1})(uv) = 1$, which means that $uv$ is also a unit.

    \item We first show that $(R, +) \leq (\Mn{2}{\R}, +)$.
    \begin{itemize}
        \item Clearly the identity of $(\Mn{2}{\R}, +)$, the zero matrix $\begin{pmatrix}0&0\\0&0\end{pmatrix}$, is inside $R$.
        \item Consider $\begin{pmatrix}a&a\\a&a\end{pmatrix}, \begin{pmatrix}b&b\\b&b\end{pmatrix} \in R$. The additive inverse of the matrix $\begin{pmatrix}b&b\\b&b\end{pmatrix}$ is the matrix $\begin{pmatrix}-b&-b\\-b&-b\end{pmatrix}$, and so their sum is
        \[
            \begin{pmatrix}a&a\\a&a\end{pmatrix} + \begin{pmatrix}-b&-b\\-b&-b\end{pmatrix} = \begin{pmatrix}a-b&a-b\\a-b&a-b\end{pmatrix} \in R
        \]
        which means $R$ is closed under addition.
    \end{itemize}
    Hence $(R, +) \leq (\Mn{2}{\R}, +)$ by subgroup test.

    We now show that $R$ is closed under multiplication. Some calculation yields that
    \[
        \begin{pmatrix}a&a\\a&a\end{pmatrix}\begin{pmatrix}b&b\\b&b\end{pmatrix} = \begin{pmatrix}2ab&2ab\\2ab&2ab\end{pmatrix}
    \]
    which is clearly in $R$. Therefore $R$ is a subring of $\Mn{2}{\R}$.
\end{questions}

\section{Integral Domains}
\begin{questions}
    \item To find a $a+bi \in \Z_5[i]$ such that there exists a $c+di \in \Z_5[i]$ where $(a+bi)(c+di) = 0$ but both $a+bi$ and $c+di$ are non-zero. Expanding $(a+bi)(c+di)$ yields $(ac-bd)+(ad+bc)i = 0$. Therefore we must have $ac-bd = 0$ and $ad+bc = 0$. For simplicity let's choose $a=c=1$. Using second equation we have $d+b = 0$ which means $d = -b$. Hence $(1 - b(-b))+(-b + b)i = 1+b^2 = 0$. Therefore choosing $b = 2$ would make it work. Therefore one solution is $a = 1, b = 2, c = 1, d = -2 = 3$; i.e. two zero divisors are $1+2i$ and $1+3i$.
    
    \item Take $w, z \in \Z[i]$ such that $w \neq 0$ and $wz = 0$. We want to show that $z = 0$. Let $z = a+bi$ and $w = c+di$. Since $w \neq 0$ we must have $c^2+d^2 \neq 0$. Now
    \[
        (a+bi)(c+di) = (ac-bd)+(ad+bc)i = 0
    \]
    which means $ac - bd = 0$ and $ad+bc = 0$. Multiplying first equation by $d$ yields $acd - bd^2 = 0$; multiplying second equation by $c$ yields $acd + bc^2 = 0$. Now summing them up yields $bc^2+bd^2 = b(c^2+d^2) = 0$ which hence means $b = 0$ since $c^2+d^2 \neq 0$. Therefore $ac - 0d = 0$ implies $ac = 0$ and $ad+0c = 0$ implies $ad = 0$. Squaring both equations and adding them up yields $a^2c^2 + a^2d^2 = a^2(c^2+d^2) = 0$ which hence means $a^2$ (and thus $a$) is zero. Therefore we have shown $z = 0$, meaning that there are no zero divisors in $\Z[i]$, so $\Z[i]$ is an integral domain.

    \item \begin{partquestions}{\alph*}
        \item Note that multiplication is commutative with identity $1 = 1 + 0\sqrt{n} \in R$. We just need to show that there are no zero divisors in $R$.
        
        Take $a+b\sqrt n, c+d\sqrt n \in R$ such that $a+b\sqrt n \neq 0$ but $(a+b\sqrt n)(c+d\sqrt n) = 0$. We want to show $c = d = 0$. Consider
        \[
            \left((a+b\sqrt n)(\underbrace{a-b\sqrt n}_{\neq 0})\right)\left((c+d\sqrt n)(\underbrace{c-d\sqrt n}_{\neq 0})\right) = 0.
        \]
        This means that $(a^2-nb^2)(c^2-nd^2) = 0$, so either $a^2-nb^2 = 0$ or $c^2-nd^2 = 0$.

        Now if $n < 0$ then clearly we have to have $c = d = 0$. Otherwise we have $a = b\sqrt n$ or $c = d\sqrt n$. But $\sqrt n$ is not an integer, so the only way for equality is if $c = d = 0$. Thus $\Z[\sqrt n]$ has no zero divisors, meaning $\Z[\sqrt n]$ is an integral domain.

        \item Consider $2 + \sqrt 2 \in \Z[\sqrt 2]$. Its multiplicative inverse is
        \begin{align*}
            \frac{1}{2+\sqrt2} &= \frac{2-\sqrt2}{(2+\sqrt2)(2-\sqrt2)}\\
            &= \frac{2-\sqrt2}{4-2}\\
            &= 1 - \frac12\sqrt2 \notin \Z[\sqrt2].
        \end{align*}
        This means that $2+\sqrt2$, a non-zero element in $\Z[\sqrt2]$, does not have an inverse in $\Z[\sqrt2]$. Therefore $\Z[\sqrt2]$ is not a field, meaning $R$ is not a field in the general case.
    \end{partquestions}

    \newpage

    \item For brevity let O$ = \begin{pmatrix}0&0\\0&0\end{pmatrix}$, I$ = \begin{pmatrix}1&0\\0&1\end{pmatrix}$, A$ = \begin{pmatrix}1&1\\1&0\end{pmatrix}$, and B$ = \begin{pmatrix}0&1\\1&1\end{pmatrix}$.
    
    \begin{partquestions}{\roman*}
        \item Clearly one sees that $R$ is a subset of $\Mn{2}{\Z_2}$.
        \begin{itemize}
            \item We show $(R, +)\leq(\Mn{2}{\Z_2},+)$.
            \begin{table}[h]
                \centering
                \begin{tabular}{|l|l|l|l|l|}
                    \hline
                    \textbf{+} & \textbf{O} & \textbf{I} & \textbf{A} & \textbf{B} \\ \hline
                    \textbf{O} & O          & I          & A          & B          \\ \hline
                    \textbf{I} & I          & O          & B          & A          \\ \hline
                    \textbf{A} & A          & B          & O          & I          \\ \hline
                    \textbf{B} & B          & A          & I          & O          \\ \hline
                \end{tabular}
            \end{table}
            
            From the Cayley table, clearly the identity of the ring $\Mn{2}{\Z_2}$ is in $R$ and $R$ is closed under addition. Hence $(R, +)\leq(\Mn{2}{\Z_2},+)$

            \item We show $R$ is closed under multiplication.
            \begin{table}[h]
                \centering
                \begin{tabular}{|l|l|l|l|l|}
                    \hline
                    $\boldsymbol{\cdot}$ & \textbf{O} & \textbf{I} & \textbf{A} & \textbf{B} \\ \hline
                    \textbf{O}           & O          & O          & O          & O          \\ \hline
                    \textbf{I}           & O          & I          & A          & B          \\ \hline
                    \textbf{A}           & O          & A          & B          & I          \\ \hline
                    \textbf{B}           & O          & B          & I          & A          \\ \hline
                \end{tabular}
            \end{table}
            
            From the Cayley table, clearly $R$ is closed under multiplication.
        \end{itemize}
        Therefore $R$ is a subring of $\Mn{2}{\Z_2}$.

        \item Since $R$ is a subring of $\Mn{2}{\Z_2}$, it is a ring. Furthermore, by the Cayley table of $(R, \cdot)$, we see that $R$ is commutative with identity I. Finally, one sees that $\mathrm{A}^{-1} = \mathrm{B}$, $\mathrm{B}^{-1} = \mathrm{A}$, and $\mathrm{I}^{-1} = \mathrm{I}$. Therefore all non-zero elements of $R$ have inverses. Hence $R$ is a field.
    \end{partquestions}
\end{questions}

\section{Ideals and Quotient Rings}
\begin{questions}
    \item Note that $36 = 2^2 \times 3^2$. So
    \begin{align*}
        \Ann{\Z_{36}}{\{15\}} &= \{r \in \Z_{36} \vert 15r = 0\}\\
        &= \{r \in \Z_{15} \vert 3(5r) = 0\}\\
        &= \{r \in \Z_{15} \vert r \text{ is a multiple of }2^2\times3 = 12\}\\
        &= \{0,12,24\}.
    \end{align*}

    \item We first show that $S$ is a subring of $\Z[i]$.
    \begin{itemize}
        \item The identity of $\Z[i]$ is $0 = 0 + 2(0)i \in S$.
        \item For any $a+2bi, c+2di \in S$, clearly $a+2bi + (-(c + 2di)) = (a-c) + 2(b-d)i \in S$.
        \item For any $a+2bi, c+2di \in S$, one sees that
        \begin{align*}
            (a+2bi)(c+2di) &= ac + 2adi + 2bci + 4bdi^2\\
            &= (ac-4bd) + 2(ad+bc)i\\
            &\in S.
        \end{align*}
    \end{itemize}
    Therefore $S$ is a subring of $\Z[i]$.

    We now show that $S$ is not an ideal of $\Z[i]$. Consider $1+2i \in \S$ and $1+i \in \Z[i]$. Then
    \begin{align*}
        (1+2i)(1+i) &= 1+i+2i+2i^2\\
        &= -1 + 3i\\
        &\notin S
    \end{align*}
    so there exists a $s \in S$ and a $r \in \Z[i]$ such that $rs\notin S$, meaning that $S$ is not a left ideal (and hence is not an ideal).

    \item We consider the test for ideal (\myref{thrm-test-for-ideal}).
    \begin{itemize}
        \item Note that $\begin{pmatrix}0&0\\0&0\end{pmatrix}=\begin{pmatrix}2(0)&2(0)\\2(0)&(0)\end{pmatrix}$ is in $I$ so $I$ is non-empty.
        \item $\begin{pmatrix}2a&2b\\2c&2d\end{pmatrix}-\begin{pmatrix}2e&2f\\2g&2h\end{pmatrix} = \begin{pmatrix}2(a-e)&2(b-f)\\2(c-g)&2(d-h)\end{pmatrix} \in I$.
        \item To show left ideal, take $\begin{pmatrix}2a&2b\\2c&2d\end{pmatrix} \in I$ and $\begin{pmatrix}e&f\\g&h\end{pmatrix} \in \Mn{2}{\Z}$. Then
        \begin{align*}
            \begin{pmatrix}2a&2b\\2c&2d\end{pmatrix}\begin{pmatrix}e&f\\g&h\end{pmatrix} &= \begin{pmatrix}2ae+2bg&2af+2bh\\2ce+2dg&2cf+2dh\end{pmatrix}\\
            &= \begin{pmatrix}2(ae+bg)&2(af+bh)\\2(ce+dg)&2(cf+dh)\end{pmatrix}\\
            &\in I
        \end{align*}
        so $I$ is a left ideal of $\Mn{2}{\Z}$.
        \item To show right ideal, take $\begin{pmatrix}a&b\\c&d\end{pmatrix} \in \Mn{2}{\Z}$ and $\begin{pmatrix}2e&2f\\2g&2h\end{pmatrix} \in I$. Then
        \begin{align*}
            \begin{pmatrix}a&b\\c&d\end{pmatrix}\begin{pmatrix}2e&2f\\2g&2h\end{pmatrix} &= \begin{pmatrix}2ae+2bg&2af+2bh\\2ce+2dg&2cf+2dh\end{pmatrix}\\
            &= \begin{pmatrix}2(ae+bg)&2(af+bh)\\2(ce+dg)&2(cf+dh)\end{pmatrix}\\
            &\in I
        \end{align*}
        so $I$ is a right ideal of $\Mn{2}{\Z}$.
    \end{itemize}
    Therefore by the test for ideal we have $I$ is an ideal of $\Mn{2}{\Z}$.

    \item \begin{partquestions}{\alph*}
        \item Suppose $I$ is not the trivial ring; we want to show that $I = R$. Since $I$ is non-trivial there there exists a non-zero element $a$ in $I$. Note that $a^{-1}$ exists since $R$ is a field, so $a$ is a unit. By \myref{exercise-ideal-containing-1-is-whole-ring} this means $I = R$. Note that $\{0\} = \princ{0}$ and $R = \princ{1}$ by \myref{exercise-trivial-ideal-and-whole-ring-are-principal-ideals}, so $R$ is indeed a PID.

        \item Take a non-zero $x \in R$ and note that $\princ{x}$ is a non-trivial ideal. Since there are no proper ideals in $R$, thus $\princ{x} = R$. This means that $1 \in \princ{x}$ (since $\princ{x} = R$ is a ring with identity), meaning that there exists an element $r \in R$ such that $xr = 1$. Therefore $x$ is a unit.
        
        Since $x$ is an arbitrary non-zero element in $R$, this thus shows that all non-zero elements of the ring $R$ are units, meaning $R$ is a division ring.

        Finally, because $R$ is commutative, thus $R$ is a field.
    \end{partquestions}

    \item \begin{partquestions}{\alph*}
        \item Suppose $r \in \sqrt{\sqrt{I}}$, meaning that $r^m \in \sqrt{I}$ for some positive integer $m$, further meaning that $(r^m)^n \in I$ for some positive integer $n$. Note $(r^m)^n = r^{mn} \in I$, so $r \in \sqrt{I}$. Therefore $\sqrt{\sqrt{I}} \subseteq \sqrt{I}$.
        
        Now suppose $r \in \sqrt{I}$, meaning that $r^n \in I$ for some positive integer $n$. Note that $r = r^1 \in \sqrt{I}$, so $r \in \sqrt{\sqrt{I}}$. Hence $\sqrt{I} \subseteq \sqrt{\sqrt{I}}$.

        Therefore, since $\sqrt{\sqrt{I}} \subseteq \sqrt{I}$ and $\sqrt{I} \subseteq \sqrt{\sqrt{I}}$, thus $\sqrt{\sqrt{I}} = \sqrt{I}$.

        \item Suppose $r \in \sqrt{I\cap J}$, so $r^n \in I \cap J$ for some positive integer $n$. This means that $r^n \in I$ and $r^n \in J$. Hence $r \in \sqrt{I}$ and $r \in \sqrt{J}$ by definition of the radical, so $r \in \sqrt{I}\cap\sqrt{J}$. Thus $\sqrt{I\cap J} \subseteq \sqrt{I}\cap\sqrt{J}$.
        
        Now suppose $r \in \sqrt{I}\cap\sqrt{J}$, meaning that $r \in \sqrt{I}$ and $r \in \sqrt{J}$. Thus $r^m \in I$ and $r^n \in J$ for some positive integers $m$ and $n$. Note that
        \[
            (\underbrace{r^m}_{\text{In }I})^n \in I \text{ and } (\underbrace{r^n}_{\text{In }J})^m \in J
        \]
        so $r^{mn} \in I$ and $r^{mn} \in J$, meaning $r^{mn} \in I \cap J$. Thus $r \in \sqrt{I \cap J}$, showing that $\sqrt{I}\cap\sqrt{J} \subseteq \sqrt{I\cap J}$.

        Therefore $\sqrt{I}\cap\sqrt{J} = \sqrt{I\cap J}$.
    \end{partquestions}

    \item \begin{partquestions}{\alph*}
        \item Suppose $a \in m\Z\cap n\Z$. Thus $a \in m\Z$ and $a \in n\Z$, meaning that $a = mx = ny$ for some integers $x$ and $y$. Therefore $a = \lcm(m,n)z = lz$ for some integer $z$, meaning $a \in l\Z$. Hence $m\Z \cap n\Z \subseteq l\Z$.
        
        Now suppose $a \in l\Z$, so $a = lx$ for some integer $x$. Write $l = m\alpha = n\beta$ for some integers $\alpha$ and $\beta$. Note that
        \begin{align*}
            a &= (m\alpha)x = m(\alpha x) \in m\Z\\
            a &= (n\beta)x = n(\beta x) \in n\Z
        \end{align*}
        so $a \in m\Z \cap n\Z$. Thus $l\Z \subseteq m\Z \cap n\Z$.

        Therefore $m\Z\cap n\Z = l\Z$.

        \item Suppose $a \in m\Z + n\Z$, meaning that there exist integers $x$ and $y$ such that $a = mx + ny$. By definition of the GCD, write $m = d\alpha$ and $n = d\beta$ for some integers $\alpha$ and $\beta$. Hence
        \begin{align*}
            a &= (d\alpha)x + (d\beta)y\\
            &= d(\alpha x + \beta y)\\
            &\in d\Z
        \end{align*}
        so $m\Z + n\Z \subseteq d\Z$.

        On the other hand, suppose $a \in d\Z$, meaning $a = dt$ for some integer $t$. By B\'{e}zout's Lemma (\myref{lemma-bezout}), we may write $d = mx + ny$ for some integers $x$ and $y$. Hence
        \begin{align*}
            a &= (mx + ny)t\\
            &= m(xt) + n(yt)\\
            &\in m\Z + n\Z
        \end{align*}
        which means $d\Z \subseteq m\Z + n\Z$.

        Therefore $m\Z + n\Z = d\Z$.
    \end{partquestions}

    \item Let $r \in R$, and suppose $x = r + \Nilr{R} \in R/\Nilr{R}$ is nilpotent, i.e. there is a positive integer $n$ such that
    \[
        x^n = (r + \Nilr{R})^n = r^n + \Nilr{R} = 0 + \Nilr{R}.
    \]
    Coset Equality (\myref{lemma-coset-equality}) thus tells us that $r^n \in \Nilr{R}$. Note that $\Nilr{R}$ contains all the nilpotents of $R$. Thus $r^n$ is a nilpotent of $R$, i.e. there exists a positive integer $m$ such that $(r^n)^m = 0$. But clearly $(r^n)^m = r^{mn} = 0$, so $r$ is nilpotent, meaning $r \in \Nilr{R}$. Hence $x = r + \Nilr{R} = 0 + \Nilr{R}$, meaning that the only nilpotent of $R/\Nilr{R}$ is the zero element. Therefore $R/\Nilr{R}$ has no non-zero nilpotents.

    \item Suppose $R$ is a PID and $I$ is a non-zero prime ideal. Let $J$ be an ideal such that $I \subseteq J \subseteq R$. Since $R$ is a PID, write $I = \princ{a}$ and $J = \princ{b}$ for some elements $a$ and $b$ in $R$. Note $a \in \princ{a} = I \subseteq J = \princ{b}$, so there exists an $r \in R$ such that $a = rb$. Now since $a = rb \in \princ{a} = I$ and $I$ is prime, therefore $r \in I$ or $b \in I$.
    \begin{itemize}
        \item If $r \in I$, write $r = sa$ for some $s \in R$. Then
        \[
            a = rb = (sa)b = (as)b = a(sb)
        \]
        since an integral domain is commutative. Thus $a - a(sb) = a(1-sb) = 0$. Now as $R$ is an integral domain thus either $a = 0$ (impossible since $a \neq 0$) or $1-sb = 0$. So $1-sb = 0$, meaning $sb = 1 \in J$ since $b \in J$. By \myref{exercise-ideal-containing-1-is-whole-ring} we have $J = R$.
        \item If instead $b \in I$, take any $x \in J = \princ{b}$, so $x = rb$ for some $r \in R$. Thus $x = rb \in I$ since $b \in I$, so $J \subseteq I$. But $I \subseteq J$, so $J = I$.
    \end{itemize}
    Therefore we have shown that $I$ is maximal.

    \item First we work in the forward direction. Suppose $\princ{a} = \princ{b}$. As $a \in \princ{a} = \princ{b}$, thus $a = bx$ for some $x \in R$. Also, as $b \in \princ{b} = \princ{a}$, thus $b = ay$ for some $y \in R$. Therefore
    \[
        b = ay = (bx)y = b(xy)
    \]
    which means $xy = 1$. Thus $x$ and $y$ are units, meaning $a = bx$ with $x$ being a unit.

    Now we work in the reverse direction; suppose $a = bu$ for some unit $u$ in $D$.
    \begin{itemize}
        \item Take $r \in \princ{a}$, so $r = ax$ for some $x$ in $D$. Thus $r = (bu)x = b(ux) \in \princ{b}$, so $\princ{a} \subseteq \princ{b}$.
        \item Note $b = au^{-1}$ since $u$ is a unit. Take $s \in \princ{b}$, so $s = by$ for some $y$ in $D$. But as $b = au^{-1}$, hence $s = (au^{-1})y = a(u^{-1}y) \in \princ{a}$, so $\princ{b} \subseteq \princ{a}$.
    \end{itemize}
    Therefore we see that $\princ{a} = \princ{b}$.
\end{questions}

\chapter{Ring Homomorphisms and Isomorphisms}
Like with groups, rings too have homomorphisms and isomorphisms, although they are defined slightly differently than in groups. Similar to how group homomorphisms preserve some structure between the two groups, ring homomorphisms and isomorphisms also preserve structure between rings.

\section{Ring Homomorphisms and Isomorphisms}
\begin{definition}
    Let $(R_1, +, \cdot)$ and $(R_2, \oplus, \otimes)$ be rings. A map $\phi: R_1 \to R_2$ is a \term{ring homomorphism}\index{homomorphism!ring} if and only if for all $a, b \in R_1$, we have
    \begin{align*}
        \phi(a+b) &= \phi(a) \oplus \phi(b) \text{ and}\\
        \phi(a\cdot b) &= \phi(a)\otimes\phi(b).
    \end{align*}
\end{definition}
\begin{remark}
    Like with group homomorphisms, we usually use ``$+$'' for both addition operations and suppress the multiplication operation. That is, the ring homomorphism conditions become
    \begin{align*}
        \phi(a+b) &= \phi(a) + \phi(b) \text{ and}\\
        \phi(ab) &= \phi(a)\phi(b).
    \end{align*}
\end{remark}

\begin{example}
    We show that the map $\phi: \Z \to \Z/n\Z, x \mapsto x + n\Z$ is a ring homomorphism. Let $a, b \in \Z$. Note
    \begin{align*}
        \phi(a+b) &= (a+b) + n\Z\\
        &= (a + n\Z) + (b + n\Z) & (\text{Definition of coset addition})\\
        &=\phi(a)+\phi(b)
    \end{align*}
    and
    \begin{align*}
        \phi(ab) &= ab + n\Z\\
        &= (a + n\Z)(b + n\Z) & (\text{Definition of coset multiplication})\\
        &= \phi(a)\phi(b)
    \end{align*}
    so $\phi$ is a homomorphism.
\end{example}

\begin{example}
    Consider the ring
    \[
        R = \left\{\begin{pmatrix}a&b\\0&c\end{pmatrix}\vert a,b,c\in\Z\right\}.
    \]
    The map $\phi: R \to \Z^2, \begin{pmatrix}a&b\\0&c\end{pmatrix} \mapsto (a,c)$ is a ring homomorphism since, for any $\begin{pmatrix}a&b\\0&c\end{pmatrix}, \begin{pmatrix}x&y\\0&z\end{pmatrix} \in R$, we have
    \begin{align*}
        \phi\left(\begin{pmatrix}a&b\\0&c\end{pmatrix} + \begin{pmatrix}x&y\\0&z\end{pmatrix}\right) &= \phi\left(\begin{pmatrix}a+x&b+y\\0&c+z\end{pmatrix}\right)\\
        &= (a+x,c+z)\\
        &= (a,c) + (x,z)\\
        &= \phi\left(\begin{pmatrix}a&b\\0&c\end{pmatrix}\right) + \phi\left(\begin{pmatrix}x&y\\0&z\end{pmatrix}\right)
    \end{align*}
    and
    \begin{align*}
        \phi\left(\begin{pmatrix}a&b\\0&c\end{pmatrix}\begin{pmatrix}x&y\\0&z\end{pmatrix}\right) &= \phi\left(\begin{pmatrix}ax&ay+bz\\0&cz\end{pmatrix}\right)\\
        &= (ax, cz)\\
        &= (a,c)(x,z)\\
        &= \phi\left(\begin{pmatrix}a&b\\0&c\end{pmatrix}\right)\phi\left(\begin{pmatrix}x&y\\0&z\end{pmatrix}\right).
    \end{align*}
\end{example}
\begin{exercise}
    Let the function $\phi: \Mn{2}{\Z} \to \Z$ be defined such that
    \[
        \phi\left(\begin{pmatrix}a&b\\c&d\end{pmatrix}\right) = a+d.
    \]
    Is $\phi$ a ring homomorphism?
\end{exercise}
\begin{exercise}
    Let $R$ and $S$ be rings with additive identities $0_R$ and $0_S$ respectively. Show that the \term{trivial homomorphism}\index{homomorphism!trivial} $\phi: R \to S, r \mapsto 0_S$ is, indeed, a ring homomorphism.
\end{exercise}

\pagebreak

An endomorphism is a specific type of homomorphism.
\begin{definition}
    A \term{ring endomorphism}\index{endomorphism!ring} of a ring $R$ is a homomorphism $\phi: R \to R$.
\end{definition}
\begin{example}
    Let $R$ be a commutative ring with prime characteristic $p$. The \term{Frobenius endomorphism}\index{Frobenius endomorphism} $\phi: R \to R$ is such that $\phi(r) = r^p$. We show that $\phi$ is a ring endomorphism.

    Note that for any $a, b \in R$ that
    \begin{align*}
        \phi(a+b) &= (a+b)^p\\
        &= a^p + pa^{p-1}b + {p \choose 2}a^{p-2}b^2 + \cdots + pab^{p-1} + b^p.
    \end{align*}
    Note that the binomial coefficients ${p \choose k}$ where $1 \leq k \leq p-1$ are all multiples of $p$ (\myref{prop-binomial-coefficient-multiple-of-p}). As the characteristic of the ring $R$ is $p$, thus $px = 0$ for any $x \in R$. Therefore,
    \begin{align*}
        \phi(a+b) = &a^p + pa^{p-1}b + {p \choose 2}a^{p-2}b^2 + \cdots + pab^{p-1} + b^p\\
        &= a^p + 0 + 0 + \cdots + 0 + b^p\\
        &= a^p + b^p\\
        &=\phi(a) + \phi(b).
    \end{align*}

    Also,
    \[
        \phi(ab) = (ab)^p = a^pb^p = \phi(a)\phi(b).
    \]
    Therefore $\phi$ is a ring endomorphism.
\end{example}

\begin{exercise}
    Let $R$ be a ring. Show that the \term{identity homomorphism}\index{homomorphism!identity} $\id: R \to R, r \mapsto r$ is a ring endomorphism.
\end{exercise}

The definition of ring isomorphisms is analogous to that of group isomorphisms.

\begin{definition}
    A \term{ring isomorphism}\index{isomorphism!ring} is a bijective ring homomorphism.
\end{definition}

Similar to groups, we write $R_1 \cong R_2$ if and only if $R_1$ and $R_2$ are (ring) isomorphic to each other.

\begin{example}\label{example-Zn-ring-isomorphic-to-Z/nZ}
    We show that $\Z_n \cong \Z/n\Z$. Consider the map $\phi:\Z_n \to \Z/n\Z$ where $m \mapsto m + n\Z$. We show that $\phi$ is an isomorphism.
    \begin{itemize}
        \item \textbf{Homomorphism}: For any $a, b \in \Z_n$ we see
        \[
            \phi(a+b) = (a+b) + n\Z = (a + n\Z) + (b + n\Z) = \phi(a) + \phi(b)
        \]
        and
        \[
            \phi(ab) = (ab) + n\Z = (a+n\Z)(b+n\Z) = \phi(a)\phi(b).
        \]

        \item \textbf{Injective}: Suppose $a, b \in \Z_n$ such that $\phi(a) = \phi(b)$. This means $a + n\Z = b + n\Z$, i.e. $a \equiv b \pmod n$. Now note that $0 \leq a,b < n$ so we have $a = b$.

        \item \textbf{Surjective}: Suppose $m + n\Z \in \Z/n\Z$. Applying Euclid's division lemma (\myref{lemma-euclid-division}) on $m$ we have $m = nq + r$ with $0 \leq r < n$. One sees that
        \begin{align*}
            \phi(r) &= r + n\Z\\
            &= r + (nq + n\Z)\\
            &= (r + nq) + n\Z\\
            &= m + n\Z
        \end{align*}
        so $m + n\Z$ has a pre-image of $r$ in $\Z_n$.
    \end{itemize}
    Since $\phi$ is a bijective ring homomorphism, thus $\phi$ is an isomorphism, meaning $\Z_n \cong \Z/n\Z$ as rings.
\end{example}
\begin{example}
    We show that $\Z \not\cong 2\Z$ as rings. Suppose $\phi: \Z \to 2\Z$ is a ring isomorphism. Set $a = \phi(1) = 2\Z$. Note that
    \[
        a = \phi(1) = \phi(1\times1) = (\phi(1))^2 = a^2
    \]
    so $a^2 = a$, which means $a = 0$ or $a = 1$. But as $a \in 2\Z$, thus $a \neq 1$ which means $a = 0$.

    But notice for any $n \in \Z$ we have
    \begin{align*}
        \phi(n) &= \phi(n1)\\
        &= \phi(n)\phi(1)\\
        &= \phi(n) \times 0\\
        &= 0.
    \end{align*}
    Thus one sees that $\phi(0) = \phi(1) = 0$ which means $\phi$ is not injective, a contradiction.
\end{example}

\begin{definition}
    A bijective ring endomorphism is called a \term{ring automorphism}\index{automorphism!ring}.
\end{definition}

\begin{exercise}\label{exercise-identity-homomorphism-is-an-isomorphism}
    Show that the identity homomorphism is actually an automorphism.
\end{exercise}

\section{Properties of Ring Homomorphisms}
For the following, let $R_1$ and $R_2$ be rings with additive identities $0_1$ and $0_2$ respectively. Also let $\phi: R_1 \to R_2$ be a ring homomorphism.

\begin{proposition}\label{prop-ring-image-of-additive-identity-is-additive-identity}
    $\phi(0_1) = 0_2$.
\end{proposition}
\begin{proof}
    See \myref{exercise-ring-image-of-identity-is-identity} (later).
\end{proof}

\begin{proposition}
    If $R_1$ and $R_2$ are division rings, then $\phi(1_1) = 1_2$ where $1_1$ and $1_2$ are the multiplicative identities of $R_1$ and $R_2$ respectively.
\end{proposition}
\begin{proof}
    See \myref{exercise-ring-image-of-identity-is-identity} (later).
\end{proof}

\begin{proposition}
    $\phi(-x) = -\phi(x)$ for all $x \in R_1$.
\end{proposition}
\begin{proof}
    See \myref{exercise-ring-image-of-inverse-is-inverse} (later).
\end{proof}

\begin{proposition}\label{prop-inverse-under-ring-homomorphism}
    If $R_1$ and $R_2$ are both division rings, then $\phi(x^{-1}) = (\phi(x))^{-1}$ for all $x \in R_1$.
\end{proposition}
\begin{proof}
    See \myref{exercise-ring-image-of-inverse-is-inverse} (later).
\end{proof}

\begin{proposition}\label{prop-homomorphism-on-subring-is-subring}
    If $S$ is a subring of $R_1$, then
    \[
        \phi(S) = \{\phi(s) | s \in S\}
    \]
    is a subring of $R_2$.
\end{proposition}
\begin{proof}
    Let $S$ be a subring of $R_1$. Take $a, b \in \phi(S)$, which means that there exist $s_a, s_b\in S$ such that $\phi(s_a) = a$ and $\phi(s_b) = b$.
    \begin{itemize}
        \item We show that $(\phi(S), +) \leq (R_2, +)$.
        \begin{itemize}
            \item Note that $\phi(S) \neq \emptyset$ since $\phi(0_1) = 0_2 \in \phi(S)$.
            \item Also note that $a - b = \phi(s_a) - \phi(s_b) = \phi(s_a-s_b) \in \phi(S)$.
        \end{itemize}

        \item One also sees that
        \[
            ab = \phi(s_a)\phi(s_b) = \phi(s_as_b) \in \phi(S).
        \]
    \end{itemize}
    Therefore $\phi(S)$ is a subring of $R_2$.
\end{proof}

\begin{proposition}
    If $\phi$ is surjective and $I$ is an ideal of $R_1$, then $\phi(I)$ is an ideal of $R_2$.
\end{proposition}
\begin{proof}
    From previous proposition $\phi(I)$ is a subring of $R_2$. We just need to show that $\phi(I)$ is an ideal of $R_2$.

    Take $a \in \phi(I)$ and $r_2 \in R_2$. As $\phi$ is surjective, we can find a $r_1 \in R_1$ such that $\phi(r_1) = r_2$. Also, let $a = \phi(i)$ for an $i \in I$.

    Note
    \begin{align*}
        ar_2 = \phi(i)\phi(r_1) = \phi(ir_1) \in \phi(I)\\
        r_2a = \phi(r_1)\phi(i) = \phi(r_1i) \in \phi(I)
    \end{align*}
    so $\phi(I)$ is an ideal of $R_2$.
\end{proof}

\begin{proposition}\label{prop-inverse-homomorphism-on-ideal-is-ideal}
    Let $J$ be an ideal of $R_2$. Then
    \[
        \phi^{-1}(J) = \{r \in R_1 \vert \phi(r) \in J\}
    \]
    is an ideal of $R_1$.
\end{proposition}
\begin{proof}
    Suppose $J$ is an ideal of $R_2$. We consider the test for ideal (\myref{thrm-test-for-ideal}) to show $\phi^{-1}(J)$ is an ideal of $R_1$.

    One sees that $\phi^{-1}(J) \neq \emptyset$ since $\phi(0_1) = 0_2 \in J$, so $0_1 \in \phi^{-1}(J)$.

    Let $a, b \in \phi^{-1}(J)$, so $\phi(a), \phi(b) \in J$. Note that
    \[
        \phi(a-b) = \phi(a) - \phi(b) \in J
    \]
    so $a-b \in \phi^{-1}(J)$ for all $a,b \in J$.

    Let $r \in R_1$ and $a \in \phi^{-1}(J)$. Note that $\phi(a) \in J$ and $\phi(r) \in R_2$, so $\underbrace{\phi(a)}_{\text{In }J}\underbrace{\phi(r)}_{\text{In }R_2} \in J$ and $\phi(r)\phi(a) \in J$. Note $\phi(a)\phi(r) = \phi(ar) \in J$, so $ar \in \phi^{-1}(J)$, and similarly we have $\phi(r)\phi(a) = \phi(ra) \in J$, so $ra \in \phi^{-1}(J)$.

    Therefore, by the test for ideal, $\phi^{-1}(J)$ is an ideal of $R_1$.
\end{proof}

\begin{exercise}\label{exercise-ring-image-of-identity-is-identity}
    Let $R_1$ and $R_2$ be rings, and $\phi: R_1 \to R_2$ be a ring homomorphism.
    \begin{partquestions}{\alph*}
        \item Show that $\phi(0_1) = 0_2$, where $0_1$ and $0_2$ are the additive identities of $R_1$ and $R_2$ respectively.
        \item If $R_1$ and $R_2$ are division rings, then show that $\phi(1_1) = 1_2$, where $1_1$ and $1_2$ are the multiplicative identities of $R_1$ and $R_2$ respectively.
    \end{partquestions}
\end{exercise}

\begin{exercise}\label{exercise-ring-image-of-inverse-is-inverse}
    Let $R_1$ and $R_2$ be rings, $x \in R_1$, and $\phi: R_1 \to R_2$ be a ring homomorphism.
    \begin{partquestions}{\alph*}
        \item Show that $\phi(-x) = -\phi(x)$.
        \item If $R_1$ and $R_2$ are division rings, show that $\phi(x^{-1}) = (\phi(x))^{-1}$.
    \end{partquestions}
\end{exercise}

\section{Image and Kernel}
Similar to group homomorphisms, ring homomorphisms too have a image and kernel.
\begin{definition}
    The \term{image}\index{image} of a ring homomorphism $\phi: R_1 \to R_2$ is
    \[
        \im\phi = \{\phi(r) \vert r \in R_1\}.
    \]
\end{definition}
\begin{definition}
    The \term{kernel}\index{kernel} of a ring homomorphism $\phi:R_1 \to R_2$ is
    \[
        \ker\phi = \{r \in R_1 \vert \phi(r) = 0\}.
    \]
\end{definition}

\begin{example}\label{example-homomorphism-on-upper-triangle-matrices}
    Consider the ring
    \[
        R = \left\{\begin{pmatrix}a&b\\0&c\end{pmatrix}\vert a,b,c\in\Z\right\}
    \]
    and the homomorphism $\phi: R \to \Z^2, \begin{pmatrix}a&b\\0&c\end{pmatrix} \mapsto (a,c)$.

    We note that $\phi$ is surjective; for any $(x,y)\in\Z^2$, we see that 
    \[
        \phi\left(\begin{pmatrix}x&0\\0&y\end{pmatrix}\right) = (x,y)
    \]
    so any $(x,y)$ has a pre-image in $R$. Therefore $\im \phi = \Z^2$.

    We now find the kernel of $\phi$.
    \begin{align*}
        \ker\phi &= \left\{\begin{pmatrix}a&b\\0&c\end{pmatrix} \in R \vert \phi\left(\begin{pmatrix}a&b\\0&c\end{pmatrix}\right) = (0,0)\right\}\\
        &= \left\{\begin{pmatrix}a&b\\0&c\end{pmatrix} \in R \vert (a,c) = (0,0)\right\}\\
        &= \left\{\begin{pmatrix}0&n\\0&0\end{pmatrix} \vert n \in \Z\right\}.
    \end{align*}
\end{example}

We look at some results regarding the image and kernel of a ring homomorphism. These results may look familiar to those in part I.
\begin{proposition}\label{prop-image-is-a-subring}
    Let $R_1$ and $R_2$ be rings, and let $\phi: R_1 \to R_2$ be a ring homomorphism. Then $\im\phi$ is a subring of $R_2$.
\end{proposition}
\begin{proof}
    \myref{prop-homomorphism-on-subring-is-subring} tells us $\im\phi = \phi(R_1)$ is a subring of $R_2$.
\end{proof}

\begin{proposition}\label{prop-kernel-is-an-ideal}
    Let $R_1$ and $R_2$ be rings, and let $\phi: R_1 \to R_2$ be a ring homomorphism. Then $\ker\phi$ is an ideal of $R_1$.
\end{proposition}
\begin{proof}
    See \myref{exercise-kernel-is-an-ideal} (later).
\end{proof}

\begin{proposition}
    Let $R_1$ and $R_2$ be rings, and let $\phi: R_1 \to R_2$ be a ring homomorphism. Then $\phi$ is injective if and only if $\ker\phi = \{0_1\}$.
\end{proposition}
\begin{proof}
    We first prove the forward direction; suppose $\phi$ is injective and let $a \in \ker\phi$. By definition of the kernel we have $\phi(a) = 0_2$. But by \myref{prop-ring-image-of-additive-identity-is-additive-identity}, we have $\phi(0_1) = 0_2$. Since $\phi$ is injective, therefore $a = 0_1$, meaning $\ker\phi = \{0_1\}$.

    We now prove the reverse direction; suppose $\ker\phi = \{0_1\}$. Now let $a,b \in R_1$ such that $\phi(a) = \phi(b)$. Therefore $\phi(a) - \phi(b) = \phi(a-b) = 0_2$. Therefore $a-b \in \ker\phi$ by definition of the kernel. However $\ker\phi = \{0_1\}$ which means that $a - b = 0_1$. Therefore $a = b$, meaning $\phi$ is injective.
\end{proof}

\begin{exercise}\label{exercise-kernel-is-an-ideal}
    Let $R_1$ and $R_2$ be rings, and let $\phi: R_1 \to R_2$ be a ring homomorphism. Prove that $\ker\phi$ is an ideal of $R_1$.
\end{exercise}

\section{The Ring Isomorphism Theorems}
Similar to group theory, there are three main ring isomorphism theorems. However, we will only explicitly prove the first ring isomorphism theorem; the other two will be left as problems.

\begin{theorem}[First Ring Isomorphism Theorem (FRIT)]\label{thrm-ring-isomorphism-1}\index{isomorphism theorem!ring!first}\index{FRIT}
    Let $R$ and $R'$ be rings. Let $\phi: R \to R'$ be a ring homomorphism, and let $\pi: R \to R/\ker\phi$, $r\mapsto r + \ker\phi$ be the natural surjective homomorphism. Then there exists a unique ring isomorphism $\psi: R / \ker\phi \to \im\phi$ such that $\psi\pi = \phi$.
\end{theorem}
\begin{remark}
    Equivalently, the FRIT states that
    \[
        R / \ker\phi \cong \im\phi
    \]
    for any ring homomorphism $\phi$. This means that the set of pre-images that have an image of $\phi(x)$ is $x\ker\phi$.
\end{remark}

We include the commutativity diagram of the stated maps for clarity.
\begin{figure}[h]
    \centering
    \pdfteximgframed[12pt]{0.3\textwidth}{part2/images/ring-homomorphisms/ring-iso-1-commutativity.pdf_tex}
    \caption{Commutativity Diagram for \myreffigures{thrm-ring-isomorphism-1}}
\end{figure}

In the diagram, $\phi$ sends elements from $R$ to $\im\phi$ and $\pi$ sends elements from $R$ to $R/\ker\phi$. Then the map $\psi$ is a unique map that sends elements from $R/\ker\phi$ to the image of $\phi$.

\begin{proof}[Proof (cf. {\cite[p.~302, Factor Theorem For Rings]{cohn_1982}})]
    Let the map $\psi$ be defined such that $\psi(r + \ker\phi) = \phi(r)$. We first show that $\psi$ is a well-defined ring isomorphism.
    \begin{itemize}
        \item \textbf{Well-Defined}: Suppose $a + \ker\phi$ and $b + \ker\phi$ are in $R/\ker\phi$ such that $a + \ker\phi = b+\ker\phi$. This means that $a - b \in \ker\phi$, i.e. $\phi(a-b) = 0$ by definition of the kernel. Hence $\phi(a) - \phi(b) = 0$ which means $\phi(a) = \phi(b)$. Therefore, we see that
        \[
            \psi(a + \ker\phi) = \phi(a) = \phi(b) = \psi(b + \ker\phi)
        \]
        which means $\psi$ is well-defined.

        \item \textbf{Homomorphism}: Let $a + \ker\phi, b + \ker\phi \in R/\ker\phi$. Then note
        \begin{align*}
            &\psi((a + \ker\phi)+(b+\ker\phi))\\
            &= \psi((a+b)+\ker\phi)\\
            &= \phi(a+b)\\
            &= \phi(a) + \phi(b)\\
            &= \psi(a + \ker\phi) + \psi(b + \ker\phi)
        \end{align*}
        and
        \begin{align*}
            \psi((a + \ker\phi)(b+\ker\phi)) &= \psi((ab)+\ker\phi)\\
            &= \phi(ab)\\
            &= \phi(a)\phi(b)\\
            &= \psi(a + \ker\phi)\psi(b + \ker\phi),
        \end{align*}
        so $\psi$ is a ring homomorphism.

        \item \textbf{Injective}: Suppose $a + \ker\phi, b + \ker\phi \in R/\ker\phi$ such that $\psi(a+\ker\phi) = \psi(b+\ker\phi)$. So,
        \begin{align*}
            \phi(a) &= \phi(b) & (\text{definition of }\psi)\\
            \phi(a) - \phi(b) &= 0\\
            \phi(a-b) &= 0 & (\phi \text{ is a ring homomorphism})\\
            a - b &\in \ker\phi & (\text{definition of kernel})\\
            a + \ker\phi &= b + \ker\phi.
        \end{align*}
        Therefore if $\psi(a+\ker\phi) = \psi(b+\ker\phi)$ then $a+\ker\phi = b+\ker\phi$, which means $\psi$ is injective.

        \item \textbf{Surjective}: Suppose $s \in \im\phi$, so there is an $r \in R$ such that $s = \phi(r)$. Clearly $\psi(r + \ker\phi) = \phi(r) = s$, so $s$ has a pre-image of $r + \ker\phi$, i.e. $\psi$ is surjective.
    \end{itemize}
    Therefore, $\psi$ is a well-defined bijective ring homomorphism, i.e. $\psi$ is a well-defined ring isomorphism.

    We now check that $\psi$ satisfies the requirement that $\psi\pi = \phi$. Note that $\pi(x) = x + \ker\phi$ and
    \[
        \psi\pi(x) = \psi(x + \ker\phi) = \phi(x)
    \]
    for all $x \in R$, so $\psi\pi = \phi$.

    Finally, we show that $\psi$ is unique. Suppose $f: R/\ker\phi \to \im\phi$ is an isomorphism satisfying $f\pi=\phi$. Note that
    \begin{align*}
        f(x + \ker\phi) &= f(\pi(x))\\
        &= (f\pi)(x)\\
        &= \phi(x)\\
        &= (\psi\pi)(x)\\
        &= \psi(\pi(x))\\
        &= \psi(x + \ker\phi)
    \end{align*}
    for all $x \in R$, meaning that $f = \psi$. Therefore $\psi$ is unique.

    Hence, $\psi$ is a unique ring isomorphism satisfying $\psi\pi = \phi$.
\end{proof}

\begin{example}
    Consider the ring
    \[
        R = \left\{\begin{pmatrix}a&b\\0&c\end{pmatrix}\vert a,b,c\in\Z\right\}
    \]
    and the homomorphism $\phi: R \to \Z^2, \begin{pmatrix}a&b\\0&c\end{pmatrix} \mapsto (a,c)$. We found in \myref{example-homomorphism-on-upper-triangle-matrices} that $\phi$ is surjective (i.e., $\im\phi = \Z^2$) with kernel
    \[
        \left\{\begin{pmatrix}0&n\\0&0\end{pmatrix} \vert n \in \Z\right\}
    \]
    which, for brevity, we shall denote by $I$. Thus the FRIT (\myref{thrm-ring-isomorphism-1}) tells us that
    \[
        R/I \cong \Z^2.
    \]
\end{example}
\begin{exercise}
    Show that $\Z_n \cong \Z/n\Z$ by considering the ring homomorphism $\phi: \Z \to \Z_n$ where $m \mapsto m \mod n$ and by using the FRIT (\myref{thrm-ring-isomorphism-1}).
\end{exercise}

We briefly mention the other two main ring isomorphism theorems, although the proof of them will be left as problems. They are much less used than the FRIT, so we make only a passing mention of them.

\newpage

\begin{theorem}[Second Ring Isomorphism Theorem]\label{thrm-ring-isomorphism-2}\index{isomorphism theorem!ring!second}
    Let $R$ be a ring with subring $S$ and ideal $I$. Then
    \begin{enumerate}
        \item $S+I = \{s+i \vert s\in S,\;i\in I\}$ is a subring of $R$;
        \item $S \cap I$ is an ideal of $S$; and
        \item $(S+I)/I \cong S/(S\cap I)$.
    \end{enumerate}
\end{theorem}
\begin{proof}
    See \myref{problem-ring-isomorphism-2} (later).
\end{proof}

\begin{theorem}[Third Ring Isomorphism Theorem]\label{thrm-ring-isomorphism-3}\index{isomorphism theorem!ring!third}
    Let $R$ be a ring with ideals $I$ and $J$ such that $I$ is a subset of $J$. Then
    \begin{enumerate}
        \item $J/I$ is an ideal of $R/I$; and
        \item $\frac{R/I}{J/I} \cong R/J$.
    \end{enumerate}
\end{theorem}
\begin{proof}
    See \myref{problem-ring-isomorphism-3} (later).
\end{proof}

\section{Restrictiveness of Ring Homomorphisms}
Although ring homomorphisms appear to be quite general, we explore how restricted they really are when dealing with certain rings.

\begin{example}\label{example-endomorphisms-of-Z}
    We find all ring endomorphisms of $\Z$.

    Let $\phi:\Z\to\Z$ be a ring endomorphism. Set $a = \phi(1)$. Note that
    \[
        a = \phi(1) = \phi(1\times1) = \phi(1)\phi(1) = a^2
    \]
    so $a^2 = a$. Thus $a = 0$ or $a = 1$ in $\Z$.

    If $a = 0$, then for any $n \in \Z$ we have
    \[
        \phi(n) = \phi(1n) = \phi(1)\phi(n) = 0\phi(n) = 0
    \]
    so $\phi(n) = 0$ for all $n \in \Z$, which is the trivial homomorphism.

    Now consider the case that $a = 1$. We claim that $\phi(n) = n$ for all $n \in \Z$. We leave the proof that $\phi(n) = n$ for all \textit{positive} integers $n$ for \myref{exercise-homomorphism-maps-n-to-n-if-n-is-positive} (later). Furthermore $\phi(0) = 0$ by the properties of ring homomorphism (specifically \myref{prop-ring-image-of-additive-identity-is-additive-identity}). Finally, note that for any non-negative integer $n$,
    \begin{align*}
        0 &= \phi(0)\\
        &= \phi(n - n)\\
        &= \phi(n) + \phi(-n)\\
        &= n + \phi(-n)
    \end{align*}
    which means $\phi(-n) = -n$. Thus $\phi(n) = n$ for all integers $n$, which is the identity endomorphism.

    Therefore, the only ring endomorphisms $\phi:\Z\to\Z$ are the trivial homomorphism and the identity endomorphism.
\end{example}
\begin{exercise}\label{exercise-homomorphism-maps-n-to-n-if-n-is-positive}
    For the map $\phi$ in \myref{example-endomorphisms-of-Z}, show that $\phi(n) = n$ for all positive integers $n$.
\end{exercise}
Note that in \myref{example-endomorphisms-of-Z} we started the entire computation with the observation that $\phi(1) = \phi(1)^2$. This means that $\phi(1)$ is an idempotent.
\begin{definition}
    Let $R$ be a ring. Then an element $x \in R$ is an \term{idempotent}\index{idempotent} if and only if $x^2 = x$.
\end{definition}
\begin{proposition}\label{prop-homomorphism-on-multiplicative-identity-is-idempotent}
    Let $R$ and $R'$ be rings, and let $\phi: R \to R'$ be a ring homomorphism. If $R$ is a ring with identity, then $\phi(1)$ is an idempotent.
\end{proposition}
\begin{proof}
    Note
    \[
        \phi(1) = \phi(1 \times 1) = \phi(1) \times \phi(1) = \left(\phi(1)\right)^2
    \]
    which means $\phi(1)$ is an idempotent.
\end{proof}

In \myref{example-endomorphisms-of-Z} we used the fact that the only idempotents of $\Z$ are 0 and 1. However, this is not true for a general ring.

\begin{example}\label{example-homomorphisms-from-Z12-to-Z28}
    We find all ring homomorphisms $\phi: \Z_{12} \to \Z_{28}$. Note that $\Z_{12}$ is a ring with an identity of 1.

    \myref{prop-homomorphism-on-multiplicative-identity-is-idempotent} tells us that $\phi(1)$ is an idempotent, so we need to find all idempotents of $\Z_{28}$. However, we cannot just assume that 0 and 1 are the \textbf{only} idempotents in $\Z_{28}$; we need to check for them exhaustively.

    By exhaustion, we see that $0^2 = 0$, $1^2 = 1$, $8^2 = 64 = 2 \times 28 + 8 = 8$, $21^2 = 441 = 15 \times 28 + 21 = 21$. So the idempotents in $\Z_{28}$ are 0, 1, 8, and 21. This is not enough to narrow down the possible values of $\phi(1)$, so we need to use more facts.

    Recall from part I that $|\phi(1)|_+$ divides $|1|_+$ by \myref{exercise-order-of-homomorphism-divides-order}. Therefore $|\phi(1)|_+$ divides 12. Furthermore, \myref{thrm-order-of-element-in-cyclic-group} tells us that the additive order of an element $k$ in the group $(\Z_n, +)$ is $\frac{n}{\gcd(k,n)}$. So we must exhaust all idempotents in $\Z_{28}$ to check whether they are valid values for $\phi(1)$.

    % \begin{itemize}
    %     \item $|0|_+ = 1$ which divides 12, so 0 is a valid value of $\phi(1)$.
    %     \item $|1|_+ = 28$ which does not divide 12, so 1 is not a valid value of $\phi(1)$. Note that this is different from the previous example, where 1 was a possible value of $\phi(1)$.
    %     \item $|8|_+ = \frac{28}{\gcd(8,28)} = \frac{28}4 = 7$ which does not divide 12, so 8 is not a valid value of $\phi(1)$.
    %     \item $|21|_+ = \frac{28}{\gcd(21,28)} = \frac{28}7 = 4$ which divides 12, so 21 is a valid value of $\phi(1)$.
    % \end{itemize}
    % Hence $\phi(1) = 0$ or $\phi(1) = 21$.

    \begin{multicols}{2}
        \begin{itemize}
            \item $|0|_+ = \frac{28}{\gcd(0,28)} = \frac{28}{28} = 1$
            \item $|1|_+ = \frac{28}{\gcd(1,28)} = \frac{28}{1} = 28$
            \item $|8|_+ = \frac{28}{\gcd(8,28)} = \frac{28}{4} = 7$
            \item $|21|_+ = \frac{28}{\gcd(21,28)} = \frac{28}{7}= 4$
        \end{itemize}
    \end{multicols}

    Of the four idempotents, only $|0|_+ = 1$ and $|21|_+ = 4$ divides 12, which means $\phi(1) = 0$ or $\phi(1) = 21$.

    If $\phi(1) = 0$, then for any $n \in \Z_{12}$,
    \begin{align*}
        \phi(n) &= \phi(\underbrace{1 + 1 + \cdot + 1}_{n \text{ times}})\\
        &= \underbrace{\phi(1) + \phi(1) + \cdot + \phi(1)}_{n \text{ times}}\\
        &= \underbrace{0 + 0 + \cdots + 0}_{n \text{ times}}\\
        &= 0
    \end{align*}
    which means that $\phi(n) = 0$ for all $n \in \Z_{12}$, i.e. $\phi$ is trivial.

    If instead $\phi(1) = 21$, then
    \begin{align*}
        \phi(n) &= \underbrace{\phi(1) + \phi(1) + \cdot + \phi(1)}_{n \text{ times}}\\
        &= \underbrace{21 + 21 + \cdots + 21}_{n \text{ times}}\\
        &= 21n
    \end{align*}
    which means $\phi(n) = 21n$ for all $n \in \Z_{12}$.

    Thus the only homomorphisms $\phi: \Z_{12} \to \Z_{28}$ are $\phi(n) = 0$ and $\phi(n) = 21n$ for all $n \in \Z_{12}$.
\end{example}

\begin{exercise}\label{exercise-homomorphism-over-Q-fixes-elements-of-Q}
    Suppose $R$ and $R'$ are rings such that $\Q$ is a subring of both $R$ and $R'$. Let $\phi: R \to R'$ be a ring homomorphism such that $\phi(1) = 1$. Show that for any $q \in \Q$ we have $\phi(q) = q$.
\end{exercise}

\newpage

\section{Problems}
\begin{problem}\label{problem-integral-domain-iff-trivial-ideal-is-prime}
    Let $R$ be a ring.
    \begin{partquestions}{\roman*}
        \item Show that $R/\{0\} \cong R$.
        \item Prove that $R$ is an integral domain if and only if $\{0\}$ is a prime ideal.
        \item Prove that $R$ is a field if and only if $\{0\}$ is a maximal ideal.
    \end{partquestions}
\end{problem}

\begin{problem}
    Find all ring endomorphisms of $\Q$.
\end{problem}

\begin{problem}
    Show $\Z^2 \not\cong \Q$.
\end{problem}

\begin{problem}
    Show $\Q[\sqrt2] \not\cong \Q[\sqrt3]$.
\end{problem}

\begin{problem}
    Consider the subring
    \[
        R = \left\{\begin{pmatrix}a&0\\0&b\end{pmatrix}\vert a,b \in \Z\right\}
    \]
    of $\Mn{2}{\Z}$. Show that $R \cong \Z^2$.
\end{problem}

\begin{problem}\label{problem-properties-of-ring-isomorphism}
    Let $R$ and $R'$ be rings, and let $\phi: R \to R'$ be a ring isomorphism. Prove or disprove the following statements.
    \begin{partquestions}{\alph*}
        \item $\phi^{-1}: R' \to R$ is a ring isomorphism.
        \item If $R$ has a subring with $n$ elements, then so does $R'$.
        \item If $R$ has an ideal, then so does $R'$.
    \end{partquestions}
\end{problem}

\begin{problem}
    Find all ring endomorphisms of $\Z_{10}$. Hence find all ring automorphisms $\psi$ of $\Z_{10}$.
\end{problem}

\begin{problem}
    Find all ring endomorphisms of $\Q[\sqrt3]$. Hence find all ring automorphisms $\psi$ of $\Q[\sqrt3]$.
\end{problem}

\begin{problem}
    Let $R$ and $R'$ be commutative rings, $I$ be an ideal of $R$, and $\phi: R\to R'$ be a ring homomorphism.
    \begin{partquestions}{\roman*}
        \item Show that $\phi(\sqrt I) \subseteq \sqrt{\phi(I)}$.
        \item If $\phi$ is surjective with $\ker\phi \subseteq I$, prove that $\phi(\sqrt I) = \sqrt{\phi(I)}$.
    \end{partquestions}
\end{problem}

\pagebreak

\begin{problem}\label{problem-ring-isomorphism-2}
    Let $R$ be a ring with a subring $S$ and ideal $I$. Prove that
    \begin{partquestions}{\roman*}
        \item $S+I$ is a subring of $R$;
        \item $S \cap I$ is an ideal of $S$; and
        \item $S/(S\cap I)\cong (S+I)/I$.
    \end{partquestions}
\end{problem}

\begin{problem}\label{problem-ring-isomorphism-3}
    Let $R$ be a ring with ideals $I$ and $J$ such that $I$ is a subset of $J$.
    \begin{partquestions}{\roman*}
        \item Prove that $J/I$ is an ideal of $R/I$.
        \item Prove that $\frac{R/I}{J/I} \cong R/J$.\newline
        (\textit{Note: remember to prove that the map is well-defined.})
    \end{partquestions}
\end{problem}

\section{Polynomial Rings}
\begin{questions}
    \item For brevity let $I = \princ{x} = \{xP(x) \vert P(x) \in \Z[x]\}$. This means that $I$ is the set of polynomials with integer coefficients and with constant term 0. Now suppose $f(x), g(x) \in \Z[x]$; write
    \begin{align*}
        f(x) &= a_0 + a_1x + \cdots + a_mx^m\\
        g(x) &= b_0 + b_1x + \cdots + b_nx^n
    \end{align*}
    where $a_i, b_i \in \Z$ and $m$ and $n$ are positive integers. Note that
    \[
        f(x)g(x) = a_0b_0 + (a_1b_0+a_0b_1)x + \cdots.
    \]
    Now if $f(x)g(x) \in I$, this means that $a_0b_0 = 0$. Hence either $a_0 = 0$ or $b_0 = 0$, meaning that either $f(x)$ has zero constant term (so $f(x) \in I$) or $g(x)$ has zero constant term (so $g(x) \in I$). Thus $I$ is prime.

    \item \begin{partquestions}{\roman*}
        \item Let $f(x), g(x) \in \Z[x]$. Then
        \[
            \phi(f(x) + g(x)) = f(-2) + g(-2) = \phi(f(x)) + \phi(g(x))
        \]
        and
        \[
            \phi(f(x)g(x)) = f(-2)g(-2) = \phi(f(x))\phi(g(x))
        \]
        so $\phi$ is a ring homomorphism.

        \item Note that
        \begin{align*}
            \ker\phi &= \{f(x) \in \Z[x] \vert \phi(f(x)) = 0\}\\
            &= \{f(x) \in \Z[x] \vert f(-2) = 0\}\\
            &= I.
        \end{align*}
        \myref{prop-kernel-is-an-ideal} tells us that $\ker\phi$ is an ideal of $\Z[x]$, so $I$ is an ideal of $\Z[x]$.

        \item We first show that $\phi$ is surjective. Let $n \in \Z$, note that $n$ is a degree zero polynomial, so $n \in \Z[x]$. Clearly $\phi(n) = n$ so $n$ is its own pre-image. Therefore $\im\phi = \Z$.
        
        By FRIT (\myref{thrm-ring-isomorphism-1}),
        \[
            \Z[x]/I \cong \Z.
        \]
        Note that $\Z$ is an integral domain but not a field. Thus $I$ is prime but not maximal.
    \end{partquestions}
\end{questions}


\chapter{Image Acknowledgements}
Unless otherwise stated, all images are the author's own work, and are released under the same licence as this book.

\begin{itemize}
    \item Image of the decorative knot with twelve crossings was modified from the one by \code{AnonMoos} on Wikimedia. You can find the original file at \url{https://tinyurl.com/crossings-diagram}.
\end{itemize}


\printbibliography[heading=bibintoc, title={References and Bibliography}]
\printindex

\end{document}
