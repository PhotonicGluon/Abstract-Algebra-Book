\section{Homomorphisms and Isomorphisms}
\begin{questions}
    \item \begin{partquestions}{\alph*}
        \item No, since $\phi(m+n) = m + n$ while $\phi(m)\phi(n) = mn \neq m+n$.
        \item Yes, since $\phi(m+n) = 2^{m+n} = 2^m2^n = \phi(m)\phi(n)$.
    \end{partquestions}

    \item Clearly $e_2 \in \phi(H_1)$ since $e_2 = \phi(e_1)$ and $e_1 \in H_1$. Now suppose $x$ and $y$ are in $\phi(H_1)$, meaning that $\phi(h_x) = x$ and $\phi(h_y) = y$ for some $h_x$ and $h_y$ in $H$. So $h_xh_y^{-1}$ is in $H$. Furthermore,
    \begin{align*}
        \phi(h_xh_y^{-1}) &= \phi(h_x)\phi(h_y^{-1})\\
        &= \phi(h_x)\left(\phi(h_y)\right)^{-1}\\
        &= xy^{-1},
    \end{align*}
    meaning that $xy^{-1}$ is in $\phi(H_1)$. Therefore, by subgroup test, $\phi(H_1) \leq G_2$.

    \item Disprove. Let $G_1 = H_1 = \mathbb{Z}$ be the additive group of integers and let $G_2 = H_2 = D_n$, the dihedral group of order $2n$. Consider the map $\phi: G_1 \to G_2$ where $\phi(m) = s^m$. Clearly, $H_1 \unlhd G_1$. Note that $\phi(H_1) = \{e, s\} = \langle s \rangle$. From \myref{example-normal-subgroups-of-d3}, we know that $\langle s \rangle$ is not a normal subgroup of $D_3 = G_2$, so $\phi(H_1)$ is not a normal subgroup of $G_2$.

    \item Suppose $|a| = n$. Note that
    \[
        \left(\phi(a)\right)^n = \phi\left(a^n\right) = \phi(e_G) = e_H
    \]
    so $|\phi(a)|$ divides $n = |a|$ by \myref{problem-element-to-power-of-multiple-of-order-is-identity}.

    \item \begin{partquestions}{\roman*}
        \item Since $3^0 = 1$, $3^1 = 3$, $3^2 = 9 \equiv 4 \pmod{5}$, and $3^3 = 27 \equiv 2 \pmod{5}$, thus $G = \langle 3 \rangle$. Since $7^0 = 1$, $7^1 = 7$, $7^2 = 49 \equiv 9 \pmod{10}$, and $7^3 = 343 \equiv 3 \pmod{10}$, thus $H = \langle 7 \rangle$.
        \item We need to prove that it is a homomorphic bijection.
        \begin{itemize}
            \item \textbf{Homomorphism}:
            \begin{align*}
                \phi(3^m3^n) &= \phi(3^{m+n})\\
                &= 7^{m+n}\\
                &= 7^m7^n\\
                &= \phi(3^m)\phi(3^n)
            \end{align*}

            \item \textbf{Bijection}: Note that $1 \mapsto 1$, $3 \mapsto 7$, $4 \mapsto 9$, $2 \mapsto 3$ which clearly shows that $\phi$ is bijective.
        \end{itemize}
        Therefore $\phi$ is an isomorphism, meaning $G \cong H$.
    \end{partquestions}

    \item Suppose $N \unlhd G$ such that $|N| = k$. Then $\phi(N)$ is a subgroup of $H$ with order $k$ by the theorem. All that remains to prove is that $\phi(N)$ is normal.

    Let $n \in N$ and $\hat{n} \in \phi(N)$ such that $\hat{n} = \phi(n)$. Let $h \in H$ be an arbitrary element. To prove that $h\hat{n}h^{-1}$ is in $\phi(N)$.

    Let $g$ be in $G$ such that $\phi(g) = h$. Then
    \begin{align*}
        h\hat{n}h^{-1} &= \phi(g)\phi(n)\phi(g^{-1})\\
        &= \phi(\underbrace{gng^{-1}}_{\text{In } N})\\
        &\in \phi(N)
    \end{align*}
    which proves that $\phi(N)$ is normal. Hence there exists a normal subgroup of order $k$, namely $\phi(N)$.

    \item Since $7^0 = 1$, $7^1 = 7$, $7^2 = 49 \equiv 9 \pmod{10}$, and $7^3 = 343 \equiv 3 \pmod{10}$, thus $G = \langle 7 \rangle$. Note $|7| = 4$ so $G \cong \mathbb{Z}_4$, i.e. $n = 4$.
\end{questions}