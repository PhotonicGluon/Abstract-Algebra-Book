\section{Sylow Theorems}
\begin{questions}
    \item We note that $12 = 2^2 \times 3$. Thus a Sylow 2-subgroup must have order 4. Clearly $|3| = 4$ so $\langle 3 \rangle = \{0, 3, 6, 9\}$ is the Sylow 2-subgroup of $\mathbb{Z}_{12}$.

    \item Recall that $|\Sn{5}| = 120 = 2^3 \times 3 \times 5$. By the First Sylow Theorem (\myref{thrm-sylow-1}), $\Syl{p}{G} \neq \emptyset$ if $p$ is 2, 3, or 5.

    \item We prove this by constructing the map $\phi: H \to gHg^{-1}$ where $h \mapsto ghg^{-1}$. We note that $\phi$ is an isomorphism.
    \begin{itemize}
        \item \textbf{Homomorphism}: Let $x, y \in H$. Then
        \[
            \phi(xy) = g(xy)g^{-1} = (gxg^{-1})(gyg^{-1}) = \phi(x)\phi(y)
        \]
        which clearly means that $\phi$ is an isomorphism.
        \item \textbf{Injective}: Suppose $x, y \in H$ such that $\phi(x) = \phi(y)$. Then $gxg^{-1} = gyg^{-1}$ which quickly implies $x = y$ by cancellation law.
        \item \textbf{Surjective}: Suppose $ghg^{-1} \in gHg^{-1}$. Clearly we have $\phi(h) = ghg^{-1}$, so any element in $gHg^{-1}$ has a pre-image inside $H$.
    \end{itemize}
    Hence $H \cong gHg^{-1}$.

    \item By \myref{prop-order-of-conjugate-element-equals-order-of-element} we know that $|xyx^{-1}| = |y|$ for all $x, y \in G$. Substituting $x = g$, and $y = hg$ yields
    \[
        |xyx^{-1}| = |gh gg^{-1}| = |gh| \text{ and } |y| = |hg|
    \]
    so the result follows.

    \item Clearly $e \in \N{G}{S}$ since $eSe^{-1} = S$. Consider $x, y \in \N{G}{S}$, meaning that $xSx^{-1} = S$ and $ySy^{-1} = S$. Note that $y^{-1} \in \N{G}{S}$ since
    \begin{align*}
        y^{-1}S\left(y^{-1}\right)^{-1} &= y^{-1}Sy\\
        &= y^{-1}\left(ySy^{-1}\right)y & (y \in \N{G}{S})\\
        &= (y^{-1}y)S(y^{-1}y)\\
        &= S.
    \end{align*}
    Therefore
    \begin{align*}
        \left(xy^{-1}\right)S\left(xy^{-1}\right)^{-1} &= \left(xy^{-1}\right)S\left(yx^{-1}\right)\\
        &= x\left(y^{-1}Sy\right)x^{-1}\\
        &= xSx^{-1} & (y^{-1} \in \N{G}{S})\\
        &= S & (x \in \N{G}{S})
    \end{align*}
    which means that $xy^{-1} \in \N{G}{S}$. Hence, by the subgroup test, we have $\N{G}{S} \leq G$.

    \item By the Second Sylow Theorem (\myref{thrm-sylow-2}), we know that $gHg^{-1} = K$. Since $H \cong gHg^{-1}$ by \myref{exercise-conjugate-subgroup-isomorphic-to-subgroup} thus $H \cong gHg^{-1} = K$ as required.

    \item We note $784 = 2^4 \times 7^2$, so $m = 16$, $p = 7$, and $k = 2$. By the Third Sylow Theorem (\myref{thrm-sylow-3}), we know that
    \begin{itemize}
        \item $n_7 = [G : \N{G}{P}] = \frac{|G|}{|\N{G}{P}|}$;
        \item $n_7 \mid 16$, which implies $n_7 \in \{1, 2, 4, 8, 16\}$; and
        \item $n_7 \equiv 1 \pmod 7$, which implies $n_7 \in \{1, 8, 15, 22, \dots\}$.
    \end{itemize}
    Hence $n_7 = 1$ or $n_7 = 8$. But since $P$ is not a normal subgroup of $G$, by \myref{corollary-sylow-subgroup-is-normal-if-it-is-unique}, $P$ cannot be the only Sylow 7-subgroup, meaning $n_7 \neq 1$. Hence $n_7 = 8$, so
    \[
        8 = n_7 = \frac{|G|}{|\N{G}{P}|} = \frac{784}{|\N{G}{P}|}
    \]
    which means that $|\N{G}{P}| = 98$.

    \item Note $130 = 2 \times 5 \times 13$. Consider the number of Sylow 13-subgroups, $n_{13}$. The Third Sylow Theorem (\myref{thrm-sylow-3}) tells us that
    \begin{itemize}
        \item $n_{13} \mid 2 \times 5 = 10$, so $n_{13} \in \{1, 2, 5, 10\}$, and
        \item $n_{13} \equiv 1 \pmod{13}$ so $n_{13} \in \{1, 14, 27, \dots\}$.
    \end{itemize}
    Hence $n_{13} = 1$. But by \myref{corollary-sylow-subgroup-is-normal-if-it-is-unique} this means that the only Sylow 13-subgroup is normal. Hence a group of order 130 is non-simple.
\end{questions}
