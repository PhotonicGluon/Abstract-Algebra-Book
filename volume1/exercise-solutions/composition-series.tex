\section{Composition Series}
\begin{questions}
    \item \begin{partquestions}{\roman*}
        \item One sees clearly that $\{0, 2\}$ is the only proper normal subgroup of $G$, so the subnormal series of length 2 is $1 \lhd \{0, 2\} \lhd G$.
        \item There are 2 factors of the above subnormal series. The first is $\{0, 2\} / 1 = \{0, 2\} \cong \mathbb{Z}_2$ and the second is
        \begin{align*}
            G / \{0, 2\} &= \{g \oplus_4 \{0, 2\} \vert g \in G\}\\
            &= \{\{0, 2\}, \{1, 3\}, \{2, 0\}, \{3, 1\}\}\\
            &= \{\{0, 2\}, \{1, 3\}\}\\
            &= \langle \{1, 3\} \rangle\\
            &\cong \mathbb{Z}_2.
        \end{align*}
        \item Since $1 \lhd G$ and $\{0, 2\} \lhd G$ thus the subnormal series in \textbf{(i)} is also a normal series of $G$.
    \end{partquestions}
    
    \item By Lagrange's Theorem (\myref{thrm-lagrange}) the order of a subgroup must divide the order of the group. Furthermore $\mathbb{Z}_{120}$ is abelian, so any subgroup of it is normal. Now the subgroup $N = \{0, 2, 4, \dots, 118\}$ has 60 elements which is the maximum possible guaranteed by Lagrange. Hence $N$ is the maximal normal subgroup of $\mathbb{Z}_{120}$, which has order 60.
    
    \item $\Cn{6}$ has these two composition series up to isomorphism
    \begin{align*}
        &1 \lhd \Cn{2} \lhd \Cn{6} \text{ and }\\
        &1 \lhd \Cn{3} \lhd \Cn{6}.
    \end{align*}
    In both cases, their composition length is 2. Their respective composition factors are:
    \begin{itemize}
        \item $\Cn{2} / 1 \cong \Cn{2}$ and $\Cn{6} / \Cn{2} \cong \Cn{3}$ by \myref{exercise-Zmn-mod-Zn-cong-Zn}; and
        \item $\Cn{3} / 1 \cong \Cn{3}$ and $\Cn{6} / \Cn{3} \cong \Cn{2}$ by \myref{exercise-Zmn-mod-Zn-cong-Zn},
    \end{itemize}
    up to isomorphism.
    
    \item Let the group in question be $G$. We know by Cauchy's Theorem (\myref{thrm-cauchy}) and \myref{exercise-group-of-order-multiple-of-prime-has-subgroup-of-prime-order}, and by writing $p^2$ as $p \times p$, that $G$ has a subgroup of order $p$ (call this $H$).
    
    Lagrange's Theorem (\myref{thrm-lagrange}) tells us that the possible orders of the subgroups of $G$ are 1, $p$, and $p^2$. These subgroups are $\{e\}$, $H$, and $G$ respectively. Furthermore, by \myref{problem-group-of-order-prime-squared-is-abelian}, $G$ must be abelian, thereby its subgroups are all normal (\myref{prop-subgroup-of-abelian-group-is-normal}). Finally, a corollary of Lagrange's Theorem (\myref{corollary-group-with-prime-order-subgroups}) says that the only subgroups of $H$ are the trivial group and the group itself. Hence, $G$ has only one composition series, namely $1 \lhd H \lhd G$.
\end{questions}
