%=============== Packages ================
% Fonts
\usepackage[utf8]{inputenc}
\usepackage[T1]{fontenc}
\usepackage{lmodern}
\usepackage{tgpagella}
\usepackage{mathpazo}

% Math formatting
\usepackage{mathtools}
\usepackage{amsfonts}
\usepackage{amsmath}
\usepackage{amssymb}
\usepackage{amsthm}
\usepackage{thmtools}

% Page formatting, colour and images
\usepackage{geometry}
\usepackage{graphicx}
\usepackage{tocloft}
\usepackage[x11names]{xcolor}
\usepackage{wrapfig}
\usepackage{cutwin}
\usepackage{fancyhdr}
\usepackage{emptypage}
\usepackage{imakeidx}
\usepackage{mdframed}
\usepackage{eso-pic}
\usepackage{float}

% Hyperlinks and references
\usepackage{hyperref}
\usepackage[capitalise, nameinlink]{cleveref}
\usepackage{crossreftools}
\usepackage[
    backend=bibtex,
    style=alphabetic,
    sorting=ynt
]{biblatex}
\usepackage[
    totoc,
    columnsep=20pt,
    hangindent=8pt,
    subindent=20pt,
    subsubindent=30pt
]{idxlayout}
\usepackage[record,symbols]{glossaries-extra}

% Miscellaneous
\usepackage{multicol}
\usepackage{multirow}
\usepackage{enumitem}

%============= PDF Metadata ==============
\hypersetup{
    pdftitle={A Complete Introduction To Abstract Algebra},
    pdfauthor={Kan Onn Kit},
    pdfsubject={Abstract Algebra},
    breaklinks=true
}

%=============== Geometry ================
\makeatletter
\Ifstr{\isebook}{true}{
    % eBook geometry
    \newgeometry{inner=2cm, outer=2cm, top=2.5cm, bottom=2cm}
}{
    % Normal paper geometry
    \newgeometry{inner=2.5cm, outer=1.5cm, top=2.5cm, bottom=2cm}
}
\makeatother

%============== Resources ================
% Bibliography
\addbibresource{abstract-algebra.bib}

% Graphics locations
\graphicspath{
    % Main location
    {images}
    % Part 0
    {part0/images}
    % Part 1
    {part1/images}
    {part1/images/intro-to-groups}
    {part1/images/basics-of-groups}
    {part1/images/homomorphisms-and-isomorphisms}
    {part1/images/symmetry-groups}
    {part1/images/further-homomorphisms}
    {part1/images/group-actions}
    % Part 2
    {part2/images}
    {part2/images/ring-homomorphisms}
    {part2/images/domains}
    % Part 3
    {part3/images}
    {part3/images/algebraic-extensions}
    {part3/images/finite-fields}
    {part3/images/geometric-constructions}
    % Part 4
    {part4/images}
}

%============= Formatting ================
% Line format
\linespread{1.05}

% Indent
\setlength{\parindent}{1.5em}
\newlength{\normalparindent}
\setlength{\normalparindent}{\parindent}

% Heading format
\setkomafont{sectioning}{\sffamily\bfseries\boldmath}

%=========== (Re)definitions =============
% Part page configuration
\let\oldpart\part
\renewcommand*{\part}[1]{\cleardoubleoddpage\oldpart{#1}\cleardoubleemptypage}
\newcommand{\unnumberedpart}[1]{
    \cleardoubleoddpage
    \addcontentsline{toc}{part}{#1}
    \oldpart*{#1}
    \cleardoubleemptypage
}

\makeatletter
\patchcmd{\set@@@@preamble}{#6}{\setlength{\parindent}{\normalparindent}#6}{}{}  % Make paragraph indent correctly for part preambles
\makeatother

% Symbols redefinitions
\let\oldemptyset\emptyset
\let\emptyset\varnothing

\let\totient\varphi

\renewcommand{\vert}{ \;\: \vline \;\: }
\newcommand{\vertalt}{ \;\: | \;\: }

\newcommand{\myref}[1]{\textbf{\crthypercref{#1}}}
\newcommand{\myreffigures}[1]{\textbf{\cref{#1}}}

\newcommand{\qedproof}{\ensuremath{\blacksquare}}
\newcommand{\qedsketch}{\ensuremath{\square}}
\renewcommand{\qedsymbol}{\qedproof}  % Actually changes the QED symbol

% Quote definitions
\newcommand{\quoteattr}[4][0.5cm]{
    \vspace{#1}
    \begin{center}
        \parbox{8cm}{
            {\itshape#2}\\
            \null\hfill--- #3\\
            \vspace{-0.1cm}
            \hfill\small(#4)
        }
    \end{center}
    \vspace{#1}
}

% PDF-TeX image definitions
\newcommand{\pdfteximg}[3][10pt]{
    {
        \fontsize{#1}{0pt}\selectfont
        \def\svgwidth{#2}
        \input{#3}
    }
}

\newcommand{\pdfteximgframed}[3][10pt]{
    {
        \fontsize{#1}{0pt}\selectfont
        \def\svgwidth{#2}
        \fbox{\input{#3}}
    }
}

%=========== Theorem Things ==============
% 'Results' declarations
\newcommand{\makenewresultstyle}[3]{
    \declaretheoremstyle[
        headfont=\normalfont\bfseries,
        bodyfont=\normalfont,
        notefont=\normalfont\bfseries\boldmath\itshape,
        spaceabove=0pt,  % Space between previous paragraph and current block
        spacebelow=0pt,  % Space between current block and next paragraph
        mdframed={
            linecolor=#3,
            skipabove=3pt,  % Space between top of block and beginning of coloured frame
            skipbelow=3pt,  % Space between bottom of block and beginning of coloured frame
            backgroundcolor=#2,
            usetwoside=false,  % Needed for `leftmargin` and `rightmargin` to work
            leftmargin=-5pt,
            rightmargin=-5pt,
            innerleftmargin=5pt,
            innerrightmargin=5pt
        }
    ]{#1-style}
}

\makenewresultstyle{theorem}{DarkSeaGreen2}{DarkSeaGreen4}
\declaretheorem[name=Theorem,style=theorem-style,within=section]{theorem}
\renewcommand*{\thetheorem}{\arabic{chapter}.\arabic{section}.\arabic{theorem}}

\makenewresultstyle{lemma}{AntiqueWhite1}{Bisque4}
\declaretheorem[style=lemma-style,sibling=theorem]{lemma}

\makenewresultstyle{proposition}{Honeydew1}{SpringGreen4}
\declaretheorem[style=proposition-style,sibling=theorem]{proposition}

\makenewresultstyle{corollary}{Cornsilk1}{Ivory4}
\declaretheorem[style=corollary-style,sibling=theorem]{corollary}

\makenewresultstyle{definition}{LightCyan1}{Cyan4}
\declaretheorem[style=definition-style,sibling=theorem]{definition}

\makenewresultstyle{axiom}{Thistle2}{Plum4}
\declaretheorem[style=axiom-style,sibling=theorem]{axiom}

% 'Questions' declarations
\declaretheoremstyle[
    headfont=\normalfont\bfseries,
    bodyfont=\normalfont,
    spaceabove=5pt,  % Space between previous paragraph and current block
    prefoothook={\vspace{5pt}},
    mdframed={
        skipabove=3pt,  % Space between top of block and beginning of frame
        skipbelow=3pt,  % Space between bottom of block and beginning of frame
        usetwoside=false,  % Needed for `leftmargin` and `rightmargin` to work
        leftmargin=-5pt,
        rightmargin=-5pt,
        innerleftmargin=5pt,
        innerrightmargin=5pt
    }
]{exercise-style}
\declaretheorem[style=exercise-style,within=chapter]{exercise}
\renewcommand*{\theexercise}{\arabic{chapter}.\arabic{exercise}}

\declaretheoremstyle[
    headfont=\normalfont\bfseries,
    bodyfont=\normalfont,
    notefont=\normalfont\bfseries\itshape,
    spaceabove=10pt,  % Space between previous paragraph and current block
    spacebelow=10pt,  % Space between current block and next paragraph
]{problem-style}
\declaretheorem[name=Problem,style=problem-style,within=chapter]{problem}
\renewcommand*{\theproblem}{\arabic{chapter}.\arabic{problem}}

% Miscellaneous declarations
\declaretheoremstyle[
    headfont=\normalfont\bfseries,
    bodyfont=\normalfont,
    spaceabove=10pt,  % Space between previous paragraph and current block
    spacebelow=10pt,  % Space between current block and next paragraph
]{general-style}
\declaretheorem[style=general-style,sibling=theorem]{example}
\declaretheorem[style=general-style,numbered=no]{remark}

%============ Environments ===============
% Questions environments
\newenvironment{questions}
{\begin{enumerate}[label=\textbf{\arabic*.}]}
{\end{enumerate}}

\newenvironment{partquestions}[1]
{\begin{enumerate}[label=\textbf{(#1)}]}
{\end{enumerate}}

% Proof environments
\newenvironment{proofsketch}
{\begin{proof}[Sketch of Proof.]}
{\renewcommand{\qedsymbol}{\qedsketch}\end{proof}\renewcommand{\qedsymbol}{\qedproof}}

% Miscellaneous environments
\newenvironment{examplewithcutout}[6]
{
    \vspace{\baselineskip}
    \begin{example}
    \renewcommand\windowpagestuff{#6}
    \Ifstr{#1}{left}{\opencutleft}{\Ifstr{#1}{right}{\opencutright}{\opencutcenter}}
    \begin{cutout}{#2}{#3}{#4}{#5}
}
{\end{cutout}\end{example}}

%=========== Custom Commands =============
\newcommand{\ZeroRoman}[1]{\ifcase\value{#1}\relax0\else\Roman{#1}\fi}  % Roman numeral that works with zero
\newcommand{\code}[1]{\texttt{#1}}  % Code block

%=========== General Symbols =============
\newcommand{\lcm}{\mathrm{lcm}}  % Lowest common multiple function
\newcommand{\sgn}{\mathrm{sgn}}  % Signum function

\newcommand{\im}{\mathrm{im}\;}  % Image of a function
\newcommand{\id}{\mathrm{id}}    % Identity function

\newcommand{\Z}{\mathbb{Z}}  % The set of integers
\newcommand{\Q}{\mathbb{Q}}  % The set of rational numbers
\newcommand{\R}{\mathbb{R}}  % The set of real numbers
\newcommand{\C}{\mathbb{C}}  % The set of complex numbers

%================ Part I =================
\newcommand{\An}[1]{\mathrm{A}_{#1}}                  % Alternating group of degree n
\newcommand{\Aut}[1]{\mathrm{Aut}(#1)}                % Group of automorphisms of G
\newcommand{\Centralizer}[2]{\mathrm{C}_{#1}(#2)}     % Centralizer of an element in G
\newcommand{\Cl}[1]{\mathrm{Cl}(#1)}                  % Conjugacy class of the element x
\newcommand{\Cn}[1]{\mathrm{C}_{#1}}                  % Cyclic group of order n
\newcommand{\GL}[2]{\mathrm{GL}_{#1}\!\left(#2\right)}  % General Linear Group of degree n
\newcommand{\Inn}[1]{\mathrm{Inn}(#1)}                % Group of inner automorphisms of G
\newcommand{\N}[2]{\mathrm{N}_{#1}(#2)}               % Normalizer of S in G
\newcommand{\Out}[1]{\mathrm{Out}(#1)}                % Group of outer automorphisms of G
\newcommand{\SL}[2]{\mathrm{SL}_{#1}\!\left(#2\right)}  % Special Linear Group of degree n
\newcommand{\Sn}[1]{\mathrm{S}_{#1}}                  % Symmetric group of degree n
\newcommand{\Syl}[2]{\mathrm{Syl}_{#1}(#2)}           % Set of Sylow p-groups of G
\newcommand{\Sym}[1]{\mathrm{Sym}(#1)}                % Symmetric group of a set
\newcommand{\Un}[1]{\mathcal{U}_{#1}}                 % Group of units modulo n
\newcommand{\CenterGrp}[1]{\mathrm{Z}(#1)}            % Center of a group G

\newcommand{\Stab}[2]{\mathrm{Stab}_{#1}(#2)}         % Stabilizer of x by G
\newcommand{\Fix}[2]{\mathrm{Fix}_{#1}(#2)}           % Set of all elements in X which is fixed by g
\newcommand{\Orb}[2]{\mathrm{Orb}_{#1}(#2)}           % Orbit of x under G

%================ Part II ================
\newcommand{\ideal}[1]{\mathfrak{#1}}                % Ideal of a ring
\newcommand{\princ}[1]{\left\langle#1\right\rangle}  % Principal ideal generated by the element

\newcommand{\Mn}[2]{\mathcal{M}_{#1\times#1}(#2)}  % The ring of n by n matrices with entries in the ring R
\newcommand{\ZeroM}[1]{\textbf{0}_{#1}}      % Zero matrix
\newcommand{\IdentityM}[1]{\textbf{I}_{#1}}  % Identity matrix

% \renewcommand{\H}{\mathbb{H}}   % Quaternion Ring
% \newcommand{\qi}{\textbf{i}}    % Quaternion i
% \newcommand{\qj}{\textbf{j}}    % Quaternion j
% \newcommand{\qk}{\textbf{k}}    % Quaternion k

\newcommand{\Ann}[2]{\mathrm{Ann}_{#1}(#2)}  % Annihilator of a subset A of a ring R
\newcommand{\Char}[1]{\mathrm{char}(#1)}     % Characteristic of a ring R
\newcommand{\Nilr}[1]{\mathfrak{N}_{#1}}     % Nilradical of a ring R
\newcommand{\Frac}[1]{\mathrm{Frac}(#1)}     % Field of fractions of an integral domain D

%=============== Part III ================
\newcommand{\Span}[1]{\mathrm{span}(#1)}  % Span of a set S of vectors
\let\olddim\dim
\renewcommand*{\dim}[1]{\olddim(#1)}      % Dimension of a vector space V
\newcommand{\GF}[1]{\mathrm{GF}(#1)}      % Galois field of order n

%================ Part IV ================
\newcommand{\Gal}[1]{\mathrm{Gal}(#1)}  % Galois group of E/F


\newcommand{\GF}[1]{\mathrm{GF}(#1)}      % Galois field of order n

%======== Custom Chapter Styling =========
% Fix part numbering
\renewcommand*{\thepart}{\ZeroRoman{part}}

\makeatletter
% Set chapter page format
\renewcommand{\chaptermark}[1]{
    \markboth{\if@mainmatter\chapapp~\thechapter.\ \fi#1}{}
}

\renewcommand*{\chapterformat}{
    \MakeUppercase{\chapapp\nobreakspace\thechapter}
}

\renewcommand*{\chapterlineswithprefixformat}[3]{
    \Ifstr{#1}{chapter}{
        \vspace{-60px}
        \Ifstr{#2}{\empty}{\vspace{40px}}{\raggedleft#2}
        \vspace{-15px}
        \rule{\linewidth}{1pt}\par\nobreak
        \centering{#3}
        \vspace{-10px}
        \rule{\linewidth}{1pt}\par\nobreak
        \vspace{-10px}
    }{#2#3}
}

% Set part page format
\renewcommand*{\partformat}{
    {\fontsize{28pt}{0pt}\selectfont \MakeUppercase{\partname\nobreakspace\thepart}}
}

\renewcommand*{\partlineswithprefixformat}[3]{
    \Ifstr{#1}{part}{
        \Ifstr{#2}{}{  % Doesn't have a part title
            \begin{mdframed}[linewidth=2pt]
                \centering#3
            \end{mdframed}
        }{
            \vspace{-250px}
            \begin{mdframed}[linewidth=2pt]
                \begin{center}
                    \vspace{5px}
                    #2
                    \vspace{-25px}
                    #3
                \end{center}
            \end{mdframed}
        }
    }{#2#3}
}

\makeatother

%=========== Hyperlink Setup =============
\makeatletter
\Ifstr{\hideboxesforlinks}{true}{
    \hypersetup{
        pdfborder = {0 0 0}
    }
}{}
\makeatother

%======== Figure Caption Format ==========
\usepackage[labelfont=bf]{caption}
\DeclareCaptionLabelFormat{custom}{#1 #2.}
\captionsetup{labelformat=custom,labelsep=space}

%============ Custom Header ==============
\fancypagestyle{plain}{\fancyhf{}\renewcommand{\headrulewidth}{0pt}}  % To clear page numbers from footer, and header line at the start of every chapter

\pagestyle{fancy}
\fancyhf{}  % Clear header/footer

\fancyhead[EL,OR]{\thepage}
\fancyhead[OL,ER]{\textit{\nouppercase\leftmark}}

\newcommand{\draftstartmark}{\fancyfoot[EC,OC]{\textsc{Work In Progress (Build \buildnumber)}}}
\newcommand{\draftendmark}{\fancyfoot[EC,OC]{}}

%====== Customise Table of Contents ======
% Customise part styling in table of contents
\newlength\parttoclen
\renewcommand\cftpartpresnum{Part~}
\settowidth\parttoclen{\bfseries\cftpartpresnum\cftpartaftersnum}
\addtolength\cftpartnumwidth{\parttoclen+1em}  % +1em to make it nicer and more spaced out

% Heading customisation
\makeatletter
\def\createtoc{
    \renewcommand\tableofcontents{
        \chapter*{\contentsname}
        \@starttoc{toc}
    }
    \tableofcontents
}
\makeatother

%============= Index Pages ===============
\makeindex[options= -s index-style.ist]

%======= Bibliography Formatting =========
% These two lines are here to ensure that URLs do not exceed the page by too much
\setcounter{biburllcpenalty}{7000}
\setcounter{biburlucpenalty}{8000}
