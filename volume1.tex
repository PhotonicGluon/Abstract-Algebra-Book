\documentclass[
    b5paper,
    pagesize,
    10pt,
    bibtotoc,
    normalheadings,
    twoside,
    openany,
    chapterprefix,
    DIV=9
]{scrbook}

%=============== Packages ================
% Fonts
\usepackage[T1]{fontenc}
\usepackage[utf8]{inputenc}
\usepackage{tgpagella}
\usepackage{mathpazo}

% Math formatting
\usepackage{mathtools}
\usepackage{amsfonts}
\usepackage{amsmath}
\usepackage{amssymb}
\usepackage{amsthm}
\usepackage{thmtools}

% Page formatting, colour and images
\usepackage[inner=1.5cm, outer=2.5cm, vmargin=2.5cm]{geometry}
\usepackage{graphicx}
\usepackage{tocloft}
\usepackage[x11names]{xcolor}
\usepackage{wrapfig}
\usepackage{parskip}
\usepackage{fancyhdr}
\usepackage{emptypage}
\usepackage{imakeidx}
\usepackage{mdframed}
\usepackage{eso-pic}

% Hyperlinks and references
\usepackage[breaklinks]{hyperref}
\usepackage[capitalise, nameinlink]{cleveref}
\usepackage{crossreftools}
\usepackage[
    backend=bibtex,
    style=alphabetic,
    sorting=ynt
]{biblatex}

% Miscellaneous
\usepackage{multicol}
\usepackage{enumitem}

%============== Resources ================
\addbibresource{abstract-algebra.bib}

%============= Formatting ================
\linespread{1.05}

%============ Redefinitions ==============
\let\oldemptyset\emptyset
\let\emptyset\varnothing

\let\totient\varphi

\renewcommand{\vert}{ \ \vline \ }
\newcommand{\vertalt}{ \ | \ }

\newcommand{\myref}[1]{\textbf{\crthypercref{#1}}}
\newcommand{\myreffigures}[1]{\textbf{\cref{#1}}}

\renewcommand{\qedsymbol}{\ensuremath{\blacksquare}}
\newcommand{\qedsymbolalt}{\ensuremath{\square}}

%=========== Theorem Things ==============
% 'Results' declarations
\newcommand{\makenewresultstyle}[2]{
    \declaretheoremstyle[
        headfont=\normalfont\bfseries,
        bodyfont=\normalfont,
        notefont=\normalfont\bfseries\itshape,
        spaceabove=0pt,  % Space between previous paragraph and current block
        spacebelow=0pt,  % Space between current block and next paragraph
        mdframed={
            skipabove=3pt,  % Space between top of block and beginning of coloured frame
            skipbelow=3pt,  % Space between bottom of block and beginning of coloured frame
            hidealllines=true,
            backgroundcolor=#2,
            usetwoside=false,  % Needed for `leftmargin` and `rightmargin` to work
            leftmargin=-5pt,
            rightmargin=-5pt,
            innerleftmargin=5pt,
            innerrightmargin=5pt
        }
    ]{#1-style}
}

\makenewresultstyle{theorem}{DarkSeaGreen2}
\declaretheorem[name=Theorem,style=theorem-style,within=section]{theorem}
\renewcommand*{\thetheorem}{\ZeroRoman{part}.\arabic{chapter}.\arabic{section}.\arabic{theorem}}

\makenewresultstyle{lemma}{Honeydew2}
\declaretheorem[style=lemma-style,sibling=theorem]{lemma}

\makenewresultstyle{proposition}{Honeydew1}
\declaretheorem[style=proposition-style,sibling=theorem]{proposition}

\makenewresultstyle{corollary}{DarkSeaGreen1}
\declaretheorem[style=corollary-style,sibling=theorem]{corollary}

\makenewresultstyle{definition}{LightCyan1}
\declaretheorem[style=definition-style,sibling=theorem]{definition}

\makenewresultstyle{axiom}{Thistle2}
\declaretheorem[style=axiom-style,sibling=theorem]{axiom}

% 'Questions' declarations
\declaretheoremstyle[
    headfont=\normalfont\bfseries,
    bodyfont=\normalfont,
    spaceabove=5pt,  % Space between previous paragraph and current block
    prefoothook={\vspace{5pt}},
    mdframed={
        skipabove=10pt,  % Space between top of block and beginning of frame
        skipbelow=10pt,  % Space between bottom of block and beginning of frame
        usetwoside=false,
        innerleftmargin=10pt,
        innerrightmargin=10pt
    }
]{exercise-style}
\declaretheorem[style=exercise-style,within=chapter]{exercise}
\renewcommand*{\theexercise}{\ZeroRoman{part}.\arabic{chapter}.\arabic{exercise}}

\declaretheoremstyle[
    headfont=\normalfont\bfseries,
    bodyfont=\normalfont,
    notefont=\normalfont\bfseries\itshape,
    spaceabove=0pt,  % Space between previous paragraph and current block
    spacebelow=0pt,  % Space between current block and next paragraph
]{problem-style}
\declaretheorem[name=Problem,style=problem-style,within=chapter]{problem}
\renewcommand*{\theproblem}{\ZeroRoman{part}.\arabic{chapter}.\arabic{problem}}

% Miscellaneous declarations
\declaretheoremstyle[
    headfont=\normalfont\bfseries,
    bodyfont=\normalfont
]{general-style}
\declaretheorem[style=general-style,sibling=theorem]{example}
\declaretheorem[style=general-style,numbered=no]{remark}

%============ Environments ===============
\newenvironment{questions}
{\begin{enumerate}[label=\textbf{\arabic*.}]}
{\end{enumerate}}

\newenvironment{partquestions}[1]
{\begin{enumerate}[label=\textbf{(#1)}]}
{\end{enumerate}}

%=========== Custom Commands =============
\newcommand{\ZeroRoman}[1]{\ifcase\value{#1}\relax0\else\Roman{#1}\fi}  % Roman numeral

\newcommand{\code}[1]{\texttt{#1}}  % Code block

\newcommand{\lcm}{\mathrm{lcm}}  % Lowest common multiple function
\newcommand{\sgn}{\mathrm{sgn}}  % Signum function

\newcommand{\im}{\mathrm{im}\;}  % Image of a function
\newcommand{\id}{\mathrm{id}}    % Identity function

%======== Custom Chapter Styling =========
\makeatletter
\renewcommand{\chaptermark}[1]{
    \markboth{\if@mainmatter\chapapp~\thechapter.\ \fi#1}{}
}

\renewcommand*{\chapterformat}{
  \MakeUppercase{\chapapp\nobreakspace\thechapter}
}

\renewcommand*{\chapterlineswithprefixformat}[3]{
    \Ifstr{#1}{chapter}{
        \vspace{-60px}
        \Ifstr{#2}{\empty}{\vspace{40px}}{\raggedleft#2}
        \vspace{-15px}
        \rule{\linewidth}{1pt}\par\nobreak
        \centering{#3}
        \vspace{-10px}
        \rule{\linewidth}{1pt}\par\nobreak
        \vspace{-10px}
    }{#2#3}
}
\makeatother

%======== Figure Caption Format ==========
\usepackage[labelfont=bf]{caption}
\DeclareCaptionLabelFormat{custom}{#1 \ZeroRoman{part}.#2.}
\captionsetup{labelformat=custom,labelsep=space}

%============ Custom Header ==============
\fancypagestyle{plain}{\fancyhf{}\renewcommand{\headrulewidth}{0pt}}  % To clear page numbers from footer, and header line at the start of every chapter

\pagestyle{fancy}
\fancyhf{}  % Clear header/footer

\fancyhead[LE,RO]{\thepage}
\fancyhead[LO,RE]{\textit{\nouppercase\leftmark}}

%========= Customise TOC Heading =========
\makeatletter
\def\createtoc{
    \renewcommand\tableofcontents{
        \chapter*{\contentsname}
        \@starttoc{toc}
    }
    \tableofcontents
}
\makeatother

%======= Customise Draft Watermark =======
\newcommand{\setasdraft}{
    \usepackage{draftwatermark}
    \SetWatermarkLightness{0.95}
    \SetWatermarkScale{1}
}

%============= Index Pages ===============
\usepackage[
    totoc,
    columnsep=20pt,
    hangindent=8pt,
    subindent=20pt,
    subsubindent=30pt
]{idxlayout}

\makeindex[options= -s index-style.ist]

%======= Bibliography Formatting =========
% These two lines are here to ensure that URLs do not exceed the page by too much
\setcounter{biburllcpenalty}{7000}
\setcounter{biburlucpenalty}{8000}

\usepackage{xr}

%=========== Global Variables ============
\newcommand{\version}{0.16}
\newcommand{\volumenumber}{1}
\newcommand{\volumename}{Groups}
\newcommand{\volumeimage}{cover/icosahedron.png}

%============= Formatting ================
\linespread{1.05}

%============== Resources ================
\externaldocument{volume0/number-theory}
\externaldocument{volume0/modular-arithmetic}

%=========== Path to images ==============
\graphicspath{{volume1/images}}

%=========== Custom Commands =============
\newcommand{\An}[1]{\mathrm{A}_{#1}}                  % Alternating group of degree n
\newcommand{\Aut}[1]{\mathrm{Aut}(#1)}                % Group of automorphisms of G
\newcommand{\C}[2]{\mathrm{C}_{#1}(#2)}               % Centralizer of an element in G
\newcommand{\Cl}[1]{\mathrm{Cl}(#1)}                  % Conjugacy class of the element x
\newcommand{\Cn}[1]{\mathrm{C}_{#1}}                  % Cyclic group of order n
\newcommand{\GL}[2]{\mathrm{GL}_{#1}\left(#2\right)}  % General Linear Group of degree n
\newcommand{\Inn}[1]{\mathrm{Inn}(#1)}                % Group of inner automorphisms of G
\newcommand{\N}[2]{\mathrm{N}_{#1}(#2)}               % Normalizer of S in G
\newcommand{\Out}[1]{\mathrm{Out}(#1)}                % Group of outer automorphisms of G
\newcommand{\SL}[2]{\mathrm{SL}_{#1}\left(#2\right)}  % Special Linear Group of degree n
\newcommand{\Sn}[1]{\mathrm{S}_{#1}}                  % Symmetric group of degree n
\newcommand{\Syl}[2]{\mathrm{Syl}_{#1}(#2)}           % Set of Sylow p-groups of G
\newcommand{\Sym}[1]{\mathrm{Sym}(#1)}                % Symmetric group of a set
\newcommand{\Un}[1]{\mathcal{U}_{#1}}                 % Group of units modulo n
\newcommand{\Z}[1]{\mathrm{Z}(#1)}                    % Center of a group G

\newcommand{\Stab}[2]{\mathrm{Stab}_{#1}(#2)}         % Stabilizer of x by G
\newcommand{\Fix}[2]{\mathrm{Fix}_{#1}(#2)}           % Set of all elements in X which is fixed by g
\newcommand{\Orb}[2]{\mathrm{Orb}_{#1}(#2)}           % Orbit of x under G

%========= Front Matter Pages ============
% Quote page
\newcommand{\quotepagetext}{
    [The] axioms for a group are short and natural... [yet] somehow hidden behind these axioms is the monster simple group, a huge and extraordinary mathematical object, which appears to rely on numerous bizarre coincidences to exist. The axioms for groups give no obvious hint that anything like this exists.
}
\newcommand{\quotepageattribution}{Richard Borcherds, 2009}
\newcommand{\quotepagecitation}{\cite{cook_borcherds_2009}}

% Preface
\newcommand{\prefacevolumetext}{
    Groups are one of the most fundamental structures in abstract algebra. They underpin the ideas of symmetry and allow us to explore the relationships between symmetrical objects. An analysis of all the different ways an object can be symmetric is also possible with groups. It would be an understatement to say that groups are important in abstract algebra; without them, there can be no further and more in-depth exploration of the other structures.

    Volume I is a simple introduction to the world of groups. As with most books on this topic, we concentrate on abstract groups, and, in particular, on finite groups. We also discuss and explore some crucial results about the structure of groups. The content covered in this volume should be ample for one to understand the fundamentals of group theory, and appreciate the wonders of groups and symmetry.
}
\newcommand{\prefacevolumedate}{22 March, 2023}

% Suggestions of use
\newcommand{\interdependencenotes}{
    \begin{itemize}
        \item This volume assumes understanding of the prerequisites in volume 0.
        \item Chapter 1 is essentially independent from the rest of the other chapters. It provides motivation for the axioms of groups, but readers who want to skip this introduction can move straight to chapter 2.
        \item Chapters 2, 3, and 4 are considered to be the essentials of group theory.
        \item Chapter 4 is required reading for chapters 5, 6, 7, and 9.
        \item Chapter 7 requires a small result from chapter 6 (specifically, \myref{prop-subgroup-product-is-subgroup}); otherwise these two chapters are relatively independent.
        \item Chapter 8, on more types of groups, assumes knowledge of chapter 7. Chapter 8 also assumes knowledge of permutations and the symmetric group from chapter 5.
        \item Chapter 9 could be read after chapter 4.
        \item Chapter 10 only requires chapters 7 and 9.
        \item Chapter 11 only require results from chapter 7, except for one problem (\myref{problem-S4-composition-series}) which uses the alternating group introduced in chapter 8.
        \item Chapter 12 assumes full knowledge of chapter 7; minor results from chapter 8 (specifically, the alternating group), chapter 10 (the Third Sylow Theorem), and chapter 11 (\myref{problem-S4-composition-series}) are required.
    \end{itemize}
}

%==== Include only relevant chapters =====
\IfFileExists{\jobname.run.xml}
{
    \includeonly{
        % Main chapters
        volume1/introduction-to-groups,
        volume1/basics-of-groups,
        volume1/subgroups,
        volume1/homomorphisms-and-isomorphisms,
        volume1/symmetry-groups,
        volume1/products,
        volume1/further-homomorphisms,
        volume1/more-groups,
        volume1/group-actions,
        volume1/sylow-theorems,
        volume1/composition-series,
        volume1/simple-groups,
        % Appendices
        volume1/exercise-solutions,
        volume1/problem-solutions,
        volume1/appendices
    }
}
{
    % Do a full document initially to generate all the aux files
}

%=========================================
\begin{document}
\frontmatterpages

%=========================================
\include{volume1/introduction-to-groups}
\section{Basics of Groups}
\subsection*{Exercises}
\begin{questions}
    \item The Cayley table of $(\Z_6, \otimes_6)$ is as follows:
    \begin{table}[H]
        \centering
        \begin{tabular}{|l|l|l|l|l|l|l|}
        \hline
        \textbf{$\otimes_n$} & \textbf{0} & \textbf{1} & \textbf{2} & \textbf{3} & \textbf{4} & \textbf{5} \\ \hline
        \textbf{0}       & 0          & 0          & 0          & 0          & 0          & 0          \\ \hline
        \textbf{1}       & 0          & 1          & 2          & 3          & 4          & 5          \\ \hline
        \textbf{2}       & 0          & 2          & 4          & 0          & 2          & 4          \\ \hline
        \textbf{3}       & 0          & 3          & 0          & 3          & 0          & 3          \\ \hline
        \textbf{4}       & 0          & 4          & 2          & 0          & 4          & 2          \\ \hline
        \textbf{5}       & 0          & 5          & 4          & 3          & 2          & 1          \\ \hline
        \end{tabular}
    \end{table}

    Since the identity is $1$, and the row (and column) of 0 does not have a $1$, thus $0$ does not have an inverse. Therefore $(\Z_6, \oplus_6)$ is not a group.

    \item Note that $(xx^{-1})^{-1} = (x^{-1})^{-1}x^{-1}$ by Shoes and Socks and $(xx^{-1})^{-1} = e^{-1} = e$. Thus $(x^{-1})^{-1}x^{-1} = e$. Multiplying both sides on the right by $x$ yields $(x^{-1})^{-1} = ex = x$, i.e. $(x^{-1})^{-1} = x$.

    \item We consider a proof by induction via inducting on $n$.

    The base case of $n = 0$ clearly holds true since
    \begin{align*}
        (x^{-1})^0 &= e & (\text{definition of }g^0 \text{ for any }g\in G)\\
        &= e^{-1} & (\myref{prop-inverse-of-identity-is-identity})\\
        &= (x^0)^{-1}. & (\text{definition of }x^0)
    \end{align*}

    Now assume that the statement holds for a non-negative integer $k$, i.e. $(x^{-1})^k = (x^k)^{-1}$. We are to show that the statement holds for $k+1$, i.e. $(x^{-1})^{k+1} = (x^{k+1})^{-1}$.

    Observe that
    \begin{align*}
        (x^{-1})^{k+1} &= (x^{-1})^k \ast x^{-1} & (\text{by statement 1})\\
        &= (x^k)^{-1} \ast x^{-1} & (\text{by hypothesis})\\
        &= (x\ast x^k)^{-1} & (\text{by Shoes and Socks})\\
        &= (x^{k+1})^{-1} & (\text{by statement 1})
    \end{align*}
    so the statement is true for $k+1$.

    Thus, by induction, we have $(x^{-1})^n = (x^n)^{-1}$ for any non-negative integer $n$.

    \item \begin{partquestions}{\roman*}
        \item The identity is $1$ since:
        \begin{itemize}
            \item $1 \times 1 = 1$;
            \item $1 \times (-1) = (-1) \times 1 = -1$;
            \item $1 \times i = i \times 1 = i$; and
            \item $1 \times (-i) = (-i) \times 1 = -i$.
        \end{itemize}
        \item The order of the identity $1$ is 1, so we look at the other elements:
        \begin{itemize}
            \item $|-1| = 2$ since $-1 \neq 1$ and $(-1)^2 = -1 \times -1 = 1$.
            \item $|i| = 4$ since $i \neq 1$, $i^2 = -1 \neq 1$, $i^3 = -i \neq 1$, but $i^4 = 1$.
            \item $|-i| = 4$ since $-i \neq 1$, $(-i)^2 = -1 \neq 1$, $(-i)^3 = i \neq 1$, but $(-i)^4 = 1$.
        \end{itemize}
    \end{partquestions}

    \item $-i$ is the other generator since $(-i)^1 = -i$, $(-i)^2 = -1$, $(-i)^3 = i$, and $(-i)^4 = 1$.

    \item We work slowly:
    \begin{align*}
        rsr^4sr^3 &= r(sr^4)(sr^3)\\
        &= r(r^2s)(r^3s)\\
        &= r^3sr^3s\\
        &= r^3(sr^3)s\\
        &= r^3(r^3s)s\\
        &= r^6s^2\\
        &= e
    \end{align*}
\end{questions}

\subsection*{Problems}
\begin{questions}
    \item The group table of $D_4$ is given as follows.
    \begin{table}[H]
        \centering
        \begin{tabular}{|l|l|l|l|l|l|l|l|l|}
        \hline
        $\ast$ & $e$    & $r$    & $r^2$  & $r^3$  & $s$    & $rs$   & $r^2s$ & $r^3s$ \\ \hline
        $e$    & $e$    & $r$    & $r^2$  & $r^3$  & $s$    & $rs$   & $r^2s$ & $r^3s$ \\ \hline
        $r$    & $r$    & $r^2$  & $r^3$  & $e$    & $rs$   & $r^2s$ & $r^3s$ & $s$    \\ \hline
        $r^2$  & $r^2$  & $r^3$  & $e$    & $r$    & $r^2s$ & $r^3s$ & $s$    & $rs$   \\ \hline
        $r^3$  & $r^3$  & $e$    & $r$    & $r^2$  & $r^3s$ & $s$    & $rs$   & $r^2s$ \\ \hline
        $s$    & $s$    & $r^3s$ & $r^2s$ & $rs$   & $e$    & $r^3$  & $r^2$  & $r$    \\ \hline
        $rs$   & $rs$   & $s$    & $r^3s$ & $r^2s$ & $r$    & $e$    & $r^3$  & $r^2$  \\ \hline
        $r^2s$ & $r^2s$ & $rs$   & $s$    & $r^3s$ & $r^2$  & $r$    & $e$    & $r^3$  \\ \hline
        $r^3s$ & $r^3s$ & $r^2s$ & $rs$   & $s$    & $r^3$  & $r^2$  & $r$    & $e$    \\ \hline
        \end{tabular}
    \end{table}
    \begin{partquestions}{\alph*}
        \item $D_4$ is not abelian because $rs \neq sr = r^3s$.
        \item We simplify $r^3srsr^3sr^3sr^2$.
        \begin{align*}
            r^3 sr sr^3 sr^3 sr^2 &= r^3srs(r^3s)(r^3s)r^2\\
            &= r^3 srs(e)r^2\\
            &= r^3 sr sr^2\\
            &= r^2(rs rs)r^2\\
            &= r^2(e)r^2\\
            &= r^4\\
            &= e
        \end{align*}
    \end{partquestions}

    \item We need to prove each of the group axioms in order to prove that $(\Q, +)$ is indeed a group.
    \begin{itemize}
        \item \textbf{Closure}: Let $\frac ab$ and $\frac cd$ be rational numbers where $b, d \neq 0$. Their sum is $\frac{ad+bc}{bd}$, which is also rational. Therefore $\Q$ is closed under addition.

        \item \textbf{Associativity}: Addition is associative by \myref{axiom-addition-is-associative}.

        \item \textbf{Identity}: 0 is the identity since
        \[
            0 + \frac ab = \frac ab + 0 = \frac ab
        \]
        for any rational number $\frac ab$ (with $b \neq 0$).

        \item \textbf{Inverse}: For any rational number $\frac ab$, its inverse is $-\frac ab$ since
        \[
            \frac ab + \left(-\frac ab\right) = \left(-\frac ab\right) + \frac ab = 0
        \]
        for any rational number $\frac ab$ (with $b \neq 0$).
    \end{itemize}
    Furthermore addition is assumed to be commutative by \myref{axiom-addition-is-commutative}. Therefore $(\Q, +)$ is an abelian group.

    \item If every element in $G$ is its own inverse, then for every element $g$ in $G$, $g^{-1} = g$. Consider $(gh)^{-1}$ where $g$ and $h$ are elements in $g$. On one hand, by Shoes and Socks, $(gh)^{-1} = h^{-1}g^{-1} = hg$ since each element is its own inverse. On the other hand, since $gh$ is an element in $G$, thus $(gh)^{-1} = gh$. Thus $gh = hg$ which means $G$ is abelian.

    \item Recall that $n = |x|$ is the smallest positive integer that satisfies $x^n = e$.

    We prove the forward direction first. Suppose $m$ is a multiple of $n$, say $m = qn$ for some integer $q$. Then
    \[
        x^m = x^{qn} = \left(x^n\right)^q = e^q = e
    \]
    which means $x^m = e$.

    We now prove the reverse direction. Suppose $x^m = e$. Using Euclid's division lemma (\myref{lemma-euclid-division}), we write $m = qn + r$ where $q$ and $r$ are integers with $0 \leq r < n$. Hence
    \[
        x^m = x^{qn + r} = x^{qn}x^r = \left(x^n\right)^qx^r = e^qx^r = x^r.
    \]
    Note that for all integers $k$ where $1 \leq k < n$, we have $x^k \neq e$ since $n$ is the smallest positive integer such that $x^n = e$. Hence, if $x^r = e$, we conclude $r = 0$. Therefore $m = qn$, meaning $m$ is a multiple of $n$.

    \item \begin{partquestions}{\alph*}
        \item Note that $(gh)^2 = ghgh$. Given that $(gh)^2 = g^2h^2 = gghh$. By cancellation law, $hg = gh$ which means $G$ is abelian.
        \item Suppose $G$ is abelian. Clearly $(gh)^1 = gh$. Suppose $(gh)^{k} = g^kh^k$ for some positive integer $k$. Then
        \begin{align*}
            (gh)^{k+1} &= (gh)(gh)^k\\
            &= (gh)(g^kh^k) & (\text{by assumption})\\
            &= ghg^kh^k\\
            &= g(hg^k)h^k\\
            &= g(g^kh)h^k & (\text{since } G \text{ is abelian})\\
            &= gg^khh^k\\
            &= g^{k+1}h^{k+1}
        \end{align*}
        so $(gh)^{k+1} = g^{k+1}h^{k+1}$ assuming $(gh)^k = g^kh^k$. Thus the claim is proven by mathematical induction.
    \end{partquestions}

    \item Note that $|1| = n$ since $1^2 = 1 \oplus_n 1 = 2$, $1^3 = 1 \oplus_n 1 \oplus_n 1 = 3$, $1^4 = 4$, ..., $1^{n-1} = n-1$ and $1^n = 0$ which is the identity. Since the group $(\Z_n, \oplus_n)$ has an element with the same order as the group, it is thus cyclic with order $n$ and generator 1.

    \item We show that $(A, \circ)$ is a group.
    \begin{itemize}
            \item \textbf{Closure}: Function composition is closed by definition.
            \item \textbf{Associativity}: Function composition is associative.
            \item \textbf{Identity}: By performing brute-force computation, we find that $T^6(x, y) = (x, y)$. Hence $T^6$ is the identity of $A$.
            \item \textbf{Inverse}: If $r = 6$ then $T^r$ is its own inverse. Otherwise, $T^{6-r}$ is the inverse of $T^r$.
    \end{itemize}
    Thus, $(A, \circ)$ is a group, with order 6.
\end{questions}

\chapter{Subgroups}
We introduced the simple properties of groups and simple examples of groups in the previous chapter. Sometimes, subsets of a group can also be a group. Such groups are called subgroups; we explore important properties and results relating to them in this chapter, with the most important result being Lagrange's theorem.

\section{Definition and Examples}
We look at the definition of a subgroup of a group.
\begin{definition}
    Let $G$ be a group with operation $\ast$. Let $H$ be a subset of $G$. Then $H$ is said to be a \textbf{subgroup}\index{subgroup} of $G$ if it satisfies the following conditions.
    \begin{enumerate}
        \item \textbf{Closure}: For all $x$ and $y$ in $H$, we have $x \ast y$ is also in $H$.
        \item \textbf{Identity}: The identity of the group $G$ is in $H$.
        \item \textbf{Inverse}: For all elements $x$ in $H$, there exists an element $x^{-1}$ in $H$ such that $x \ast x^{-1} = x^{-1} \ast x = e$.
    \end{enumerate}
\end{definition}
\begin{remark}
    Equivalently, $H$ is a subgroup of $G$ if $H$ is a subset of $G$ and $H$ is a group under $\ast$.
\end{remark}
We write $H \leq G$ if $H$ is a subgroup of $G$.

\begin{example}\label{example-subgroups-of-Z}
    Let's look at all possible subgroups of the group $(\mathbb{Z}, +)$. Suppose $H$ is a subgroup of $G$ with $H \neq \{0\}$.

    Let $n$ be the smallest positive integer in $H$. Let $m$ be any other number in $H$. Then by Euclid's division lemma (\myref{lemma-euclid-division}), we may write $m = nq + r$ where $q$ and $r$ are integers such that $0 \leq r < n$. Hence,
    \[
        r = m + \underbrace{(-n) +(-n) +(-n) + \cdots + (-n)}_{q\text{ times}}.
    \]
    Note that since both $m$ and $-n$ are in $H$, thus $r = m + (-n) + \cdots + (-n)$ is also in $H$, since $H$ is closed under addition. But $0 \leq r < n$ and $n$ is the \textit{smallest} positive integer in $H$. Thus, $r \not> 0$ which means $r = 0$. Therefore $m = nq$, i.e. every element in $H$ is a multiple of the smallest positive integer in $H$.

    Hence,
    \[
    H = \{nk \vert k \in \mathbb{Z}\},
    \]
    which is often written as the group $n\mathbb{Z}$.
\end{example}

\begin{exercise}
    Let $G$ be a group with identity $e$. Prove that $\{e\} \leq G$.
\end{exercise}

We note that the trivial group (i.e., $\{e\}$) is always a subgroup of any group. This is known as the \textbf{trivial subgroup}\index{subgroup!trivial}. We also note that the group itself is a subgroup of itself (that is, $G \leq G$). Any subgroup that are \textbf{not} these two groups is known as a \textbf{proper subgroup}\index{subgroup!proper} of $G$, and we write $H < G$ if $H$ is a proper subgroup of $G$.

\section{Subgroup Test}
To prove that a subset of a group is a subgroup using the axioms is too tedious. Wouldn't it be nice if we have a simple test to determine if a subset is a subgroup? Well, there is; it is called the \textbf{subgroup test}\index{subgroup test}.
\begin{theorem}[Subgroup Test]\label{thrm-subgroup-test}
    Let $G$ be a group and $H \subseteq G$ such that $H \neq \emptyset$ (i.e., $H$ is not the empty set). Then $H \leq G$ if and only if $xy^{-1} \in H$ for all $x, y \in H$.
\end{theorem}
\begin{proof}
    We prove the forward direction first. Suppose $H \leq G$ and let $x$ and $y$ be elements in $H$. Since $H$ is a subgroup of $G$, inverses exist. Thus $y^{-1}$ is in $H$. Also, since $H$ is a subgroup of $G$, thus $H$ is closed under the group operation. Hence, $xy^{-1}$ is in $H$. Therefore, if $H \leq G$ then $xy^{-1} \in H$ for all $x, y \in H$.

    We now prove the reverse direction. Suppose $xy^{-1}\in H$ for all $x$ and $y$ in $H$.
    \begin{itemize}
        \item Suppose $h$ is an arbitrary element in $H$. Set $x = h$ and $y = h$. Then $xy^{-1} = hh^{-1} = e$ is in $H$. Thus the identity of $G$ is present in $H$.
        \item Set $x = e$ and let $y = h$. Then $xy^{-1} = eh^{-1} = h^{-1}$ is in $H$. Hence every element in $H$ has an inverse in $H$.
        \item Let $a$ and $b$ be elements in $H$. Then by above point, $b^{-1}$ is in $H$. Set $x = a$ and $y = b^{-1}$. Then $xy^{-1} = a\left(b^{-1}\right)^{-1} = ab$ is in $H$. Hence $H$ is closed under the group operation.
    \end{itemize}
    Therefore if $xy^{-1}$ is in $H$ if for all $x$ and $y$ in $H$, then $H \leq G$.

    This completes the proof of the subgroup test.
\end{proof}

\begin{remark}
We usually verify the `non-empty' and `subset' requirement by checking if the identity of $G$ is in $H$.
\end{remark}

We look at some examples of the use of the subgroup test.
\begin{example}\label{example-centralizer-of-a-subset}
    Let $G$ be a group and $S \subseteq G$ where $S \neq \emptyset$. The \textbf{centralizer}\index{centralizer} of $S$ in $G$ is the set
    \[
        \C{G}{S} = \left\{g \in G \vert \text{for all } s \in S \text{ we have } gs = sg\right\}.
    \]
    We claim that $\C{G}{S}$ is a subgroup of $G$.
    \begin{proof}
        We show that for all elements $x$ and $y$ in $\C{G}{S}$, $xy^{-1}$ is also in the centralizer. Note that the condition $gs = sg$ is equivalent to the condition $sg^{-1} = g^{-1}s$.

        Clearly for all $s \in S$ we have $es = se$, so $e \in \C{G}{S}$. This means that $\C{G}{S}$ is non-empty and $\C{G}{S}$ is a subset of $G$.
        
        Now suppose $x$ and $y$ are in $\C{G}{S}$. Let $s$ be in $S$. Then
        \begin{align*}
            (xy^{-1})s &= x(y^{-1}s) & (\text{associativity})\\
            &= x(sy^{-1}) & (\text{since } g^{-1}s = sg^{-1})\\
            &= (xs)y^{-1} & (\text{associativity})\\
            &= (sx)y^{-1} & (\text{since } gs = sg)\\
            &= s(xy^{-1}) & (\text{associativity})
        \end{align*}
        so $(xy^{-1})s = s(xy^{-1})$ for all $s$ in $S$. Thus, $xy^{-1}$ is in $\C{G}{S}$ for all $x$ and $y$ in $\C{G}{S}$, so by subgroup test $\C{G}{S} \leq G$.
    \end{proof}
\end{example}
\begin{remark}
    In the case where the set $S$ contains a single element, say $S = \{x\}$, we write $\C{G}{x}$.
\end{remark}

\begin{example}\label{example-center-of-group}
    Let $G$ be a group. The \textbf{center}\index{center} of a group $G$ is given by the set
    \[
        \Z{G} = \{z \in G \vert gz = zg \text{ for all } g \in G\}.
    \]
    We claim that $\Z{G}$ is a subgroup of $G$.

    \begin{proof}
	    Note that $e \in \Z{G}$ since $ge = eg$ for all $g \in G$.

	    Let $x$ and $y$ be in $\Z{G}$, meaning that $gx = xg$ and $gy = yg$ for all $g \in G$. Note that $gy = yg$ implies $y^{-1}g = gy^{-1}$, so $y^{-1} \in \Z{G}$. Now,
	    \begin{align*}
	        (xy^{-1})g &= x(y^{-1}g)\\
	        &= x(gy^{-1}) & (\text{since } y^{-1} \in \Z{G})\\
	        &= (xg)y^{-1} & (\text{associativity})\\
	        &= (gx)y^{-1} & (\text{since } x \in \Z{G})\\
	        &= g(xy^{-1}) & (\text{associativity})
	    \end{align*}
	    which means that $xy^{-1} \in \Z{G}$. Hence $\Z{G} \leq G$ by the subgroup test.
    \end{proof}
\end{example}

\begin{exercise}\label{exercise-conjugate-subgroup}
    Let $G$ be a group, $H \leq G$, and $g \in G$. Define the set
    \[
        S = \{ghg^{-1} \vert h \text{ is in } H\}.
    \]
    Prove that $S \leq G$ under the group operation of $G$.
\end{exercise}

\newpage

\section{Cosets}
We now introduce the idea of \textbf{cosets}\index{coset} of a group $G$.
\begin{definition}
    Let $G$ be a group, $H \leq G$, and $g$ be an element of $G$. Then
    \begin{itemize}
        \item the \textbf{left coset}\index{coset!left} of $H$ in $G$ by $g$ is $gH = \{gh \vert h \in H\}$; and
        \item the \textbf{right coset}\index{coset!right} of $H$ in $G$ by $g$ is $Hg = \{hg \vert h \in H\}$.
    \end{itemize}
\end{definition}

\begin{remark}
    The subgroup $H$ is both a left and right coset of $H$ in $G$. This is because the element $g$ in question is $g = e$, so $gH = eH = H$ and $Hg = He = H$.
\end{remark}

\begin{example}
    Let $G = D_3$, and $H = \{e, s\} \leq G$. The distinct left cosets of $H$ in $G$ are
    \begin{itemize}
        \item $eH = \{e, s\} = H$;
        \item $rH = \{r, rs\}$; and
        \item $r^2H = \{r^2, r^2s\}$.
    \end{itemize}
    Since each element of $G$ appears in one of these cosets, generating more using any other element would not give new cosets. This is because a new coset would have to contain an element in common with one of the above cosets and therefore be identical to it. For example, $rsH = \{rs, rss\} = \{rs, r\} = rH$.

    The distinct right cosets of $H$ in $G$ are
    \begin{itemize}
        \item $He = \{e, s\} = H$;
        \item $Hr = \{r, sr\} = \{r, r^2s\}$; and
        \item $Hr^2 = \{r^2, sr^2\} = \{r^2, rs\}$.
    \end{itemize}
    Thus, for $D_3$, no left coset is a right coset, except for $H = eH = He$.
\end{example}

\begin{exercise}
    Let $G$ be the group $(\mathbb{Z}_8, \oplus_8)$. Let $H = \{0, 4\} \leq G$.
    \begin{partquestions}{\alph*}
        \item Explain why any left coset of $H$ in $G$ by an element $g$ is the same as the right coset of $H$ in $G$ by $g$.
        \item Find all distinct left cosets of $H$ in $G$.
    \end{partquestions}
\end{exercise}

We now state and prove a result that relates the equality of cosets.
\begin{lemma}[Coset Equality]\label{lemma-coset-equality}\index{Coset Equality}
    Let $G$ be a group, $H \leq G$, and $g_1$ and $g_2$ be elements in $G$. Then the following statements are equivalent.
    \begin{enumerate}[label=$(\arabic*)$]
        \item $g_1H = g_2H$
        \item $Hg_1^{-1} = Hg_2^{-1}$
        \item $g_1H \subseteq g_2H$
        \item $g_2 \in g_1H$
        \item $g_1^{-1}g_2 \in H$
    \end{enumerate}
\end{lemma}
\begin{proof}
    We prove the statements in order.

    \begin{itemize}
        \item $\boxed{(1) \implies (2)}$ Suppose $g_1H = g_2H$. We show $Hg_1^{-1} = Hg_2^{-1}$.

        Let $x \in Hg_1^{-1}$. Then $x = hg_1^{-1}$ for some $h$ in $H$. Thus $x^{-1} = \left(hg_1^{-1}\right)^{-1} = g_1h^{-1}$ by Shoes and Socks. Since $h^{-1}$ is in $H$, thus $x^{-1} = g_1h^{-1}$ is in $g_1H$.

        Since $g_1H = g_2H$, thus $x^{-1} \in g_2H$. Write $x^{-1} = g_2\hat{h}$ for some $\hat{h}$ in $H$. Thus $x = (g_2\hat{h})^{-1} = \hat{h}^{-1}g_2^{-1}$ by Shoes and Socks. Since $\hat{h}^{-1}$ is in $H$ thus $x = \hat{h}^{-1}g_2^{-1}$ is in $Hg_2^{-1}$.

        Hence, any element $x \in Hg_1^{-1}$ is also in $Hg_2^{-1}$, i.e. $Hg_1^{-1} \subseteq Hg_2^{-1}$. A similar argument shows that $Hg_2^{-1} \subseteq Hg_1^{-1}$. Thus $Hg_1^{-1} = Hg_2^{-1}$ as required.

        \item $\boxed{(2) \implies (3)}$ Suppose $Hg_1^{-1} = Hg_2^{-1}$ and take $x \in g_1H$. Thus $x = g_1h$ for some $h$ in $H$. Therefore $x^{-1} = (g_1h)^{-1} = h^{-1}g_1^{-1} \in Hg_1^{-1}$ since $h^{-1}$ is in $H$. By assumption $Hg_1^{-1} = Hg_2^{-1}$ so $x^{-1} \in Hg_2^{-1}$. Let $x^{-1} = \hat{h}g_2^{-1}$ for some $\hat{h}$ in $H$. Then $x = \left(\hat{h}g_2^{-1}\right)^{-1} = g_2\hat{h}^{-1} \in g_2H$ since $\hat{h}^{-1}$ is in $H$. Hence $x$ is in $g_2H$. Therefore, for any $x \in g_1H$, $x$ will also be in $g_2H$. Thus $g_1H \subseteq g_2H$.

        \item $\boxed{(3) \implies (4)}$
        Suppose $g_1H \subseteq g_2H$. Then for all $x$ in $g_1H$, $x$ is also in $g_2H$. Note that $g_1 = g_1e$. Since $e$ is in $H$ (as $H \leq G$) thus $g_1e \in g_1H \subseteq g_2H$ by assumption. Thus $g_1 \in g_2H$ as needed.

        \item $\boxed{(4) \implies (5)}$
        Suppose $g_1 \in g_2H$. Then $g_1 = g_2h$ for some $h$ in $H$. Thus,
        \begin{align*}
            &g_1^{-1}g_1 = g_1^{-1}g_2h & (\text{left multiply by } g_1^{-1})\\
            &e = g_1^{-1}g_2h\\
            &h^{-1} = g_1^{-1}g_2hh^{-1} & (\text{right multiply by } h^{-1})\\
            &h^{-1} = g_1^{-1}g_2.
        \end{align*}
        Since $h^{-1}$ is an element in $H$ thus $g_1^{-1}g_2 = h^{-1}$ is also in $H$, meaning $g_1^{-1}g_2 \in H$.

        \newpage

        \item $\boxed{(5) \implies (1)}$
        Suppose $g_1^{-1}g_2$ is an element of $H$. Thus, $g_1^{-1}g_2 = \hat{h}$ for some $\hat{h}$ in $H$. Let $x$ be an element from $g_1H$, so $x = g_1h$ for some $h$ in $H$. Then
        \begin{align*}
            x^{-1} &= (g_1h)^{-1}\\
            &= h^{-1}g_1^{-1} & (\text{Shoes and Socks})\\
            &= h_1^{-1}g_1^{-1}(g_2g_2^{-1}) & (\text{since }g_2g_2^{-1} = e)\\
            &= h_1{-1}(g_1^{-1}g_2)g_2^{-1}\\
            &= h^{-1}\hat{h}g_2^{-1}
        \end{align*}
        which means $x = \left(h^{-1}\hat{h}g_2^{-1}\right)^{-1} = g_2\hat{h}^{-1}h$, which is an element of $g_2H$ since $\hat{h}^{-1}h$ is in $H$.

        Thus, for all $x$ in $g_1H$, $x$ is also in $g_2H$ which means $g_1H \subseteq g_2H$. A similar argument can be used to show that $g_2H \subseteq g_1H$. Hence $g_1H = g_2H$.
    \end{itemize}

    Thus, $(1) \implies (2) \implies (3) \implies (4) \implies (5) \implies (1)$, completing the proof.
\end{proof}

\begin{exercise}\label{exercise-intersection-of-cosets}
    Let $G$ be a group, $H \leq G$, and $g_1$ and $g_2$ be elements in $G$. Prove that if $g_1H \cap g_2H \neq \emptyset$ then $g_1H = g_2H$.
\end{exercise}

\newpage

\section{Lagrange's Theorem}
Lagrange's theorem is an important result relating the order of a subgroup and the order of the group itself. Before that, we introduce the idea of the \textbf{index} of a subgroup.
\begin{definition}
    Let $G$ be a group and $H \leq G$. The \textbf{index}\index{index} of $H$ in $G$, denoted by $[G : H]$, is the number of left cosets of $H$ in $G$.
\end{definition}
Note that since the number of left cosets is the number of right cosets, $[G : H]$ can be defined as the number of cosets of $H$ in $G$.

\begin{example}
    If $G = D_3$ and $H = \{e, s\}$, then there are 3 distinct left cosets, namely $H$, $rH$, and $r^2H$. Thus, $[G:H] = 3$ in this case.
\end{example}

We require two lemmas to properly prove Lagrange's theorem.
\begin{lemma}\label{lemma-left-coset-partition}
    Let $G$ be a group and $H \leq G$. Then distinct left cosets of $H$ in $G$ partition $G$.
\end{lemma}
\begin{proof}
    To prove that the distinct left cosets of $H$ in $G$ partition $G$, we need to show two things.
    \begin{itemize}
        \item The intersection of any 2 left cosets is either the empty set or is one of the left cosets (i.e., distinct left cosets are disjoint).
        \item The union of all left cosets is the group.
    \end{itemize}
    The first bullet point is proven by \myref{exercise-intersection-of-cosets}, so we work only on the second bullet point.

    Suppose $g$ is in $G$. We will find a left coset that $g$ is in. Clearly $g = ge$, and since $e$ is an element of $H$, thus $g = ge$ is an element of the left coset $gH$. So any element in $G$ belongs to a left coset of $H$ in $G$. Thus the union of all left cosets is the group.

    Hence, distinct left cosets of $H$ in $G$ partition $G$.
\end{proof}

\begin{lemma}\label{lemma-order-of-coset}
    Let $G$ be a group and $H \leq G$. Then $|H| = |gH|$ for all $g$ in $G$.
\end{lemma}
\begin{proof}
    Define the map $\phi: H \to gH$ such that $\phi(h) = gh$. To prove that $|H| = |gH|$, we need to show that $\phi$ is a bijection.
    \begin{itemize}
        \item \textbf{Injective}: Let $h$ and $\hat{h}$ be elements in $H$ such that $\phi(h) = \phi(\hat{h})$. Then $gh = g\hat{h}$ by definition of $\phi$. By cancellation law, we have $h = \hat{h}$ which means $\phi$ is injective.
        \item \textbf{Surjective}: Let $x$ be in $gH$. Thus $x = gh$ for some $h$ in $H$. Clearly $\phi(h) = gh = x$ so a pre-image of $x$ exists in $H$. Therefore $\phi$ is surjective.
    \end{itemize}
    Therefore $\phi$ is a bijection, which means $|H| = |gH|$.
\end{proof}

We are now ready to state and prove Lagrange's theorem.
\begin{theorem}[Lagrange]\label{thrm-lagrange}\index{Lagrange's Theorem}
    Let $G$ be a group and $H \leq G$. Then $|G| = [G:H]|H|$.
\end{theorem}
\begin{proof}
    Suppose $|G| = n$. Let $\mathcal{S} = \{g_1H, g_2H, g_3H, \dots, g_kH\}$ be the set containing all distinct left cosets of $H$ in $G$. Thus $k = [G: H]$.

    Note that distinct left cosets partition $G$ (\myref{lemma-left-coset-partition}). Hence
    \[
        G = \bigcup_{i=1}^k g_iH = g_1H \cup g_2H \cup \cdots \cup g_kH
    \]
    with $g_iH \cap g_jH = \emptyset$ if $i \neq j$. This means
    \begin{align*}
        |G| &= \sum_{i=1}^k|g_iH|\\
        &=\sum_{i=1}^k|H| & (\myref{lemma-order-of-coset})\\
        &= k|H|.
    \end{align*}
    Recall that we set $k = [G: H]$, therefore $|G| = [G:H]|H|$, proving Lagrange's theorem.
\end{proof}

\begin{exercise}
    Let $G$ be the group $(\mathbb{Z}_{99}, \oplus_{99})$. It is given that $H = \{0, 33, 66\}$ is a subgroup of $G$. What is the index of $H$ in $G$?
\end{exercise}

Let's look at some corollaries of Lagrange's theorem.
\begin{corollary}\label{corollary-order-of-group-multiple-of-order-of-element}
    Let $G$ be a finite group and let $g$ be an element in $G$. Then $|G|$ is a multiple of $|g|$.
\end{corollary}
\begin{proof}
    We show that $|G| = m|g|$ for some positive integer $m$. Clearly $|e| = 1$ which means $|G| = |G| \times 1 = |G||e|$, so suppose $g$ is a non-identity element of $G$ with order $n$. Consider $\mathcal{S} = \langle g \rangle = \{g, g^2, g^3, \dots, g^n\}$. Note that $\mathcal{S}$ is a (cyclic) subgroup of $G$ and $|\mathcal{S}| = n$. By Lagrange's theorem (\myref{thrm-lagrange}), $|G| = [G:\mathcal{S}]|\mathcal{S}| = [G:\mathcal{S}]|g|$. Hence, in this case, $m = [G:\mathcal{S}]$, proving the claim.
\end{proof}

\newpage

\begin{corollary}\label{corollary-group-with-prime-order-subgroups}
    A finite group $G$ with prime order $p$ has no proper subgroups.
\end{corollary}
\begin{proof}
    By Lagrange's theorem (\myref{thrm-lagrange}), the order of a subgroup must be a factor of the order of the group. Since the order of the group is prime, it only has 2 factors, namely 1 and $p$. The subgroup of order 1 is the trivial subgroup and the subgroup of order $p$ is $G$ itself. Hence $G$ has no proper subgroups.
\end{proof}

\begin{exercise}\label{exercise-prime-order-element}
    Let $G$ be a finite group with prime order $p$. Let $x$ be a non-identity element in $G$. Prove that $|x| = p$.
\end{exercise}

\begin{corollary}\label{corollary-group-with-prime-order-is-cyclic}
    A finite group $G$ with prime order $p$ is cyclic.
\end{corollary}
\begin{proof}
    By \myref{exercise-prime-order-element}, any non-identity element $g$ in $G$ has $|g| = p$. Thus $\langle g \rangle \neq \{e\}$.

    Note that $\langle g \rangle = \{g, g^2, g^3, \dots, g^p\} \leq G$. However, by \myref{corollary-group-with-prime-order-subgroups}, the only subgroups of $G$ are $\{e\}$ and $G$. Since $\langle g \rangle \neq \{e\}$ thus $\langle g \rangle = G$, meaning $G$ is cyclic with generator $g$.
\end{proof}

\section{Normal Subgroups}
We now look at a special type of subgroup, known as a \textbf{normal subgroup}.
\begin{definition}
    Let $G$ be a group and $N \leq G$. We say that $N$ is a \textbf{normal subgroup}\index{subgroup!normal} of $G$ if $gN = Ng$ for all $g \in G$.
\end{definition}
If $N$ is a normal subgroup of $G$, we write $N \unlhd G$. Furthermore if $N$ is a \textit{proper} normal subgroup of $G$, we write $N \lhd G$.

\newpage

Note that $gN = Ng$ is equivalent to the following two statements:
\begin{itemize}
    \item $gNg^{-1} = N$ for all $g \in G$. (One may interpret $gNg^{-1}$ as either the left coset $g(Ng^{-1})$ or the right coset $(gN)g^{-1}$.)
    \item $gng^{-1} \in N$ for all $g \in G$ and $n \in N$.
\end{itemize}

\begin{proposition}\label{prop-subgroup-of-abelian-group-is-normal}
    Any subgroup of an abelian group is normal.
\end{proposition}
\begin{proof}
    Let $G$ be an abelian group and $H \leq G$. Let $g$ be in $G$ and $h$ be in $H$. Then
    \begin{align*}
        ghg^{-1} &= g(hg^{-1})\\
        &= g(g^{-1}h) & (G \text{ is abelian})\\
        &= (gg^{-1})h & (\text{associativity})\\
        &= eh\\
        &= h
    \end{align*}
    which is an element of $H$. Thus $ghg^{-1}$ is an element of $H$ for all $g$ in $G$ and $h$ in $H$, meaning $H \unlhd G$.
\end{proof}

\begin{example}\label{example-normal-subgroups-of-d3}
    Let's find the normal subgroups of the dihedral group of order 6, $D_3$.

    Recall $D_3 = \{e, r, r^2, s, rs, r^2s\}$. Note that $|r| = 3$, $|s| = 2$, $\langle r \rangle = \{e, r, r^2\}$, and $\langle s \rangle = \{e, s\}$. We will show that $\langle r \rangle$ is a (proper) normal subgroup of $D_3$ but not $\langle s \rangle$. Note that since $r$ and $s$ are generators, we just need to check $s\langle r\rangle$, $\langle r\rangle s$, $r\langle s\rangle$, and $\langle s\rangle r$ to verify normality.

    \newpage

    For $\langle r \rangle$,
    \begin{itemize}
        \item $s\langle r\rangle = \{s, sr, sr^2\} = \{s, r^2s, rs\}$; and
        \item $\langle r\rangle s = \{s, rs, r^2s\}$,
    \end{itemize}
    so $s\langle r\rangle = \langle r \rangle s$ which means $\langle r \rangle \lhd D_3$.

    For $\langle s \rangle$,
    \begin{itemize}
        \item $r\langle s\rangle = \{r, rs\}$; and
        \item $\langle s \rangle r = \{r, sr\} = \{r, r^2s\}$,
    \end{itemize}
    and since $rs \neq r^2s$ thus $r\langle s\rangle \neq \langle s \rangle r$. Hence, $\langle s \rangle$ is not a normal subgroup of $D_3$.
\end{example}

\section{Quotient Groups}
We end this chapter by looking at a special (and useful) type of group: the quotient group. But before we can do that, we look at the idea of the set of left cosets.

\begin{definition}
    Let $G$ be a group and $H \leq G$. The \textbf{set of left cosets}\index{set of left cosets} is denoted by
    \[
        G/H = \{gH \vert g \in G \}.
    \]
\end{definition}
\begin{remark}
    We may sometimes write $G/H$ using fractions (i.e. $\frac GH$) if it serves to improve readability.
\end{remark}

Note that if $G/H$ is \textbf{not} a group, it is read ``$G$ by $H$''. Recall that the number of left cosets is $[G:H] = \frac{|G|}{|H|}$ by Lagrange's Theorem (\myref{thrm-lagrange}).

\begin{theorem}\label{thrm-quotient-group-requirement}
    Let $G$ be a group and $N \unlhd G$. Then $G / N$ forms a group called the \textbf{quotient group}\index{quotient group} with group operation $\star$ where
    \[
        (xN) \star (yN) = (xy)N.
    \]
\end{theorem}
For the quotient group, $G / N$ is read ``$G$ mod $N$''. Also note that, as per usual, we suppress the operation $\star$ and just write $(xN)(yN) = (xy)N$.
\begin{proof}
    Before we can prove that it forms a group, we need to show that $\star$ is well defined. This is because we worry that if $x_1N = x_2N$ and $y_1N = y_2N$ there may be a situation that $(x_1y_1)N \neq (x_2y_2)N$ under this operation. Thus we need to check if it is well defined before we can proceed.

    Suppose $x_1N = x_2N$ and $y_1N = y_2N$ where $x_1, x_2, y_1, y_2 \in G$. Then
    \begin{align*}
        (x_1N)(y_1N) &= (x_1y_1)N & (\text{by definition})\\
        &= x_1(y_1N) & (\text{left coset of } y_1N \text{ in } G \text{ by } x_1)\\
        &= x_1(y_2N) & (y_1N = y_2N)\\
        &= x_1(Ny_2) & (N \text{ is normal, so } y_2N=Ny_2)\\
        &= (x_1N)y_2 & (\text{right coset of } x_1N \text{ in } G \text{ by } y_2)\\
        &= (x_2N)y_2 & (x_1N = x_2N)\\
        &= x_2(Ny_2) & (\text{left coset of } Ny_2 \text{ in } G \text{ by } x_2)\\
        &= x_2(y_2N) & (N \text{ is normal, so } y_2N=Ny_2)\\
        &= (x_2y_2)N & (\text{left coset of } N \text{ in } G \text{ by } x_2y_2)\\
        &= (x_2N)(y_2N) & \text{(by definition)}
    \end{align*}
    so if $x_1N = x_2N$ and $y_1N = y_2N$ then we see that $(x_1N)(y_1N) = (x_2N)(y_2N)$, meaning that $\star$ is well defined.

    We can now show that $G/N$ under $\star$ satisfies the group axioms.
    \begin{enumerate}
        \item \textbf{Closure}: Assume $xN$ and $yN$ are in the set $G/N$. Then $(xN)(yN) = (xy)N$. Since $xy$ is in $G$ thus $(xy)N$ is in $G/N$, meaning that $G/N$ is closed under $\star$.
        \item \textbf{Associativity}: Take $xN$, $yN$, and $zN$ from $G/N$. Then
        \begin{align*}
            (xN)\left((yN)(zN)\right) &= (xN)\left((yz)N\right)\\
            &= (xyz)N\\
            &= \left((xy)z\right)N\\
            &= \left((xy)N\right)(zN)\\
            &= \left((xN)(yN)\right)(zN)
        \end{align*}
        so $\star$ is associative.
        \item \textbf{Identity}: Observe that $e$ is in $G$ so $eN = N$ is in $G / N$. Note that
        \[
        (eN)(xN) = (ex)N = xN \text{ and } (xN)(eN) = (xe)N = xN
        \]
        for any $x$ in $G$, so $eN = N$ is the identity in $G/N$.
        \item \textbf{Inverse}: Observe that for any $x$ in $G$, $x^{-1}$ is also in $G$, so
        \[
        (xN)(x^{-1}N) = (xx^{-1})N = eN = N
        \]
        and
        \[
        (x^{-1}N)(xN) = (x^{-1}x)N = eN = N,
        \]
        which means $(xN)^{-1}$ is $x^{-1}N$.
    \end{enumerate}
    Since the four group axioms are satisfied, thus $G/N$ is a group under the operation $\star$.
\end{proof}

\begin{example}
    Let's look at possible quotients of the group $D_3$, which has the underlying set of $\{e, r, r^2, s, rs, r^2s\}$. Recall from a previous example (\myref{example-normal-subgroups-of-d3}) that $\langle r \rangle \lhd D_3$. Thus $D_3 / \langle r \rangle$ is a quotient group, with $|D_3 / \langle r \rangle| = \frac{|D_3|}{|\langle r\rangle|} = \frac63 = 2$.

    Let's now look at the elements of $D_3 / \langle r \rangle$.
    \begin{align*}
        D_3 / \langle r \rangle  &= \{x\langle r \rangle \vert x \in D_3\}\\
        &= \left\{\{x, xr, xr^2\} \vert x \in D_3\right\}\\
        &= \{\{e, r, r^2\}, \{r, r^2, r^3\}, \{r^2, r^3, r^4\}, \\ &\quad\quad \{s, sr, sr^2\}, \{rs, rsr, rsr^2\}, \{r^2s, r^2sr, r^2sr^2\}\}\\
        &= \{\{e, r, r^2\}, \{r, r^2, e\}, \{r^2, e, r\}, \\ &\quad\quad \{s, sr, sr^2\}, \{sr^2, s, sr\}, \{sr, sr^2, s\}\}\\
        &= \left\{\{e, r, r^2\}, \{s, sr, sr^2\}\right\}\\
        &= \left\{\langle r\rangle, s\langle r \rangle\right\}
    \end{align*}
    Note also that $(s\langle r \rangle)^2 = s^2\langle r \rangle = \langle r\rangle$, so in fact $D_3 / \langle r \rangle$ has generator $s\langle r \rangle$, i.e. $D_3 / \langle r \rangle = \left\langle s\langle r \rangle \right\rangle$.
\end{example}

\begin{exercise}\label{exercise-quotient-group-of-cyclic-group-is-cyclic}
    Let $G$ be a finite cyclic group. Let $H$ be a subgroup of $G$.
    \begin{partquestions}{\roman*}
        \item Explain why $G/H$ is a quotient group.
        \item Show that $G/H$ is cyclic.
    \end{partquestions}    
\end{exercise}

\newpage

\section{Problems}
\begin{problem}
    Let $G = D_4$, the dihedral group of order 8. By considering the subgroup axioms, determine if the following are subgroups of $G$.
    \begin{partquestions}{\alph*}
        \item $\{e\}$
        \item $\{e, r, s\}$
        \item $\{r, r^2, r^3\}$
        \item $\{r, r^3, r^4, r^6\}$
    \end{partquestions}
\end{problem}

\begin{problem}
    Let $G$ be a group and $H \leq G$. Let
    \[
        K = \{x \in G \vert x^2 \in H\}.
    \]
    Prove the following statements.
    \begin{partquestions}{\alph*}
        \item $K \leq G$
        \item $H \leq K$
    \end{partquestions}
\end{problem}

\begin{problem}\label{problem-center-of-G}
    Let $G$ be a group.
    \begin{partquestions}{\alph*}
        \item Prove that $\Z{G}$ is a normal subgroup of $G$.
        \item Prove that $\Z{G} = G$ if and only if $G$ is abelian.
        \item Find the center of the group $D_4$.\newline
        (You may assume $|\Z{G}| < \frac12 |G|$. The reason becomes apparent with \myref{problem-quotient-of-group-mod-center-is-cyclic-implies-abelian}'s solution.)
    \end{partquestions}
\end{problem}

\newpage

\begin{problem}\label{problem-intersection-of-subgroups}
    Let $G$ be a group, and $H, K \leq G$. Prove or disprove the following statements.
    \begin{partquestions}{\alph*}
        \item $H \cap K \leq G$
        \item $H \cap K \leq H$
        \item $H \cup K \leq G$
        \item $H \cup K \leq H$
    \end{partquestions}
\end{problem}

\begin{problem}
    Let $G$ be a group of order 1024 and let $H$ be a proper subgroup of $G$. Determine the maximum order of $H$. Give an example of the groups $G$ and $H$ such that $H$ has this maximum order.
\end{problem}

\begin{problem}
    Let $G$ be a finite group with even order. Show that there exists an element with order 2 in $G$.
\end{problem}

\begin{problem}\label{problem-subgroup-of-cyclic-group-is-cyclic}
    Let $G$ be a cyclic group with generator $g$. Prove that any subgroup of $G$ must also be cyclic.\newline
    (\textit{Hint: Consider the solution for \myref{example-subgroups-of-Z}.})
\end{problem}

\begin{problem}\label{problem-subgroup-of-index-2}
    Let $G$ be a finite group. Suppose $H < G$ such that the index of $H$ in $G$ is 2. Prove that
    \begin{partquestions}{\roman*}
        \item $H \lhd G$;
        \item $H$ contains the squares of all elements of $G$; and
        \item an element $x \in G$ is in $H$ if $x$ has odd order.
    \end{partquestions}
\end{problem}

\newpage

\begin{problem}\label{problem-intersection-of-coprime-subgroups}
    Let $G$ be a finite group, $H \leq G$, and $K \leq G$. Suppose the greatest common divisor (GCD) of the order of $H$ and the order of $K$ is 1.
    \begin{partquestions}{\alph*}
        \item Show that the intersection of the groups $H$ and $K$ contains only the identity.
        \item Show that, if $H$ and $K$ are normal subgroups of $G$, then for any $h \in H$ and $k \in K$ we have $hk = kh$.
    \end{partquestions}
\end{problem}

\begin{problem}\label{problem-smallest-nonabelian-group}
    Let $G$ be a finite group with order $m$.
    \begin{partquestions}{\alph*}
        \item State the smallest value of $m$ such that $G$ is non-abelian.
        \item Prove that the value of $m$ found in \textbf{(a)} is the smallest value that allows $G$ to be non-abelian.
        \item Hence prove that for all even integers $n \geq m$, there exists a non-abelian group of order $n$.
    \end{partquestions}
\end{problem}

\begin{problem}\label{problem-quotient-of-group-mod-center-is-cyclic-implies-abelian}
    Let $G$ be a group, and suppose $G / \Z{G}$ is cyclic. Prove that $G$ is abelian.
\end{problem}

\section{Homomorphisms and Isomorphisms}
\begin{questions}
    \item We will prove that $f$ is a homomorphism, is injective, and is surjective.
    \begin{itemize}
        \item \textbf{Homomorphism}: Let $x, y \in G$. Then
        \begin{align*}
            f(xy) &= g(xy)g^{-1}\\
            &= (gxg^{-1})(gyg^{-1})\\
            &= f(x)f(y)
        \end{align*}
        which means that $f$ is a homomorphism.
        \item \textbf{Injective}: Let $x, y \in G$ be such that $f(x) = f(y)$. Then $gxg^{-1} = gyg^{-1}$. By cancellation law, $x = y$.
        \item \textbf{Surjective}: Suppose $y \in G$. Set $x = g^{-1}yg$. Since $G$ is closed, thus $x \in G$. Note $f(x) = g(g^{-1}yg)g^{-1} = y$. Hence $y$ has a pre-image of $x = g^{-1}yg$ in $G$.
    \end{itemize}
    Therefore $f$ is an isomorphism.

    \item Suppose on the contrary there exists an isomorphism $\phi: G \to H$. Since $\phi$ is an isomorphism, it is surjective. Hence, there must exists a rational number $r \in G$ such that $\phi(r) = 2$. As $r$ is rational, so is $\frac r2$.

    Now consider $\phi\left(\frac r2 + \frac r2\right)$. On one hand, $\phi\left(\frac r2 + \frac r2\right) = \phi(r) = 2$. On another hand, $\phi(\frac r2 + \frac r2) = \left(\phi\left(\frac r2\right)\right)^2$ as $\phi$ is a homomorphism. Therefore, $\left(\phi\left(\frac r2\right)\right)^2 = 2$ which quickly implies $\phi\left(\frac r2\right) = \sqrt 2$ since $\phi\left(\frac r2\right)$ must be positive. However, $\sqrt 2 \notin H$ while $\phi\left(\frac r2\right) \in H$, a contradiction.

    Hence, $G \not\cong H$.

    \item \begin{partquestions}{\alph*}
        \item Let $m, n \in G$. Then
        \[
            \phi(m + n) = 2(m + n) = 2m + 2n = \phi(m) + \phi(n)
        \]
        which means $\phi$ is a homomorphism.

        \item Suppose $m, n \in G$ such that $\phi(m) = \phi(n)$. Then $2m = 2n$. Clearly this means that $m = n$. Thus $\phi$ is injective.

        \item Suppose on the contrary there existed a homomorphism $\psi: H \to G$ such that $\psi(\phi(n)) = n$. Then $\psi(2n) = n$ by definition of $\phi$. Note that
        \[
            \psi(2n) = \psi(n + n) = \psi(n) + \psi(n) = 2\psi(n)
        \]
        since $\psi$ is a homomorphism. Hence $2\psi(n) = n$ which implies that $\psi(n) = \frac n2$. But for the case of $n = 1$, $\psi(1) = \frac 12 \notin G$. Hence $\psi$ does not exist.
    \end{partquestions}

    \item We prove the forward direction first: assume that $G$ is abelian. Then $f$ is a homomorphism since
    \[
        f(gh) = (gh)^{-1} = h^{-1}g^{-1} = g^{-1}h^{-1} = f(g)h(g).
    \]

    We now prove the reverse direction: assume that $f$ is a homomorphism, meaning $f(gh) = f(g)f(h) = g^{-1}h^{-1}$. But $f(gh) = (gh)^{-1} = h^{-1}g^{-1}$. Therefore we have $g^{-1}h^{-1} = h^{-1}g^{-1}$ which clearly shows that the group is abelian.

    \item Suppose $\phi: G \to H$ is a surjective homomorphism and $G$ is abelian. Since $\phi$ is surjective, thus $\im \phi = H$. Let $g_1, g_2 \in G$ and $h_1, h_2 \in H$ such that $\phi(g_1) = h_1$ and $\phi(g_2) = h_2$. Consider $\phi(g_1g_2)$.
    \begin{itemize}
        \item On one hand, $\phi(g_1g_2) = \phi(g_1)\phi(g_2) = h_1h_2$.
        \item On another hand, $\phi(g_1g_2) = \phi(g_2g_1) = \phi(g_2)\phi(g_1) = h_2h_1$.
    \end{itemize}
    Hence $h_1h_2 = h_2h_1$ which means that $H$ is abelian.

    \item We first prove $\phi(N)$ is a subgroup of $H$ by using subgroup test before proving normality.

    Note that $e_H \in \phi(N)$ since $e_G \in N$ and $\phi(e_G) = e_H$. Now let $x, y \in \phi(N)$. As $\phi$ is surjective, we know that there exists $n_x, n_y \in N$ where $\phi(n_x) = x$ and $\phi(n_y) = y$. Note that $\phi(n_y^{-1}) = y^{-1}$ and $n_xn_y^{-1} \in N$. Hence, $xy^{-1} = \phi(n_xn_y^{-1}) \in \phi(N)$. By subgroup test, $\phi(N) \leq H$.

    We now show that $\phi(N)$ is a normal subgroup of $H$. Take $g \in G$, $h \in H$, $n \in N$, and $x \in \phi(N)$, such that $\phi(g) = h$ and $\phi(n) = x$. Note that since $N \unlhd G$, thus $gng^{-1} \in N$. Therefore,
    \begin{align*}
        hxh^{-1} &= \phi(g)\phi(n)\phi(g^{-1})\\
        &= \phi(\underbrace{gng^{-1}}_{\text{In }N})\\
        &\in \phi(N)
    \end{align*}
    which means that $\phi(N) \unlhd H$.

    \item Consider the map $\phi: G \to H, a \mapsto a + n\mathbb{Z}$. We show that $\phi$ is an isomorphism:
    \begin{itemize}
        \item \textbf{Homomorphism}: Let $a$ and $b$ be in $G$. Then
        \begin{align*}
            \phi(a\oplus_n b) &= (a\oplus_n b) + n\mathbb{Z}\\
            &= \{(a \oplus_n b) + pn \vert p \in \mathbb{Z}\}\\
            &= \{a+b + pn \vert p \in \mathbb{Z}\}\\
            &= \{a+b + pn + qn\vert p, q \in \mathbb{Z}\}\\
            &= a+b+n\mathbb{Z} + n\mathbb{Z}\\
            &= (a+n\mathbb{Z}) + (b + n\mathbb{Z})\\
            &= \phi(a) + \phi(b).
        \end{align*}
        \item \textbf{Injective}: Let $a$ and $b$ be in $G$ such that $\phi(a) = \phi(b)$. Thus
        \[
            \{a + pn \vert p \in \mathbb{Z} \} = \ \{b + qn \vert q \in \mathbb{Z} \}
        \]
        by definition of $\phi$. Hence $a \equiv b \pmod n$. But since $0 \leq a, b < n$, we must have $a = b$.
        \item \textbf{Surjective}: Let $x + n\mathbb{Z} \in H$. We use Euclid's division lemma (\myref{lemma-euclid-division}) on $x$ to yield
        \[
            x = qn + r, \text{ where } 0 \leq r < n.
        \]
        Note that
        \begin{align*}
            x + n\mathbb{Z} &= \{x + kn \vert k \in \mathbb{Z}\}\\
            &= \{(qn + r) + kn \vert k \in \mathbb{Z}\}\\
            &= \{r + n(\underbrace{q + k}_{\text{In }\mathbb{Z}}) \vertalt k \in \mathbb{Z} \}\\
            &= r + n\mathbb{Z}
        \end{align*}
        with $0 \leq r < n$, meaning $r \in G$. Now observe $\phi(r) = r+n\mathbb{Z} = x+n\mathbb{Z}$ which means that there is a pre-image for every element in $H$, hence proving that $\phi$ is surjective.
    \end{itemize}
    Therefore $\phi$ is an isomorphism, proving $G \cong H$.
    
    \item Consider the map $\phi: G \to G/N$ such that $g \mapsto gN$. We note that $\phi$ is a homomorphism as
    \[
        \phi(gh) = (gh)N = (gN)(hN) = \phi(g)\phi(H).
    \]
    We note by \myref{prop-homomorphism-inverse-is-subgroup} that $A = \phi^{-1}(B) \leq G$. Thus
    \begin{align*}
        \phi^{-1}(N) &= \{g \in G \vert \phi(g) = N\}\\
        &= \{g \in G \vert gN = N\}\\
        &= \{g \in G \vert g \in N\}\\
        &= G \cap N\\
        &= N\\
        &\subseteq A
    \end{align*}
    by assumption. Since $N$ is a group, we know $N \leq A$. Furthermore $N \leq A \leq G$ and $N \unlhd G$, meaning $N \unlhd A$ (since $gN = Ng$ for all $g \in G$, including those in $A$). Hence $A/N$ is a group.
    
    Now clearly $\phi$ is surjective (since for any $gN \in G/N$ we know $\phi(g) = gN$), which means that $\phi(\phi^{-1}(B)) = B$. Since $\phi^{-1}(B) = A$, so $\phi(A) = B$. Finally,
    \begin{align*}
        \phi(A) &= \{\phi(a) \vert a \in A\}\\
        &= \{aN \vert a \in A\}\\
        &= A/N
    \end{align*}
    which means $B = A/N$.
\end{questions}

\section{Symmetric Groups}
\begin{questions}
    \item We work from the right to the left.
    \begin{itemize}
        \item $\gamma \delta$ has cycle notation
        \begin{align*}
            &\begin{pmatrix}1 & 2 & 5\end{pmatrix}\begin{pmatrix}3 & 4\end{pmatrix}\begin{pmatrix}1 & 3 & 2 & 5\end{pmatrix}\\
            &= \begin{pmatrix}1 & 4 & 3 & 5 & 2\end{pmatrix};
        \end{align*}
        \item $\beta \gamma \delta$ has cycle notation
        \begin{align*}
            &\begin{pmatrix}1 & 5 & 2\end{pmatrix}\begin{pmatrix}3 & 4\end{pmatrix}\begin{pmatrix}1 & 4 & 3 & 5 & 2\end{pmatrix}\\
            &= \begin{pmatrix}1 & 3 & 2 & 5\end{pmatrix}\\
            &= \delta;
        \end{align*}
        and
        \item $\alpha \beta \gamma \delta$ has cycle notation
        \begin{align*}
            &\begin{pmatrix}1 & 5 & 2 & 3\end{pmatrix}\begin{pmatrix}1 & 3 & 2 & 5\end{pmatrix}\\
            &= \id,
        \end{align*}
        the identity.
    \end{itemize}

    \item Recall that $D_3$ has presentation
    \[
        \langle r, s \vert r^3 = s^2 = e, rs = sr^2 \rangle.
    \]

    Let the map $\phi: D_3 \to \Sn{3}$ be given such that $r \mapsto \begin{pmatrix}1 & 2 & 3\end{pmatrix}$ and $s \mapsto \begin{pmatrix}1 & 2\end{pmatrix}$. We show that $\begin{pmatrix}1 & 2 & 3\end{pmatrix}$ and $\begin{pmatrix}1 & 2\end{pmatrix}$ satisfy the two rules above. For brevity let $\sigma = \begin{pmatrix}1 & 2 & 3\end{pmatrix}$ and $\tau = \begin{pmatrix}1 & 2\end{pmatrix}$.
    \begin{itemize}
        \item We check that $\phi(r^3) = \phi(s^2) = \phi(e)$.
        \begin{itemize}
            \item $\sigma^2 = \begin{pmatrix}1 & 2 & 3\end{pmatrix}\begin{pmatrix}1 & 2 & 3\end{pmatrix} = \begin{pmatrix}1 & 3 & 2\end{pmatrix} \neq \id$;
            \item $\sigma^3 = \begin{pmatrix}1 & 2 & 3\end{pmatrix}\begin{pmatrix}1 & 3 & 2\end{pmatrix} = \id$; and
            \item $\tau^2 = \begin{pmatrix}1 & 2\end{pmatrix}\begin{pmatrix}1 & 2\end{pmatrix} = \id$.
        \end{itemize}
        \item We check that $\phi(rs) = \phi(sr^2)$.
        \begin{itemize}
            \item $rs \mapsto \sigma\tau = \begin{pmatrix}1 & 2 & 3\end{pmatrix}\begin{pmatrix}1 & 2\end{pmatrix} = \begin{pmatrix}1 & 3\end{pmatrix}$; and
            \item $sr^2 \mapsto \tau\sigma^2 = \begin{pmatrix}1 & 2\end{pmatrix}\begin{pmatrix}1 & 3 & 2\end{pmatrix} = \begin{pmatrix}1 & 3\end{pmatrix}$.
        \end{itemize}
    \end{itemize}
    Thus $\phi$ is an isomorphism and so $D_3 \cong \Sn{3}$.

    \item We note that $|\Sn{4}| = 4! = 24$.
    \begin{partquestions}{\alph*}
        \item Consider $H = \left\langle \begin{pmatrix}1 & 2 & 3 & 4\end{pmatrix} \right\rangle$. For brevity, let $\sigma = \begin{pmatrix}1 & 2 & 3 & 4\end{pmatrix}$. Note that
        \begin{itemize}
            \item $\sigma^2 = \begin{pmatrix}1 & 3\end{pmatrix}\begin{pmatrix}2 & 4\end{pmatrix} \neq \id$;
            \item $\sigma^3 = \begin{pmatrix}1 & 4 & 3 & 2\end{pmatrix} \neq \id$; and
            \item $\sigma^4 = \id$.
        \end{itemize}
        Thus, $|\sigma| = 4$ which means $|H| = 4$. Therefore, $G \cong H \leq \Sn{4}$.

        \item Let $\sigma = \begin{pmatrix}1 & 2\end{pmatrix}$ and $\tau = \begin{pmatrix}3 & 4\end{pmatrix}$. Let $H$ have presentation $\langle \sigma, \tau \rangle$. Notice that
        \begin{itemize}
            \item $\sigma^2 = \id$;
            \item $\tau^2 = \id$; and
            \item $(\sigma\tau)^2 = \id$.
        \end{itemize}
        Therefore $H = \{\id, \sigma, \tau, \sigma\tau\}$, so $G \cong H \leq \Sn{4}$.
    \end{partquestions}
\end{questions}

\chapter{Direct Products of Groups}
Previously, we considered ways to transform elements from one group to another using homomorphisms. We now slightly shift our focus to ways to combine groups together, using the direct product.

\section{External Direct Product}
The external direct product is one method of `combining' groups together.
\begin{definition}
    Let $(G, \ast)$ and $(H, \star)$ be groups. The \textbf{external direct product}\index{direct product!external} of $G$ and $H$ is denoted by $G\times H$ and is the group $(G\times H, (\ast, \star))$, where $(\ast, \star)$ are the operators performed component-wise.
\end{definition}
Verifying that $(G\times H, (\ast, \star))$ is a group is left as an exercise for the reader. We note the identity of $G \times H$ is $(e_G, e_H)$, where $e_G$ is the identity of $G$ and $e_H$ is the identity of $H$. Also, $|G \times H| = |G||H|$ by definition of the Cartesian product.

\begin{example}
    $\mathbb{Z}_5 \times \mathbb{Z} = \{(m,n) \vert m \in \mathbb{Z}_5 \text{ and } n \in \mathbb{Z}\}$.

    Thus, $(2, 5)(5, 9) = (2 \oplus_5 5, 5 + 9) = (7 \mod 5, 14) = (2, 14)$.
\end{example}

\begin{example}
    Let $\mathcal{S} = \mathbb{R}\setminus\{0\}$ be a group under multiplication. Then $\mathcal{S} \times \mathbb{Z}_3 = \{ (x, n) \vert x \in \mathcal{S} \text{ and } n \in \mathbb{Z}_3\}$.

    So, $(3, 2)^{-1} = \left(\frac13, 1\right)$ since
    \begin{align*}
        (3, 2)\left(\frac13, 1\right) &= \left(3 \times \frac13, 2 \oplus_3 1\right)\\
        &= (1, 0)\\
        &= (\text{Identity in }\mathcal{S}, \text{Identity in }\mathbb{Z}_3)
    \end{align*}
\end{example}

\begin{exercise}
    Simplify $(s, rs)(r^2s, r^3)$ in $D_3 \times D_4$.
\end{exercise}

We prove some results relating the external direct product.
\begin{proposition}\label{prop-order-of-element-in-external-direct-product}
    Let $G_1$ and $G_2$ be groups with identities $e_1$ and $e_2$ respectively. Let $(x, y) \in G_1 \times G_2$, $|x| = r$, and $|y| = s$. Then $|(x, y)| = \lcm(r, s)$.
\end{proposition}
\begin{proof}[Proof (see \cite{proofwiki_order-of-group-element-in-external-direct-product})]
    For brevity let $l = \lcm(r, s)$, so $l = \alpha r = \beta s$ for some positive integers $\alpha$ and $\beta$. Let $m = |(x, y)|$.

    Note that
    \begin{align*}
        (x, y)^l &= (x^l, y^l)\\
        &= (x^{\alpha r}, y^{\beta s})\\
        &= \left((x^r)^\alpha, (y^s)^\beta\right)\\
        &= (e_1^\alpha, e_2^\beta)\\
        &= (e_1, e_2).
    \end{align*}
    Therefore $l = k|(x, y)| = km$ for some positive integer $k$, i.e. $m\vert l$.

    Note also that $(x, y)^m = (x^m, y^m)$ and $(x, y)^m = (e_1, e_2)$. Therefore $x^m = e_1$ and $y^m = e_2$, which means $m = p|x| = pr$ and $m = q|y| = qs$ for some positive integers $p$ and $q$, so $r$ and $s$ both divide $m$. Therefore $\lcm(r, s) = l\vert m$.

    Since $m\vert l$ and $l\vert m$, thus $m = l$. Therefore we see that $|(x, y)| = \lcm(|x|, |y|)$.
\end{proof}

\begin{theorem}\label{thrm-Zm-cross-Zn-isomorphic-to-Zmn-condition}
    $\mathbb{Z}_m \times \mathbb{Z}_n \cong \mathbb{Z}_{mn}$ if and only if $\gcd(m,n) = 1$.
\end{theorem}
\begin{proof}[Proof (see {\cite[Proposition 13.1 (3)]{humphreys_1996}})]
    We first work on the forward direction. Suppose $\mathbb{Z}_m \times \mathbb{Z}_n \cong \mathbb{Z}_{mn}$, and for brevity let $d = \gcd(m,n)$.
    
    Suppose on the contrary that $d > 1$. Take $(a, b) \in \mathbb{Z}_m \times \mathbb{Z}_n$. Note that $\frac{mn}{d} = \lcm(m,n)$ (\myref{prop-product-of-gcd-and-lcm}) which is a positive integer. Then
    \begin{align*}
        \underbrace{(a,b)(a,b)(a,b)\cdots(a,b)}_{\frac{mn}{d}\text{ times}} &= \left(\frac{mn}{d}a, \frac{mn}{d}b\right)\\
        &= \left(m\frac{na}{d}, n\frac{mb}{d}\right) & (\text{as } d \vert m \text{ and } d \vert n)\\
        &= (\underbrace{0}_{\text{In } \mathbb{Z}_m}, \underbrace{0}_{\text{In } \mathbb{Z}_n})
    \end{align*}
    which implies $|(a, b)| \leq \frac{mn}{d} < mn$ for all $(a, b) \in \mathbb{Z}_m \times \mathbb{Z}_n$. Hence, this means that $\mathbb{Z}_m \times \mathbb{Z}_n$ is \textit{not} cyclic, since $|\mathbb{Z}_m \times \mathbb{Z}_n| = mn$ and no element in $\mathbb{Z}_m \times \mathbb{Z}_n$ has order $mn$. However, $\mathbb{Z}_m \times \mathbb{Z}_n \cong \mathbb{Z}_{mn}$ is cyclic, a clear contradiction. Hence, $d \not>1$ which means $d = 1$, so $\gcd(m,n) = 1$.

    We now work on the reverse direction. Suppose $\gcd(m,n) = 1$. Note that $|1| = m$ in $\mathbb{Z}_m$ and $|1| = n$ in $\mathbb{Z}_n$. Thus
    \begin{align*}
        |(1, 1)| &= \lcm(m, n) & (\myref{prop-order-of-element-in-external-direct-product})\\
        &= \frac{mn}{\gcd(m,n)}\\
        &= mn & (\gcd(m,n) = 1).
    \end{align*}
    Since $|\mathbb{Z}_m \times \mathbb{Z}_n| = mn$ and $|(1,1)| = mn$, thus $\mathbb{Z}_m \times \mathbb{Z}_n$ is cyclic with generator $(1,1)$. By \myref{thrm-finite-cyclic-group-isomorphic-to-Zn}, $\mathbb{Z}_m \times \mathbb{Z}_n \cong \mathbb{Z}_{mn}$.
\end{proof}

\begin{exercise}
    Find all pairs of integers $(m, n)$ with $1 < m < n$ such that $\mathbb{Z}_m \times \mathbb{Z}_n \cong \mathbb{Z}_{180}$.
\end{exercise}

\section{Internal Direct Product}
Before we look at the internal direct product, we look at the \textbf{subgroup product}\index{subgroup product}.
\begin{definition}
    Let $G$ be a group and $H, K \leq G$. Then
    \[
        HK = \{hk \vert h \in H, k \in K\}
    \]
    is called the \textbf{subgroup product of $H$ and $K$}.
\end{definition}

\begin{proposition}\label{prop-subgroup-product-is-subgroup}
    Let $G$ be a group and $H, K \leq G$. Then $HK \leq G$ if and only if $HK = KH$.
\end{proposition}

An important point to note is that $HK = KH$ does not imply that $hk = kh$ for all $h \in H$ and $k \in K$. Finding a counterexample to this claim is left as an exercise to the reader.

\begin{proof}
    We first prove the forward direction; assume that $HK \leq G$. First take an arbitrary $kh \in KH$. Then we note
    \begin{align*}
        kh &= \left(\left(kh\right)^{-1}\right)^{-1}\\
        &= (\underbrace{h^{-1}}_{\text{In } H}\underbrace{k^{-1}}_{\text{In } K})^{-1}\\
        &\in HK
    \end{align*}
    since $HK \leq G$ so the inverse of any element is in $HK$. Therefore any element in $KH$ is also in $HK$, meaning that $KH \subseteq HK$. Now take an arbitrary $hk \in HK$. Note that $k^{-1}h^{-1} = (hk)^{-1} \in HK$, so set $k^{-1}h^{-1} = \hat{h}\hat{k}$ for some $\hat{h} \in H$ and $\hat{k} \in K$. Hence $hk = \left(k^{-1}h^{-1}\right)^{-1} = \hat{k}^{-1}\hat{h}^{-1} \in KH$, meaning $HK \subseteq KH$. Therefore $HK = KH$ as needed.

    We now prove the reverse direction; assume that $HK = KH$. Clearly $e \in HK$ since $e \in H$ and $e \in K$. Now suppose $h_1k_1$ and $h_2k_2$ are in $HK$. We note
    \begin{align*}
        (h_1k_1)(h_2k_2)^{-1} &= h_1k_1k_2^{-1}h_2^{-1}\\
        &= h_1(\underbrace{k_1k_2^{-1}h_2^{-1}}_{\text{In } KH = HK})\\
        &= h_1(h'k') & (\text{set }(k_1k_2^{-1})h_2^{-1} = h'k')\\
        &= (\underbrace{h_1h'}_{\text{In } H})k'\\
        &\in HK
    \end{align*}
    so by subgroup test, $HK \leq G$.
\end{proof}

We now look at the \textbf{internal direct product}, another way of combining elements of two groups.

\begin{definition}
    Let $G$ be a group and $H$ and $K$ be subgroups of $G$ such that
    \begin{enumerate}
        \item $G = HK$,
        \item $H \cap K = \{e\}$, and
        \item for all $h \in H$ and $k \in K$, $hk = kh$.
    \end{enumerate}
    Then $G$ is said to be the \textbf{internal direct product}\index{direct product!internal} of $H$ and $K$.
\end{definition}

\begin{example}
    Consider $G = D_6$, $H = \langle r^3 \rangle$ and $K = \langle s, r^2 \rangle$. Note that
    \begin{align*}
        HK &= \{hk \vert h \in H, k \in K\}\\
        &= \{e, r^2, r^4, s, r^2s, r^4s, r^3, r^5, r^7, r^3s, r^5s, r^7s\}\\
        &= \{e, r^2, r^4, s, r^2s, r^4s, r^3, r^5, r, r^3s, r^5s, rs\}\\
        &= D_6\\
        &= G
    \end{align*}
    so $G = HK$. Note also that $H \cap K = \{e\}$ and $hk = kh$ for all $h$ in $H$ and $k$ in $K$. Thus $G$ is the internal direct product of $H$ and $K$.
\end{example}

\begin{exercise}
    Let $G = \{1, 5\}$ and $H = \{1, 7\}$ be groups under $\otimes_{12}$. Find the internal direct product of $G$ and $H$.
\end{exercise}

\section{The Isomorphism Between Them}
It is certainly tiring to remember that there is an \textit{external} direct product and an \textit{internal} direct product. One might rightly wonder whether we can simplify both into one unified ``direct product''. In fact, there exists an isomorphism between the external direct product and the internal direct product of two groups.

\begin{theorem}\label{thrm-direct-product-equivilance}
    If $G$ is the internal direct product of $H$ and $K$, then $G \cong H \times K$.
\end{theorem}

\newpage

\begin{proof}
    Let $\phi: G \to H \times K$, $g \mapsto (h, k)$ where $g = hk$. We will show that $\phi$ is a well-defined isomorphism.
    \begin{itemize}
        \item \textbf{Well-defined}: Suppose $g = hk = h'k'$ where $h, h' \in H$ and $k, k' \in K$. Since $hk = h'k'$ thus $h^{-1}h' = k(k')^{-1}$.

        Note that $h^{-1}h' \in H$ and $k(k')^{-1} \in K$. So if $h^{-1}h' = k(k')^{-1}$ then $h^{-1}h' \in H \cap K$ and $k(k')^{-1} \in H \cap K$. But $H \cap K = \{e\}$. Thus $h^{-1}h' = k(k')^{-1} = e$, which means that $h = h'$ and $k = k'$.
        
        \item \textbf{Homomorphism}: Let $g, g' \in G$, $h, h' \in H$, and $k, k' \in K$ such that $g = hk$ and $g' = h'k'$. Then $gg' = hkh'k' = h(kh')k'$. We note that $kh' = h'k$ since $hk = kh$ for all $h \in H$ and $k \in K$. Hence $gg' = hh'kk'$.
        
        Thus
        \[
            \phi(gg') = \phi(\underbrace{hh'}_{\text{In }H}\underbrace{kk'}_{\text{In }K}) = (hh', kk').
        \]
        Now we are in the group $H \times K$, so $(h, k)(h', k') = (hh', kk')$. Therefore,
        \begin{align*}
            \phi(gg') &= (h,k)(h',k')\\
            &= \phi(hk)\phi(h'k')\\
            &= \phi(g)\phi(g')
        \end{align*}
        which means that $\phi$ is a homomorphism.
        
        \item \textbf{Injective}: Let $g, g' \in G$, $h, h' \in H$, and $k, k' \in K$ such that $g = hk$, $g' = h'k'$, and $\phi(g) = \phi(g')$. Then $\phi(hk) = \phi(h'k')$, meaning $(h,k) = (h',k')$. Thus $h = h'$ and $k = k'$ by equality of ordered pairs, meaning $g = hk = h'k' = g'$.
        
        \item \textbf{Surjective}: Let $(h, k) \in H \times K$. Note that $hk \in G$ since $G$ is the internal direct product of $H$ and $K$, so $\phi(hk) = (h, k)$. Thus a pre-image of $(h, k)$ is $hk$.
    \end{itemize}
    Therefore $\phi$ is a well-defined isomorphism from $G$ to $H \times K$, meaning $G \cong H \times K$.
\end{proof}

\begin{example}
    Consider again $G = D_6$, $H = \langle r^3 \rangle$ and $K = \langle s, r^2 \rangle$. As we have found before, $G$ is the internal direct product of $H$ and $K$; but we now know that $G \cong H \times K$.

    Note that $|H| = 2$ so $H \cong \mathbb{Z}_2$ and $K \cong D_3$ (we leave the latter as an exercise for the reader to prove). Thus,
    \[
        D_6 = G = HK \cong H \times K = \mathbb{Z}_2 \times D_3.
    \]
\end{example}

\begin{exercise}
    Let $\mathcal{S} = \{1, 5, 7, 11\}$, $G = \{1, 5\}$, and $H = \{1, 7\}$ be groups under $\otimes_{12}$. Find the value of $n$ such that $\mathcal{S} \cong (\mathbb{Z}_n)^2$.
\end{exercise}

\newpage

\section{Problems}
\begin{problem}\label{problem-external-direct-product-of-abelian-groups-is-abelian}
    Let $G$ and $H$ be abelian groups. Prove that $G \times H$ is also an abelian group.
\end{problem}

\begin{problem}
    Let $G$ and $H$ be groups. Prove that $G \times H \cong H \times G$.
\end{problem}

\begin{problem}
    Let $G = \mathbb{Z}_6$, and let the subgroups $H = \{0, 2, 4\}$ and $K = \{0, 3\}$. Determine whether $G$ is the internal direct product of $H$ and $K$.
\end{problem}

\begin{problem}
    Consider the \textit{Klein four-group}\index{Klein four-group} $\mathrm{V}$ with presentation
    \[
        \langle a, b \vert a^2 = b^2 = (ab)^2 = e \rangle.
    \]
    Show that $\mathrm{V} \cong (\mathbb{Z}_2)^2$.
\end{problem}

\chapter{Further Properties of Homomorphisms}
Earlier in this book, we introduced homomorphisms and isomorphisms, special types of maps that transform elements of one group to another. We look at more properties of such maps in this chapter and describe the uses of these new properties.

\section{Image of a Homomorphism}
As a homomorphism is a mapping between two groups, it is worthy to look at the \textbf{image} of the homomorphism.
\begin{definition}
    The \textbf{image}\index{homomorphism!image} (or \textbf{range}\index{homomorphism!range}) of a homomorphism $\phi: G \to H$ is the set
    \[
        \im\phi = \{\phi(g) \vert g \in G\}.
    \]
\end{definition}
\begin{remark}
    Some authors (e.g. {\cite[Definition 4.2.0]{libretexts_im-and-ker}}) will use the notation $\phi(G)$ for the image of $\phi$. The alternate notation $\mathrm{Im}\;\phi$ may also be used (e.g. by {\cite[Definition I.2.2]{hungerford_1980}} and \cite[\S 66]{clark_1984}).
\end{remark}

\begin{example}
    Consider the homomorphism $f: \Z \to \Z, x \mapsto 0$. Clearly, all possible values of $x$ maps to 0, so $\im f = \{0\}$.
\end{example}
\begin{example}
    The homomorphism $f: \R \to \R$ where $f(x) = |x|$ has an image of $\{x \in \R \vert x \geq 0\}$, i.e. all non-negative real numbers.
\end{example}

\begin{proposition}\label{prop-image-is-subgroup-of-codomain}
    Let $\phi: G \to H$ be a homomorphism. Then $\im\phi \leq H$.
\end{proposition}
\begin{proof}
    Note that $\phi(e_G) = e_H \in \im\phi$, where $e_G$ and $e_H$ are the identities of $G$ and $H$ respectively.
    
    Now suppose $h_1$ and $h_2$ are in the image of $\phi$, meaning that there exists $g_1$ and $g_2$ such that $\phi(g_1) = h_1$ and $\phi(g_2) = h_2$. Note that $\phi(g_2^{-1}) = h_2^{-1}$ by homomorphism property. Hence $\phi(g_1g_2^{-1}) = h_1h_2^{-1} \in \im\phi$.

    Therefore, by subgroup test, $\im\phi \leq H$.
\end{proof}

\begin{exercise}
    Consider the map $\phi: \Z_3 \to \Z_6, n \mapsto 2n$. Determine whether $\phi$ is a homomorphism and, if so, find its image.
\end{exercise}

\section{Kernel of a Homomorphism}
\begin{definition}
    The \textbf{kernel}\index{homomorphism!kernel} of a homomorphism $\phi: G \to H$ is the set of elements in the group $G$ which map to the identity in the group $H$. That is, if the identity of $H$ is $e_H$, then the kernel of $\phi$ is the set
    \[
        \ker\phi = \{x \in G \vert \phi(x) = e_H\}.
    \]
\end{definition}


\begin{remark}
    Some authors (e.g. {\cite[Definition 4.2.0]{libretexts_im-and-ker}}) will use the notation $\phi^{-1}(e_H)$ for the kernel of $\phi$. The alternate notation $\mathrm{Ker}\;\phi$ may also be used by some authors (e.g. by {\cite[Definition I.2.2]{hungerford_1980}} and \cite[\S 65]{clark_1984}).
\end{remark}

\begin{example}
    Let the groups $G = (\Z^2, (+, +))$ and $H = (\Z, +)$. Let the map $\phi: G \to H, (a, b) \mapsto a+b$. Then, $(a, b) \in \ker\phi$ if $\phi((a,b)) = 0$. This means that $a+b = 0$, implying $ b = -a$. Hence the kernel of $\phi$ is $\{(a, -a) \vert a \in \Z\}$.
\end{example}

\begin{exercise}
    Let $i$ be the imaginary unit, that is $i^2 = -1$. Let the group $G$ be the integers under addition and $H = \langle i \rangle$ be under multiplication. Let the map $\phi: G \to H, n \mapsto i^n$.  Show that $\phi$ is a homomorphism and hence find $\ker\phi$.
\end{exercise}

\begin{proposition}\label{prop-kernel-is-normal-subgroup-of-domain}
    Let $\phi: G \to H$ be a homomorphism. Then $\ker\phi \unlhd G$.
\end{proposition}
\begin{proof}
    We will first show $\ker\phi\leq G$. Clearly $e_G \in \ker\phi$ since $\phi(e_G) = e_H$, so $\ker\phi$ is non-empty. Now let $x, y \in \ker\phi$. This means that $\phi(x) = \phi(y) = e_H$. Note
    \begin{align*}
        \phi(xy^{-1}) &= \phi(x)\left(\phi(y)\right)^{-1}\\
        &= e_H(e_H)^{-1}\\
        &= e_H
    \end{align*}
    which means that $xy^{-1}\in\ker\phi$. By subgroup test, $\ker\phi\leq G$.

    Now we prove normality. Let $x \in G$ and $n \in \ker\phi$. We need to show that $xnx^{-1}\in\ker\phi$ to prove normality. Observe that
    \begin{align*}
        \phi(xnx^{-1}) &= \phi(x)\phi(n)\phi(x^{-1})\\
        &= \phi(x)e_H\phi(x)^{-1} & (n \in \ker\phi)\\
        &= \phi(x)\phi(x)^{-1}\\
        &= e_H,
    \end{align*}
    which means that $xnx^{-1} \in \ker\phi$. Hence, $\ker\phi \unlhd G$.
\end{proof}

\begin{exercise}\label{exercise-trivial-kernel-means-injective}
    Prove that a homomorphism $\phi:G\to H$ is injective if and only if $\ker \phi$ is trivial, i.e. $\ker \phi = \{e_G\}$.
\end{exercise}

\section{The Fundamental Homomorphism Theorem}
We are now ready to tackle the three most important theorems regarding homomorphisms. We first state the \textbf{Fundamental Homomorphism Theorem}, which is also sometimes called the \textbf{First Isomorphism Theorem} (e.g. in {\cite[p.~251, Theorem 3]{cohn_1982}}).
\begin{theorem}[Fundamental Homomorphism Theorem]\label{thrm-isomorphism-1}\index{Fundamental Homomorphism Theorem}\index{Isomorphism Theorem!First}
    Let $G$ and $H$ be groups. Let $\phi: G \to H$ be a homomorphism, and let $\pi: G \to G/\ker\phi$ where $g\mapsto g\ker\phi$ be the natural surjective homomorphism. Then there exists a unique isomorphism $\psi: G/\ker\phi \to \im\phi$ such that $\psi\pi = \phi$.
\end{theorem}
\begin{remark}
    Equivalently, the Fundamental Homomorphism Theorem states that
    \[
        G/\ker\phi \cong \im\phi
    \]
    for any homomorphism $\phi$.
\end{remark}

We include the commutativity diagram of the homomorphisms stated to aid clarity:

\begin{figure}[h]
    \centering
    \fbox{\includegraphics[width=0.35\textwidth]{further-homomorphisms/iso-1-comm-diagram.png}}
    \caption{Commutativity Diagram for \myreffigures{thrm-isomorphism-1}}
\end{figure}

In the diagram, $\phi$ sends elements from $G$ to $\im\phi$ and $\pi$ sends elements from $G$ to $G/\ker\phi$. Then the map $\psi$ is a unique map that sends elements from $G/\ker\phi$ to the image of $\phi$.

\begin{proof}
    We know by \myref{prop-image-is-subgroup-of-codomain} that $\im\phi \leq H$. Let $\psi: G/\ker\phi \to \im\phi$ such that $\psi(x\ker\phi) = \phi(x)$. We need to check that $\psi$ is a well-defined isomorphism.
    \begin{itemize}
        \item \textbf{Well-defined}: Suppose $x\ker\phi = y\ker\phi$ where $x, y \in G$. Then $xy^{-1} \in \ker\phi$ by Coset Equality (\myref{lemma-coset-equality}), statements 1 and 5. This means that $\phi(xy^{-1}) = e_H$ by definition of the kernel. Note $\phi(xy^{-1}) = \phi(x)\left(\phi(y)\right)^{-1}$, so $\phi(x)\left(\phi(y)\right)^{-1} = e_H$. Hence $\phi(x) = \phi(y)$. Thus,
        \[
            \psi(x\ker\phi) = \phi(x) = \phi(y) = \psi(y\ker\phi)
        \]
        so $\psi$ is well-defined.

        \item \textbf{Homomorphism}: Note that
        \begin{align*}
            \psi((x\ker\phi)(y\ker\phi)) &= \psi((xy)\ker\phi)\\
            &= \phi(xy)\\
            &= \phi(x)\phi(y)\\
            &= \psi(x\ker\phi)\psi(y\ker\phi)
        \end{align*}
        so $\psi$ is a homomorphism.
        \item \textbf{Injective}: By \myref{exercise-trivial-kernel-means-injective}, we check that $\psi$ is injective by showing that $\ker\psi$ is trivial, i.e. $\ker\psi = \{\ker\phi\}$.

        Suppose $x\ker\phi\in\ker\psi$. Then $\psi(x\ker\phi) = e_H$ by definition of kernel. Hence $\phi(x) = e_H$ by definition of $\psi$, which means $x \in \ker\phi$ by definition of kernel. Thus $x\ker\phi = \ker\phi$ by Element in Coset (\myref{corollary-equivalence-of-element-in-coset}). Therefore $\psi$ is injective.

        \item \textbf{Surjective}: Suppose $y$ is in the image of $\phi$, meaning there exists a $x \in G$ such that $\phi(x) = y$. Note that $\psi(x\ker\phi) = \phi(x) = y$. Thus $\psi$ is surjective.
    \end{itemize}
    Thus $\psi$ is a well-defined isomorphism.

    We now check that $\psi$ satisfies the requirement that $\psi\pi = \phi$. Let $x \in G$. Note that $\pi(x) = x\ker\phi$, and
    \[
        \psi\pi(x) = \psi(x\ker\phi) = \phi(x)
    \]
    for all $x \in G$, so $\psi\pi = \phi$.

    Finally we show that $\psi$ is unique. Suppose $f: G/\ker\phi \to \im\phi$ is an isomorphism satisfying $f\pi=\phi$. Take $x\ker\phi \in G/\ker\phi$. Note that
    \begin{align*}
        f(x\ker\phi) &= f(\pi(x))\\
        &= (f\pi)(x)\\
        &= \phi(x)\\
        &= (\psi\pi)(x)\\
        &= \psi(\pi(x))\\
        &= \psi(x\ker\phi)
    \end{align*}
    for all $x \in G$, meaning that $f = \psi$. Therefore $\psi$ is unique.

    Hence, $\psi$ is a unique isomorphism satisfying $\psi\pi = \phi$.
\end{proof}

\begin{example}
    Let $R = \{x \in \R \vert x > 0\}$, $G = \{x \in \R \vert x \neq 0\}$, and $H = \{1, -1\}$ be groups under multiplication. We show $G / H \cong R$.

    Consider the map $\phi: G \to R$ where $x \mapsto |x|$. We show that $\phi$ is a homomorphism, then find the image of $\phi$, and finally find its kernel.

    \begin{itemize}
        \item \textbf{Homomorphism}: $\phi$ is a homomorphism since $\phi(xy) = |xy| = |x||y| = \phi(x)\phi(y)$.
        \item \textbf{Image}: We find the image of $\phi$.
        \begin{align*}
            \im\phi &= \{\phi(x) \vert x \in G\}\\
            &= \{|x| \vert x \neq 0\}\\
            &= \{x \in \R \vert x > 0\} & (\text{by definition of } |x|)\\
            &= R
        \end{align*}
        which actually means that $\phi$ is surjective.
        \item \textbf{Kernel}: We find the kernel of $\phi$.
        \begin{align*}
            \ker\phi &= \{x \in G \vert \phi(x) = 1\} & (1 \text{ is the identity in } R)\\
            &= \{x \in G \vert |x| = 1\}\\
            &= \{1, -1\}\\
            &= H
        \end{align*}
    \end{itemize}
    Thus $G/H \cong R$ by the Fundamental Homomorphism Theorem (\myref{thrm-isomorphism-1}).
\end{example}

\begin{exercise}
    Let $\phi: G \to H$ be a homomorphism between finite groups $G$ and $H$. Prove that
    \[
        |G| = |\im \phi|\times|\ker \phi|.
    \]
\end{exercise}

\section{The Diamond Isomorphism Theorem}
We now look at the next theorem, called the \textbf{Diamond Isomorphism Theorem} (e.g. in {\cite[Theorem 3.18]{dummit_foote_2004}}) or the \textbf{Second Isomorphism Theorem} (e.g. in {\cite[\S 69]{clark_1984}}).
\begin{theorem}[Diamond Isomorphism Theorem]\label{thrm-isomorphism-2}\index{Diamond Isomorphism Theorem}\index{Isomorphism Theorem!Second}
    Let $G$ be a group and let $H$ and $K$ be subgroups of $G$. Then
    \begin{enumerate}
        \item $H \cap K \leq H$; and
        \item $H \leq HK$.
    \end{enumerate}
    Furthermore, if $N \unlhd G$, then
    \begin{enumerate}[start=3]
        \item $HN \leq G$;
        \item $H \cap N \unlhd H$;
        \item $N \unlhd HN$; and
        \item $H / (H\cap N) \cong HN / N$.
    \end{enumerate}
\end{theorem}

\newpage

We can capture the overall relationships of the subgroups of $G$ using a \textbf{subgroup lattice}.
\begin{figure}[h]
    \centering
    \fbox{\includegraphics[width=0.25\textwidth]{further-homomorphisms/iso-2-subgroup-diagram.png}}
    \caption{Subgroup Lattice for \myreffigures{thrm-isomorphism-2}}
\end{figure}

We only show subgroups that we care about in the diagram. The group $G$ has a (direct) subgroup $HK$; $HK$ has subgroups $H$ and $K$; and $H$ and $K$ has a common subgroup $H\cap K$. The dotted quotient groups are isomorphic to each other if $H \unlhd G$.

\begin{proof}
    We prove each statement in sequence.

    \begin{enumerate}
        \item Clearly $e_G \in H$ and $e_G \in K$ so $e_G \in H \cap K$. Now take $x, y \in H \cap K$, meaning $x, y \in H$ and $x, y \in K$. Since $H, K \leq G$ so $xy^{-1} \in H$ and $xy^{-1} \in K$. Thus $xy^{-1} \in H \cap K$. By the subgroup test, this means that $H \cap K \leq H$.
        
        \item Note that $H = \{he_G \vert h \in H\} \subseteq \{hk \vert h \in H, k \in K\} = HK$, and $H$ is a group (as $H$ is a subgroup). Therefore $H \leq HK$.
        
        \item We note that, because $N$ is normal, hence $hN = Nh$ for all $h \in H \subseteq G$, meaning that $HN = NH$. Therefore by \myref{prop-subgroup-product-is-subgroup}, we have $HN \leq G$.

        \item We know $H \cap N \leq H$ by statement 1, so we only prove normality. Take $x \in H \cap N$. Since $H \leq G$, thus $x \in H \cap N \subseteq H$, meaning for all $g \in H$, $gxg^{-1} \in H$ (where we think of $g$ and $x$ as being in $H$). But since $x \in H \cap N \subseteq N$ and $N \unlhd G$, thus $gxg^{-1} \in N$ (where we think of $g \in H$ and $x \in N$). Therefore $H \cap N \unlhd H$.

        \item We know $N \leq HN$ by statement 2, so we only prove normality. Take $n \in N$ and $x \in HN$ such that $x = h_xn_x$. Then
        \begin{align*}
            xnx^{-1} &= (h_xn_x)n(h_xn_x)^{-1}\\
            &= (h_xn_x)n(n_x^{-1}h_x^{-1}) & (\text{Shoes and Socks})\\
            &= \underbrace{h_x}_{\text{In }G}\underbrace{n_xnn_x^{-1}}_{\text{In }N}\underbrace{h_x^{-1}}_{\text{In G}}\\
            &\in N
        \end{align*}
        since $N \unlhd G$. This proves that $N \unlhd HN$.

        \item This is the main result of this theorem. We define $\phi: H \to HN/N, h \mapsto hN$. We show that $\phi$ is a homomorphism and then find its image and kernel.
        \begin{itemize}
            \item \textbf{Homomorphism}:
            \[
                \phi(xy) = (xy)N = (xN)(yN) = \phi(x)\phi(y)    
            \]

            \item \textbf{Image}: We show that $\phi$ is surjective to show that $\im\phi = HN/N$. Suppose $x \in HN$, meaning $x = hn$ where $h \in H$ and $n \in N$. Thus $xN \in HN/N$, so
            \[
                xN = (hn)N = h(nN) = hN
            \]
            meaning $\phi(h) = hN = xN$. Hence we have found a pre-image of the coset $xN$, meaning $\phi$ is surjective. Thus $\im \phi = HN/N$.

            \item \textbf{Kernel}: We claim that $\ker\phi = H \cap N$.
            
            Note that $\ker\phi = \{h \in H \vert \phi(h) = eN = N\}$ by definition of kernel. This means that if $h \in \ker\phi$ then $\phi(h) = N$. Hence $\phi(h) = hN = N$, which means $h \in N$ by Element in Coset (\myref{corollary-equivalence-of-element-in-coset}). Thus, $h \in H$ and $h \in N$, meaning $h \in H \cap N$. Therefore $\ker \phi \subseteq H \cap N$.
    
            Now suppose $x \in H \cap N$. This means that $x \in N$ necessarily, implying $xN = N$. Thus $\phi(x) = N$ which quickly implies $x \in \ker\phi$. Therefore $H \cap N \subseteq \ker\phi$.
    
            Since $\ker \phi \subseteq H \cap N$ and  $H \cap N \subseteq \ker\phi$ therefore $\ker\phi = H\cap N$.
        \end{itemize}

        By the Fundamental Homomorphism Theorem (\myref{thrm-isomorphism-1}),
        \[
            H / \ker\phi \cong \im \phi,
        \]
        which means
        \[
            H/(H\cap N) \cong HN/N.
        \]
    \end{enumerate}
    This completes the proof of the theorem.
\end{proof}

\begin{corollary}\label{corollary-subgroup-product-is-normal-subgroup-if-subgroups-are-normal}
    Let $G$ be a group with proper subgroups $H$ and $K$. Then $HK \unlhd G$ if $H$ and $K$ are normal subgroups of $G$.
\end{corollary}
\begin{proof}
    Assume that $H, K \lhd G$. By the Diamond Isomorphism Theorem (\myref{thrm-isomorphism-2}), statement 3, we know that $HK \leq G$ since $H \lhd G$. We just need to prove normality. Suppose $hk \in HK$ and take $g \in G$. Then
    \begin{align*}
        g(hk)g^{-1} &= (gh)(kg^{-1})\\
        &= (hg)(g^{-1}k) & (\text{as } H, K \lhd G)\\
        &= h(gg^{-1})k\\
        &= hk \in HK
    \end{align*}
    which means that $HK \unlhd G$.
\end{proof}

We look at two examples using the Diamond Isomorphism Theorem.
\begin{example}
    We say a group $G$ is \textbf{metabelian}\index{metabelian} if and only if there exists $A \unlhd G$ such that $A$ and $G/A$ are both abelian. We will prove that any subgroup of a metabelian group is also metabelian.

    Let $H \leq G$. Then, by the Diamond Isomorphism Theorem (\myref{thrm-isomorphism-2}), we know $H \cap A \unlhd H$ (statement 4) and $H/(H \cap A) \cong HA / A$ (statement 6). We just need to prove that both $H \cap A$ and $H/(H \cap A)$ are abelian.
    \begin{itemize}
        \item Consider any two elements from $H \cap A$, say $x$ and $y$. Then $x \in A$ and $y \in A$, so $xy = yx$ as $A$ is abelian. Hence, elements from $H \cap A$ commute, meaning that $H \cap A$ is abelian.
        \item Consider $H/(H\cap A) \cong HA / A$. Note that $HA \leq G$ since $H \leq G$ and $A \leq G$. Thus, $HA / A \leq G / A$. Note that $G/A$ is abelian by definition of metabelian group. Hence, $H/(H \cap A)$ is also abelian.
    \end{itemize}
    Therefore, we have found a subgroup of $H$ (in particular $H \cap A$) such that both $H \cap A$ and $H/(H\cap A)$ are both abelian. Hence, $H$ is metabelian.
\end{example}

We look at another application of the Diamond Isomorphism Theorem, which has use in Number Theory.
\begin{example}
    We will prove that $\lcm(m,n)\times\gcd(m,n) = mn$ (\myref{prop-product-of-gcd-and-lcm}) by considering the Diamond Isomorphism Theorem. For brevity, let $d = \gcd(m,n)$ and $l = \lcm(m,n)$.

    Consider the groups $G = \Z$, $H = m\Z$, and $N = n\Z$ under addition. By Diamond Isomorphism Theorem (\myref{thrm-isomorphism-2}),
    \[
        m\Z/(m\Z \cap n\Z) \cong (m\Z + n\Z)/(n\Z).
    \]

    Now $m\Z \cap n\Z$ is the set of integers that are both a multiple of $m$ and $n$. Hence, $m\Z \cap n\Z = \lcm(m,n)\Z = l\Z$. On the other hand, $m\Z + n\Z$ is the set of all integers of the form $mx+ny$ where $x$ and $y$ are integers. B\'{e}zout's lemma (\myref{lemma-bezout}) tells us that this set consists of the multiples of $\gcd(m,n)$, i.e. $m\Z + n\Z = \gcd(m,n)\Z = d\Z$. Hence,
    \[
        m\Z/(l\Z) \cong d\Z/(n\Z).
    \]

    We claim that $m\Z / (l\Z) \cong \Z_{\frac lm} \text{ and } d\Z / (n\Z) \cong \Z_{\frac nd}$. This is a specific case of \myref{problem-mZ/nZ-isomorphic-to-Zn/m} which we have left as a problem for later. Hence,
    \[
    \Z_{\frac lm} \cong m\Z/(l\Z) \cong d\Z/(n\Z) \cong \Z_{\frac nd},
    \]
    which means that $\Z_{\frac lm} \cong \Z_{\frac nd}$. We can now finally take orders on both sides:
    \[
        \frac{l}{m} = \frac{n}{d},
    \]
    which means that $ld = mn$. Hence, $\lcm(m,n)\times\gcd(m,n) = mn$.
\end{example}

\begin{exercise}\label{exercise-order-of-subgroup-product}
    Let $G$ be a finite group, $H \leq G$, and $N \lhd G$. Prove that
    \[
        |HN| = \frac{|H||N|}{|H \cap N|}.
    \]
\end{exercise}

\section{The Third Isomorphism Theorem}
We look at the last important theorem regarding homomorphisms and isomorphisms. This is often called the \textbf{Third Isomorphism Theorem} (e.g. in {\cite[Corollary I.5.10]{hungerford_1980}} and {\cite[pp.~253--254, Theorem 5]{cohn_1982}}).

It should be noted that there is no consistency with the numbering of these theorems in books (cf. {\cite[\S 68]{clark_1984}} as ``First Isomorphism Theorem'', {\cite[Theorem 8.16]{humphreys_1996}} as ``Second Isomorphism Theorem''), but the name ``Third Isomorphism Theorem'' is the easiest to research. Hence, we use that name here.

\begin{theorem}[Third Isomorphism Theorem]\label{thrm-isomorphism-3}\index{Isomorphism Theorem!Third}
    Let $G$ be a group. Let $H \unlhd G$ and $N \unlhd G$. Suppose $N \subseteq H$. Then
    \begin{enumerate}
        \item $N \unlhd H$;
        \item $H/N \unlhd G/N$; and
        \item $\frac{G/N}{H/N} \cong G/H$.
    \end{enumerate}
\end{theorem}
\begin{proof}
    We prove the statements in sequence.

    \begin{enumerate}
        \item We note that since $N \subseteq H$ and $N$ is a group (since $N$ is a normal subgroup of $G$) thus $N \leq H$. We just need to prove normality. Since $H$ and $N$ are normal subgroups of $G$, thus for all $g \in G$,
        \[
            gH = Hg \text{ and } gN = Ng.
        \]
        Now since $N \subseteq H \subseteq G$, thus for all $n$ in $N$, $nH = Hn$ (since $n \in G$). This means that $N \unlhd H$.

        \item We first prove that it is a subgroup before proving normality.

        Clearly $N = eN \in H/N$. Let $x$ and $y$ be in $H/N$. Then $x=h_xN$ and $y=h_yN$ for some $h_x, h_y \in H$. Note that $y^{-1} = (h_y^{-1})N$ by group operator on cosets. Hence,
        \begin{align*}
            xy^{-1} &= (h_xN)(h_y^{-1}N)\\
            &= (\underbrace{h_xh_y^{-1}}_{\text{In }H})N\\
            &\in H/N
        \end{align*}
        Hence, by subgroup test, $H/N \leq G/N$.

        Now let $gN \in G/N$ and $hN \in H/N$. We need to show that $(gN)(hN)(gN)^{-1} \in H/N$. Note $(gN)(hN)(gN)^{-1} = (ghg^{-1})N$. Since $H \unlhd G$, thus $ghg^{-1} \in H$ which means that $(ghg^{-1})N \in H/N$.

        Therefore $H/N \unlhd G/N$.

        \item This is the main result of the theorem.

        Define $\phi: G/N \to G/H, gN \mapsto gH$. We check that $\phi$ is a well-defined homomorphism and find its image and kernel.
        \begin{itemize}
            \item \textbf{Well-defined}: Suppose $gN = g'N$. Then $g(g')^{-1} \in N$ by Coset Equality (\myref{lemma-coset-equality}), statements 1 and 5. Since $N \subseteq H$, thus $g(g')^{-1} \in H$ which implies $gH = g'H$, again by Coset Equality, statements 1 and 5. Hence
            \[
                \phi(gN) = gH = g'H = \phi(g'N)
            \]
            so $\phi$ is well-defined.

            \item \textbf{Homomorphism}: Take $gN, g'N \in G/N$. Then
            \begin{align*}
                \phi((gN)(g'N)) &= \phi((gg')N)\\
                &= (gg')H\\
                &= (gH)(g'H)\\
                &= \phi(gN)\phi(g'N)
            \end{align*}
            which means that $\phi$ is a homomorphism.
            
            \item \textbf{Image}: We show $\phi$ is surjective to prove that $\im\phi = G/H$. Suppose $gH \in G/H$. Clearly $\phi(gN) = gH$. Thus $gN$ is a pre-image of $gH$, meaning that $\phi$ is surjective. Hence $\im\phi = G/H$.
            
            \item \textbf{Kernel}: Suppose $gN \in \ker\phi = \{gN \vert \phi(gN) = eH = H\}$. Thus $\phi(gN) = gH = H$, which means $g \in H$. Hence $gN \in H/N$, so $\ker\phi \subseteq H/N$.

            Now assume $hN \in H/N$. Since $H\subseteq G$ (as $H \unlhd G$), thus $h \in G$. Therefore $hN \in G/N$, so $\phi(hN) = hH = H$. Hence $hN \in \ker\phi$ which means $H/N \subseteq \ker\phi$.

            Since $\ker\phi \subseteq H/N$ and $H/N \subseteq \ker\phi$, we must have $\ker\phi = H/N$.
        \end{itemize}

        By the Fundamental Homomorphism Theorem (\myref{thrm-isomorphism-1}), we have $\frac{G/N}{\ker\phi} \cong \im\phi$, which means
        \[
            \frac{G/N}{H/N} \cong G/H,
        \]
        proving statement 3.
    \end{enumerate}
    This proves the theorem.
\end{proof}

\begin{example}
    Take $G = \Z$, $H = m\Z$, and $N = mn\Z$. Note that clearly $H, N \leq G$, and since $G$ is abelian, we must also have $H \unlhd G$ and $N \unlhd G$. By the Third Isomorphism Theorem (\myref{thrm-isomorphism-3}), statement 3,
    \[
        \frac{G/N}{H/N} \cong G/H.
    \]
    Note $G/H = \Z/(m\Z) \cong \Z_m$ by \myref{problem-Zn-isomorphic-to-Z-by-nZ}. Note also
    \[
        \frac{G/N}{H/N} = \frac{\Z/(mn\Z)}{m\Z/(mn\Z)}.
    \]
    Hence we see that
    \[
        \frac{\Z/(mn\Z)}{m\Z/(mn\Z)} \cong \Z/(m\Z) \cong \Z_m.
    \]
\end{example}

\begin{exercise}
    Suppose $x$ and $y$ are positive integers such that $y = mx$ for some integer $m$. Let $H = x\Z$ and $N = y\Z$ be groups under addition.
    \begin{partquestions}{\roman*}
        \item Explain why $N \subseteq H$.
        \item Find a group $G$ such that $H \lhd G$ and $N \lhd G$.
        \item Hence find the order of $H/N$.
    \end{partquestions}
\end{exercise}

\newpage

\section{Problems}
\begin{problem}
    Let $G$ be a group. Prove that $G/G$ is isomorphic to the trivial group.
\end{problem}

\begin{problem}
    Let $R = (\R, +)$. Also let $G = R^2$ and $H = \left\{(r\sqrt2, r\sqrt3) \vert r\in R\right\}$ be groups under component-wise addition. Prove that $G/H \cong R$.
\end{problem}

\begin{problem}\label{problem-subgroup-product-equal-to-subgroup-if-one-is-subgroup-of-another}
    Let $G$ be a group. Let $H$ and $K$ be subgroups of $G$ such that $K \subseteq H$. Prove that $HK = H$.
\end{problem}

\begin{problem}\label{problem-cartesian-product-of-group-by-group-isomorphic-to-group}
    Let $G$ be an abelian group with operation $\ast$. Let $I = \{(g, g^{-1}) \vert g \in G\}$ be a group under component-wise application of $\ast$.
    \begin{partquestions}{\roman*}
        \item Show that $I \cong G$.
        \item Hence prove $G^2/G \cong G$ by considering a suitable homomorphism.
    \end{partquestions}
\end{problem}

\begin{problem}\label{problem-mZ/nZ-isomorphic-to-Zn/m}
    Let $G = m\Z$ and $H = n\Z$ be groups under addition, where $m\vert n$ and $m \neq n$. Let the map $\phi: G \to \Z/({\frac nm}\Z)$ be defined such that
    \[
        \phi(am) = a + \frac nm \Z.
    \]
    Prove that $G/H \cong \Z_{\frac nm}$.
\end{problem}

\section{More Types of Groups}
\subsection*{Exercises}
\begin{questions}
    \item Let $G = \Z_{mn}$ and $H = \{0, n, 2n, \dots, (m-1)n\}$. Clearly $H$ is a subgroup of $G$ of order $m$. By \myref{problem-subgroup-of-cyclic-group-is-cyclic} we know $H$ is normal and cyclic with order $m$ and by \myref{exercise-quotient-group-of-cyclic-group-is-cyclic} we know $G/H$ is cyclic. The order of $G/H$ is $\frac{|G|}{|H|} = \frac{mn}{m} = n$ by Lagrange's theorem (\myref{thrm-lagrange}), meaning that $G/H \cong \Z_n$. Hence, $\Z_{mn}/\Z_m \cong G/H \cong \Z_n$.

    \item Note 0 is the identity in $\Z_n$. By \myref{lemma-order-of-an-element-that-is-equivalent-to-identity} we know that if $12$ is equivalent to the identity in $\Z_n$, then $12 = mn$ for some integer $m$. Since $n > 0$ we restrict $m$ to positive integers. Now $12 = 2^2 \times 3$. Thus the possible values of $n$ are
    \begin{itemize}
        \item $n = 1$ with $m = 12$;
        \item $n = 2$ with $m = 6$;
        \item $n = 3$ with $m = 4$;
        \item $n = 4$ with $m = 3$;
        \item $n = 6$ with $m = 2$; and
        \item $n = 12$ with $m = 1$.
    \end{itemize}

    \item $|10| = \frac{210}{\gcd(10, 210)} = \frac{210}{10} = 21$, $|42| = \frac{210}{\gcd(42, 210)} = \frac{210}{42} = 5$, $|75| = \frac{210}{\gcd(75, 210)} = \frac{210}{15} = 14$, and $|140| = \frac{210}{\gcd(140, 210)} = \frac{210}{70} = 3$.

    \item \begin{partquestions}{\alph*}
        \item Note that $10 = 2 \times 5$. Generators of the group $\Z_{10}$ has to satisfy $\gcd(m, 10) = 1$ by \myref{corollary-element-in-cyclic-group-is-generator-iff-gcd-is-1}. The positive integers that satisfy this requirement (and which are less than 10) are 1, 3, 7, 9. Thus they are the generators of $\Z_{10}$.
        \item Note that 101 is prime. Hence all positive integers from 1 to 100 (inclusive) are generators. (Note that 0 is not a generator of $\Z_{101}$ since 0 is the identity.)
    \end{partquestions}

    \item We show that all subgroups of $\mathrm{Q}$ are, in fact, normal. We consider the first definition of the quaternion group.
    \begin{itemize}
        \item Clearly $\{1\} \lhd \mathrm{Q}$ and $\mathrm{Q} \unlhd \mathrm{Q}$.
        \item The subgroups $\langle i\rangle$, $\langle j\rangle$, and $\langle k\rangle$ have order 4. Therefore, Lagrange's theorem (\myref{thrm-lagrange}) tells us that they have index 2. Hence these subgroups are normal by \myref{problem-subgroup-of-index-2}.
        \item Consider the subgroup $\langle -1 \rangle = \{1, -1\}$. \begin{itemize}
            \item $1\langle -1 \rangle = \langle -1 \rangle1$, since 1 is the identity;
            \item $-1\langle -1 \rangle = \{1, -1\} = \langle -1 \rangle(-1)$;
            \item $i\langle -1 \rangle = \{-i, i\} = \langle -1 \rangle i$;
            \item $-i\langle -1 \rangle = \{i, -i\} = \langle -1 \rangle (-i)$;
            \item $j\langle -1 \rangle = \{-j, j\} = \langle -1 \rangle j$;
            \item $-j\langle -1 \rangle = \{j, -j\} = \langle -1 \rangle (-j)$;
            \item $k\langle -1 \rangle = \{-k, k\} = \langle -1 \rangle k$; and
            \item $-k\langle -1 \rangle = \{k, -k\} = \langle -1 \rangle (-k)$.
        \end{itemize}
        Thus $\langle -1 \rangle$ is normal.
    \end{itemize}
    Hence all subgroups of $\mathrm{Q}$ are normal.

    \item $\begin{pmatrix}2&6\end{pmatrix} = \begin{pmatrix}2&3\end{pmatrix}\begin{pmatrix}3&4\end{pmatrix}\begin{pmatrix}4&5\end{pmatrix}\begin{pmatrix}5&6\end{pmatrix}\begin{pmatrix}4&5\end{pmatrix}\begin{pmatrix}3&4\end{pmatrix}\begin{pmatrix}2&3\end{pmatrix}$.

    \item Note that $\begin{pmatrix}1&3&2&5&4\end{pmatrix} = \begin{pmatrix}1&4\end{pmatrix}\begin{pmatrix}1&5\end{pmatrix}\begin{pmatrix}1&2\end{pmatrix}\begin{pmatrix}1&3\end{pmatrix}$. \myref{thrm-parity-of-permutation} tells us that $\begin{pmatrix}1&3&2&5&4\end{pmatrix}$ is even and thus has a sign of $+1$.

    \item Note that $\An{3}$ has order $\frac{3!}{2} = 3$ so we should expect 3 permutations. Clearly the identity is one such permutation. Looking at \myref{example-symmetric-group-of-degree-3} we can find two more, namely $\begin{pmatrix}1&2&3\end{pmatrix}$ and $\begin{pmatrix}1&3&2\end{pmatrix}$.
    
    For $\An{4}$, note that it has order $\frac{4!}{2} = 12$ so we expect 12 permutations. Again the identity is one of them. Like $\An{3}$ we now find the 3-cycles in $\An{4}$, which are $\begin{pmatrix}1&2&3\end{pmatrix}$, $\begin{pmatrix}1&2&4\end{pmatrix}$, $\begin{pmatrix}1&3&2\end{pmatrix}$, $\begin{pmatrix}1&3&4\end{pmatrix}$, \linebreak $\begin{pmatrix}1&4&2\end{pmatrix}$, $\begin{pmatrix}1&4&3\end{pmatrix}$, $\begin{pmatrix}2&3&4\end{pmatrix}$, and $\begin{pmatrix}2&4&3\end{pmatrix}$. So there are 3 more permutations unaccounted, which are permutations of products of 2-cycles: $\begin{pmatrix}1&2\end{pmatrix}\begin{pmatrix}3&4\end{pmatrix}$, $\begin{pmatrix}1&3\end{pmatrix}\begin{pmatrix}2&4\end{pmatrix}$, and $\begin{pmatrix}1&4\end{pmatrix}\begin{pmatrix}2&3\end{pmatrix}$.

    \item $\Un{10} = \{1, 3, 7, 9\}$.

    \item By a corollary of Lagrange's theorem (\myref{corollary-order-of-group-multiple-of-order-of-element}), the order of $a$ dives the order of the group $\Un{n}$. Now since $|\Un{n}| = \totient(n)$, thus the order of $a$ divides $\totient(n)$.

    \item $\begin{pmatrix}2&1&2\\1&0&1\\2&1&2\end{pmatrix}$

    \item We already proved that $\Inn{G} \leq \Aut{G}$ so we only need to prove normality.

    Let $\phi \in \Aut{G}$ and $\iota_g \in \Inn{G}$. For brevity let $f = \phi\iota_g\phi^{-1}$. We note that $f \in \Aut{G}$; we need to prove that $f \in \Inn{G}$.

    Suppose $x \in G$. Since $\phi$ is an isomorphism, there exists $w \in G$ such that $x = \phi(w)$, i.e. $w = \phi^{-1}(x)$. So
    \begin{align*}
        f(x) &= \left(\phi\iota_g\phi^{-1}\right)(x)\\
        &= \phi(\iota_g(\phi^{-1}(x)))\\
        &= \phi(\iota_g(w))\\
        &= \phi(gwg^{-1})\\
        &= \phi(g)\phi(w)\phi(g^{-1})\\
        &= \phi(g)x\left(\phi(g)\right)^{-1}
    \end{align*}
    which shows that $f \in \Inn{G}$. Hence, $\Inn{G} \unlhd \Aut{G}$.
\end{questions}

\subsection*{Problems}
\begin{questions}
    \item We note that the two questions are equivalent to finding the orders of 3774 and 1870 in the group $\Z_{10101}$. We note that
    \begin{align*}
        1870 &= 2 \times 5 \times 11 \times 17,\\
        3774 &= 2 \times 3 \times 17 \times 37, \text{ and}\\
        10101 &= 3 \times 7 \times 13 \times 37.
    \end{align*}
    Therefore, $\gcd(1870, 10101) = 1$ and $\gcd(3774, 10101) = 3 \times 37 = 111$. Hence $|1870| = 10101$ and $|3774| = \frac{10101}{111} = 91$. Therefore, $a = 10101$ and $b = 91$.

    \item We claim that $\An{n}$ is non-abelian for any $n > 3$. Note that both $\pi = \begin{pmatrix}1 & 2 & 3\end{pmatrix}$ and $\sigma = \begin{pmatrix}2 & 3 & 4\end{pmatrix}$ are even permutations, and hence are in $\An{n}$ for any $n > 3$. We note
    \begin{itemize}
        \item $\pi\sigma = \begin{pmatrix}1 & 2 & 3\end{pmatrix}\begin{pmatrix}2 & 3 & 4\end{pmatrix} = \begin{pmatrix}1 & 2\end{pmatrix}\begin{pmatrix}3 & 4\end{pmatrix}$; and
        \item $\sigma\pi = \begin{pmatrix}2 & 3 & 4\end{pmatrix}\begin{pmatrix}1 & 2 & 3\end{pmatrix} = \begin{pmatrix}1 & 3\end{pmatrix}\begin{pmatrix}2 & 4\end{pmatrix}$.
    \end{itemize}
    So $\pi\sigma \neq \sigma\pi$. Thus $\An{n}$ is non-abelian for any $n > 3$.

    We note that
    \begin{itemize}
        \item $\An{2}$ has order 1 so $\An{2}$ is the trivial group, which is abelian (and cyclic); and
        \item $\An{3}$ has order 3 so $\An{3}$ is cyclic and thus abelian.
    \end{itemize}
    Thus the largest integer $n$ for which $\An{n}$ is abelian is $n = 3$. Furthermore $\An{k}$ is cyclic if $k = 2$ or $k = 3$.

    \item We first note that
    \[
        \totient(2p^k) = 2p^k\left(1-\frac12\right)\left(1-\frac1p\right) = p^k\left(1-\frac1p\right) = \totient(p^k).
    \]
    Now we are given that $r$ is an odd primitive root of $p^k$. Since $r \in \Un{p^k}$, thus $\gcd(r, 2p^k) = 1$ because $\gcd(r, p^k) = 1$. Now as $r$ is odd, thus $r \in \Un{2p^k}$. Let $n$ be the order of $r$ in $\Un{2p^k}$. Then by \myref{exercise-order-of-a-divides-phi-a} we know $n$ divides $\totient(2p^k)$. At the same time, because $r$ is a generator in $\Un{p^k} \cong \Z_{\phi(p^k)}$, so $\totient(p^k) = \totient(2p^k)$ divides $n$ by \myref{lemma-order-of-an-element-that-is-equivalent-to-identity}. Since $n$ divides $\totient(2p^k)$ and $\totient(2p^k)$ divides $n$ simultaneously, therefore $n = \totient(2p^k) = |\Un{2p^k}|$ which means that $r$ is a primitive root modulo $2p^k$.

    \item \begin{partquestions}{\roman*}
        \item The forward direction is clearly true since if $f_1 = f_2$, then $f_1(x) = f_2(x)$ for all $x \in G$, including $g \in G$. For the reverse direction, assume $f_1(g) = f_2(g)$. Note that
        \[
            f_1(g^k) = (f_1(g))^k = (f_2(g))^k = f_2(g^k)
        \]
        for any integer $k$. Since $g$ is a generator, thus we have $f_1(x) = f_2(x)$ for all $x \in G$, meaning $f_1 = f_2$.

        \item We note $f(g) \in G$. Since $g$ is a generator, hence $f(g) = g^k$ for some integer $k$. Hence any homomorphism from $G$ to $G$ is of the form $f(g) = g^{m_f}$ where $0 \leq m_f \leq n-1$, which means $m_f \in \Z_n$.

        \item Suppose the map $f_2: G \to G$ is another homomorphism where $f_2(g) = g^{m_f}$. Then we see
        \[
            f(g) = g^{m_f} = f_2(g)
        \]
        which means $f = f_2$ by \textbf{(i)}. Hence $m_f$ is unique to $f$.

        \item Consider $f_1(f_2(g))$. On one hand,
        \[
            f_1(f_2(g)) = f_1(g^{m_{f_2}}) = (f_1(g))^{m_{f_2}} = g^{m_{f_1}m_{f_2}},
        \]
        while on the other,
        \[
            f_1(f_2(g)) = (f_1 \circ f_2)(g) = g^{m_{f_1\circ f_2}}
        \]
        by definition of $m_f$ as introduced in \textbf{(ii)}. Therefore $m_{f_1\circ f_2} \equiv m_{f_1}m_{f_2} \pmod n$. In other words, $m_{f_1\circ f_2} = m_{f_1} \otimes_n m_{f_2}$.

        \item We prove the forward direction first by assuming that the map $f$ is an automorphism. Hence $f$ is surjective, meaning that there exists $a \in G$ such that $f(a) = g$. Since $a \in G$ thus $a = g^k$ for some $k \in \Z_n$ (we will show $k \in \Un{n}$ later). Observe
        \[
            g = f(a) = f(g^k) = (f(g))^k = g^{m_fk}
        \]
        which means $m_fk \equiv 1 \pmod n$. By \myref{prop-multiplicative-inverse-exists-iff-coprime}, this means that we have $\gcd(m_f, n) = 1$ and $\gcd(k, n) = 1$. Therefore, $m_f$ and $k$ are in $\Un{n}$. Hence $k$ is the multiplicative inverse of $m_f$.

        We now prove the reverse direction. Assume $m_f$ has a multiplicative inverse (say $k$), meaning $m_fk \equiv 1 \pmod n$. As above this means that both $m_f$ and $k$ are in $\Un{n}$. We show that $f$ is a bijection.
        \begin{itemize}
            \item \textbf{Injective}: Suppose $x, y \in G$ such that $f(x) = f(y)$. Since $g$ is a generator we may take $x = g^p$ and $y = g^q$ for some integers $p$ and $q$. Hence we have $g^{m_fp} = g^{m_fq}$. Then
            \[
                \left(g^{m_fp}\right)^k = g^{km_fp} = \left(g^{km_f}\right)^p = g^p
            \]
            and $\left(g^{m_fq}\right)^k = g^q$. Hence this implies $g^p = g^q$ which means $x = y$.
            \item \textbf{Surjective}: Suppose $x \in G$. Since $g$ is a generator we may write $x = g^p$ for some integer $p$. Then $f(g^{kp}) = g^{m_fkp} = g^p = x$.
        \end{itemize}
        Also $f$ is given to be a homomorphism. Hence $f$ is an isomorphism. Since $f: G \to G$, it is thus an automorphism.

        \item We prove that $\phi$ is an isomorphism.
        \begin{itemize}
            \item \textbf{Homomorphism}: Let $f_1, f_2 \in \Aut{G}$. Then
            \begin{align*}
                \phi(f_1\circ f_2) &= m_{f_1\circ f_2} & (\text{definition of } m_f \text{ in }\textbf{(ii)})\\
                &= m_{f_1} \otimes_n m_{f_2} & (\text{by \textbf{(iv)}})\\
                &= \phi(f_1)\otimes_n\phi(f_2),
            \end{align*}
            which means $\phi$ is a homomorphism.

            \item \textbf{Injective}: Suppose we have $f_1, f_2 \in \Aut{G}$ such that $\phi(f_1) = \phi(f_2)$. Thus $m_{f_1} = m_{f_2}$ by definition of $\phi$. However, we know that the value of $m$ uniquely defines a homomorphism from $G$ to $G$ from \textbf{(iii)}. Hence $f_1 = f_2$, which shows that $\phi$ is injective.

            \item \textbf{Surjective}: Suppose $r \in \Un{n}$. Define the map $f: G \to G$ where $f(g) = g^r$. Since $r \in \Un{n}$ it has a multiplicative inverse, which means that $f$ is an automorphism by \textbf{(v)}. Clearly $\phi(f) = r$, so $r$ has a pre-image. So $\phi$ is surjective.
        \end{itemize}
        Hence $\phi$ is an isomorphism, meaning $\Aut{G} \cong \Un{n}$.
    \end{partquestions}
\end{questions}

\section{Group Actions}
\begin{questions}
    \item We prove the two group action axioms.
    \begin{itemize}
        \item \textbf{Identity}: $\alpha(e, x) = exe^{-1} = x$.
        \item \textbf{Compatibility}: Note
        \begin{align*}
            \alpha(g, \alpha(h, x)) &= \alpha(g, hxh^{-1})\\
            &= gh x h^{-1}g^{-1}\\
            &= (gh)x(gh)^{-1}\\
            &= \alpha(gh, x).
        \end{align*}
    \end{itemize}
    Therefore $\alpha$ is a group action of $G$ on $G$.

    \item Recall there are 6 elements in $\Sn{3}$: $\id$, $\begin{pmatrix}1 & 2 & 3\end{pmatrix}$, $\begin{pmatrix}1 & 3 & 2\end{pmatrix}$, $\begin{pmatrix}1 & 2\end{pmatrix}$, $\begin{pmatrix}1 & 3\end{pmatrix}$, and $\begin{pmatrix}2 & 3\end{pmatrix}$. Clearly the identity has all elements of $X$ as fixed points. It is also clear that $\begin{pmatrix}1 & 2 & 3\end{pmatrix}$ and $\begin{pmatrix}1 & 3 & 2\end{pmatrix}$ have no fixed points since they permute all elements. For the rest, the fixed points are the missing element from the cycle notation, i.e. $\begin{pmatrix}1 & 2\end{pmatrix}$ has fixed point 3, $\begin{pmatrix}1 & 3\end{pmatrix}$ has fixed point 2, and $\begin{pmatrix}2 & 3\end{pmatrix}$ has fixed point 1.

    \item For 1, it is $\{\id, \begin{pmatrix}2 & 3\end{pmatrix}\}$. For 2, it is $\{\id, \begin{pmatrix}1 & 3\end{pmatrix}\}$. For 3, it is $\{\id, \begin{pmatrix}1 & 2\end{pmatrix}\}$.

    \item We work from the statement forwards. Note that each of these statements are ``if and only if'' statements.
    \begin{align*}
	    g \cdot x = h \cdot x &\iff g^{-1} \cdot (g \cdot x) = g^{-1} \cdot (h \cdot x)\\
	    &\iff (g^{-1}g) \cdot x = (g^{-1}h) \cdot x\\
	    &\iff e \cdot x = (g^{-1}h) \cdot x\\
	    &\iff x = (g^{-1}h) \cdot x\\
	    &\iff (g^{-1}h) \cdot x = x\\
	    &\iff g^{-1}h \in \Stab{G}{x}
	\end{align*}

	\item \begin{partquestions}{\alph*}
		\item An orbit takes the form $\Orb{G}{x}$. Clearly $e \cdot x = x$ so $x \in \Orb{G}{x}$ and thus $\Orb{G}{x}$ is non-empty.
	    \item Let $x \in X$. Since $e \cdot x = x$, so $x \in \Orb{G}{x}$.
	    \item Suppose $x \in \Orb{G}{x_1} \cap \Orb{G}{x_2}$ (as their intersection is non-empty). Then there exists $g_1, g_2 \in G$ such that $g_1\cdot x_1 = x = g_2\cdot x_2$. Thus,
	    \begin{align*}
	        x_1 &= e \cdot x_1\\
	        &= (g_1^{-1}g_1)\cdot x_1\\
	        &= g_1^{-1} \cdot (g_1 \cdot x_1)\\
	        &= g_1^{-1} \cdot (g_2 \cdot x_2)\\
	        &= (g_1^{-1}g_2) \cdot x_2.
	    \end{align*}
	    Now suppose $y \in \Orb{G}{x_1}$. Then $y = g\cdot x_1$ for some $g \in G$. Hence,
	    \begin{align*}
	        y &= g\cdot x_1 \\
	        &= g \cdot \left((g_1^{-1}g_2) \cdot x_2\right)\\
	        &= (\underbrace{gg_1^{-1}g_2}_{\text{In } G})\cdot x_2\\
	        &\in \Orb{G}{x_2}
	    \end{align*}
	    which means any element in $\Orb{G}{x_1}$ is also in $\Orb{G}{x_2}$. Hence, $\Orb{G}{x_1}$ is a subset of $\Orb{G}{x_2}$. A similar argument can be used to show that $\Orb{G}{x_2}$ is a subset of $\Orb{G}{x_1}$. Hence $\Orb{G}{x_1} = \Orb{G}{x_2}$.
	\end{partquestions}

	\item We prove the forward direction first: suppose the action is transitive. Then there exists $x \in X$ such that $\Orb{G}{x} = X$. Now consider any other element $y \in X$. Since the action is transitive, this means that there exists a $\hat{g} \in G$ such that $\hat{g} \cdot x = y$. Note that $\Orb{G}{y} = \Orb{G}{\hat{g} \cdot x}$, and that $\Orb{G}{x} = \{g \cdot x \vert g \in G\}$. Hence,
	\[
        \Orb{G}{\hat{g} \cdot x} = \{g\cdot (\hat{g} \cdot x) \vert g \in G\} = \{(g\hat{g}) \cdot x \vert g \in G\}.
	\]
	Since $G$ is a group, $g\hat{g} \in G$. In particular, we may pick $g = g'\hat{g}^{-1}$ to obtain any arbitrary element $g' \in G$. Thus, this means that
	\[
        	\{(g\hat{g}) \cdot x \vert g \in G\} = \{g' \cdot x \vert g' \in G \} = \Orb{G}{x} = X.
	\]
	Hence, for any element $y \in X$, $\Orb{G}{y} = \Orb{G}{g \cdot x} = X$.

	The reverse direction is trivial: suppose $\Orb{G}{x} = X$ for all $x \in X$. Then certainly there exists an element $x \in X$ such that $\Orb{G}{x} = X$, meaning that the group action is transitive.

	\item \begin{partquestions}{\alph*}
	    \item Consider $x = n$. The orbit of $n$ is all of $X$. Consider the permutation $\sigma = \begin{pmatrix}k & n\end{pmatrix}$ where $1 \leq k \leq n$. Clearly $\sigma \in \Sn{n}$. Note that $\sigma \cdot n = \sigma(n) = k$. Thus, $\Orb{G}{n} = X$, meaning that the group action ``$\cdot$'' given by $g \cdot x \mapsto g(x)$ is transitive.
	    \item Note that $|X| = n$ and $|\Sn{n}| = n!$. By Orbit-Stabilizer theorem (\myref{thrm-orbit-stabilizer}), the stabilizer of $x$ by $G$ must have order $\frac{n!}{n} = (n-1)!$.
	\end{partquestions}

	\item By the Orbit-Stabilizer theorem (\myref{thrm-orbit-stabilizer}),
	\[
        |\Orb{G}{x}| = \frac{|G|}{|\Stab{G}{x}|} = [G : \Stab{G}{x}]	.
	\]
	Under the group action of conjugation, $\Orb{G}{x} = \Cl{x}$ and $\Stab{G}{x} = \Centralizer{G}{x}$. Hence, $|\Cl{x}| = [G : \Centralizer{G}{x}]$ as required.

	\item \begin{partquestions}{\alph*}
	    \item One sees that $\Z{D_3} = \{e\}$ based on the group table of $D_3$.
	    \item Recall that every element in $D_3$ can be expressed in the form $r^as^b$ where $a \in \{0, 1, 2\}$ and $b \in \{0, 1\}$. One finds that $\Cl{r} = \{r, r^2\}$ and $\Cl{s} = \{s, rs, rs^2\}$.
	    \item The class equation is $6 = 1 + 2 + 3$.
	\end{partquestions}

	\item By Cauchy's Theorem (\myref{thrm-cauchy}) there exists an element (say $x$) with order $p$. Consider $H = \langle x \rangle$. Note that $|H| = p$ and $H \leq G$. Hence we found a subgroup of $G$ of order $p$.
\end{questions}

\section{Sylow Theorems}
\subsection*{Exercises}
\begin{questions}
    \item We note that $12 = 2^2 \times 3$. Thus a Sylow 2-subgroup must have order 4. Clearly $|3| = 4$ so $\langle 3 \rangle = \{0, 3, 6, 9\}$ is the Sylow 2-subgroup of $\Z_{12}$.

    \item Recall that $|\Sn{5}| = 120 = 2^3 \times 3 \times 5$. By a corollary of the First Sylow Theorem (\myref{corollary-sylow-p-subgroup-exists}), $\Syl{p}{G} \neq \emptyset$ if $p$ is 2, 3, or 5.

    \item We prove this by constructing the map $\phi: H \to gHg^{-1}$ where $h \mapsto ghg^{-1}$. We note that $\phi$ is an isomorphism.
    \begin{itemize}
        \item \textbf{Homomorphism}: Let $x, y \in H$. Then
        \[
            \phi(xy) = g(xy)g^{-1} = (gxg^{-1})(gyg^{-1}) = \phi(x)\phi(y)
        \]
        which clearly means that $\phi$ is a homomorphism.
        
        \item \textbf{Injective}: Suppose $x, y \in H$ such that $\phi(x) = \phi(y)$. Then $gxg^{-1} = gyg^{-1}$ which quickly implies $x = y$.
        
        \item \textbf{Surjective}: Suppose $ghg^{-1} \in gHg^{-1}$. Clearly we have $\phi(h) = ghg^{-1}$, so any element in $gHg^{-1}$ has a pre-image inside $H$.
    \end{itemize}
    Hence $H \cong gHg^{-1}$.

    \item By \myref{prop-order-of-conjugate-element-equals-order-of-element} we know that $|xyx^{-1}| = |y|$ for all $x, y \in G$. Substituting $x = g$, and $y = hg$ yields
    \[
        |xyx^{-1}| = |g(hg)g^{-1}| = |gh| \text{ and } |y| = |hg|
    \]
    so the result follows.

    \item Clearly $e \in \N{G}{S}$ since $eSe^{-1} = S$. Consider $x, y \in \N{G}{S}$, meaning that $xSx^{-1} = S$ and $ySy^{-1} = S$. Note that $y^{-1} \in \N{G}{S}$ since
    \begin{align*}
        y^{-1}S\left(y^{-1}\right)^{-1} &= y^{-1}Sy\\
        &= y^{-1}\left(ySy^{-1}\right)y & (\text{since } y \in \N{G}{S})\\
        &= (y^{-1}y)S(y^{-1}y)\\
        &= S.
    \end{align*}
    Therefore
    \begin{align*}
        \left(xy^{-1}\right)S\left(xy^{-1}\right)^{-1} &= \left(xy^{-1}\right)S\left(yx^{-1}\right)\\
        &= x\left(y^{-1}Sy\right)x^{-1}\\
        &= xSx^{-1} & (\text{since } y^{-1} \in \N{G}{S})\\
        &= S & (\text{since } x \in \N{G}{S})
    \end{align*}
    which means that $xy^{-1} \in \N{G}{S}$. Hence, by the subgroup test, we have $\N{G}{S} \leq G$.

    \item By the Second Sylow Theorem (\myref{thrm-sylow-2}), we know that $gHg^{-1} = K$. Since $H \cong gHg^{-1}$ by \myref{exercise-conjugate-subgroup-isomorphic-to-subgroup} thus $H \cong gHg^{-1} = K$ as required.

    \item We note $784 = 2^4 \times 7^2$, so $m = 16$, $p = 7$, and $k = 2$. By the Third Sylow Theorem (\myref{thrm-sylow-3}), we know that
    \begin{itemize}
        \item $n_7 = [G : \N{G}{P}] = \frac{|G|}{|\N{G}{P}|}$;
        \item $n_7 \mid 16$, which implies $n_7 \in \{1, 2, 4, 8, 16\}$; and
        \item $n_7 \equiv 1 \pmod 7$, which implies $n_7 \in \{1, 8, 15, 22, \dots\}$.
    \end{itemize}
    Hence $n_7 = 1$ or $n_7 = 8$. But since $P$ is not a normal subgroup of $G$, by \myref{corollary-sylow-subgroup-is-normal-if-it-is-unique}, $P$ cannot be the only Sylow 7-subgroup, meaning $n_7 \neq 1$. Hence $n_7 = 8$, so
    \[
        8 = n_7 = \frac{|G|}{|\N{G}{P}|} = \frac{784}{|\N{G}{P}|}
    \]
    which means that $|\N{G}{P}| = 98$.

    \item Note $130 = 2 \times 5 \times 13$. Consider the number of Sylow 13-subgroups, $n_{13}$. The Third Sylow Theorem (\myref{thrm-sylow-3}) tells us that
    \begin{itemize}
        \item $n_{13} \mid 2 \times 5 = 10$, so $n_{13} \in \{1, 2, 5, 10\}$, and
        \item $n_{13} \equiv 1 \pmod{13}$ so $n_{13} \in \{1, 14, 27, \dots\}$.
    \end{itemize}
    Hence $n_{13} = 1$. But by \myref{corollary-sylow-subgroup-is-normal-if-it-is-unique} this means that the only Sylow 13-subgroup is normal. Hence a group of order 130 is non-simple.
\end{questions}

\subsection*{Problems}
\begin{questions}
    \item Note $200 = 2^3 \times 5^2$. Note that for $p = 5$ we have $m = 8$ and the factors of 8 are 1, 2, 4, and 8. Furthermore by the Third Sylow Theorem (\myref{thrm-sylow-3}) we must have $n_5 \equiv 1 \pmod 5$. Hence $n_5 = 1$. By a corollary of the Second Sylow Theorem (\myref{corollary-sylow-subgroup-is-normal-if-it-is-unique}) this means that the only Sylow 5-subgroup is normal.

    \item Note $33 = 3 \times 11$,
    \begin{itemize}
        \item when $p = 3$ we have $m = 11$ and the factors of 11 are 1 and 11; and
        \item when $p = 11$ we have $m = 3$ and the factors of 3 are 1 and 3.
    \end{itemize}
    The Third Sylow Theorem (\myref{thrm-sylow-3}) tells us that $n_p \equiv 1 \pmod p$. Hence we must have $n_3 = n_{11} = 1$. A corollary of the Second Sylow Theorem (\myref{corollary-sylow-subgroup-is-normal-if-it-is-unique}) tells us that the only Sylow 3-subgroup and Sylow 11-subgroup are normal.

    \item For brevity let $q = 2^p - 1$, and we are given that $q$ is a prime. By the Third Sylow Theorem (\myref{thrm-sylow-3}), $n_q \mid 2^{p-1}$ and $n_q \equiv 1 \pmod p$. The factors of $2^{p-1}$ are $1, 2, 4, 8, \dots, 2^{p-1}$. We note $2^{p-1} < 2^p - 1 = q$ for any prime $p$ since
    \[
        2^{p-1} + 1 < 2^{p-1} + 2^{p-1} = 2(2^{p-1}) = 2^p
    \]
    which result immediately follows by subtracting 1 on both sides. Hence, the only possible value that satisfies both conditions is $n_q = 1$. By a corollary of the Second Sylow Theorem (\myref{corollary-sylow-subgroup-is-normal-if-it-is-unique}) this means that the only Sylow $q$-subgroup is normal, hence showing that a group with an even perfect number order is non-simple.

    \item \begin{partquestions}{\roman*}
        \item The divisors of $p$ are 1 and $p$ itself. By the Third Sylow Theorem (\myref{thrm-sylow-3}), $n_q$ divides $p$ and $n_q \equiv 1 \pmod q$. Since $p < q$ hence $p \not\equiv 1 \pmod q$ meaning that $n_q = 1$. By a corollary of the Second Sylow Theorem (\myref{corollary-sylow-subgroup-is-normal-if-it-is-unique}) the only Sylow $q$-subgroup is normal.

        \item The divisors of $q$ are 1 and $q$ itself. By the Third Sylow Theorem (\myref{thrm-sylow-3}), $n_p$ divides $q$ and $n_p \equiv 1 \pmod p$. Since $q \not\equiv 1 \pmod p$ by assumption, we must have $n_p = 1$.

        Recall that the order of an element in a group of order $pq$ must divide $pq$ (\myref{corollary-order-of-group-multiple-of-order-of-element}). Hence the possible orders of an element in such a group are 1, $p$, $q$, or $pq$.
        \begin{itemize}
            \item There is only one element of order 1, the identity.
            \item There are $p - 1$ elements of order $p$, all belonging in the single Sylow $p$-subgroup. Note that we subtract 1 because one element in the Sylow $p$-subgroup is the identity.
            \item There are $q - 1$ elements of order $q$, all in the single Sylow $q$-subgroup.
        \end{itemize}
        Hence, since the total number of elements in a group of order $pq$ is $pq$, the number of elements of order $pq$ is
        \begin{align*}
            pq - \left((p-1)+(q-1)+1\right) &= pq - (p+q - 1)\\
            &= pq - p - q + 1\\
            &> 2q - 2 - q + 1\\
            &= 2q - q - 1\\
            &= q - 1\\
            &> 0
        \end{align*}
        which means that there is at least one element of order $pq$. By \myref{thrm-cyclic-group-has-element-with-same-order} this means that such a group is cyclic.
    \end{partquestions}

    \item \begin{partquestions}{\roman*}
        \item Let $P$ be a Sylow $p$-subgroup of $N$. Lagrange's Theorem (\myref{thrm-lagrange}) tells us that $|G| = [G:N]|N|$. Since $p$ does not divide $[G:N]$ we must have $|N| = p^ka$ where $a$ divides $m$. Hence $|P| = p^k$ as $P$ is a Sylow $p$-subgroup of $N$. Since $P$ has order $p^k$ and $P \leq N \leq G$, thus $P$ is also a Sylow $p$-subgroup of $G$.
        \item Let $Q$ be a Sylow $p$-subgroup of $G$. The Second Sylow Theorem (\myref{thrm-sylow-2}) tells us there exist  a $g \in G$ such that $Q = gPg^{-1}$. Recall by definition of normality that $gNg^{-1} = N$ for any $g \in G$. Note also that $P \leq N$. Hence,
        \[
            Q = gPg^{-1} \leq gNg^{-1} = N
        \]
        which means that $Q$ is also a Sylow $p$-subgroup of $N$.
    \end{partquestions}

    \item We note $3325 = 5^2 \times 7 \times 19$. Let the group of order 3325 be $G$. We know that
    \begin{itemize}
        \item for $p = 5$ we have $m = 7 \times 19 = 133$ and so the possible divisors of $m$ are $\{1, 7, 19, 133\}$;
        \item for $p = 7$ we have $m = 5^2 \times 19 = 475$ and so the possible divisors of $m$ are $\{1, 5, 19, 25, 95, 475\}$; and
        \item for $p = 19$ we have $m = 5^2 \times 7 = 175$ and so the possible divisors of $m$ are $\{1, 5, 7, 25, 35, 175\}$.
    \end{itemize}
    The Third Sylow Theorem (\myref{thrm-sylow-3}) tells us that $n_p \equiv 1 \pmod p$. Thus $n_5 = n_7 = n_{19} = 1$. Let $P$, $Q$, and $R$ be the Sylow 5-subgroup, the Sylow 7-subgroup, and the Sylow 19-subgroup respectively. We note that $P$, $Q$, and $R$ are all normal subgroups of $G$ by \myref{corollary-sylow-subgroup-is-normal-if-it-is-unique}.

    Denote the group $QR$ by $H$. Since $Q$ and $R$ are of prime order, their intersection is the identity (\myref{problem-intersection-of-coprime-subgroups}). Furthermore, as $Q$ and $R$ are normal subgroups of $G$, thus they commute by \myref{problem-intersection-of-coprime-subgroups}. Therefore $H$ is the internal direct product of $Q$ and $R$, meaning $H \cong Q \times R$ by \myref{thrm-direct-product-equivalence}. Hence $|H| = |Q||R| = 7 \times 19 = 133$. Now because as $Q$ and $R$ are of prime order, thus $Q$ and $R$ are abelian and so is $H$. Hence $H$ is an abelian group of order 133.

    Now consider the group $PH$. Since 5 and 133 are coprime, thus $P \cap H = \{e\}$. In addition, since $P \lhd G$ thus $PH \leq G$ by Diamond Isomorphism Theorem (\myref{thrm-isomorphism-2}), statement 3. Also,
    \[
        |PH| = \frac{|P||H|}{|P \cap H|} = |P||H| = 5^2 \times 133 = 3325 = |G|
    \]
    which means that $G = PH$. Since $P \lhd G$, thus $ph = hp$ for any element $h \in H$, meaning elements in $P$ and $H$ commute. Hence, $G$ is the internal direct product of $P$ and $H$, meaning $G \cong P \times H$. As the external direct product of two abelian groups is also abelian (\myref{problem-external-direct-product-of-abelian-groups-is-abelian}) thus $G$ is abelian.

    \item Let $P$ be a Sylow $p$-subgroup of $G$. We note that $|G/P| = \frac{p^km}{p^k} = m$. Let $G$ act on the set of cosets $G/P$ by left multiplication, meaning $g\cdot (xP) = (gx)P$. We know by \myref{thrm-group-action-definition-equivalence} that this induces a homomorphism $\phi: G \to \Sn{m}$ where $\phi(g) = \sigma_g$ such that $\sigma_g(xP) = g\cdot (xP) = (gx)P$. By \myref{example-using-kernel-to-show-non-simple}, $\ker\phi = \bigcap_{x \in G}xPx^{-1}$.

    We note $\ker\phi \neq \{e\}$ since otherwise it would imply that $\phi$ is injective (\myref{exercise-trivial-kernel-means-injective}), which is impossible as that would mean $p^km = |G| \leq |\Sn{m}| = m!$ which is a contradiction. Also $\ker\phi \neq G$ as otherwise
    \[
        p^km = |G| = |\ker\phi| = \left|\bigcap_{x \in G} xPx^{-1}\right| \leq |xPx^{-1}| = |P| = p^k,
    \]
    which would mean $m = 1$, a contradiction. Hence $\ker\phi$ is a non-trivial proper subgroup of $G$. We note that $\ker\phi \lhd G$, so we have found a non-trivial proper normal subgroup of $G$, meaning that $G$ is non-simple.

    \item Let $G$ be a group of order 30. Note $30 = 2 \times 3 \times 5$, and consider $n_5$. The Third Sylow Theorem (\myref{thrm-sylow-3}) tells us that
    \begin{itemize}
        \item $6 \vert n_5$, so $n_5 \in \{1, 2, 3, 6\}$; and
        \item $n_5 \equiv 1 \pmod 5$, so $n_5 \in \{1, 6, 11, 16, \dots\}$.
    \end{itemize}
    Hence $n_5$ is 1 or 6. Seeking a contradiction, assume $n_5 = 6$, and let $P_5$ be a Sylow 5-subgroup.

    Since $|P_5| = 5$, which is prime, each non-identity element of $P_5$ is a generator. Hence, no two Sylow 5-subgroups can share any non-identity elements (otherwise they will be the same group), thereby meaning any two Sylow 5-subgroups intersect in the identity only. Thus there exists $6(5-1) = 24$ elements of order 5, meaning there must be 6 elements of order not equal to 5.

    Now consider $n_3$. Note by the Third Sylow Theorem again,
    \begin{itemize}
        \item $10 \vert n_3$, so $n_3 \in \{1, 2, 5, 10\}$; and
        \item $n_3 \equiv 1 \pmod 3$, so $n_3 \in \{1, 4, 7, 10, 13, \dots\}$.
    \end{itemize}
    Thus $n_3$ is 1 or 10. Now if $n_3 = 10$ then there must be $10(3-1) = 20$ elements of order 3, a contradiction to the fact there exists only 6 elements with order not 5. Hence $n_3 = 1$, meaning the only Sylow 3-subgroup (call it $P_3$) is normal in $G$.

    As $P_5 \leq G$ and $P_3 \lhd G$, by the Diamond Isomorphism Theorem (\myref{thrm-isomorphism-2}), statement 3, we have $P_5P_3 \leq G$. Note $P_5 \cap P_3 = \{e\}$ by \myref{problem-intersection-of-coprime-subgroups}. So \myref{exercise-order-of-subgroup-product} tells us
    \[
        |P_5P_3| = \frac{|P_5||P_3|}{|P_5 \cap P_3|} = \frac{5\times3}{1} = 15.
    \]
    One sees that $[G:P_5P_3] = \frac{30}{15} = 2$, so $P_5P_3 \lhd G$ by \myref{problem-subgroup-of-index-2}.

    Now \myref{problem-group-of-order-pq-has-normal-subgroup-of-order-q} tells us there exists a unique $H \lhd P_5P_3$ with $|H| = 5$. But since $P_5P_3 \lhd G$, \myref{problem-normal-subgroup-of-G-contains-all-sylow-p-subgroups} tells us that $P_5P_3$ contains all Sylow 5-subgroups of $G$, meaning $G$ has only 1 Sylow 5-subgroup, i.e. $n_5 = 1$, a contradiction to our assumption that $n_5 = 6$.

    Hence $n_5 = 1$. Therefore the unique Sylow 5-subgroup is a normal subgroup of $G$ by \myref{corollary-sylow-subgroup-is-normal-if-it-is-unique}.

    \item We prove that $G$ has a normal subgroup of order $p$, $q$, or $r$. By \myref{corollary-sylow-subgroup-is-normal-if-it-is-unique}, subgroups of order $p$, $q$, or $r$ are normal if they are unique. By way of contradiction, assume that they are not unique, meaning $n_p, n_q, n_r > 1$.

    By the Third Sylow Theorem (\myref{thrm-sylow-3}), $n_r \equiv 1 \pmod r$ and $n_r \mid pq$. The divisors of $pq$ are 1, $p$, $q$, and $pq$. We note that since both $p$ and $q$ are less than $r$, thus $p \not\equiv 1 \pmod r$ and $q \not\equiv 1 \pmod r$. The only possibility that is left is $n_r = pq$ as we assume $n_r \neq 1$. Similarly, $n_q \equiv 1 \pmod q$ and $n_q \mid pr$. The divisors of $pr$ are 1, $p$, $r$, and $pr$. Since $p < q$ thus $p \not\equiv 1 \pmod q$. Hence $n_q \geq r$ as we assume $n_q \neq 1$. Similarly, $n_p \geq q$.

    We now consider the number of elements with order $p$, $q$, and $r$.
    \begin{itemize}
        \item $\boxed{p}$ With $n_p \geq q$, there are at least $q(p-1)$ elements of order $p$. We minus 1 because one of the elements in a Sylow $p$-subgroup is the identity with order 1.
        \item $\boxed{q}$ With $n_q \geq r$, there are at least $r(q-1)$ elements of order $q$.
        \item $\boxed{r}$ We know $n_r = pq$ so there are exactly $pq(r-1)$ elements of order $r$.
    \end{itemize}
    Since the total number of elements, $pqr$, must be at least the sum of the numbers of these elements, thus
    \begin{align*}
        pqr &\geq q(p-1) + r(q-1) + pq(r-1)\\
        &= pq - q + qr - r + pqr - pq\\
        &= pqr + qr - q - r
    \end{align*}
    which means $qr - q - r \leq 0$. Rearranging, we see
    \[
        q \leq \frac{r}{r-1} = 1 + \frac{1}{r-1}.
    \]
    Since $p < q$ and they are both primes, we must have $q \geq 3$. Hence one sees
    \[
        3 \leq q \leq 1 + \frac{1}{r-1} \leq 2
    \]
    which is a clear contradiction. Hence, at least one of $n_p$, $n_q$, or $n_r$ is 1, meaning that there exists a non-trivial proper normal subgroup in $G$ by \myref{corollary-sylow-subgroup-is-normal-if-it-is-unique}. Therefore $G$ is non-simple.
\end{questions}

\section{Composition Series}
\subsection*{Exercises}
\begin{questions}
    \item \begin{partquestions}{\roman*}
        \item One sees clearly that $\{0, 2\}$ is the only non-trivial proper normal subgroup of $G$, so the subnormal series of length 2 is $1 \lhd \{0, 2\} \lhd G$.
        \item There are 2 factor groups of the above subnormal series. The first is $\{0, 2\} / 1 \cong \Z_2$ and the second is
        \begin{align*}
            G / \{0, 2\} &= \{g \oplus_4 \{0, 2\} \vert g \in G\}\\
            &= \{\{0, 2\}, \{1, 3\}, \{2, 0\}, \{3, 1\}\}\\
            &= \{\{0, 2\}, \{1, 3\}\}\\
            &= \langle \{1, 3\} \rangle\\
            &\cong \Z_2.
        \end{align*}
        \item Since $1 \lhd G$ and $\{0, 2\} \lhd G$ thus the subnormal series in \textbf{(i)} is also a normal series of $G$.
    \end{partquestions}

    \item By Lagrange's theorem (\myref{thrm-lagrange}) we know that the order of a subgroup must divide the order of the group. Furthermore $\Z_{120}$ is abelian, so any subgroup of it is normal. Now the subgroup $N = \{0, 2, 4, \dots, 118\}$ has 60 elements which is the maximum possible guaranteed by Lagrange. Hence $N$ is the maximal normal subgroup of $\Z_{120}$, which has order 60.

    \item $\Cn{6}$ has
    \begin{align*}
        &1 \lhd \Cn{2} \lhd \Cn{6} \text{ and }\\
        &1 \lhd \Cn{3} \lhd \Cn{6}
    \end{align*}
    as composition series up to isomorphism. In both cases, their composition length is 2. Their respective composition factors are
    \begin{itemize}
        \item $\Cn{2} / 1 \cong \Cn{2}$ and $\Cn{6} / \Cn{2} \cong \Cn{3}$ by \myref{exercise-Zmn-mod-Zn-cong-Zn}; and
        \item $\Cn{3} / 1 \cong \Cn{3}$ and $\Cn{6} / \Cn{3} \cong \Cn{2}$ by \myref{exercise-Zmn-mod-Zn-cong-Zn},
    \end{itemize}
    up to isomorphism.

    \item Let the group in question be $G$. We know by Cauchy's theorem (\myref{thrm-cauchy}) and \myref{exercise-group-of-order-multiple-of-prime-has-subgroup-of-prime-order}, and by writing $p^2$ as $p \times p$, that $G$ has a subgroup of order $p$. Let this subgroup be $H$.

    Lagrange's theorem (\myref{thrm-lagrange}) tells us that the possible orders of the subgroups of $G$ are 1, $p$, and $p^2$. These subgroups are $\{e\}$, $H$, and $G$ respectively. Furthermore, by \myref{problem-group-of-order-prime-squared-is-abelian}, $G$ must be abelian, therefore its subgroups are all normal (\myref{prop-subgroup-of-abelian-group-is-normal}). Finally, a corollary of Lagrange's theorem (\myref{corollary-group-with-prime-order-subgroups}) says that the only subgroups of $H$ are the trivial group and the group itself. Hence, $G$ has only one composition series, namely $1 \lhd H \lhd G$.
\end{questions}

\subsection*{Problems}
\begin{questions}
    \item \begin{partquestions}{\roman*}
        \item We note $\mathrm{V}$ has order 4. As $4 = 2 \times 2$, thus we know that $\mathrm{V}$ has a subgroup of order 2 (which is cyclic) by Cauchy's theorem (\myref{thrm-cauchy}). Now $\mathrm{V}$ is abelian (\myref{problem-group-of-order-prime-squared-is-abelian}) which means that the subgroup of order 2 is normal (\myref{prop-subgroup-of-abelian-group-is-normal}). Finally, the only possible order for a non-trivial proper subgroup of $\mathrm{V}$ is 2 by Lagrange's theorem (\myref{thrm-lagrange}). Hence, the only composition series for $\mathrm{V}$ is $1 \lhd \Cn{2} \lhd \mathrm{V}$ up to isomorphism.\newline
        (Note that this analysis applies for any group of order 4.)

        \item Recall that $\mathrm{Q} = \langle \alpha, \beta \vert \alpha^4 = e, \alpha^2 = \beta^2, \text{ and } \beta\alpha = \alpha^3\beta \rangle$. From the solution of \myref{exercise-normal-subgroups-of-quarternion-group}, the maximal subgroups of $\mathrm{Q}$ are $G_1 = \langle \alpha \rangle$, $G_2 = \langle \beta \rangle$, and $G_3 = \langle \alpha\beta \rangle$ (by setting $\alpha = i$ and $\beta = j$). We note the following.
        \begin{itemize}
            \item $G_1 = \{e, \alpha, \alpha^2, \alpha^3\} \cong \Cn{4}$.
            \item $G_2 = \{e, \beta, \beta^2, \beta^3\} = \{e, \beta, \alpha^2, \alpha^2\beta\} \cong \mathrm{V}$ where $a = \alpha^2$ and $b = \beta$.
            \item $G_3 = \{e, \alpha\beta, (\alpha\beta)^2, (\alpha\beta)^3\} = \{e, \alpha\beta, \alpha^2, \alpha^3\beta\} \cong \mathrm{V}$ with $a = \alpha\beta$ and $b = \alpha^2$.
        \end{itemize}
        Also, note that $\Cn{2} \cong \langle \alpha^2 \rangle \lhd G_1$, $\Cn{2} \cong \langle \beta^2 \rangle \lhd G_2$, and $\Cn{2} \cong \langle (\alpha\beta)^2 \rangle \lhd G_3$. Hence, the two series up to isomorphism are
        \begin{align*}
            1 \lhd \Cn{2} \lhd \Cn{4} \lhd \mathrm{Q} & \text{ and }\\
            1 \lhd \Cn{2} \lhd \mathrm{V} \lhd \mathrm{Q}.
        \end{align*}

        \item By the Jordan-H\"older theorem (\myref{thrm-jordan-holder}), the composition factors are isomorphic to each other. We note
        \begin{itemize}
            \item $\Cn{2} / 1 \cong \Cn{2}$;
            \item $\Cn{4} / \Cn{2} \cong \Cn{2}$ by \myref{exercise-Zmn-mod-Zn-cong-Zn}; and
            \item $\mathrm{V} / \Cn{2} \cong (\Cn{2})^2 / \Cn{2} \cong \Cn{2}$ by \myref{problem-cartesian-product-of-group-by-group-isomorphic-to-group}.
        \end{itemize}
        The only unaccounted set of factors is $\mathrm{Q}/\mathrm{V}$ and $\mathrm{Q}/\Cn{4}$. So, either $\mathrm{Q}/\mathrm{V} \cong \Cn{2}$ and $\mathrm{Q}/\Cn{4} \cong \Cn{2}$, or $\mathrm{Q}/\mathrm{V} \cong \mathrm{Q}/\Cn{4}$. Hence $\mathrm{Q}/H \cong \mathrm{Q}/K$.
    \end{partquestions}

    \item We know that $\An{4} \lhd \Sn{4}$ by \myref{prop-An-normal-subgroup-of-Sn}. Note $\An{4}$ is a maximal normal subgroup since $|\An{4}| = \frac{4!}2 = 12$ by \myref{prop-order-of-An}, and a subgroup's order must divide the order of the group by Lagrange's theorem (\myref{thrm-lagrange}).

    Now applying that theorem on $\An{4}$, we see that the possible orders of a subgroup of $\An{4}$ are 6, 4, 3, 2, and 1. We claim that a subgroup of order 6 does not exist. Note that $\An{4}$ contains
    \begin{itemize}
        \item 1 element of order 1;
        \item 3 elements of order 2; and
        \item 8 elements of order 3.
    \end{itemize}
    If a subgroup of order 6 exists (say, $H$), then its index would be $\frac{12}{6} = 2$ by Lagrange, meaning $H$ contains all odd order elements (\myref{problem-subgroup-of-index-2}). However, there are $1 + 8 = 9$ odd order elements, meaning that $H$ has an order of at least 9, a contradiction. Hence a subgroup with $\An{4}$ of order 6 is impossible.

    Now we note that a subgroup of order $4 = 2^2$ exists by a corollary of the First Sylow Theorem (\myref{corollary-sylow-p-subgroup-exists}) as it is a Sylow 2-subgroup. The Third Sylow Theorem (\myref{thrm-sylow-3}) tells us how many Sylow 2-subgroups there are, in particular
    \begin{itemize}
        \item $n_2 \vert 3$, so $n_2$ is 1 or 3; and
        \item $n_2 \equiv 1 \pmod2$, so $n_2 \in \{1, 3, 5, \dots\}$.
    \end{itemize}
    Hence $n_2 = 1$ or $n_2 = 3$. Now if $n_2 = 3$, then the number of elements of order of 1, 2, or 4 is
    \[
        3 \times (4 - 1) + 1 = 10,
    \]
    where the 3 is $n_2$, the $4-1$ is the number of non-identity elements in each Sylow 2-subgroup, and the $+1$ is to add the identity element. However, as noted above, there are only 4 elements of order 1, 2, or 4, a contradiction. Hence $n_2 = 1$, meaning the Sylow 2-subgroup (which is a subgroup of order 4) is normal (\myref{corollary-sylow-subgroup-is-normal-if-it-is-unique}). Therefore the subgroup of order 4 is the maximal normal subgroup of $\An{4}$.

    We note that the subgroup of order 4 of $\An{4}$ is not $\Cn{4}$ (as this would imply that $\An{4}$ has an element of order 4, which it does not). Hence, from \myref{problem-smallest-nonabelian-group}, the subgroup of order 4 must be isomorphic to the Klein-4 group, $\mathrm{V}$.

    Note that a group of order 4 has a subgroup of order 2 by Cauchy's theorem (\myref{thrm-cauchy}). Clearly such a subgroup is cyclic (since 2 is prime), and has index $\frac42 = 2$, meaning that it is normal in the group of order 4. Furthermore the trivial group is always a subgroup of any group.

    Hence, the composition series for $\Sn{4}$, up to isomorphism, is
    \[
        1 \lhd \Cn{2} \lhd \mathrm{V} \lhd \An{4} \lhd \Sn{4}.
    \]
    \begin{remark}
        We list the actual subgroups that are isomorphic to the above terms in the composition series here.
        \begin{itemize}
            \item $\Cn{2}$: $\{e, \begin{pmatrix}1&2\end{pmatrix}\begin{pmatrix}3&4\end{pmatrix}\}$
            \item V: $\{e, \begin{pmatrix}1&2\end{pmatrix}\begin{pmatrix}3&4\end{pmatrix}, \begin{pmatrix}1&3\end{pmatrix}\begin{pmatrix}2&4\end{pmatrix}, \begin{pmatrix}1&4\end{pmatrix}\begin{pmatrix}2&3\end{pmatrix}\}$
            \item $\An{4}$ is an actual subgroup of $\Sn{4}$
        \end{itemize}
    \end{remark}
\end{questions}

\chapter{Simple Groups}
Simple groups can be thought of as the `building blocks' of all (finite) groups. The finite simple groups have been completely classified; each belongs to one of 18 infinite families, or is one of 26 sporadic groups that do not follow a specific pattern. We look at the classification of the families of these simple groups here.

\section{Cyclic Groups of Prime Order}
The first infinite family of simple groups we will look at is the family of Cyclic Groups of Prime Order\index{cyclic group!of prime order}.

\begin{lemma}\label{lemma-cyclic-group-simple-iff-order-is-prime}
    $\Cn{n}$ is simple if and only if $n$ is prime.
\end{lemma}
\begin{proof}
    We first prove the forward direction. Suppose $\Cn{n}$ is simple with generator $g$. Then the only normal subgroups of $\Cn{n}$ are the trivial group and the group itself. Seeking a contradiction, assume $n$ is not prime; write $n = ab$ where $a$ and $b$ are positive integers that are both smaller than $n$. Then clearly $\langle g^a\rangle$ is a proper subgroup of $\Cn{n}$. Now $\Cn{n}$ is abelian (\myref{prop-cyclic-group-is-abelian}) which means all subgroups are normal (\myref{prop-subgroup-of-abelian-group-is-normal}). Hence we have found a non-trivial proper normal subgroup of $\Cn{n}$, namely $\langle g^a \rangle$, contradicting that $\Cn{n}$ has no non-trivial proper normal subgroups. Therefore $n$ is prime.

    We now prove the reverse direction. Suppose $n$ is a prime. Then by a corollary of Lagrange's theorem (\myref{corollary-group-with-prime-order-subgroups}), $\Cn{n}$ has no non-trivial proper subgroups. So the only subgroup with order smaller than $n$ is the trivial group, $\{e\}$. Clearly $\Cn{n}$ is normal in itself, and the trivial group is always a normal subgroup. Hence, as the only normal subgroups of $\Cn{n}$ are the trivial group and itself, thus $\Cn{n}$ is simple. Therefore, if $n$ is prime then $\Cn{n}$ is simple.
\end{proof}

In fact, we have a much stronger result which we prove here.

\begin{theorem}\label{thrm-abelian-group-simple-iff-cylic-group-of-prime-order}
    An abelian group is simple if and only if it has prime order.
\end{theorem}
Note that we do not assume that the abelian group is finite; we will show that the group is finite in the proof below.
\begin{proof}
    The reverse direction follows immediately from \myref{lemma-cyclic-group-simple-iff-order-is-prime}, so we prove the forward direction only.

    Suppose $G$ is a simple abelian group; we show that $G$ is finite. Let $g$ a non-identity element of $G$. Then $H = \langle g \rangle$ is a subgroup of $G$. In fact, since $G$ is abelian, $H \unlhd G$ (\myref{prop-subgroup-of-abelian-group-is-normal}). As $G$ is simple, therefore $H = G$, meaning that $g$ is a generator of $G$. Now if $G$ is an infinite group, then one also sees that $\langle g^2 \rangle < G$ which implies $\langle g^2 \rangle \lhd G$, contradicting the fact that $G$ is simple. Hence $G$ is a finite abelian group with generator $g$, meaning $G$ is cyclic. Result follows directly from \myref{lemma-cyclic-group-simple-iff-order-is-prime}.
\end{proof}

From this, we conclude that the only family of simple abelian groups is the family of cyclic groups of prime order.

\section{Alternating Group With Degree $>4$}
The other family of simple groups that is relatively easy to find (and define) is the family of alternating groups with degree above 4\index{alternating group!of degree $>4$}. However, to prove this claim, we need several preliminary results.

\begin{theorem}\label{thrm-group-of-order-60-with->1-sylow-5-subgroup-is-simple}
    Let $G$ be a group of order 60. If $G$ has more than one Sylow 5-subgroup then $G$ is simple.
\end{theorem}
\begin{proof}[Proof (see {\cite[Proposition 4.21]{dummit_foote_2004}})]
    By way of contradiction assume $G$ is a group of order 60 with more than one Sylow 5-subgroup, but has a non-trivial proper normal subgroup $H$. Note $60 = 5 \times 12$, so by the Third Sylow Theorem (\myref{thrm-sylow-3}),
    \begin{itemize}
        \item $12 \vert n_5$, so $n_5 \in \{1, 2, 3, 4, 6, 12\}$; and
        \item $n_5 \equiv 1 \pmod 5$, so $n_5 \in \{1, 6, 11, 16, \dots\}$.
    \end{itemize}
    Therefore $n_5 = 6$ as $n_5 > 1$ (given), i.e. there are 6 Sylow 5-subgroups.

    We note by Lagrange's theorem (\myref{thrm-lagrange}) that the order of $H$ belongs in the set $\{1, 2, 3, 4, 5, 6, 10, 12, 15, 20, 30, 60\}$. As $H$ is a non-trivial proper subgroup of $G$, thus $|H| \neq 1$ and $|H| \neq 60$. That leaves 4 cases which we will deal with separately.
    \begin{enumerate}
        \item $|H| = 6$. Note $6 = 2 \times 3$, so \myref{problem-group-of-order-pq-has-normal-subgroup-of-order-q} tells us that there exists a $N \lhd H$ with $|N| = 3$. Note also $[G:H] = 10$ which is not a multiple of 3, so \myref{problem-normal-subgroup-of-G-contains-all-sylow-p-subgroups} tells us that all Sylow 3-subgroups of $G$ are in $H$. But $N \lhd H$ means that $N$ is the unique Sylow 3-subgroup of $H$ and $G$ (\myref{corollary-sylow-subgroup-is-normal-if-it-is-unique}), so $N \lhd G$ (by the same corollary). Proceed to case 3, using $N$ in place of $H$.

        \item $|H| = 12$. Note $12 = 2^2 \times 3$. Now \myref{exercise-group-of-order-12-has-normal-subgroup-of-3-or-4} (later) tells us that there exists a normal subgroup of $H$ with order 3 or 4. Call that subgroup $N$. If $|N| = 3$ then it is a Sylow 3-subgroup; if $|N| = 4 = 2^2$ it is a Sylow 2-subgroup. As $N \lhd H$, thus $N$ is the unique Sylow 2- or 3- subgroup (\myref{corollary-sylow-subgroup-is-normal-if-it-is-unique}). Since $H \lhd G$, thus $H$ contains all Sylow 2- and 3-subgroups of $G$ (\myref{problem-normal-subgroup-of-G-contains-all-sylow-p-subgroups}), meaning $G$ has only one Sylow 2-subgroup or one Sylow 3-subgroup (or both), in particular $N$. Hence, $N \lhd G$ since a Sylow $p$-subgroup is unique if and only if it is normal (\myref{corollary-sylow-subgroup-is-normal-if-it-is-unique}). Proceed with case 3, using $N$ instead of $H$.

        \item $|H| \in \{2, 3, 4\}$. Since $H \lhd G$, thus $G/H$ is a group. Note $|G/H| \in \{15, 20, 30\}$. We claim that each of these cases produces a new normal subgroup of $G/H$ (call it $\bar{P}$) with order 5. This is proven for the case where $|G/H| = 30$ in \myref{problem-group-of-order-30-has-normal-subgroup-of-order-5}; the other two cases are for \myref{exercise-group-of-order-15-or-20-has-normal-subgroup-of-order-5} (later).

        Now \myref{problem-subgroup-of-quotient-group-is-quotient-group} tells us that $\bar{P}$ has the form $K/H$ where $K < G$ and $H \subseteq K$. Since $\bar{P} = K/H \lhd G/H$, thus for any $g \in G$ and $kH \in \bar{P}$ we have
        \[
            (gH)(kH)(g^{-1}H) = (gkg^{-1})H \in K/H,
        \]
        which means $gkg^{-1} \in K$. Therefore $K \lhd G$ by definition of normality.

        Observe that this means that
        \[
            |K| = |K/H||H| = |\bar{P}||H| = 5|H|,
        \]
        meaning $K$ is a normal subgroup of $G$ with an order that is a multiple of 5. Proceed to case 4, using $K$ in place of $H$.

        \item $|H|$ is a multiple of 5, meaning $H$ has a Sylow 5-subgroup. Note that there are $5-1=4$ non-identity elements in each Sylow 5-subgroup; therefore
        \[
            |H| \geq n_5(5-1) = 24
        \]
        which means that $|H| = 30$. By \myref{problem-group-of-order-30-has-normal-subgroup-of-order-5} again, such a group has only a unique Sylow 5-subgroup.  Note $5 \nmid [G:H]$, so \myref{problem-normal-subgroup-of-G-contains-all-sylow-p-subgroups} implies all Sylow 5-subgroups of $G$ are in $H$. However, right at the start, we concluded that there are 6 Sylow 5-subgroups in $G$, so $H$ must have 6 Sylow 5-subgroups, a contradiction.
    \end{enumerate}
    Hence, $H$ does not exist, and so $G$ is simple.
\end{proof}

\begin{exercise}\label{exercise-group-of-order-12-has-normal-subgroup-of-3-or-4}
    Prove that a group of order 12 either has a normal subgroup of order 3, or a normal subgroup of order 4, or both.
\end{exercise}

\begin{exercise}\label{exercise-group-of-order-15-or-20-has-normal-subgroup-of-order-5}
    Prove that a group of each of the following orders has a normal subgroup of order 5.
    \begin{partquestions}{\alph*}
        \item 15
        \item 20
    \end{partquestions}
\end{exercise}

\begin{corollary}\label{corollary-A5-is-simple}
    The group $\An5$ is simple.
\end{corollary}
\begin{proof}
    \myref{exercise-A5-has-two-distinct-subgroups-of-order-5} (later) gives two distinct subgroups of order 5. Since $|\An{5}| = 60 = 2^2 \times 3 \times 5$, thus subgroups of order 5 are Sylow 5-subgroups. Therefore $\An5$ is simple by \myref{thrm-group-of-order-60-with->1-sylow-5-subgroup-is-simple}.
\end{proof}
\begin{exercise}\label{exercise-A5-has-two-distinct-subgroups-of-order-5}
    Consider the permutation $\sigma = \begin{pmatrix}1&3&2&4&5\end{pmatrix}$.
    \begin{partquestions}{\roman*}
        \item Explain why $\sigma \in \An{5}$.
        \item Find the order of the subgroup $\langle \sigma \rangle$.
        \item Find another subgroup of $\An{5}$ with order 5.
    \end{partquestions}
\end{exercise}

We also state and prove a fairly obvious proposition.

\begin{proposition}\label{prop-An-stabilizer-of-i-is-isomorphic-to-A(n-1)}
    Let the integer $n \geq 3$. Let the set $\{1, 2, 3, \dots, n\}$ be denoted by $\mathcal{N}_n$. Suppose $\An{n}$ acts on $\mathcal{N}_n$ naturally. Then $\Stab{\An{n}}{r} \cong \An{n-1}$.
\end{proposition}
\begin{proof}
    We note that elements of $\Stab{\An{n}}{r}$ are permutations that fix $r$, thereby permuting the $n - 1$ other elements. Therefore the elements of $\Stab{\An{n}}{r}$ are even permutations on $n - 1$ elements, i.e. $\Stab{\An{n}}{r} \cong \An{n-1}$.
\end{proof}

With these results, we are ready to prove the main result of this section.

\begin{theorem}\label{thrm-An-is-simple-for-n>=5}
    The group $\An{n}$ is simple if $n \geq 5$.
\end{theorem}
\begin{proof}[Proof (see {\cite[Theorem 4.24]{dummit_foote_2004}})]
    We induct on $n$. For brevity, let $\mathcal{N}_n = \{1, 2, 3, \dots, n\}$. The base case of $n = 5$ is covered by \myref{corollary-A5-is-simple}. Assume that $\An{k-1}$ is simple for some $k \geq 6$; we will prove that $\An{k}$ is also simple.

    Let $G = \An{k}$ and, seeking a contradiction, assume that $G$ has a non-trivial proper normal subgroup $H$. Let $G$ act on $\mathcal{N}_{k}$ naturally; thus we see that $\Stab{G}{i} \leq G$ with $\Stab{G}{i} \cong \An{k-1}$ (\myref{prop-An-stabilizer-of-i-is-isomorphic-to-A(n-1)}) for any $i \in \mathcal{N}_k$. Note $\An{k-1}$ is simple by the induction hypothesis, so $\Stab{G}{i}$ is simple for each $i \in \mathcal{N}_{k}$.

    Suppose first that there is some non-identity $\pi \in H$ such that $\pi(i) = i$ for some $i \in \mathcal{N}_{k}$. This means that $\pi$ fixes $i$; thus $\pi \in H \cap \Stab{G}{i}$. Note that since $H \lhd G$ and $\Stab{G}{i} \leq G$ thus $H \cap \Stab{G}{i} \lhd \Stab{G}{i}$ by the Second Isomorphism Theorem (\myref{thrm-isomorphism-2}), statement 4. But as $\Stab{G}{i}$ is simple (and non-trivial) we must have $H \cap \Stab{G}{i} = \Stab{G}{i}$. Therefore $\Stab{G}{i} \subseteq H$ which means $\Stab{G}{i} \leq H$. Now by \myref{exercise-conjugate-of-stabilizer} (later), for any $\sigma \in G$, we know that $\sigma\Stab{G}{i}\sigma^{-1} = \Stab{G}{\sigma(i)}$. Therefore we see
    \[
        \sigma\Stab{G}{i}\sigma^{-1} \leq \sigma H\sigma^{-1} = H
    \]
    since $H \lhd G$. Thus, for any $j \in \mathcal{N}_{k+1}$, there exists $\sigma \in G$ where $\sigma(i) = j$ such that
    \[
        \sigma\Stab{G}{i}\sigma^{-1} = \Stab{G}{j} \leq H.
    \]

    Note that any $\lambda \in G$ may be written as a product of an even number of transpositions (\myref{thrm-parity-of-permutation}), say $2t$ transpositions. Thus, we may write $\lambda = \lambda_1\lambda_2\cdots\lambda_t$ where each $\lambda_i$ is a product of two transpositions. Now as $k \geq 5$, each $\lambda_i$ (which could at most consist of two disjoint cycles of 4 elements) must fix at least one element in $\mathcal{N}_{k}$, say $j$. That is, $\lambda_i \in \Stab{G}{j}$ for some $j \in \mathcal{N}_{k}$. Since $\lambda_i \in \Stab{G}{j} \leq H$ for some $j \in \mathcal{N}_{k}$, thus $\lambda_i \in H$. Hence $\lambda = \lambda_1\lambda_2\cdots\lambda_t \in H$. Therefore, any element in $G$ is also in $H$, meaning $G \subseteq H$, contradicting the fact that $H \lhd G$.

    We conclude that for any $\pi \in H$, if $\pi \neq \id$ then $\pi(i) \neq i$ for all $i \in \mathcal{N}_k$. The contrapositive of this statement is that if $\pi(i) = i$ for some $i \in \mathcal{N}_k$ then $\pi = \id$. Now suppose $\pi_1, \pi_2 \in H$ and $\pi_1(i) = \pi_2(i)$ for some $i \in \mathcal{N}_{k}$. Then $\pi_2^{-1}\pi_1(i) = i$, which implies $\pi_2^{-1}\pi_1 = \id$. Hence $\pi_1 = \pi_2$. Therefore, if $\pi_1, \pi_2 \in H$ and $\pi_1(i) = \pi_2(i)$ for some $i \in \mathcal{N}_{k}$, then $\pi_1 = \pi_2$.

    Now suppose a non-identity $\pi_1 \in H$ exists such that the cycle decomposition of $\pi_1$ contains a cycle of length of at least 3, say
    \[
        \pi_1 = \begin{pmatrix}a_1&a_2&a_3&\cdots\end{pmatrix} \begin{pmatrix}b_1&b_2&\cdots\end{pmatrix}\cdots
    \]
    where $a_1$, $a_2$, $a_3$, $b_1$, $b_2$, etc. are distinct (which is possible since $k \geq 5$). We note an element $\sigma \in G$ exists such that $\sigma(a_1) = a_1$, $\sigma(a_2) = a_2$, but $\sigma(a_3) \neq a_3$ because $k \geq 4$ (for example, the permutation $\begin{pmatrix}a_3 & a_4\end{pmatrix}$). Then \myref{exercise-conjugation-of-permutation-by-another} (later) tells us that
    \[
        \sigma\pi_1\sigma^{-1} = \begin{pmatrix}a_1&a_2&\sigma(a_3)&\cdots\end{pmatrix} \begin{pmatrix}\sigma(b_1)&\sigma(b_2)&\cdots\end{pmatrix}\cdots.
    \]
    Set $\pi_2 = \sigma\pi_1\sigma^{-1}$, which is clearly distinct from $\pi_1$. Then we see $\pi_1(a_1) = \pi_2(a_1) = a_2$, contrary to the above observation that $\pi_1(i) = \pi_2(i)$ for any $i \in \mathcal{N}_k$ implies $\pi_1 = \pi_2$. Therefore only 2-cycles can appear in the cycle decomposition of non-identity elements of $H$.

    Let $\pi_1 \in H$ be a non-identity element, so that
    \[
        \pi_1 = \begin{pmatrix}a_1&a_2\end{pmatrix} \begin{pmatrix}a_3&a_4\end{pmatrix} \begin{pmatrix}a_5&a_6\end{pmatrix}\cdots
    \]
    where each $a_i$ is distinct (note that $k \geq 6$ is used above). Consider the permutation $\sigma = \begin{pmatrix}a_1&a_2\end{pmatrix} \begin{pmatrix}a_3&a_5\end{pmatrix}$, which is in $G$ since it is made up of 2 transpositions. Then \myref{exercise-conjugation-of-permutation-by-another} again gives
    \[
        \sigma\pi_1\sigma^{-1} = \begin{pmatrix}a_1&a_2\end{pmatrix} \begin{pmatrix}a_5&a_4\end{pmatrix} \begin{pmatrix}a_3&a_6\end{pmatrix}\cdots.
    \]
    Setting $\pi_2 = \sigma\pi_1\sigma^{-1}$ again gives two distinct permutations $\pi_1$ and $\pi_2$ where $\pi_1(a_1) = \pi_2(a_1) = a_2$, again contrary to the above observation.

    We conclude that such a non-trivial proper normal subgroup $H$ of $\An{k}$ cannot exist. Thus, $\An{k-1}$ being simple implies that $\An{k}$ is also simple.

    By mathematical induction, $\An{n}$ is simple for all $n \geq 5$.
\end{proof}

\begin{exercise}\label{exercise-conjugate-of-stabilizer}
    Let $S$ be a non-empty set and let $G \leq \Sym{S}$ act on $S$. Show that $\sigma\Stab{G}{x}\sigma^{-1} = \Stab{G}{\sigma(x)}$ for any $\sigma \in G$ and $x \in S$.
\end{exercise}

\begin{exercise}\label{exercise-conjugation-of-permutation-by-another}
    Let $\sigma, \pi \in \Sn{n}$. Suppose $\sigma$ has cycle decomposition
    \[
        \begin{pmatrix}a_1&a_2&\cdots&a_{k_1}\end{pmatrix} \begin{pmatrix}b_1&b_2&\cdots&b_{k_2}\end{pmatrix}\cdots,
    \]
    where $a_1, a_2, \dots, a_{k_1}, b_1, b_2, \dots, b_{k_2}, \dots$ are all distinct. Show that
    \[
        \pi\sigma\pi^{-1} = \begin{pmatrix}\pi(a_1)&\cdots&\pi(a_{k_1})\end{pmatrix} \begin{pmatrix}\pi(b_1)&\cdots&\pi(b_{k_2})\end{pmatrix}\cdots,
    \]
    that is, $\pi\sigma\pi^{-1}$ is obtained from $\sigma$ by replacing each entry $i$ by $\pi(i)$.
\end{exercise}

\begin{corollary}
    The group $\An{n}$ is simple for $n \geq 3$ and $n \neq 4$.
\end{corollary}
\begin{proof}
    We note $\An3$ has order $\frac{3!}{2} = 3$ which is prime, so $\An3 \cong \Cn3$ which is simple. Also $\An{n}$ is simple for $n \geq 5$ by \myref{thrm-An-is-simple-for-n>=5}.
\end{proof}

We note that $\An4$ is non-simple by the solution of \myref{problem-S4-composition-series}, in which we found that $\An4$ has a unique composition series of
\[
    1 \lhd \Cn2 \lhd \mathrm{V} \lhd \An4
\]
up to isomorphism.

\section{Groups of Lie Type}
We briefly mention groups of Lie type; we will not prove any significant results here.

Groups of Lie (pronounced ``lee'') type\index{groups of Lie type} usually refers to finite groups that are closely related to the group of rational points of a reductive linear algebraic group with values in a finite field. We will cover finite fields in part III. We briefly mention these groups here.

The list below, taken from \cite{wikipedia_list-of-simple-groups}, is a list of the families of simple groups of Lie type. In what follows, $n$ is a positive integer and $q$ is a positive power of a prime number $p$.
\begin{itemize}
    \item \term{Classical Chevalley groups}\index{Chevalley groups!classical}: there are 4 families of simple groups.
    \begin{itemize}
        \item $A_n(q)$, except for $A_1(2)$ and $A_1(3)$. There are several duplicates, which are
        \begin{itemize}
            \item $A_1(4) \cong A_1(5) \cong \An{5}$;
            \item $A_1(7) \cong A_2(2)$;
            \item $A_1(9) \cong \An{6}$; and
            \item $A_3(2) \cong \An{8}$.
        \end{itemize}
        We note that $\An{n}$ is not the same as $A_n(q)$. We distinguish between the alternating group of degree $n$ ($\An{n}$) and the groups of Lie type $A_n(q)$ by letting the latter be in italics and the former be in `normal' font.

        \item $B_n(q)$ for $n > 1$, except for $B_2(2)$. There are several duplicates, which are
        \begin{itemize}
            \item $B_n(2^m) \cong C_n(2^m)$; and
            \item $B_2(3) \cong {^2A_3(2^2)} = {^2A_3(4)}$, where ${^2A_3(4)}$ is a classical Steinberg group.
        \end{itemize}
        \item $C_n(q)$ for $n > 2$. The only duplicate is $C_n(2^m) \cong B_n(2^m)$ mentioned earlier.
        \item $D_n(q)$ for $n > 3$.
    \end{itemize}

    \item \term{Exceptional Chevalley groups}\index{Chevalley groups!exceptional}: there are 5 families of such groups.
    \begin{itemize}
        \item $E_6(q)$;
        \item $E_7(q)$;
        \item $E_8(q)$;
        \item $F_4(q)$; and
        \item $G_2(q)$, except for $G_2(2)$.
    \end{itemize}

    \item \term{Classical Steinberg groups}\index{Steinberg groups!classical}: there are 2 families of simple groups.
    \begin{itemize}
        \item ${^2A_n(q^2)}$ for $n > 1$, except for ${^2A_2(2^2)} = {^2A_2(4)}$. The only duplicate is ${^2A_3(2^2)} \cong B_2(3)$ mentioned earlier.
        \item ${^2D_n(q^2)}$ for $n > 3$.
    \end{itemize}

    \item \term{Exceptional Steinberg groups}\index{Steinberg groups!exceptional}: there are 2 families of simple groups.
    \begin{itemize}
        \item ${^2E_6(q^2)}$; and
        \item ${^3D_4(q^3)}$.
    \end{itemize}

    \item \term{Suzuki groups}\index{Suzuki groups}: there is 1 family of simple groups, which is ${^2B_2(q)}$ where $q = 2^{2n+1}$ and $n \geq 1$. Such a group has order $q^2(q^2+1)(q-1)$.

    \item \term{Ree groups}\index{Ree groups}: there are 2 families of simple groups.
    \begin{itemize}
        \item $^2F_4(q)$ where $q = 2^{2n+1}$ and $n \geq 1$. The order of such a group is $q^{12}(q^6+1)(q^4-1)(q^3+1)(q-1)$.
        \item $^2G_2(q)$ where $q = 3^{2n+1}$ and $n \geq 1$. The order of such a group is $q^3(q^3+1)(q-1)$.
    \end{itemize}
\end{itemize}

There is also the \term{Tits group}\index{Tits group}, $^2F_4(2)'$, with an order of $17,971,200$. It is the commutator subgroup of $^2F_4(2)$, which is a Lie group but not a simple group. The fact that $^2F_4(2)'$ is linked to Ree groups makes most authors consider it not a sporadic group (see below).

\section{The Sporadic Groups}
Along with the 18 infinite families of simple groups, there are also 26 sporadic simple groups\index{sporadic group} that do not fall within the families (27 if the Tits group is considered a sporadic group).

We first list four categories of sporadic groups.
\begin{table}[h]
    \centering
    \begin{tabular}{|l|l|l|}
        \hline
        \textbf{Name} & \textbf{Symbol} & \textbf{Order} \\ \hline
        \multirow{5}{*}{\term{Mathieu Groups}\index{Mathieu groups}} & $\mathrm{M}_{11}$ & 7,290 \\ \cline{2-3}
        & $\mathrm{M}_{12}$ & 95,040 \\ \cline{2-3}
        & $\mathrm{M}_{22}$ & 443,520 \\ \cline{2-3}
        & $\mathrm{M}_{23}$ & 10,200,960 \\ \cline{2-3}
        & $\mathrm{M}_{24}$ & 244,823,040 \\ \hline
        \multirow{4}{*}{\term{Janko Groups}\index{Janko groups}} & $\mathrm{J}_1$ & 175,560 \\ \cline{2-3}
        & $\mathrm{J}_2$ & 604,800 \\ \cline{2-3}
        & $\mathrm{J}_3$ & 50,232,960 \\ \cline{2-3}
        & $\mathrm{J}_4$ & 86,775,571,046,077,562,880 \\ \hline
        \multirow{3}{*}{\term{Conway Groups}\index{Conway groups}} & $\mathrm{Co}_3$ & 495,766,656,000 \\ \cline{2-3}
        & $\mathrm{Co}_2$ & 42,305,421,312,000 \\ \cline{2-3}
        & $\mathrm{Co}_1$ & 4,157,776,806,543,360,000 \\ \hline
        \multirow{3}{*}{\term{Fischer Groups}\index{Fischer groups}} & $\mathrm{Fi}_{22}$ & 64,561,751,654,400 \\ \cline{2-3}
        & $\mathrm{Fi}_{23}$ & 4,089,470,473,293,004,800 \\ \cline{2-3}
        & $\mathrm{Fi}_{24}$ & 1,255,205,709,190,661,721,292,800 \\ \hline
    \end{tabular}
\end{table}



More sporadic groups are listed below.
\begin{table}[h]
    \centering
    \begin{tabular}{|l|l|l|}
        \hline
        \textbf{Group}        & \textbf{Symbol} & \textbf{Order}  \\ \hline
        \term{Higman-Sims group}\index{Higman-Sims group}     & HS              & 44,352,000      \\ \hline
        \term{McLaughlin group}\index{McLaughlin group}      & McL             & 898,128,000     \\ \hline
        \term{Held group}\index{Held group}            & He              & 4,030,387,200   \\ \hline
        \term{Rudvalis group}\index{Rudvalis group}        & Ru              & 145,926,144,000 \\ \hline
        \term{Suzuki sporadic group}\index{Suzuki sporadic group} & Suz             & 448,345,497,600 \\ \hline
        \term{O'Nan group}\index{O'Nan group}           & $\mathrm{O'N}$  & 460,815,505,920 \\ \hline
        \term{Harada-Norton group}\index{Harada-Norton group}   & HN              & 273,030,912,000,000    \\ \hline
        \term{Lyons group}\index{Lyons group}           & Ly              & 51,765,179,004,000,000 \\ \hline
        \term{Thompson group}\index{Thompson group}        & Th              & 90,745,943,887,872,000 \\ \hline
    \end{tabular}
\end{table}

The remaining 2 sporadic groups are special in that they have extremely large order.
\begin{itemize}
    \item The \term{Baby Monster group}\index{Baby Monster group}, usually denoted $\mathrm{B}$, has order
    \begin{align*}
        &2^{41} \times 3^{13} \times 5^6 \times 7^2 \times 11 \times 13 \times 17 \times 19 \times 23 \times 31 \times 47\\
        &= 4,154,781,481,226,426,191,177,580,544,000,000.
    \end{align*}
    \item The \term{Monster group}\index{Monster group}, usually denoted $\mathrm{M}$, has order $2^{46} \times 3^{20} \times 5^9 \times 7^6 \times 11^{2} \times 13^3 \times 17 \times 19 \times 23 \times 29 \times 31 \times 41 \times 47 \times 59 \times 71$ which equals 808,017,424,794,512,875,886,459,904,961,710,757,005,754,368,\linebreak000,000,000. It is the largest sporadic group.
\end{itemize}

\section{The Classification Theorem of Finite Simple Groups}
One might rightly wonder what the importance of listing out all of these different types of simple groups are. It turns out that, amazingly, that these results provide a complete classification of what a finite simple group can really be. This is captured in the Classification Theorem of Finite Simple Groups\index{Classification Theorem of Finite Simple Groups}, which is sometimes called the enormous theorem\index{Enormous Theorem}.

\begin{theorem}[Classification Theorem]
    Every finite simple group is isomorphic to either
    \begin{itemize}
        \item a cyclic group of prime order;
        \item an alternating group with degree of at least 5;
        \item a group in the 16 infinite families of groups of Lie type, or the Tits group; or
        \item one of 26 sporadic groups.
    \end{itemize}
\end{theorem}

The proof of this theorem required tens of thousands of pages in hundreds of articles, written by a large number of authors that were  published mostly between 1955 to 2004. The longest paper, and the last paper needed to fill in the gap for quasithin groups, was published in 2004 by Aschbacher and Smith and spanned in a 1221 pages. But after all that work, mathematicians had a complete classification of all finite simple groups.


%=========================================
\appendix
\chapter{Exercise Solutions}
\section{Introduction to Groups}
\subsection*{Exercises}
\begin{questions}
    \item There are 6! = 720 possible permutations of 6 points, so there are 720 symmetries in the group given. That is, the order of the symmetric group of degree 6 is 720.
\end{questions}

\subsection*{Problems}
\begin{questions}
    \item \begin{partquestions}{\alph*}
        \item This is a group. Addition is clearly closed and associative. The identity is 0. The inverse of any element $x$ is $-x$.
        \item This is not a group. Inverses do not exist. For example, the element $2$ does not have an inverse under multiplication.
        \item This is a group. Multiplication is clearly closed and associative. The identity is $1$. The inverse of any element $x$ is $\frac1x$.
        \item This is a group. Multiplication is clearly closed and associative. The identity is $0$ and the inverse is $0$.
        \item This is not a group. Addition is not closed: $1 + 1 = 2$ which is not in the group.
        \item This is a group. Multiplication is clearly closed and associative. The identity is $1$ and the inverse is $1$.
    \end{partquestions}

    \item We show that the trivial group is indeed a group by showing that the four group axioms hold.
    \begin{itemize}
        \item \textbf{Closure}: The only element in the underlying set is $e$, and $e \ast e = e \in \{e\}$. Thus the structure is closed under $\ast$.
        \item \textbf{Associativity}: Clearly $e \ast (e \ast e) = e \ast e = e$ and $(e \ast e) \ast e = e \ast e = e$ which means that $\ast$ is associative.
        \item \textbf{Identity}: The identity is clearly $e$.
        \item \textbf{Inverse}: The only element is $e$, and because $e \ast e = e$ hence $e$ is its own inverse.
    \end{itemize}
    Therefore $(\{e,\}, \ast)$ is a group.
\end{questions}

\section{Basics of Groups}
\subsection*{Exercises}
\begin{questions}
    \item The Cayley table of $(\Z_6, \otimes_6)$ is as follows:
    \begin{table}[H]
        \centering
        \begin{tabular}{|l|l|l|l|l|l|l|}
        \hline
        \textbf{$\otimes_n$} & \textbf{0} & \textbf{1} & \textbf{2} & \textbf{3} & \textbf{4} & \textbf{5} \\ \hline
        \textbf{0}       & 0          & 0          & 0          & 0          & 0          & 0          \\ \hline
        \textbf{1}       & 0          & 1          & 2          & 3          & 4          & 5          \\ \hline
        \textbf{2}       & 0          & 2          & 4          & 0          & 2          & 4          \\ \hline
        \textbf{3}       & 0          & 3          & 0          & 3          & 0          & 3          \\ \hline
        \textbf{4}       & 0          & 4          & 2          & 0          & 4          & 2          \\ \hline
        \textbf{5}       & 0          & 5          & 4          & 3          & 2          & 1          \\ \hline
        \end{tabular}
    \end{table}

    Since the identity is $1$, and the row (and column) of 0 does not have a $1$, thus $0$ does not have an inverse. Therefore $(\Z_6, \oplus_6)$ is not a group.

    \item Note that $(xx^{-1})^{-1} = (x^{-1})^{-1}x^{-1}$ by Shoes and Socks and $(xx^{-1})^{-1} = e^{-1} = e$. Thus $(x^{-1})^{-1}x^{-1} = e$. Multiplying both sides on the right by $x$ yields $(x^{-1})^{-1} = ex = x$, i.e. $(x^{-1})^{-1} = x$.

    \item We consider a proof by induction via inducting on $n$.

    The base case of $n = 0$ clearly holds true since
    \begin{align*}
        (x^{-1})^0 &= e & (\text{definition of }g^0 \text{ for any }g\in G)\\
        &= e^{-1} & (\myref{prop-inverse-of-identity-is-identity})\\
        &= (x^0)^{-1}. & (\text{definition of }x^0)
    \end{align*}

    Now assume that the statement holds for a non-negative integer $k$, i.e. $(x^{-1})^k = (x^k)^{-1}$. We are to show that the statement holds for $k+1$, i.e. $(x^{-1})^{k+1} = (x^{k+1})^{-1}$.

    Observe that
    \begin{align*}
        (x^{-1})^{k+1} &= (x^{-1})^k \ast x^{-1} & (\text{by statement 1})\\
        &= (x^k)^{-1} \ast x^{-1} & (\text{by hypothesis})\\
        &= (x\ast x^k)^{-1} & (\text{by Shoes and Socks})\\
        &= (x^{k+1})^{-1} & (\text{by statement 1})
    \end{align*}
    so the statement is true for $k+1$.

    Thus, by induction, we have $(x^{-1})^n = (x^n)^{-1}$ for any non-negative integer $n$.

    \item \begin{partquestions}{\roman*}
        \item The identity is $1$ since:
        \begin{itemize}
            \item $1 \times 1 = 1$;
            \item $1 \times (-1) = (-1) \times 1 = -1$;
            \item $1 \times i = i \times 1 = i$; and
            \item $1 \times (-i) = (-i) \times 1 = -i$.
        \end{itemize}
        \item The order of the identity $1$ is 1, so we look at the other elements:
        \begin{itemize}
            \item $|-1| = 2$ since $-1 \neq 1$ and $(-1)^2 = -1 \times -1 = 1$.
            \item $|i| = 4$ since $i \neq 1$, $i^2 = -1 \neq 1$, $i^3 = -i \neq 1$, but $i^4 = 1$.
            \item $|-i| = 4$ since $-i \neq 1$, $(-i)^2 = -1 \neq 1$, $(-i)^3 = i \neq 1$, but $(-i)^4 = 1$.
        \end{itemize}
    \end{partquestions}

    \item $-i$ is the other generator since $(-i)^1 = -i$, $(-i)^2 = -1$, $(-i)^3 = i$, and $(-i)^4 = 1$.

    \item We work slowly:
    \begin{align*}
        rsr^4sr^3 &= r(sr^4)(sr^3)\\
        &= r(r^2s)(r^3s)\\
        &= r^3sr^3s\\
        &= r^3(sr^3)s\\
        &= r^3(r^3s)s\\
        &= r^6s^2\\
        &= e
    \end{align*}
\end{questions}

\subsection*{Problems}
\begin{questions}
    \item The group table of $D_4$ is given as follows.
    \begin{table}[H]
        \centering
        \begin{tabular}{|l|l|l|l|l|l|l|l|l|}
        \hline
        $\ast$ & $e$    & $r$    & $r^2$  & $r^3$  & $s$    & $rs$   & $r^2s$ & $r^3s$ \\ \hline
        $e$    & $e$    & $r$    & $r^2$  & $r^3$  & $s$    & $rs$   & $r^2s$ & $r^3s$ \\ \hline
        $r$    & $r$    & $r^2$  & $r^3$  & $e$    & $rs$   & $r^2s$ & $r^3s$ & $s$    \\ \hline
        $r^2$  & $r^2$  & $r^3$  & $e$    & $r$    & $r^2s$ & $r^3s$ & $s$    & $rs$   \\ \hline
        $r^3$  & $r^3$  & $e$    & $r$    & $r^2$  & $r^3s$ & $s$    & $rs$   & $r^2s$ \\ \hline
        $s$    & $s$    & $r^3s$ & $r^2s$ & $rs$   & $e$    & $r^3$  & $r^2$  & $r$    \\ \hline
        $rs$   & $rs$   & $s$    & $r^3s$ & $r^2s$ & $r$    & $e$    & $r^3$  & $r^2$  \\ \hline
        $r^2s$ & $r^2s$ & $rs$   & $s$    & $r^3s$ & $r^2$  & $r$    & $e$    & $r^3$  \\ \hline
        $r^3s$ & $r^3s$ & $r^2s$ & $rs$   & $s$    & $r^3$  & $r^2$  & $r$    & $e$    \\ \hline
        \end{tabular}
    \end{table}
    \begin{partquestions}{\alph*}
        \item $D_4$ is not abelian because $rs \neq sr = r^3s$.
        \item We simplify $r^3srsr^3sr^3sr^2$.
        \begin{align*}
            r^3 sr sr^3 sr^3 sr^2 &= r^3srs(r^3s)(r^3s)r^2\\
            &= r^3 srs(e)r^2\\
            &= r^3 sr sr^2\\
            &= r^2(rs rs)r^2\\
            &= r^2(e)r^2\\
            &= r^4\\
            &= e
        \end{align*}
    \end{partquestions}

    \item We need to prove each of the group axioms in order to prove that $(\Q, +)$ is indeed a group.
    \begin{itemize}
        \item \textbf{Closure}: Let $\frac ab$ and $\frac cd$ be rational numbers where $b, d \neq 0$. Their sum is $\frac{ad+bc}{bd}$, which is also rational. Therefore $\Q$ is closed under addition.

        \item \textbf{Associativity}: Addition is associative by \myref{axiom-addition-is-associative}.

        \item \textbf{Identity}: 0 is the identity since
        \[
            0 + \frac ab = \frac ab + 0 = \frac ab
        \]
        for any rational number $\frac ab$ (with $b \neq 0$).

        \item \textbf{Inverse}: For any rational number $\frac ab$, its inverse is $-\frac ab$ since
        \[
            \frac ab + \left(-\frac ab\right) = \left(-\frac ab\right) + \frac ab = 0
        \]
        for any rational number $\frac ab$ (with $b \neq 0$).
    \end{itemize}
    Furthermore addition is assumed to be commutative by \myref{axiom-addition-is-commutative}. Therefore $(\Q, +)$ is an abelian group.

    \item If every element in $G$ is its own inverse, then for every element $g$ in $G$, $g^{-1} = g$. Consider $(gh)^{-1}$ where $g$ and $h$ are elements in $g$. On one hand, by Shoes and Socks, $(gh)^{-1} = h^{-1}g^{-1} = hg$ since each element is its own inverse. On the other hand, since $gh$ is an element in $G$, thus $(gh)^{-1} = gh$. Thus $gh = hg$ which means $G$ is abelian.

    \item Recall that $n = |x|$ is the smallest positive integer that satisfies $x^n = e$.

    We prove the forward direction first. Suppose $m$ is a multiple of $n$, say $m = qn$ for some integer $q$. Then
    \[
        x^m = x^{qn} = \left(x^n\right)^q = e^q = e
    \]
    which means $x^m = e$.

    We now prove the reverse direction. Suppose $x^m = e$. Using Euclid's division lemma (\myref{lemma-euclid-division}), we write $m = qn + r$ where $q$ and $r$ are integers with $0 \leq r < n$. Hence
    \[
        x^m = x^{qn + r} = x^{qn}x^r = \left(x^n\right)^qx^r = e^qx^r = x^r.
    \]
    Note that for all integers $k$ where $1 \leq k < n$, we have $x^k \neq e$ since $n$ is the smallest positive integer such that $x^n = e$. Hence, if $x^r = e$, we conclude $r = 0$. Therefore $m = qn$, meaning $m$ is a multiple of $n$.

    \item \begin{partquestions}{\alph*}
        \item Note that $(gh)^2 = ghgh$. Given that $(gh)^2 = g^2h^2 = gghh$. By cancellation law, $hg = gh$ which means $G$ is abelian.
        \item Suppose $G$ is abelian. Clearly $(gh)^1 = gh$. Suppose $(gh)^{k} = g^kh^k$ for some positive integer $k$. Then
        \begin{align*}
            (gh)^{k+1} &= (gh)(gh)^k\\
            &= (gh)(g^kh^k) & (\text{by assumption})\\
            &= ghg^kh^k\\
            &= g(hg^k)h^k\\
            &= g(g^kh)h^k & (\text{since } G \text{ is abelian})\\
            &= gg^khh^k\\
            &= g^{k+1}h^{k+1}
        \end{align*}
        so $(gh)^{k+1} = g^{k+1}h^{k+1}$ assuming $(gh)^k = g^kh^k$. Thus the claim is proven by mathematical induction.
    \end{partquestions}

    \item Note that $|1| = n$ since $1^2 = 1 \oplus_n 1 = 2$, $1^3 = 1 \oplus_n 1 \oplus_n 1 = 3$, $1^4 = 4$, ..., $1^{n-1} = n-1$ and $1^n = 0$ which is the identity. Since the group $(\Z_n, \oplus_n)$ has an element with the same order as the group, it is thus cyclic with order $n$ and generator 1.

    \item We show that $(A, \circ)$ is a group.
    \begin{itemize}
            \item \textbf{Closure}: Function composition is closed by definition.
            \item \textbf{Associativity}: Function composition is associative.
            \item \textbf{Identity}: By performing brute-force computation, we find that $T^6(x, y) = (x, y)$. Hence $T^6$ is the identity of $A$.
            \item \textbf{Inverse}: If $r = 6$ then $T^r$ is its own inverse. Otherwise, $T^{6-r}$ is the inverse of $T^r$.
    \end{itemize}
    Thus, $(A, \circ)$ is a group, with order 6.
\end{questions}

\section{Subgroups}
\begin{questions}
    \item We will prove this claim by using the 3 axioms. For brevity let $H = \{e\}$. Clearly $H \subseteq G$.
    \begin{itemize}
        \item The only element in $H$ is $e$, and $e \ast e = e \in H$. Hence $H$ is closed.
        \item The identity of the group $G$ is $e$ which is in $H$.
        \item The inverse of $e$ is $e$ which is in $H$.
    \end{itemize}
    Hence, $\{e\} \leq G$.

    \item Clearly $e$ is in $S$ since $e \in H$ and $geg^{-1} = gg^{-1} = e$, so $S$ is non-empty and $S \subseteq G$.

    Now suppose $x$ and $y$ are in $S$. Then there exist elements $h_x$ and $h_y$ in $H$ such that $x = gh_xg^{-1}$ and $y = gh_yg^{-1}$. Note that
    \begin{align*}
        xy^{-1} &= (gh_xg^{-1})(gh_yg^{-1})^{-1}\\
        &= (gh_xg^{-1})(g{h_y}^{-1}g^{-1}) & (\text{Shoes and Shocks})\\
        &= gh_xg^{-1}g{h_y}^{-1}g^{-1} & (\text{associativity})\\
        &= gh_x{h_y}^{-1}g^{-1} & (g^{-1}g = e).
    \end{align*}
    Note that since $H \leq G$, thus $h_x{h_y}^{-1} \in H$. Hence $xy^{-1} = g(h_x{h_y}^{-1})g^{-1} \in S$. By subgroup test, $S \leq G$.

    \item \begin{partquestions}{\alph*}
        \item Since $\oplus_8$ is commutative, thus $gH = Hg$.\newline
        (Actually, since $G$ is an additive group, the better thing to write is $g \oplus_8 H = H \oplus_8 g$.)
        \item There are 4 distinct left cosets of $H$ in $G$.
        \begin{itemize}
            \item $0 \oplus_8 H = \{0, 4\} = H$
            \item $1 \oplus_8 H = \{1, 5\}$
            \item $2 \oplus_8 H = \{2, 6\}$
            \item $3 \oplus_8 H = \{3, 7\}$
        \end{itemize}
    \end{partquestions}

    \item Let $x$ be in $g_1H \cap g_2H$. Then $x \in g_1H$ and $x \in g_2H$ simultaneously. Hence, $x = g_1h = g_2\hat{h}$ for some $h, \hat{h} \in H$. Thus, by rearrangement, $g_2^{-1}g_1 = \hat{h}h^{-1} \in H$. By Coset Equality (\myref{lemma-coset-equality}), statements 1 and 5, $g_1H = g_2H$.

    \item Note that $|G| = 99$ and $|H| = 3$, so $[G:H] = \frac{99}{3} = 33$ by Lagrange's theorem.

    \item Let $x \in G$ with $x \neq e$. Then $|x| > 1$. By \myref{corollary-order-of-group-multiple-of-order-of-element}, the order of $x$ is a factor of $|G| = p$. Since $p$ is prime, $|x| = 1$ (which is not possible) or $|x| = p$. Hence $|x| = p$.
    
    \item \begin{partquestions}{\roman*}
        \item By \myref{prop-subgroup-of-abelian-group-is-normal} every subgroup of $G$ is normal. Hence $H$ is a normal subgroup of $G$, meaning $G/H$ is a quotient group by \myref{thrm-quotient-group-requirement}.
        \item Let $g$ be the generator of $G$. Consider $xH \in G/H$. Since $x \in G$ and $G$ is cyclic, thus there exists an integer $k$ such that $x = g^k$. Hence, $xH = g^kH = (gH)^k$ which means that $gH$ generates any element in $G/H$. Therefore $gH$ is a generator of $G/H$, meaning $G/H$ is cyclic.
    \end{partquestions}
\end{questions}

\section{Homomorphisms and Isomorphisms}
\begin{questions}
    \item \begin{partquestions}{\alph*}
        \item No, since $\phi(m+n) = m + n$ while $\phi(m)\phi(n) = mn \neq m+n$.
        \item Yes, since $\phi(m+n) = 2^{m+n} = 2^m2^n = \phi(m)\phi(n)$.
    \end{partquestions}

    \item Clearly $e_2 \in \phi(H_1)$ since $e_2 = \phi(e_1)$ and $e_1 \in H_1$. Now suppose $x$ and $y$ are in $\phi(H_1)$, meaning that $\phi(h_x) = x$ and $\phi(h_y) = y$ for some $h_x$ and $h_y$ in $H$. So $h_xh_y^{-1}$ is in $H$. Furthermore,
    \begin{align*}
        \phi(h_xh_y^{-1}) &= \phi(h_x)\phi(h_y^{-1})\\
        &= \phi(h_x)\left(\phi(h_y)\right)^{-1}\\
        &= xy^{-1},
    \end{align*}
    meaning that $xy^{-1}$ is in $\phi(H_1)$. Therefore, by subgroup test, $\phi(H_1) \leq G_2$.

    \item Disprove. Let $G_1 = H_1 = \mathbb{Z}$ be the additive group of integers and let $G_2 = H_2 = D_n$, the dihedral group of order $2n$. Consider the map $\phi: G_1 \to G_2$ where $\phi(m) = s^m$. Clearly, $H_1 \unlhd G_1$. Note that $\phi(H_1) = \{e, s\} = \langle s \rangle$. From \myref{example-normal-subgroups-of-d3}, we know that $\langle s \rangle$ is not a normal subgroup of $D_3 = G_2$, so $\phi(H_1)$ is not a normal subgroup of $G_2$.

    \item Suppose $|a| = n$. Note that
    \[
        \left(\phi(a)\right)^n = \phi\left(a^n\right) = \phi(e_G) = e_H
    \]
    so $|\phi(a)|$ divides $n = |a|$ by \myref{problem-element-to-power-of-multiple-of-order-is-identity}.

    \item \begin{partquestions}{\roman*}
        \item Since $3^0 = 1$, $3^1 = 3$, $3^2 = 9 \equiv 4 \pmod{5}$, and $3^3 = 27 \equiv 2 \pmod{5}$, thus $G = \langle 3 \rangle$. Since $7^0 = 1$, $7^1 = 7$, $7^2 = 49 \equiv 9 \pmod{10}$, and $7^3 = 343 \equiv 3 \pmod{10}$, thus $H = \langle 7 \rangle$.
        \item We need to prove that it is a homomorphic bijection.
        \begin{itemize}
            \item \textbf{Homomorphism}:
            \begin{align*}
                \phi(3^m3^n) &= \phi(3^{m+n})\\
                &= 7^{m+n}\\
                &= 7^m7^n\\
                &= \phi(3^m)\phi(3^n)
            \end{align*}

            \item \textbf{Bijection}: Note that $1 \mapsto 1$, $3 \mapsto 7$, $4 \mapsto 9$, $2 \mapsto 3$ which clearly shows that $\phi$ is bijective.
        \end{itemize}
        Therefore $\phi$ is an isomorphism, meaning $G \cong H$.
    \end{partquestions}

    \item Suppose $N \unlhd G$ such that $|N| = k$. Then $\phi(N)$ is a subgroup of $H$ with order $k$ by the theorem. All that remains to prove is that $\phi(N)$ is normal.

    Let $n \in N$ and $\hat{n} \in \phi(N)$ such that $\hat{n} = \phi(n)$. Let $h \in H$ be an arbitrary element. To prove that $h\hat{n}h^{-1}$ is in $\phi(N)$.

    Let $g$ be in $G$ such that $\phi(g) = h$. Then
    \begin{align*}
        h\hat{n}h^{-1} &= \phi(g)\phi(n)\phi(g^{-1})\\
        &= \phi(\underbrace{gng^{-1}}_{\text{In } N})\\
        &\in \phi(N)
    \end{align*}
    which proves that $\phi(N)$ is normal. Hence there exists a normal subgroup of order $k$, namely $\phi(N)$.

    \item Since $7^0 = 1$, $7^1 = 7$, $7^2 = 49 \equiv 9 \pmod{10}$, and $7^3 = 343 \equiv 3 \pmod{10}$, thus $G = \langle 7 \rangle$. Note $|7| = 4$ so $G \cong \mathbb{Z}_4$, i.e. $n = 4$.
\end{questions}
\section{Symmetric Groups}
\begin{questions}
    \item \begin{partquestions}{\alph*}
        \item $\begin{pmatrix}1 & 2 & 3\end{pmatrix}$
        \item $\begin{pmatrix}1 & 3\end{pmatrix}$
        \item $\begin{pmatrix}1 & 3\end{pmatrix}\begin{pmatrix}2 & 4 & 5\end{pmatrix}$
    \end{partquestions}

    \item This exercise can be solved in two ways.
    \begin{enumerate}
        \item Notice that
        \[
            \pi = \begin{pmatrix}1 & 5 & 2\end{pmatrix}\begin{pmatrix}2 & 5 & 3 & 4\end{pmatrix} = \begin{pmatrix}1 & 5 & 3 & 4 \end{pmatrix}
        \]
        and so $\pi^{-1} = \begin{pmatrix}4 & 3 & 5 & 1\end{pmatrix} = \begin{pmatrix}1 & 4 & 3 & 5\end{pmatrix}$.
        \item Using Shoes and Socks,
        \begin{align*}
            \pi^{-1} &= \left(\begin{pmatrix}1 & 5 & 2\end{pmatrix}\begin{pmatrix}2 & 5 & 3 & 4\end{pmatrix}\right)^{-1}\\
            &= \begin{pmatrix}2 & 5 & 3 & 4\end{pmatrix}^{-1} \begin{pmatrix}1 & 5 & 2\end{pmatrix}^{-1}.
        \end{align*}
        Now, $\begin{pmatrix}2 & 5 & 3 & 4\end{pmatrix}^{-1} = \begin{pmatrix}4 & 3 & 5 & 2\end{pmatrix} = \begin{pmatrix}2 & 4 & 3 & 5\end{pmatrix}$ and $\begin{pmatrix}1 & 5 & 2\end{pmatrix}^{-1} = \begin{pmatrix}2 & 5 & 1\end{pmatrix} = \begin{pmatrix}1 & 2 & 5\end{pmatrix}$. Therefore
        \[
            \pi^{-1} = \begin{pmatrix}2 & 4 & 3 & 5\end{pmatrix}\begin{pmatrix}1 & 2 & 5\end{pmatrix} = \begin{pmatrix}1 & 4 & 3 & 5\end{pmatrix}.
        \]
    \end{enumerate}

    \item To see why, consider the fact that elements of $\Sn{n}$ are permutations. Each element is only able to permute the elements of the set $X = \{1, 2, 3, \dots, n\}$. For a set of $n$ elements, there are $n!$ permutations. Thus, $|\Sn{n}| = n!$ since $\Sn{n}$ is the set of all permutations of $n$ letters.
\end{questions}

\section{Direct Products of Groups}
\begin{questions}
    \item We work component-wise:
    \begin{align*}
        (s, rs)(r^2s, r^3) &= (sr^2s, rsr^3)\\
        &= (s(r^2s), r(sr^3))\\
        &= (s(sr), r(rs))\\
        &= ((ss)r, (rr)s)\\
        &= (r, r^2s)
    \end{align*}

    \item Note that $180 = 2^2 \times 3^2 \times 5$. By \myref{thrm-Zm-cross-Zn-isomorphic-to-Zmn-condition}, we must have $mn = 180$ and $\gcd(m, n) = 1$. Thus, the valid pairs of $(m,n)$ are (4, 45), (5, 36), and (9, 20).

    \item Note that $5 \otimes_{12} 7 = 11$. Hence $GH = \{1, 5, 7, 11\}$.

    \item From above exercise, $GH = \mathcal{S}$. Now $G = \langle 5 \rangle \cong \mathbb{Z}_2$ and $H \langle 7 \rangle \cong \mathbb{Z}_2$. Thus, $\mathcal{S} = GH = \cong G \times H \cong \mathbb{Z}_2 \times \mathbb{Z}_2 = (\mathbb{Z}_2)^2$, meaning $n = 2$.
\end{questions}

\section{Further Properties of Homomorphisms}
\begin{questions}
    \item $\phi$ is a homomorphism since
    \begin{align*}
        \phi(a \oplus_3 b) &= 2(a\oplus_3 b)\\
        &= (2a) \oplus_6 (2b)\\
        &= \phi(a) \oplus_6 \phi(b).
    \end{align*}
    The image is $\{0, 2, 4\}$.

    \item $\phi$ is a homomorphism since
    \begin{align*}
        \phi(a+b) &= i^{a+b}\\
        &=i^ai^b\\
        &=\phi(a)\phi(b).
    \end{align*}
    The kernel is the set of values which map to the identity of $H$, i.e. $\{n \in \mathbb{Z} \vert \phi(n) = 1\}$. Now note $H$ is a cyclic group and $|i| = 4$. Thus $i^4 = 1$. Furthermore $i^8 = (i^4)^2 = 1, i^{12} = 1, \dots, i^{4k} = 1$. Thus $\ker\phi = \{4n \vert n \in \mathbb{Z}\} = 4\mathbb{Z}$ (using coset notation).

    \item We prove the forward direction first. Suppose that $\phi$ is injective. Clearly $\phi(e_G) = e_H$. Let $x$ be an element which is in the kernel of $\phi$, meaning $\phi(x) = e_H$. Then, $\phi(x) = \phi(e_G) = e_H$ which means $x = e_G$ by injectivity of $\phi$. Hence the kernel is trivial.

    Now we prove the reverse direction. Suppose the kernel of $\phi$ is trivial, i.e. $\ker \phi = \{e_G\}$. Suppose now there exists elements $x$ and $y$ in $G$ such that $\phi(x) = \phi(y)$. This means that $(\phi(x))^{-1} = \phi(x^{-1}) = \phi(y^{-1}) = (\phi(y))^{-1}$. Hence,
    \[
        \phi(xy^{-1})
        = \phi(x)\phi(y^{-1})
        = \phi(x)\left(\phi(y)\right)^{-1}
        = e_H.
    \]
    Now since the kernel is trivial, this must mean that $xy^{-1} = e_G$ which immediately leads $x=y$. Hence $\phi$ is injective.

    \item Recall that $G / \ker \phi \cong \im \phi$ by the Fundamental Homomorphism Theorem (\myref{thrm-isomorphism-1}). Furthermore, we note that $|G / \ker \phi| = \frac{|G|}{|\ker\phi|}$ by Lagrange's Theorem (\myref{thrm-lagrange}). Hence, $\frac{|G|}{|\ker\phi|} = |\im\phi|$ which leads to the result quickly.

    \item The Diamond Isomorphism Theorem (\myref{thrm-isomorphism-2}), statement 6, states that $H / (H\cap N) \cong HN / N$. Taking orders on both sides yields $\frac{|H|}{|H \cap N|} = \frac{|HN|}{|N|}$. Rearranging yields required result.

    \item \begin{partquestions}{\roman*}
        \item Note $H = x\mathbb{Z} = \{ax \vert a \in \mathbb{Z}\}$ and $N = mx\mathbb{Z} = \{a(mx) \vert a \in \mathbb{Z}\}$, which necessarily means $N \subseteq H$.
        \item Let $G = \mathbb{Z}$. Then both $H$ and $N$ are clearly subgroups of $G$. Now since $G$ is abelian (since addition is commutative), therefore $H$ and $N$ are normal by \myref{prop-subgroup-of-abelian-group-is-normal}.
        \item The Third Isomorphism Theorem (\myref{thrm-isomorphism-3}) tells us that
        \[
            (G/N)/(H/N) \cong G/H.
        \]
        Now by \myref{problem-Zn-isomorphic-to-Z-by-nZ} we have $|G/H| = |\mathbb{Z}/(x\mathbb{Z})| = x$ and $|G/N| = |\mathbb{Z}/(y\mathbb{Z})| = y$. Hence
        \[
            \frac{x}{|H/N|} = y
        \]
        which quickly implies $|H/N| = \frac yx$.
    \end{partquestions}
\end{questions}

\section{More Types of Groups}
\subsection*{Exercises}
\begin{questions}
    \item Let $G = \Z_{mn}$ and $H = \{0, n, 2n, \dots, (m-1)n\}$. Clearly $H$ is a subgroup of $G$ of order $m$. By \myref{problem-subgroup-of-cyclic-group-is-cyclic} we know $H$ is normal and cyclic with order $m$ and by \myref{exercise-quotient-group-of-cyclic-group-is-cyclic} we know $G/H$ is cyclic. The order of $G/H$ is $\frac{|G|}{|H|} = \frac{mn}{m} = n$ by Lagrange's theorem (\myref{thrm-lagrange}), meaning that $G/H \cong \Z_n$. Hence, $\Z_{mn}/\Z_m \cong G/H \cong \Z_n$.

    \item Note 0 is the identity in $\Z_n$. By \myref{lemma-order-of-an-element-that-is-equivalent-to-identity} we know that if $12$ is equivalent to the identity in $\Z_n$, then $12 = mn$ for some integer $m$. Since $n > 0$ we restrict $m$ to positive integers. Now $12 = 2^2 \times 3$. Thus the possible values of $n$ are
    \begin{itemize}
        \item $n = 1$ with $m = 12$;
        \item $n = 2$ with $m = 6$;
        \item $n = 3$ with $m = 4$;
        \item $n = 4$ with $m = 3$;
        \item $n = 6$ with $m = 2$; and
        \item $n = 12$ with $m = 1$.
    \end{itemize}

    \item $|10| = \frac{210}{\gcd(10, 210)} = \frac{210}{10} = 21$, $|42| = \frac{210}{\gcd(42, 210)} = \frac{210}{42} = 5$, $|75| = \frac{210}{\gcd(75, 210)} = \frac{210}{15} = 14$, and $|140| = \frac{210}{\gcd(140, 210)} = \frac{210}{70} = 3$.

    \item \begin{partquestions}{\alph*}
        \item Note that $10 = 2 \times 5$. Generators of the group $\Z_{10}$ has to satisfy $\gcd(m, 10) = 1$ by \myref{corollary-element-in-cyclic-group-is-generator-iff-gcd-is-1}. The positive integers that satisfy this requirement (and which are less than 10) are 1, 3, 7, 9. Thus they are the generators of $\Z_{10}$.
        \item Note that 101 is prime. Hence all positive integers from 1 to 100 (inclusive) are generators. (Note that 0 is not a generator of $\Z_{101}$ since 0 is the identity.)
    \end{partquestions}

    \item We show that all subgroups of $\mathrm{Q}$ are, in fact, normal. We consider the first definition of the quaternion group.
    \begin{itemize}
        \item Clearly $\{1\} \lhd \mathrm{Q}$ and $\mathrm{Q} \unlhd \mathrm{Q}$.
        \item The subgroups $\langle i\rangle$, $\langle j\rangle$, and $\langle k\rangle$ have order 4. Therefore, Lagrange's theorem (\myref{thrm-lagrange}) tells us that they have index 2. Hence these subgroups are normal by \myref{problem-subgroup-of-index-2}.
        \item Consider the subgroup $\langle -1 \rangle = \{1, -1\}$. \begin{itemize}
            \item $1\langle -1 \rangle = \langle -1 \rangle1$, since 1 is the identity;
            \item $-1\langle -1 \rangle = \{1, -1\} = \langle -1 \rangle(-1)$;
            \item $i\langle -1 \rangle = \{-i, i\} = \langle -1 \rangle i$;
            \item $-i\langle -1 \rangle = \{i, -i\} = \langle -1 \rangle (-i)$;
            \item $j\langle -1 \rangle = \{-j, j\} = \langle -1 \rangle j$;
            \item $-j\langle -1 \rangle = \{j, -j\} = \langle -1 \rangle (-j)$;
            \item $k\langle -1 \rangle = \{-k, k\} = \langle -1 \rangle k$; and
            \item $-k\langle -1 \rangle = \{k, -k\} = \langle -1 \rangle (-k)$.
        \end{itemize}
        Thus $\langle -1 \rangle$ is normal.
    \end{itemize}
    Hence all subgroups of $\mathrm{Q}$ are normal.

    \item $\begin{pmatrix}2&6\end{pmatrix} = \begin{pmatrix}2&3\end{pmatrix}\begin{pmatrix}3&4\end{pmatrix}\begin{pmatrix}4&5\end{pmatrix}\begin{pmatrix}5&6\end{pmatrix}\begin{pmatrix}4&5\end{pmatrix}\begin{pmatrix}3&4\end{pmatrix}\begin{pmatrix}2&3\end{pmatrix}$.

    \item Note that $\begin{pmatrix}1&3&2&5&4\end{pmatrix} = \begin{pmatrix}1&4\end{pmatrix}\begin{pmatrix}1&5\end{pmatrix}\begin{pmatrix}1&2\end{pmatrix}\begin{pmatrix}1&3\end{pmatrix}$. \myref{thrm-parity-of-permutation} tells us that $\begin{pmatrix}1&3&2&5&4\end{pmatrix}$ is even and thus has a sign of $+1$.

    \item Note that $\An{3}$ has order $\frac{3!}{2} = 3$ so we should expect 3 permutations. Clearly the identity is one such permutation. Looking at \myref{example-symmetric-group-of-degree-3} we can find two more, namely $\begin{pmatrix}1&2&3\end{pmatrix}$ and $\begin{pmatrix}1&3&2\end{pmatrix}$.
    
    For $\An{4}$, note that it has order $\frac{4!}{2} = 12$ so we expect 12 permutations. Again the identity is one of them. Like $\An{3}$ we now find the 3-cycles in $\An{4}$, which are $\begin{pmatrix}1&2&3\end{pmatrix}$, $\begin{pmatrix}1&2&4\end{pmatrix}$, $\begin{pmatrix}1&3&2\end{pmatrix}$, $\begin{pmatrix}1&3&4\end{pmatrix}$, \linebreak $\begin{pmatrix}1&4&2\end{pmatrix}$, $\begin{pmatrix}1&4&3\end{pmatrix}$, $\begin{pmatrix}2&3&4\end{pmatrix}$, and $\begin{pmatrix}2&4&3\end{pmatrix}$. So there are 3 more permutations unaccounted, which are permutations of products of 2-cycles: $\begin{pmatrix}1&2\end{pmatrix}\begin{pmatrix}3&4\end{pmatrix}$, $\begin{pmatrix}1&3\end{pmatrix}\begin{pmatrix}2&4\end{pmatrix}$, and $\begin{pmatrix}1&4\end{pmatrix}\begin{pmatrix}2&3\end{pmatrix}$.

    \item $\Un{10} = \{1, 3, 7, 9\}$.

    \item By a corollary of Lagrange's theorem (\myref{corollary-order-of-group-multiple-of-order-of-element}), the order of $a$ dives the order of the group $\Un{n}$. Now since $|\Un{n}| = \totient(n)$, thus the order of $a$ divides $\totient(n)$.

    \item $\begin{pmatrix}2&1&2\\1&0&1\\2&1&2\end{pmatrix}$

    \item We already proved that $\Inn{G} \leq \Aut{G}$ so we only need to prove normality.

    Let $\phi \in \Aut{G}$ and $\iota_g \in \Inn{G}$. For brevity let $f = \phi\iota_g\phi^{-1}$. We note that $f \in \Aut{G}$; we need to prove that $f \in \Inn{G}$.

    Suppose $x \in G$. Since $\phi$ is an isomorphism, there exists $w \in G$ such that $x = \phi(w)$, i.e. $w = \phi^{-1}(x)$. So
    \begin{align*}
        f(x) &= \left(\phi\iota_g\phi^{-1}\right)(x)\\
        &= \phi(\iota_g(\phi^{-1}(x)))\\
        &= \phi(\iota_g(w))\\
        &= \phi(gwg^{-1})\\
        &= \phi(g)\phi(w)\phi(g^{-1})\\
        &= \phi(g)x\left(\phi(g)\right)^{-1}
    \end{align*}
    which shows that $f \in \Inn{G}$. Hence, $\Inn{G} \unlhd \Aut{G}$.
\end{questions}

\subsection*{Problems}
\begin{questions}
    \item We note that the two questions are equivalent to finding the orders of 3774 and 1870 in the group $\Z_{10101}$. We note that
    \begin{align*}
        1870 &= 2 \times 5 \times 11 \times 17,\\
        3774 &= 2 \times 3 \times 17 \times 37, \text{ and}\\
        10101 &= 3 \times 7 \times 13 \times 37.
    \end{align*}
    Therefore, $\gcd(1870, 10101) = 1$ and $\gcd(3774, 10101) = 3 \times 37 = 111$. Hence $|1870| = 10101$ and $|3774| = \frac{10101}{111} = 91$. Therefore, $a = 10101$ and $b = 91$.

    \item We claim that $\An{n}$ is non-abelian for any $n > 3$. Note that both $\pi = \begin{pmatrix}1 & 2 & 3\end{pmatrix}$ and $\sigma = \begin{pmatrix}2 & 3 & 4\end{pmatrix}$ are even permutations, and hence are in $\An{n}$ for any $n > 3$. We note
    \begin{itemize}
        \item $\pi\sigma = \begin{pmatrix}1 & 2 & 3\end{pmatrix}\begin{pmatrix}2 & 3 & 4\end{pmatrix} = \begin{pmatrix}1 & 2\end{pmatrix}\begin{pmatrix}3 & 4\end{pmatrix}$; and
        \item $\sigma\pi = \begin{pmatrix}2 & 3 & 4\end{pmatrix}\begin{pmatrix}1 & 2 & 3\end{pmatrix} = \begin{pmatrix}1 & 3\end{pmatrix}\begin{pmatrix}2 & 4\end{pmatrix}$.
    \end{itemize}
    So $\pi\sigma \neq \sigma\pi$. Thus $\An{n}$ is non-abelian for any $n > 3$.

    We note that
    \begin{itemize}
        \item $\An{2}$ has order 1 so $\An{2}$ is the trivial group, which is abelian (and cyclic); and
        \item $\An{3}$ has order 3 so $\An{3}$ is cyclic and thus abelian.
    \end{itemize}
    Thus the largest integer $n$ for which $\An{n}$ is abelian is $n = 3$. Furthermore $\An{k}$ is cyclic if $k = 2$ or $k = 3$.

    \item We first note that
    \[
        \totient(2p^k) = 2p^k\left(1-\frac12\right)\left(1-\frac1p\right) = p^k\left(1-\frac1p\right) = \totient(p^k).
    \]
    Now we are given that $r$ is an odd primitive root of $p^k$. Since $r \in \Un{p^k}$, thus $\gcd(r, 2p^k) = 1$ because $\gcd(r, p^k) = 1$. Now as $r$ is odd, thus $r \in \Un{2p^k}$. Let $n$ be the order of $r$ in $\Un{2p^k}$. Then by \myref{exercise-order-of-a-divides-phi-a} we know $n$ divides $\totient(2p^k)$. At the same time, because $r$ is a generator in $\Un{p^k} \cong \Z_{\phi(p^k)}$, so $\totient(p^k) = \totient(2p^k)$ divides $n$ by \myref{lemma-order-of-an-element-that-is-equivalent-to-identity}. Since $n$ divides $\totient(2p^k)$ and $\totient(2p^k)$ divides $n$ simultaneously, therefore $n = \totient(2p^k) = |\Un{2p^k}|$ which means that $r$ is a primitive root modulo $2p^k$.

    \item \begin{partquestions}{\roman*}
        \item The forward direction is clearly true since if $f_1 = f_2$, then $f_1(x) = f_2(x)$ for all $x \in G$, including $g \in G$. For the reverse direction, assume $f_1(g) = f_2(g)$. Note that
        \[
            f_1(g^k) = (f_1(g))^k = (f_2(g))^k = f_2(g^k)
        \]
        for any integer $k$. Since $g$ is a generator, thus we have $f_1(x) = f_2(x)$ for all $x \in G$, meaning $f_1 = f_2$.

        \item We note $f(g) \in G$. Since $g$ is a generator, hence $f(g) = g^k$ for some integer $k$. Hence any homomorphism from $G$ to $G$ is of the form $f(g) = g^{m_f}$ where $0 \leq m_f \leq n-1$, which means $m_f \in \Z_n$.

        \item Suppose the map $f_2: G \to G$ is another homomorphism where $f_2(g) = g^{m_f}$. Then we see
        \[
            f(g) = g^{m_f} = f_2(g)
        \]
        which means $f = f_2$ by \textbf{(i)}. Hence $m_f$ is unique to $f$.

        \item Consider $f_1(f_2(g))$. On one hand,
        \[
            f_1(f_2(g)) = f_1(g^{m_{f_2}}) = (f_1(g))^{m_{f_2}} = g^{m_{f_1}m_{f_2}},
        \]
        while on the other,
        \[
            f_1(f_2(g)) = (f_1 \circ f_2)(g) = g^{m_{f_1\circ f_2}}
        \]
        by definition of $m_f$ as introduced in \textbf{(ii)}. Therefore $m_{f_1\circ f_2} \equiv m_{f_1}m_{f_2} \pmod n$. In other words, $m_{f_1\circ f_2} = m_{f_1} \otimes_n m_{f_2}$.

        \item We prove the forward direction first by assuming that the map $f$ is an automorphism. Hence $f$ is surjective, meaning that there exists $a \in G$ such that $f(a) = g$. Since $a \in G$ thus $a = g^k$ for some $k \in \Z_n$ (we will show $k \in \Un{n}$ later). Observe
        \[
            g = f(a) = f(g^k) = (f(g))^k = g^{m_fk}
        \]
        which means $m_fk \equiv 1 \pmod n$. By \myref{prop-multiplicative-inverse-exists-iff-coprime}, this means that we have $\gcd(m_f, n) = 1$ and $\gcd(k, n) = 1$. Therefore, $m_f$ and $k$ are in $\Un{n}$. Hence $k$ is the multiplicative inverse of $m_f$.

        We now prove the reverse direction. Assume $m_f$ has a multiplicative inverse (say $k$), meaning $m_fk \equiv 1 \pmod n$. As above this means that both $m_f$ and $k$ are in $\Un{n}$. We show that $f$ is a bijection.
        \begin{itemize}
            \item \textbf{Injective}: Suppose $x, y \in G$ such that $f(x) = f(y)$. Since $g$ is a generator we may take $x = g^p$ and $y = g^q$ for some integers $p$ and $q$. Hence we have $g^{m_fp} = g^{m_fq}$. Then
            \[
                \left(g^{m_fp}\right)^k = g^{km_fp} = \left(g^{km_f}\right)^p = g^p
            \]
            and $\left(g^{m_fq}\right)^k = g^q$. Hence this implies $g^p = g^q$ which means $x = y$.
            \item \textbf{Surjective}: Suppose $x \in G$. Since $g$ is a generator we may write $x = g^p$ for some integer $p$. Then $f(g^{kp}) = g^{m_fkp} = g^p = x$.
        \end{itemize}
        Also $f$ is given to be a homomorphism. Hence $f$ is an isomorphism. Since $f: G \to G$, it is thus an automorphism.

        \item We prove that $\phi$ is an isomorphism.
        \begin{itemize}
            \item \textbf{Homomorphism}: Let $f_1, f_2 \in \Aut{G}$. Then
            \begin{align*}
                \phi(f_1\circ f_2) &= m_{f_1\circ f_2} & (\text{definition of } m_f \text{ in }\textbf{(ii)})\\
                &= m_{f_1} \otimes_n m_{f_2} & (\text{by \textbf{(iv)}})\\
                &= \phi(f_1)\otimes_n\phi(f_2),
            \end{align*}
            which means $\phi$ is a homomorphism.

            \item \textbf{Injective}: Suppose we have $f_1, f_2 \in \Aut{G}$ such that $\phi(f_1) = \phi(f_2)$. Thus $m_{f_1} = m_{f_2}$ by definition of $\phi$. However, we know that the value of $m$ uniquely defines a homomorphism from $G$ to $G$ from \textbf{(iii)}. Hence $f_1 = f_2$, which shows that $\phi$ is injective.

            \item \textbf{Surjective}: Suppose $r \in \Un{n}$. Define the map $f: G \to G$ where $f(g) = g^r$. Since $r \in \Un{n}$ it has a multiplicative inverse, which means that $f$ is an automorphism by \textbf{(v)}. Clearly $\phi(f) = r$, so $r$ has a pre-image. So $\phi$ is surjective.
        \end{itemize}
        Hence $\phi$ is an isomorphism, meaning $\Aut{G} \cong \Un{n}$.
    \end{partquestions}
\end{questions}

\section{Group Actions}
\begin{questions}
    \item We prove the two group action axioms.
    \begin{itemize}
        \item \textbf{Identity}: $\alpha(e, x) = exe^{-1} = x$.
        \item \textbf{Compatibility}: Note
        \begin{align*}
            \alpha(g, \alpha(h, x)) &= \alpha(g, hxh^{-1})\\
            &= gh x h^{-1}g^{-1}\\
            &= (gh)x(gh)^{-1}\\
            &= \alpha(gh, x).
        \end{align*}
    \end{itemize}
    Therefore $\alpha$ is a group action of $G$ on $G$.

    \item Recall there are 6 elements in $\Sn{3}$: $\id$, $\begin{pmatrix}1 & 2 & 3\end{pmatrix}$, $\begin{pmatrix}1 & 3 & 2\end{pmatrix}$, $\begin{pmatrix}1 & 2\end{pmatrix}$, $\begin{pmatrix}1 & 3\end{pmatrix}$, and $\begin{pmatrix}2 & 3\end{pmatrix}$. Clearly the identity has all elements of $X$ as fixed points. It is also clear that $\begin{pmatrix}1 & 2 & 3\end{pmatrix}$ and $\begin{pmatrix}1 & 3 & 2\end{pmatrix}$ have no fixed points since they permute all elements. For the rest, the fixed points are the missing element from the cycle notation, i.e. $\begin{pmatrix}1 & 2\end{pmatrix}$ has fixed point 3, $\begin{pmatrix}1 & 3\end{pmatrix}$ has fixed point 2, and $\begin{pmatrix}2 & 3\end{pmatrix}$ has fixed point 1.

    \item For 1, it is $\{\id, \begin{pmatrix}2 & 3\end{pmatrix}\}$. For 2, it is $\{\id, \begin{pmatrix}1 & 3\end{pmatrix}\}$. For 3, it is $\{\id, \begin{pmatrix}1 & 2\end{pmatrix}\}$.

    \item We work from the statement forwards. Note that each of these statements are ``if and only if'' statements.
    \begin{align*}
	    g \cdot x = h \cdot x &\iff g^{-1} \cdot (g \cdot x) = g^{-1} \cdot (h \cdot x)\\
	    &\iff (g^{-1}g) \cdot x = (g^{-1}h) \cdot x\\
	    &\iff e \cdot x = (g^{-1}h) \cdot x\\
	    &\iff x = (g^{-1}h) \cdot x\\
	    &\iff (g^{-1}h) \cdot x = x\\
	    &\iff g^{-1}h \in \Stab{G}{x}
	\end{align*}

	\item \begin{partquestions}{\alph*}
		\item An orbit takes the form $\Orb{G}{x}$. Clearly $e \cdot x = x$ so $x \in \Orb{G}{x}$ and thus $\Orb{G}{x}$ is non-empty.
	    \item Let $x \in X$. Since $e \cdot x = x$, so $x \in \Orb{G}{x}$.
	    \item Suppose $x \in \Orb{G}{x_1} \cap \Orb{G}{x_2}$ (as their intersection is non-empty). Then there exists $g_1, g_2 \in G$ such that $g_1\cdot x_1 = x = g_2\cdot x_2$. Thus,
	    \begin{align*}
	        x_1 &= e \cdot x_1\\
	        &= (g_1^{-1}g_1)\cdot x_1\\
	        &= g_1^{-1} \cdot (g_1 \cdot x_1)\\
	        &= g_1^{-1} \cdot (g_2 \cdot x_2)\\
	        &= (g_1^{-1}g_2) \cdot x_2.
	    \end{align*}
	    Now suppose $y \in \Orb{G}{x_1}$. Then $y = g\cdot x_1$ for some $g \in G$. Hence,
	    \begin{align*}
	        y &= g\cdot x_1 \\
	        &= g \cdot \left((g_1^{-1}g_2) \cdot x_2\right)\\
	        &= (\underbrace{gg_1^{-1}g_2}_{\text{In } G})\cdot x_2\\
	        &\in \Orb{G}{x_2}
	    \end{align*}
	    which means any element in $\Orb{G}{x_1}$ is also in $\Orb{G}{x_2}$. Hence, $\Orb{G}{x_1}$ is a subset of $\Orb{G}{x_2}$. A similar argument can be used to show that $\Orb{G}{x_2}$ is a subset of $\Orb{G}{x_1}$. Hence $\Orb{G}{x_1} = \Orb{G}{x_2}$.
	\end{partquestions}

	\item We prove the forward direction first: suppose the action is transitive. Then there exists $x \in X$ such that $\Orb{G}{x} = X$. Now consider any other element $y \in X$. Since the action is transitive, this means that there exists a $\hat{g} \in G$ such that $\hat{g} \cdot x = y$. Note that $\Orb{G}{y} = \Orb{G}{\hat{g} \cdot x}$, and that $\Orb{G}{x} = \{g \cdot x \vert g \in G\}$. Hence,
	\[
        \Orb{G}{\hat{g} \cdot x} = \{g\cdot (\hat{g} \cdot x) \vert g \in G\} = \{(g\hat{g}) \cdot x \vert g \in G\}.
	\]
	Since $G$ is a group, $g\hat{g} \in G$. In particular, we may pick $g = g'\hat{g}^{-1}$ to obtain any arbitrary element $g' \in G$. Thus, this means that
	\[
        	\{(g\hat{g}) \cdot x \vert g \in G\} = \{g' \cdot x \vert g' \in G \} = \Orb{G}{x} = X.
	\]
	Hence, for any element $y \in X$, $\Orb{G}{y} = \Orb{G}{g \cdot x} = X$.

	The reverse direction is trivial: suppose $\Orb{G}{x} = X$ for all $x \in X$. Then certainly there exists an element $x \in X$ such that $\Orb{G}{x} = X$, meaning that the group action is transitive.

	\item \begin{partquestions}{\alph*}
	    \item Consider $x = n$. The orbit of $n$ is all of $X$. Consider the permutation $\sigma = \begin{pmatrix}k & n\end{pmatrix}$ where $1 \leq k \leq n$. Clearly $\sigma \in \Sn{n}$. Note that $\sigma \cdot n = \sigma(n) = k$. Thus, $\Orb{G}{n} = X$, meaning that the group action ``$\cdot$'' given by $g \cdot x \mapsto g(x)$ is transitive.
	    \item Note that $|X| = n$ and $|\Sn{n}| = n!$. By Orbit-Stabilizer theorem (\myref{thrm-orbit-stabilizer}), the stabilizer of $x$ by $G$ must have order $\frac{n!}{n} = (n-1)!$.
	\end{partquestions}

	\item By the Orbit-Stabilizer theorem (\myref{thrm-orbit-stabilizer}),
	\[
        |\Orb{G}{x}| = \frac{|G|}{|\Stab{G}{x}|} = [G : \Stab{G}{x}]	.
	\]
	Under the group action of conjugation, $\Orb{G}{x} = \Cl{x}$ and $\Stab{G}{x} = \Centralizer{G}{x}$. Hence, $|\Cl{x}| = [G : \Centralizer{G}{x}]$ as required.

	\item \begin{partquestions}{\alph*}
	    \item One sees that $\Z{D_3} = \{e\}$ based on the group table of $D_3$.
	    \item Recall that every element in $D_3$ can be expressed in the form $r^as^b$ where $a \in \{0, 1, 2\}$ and $b \in \{0, 1\}$. One finds that $\Cl{r} = \{r, r^2\}$ and $\Cl{s} = \{s, rs, rs^2\}$.
	    \item The class equation is $6 = 1 + 2 + 3$.
	\end{partquestions}

	\item By Cauchy's Theorem (\myref{thrm-cauchy}) there exists an element (say $x$) with order $p$. Consider $H = \langle x \rangle$. Note that $|H| = p$ and $H \leq G$. Hence we found a subgroup of $G$ of order $p$.
\end{questions}

\section{Sylow Theorems}
\subsection*{Exercises}
\begin{questions}
    \item We note that $12 = 2^2 \times 3$. Thus a Sylow 2-subgroup must have order 4. Clearly $|3| = 4$ so $\langle 3 \rangle = \{0, 3, 6, 9\}$ is the Sylow 2-subgroup of $\Z_{12}$.

    \item Recall that $|\Sn{5}| = 120 = 2^3 \times 3 \times 5$. By a corollary of the First Sylow Theorem (\myref{corollary-sylow-p-subgroup-exists}), $\Syl{p}{G} \neq \emptyset$ if $p$ is 2, 3, or 5.

    \item We prove this by constructing the map $\phi: H \to gHg^{-1}$ where $h \mapsto ghg^{-1}$. We note that $\phi$ is an isomorphism.
    \begin{itemize}
        \item \textbf{Homomorphism}: Let $x, y \in H$. Then
        \[
            \phi(xy) = g(xy)g^{-1} = (gxg^{-1})(gyg^{-1}) = \phi(x)\phi(y)
        \]
        which clearly means that $\phi$ is a homomorphism.
        
        \item \textbf{Injective}: Suppose $x, y \in H$ such that $\phi(x) = \phi(y)$. Then $gxg^{-1} = gyg^{-1}$ which quickly implies $x = y$.
        
        \item \textbf{Surjective}: Suppose $ghg^{-1} \in gHg^{-1}$. Clearly we have $\phi(h) = ghg^{-1}$, so any element in $gHg^{-1}$ has a pre-image inside $H$.
    \end{itemize}
    Hence $H \cong gHg^{-1}$.

    \item By \myref{prop-order-of-conjugate-element-equals-order-of-element} we know that $|xyx^{-1}| = |y|$ for all $x, y \in G$. Substituting $x = g$, and $y = hg$ yields
    \[
        |xyx^{-1}| = |g(hg)g^{-1}| = |gh| \text{ and } |y| = |hg|
    \]
    so the result follows.

    \item Clearly $e \in \N{G}{S}$ since $eSe^{-1} = S$. Consider $x, y \in \N{G}{S}$, meaning that $xSx^{-1} = S$ and $ySy^{-1} = S$. Note that $y^{-1} \in \N{G}{S}$ since
    \begin{align*}
        y^{-1}S\left(y^{-1}\right)^{-1} &= y^{-1}Sy\\
        &= y^{-1}\left(ySy^{-1}\right)y & (\text{since } y \in \N{G}{S})\\
        &= (y^{-1}y)S(y^{-1}y)\\
        &= S.
    \end{align*}
    Therefore
    \begin{align*}
        \left(xy^{-1}\right)S\left(xy^{-1}\right)^{-1} &= \left(xy^{-1}\right)S\left(yx^{-1}\right)\\
        &= x\left(y^{-1}Sy\right)x^{-1}\\
        &= xSx^{-1} & (\text{since } y^{-1} \in \N{G}{S})\\
        &= S & (\text{since } x \in \N{G}{S})
    \end{align*}
    which means that $xy^{-1} \in \N{G}{S}$. Hence, by the subgroup test, we have $\N{G}{S} \leq G$.

    \item By the Second Sylow Theorem (\myref{thrm-sylow-2}), we know that $gHg^{-1} = K$. Since $H \cong gHg^{-1}$ by \myref{exercise-conjugate-subgroup-isomorphic-to-subgroup} thus $H \cong gHg^{-1} = K$ as required.

    \item We note $784 = 2^4 \times 7^2$, so $m = 16$, $p = 7$, and $k = 2$. By the Third Sylow Theorem (\myref{thrm-sylow-3}), we know that
    \begin{itemize}
        \item $n_7 = [G : \N{G}{P}] = \frac{|G|}{|\N{G}{P}|}$;
        \item $n_7 \mid 16$, which implies $n_7 \in \{1, 2, 4, 8, 16\}$; and
        \item $n_7 \equiv 1 \pmod 7$, which implies $n_7 \in \{1, 8, 15, 22, \dots\}$.
    \end{itemize}
    Hence $n_7 = 1$ or $n_7 = 8$. But since $P$ is not a normal subgroup of $G$, by \myref{corollary-sylow-subgroup-is-normal-if-it-is-unique}, $P$ cannot be the only Sylow 7-subgroup, meaning $n_7 \neq 1$. Hence $n_7 = 8$, so
    \[
        8 = n_7 = \frac{|G|}{|\N{G}{P}|} = \frac{784}{|\N{G}{P}|}
    \]
    which means that $|\N{G}{P}| = 98$.

    \item Note $130 = 2 \times 5 \times 13$. Consider the number of Sylow 13-subgroups, $n_{13}$. The Third Sylow Theorem (\myref{thrm-sylow-3}) tells us that
    \begin{itemize}
        \item $n_{13} \mid 2 \times 5 = 10$, so $n_{13} \in \{1, 2, 5, 10\}$, and
        \item $n_{13} \equiv 1 \pmod{13}$ so $n_{13} \in \{1, 14, 27, \dots\}$.
    \end{itemize}
    Hence $n_{13} = 1$. But by \myref{corollary-sylow-subgroup-is-normal-if-it-is-unique} this means that the only Sylow 13-subgroup is normal. Hence a group of order 130 is non-simple.
\end{questions}

\subsection*{Problems}
\begin{questions}
    \item Note $200 = 2^3 \times 5^2$. Note that for $p = 5$ we have $m = 8$ and the factors of 8 are 1, 2, 4, and 8. Furthermore by the Third Sylow Theorem (\myref{thrm-sylow-3}) we must have $n_5 \equiv 1 \pmod 5$. Hence $n_5 = 1$. By a corollary of the Second Sylow Theorem (\myref{corollary-sylow-subgroup-is-normal-if-it-is-unique}) this means that the only Sylow 5-subgroup is normal.

    \item Note $33 = 3 \times 11$,
    \begin{itemize}
        \item when $p = 3$ we have $m = 11$ and the factors of 11 are 1 and 11; and
        \item when $p = 11$ we have $m = 3$ and the factors of 3 are 1 and 3.
    \end{itemize}
    The Third Sylow Theorem (\myref{thrm-sylow-3}) tells us that $n_p \equiv 1 \pmod p$. Hence we must have $n_3 = n_{11} = 1$. A corollary of the Second Sylow Theorem (\myref{corollary-sylow-subgroup-is-normal-if-it-is-unique}) tells us that the only Sylow 3-subgroup and Sylow 11-subgroup are normal.

    \item For brevity let $q = 2^p - 1$, and we are given that $q$ is a prime. By the Third Sylow Theorem (\myref{thrm-sylow-3}), $n_q \mid 2^{p-1}$ and $n_q \equiv 1 \pmod p$. The factors of $2^{p-1}$ are $1, 2, 4, 8, \dots, 2^{p-1}$. We note $2^{p-1} < 2^p - 1 = q$ for any prime $p$ since
    \[
        2^{p-1} + 1 < 2^{p-1} + 2^{p-1} = 2(2^{p-1}) = 2^p
    \]
    which result immediately follows by subtracting 1 on both sides. Hence, the only possible value that satisfies both conditions is $n_q = 1$. By a corollary of the Second Sylow Theorem (\myref{corollary-sylow-subgroup-is-normal-if-it-is-unique}) this means that the only Sylow $q$-subgroup is normal, hence showing that a group with an even perfect number order is non-simple.

    \item \begin{partquestions}{\roman*}
        \item The divisors of $p$ are 1 and $p$ itself. By the Third Sylow Theorem (\myref{thrm-sylow-3}), $n_q$ divides $p$ and $n_q \equiv 1 \pmod q$. Since $p < q$ hence $p \not\equiv 1 \pmod q$ meaning that $n_q = 1$. By a corollary of the Second Sylow Theorem (\myref{corollary-sylow-subgroup-is-normal-if-it-is-unique}) the only Sylow $q$-subgroup is normal.

        \item The divisors of $q$ are 1 and $q$ itself. By the Third Sylow Theorem (\myref{thrm-sylow-3}), $n_p$ divides $q$ and $n_p \equiv 1 \pmod p$. Since $q \not\equiv 1 \pmod p$ by assumption, we must have $n_p = 1$.

        Recall that the order of an element in a group of order $pq$ must divide $pq$ (\myref{corollary-order-of-group-multiple-of-order-of-element}). Hence the possible orders of an element in such a group are 1, $p$, $q$, or $pq$.
        \begin{itemize}
            \item There is only one element of order 1, the identity.
            \item There are $p - 1$ elements of order $p$, all belonging in the single Sylow $p$-subgroup. Note that we subtract 1 because one element in the Sylow $p$-subgroup is the identity.
            \item There are $q - 1$ elements of order $q$, all in the single Sylow $q$-subgroup.
        \end{itemize}
        Hence, since the total number of elements in a group of order $pq$ is $pq$, the number of elements of order $pq$ is
        \begin{align*}
            pq - \left((p-1)+(q-1)+1\right) &= pq - (p+q - 1)\\
            &= pq - p - q + 1\\
            &> 2q - 2 - q + 1\\
            &= 2q - q - 1\\
            &= q - 1\\
            &> 0
        \end{align*}
        which means that there is at least one element of order $pq$. By \myref{thrm-cyclic-group-has-element-with-same-order} this means that such a group is cyclic.
    \end{partquestions}

    \item \begin{partquestions}{\roman*}
        \item Let $P$ be a Sylow $p$-subgroup of $N$. Lagrange's Theorem (\myref{thrm-lagrange}) tells us that $|G| = [G:N]|N|$. Since $p$ does not divide $[G:N]$ we must have $|N| = p^ka$ where $a$ divides $m$. Hence $|P| = p^k$ as $P$ is a Sylow $p$-subgroup of $N$. Since $P$ has order $p^k$ and $P \leq N \leq G$, thus $P$ is also a Sylow $p$-subgroup of $G$.
        \item Let $Q$ be a Sylow $p$-subgroup of $G$. The Second Sylow Theorem (\myref{thrm-sylow-2}) tells us there exist  a $g \in G$ such that $Q = gPg^{-1}$. Recall by definition of normality that $gNg^{-1} = N$ for any $g \in G$. Note also that $P \leq N$. Hence,
        \[
            Q = gPg^{-1} \leq gNg^{-1} = N
        \]
        which means that $Q$ is also a Sylow $p$-subgroup of $N$.
    \end{partquestions}

    \item We note $3325 = 5^2 \times 7 \times 19$. Let the group of order 3325 be $G$. We know that
    \begin{itemize}
        \item for $p = 5$ we have $m = 7 \times 19 = 133$ and so the possible divisors of $m$ are $\{1, 7, 19, 133\}$;
        \item for $p = 7$ we have $m = 5^2 \times 19 = 475$ and so the possible divisors of $m$ are $\{1, 5, 19, 25, 95, 475\}$; and
        \item for $p = 19$ we have $m = 5^2 \times 7 = 175$ and so the possible divisors of $m$ are $\{1, 5, 7, 25, 35, 175\}$.
    \end{itemize}
    The Third Sylow Theorem (\myref{thrm-sylow-3}) tells us that $n_p \equiv 1 \pmod p$. Thus $n_5 = n_7 = n_{19} = 1$. Let $P$, $Q$, and $R$ be the Sylow 5-subgroup, the Sylow 7-subgroup, and the Sylow 19-subgroup respectively. We note that $P$, $Q$, and $R$ are all normal subgroups of $G$ by \myref{corollary-sylow-subgroup-is-normal-if-it-is-unique}.

    Denote the group $QR$ by $H$. Since $Q$ and $R$ are of prime order, their intersection is the identity (\myref{problem-intersection-of-coprime-subgroups}). Furthermore, as $Q$ and $R$ are normal subgroups of $G$, thus they commute by \myref{problem-intersection-of-coprime-subgroups}. Therefore $H$ is the internal direct product of $Q$ and $R$, meaning $H \cong Q \times R$ by \myref{thrm-direct-product-equivalence}. Hence $|H| = |Q||R| = 7 \times 19 = 133$. Now because as $Q$ and $R$ are of prime order, thus $Q$ and $R$ are abelian and so is $H$. Hence $H$ is an abelian group of order 133.

    Now consider the group $PH$. Since 5 and 133 are coprime, thus $P \cap H = \{e\}$. In addition, since $P \lhd G$ thus $PH \leq G$ by Diamond Isomorphism Theorem (\myref{thrm-isomorphism-2}), statement 3. Also,
    \[
        |PH| = \frac{|P||H|}{|P \cap H|} = |P||H| = 5^2 \times 133 = 3325 = |G|
    \]
    which means that $G = PH$. Since $P \lhd G$, thus $ph = hp$ for any element $h \in H$, meaning elements in $P$ and $H$ commute. Hence, $G$ is the internal direct product of $P$ and $H$, meaning $G \cong P \times H$. As the external direct product of two abelian groups is also abelian (\myref{problem-external-direct-product-of-abelian-groups-is-abelian}) thus $G$ is abelian.

    \item Let $P$ be a Sylow $p$-subgroup of $G$. We note that $|G/P| = \frac{p^km}{p^k} = m$. Let $G$ act on the set of cosets $G/P$ by left multiplication, meaning $g\cdot (xP) = (gx)P$. We know by \myref{thrm-group-action-definition-equivalence} that this induces a homomorphism $\phi: G \to \Sn{m}$ where $\phi(g) = \sigma_g$ such that $\sigma_g(xP) = g\cdot (xP) = (gx)P$. By \myref{example-using-kernel-to-show-non-simple}, $\ker\phi = \bigcap_{x \in G}xPx^{-1}$.

    We note $\ker\phi \neq \{e\}$ since otherwise it would imply that $\phi$ is injective (\myref{exercise-trivial-kernel-means-injective}), which is impossible as that would mean $p^km = |G| \leq |\Sn{m}| = m!$ which is a contradiction. Also $\ker\phi \neq G$ as otherwise
    \[
        p^km = |G| = |\ker\phi| = \left|\bigcap_{x \in G} xPx^{-1}\right| \leq |xPx^{-1}| = |P| = p^k,
    \]
    which would mean $m = 1$, a contradiction. Hence $\ker\phi$ is a non-trivial proper subgroup of $G$. We note that $\ker\phi \lhd G$, so we have found a non-trivial proper normal subgroup of $G$, meaning that $G$ is non-simple.

    \item Let $G$ be a group of order 30. Note $30 = 2 \times 3 \times 5$, and consider $n_5$. The Third Sylow Theorem (\myref{thrm-sylow-3}) tells us that
    \begin{itemize}
        \item $6 \vert n_5$, so $n_5 \in \{1, 2, 3, 6\}$; and
        \item $n_5 \equiv 1 \pmod 5$, so $n_5 \in \{1, 6, 11, 16, \dots\}$.
    \end{itemize}
    Hence $n_5$ is 1 or 6. Seeking a contradiction, assume $n_5 = 6$, and let $P_5$ be a Sylow 5-subgroup.

    Since $|P_5| = 5$, which is prime, each non-identity element of $P_5$ is a generator. Hence, no two Sylow 5-subgroups can share any non-identity elements (otherwise they will be the same group), thereby meaning any two Sylow 5-subgroups intersect in the identity only. Thus there exists $6(5-1) = 24$ elements of order 5, meaning there must be 6 elements of order not equal to 5.

    Now consider $n_3$. Note by the Third Sylow Theorem again,
    \begin{itemize}
        \item $10 \vert n_3$, so $n_3 \in \{1, 2, 5, 10\}$; and
        \item $n_3 \equiv 1 \pmod 3$, so $n_3 \in \{1, 4, 7, 10, 13, \dots\}$.
    \end{itemize}
    Thus $n_3$ is 1 or 10. Now if $n_3 = 10$ then there must be $10(3-1) = 20$ elements of order 3, a contradiction to the fact there exists only 6 elements with order not 5. Hence $n_3 = 1$, meaning the only Sylow 3-subgroup (call it $P_3$) is normal in $G$.

    As $P_5 \leq G$ and $P_3 \lhd G$, by the Diamond Isomorphism Theorem (\myref{thrm-isomorphism-2}), statement 3, we have $P_5P_3 \leq G$. Note $P_5 \cap P_3 = \{e\}$ by \myref{problem-intersection-of-coprime-subgroups}. So \myref{exercise-order-of-subgroup-product} tells us
    \[
        |P_5P_3| = \frac{|P_5||P_3|}{|P_5 \cap P_3|} = \frac{5\times3}{1} = 15.
    \]
    One sees that $[G:P_5P_3] = \frac{30}{15} = 2$, so $P_5P_3 \lhd G$ by \myref{problem-subgroup-of-index-2}.

    Now \myref{problem-group-of-order-pq-has-normal-subgroup-of-order-q} tells us there exists a unique $H \lhd P_5P_3$ with $|H| = 5$. But since $P_5P_3 \lhd G$, \myref{problem-normal-subgroup-of-G-contains-all-sylow-p-subgroups} tells us that $P_5P_3$ contains all Sylow 5-subgroups of $G$, meaning $G$ has only 1 Sylow 5-subgroup, i.e. $n_5 = 1$, a contradiction to our assumption that $n_5 = 6$.

    Hence $n_5 = 1$. Therefore the unique Sylow 5-subgroup is a normal subgroup of $G$ by \myref{corollary-sylow-subgroup-is-normal-if-it-is-unique}.

    \item We prove that $G$ has a normal subgroup of order $p$, $q$, or $r$. By \myref{corollary-sylow-subgroup-is-normal-if-it-is-unique}, subgroups of order $p$, $q$, or $r$ are normal if they are unique. By way of contradiction, assume that they are not unique, meaning $n_p, n_q, n_r > 1$.

    By the Third Sylow Theorem (\myref{thrm-sylow-3}), $n_r \equiv 1 \pmod r$ and $n_r \mid pq$. The divisors of $pq$ are 1, $p$, $q$, and $pq$. We note that since both $p$ and $q$ are less than $r$, thus $p \not\equiv 1 \pmod r$ and $q \not\equiv 1 \pmod r$. The only possibility that is left is $n_r = pq$ as we assume $n_r \neq 1$. Similarly, $n_q \equiv 1 \pmod q$ and $n_q \mid pr$. The divisors of $pr$ are 1, $p$, $r$, and $pr$. Since $p < q$ thus $p \not\equiv 1 \pmod q$. Hence $n_q \geq r$ as we assume $n_q \neq 1$. Similarly, $n_p \geq q$.

    We now consider the number of elements with order $p$, $q$, and $r$.
    \begin{itemize}
        \item $\boxed{p}$ With $n_p \geq q$, there are at least $q(p-1)$ elements of order $p$. We minus 1 because one of the elements in a Sylow $p$-subgroup is the identity with order 1.
        \item $\boxed{q}$ With $n_q \geq r$, there are at least $r(q-1)$ elements of order $q$.
        \item $\boxed{r}$ We know $n_r = pq$ so there are exactly $pq(r-1)$ elements of order $r$.
    \end{itemize}
    Since the total number of elements, $pqr$, must be at least the sum of the numbers of these elements, thus
    \begin{align*}
        pqr &\geq q(p-1) + r(q-1) + pq(r-1)\\
        &= pq - q + qr - r + pqr - pq\\
        &= pqr + qr - q - r
    \end{align*}
    which means $qr - q - r \leq 0$. Rearranging, we see
    \[
        q \leq \frac{r}{r-1} = 1 + \frac{1}{r-1}.
    \]
    Since $p < q$ and they are both primes, we must have $q \geq 3$. Hence one sees
    \[
        3 \leq q \leq 1 + \frac{1}{r-1} \leq 2
    \]
    which is a clear contradiction. Hence, at least one of $n_p$, $n_q$, or $n_r$ is 1, meaning that there exists a non-trivial proper normal subgroup in $G$ by \myref{corollary-sylow-subgroup-is-normal-if-it-is-unique}. Therefore $G$ is non-simple.
\end{questions}

\section{Composition Series}
\subsection*{Exercises}
\begin{questions}
    \item \begin{partquestions}{\roman*}
        \item One sees clearly that $\{0, 2\}$ is the only non-trivial proper normal subgroup of $G$, so the subnormal series of length 2 is $1 \lhd \{0, 2\} \lhd G$.
        \item There are 2 factor groups of the above subnormal series. The first is $\{0, 2\} / 1 \cong \Z_2$ and the second is
        \begin{align*}
            G / \{0, 2\} &= \{g \oplus_4 \{0, 2\} \vert g \in G\}\\
            &= \{\{0, 2\}, \{1, 3\}, \{2, 0\}, \{3, 1\}\}\\
            &= \{\{0, 2\}, \{1, 3\}\}\\
            &= \langle \{1, 3\} \rangle\\
            &\cong \Z_2.
        \end{align*}
        \item Since $1 \lhd G$ and $\{0, 2\} \lhd G$ thus the subnormal series in \textbf{(i)} is also a normal series of $G$.
    \end{partquestions}

    \item By Lagrange's theorem (\myref{thrm-lagrange}) we know that the order of a subgroup must divide the order of the group. Furthermore $\Z_{120}$ is abelian, so any subgroup of it is normal. Now the subgroup $N = \{0, 2, 4, \dots, 118\}$ has 60 elements which is the maximum possible guaranteed by Lagrange. Hence $N$ is the maximal normal subgroup of $\Z_{120}$, which has order 60.

    \item $\Cn{6}$ has
    \begin{align*}
        &1 \lhd \Cn{2} \lhd \Cn{6} \text{ and }\\
        &1 \lhd \Cn{3} \lhd \Cn{6}
    \end{align*}
    as composition series up to isomorphism. In both cases, their composition length is 2. Their respective composition factors are
    \begin{itemize}
        \item $\Cn{2} / 1 \cong \Cn{2}$ and $\Cn{6} / \Cn{2} \cong \Cn{3}$ by \myref{exercise-Zmn-mod-Zn-cong-Zn}; and
        \item $\Cn{3} / 1 \cong \Cn{3}$ and $\Cn{6} / \Cn{3} \cong \Cn{2}$ by \myref{exercise-Zmn-mod-Zn-cong-Zn},
    \end{itemize}
    up to isomorphism.

    \item Let the group in question be $G$. We know by Cauchy's theorem (\myref{thrm-cauchy}) and \myref{exercise-group-of-order-multiple-of-prime-has-subgroup-of-prime-order}, and by writing $p^2$ as $p \times p$, that $G$ has a subgroup of order $p$. Let this subgroup be $H$.

    Lagrange's theorem (\myref{thrm-lagrange}) tells us that the possible orders of the subgroups of $G$ are 1, $p$, and $p^2$. These subgroups are $\{e\}$, $H$, and $G$ respectively. Furthermore, by \myref{problem-group-of-order-prime-squared-is-abelian}, $G$ must be abelian, therefore its subgroups are all normal (\myref{prop-subgroup-of-abelian-group-is-normal}). Finally, a corollary of Lagrange's theorem (\myref{corollary-group-with-prime-order-subgroups}) says that the only subgroups of $H$ are the trivial group and the group itself. Hence, $G$ has only one composition series, namely $1 \lhd H \lhd G$.
\end{questions}

\subsection*{Problems}
\begin{questions}
    \item \begin{partquestions}{\roman*}
        \item We note $\mathrm{V}$ has order 4. As $4 = 2 \times 2$, thus we know that $\mathrm{V}$ has a subgroup of order 2 (which is cyclic) by Cauchy's theorem (\myref{thrm-cauchy}). Now $\mathrm{V}$ is abelian (\myref{problem-group-of-order-prime-squared-is-abelian}) which means that the subgroup of order 2 is normal (\myref{prop-subgroup-of-abelian-group-is-normal}). Finally, the only possible order for a non-trivial proper subgroup of $\mathrm{V}$ is 2 by Lagrange's theorem (\myref{thrm-lagrange}). Hence, the only composition series for $\mathrm{V}$ is $1 \lhd \Cn{2} \lhd \mathrm{V}$ up to isomorphism.\newline
        (Note that this analysis applies for any group of order 4.)

        \item Recall that $\mathrm{Q} = \langle \alpha, \beta \vert \alpha^4 = e, \alpha^2 = \beta^2, \text{ and } \beta\alpha = \alpha^3\beta \rangle$. From the solution of \myref{exercise-normal-subgroups-of-quarternion-group}, the maximal subgroups of $\mathrm{Q}$ are $G_1 = \langle \alpha \rangle$, $G_2 = \langle \beta \rangle$, and $G_3 = \langle \alpha\beta \rangle$ (by setting $\alpha = i$ and $\beta = j$). We note the following.
        \begin{itemize}
            \item $G_1 = \{e, \alpha, \alpha^2, \alpha^3\} \cong \Cn{4}$.
            \item $G_2 = \{e, \beta, \beta^2, \beta^3\} = \{e, \beta, \alpha^2, \alpha^2\beta\} \cong \mathrm{V}$ where $a = \alpha^2$ and $b = \beta$.
            \item $G_3 = \{e, \alpha\beta, (\alpha\beta)^2, (\alpha\beta)^3\} = \{e, \alpha\beta, \alpha^2, \alpha^3\beta\} \cong \mathrm{V}$ with $a = \alpha\beta$ and $b = \alpha^2$.
        \end{itemize}
        Also, note that $\Cn{2} \cong \langle \alpha^2 \rangle \lhd G_1$, $\Cn{2} \cong \langle \beta^2 \rangle \lhd G_2$, and $\Cn{2} \cong \langle (\alpha\beta)^2 \rangle \lhd G_3$. Hence, the two series up to isomorphism are
        \begin{align*}
            1 \lhd \Cn{2} \lhd \Cn{4} \lhd \mathrm{Q} & \text{ and }\\
            1 \lhd \Cn{2} \lhd \mathrm{V} \lhd \mathrm{Q}.
        \end{align*}

        \item By the Jordan-H\"older theorem (\myref{thrm-jordan-holder}), the composition factors are isomorphic to each other. We note
        \begin{itemize}
            \item $\Cn{2} / 1 \cong \Cn{2}$;
            \item $\Cn{4} / \Cn{2} \cong \Cn{2}$ by \myref{exercise-Zmn-mod-Zn-cong-Zn}; and
            \item $\mathrm{V} / \Cn{2} \cong (\Cn{2})^2 / \Cn{2} \cong \Cn{2}$ by \myref{problem-cartesian-product-of-group-by-group-isomorphic-to-group}.
        \end{itemize}
        The only unaccounted set of factors is $\mathrm{Q}/\mathrm{V}$ and $\mathrm{Q}/\Cn{4}$. So, either $\mathrm{Q}/\mathrm{V} \cong \Cn{2}$ and $\mathrm{Q}/\Cn{4} \cong \Cn{2}$, or $\mathrm{Q}/\mathrm{V} \cong \mathrm{Q}/\Cn{4}$. Hence $\mathrm{Q}/H \cong \mathrm{Q}/K$.
    \end{partquestions}

    \item We know that $\An{4} \lhd \Sn{4}$ by \myref{prop-An-normal-subgroup-of-Sn}. Note $\An{4}$ is a maximal normal subgroup since $|\An{4}| = \frac{4!}2 = 12$ by \myref{prop-order-of-An}, and a subgroup's order must divide the order of the group by Lagrange's theorem (\myref{thrm-lagrange}).

    Now applying that theorem on $\An{4}$, we see that the possible orders of a subgroup of $\An{4}$ are 6, 4, 3, 2, and 1. We claim that a subgroup of order 6 does not exist. Note that $\An{4}$ contains
    \begin{itemize}
        \item 1 element of order 1;
        \item 3 elements of order 2; and
        \item 8 elements of order 3.
    \end{itemize}
    If a subgroup of order 6 exists (say, $H$), then its index would be $\frac{12}{6} = 2$ by Lagrange, meaning $H$ contains all odd order elements (\myref{problem-subgroup-of-index-2}). However, there are $1 + 8 = 9$ odd order elements, meaning that $H$ has an order of at least 9, a contradiction. Hence a subgroup with $\An{4}$ of order 6 is impossible.

    Now we note that a subgroup of order $4 = 2^2$ exists by a corollary of the First Sylow Theorem (\myref{corollary-sylow-p-subgroup-exists}) as it is a Sylow 2-subgroup. The Third Sylow Theorem (\myref{thrm-sylow-3}) tells us how many Sylow 2-subgroups there are, in particular
    \begin{itemize}
        \item $n_2 \vert 3$, so $n_2$ is 1 or 3; and
        \item $n_2 \equiv 1 \pmod2$, so $n_2 \in \{1, 3, 5, \dots\}$.
    \end{itemize}
    Hence $n_2 = 1$ or $n_2 = 3$. Now if $n_2 = 3$, then the number of elements of order of 1, 2, or 4 is
    \[
        3 \times (4 - 1) + 1 = 10,
    \]
    where the 3 is $n_2$, the $4-1$ is the number of non-identity elements in each Sylow 2-subgroup, and the $+1$ is to add the identity element. However, as noted above, there are only 4 elements of order 1, 2, or 4, a contradiction. Hence $n_2 = 1$, meaning the Sylow 2-subgroup (which is a subgroup of order 4) is normal (\myref{corollary-sylow-subgroup-is-normal-if-it-is-unique}). Therefore the subgroup of order 4 is the maximal normal subgroup of $\An{4}$.

    We note that the subgroup of order 4 of $\An{4}$ is not $\Cn{4}$ (as this would imply that $\An{4}$ has an element of order 4, which it does not). Hence, from \myref{problem-smallest-nonabelian-group}, the subgroup of order 4 must be isomorphic to the Klein-4 group, $\mathrm{V}$.

    Note that a group of order 4 has a subgroup of order 2 by Cauchy's theorem (\myref{thrm-cauchy}). Clearly such a subgroup is cyclic (since 2 is prime), and has index $\frac42 = 2$, meaning that it is normal in the group of order 4. Furthermore the trivial group is always a subgroup of any group.

    Hence, the composition series for $\Sn{4}$, up to isomorphism, is
    \[
        1 \lhd \Cn{2} \lhd \mathrm{V} \lhd \An{4} \lhd \Sn{4}.
    \]
    \begin{remark}
        We list the actual subgroups that are isomorphic to the above terms in the composition series here.
        \begin{itemize}
            \item $\Cn{2}$: $\{e, \begin{pmatrix}1&2\end{pmatrix}\begin{pmatrix}3&4\end{pmatrix}\}$
            \item V: $\{e, \begin{pmatrix}1&2\end{pmatrix}\begin{pmatrix}3&4\end{pmatrix}, \begin{pmatrix}1&3\end{pmatrix}\begin{pmatrix}2&4\end{pmatrix}, \begin{pmatrix}1&4\end{pmatrix}\begin{pmatrix}2&3\end{pmatrix}\}$
            \item $\An{4}$ is an actual subgroup of $\Sn{4}$
        \end{itemize}
    \end{remark}
\end{questions}

\section{Simple Groups}
\begin{questions}
    \item We find the number of Sylow 2- and Sylow 3-subgroups (denoted $n_2$ and $n_3$ respectively) of the group of order 12 (call it $G$) using the Third Sylow Theorem (\myref{thrm-sylow-3}):
    \begin{itemize}
        \item For $n_2$, note $12 = 2^2 \times 3$. So,
        \begin{itemize}
            \item $3 \vert n_2$ meaning $n_2 \in \{1, 3\}$; and
            \item $n_2 \equiv 1 \pmod 2$ meaning $n_2 \in \{1, 3, 5, \dots\}$.
        \end{itemize}
        Hence $n_2 = 1$ or $n_2 = 3$.

        \item For $n_3$, note $12 = 3 \times 2^2$. So,
        \begin{itemize}
            \item $4 \vert n_3$ meaning $n_3 \in \{1, 2, 4\}$; and
            \item $n_3 \equiv 1 \pmod 3$ meaning $n_3 \in \{1, 4, 7, \dots\}$.
        \end{itemize}
        Hence $n_3 = 1$ or $n_3 = 4$.
    \end{itemize}

    Now by way of contradiction suppose both $n_2$ and $n_3$ are not 1. Thus $n_2 = 3$ and $n_3 = 4$. We consider the number of elements of a certain order.
    \begin{itemize}
        \item Number of elements with order 2 or 4 is $3(4-1) = 9$, since each of the 3 Sylow 2-subgroups has 4 elements, 1 of which is the identity.
        \item Number of elements with order 3 is $4(3-1) = 8$, since each of the 4 Sylow 3-subgroups has 3 elements, 1 of which is the identity.
    \end{itemize}
    Therefore, the number of elements in $G$ must be at least $9 + 8 = 17 > 12$, a contradiction.

    Thus, at least one of $n_2$ and $n_3$ must be 1, meaning that there must exist a normal subgroup of order 4 or 3 (or both) by \myref{corollary-sylow-subgroup-is-normal-if-it-is-unique}.

    \item \begin{partquestions}{\alph*}
        \item Note $15 = 3 \times 5$, so \myref{problem-group-of-order-pq-has-normal-subgroup-of-order-q} means that there exists a unique (and hence normal) subgroup of order 5.
        \item Note $20 = 2^2 \times 5$. Then the Third Sylow Theorem (\myref{thrm-sylow-3}) tells us that
        \begin{itemize}
            \item $2^2 \vert n_5$, so $n_5 \in \{1, 2, 4\}$; and
            \item $n_5 \equiv 1 \pmod 5$, so $n_5 \in \{1, 6, 11, 16, \dots\}$.
        \end{itemize}
        Hence $n_5 = 1$, so there exists a unique (and hence normal) subgroup of order 5.
    \end{partquestions}

    \item \begin{partquestions}{\alph*}
        \item Note that $\sigma = \begin{pmatrix}1&3\end{pmatrix}\begin{pmatrix}2&3\end{pmatrix}\begin{pmatrix}2&4\end{pmatrix}\begin{pmatrix}4&5\end{pmatrix}$, so $\sigma$ is an even permutation (\myref{thrm-parity-of-permutation}), and since the highest integer that appears in $\sigma$ is 5, thus $\sigma \in \An5$.

        \item Observe that
        \begin{itemize}
            \item $\sigma^0 = \id$;
            \item $\sigma^1 = \sigma \neq \id$;
            \item $\sigma^2 = \begin{pmatrix}1&2&5&3&4\end{pmatrix} \neq \id$;
            \item $\sigma^3 = \begin{pmatrix}1&4&3&5&2\end{pmatrix} \neq \id$;
            \item $\sigma^4 = \begin{pmatrix}1&5&4&2&3\end{pmatrix} \neq \id$; and
            \item $\sigma^5 = \id$.
        \end{itemize}
        Hence the order of $\langle \sigma \rangle$ is 5 since that cyclic subgroup contains 5 distinct elements.

        \item Consider the other permutation $\pi = \begin{pmatrix}1&2&3&4&5\end{pmatrix}$. We note $\pi = \begin{pmatrix}1&2\end{pmatrix}\begin{pmatrix}2&3\end{pmatrix}\begin{pmatrix}3&4\end{pmatrix}\begin{pmatrix}4&5\end{pmatrix} \in \An5$. Also,
        \begin{itemize}
            \item $\pi^0 = \id$;
            \item $\pi^1 = \pi \neq \id$;
            \item $\pi^2 = \begin{pmatrix}1&3&5&2&4\end{pmatrix} \neq \id$;
            \item $\pi^3 = \begin{pmatrix}1&4&2&5&3\end{pmatrix} \neq \id$;
            \item $\pi^4 = \begin{pmatrix}1&5&4&3&2\end{pmatrix} \neq \id$; and
            \item $\pi^5 = \id$,
        \end{itemize}
        meaning $|\langle \pi \rangle| = 5$ with $\langle \pi \rangle \neq \langle \sigma \rangle$, so we have found another subgroup of $\An5$.
    \end{partquestions}

    \item Note that $\sigma \in G$, and
    \begin{align*}
        \sigma\Stab{G}{x}\sigma^{-1} &= \{\underbrace{\sigma\pi\sigma^{-1}}_{\text{Set as }\pi'} \vertalt \pi \in G,\; \pi(x) = x\}\\
        &= \{\pi' \vertalt \underbrace{\sigma^{-1}\pi'\sigma \in G}_{\text{True if } \pi' \in G},\; \sigma^{-1}\pi'\sigma(x) = x\}\\
        &= \{\pi' \vert \pi' \in G,\; \sigma^{-1}\pi'(\sigma(x)) = x\}\\
        &= \{\pi' \vert \pi' \in G,\; \pi'(\sigma(x)) = \sigma(x)\}\\
        &= \{\pi \vert \pi \in G,\; \pi(\sigma(x)) = \sigma(x)\}\\
        &= \Stab{G}{\sigma(x)}
    \end{align*}
    which proves the claim.

    \item Observe that if $\sigma(i) = j$, then we must have
    \[
        \pi\sigma\pi^{-1}(\pi(i)) = \pi\sigma(i) = \pi(j).  
    \]
    Thus if the ordered pair $(i, j)$ appears in the cycle decomposition of $\sigma$, then the ordered pair $(\pi(i), \pi(j))$ appears in the cycle decomposition of $\pi\sigma\pi^{-1}$, completing the proof of the claim.
\end{questions}


\chapter{Problem Solutions}
\section{Introduction to Groups}
\subsection*{Exercises}
\begin{questions}
    \item There are 6! = 720 possible permutations of 6 points, so there are 720 symmetries in the group given. That is, the order of the symmetric group of degree 6 is 720.
\end{questions}

\subsection*{Problems}
\begin{questions}
    \item \begin{partquestions}{\alph*}
        \item This is a group. Addition is clearly closed and associative. The identity is 0. The inverse of any element $x$ is $-x$.
        \item This is not a group. Inverses do not exist. For example, the element $2$ does not have an inverse under multiplication.
        \item This is a group. Multiplication is clearly closed and associative. The identity is $1$. The inverse of any element $x$ is $\frac1x$.
        \item This is a group. Multiplication is clearly closed and associative. The identity is $0$ and the inverse is $0$.
        \item This is not a group. Addition is not closed: $1 + 1 = 2$ which is not in the group.
        \item This is a group. Multiplication is clearly closed and associative. The identity is $1$ and the inverse is $1$.
    \end{partquestions}

    \item We show that the trivial group is indeed a group by showing that the four group axioms hold.
    \begin{itemize}
        \item \textbf{Closure}: The only element in the underlying set is $e$, and $e \ast e = e \in \{e\}$. Thus the structure is closed under $\ast$.
        \item \textbf{Associativity}: Clearly $e \ast (e \ast e) = e \ast e = e$ and $(e \ast e) \ast e = e \ast e = e$ which means that $\ast$ is associative.
        \item \textbf{Identity}: The identity is clearly $e$.
        \item \textbf{Inverse}: The only element is $e$, and because $e \ast e = e$ hence $e$ is its own inverse.
    \end{itemize}
    Therefore $(\{e,\}, \ast)$ is a group.
\end{questions}

\section{Basics of Groups}
\subsection*{Exercises}
\begin{questions}
    \item The Cayley table of $(\Z_6, \otimes_6)$ is as follows:
    \begin{table}[H]
        \centering
        \begin{tabular}{|l|l|l|l|l|l|l|}
        \hline
        \textbf{$\otimes_n$} & \textbf{0} & \textbf{1} & \textbf{2} & \textbf{3} & \textbf{4} & \textbf{5} \\ \hline
        \textbf{0}       & 0          & 0          & 0          & 0          & 0          & 0          \\ \hline
        \textbf{1}       & 0          & 1          & 2          & 3          & 4          & 5          \\ \hline
        \textbf{2}       & 0          & 2          & 4          & 0          & 2          & 4          \\ \hline
        \textbf{3}       & 0          & 3          & 0          & 3          & 0          & 3          \\ \hline
        \textbf{4}       & 0          & 4          & 2          & 0          & 4          & 2          \\ \hline
        \textbf{5}       & 0          & 5          & 4          & 3          & 2          & 1          \\ \hline
        \end{tabular}
    \end{table}

    Since the identity is $1$, and the row (and column) of 0 does not have a $1$, thus $0$ does not have an inverse. Therefore $(\Z_6, \oplus_6)$ is not a group.

    \item Note that $(xx^{-1})^{-1} = (x^{-1})^{-1}x^{-1}$ by Shoes and Socks and $(xx^{-1})^{-1} = e^{-1} = e$. Thus $(x^{-1})^{-1}x^{-1} = e$. Multiplying both sides on the right by $x$ yields $(x^{-1})^{-1} = ex = x$, i.e. $(x^{-1})^{-1} = x$.

    \item We consider a proof by induction via inducting on $n$.

    The base case of $n = 0$ clearly holds true since
    \begin{align*}
        (x^{-1})^0 &= e & (\text{definition of }g^0 \text{ for any }g\in G)\\
        &= e^{-1} & (\myref{prop-inverse-of-identity-is-identity})\\
        &= (x^0)^{-1}. & (\text{definition of }x^0)
    \end{align*}

    Now assume that the statement holds for a non-negative integer $k$, i.e. $(x^{-1})^k = (x^k)^{-1}$. We are to show that the statement holds for $k+1$, i.e. $(x^{-1})^{k+1} = (x^{k+1})^{-1}$.

    Observe that
    \begin{align*}
        (x^{-1})^{k+1} &= (x^{-1})^k \ast x^{-1} & (\text{by statement 1})\\
        &= (x^k)^{-1} \ast x^{-1} & (\text{by hypothesis})\\
        &= (x\ast x^k)^{-1} & (\text{by Shoes and Socks})\\
        &= (x^{k+1})^{-1} & (\text{by statement 1})
    \end{align*}
    so the statement is true for $k+1$.

    Thus, by induction, we have $(x^{-1})^n = (x^n)^{-1}$ for any non-negative integer $n$.

    \item \begin{partquestions}{\roman*}
        \item The identity is $1$ since:
        \begin{itemize}
            \item $1 \times 1 = 1$;
            \item $1 \times (-1) = (-1) \times 1 = -1$;
            \item $1 \times i = i \times 1 = i$; and
            \item $1 \times (-i) = (-i) \times 1 = -i$.
        \end{itemize}
        \item The order of the identity $1$ is 1, so we look at the other elements:
        \begin{itemize}
            \item $|-1| = 2$ since $-1 \neq 1$ and $(-1)^2 = -1 \times -1 = 1$.
            \item $|i| = 4$ since $i \neq 1$, $i^2 = -1 \neq 1$, $i^3 = -i \neq 1$, but $i^4 = 1$.
            \item $|-i| = 4$ since $-i \neq 1$, $(-i)^2 = -1 \neq 1$, $(-i)^3 = i \neq 1$, but $(-i)^4 = 1$.
        \end{itemize}
    \end{partquestions}

    \item $-i$ is the other generator since $(-i)^1 = -i$, $(-i)^2 = -1$, $(-i)^3 = i$, and $(-i)^4 = 1$.

    \item We work slowly:
    \begin{align*}
        rsr^4sr^3 &= r(sr^4)(sr^3)\\
        &= r(r^2s)(r^3s)\\
        &= r^3sr^3s\\
        &= r^3(sr^3)s\\
        &= r^3(r^3s)s\\
        &= r^6s^2\\
        &= e
    \end{align*}
\end{questions}

\subsection*{Problems}
\begin{questions}
    \item The group table of $D_4$ is given as follows.
    \begin{table}[H]
        \centering
        \begin{tabular}{|l|l|l|l|l|l|l|l|l|}
        \hline
        $\ast$ & $e$    & $r$    & $r^2$  & $r^3$  & $s$    & $rs$   & $r^2s$ & $r^3s$ \\ \hline
        $e$    & $e$    & $r$    & $r^2$  & $r^3$  & $s$    & $rs$   & $r^2s$ & $r^3s$ \\ \hline
        $r$    & $r$    & $r^2$  & $r^3$  & $e$    & $rs$   & $r^2s$ & $r^3s$ & $s$    \\ \hline
        $r^2$  & $r^2$  & $r^3$  & $e$    & $r$    & $r^2s$ & $r^3s$ & $s$    & $rs$   \\ \hline
        $r^3$  & $r^3$  & $e$    & $r$    & $r^2$  & $r^3s$ & $s$    & $rs$   & $r^2s$ \\ \hline
        $s$    & $s$    & $r^3s$ & $r^2s$ & $rs$   & $e$    & $r^3$  & $r^2$  & $r$    \\ \hline
        $rs$   & $rs$   & $s$    & $r^3s$ & $r^2s$ & $r$    & $e$    & $r^3$  & $r^2$  \\ \hline
        $r^2s$ & $r^2s$ & $rs$   & $s$    & $r^3s$ & $r^2$  & $r$    & $e$    & $r^3$  \\ \hline
        $r^3s$ & $r^3s$ & $r^2s$ & $rs$   & $s$    & $r^3$  & $r^2$  & $r$    & $e$    \\ \hline
        \end{tabular}
    \end{table}
    \begin{partquestions}{\alph*}
        \item $D_4$ is not abelian because $rs \neq sr = r^3s$.
        \item We simplify $r^3srsr^3sr^3sr^2$.
        \begin{align*}
            r^3 sr sr^3 sr^3 sr^2 &= r^3srs(r^3s)(r^3s)r^2\\
            &= r^3 srs(e)r^2\\
            &= r^3 sr sr^2\\
            &= r^2(rs rs)r^2\\
            &= r^2(e)r^2\\
            &= r^4\\
            &= e
        \end{align*}
    \end{partquestions}

    \item We need to prove each of the group axioms in order to prove that $(\Q, +)$ is indeed a group.
    \begin{itemize}
        \item \textbf{Closure}: Let $\frac ab$ and $\frac cd$ be rational numbers where $b, d \neq 0$. Their sum is $\frac{ad+bc}{bd}$, which is also rational. Therefore $\Q$ is closed under addition.

        \item \textbf{Associativity}: Addition is associative by \myref{axiom-addition-is-associative}.

        \item \textbf{Identity}: 0 is the identity since
        \[
            0 + \frac ab = \frac ab + 0 = \frac ab
        \]
        for any rational number $\frac ab$ (with $b \neq 0$).

        \item \textbf{Inverse}: For any rational number $\frac ab$, its inverse is $-\frac ab$ since
        \[
            \frac ab + \left(-\frac ab\right) = \left(-\frac ab\right) + \frac ab = 0
        \]
        for any rational number $\frac ab$ (with $b \neq 0$).
    \end{itemize}
    Furthermore addition is assumed to be commutative by \myref{axiom-addition-is-commutative}. Therefore $(\Q, +)$ is an abelian group.

    \item If every element in $G$ is its own inverse, then for every element $g$ in $G$, $g^{-1} = g$. Consider $(gh)^{-1}$ where $g$ and $h$ are elements in $g$. On one hand, by Shoes and Socks, $(gh)^{-1} = h^{-1}g^{-1} = hg$ since each element is its own inverse. On the other hand, since $gh$ is an element in $G$, thus $(gh)^{-1} = gh$. Thus $gh = hg$ which means $G$ is abelian.

    \item Recall that $n = |x|$ is the smallest positive integer that satisfies $x^n = e$.

    We prove the forward direction first. Suppose $m$ is a multiple of $n$, say $m = qn$ for some integer $q$. Then
    \[
        x^m = x^{qn} = \left(x^n\right)^q = e^q = e
    \]
    which means $x^m = e$.

    We now prove the reverse direction. Suppose $x^m = e$. Using Euclid's division lemma (\myref{lemma-euclid-division}), we write $m = qn + r$ where $q$ and $r$ are integers with $0 \leq r < n$. Hence
    \[
        x^m = x^{qn + r} = x^{qn}x^r = \left(x^n\right)^qx^r = e^qx^r = x^r.
    \]
    Note that for all integers $k$ where $1 \leq k < n$, we have $x^k \neq e$ since $n$ is the smallest positive integer such that $x^n = e$. Hence, if $x^r = e$, we conclude $r = 0$. Therefore $m = qn$, meaning $m$ is a multiple of $n$.

    \item \begin{partquestions}{\alph*}
        \item Note that $(gh)^2 = ghgh$. Given that $(gh)^2 = g^2h^2 = gghh$. By cancellation law, $hg = gh$ which means $G$ is abelian.
        \item Suppose $G$ is abelian. Clearly $(gh)^1 = gh$. Suppose $(gh)^{k} = g^kh^k$ for some positive integer $k$. Then
        \begin{align*}
            (gh)^{k+1} &= (gh)(gh)^k\\
            &= (gh)(g^kh^k) & (\text{by assumption})\\
            &= ghg^kh^k\\
            &= g(hg^k)h^k\\
            &= g(g^kh)h^k & (\text{since } G \text{ is abelian})\\
            &= gg^khh^k\\
            &= g^{k+1}h^{k+1}
        \end{align*}
        so $(gh)^{k+1} = g^{k+1}h^{k+1}$ assuming $(gh)^k = g^kh^k$. Thus the claim is proven by mathematical induction.
    \end{partquestions}

    \item Note that $|1| = n$ since $1^2 = 1 \oplus_n 1 = 2$, $1^3 = 1 \oplus_n 1 \oplus_n 1 = 3$, $1^4 = 4$, ..., $1^{n-1} = n-1$ and $1^n = 0$ which is the identity. Since the group $(\Z_n, \oplus_n)$ has an element with the same order as the group, it is thus cyclic with order $n$ and generator 1.

    \item We show that $(A, \circ)$ is a group.
    \begin{itemize}
            \item \textbf{Closure}: Function composition is closed by definition.
            \item \textbf{Associativity}: Function composition is associative.
            \item \textbf{Identity}: By performing brute-force computation, we find that $T^6(x, y) = (x, y)$. Hence $T^6$ is the identity of $A$.
            \item \textbf{Inverse}: If $r = 6$ then $T^r$ is its own inverse. Otherwise, $T^{6-r}$ is the inverse of $T^r$.
    \end{itemize}
    Thus, $(A, \circ)$ is a group, with order 6.
\end{questions}

\section{Subgroups}
\begin{questions}
    \item We note that $G$ contains $\{e, r, r^2, r^3, s, rs, r^2s, r^3s\}$.
    \begin{partquestions}{\alph*}
        \item Yes, this is the trivial subgroup.
        \item No, it is not closed. ($rs$ can be generated by $r \ast s$ but is not in the set)
        \item No, the identity $e$ is missing.
        \item Yes, $\{r, r^3, r^4, r^6\} = \{r, r^3, e, r^2\} = \langle r \rangle$.
    \end{partquestions}

    \item \begin{partquestions}{\alph*}
        \item Clearly $e \in K$ since $e^2 = e \in H$.
        
        Let $x, y \in K$, so $x^2 \in H$ and $y^2 \in H$. We note $y^{-1} \in K$ since $(y^{-1})^2 = (y^2)^{-1} \in H$. Therefore $(xy^{-1})^2 = xy^{-1}xy^{-1} \in H$, so $xy^{-1} \in K$. Hence $K \leq G$ by subgroup test.

        \item Note that the identity of $K$, $e$, is also the identity of $G$. Since $H \leq G$, thus $e \in H$. Since $H$ is a subgroup of $G$, thus for any $x$ and $y$ in $H$ we have $xy^{-1} \in H$. Hence, by subgroup test, $H \leq K$.
    \end{partquestions}

    \item \begin{partquestions}{\alph*}
        \item We have proved that $\Z{G} \leq G$ so we only prove normality. Let $g$ and $z$ be arbitrary elements from $G$ and $\mathrm{Z}(G)$ respectively. Then
        \begin{align*}
            gzg^{-1} &= g(zg^{-1})\\
            &= g(g^{-1}z) & (\text{since }z \in \mathrm{Z}(G))\\
            &= (gg^{-1})z \\
            &= z\\
            &\in \mathrm{Z}(G)
        \end{align*}
        which proves that $\mathrm{Z}(G) \unlhd G$.

        \item We first work in the forward direction by assuming $G = \Z{G}$. Then for all $z \in \Z{G} = G$ we have $gz = zg$ for any $g \in G$ by definition, which means that $G$ is abelian.

        We now work in the reverse direction by assuming that $G$ is abelian. Note $\mathrm{Z}(G) = \{z \in G \vert gz = zg \text{ for all } g \in G\}$. But since $G$ is abelian, $gh = hg$ for all $g$ and $h$ in $G$. Thus every element in $G$ satisfies the condition to be in the center of $G$, meaning $\mathrm{Z}(G) = G$.

        \item We note that $D_4 = \{e, r, r^2, r^3, s, rs, r^2s, r^3s\}$. Since $\mathrm{Z}(D_4)$ is a subgroup of $D_4$ it has a maximum order of 2, by Lagrange's theorem (\myref{thrm-lagrange}). Since 2 is prime the subgroups must be cyclic. Thus the non-trivial subgroups of $D_4$ are $\{e, r^2\}$ and $\{e, s\}$ (since $|r^2| = |s| = 2$). Now like how we proved that $\langle s \rangle = \{e, s\}$ is not a normal subgroup in $D_3$ in \myref{example-normal-subgroups-of-d3}, $\{e, s\}$ is not a normal subgroup of $D_4$. One verifies easily that $\{e, r^2\} = \langle r^2 \rangle$ is a normal subgroup of $D_4$. Thus $\mathrm{Z}(D_4) = \langle r^2 \rangle$ since $\mathrm{Z}(D_4)$ must be a normal subgroup of $D_4$ with order not exceeding 2.
    \end{partquestions}

    \item \begin{partquestions}{\alph*}
        \item We will prove this statement.

        Clearly $e \in H \cap K$ since $e \in H$ and $e \in K$ as both are subgroups of $G$.

        Let $x$ and $y$ be in $H \cap K$, meaning that $x, y \in H$ and $x, y \in K$. Thus $xy^{-1} \in H$ and $xy^{-1} \in K$ as both are subgroups of $G$. Hence $xy^{-1} \in H \cap K$.
        By subgroup test, $H \cap K \leq G$.

        \item We will prove this statement. One sees that $H \cap K \subseteq H$. Since $H \cap K \leq G$, it is thus a group. Hence $H \cap K \leq H$ by definition of a subgroup.

        \item We will disprove this statement. Consider:
        \begin{align*}
            &G = \mathbb{Z}_6 \text{ under }\oplus_6,\\
            &H = \{0, 2, 4\},\text{ and}\\
            &K = \{0, 3\}.
        \end{align*}
        Clearly $H \leq G$ and $K \leq G$. Note $H \cup K = \{0, 2, 3, 4\}$. But $H \cup K$ is not closed since $2 \oplus_6 3 = 5 \not \in H \cup K$. Hence $H \cup K \not\leq G$.

        \item We will disprove this statement. Since $H \cup K$ is not closed it is not a group, meaning it cannot be a subgroup.
    \end{partquestions}

    \item By Lagrange's Theorem (\myref{thrm-lagrange}), the order of a subgroup must divide the order of the group. Since $H \leq G$ is non-trivial, and since $1024 = 2^{10}$, the largest order that $H$ can be is $512$ with $[G:H] = 2$. An example is $G = \mathbb{Z}_{1024}$ and $H = \langle 2 \rangle$, since $|H| = |2| = 512$ as $2 \times 512 = 1024 \equiv 0 \pmod{1024}$.

    \item Let $|G| = 2n$. The identity is its own inverse, leaving an odd number of non-identity elements.
    
    Suppose $x$ is an element of $G$ with $|x| > 2$; we cannot have $x^{-1} = x$ (otherwise $x^2 = e$). Thus $x^{-1}$ and $x$ are distinct. Pair every one of these $x$'s with its inverse $x^{-1}$.

    Remember that there is an odd number of non-identity elements. Hence, there must be at least one element which has not been paired off with any of the others, which is therefore its own self inverse.

    Since this element is not the identity, thus it has to have order 2 (as $g^{-1} = g$ implies $g^2 = e$).

    \item Suppose $G = \langle g \rangle$ and $H \leq G$. Then any element in $H$ is of the form $g^a$ where $a$ is an integer. Suppose $m$ is the smallest positive integer $m$ such that $g^m \in H$. Suppose now $g^n \in H$ for some $n$. By Euclid's division lemma (\myref{lemma-euclid-division}), $n = mq + r$ where $q$ and $r$ are non-negative integers such that $0 \leq r < m$. Hence,
    \[
        g^n = g^{mq}g^r = (g^m)^q g^r.
    \]
    Now, $m$ is the smallest positive integer such that $g^m \in H$. This means that if $r \neq 0$, $g^r \not\in H$ as $0 \leq r < m$. Hence, $r = 0$, which means
    \[
        g^n = (g^m)^q.
    \]
    Thus, every element in the subgroup $H$ can be formed by applying $g^m$ a certain number of times, meaning $H$ is cyclic with generator $g^m$.

    \item \begin{partquestions}{\roman*}
        \item Let $xH$ be a coset in $G$. Since cosets partition $G$, either $xH = H$ or $xH = G \setminus H$ (since there are only two distinct cosets).
    \begin{itemize}
            \item If $xH = H$, then $x \in H$, meaning $xH = H = Hx$.
            \item If $xH \neq H$, then $x \in G \setminus H$. Hence $xH = G \setminus H = Hx$.
    \end{itemize}
    Therefore $H$ is a normal subgroup of $G$, i.e. $H \lhd G$.
        \item Note that $G/H$ is a group since $H \lhd G$. Also since $[G:H] = 2$ thus $|G/H| = 2$.
        
        If $x \in G$ then $x^2H = (xH)^2 = H$ since the order of $G/H$ is 2, meaning that any non-identity element inside it (like $xH$) has an order of at most 2. Since $x^2H = H$, therefore by Coset Equality (\myref{lemma-coset-equality}), statements 1 and 4, we must have $x^2 \in H$.
        
        \item Suppose $x$ has odd order. Write $|x| = 2k - 1$ where $k$ is a positive integer. Hence $x^{2k-1} = e \in H$ since $H < G$. Therefore
        \[
            x = x^{2k} = \left(x^k\right)^2 \in H        
        \]
        which means $x \in H$.
    \end{partquestions}

    \item \begin{partquestions}{\alph*}
        \item Suppose we have an element $x \in H \cap K$, meaning that $x \in H$ and $x \in K$. By a corollary of Lagrange's Theorem (\myref{corollary-order-of-group-multiple-of-order-of-element}), the order of $x$ must divide the order of its group. Hence, $|x|$ divides $|H|$ and $|x|$ divides $|K|$ simultaneously, meaning that $|x| = \gcd(|H|, |K|)$. But the GCD of the orders of both subgroups is 1. Hence, $|x| = 1$, meaning the only element in the intersection $H \cap K$ is the identity $e$.
        
        \item Consider $hkh^{-1}k^{-1}$.
        \begin{itemize}
            \item On one hand, note that $hkh^{-1}k^{-1} = h(kh^{-1}k^{-1})$. Clearly $h \in H$ and $kh^{-1}k^{-1} \in H$ by normality of $H$. Therefore $hkh^{-1}k^{-1} \in H$.
            \item On another hand, $hkh^{-1}k^{-1} = (hkh^{-1})k^{-1}$. Note $hkh^{-1} \in K$ by normality of $K$ and $k^{-1} \in K$, so $hkh^{-1}k^{-1} \in K$.
        \end{itemize}
        Therefore $hkh^{-1}k^{-1} \in H \cap K$. But by \textbf{(a)}, the only element in $H \cap K$ is the identity. Thus, $hkh^{-1}k^{-1} = e$ which the result follows quickly.
    \end{partquestions}
    
    \item \begin{partquestions}{\alph*}
        \item $m = 6$.
        \item We first prove that all groups of order less than 6 are abelian, and then find a non-abelian group of order 6.

        We note that a group of order 1 is the trivial group which is abelian. The groups of order 2, 3, and 5 are groups of prime order, meaning that they are cyclic and hence abelian. We are left with a group of order 4.

        We note that the order of an element of a group of order 4 must divide 4 (\myref{corollary-order-of-group-multiple-of-order-of-element}). Hence the possible orders of an element in such a group is 1, 2, or 4. An element of order 1 is the identity. If an element with order 4 exists, then the group is cyclic and hence abelian. So we assume that all elements are either order 1 or order 2 (in fact, the orders must be 1, 2, 2, 2). This is precisely the group
        \[
            D_2 = \langle r, s \vert r^2 = s^2 = e, rs = sr\rangle
        \]
        which is abelian. Hence all groups of order 4 are abelian.

        We now show that a group of order 6 can be non-abelian. We note that the group
        \[
            D_3 =  \langle r, s \vert r^3 = s^2 = e, rs = sr^2\rangle
        \]
        has order 6 and because $rs = sr^2 \neq sr$, thus $D_3$ is non-abelian. Hence $m = 6$.

        \item For all even $n \geq 6$, the group $D_{\frac n2}$ has $n$ elements and $rs = sr^{\frac n2 - 1} \neq sr$, so $D_{\frac n2}$ is non-abelian.
    \end{partquestions}

    \item Suppose $G / \Z{G}$ is cyclic. Then by definition, $G / \Z{G} = \langle g\Z{G}\rangle$ for some $g \in G$, and any element in $G/\Z{G}$ is of the form $g^n\Z{G}$.

    Now take $x, y \in G$. By \myref{lemma-left-coset-partition}, left cosets partition the group, so we may assume $x \in g^m\Z{G}$ and $y \in g^n\Z{G}$, meaning $x = g^mz_1$ and $y = g^nz_2$ for some $z_1, z_2 \in \Z{G}$. We note
    \begin{align*}
        xy &= (g^mz_1)(g^nz_2)\\
        &= g^m(z_1g^n)z_2\\
        &= g^m(g^nz_1)z_2 & (\text{since }z_1 \in \Z{G})\\
        &= (g^mg^n)(z_1z_2)\\
        &= g^{m+n}z_1z_2\\
        &= g^{n+m}z_2z_1\\
        &= g^ng^mz_2z_1\\
        &= g^n(g^mz_2)z_1\\
        &= g^n(z_2g^m)z_1\\
        &= (g^nz_2)(g^mz_1)\\
        &= yx
    \end{align*}
    which means that $xy = yx$ for any $x, y \in G$. Hence $G$ is abelian.
\end{questions}

\section{Homomorphisms and Isomorphisms}
\begin{questions}
    \item We will prove that $f$ is a homomorphism, is injective, and is surjective.
    \begin{itemize}
        \item \textbf{Homomorphism}: Let $x, y \in G$. Then
        \begin{align*}
            f(xy) &= g(xy)g^{-1}\\
            &= (gxg^{-1})(gyg^{-1})\\
            &= f(x)f(y)
        \end{align*}
        which means that $f$ is a homomorphism.
        \item \textbf{Injective}: Let $x, y \in G$ be such that $f(x) = f(y)$. Then $gxg^{-1} = gyg^{-1}$. By cancellation law, $x = y$.
        \item \textbf{Surjective}: Suppose $y \in G$. Set $x = g^{-1}yg$. Since $G$ is closed, thus $x \in G$. Note $f(x) = g(g^{-1}yg)g^{-1} = y$. Hence $y$ has a pre-image of $x = g^{-1}yg$ in $G$.
    \end{itemize}
    Therefore $f$ is an isomorphism.

    \item Suppose on the contrary there exists an isomorphism $\phi: G \to H$. Since $\phi$ is an isomorphism, it is surjective. Hence, there must exists a rational number $r \in G$ such that $\phi(r) = 2$. As $r$ is rational, so is $\frac r2$.

    Now consider $\phi\left(\frac r2 + \frac r2\right)$. On one hand, $\phi\left(\frac r2 + \frac r2\right) = \phi(r) = 2$. On another hand, $\phi(\frac r2 + \frac r2) = \left(\phi\left(\frac r2\right)\right)^2$ as $\phi$ is a homomorphism. Therefore, $\left(\phi\left(\frac r2\right)\right)^2 = 2$ which quickly implies $\phi\left(\frac r2\right) = \sqrt 2$ since $\phi\left(\frac r2\right)$ must be positive. However, $\sqrt 2 \notin H$ while $\phi\left(\frac r2\right) \in H$, a contradiction.

    Hence, $G \not\cong H$.

    \item \begin{partquestions}{\alph*}
        \item Let $m, n \in G$. Then
        \[
            \phi(m + n) = 2(m + n) = 2m + 2n = \phi(m) + \phi(n)
        \]
        which means $\phi$ is a homomorphism.

        \item Suppose $m, n \in G$ such that $\phi(m) = \phi(n)$. Then $2m = 2n$. Clearly this means that $m = n$. Thus $\phi$ is injective.

        \item Suppose on the contrary there existed a homomorphism $\psi: H \to G$ such that $\psi(\phi(n)) = n$. Then $\psi(2n) = n$ by definition of $\phi$. Note that
        \[
            \psi(2n) = \psi(n + n) = \psi(n) + \psi(n) = 2\psi(n)
        \]
        since $\psi$ is a homomorphism. Hence $2\psi(n) = n$ which implies that $\psi(n) = \frac n2$. But for the case of $n = 1$, $\psi(1) = \frac 12 \notin G$. Hence $\psi$ does not exist.
    \end{partquestions}

    \item We prove the forward direction first: assume that $G$ is abelian. Then $f$ is a homomorphism since
    \[
        f(gh) = (gh)^{-1} = h^{-1}g^{-1} = g^{-1}h^{-1} = f(g)h(g).
    \]

    We now prove the reverse direction: assume that $f$ is a homomorphism, meaning $f(gh) = f(g)f(h) = g^{-1}h^{-1}$. But $f(gh) = (gh)^{-1} = h^{-1}g^{-1}$. Therefore we have $g^{-1}h^{-1} = h^{-1}g^{-1}$ which clearly shows that the group is abelian.

    \item Suppose $\phi: G \to H$ is a surjective homomorphism and $G$ is abelian. Since $\phi$ is surjective, thus $\im \phi = H$. Let $g_1, g_2 \in G$ and $h_1, h_2 \in H$ such that $\phi(g_1) = h_1$ and $\phi(g_2) = h_2$. Consider $\phi(g_1g_2)$.
    \begin{itemize}
        \item On one hand, $\phi(g_1g_2) = \phi(g_1)\phi(g_2) = h_1h_2$.
        \item On another hand, $\phi(g_1g_2) = \phi(g_2g_1) = \phi(g_2)\phi(g_1) = h_2h_1$.
    \end{itemize}
    Hence $h_1h_2 = h_2h_1$ which means that $H$ is abelian.

    \item We first prove $\phi(N)$ is a subgroup of $H$ by using subgroup test before proving normality.

    Note that $e_H \in \phi(N)$ since $e_G \in N$ and $\phi(e_G) = e_H$. Now let $x, y \in \phi(N)$. As $\phi$ is surjective, we know that there exists $n_x, n_y \in N$ where $\phi(n_x) = x$ and $\phi(n_y) = y$. Note that $\phi(n_y^{-1}) = y^{-1}$ and $n_xn_y^{-1} \in N$. Hence, $xy^{-1} = \phi(n_xn_y^{-1}) \in \phi(N)$. By subgroup test, $\phi(N) \leq H$.

    We now show that $\phi(N)$ is a normal subgroup of $H$. Take $g \in G$, $h \in H$, $n \in N$, and $x \in \phi(N)$, such that $\phi(g) = h$ and $\phi(n) = x$. Note that since $N \unlhd G$, thus $gng^{-1} \in N$. Therefore,
    \begin{align*}
        hxh^{-1} &= \phi(g)\phi(n)\phi(g^{-1})\\
        &= \phi(\underbrace{gng^{-1}}_{\text{In }N})\\
        &\in \phi(N)
    \end{align*}
    which means that $\phi(N) \unlhd H$.

    \item Consider the map $\phi: G \to H, a \mapsto a + n\mathbb{Z}$. We show that $\phi$ is an isomorphism:
    \begin{itemize}
        \item \textbf{Homomorphism}: Let $a$ and $b$ be in $G$. Then
        \begin{align*}
            \phi(a\oplus_n b) &= (a\oplus_n b) + n\mathbb{Z}\\
            &= \{(a \oplus_n b) + pn \vert p \in \mathbb{Z}\}\\
            &= \{a+b + pn \vert p \in \mathbb{Z}\}\\
            &= \{a+b + pn + qn\vert p, q \in \mathbb{Z}\}\\
            &= a+b+n\mathbb{Z} + n\mathbb{Z}\\
            &= (a+n\mathbb{Z}) + (b + n\mathbb{Z})\\
            &= \phi(a) + \phi(b).
        \end{align*}
        \item \textbf{Injective}: Let $a$ and $b$ be in $G$ such that $\phi(a) = \phi(b)$. Thus
        \[
            \{a + pn \vert p \in \mathbb{Z} \} = \ \{b + qn \vert q \in \mathbb{Z} \}
        \]
        by definition of $\phi$. Hence $a \equiv b \pmod n$. But since $0 \leq a, b < n$, we must have $a = b$.
        \item \textbf{Surjective}: Let $x + n\mathbb{Z} \in H$. We use Euclid's division lemma (\myref{lemma-euclid-division}) on $x$ to yield
        \[
            x = qn + r, \text{ where } 0 \leq r < n.
        \]
        Note that
        \begin{align*}
            x + n\mathbb{Z} &= \{x + kn \vert k \in \mathbb{Z}\}\\
            &= \{(qn + r) + kn \vert k \in \mathbb{Z}\}\\
            &= \{r + n(\underbrace{q + k}_{\text{In }\mathbb{Z}}) \vertalt k \in \mathbb{Z} \}\\
            &= r + n\mathbb{Z}
        \end{align*}
        with $0 \leq r < n$, meaning $r \in G$. Now observe $\phi(r) = r+n\mathbb{Z} = x+n\mathbb{Z}$ which means that there is a pre-image for every element in $H$, hence proving that $\phi$ is surjective.
    \end{itemize}
    Therefore $\phi$ is an isomorphism, proving $G \cong H$.
    
    \item Consider the map $\phi: G \to G/N$ such that $g \mapsto gN$. We note that $\phi$ is a homomorphism as
    \[
        \phi(gh) = (gh)N = (gN)(hN) = \phi(g)\phi(H).
    \]
    We note by \myref{prop-homomorphism-inverse-is-subgroup} that $A = \phi^{-1}(B) \leq G$. Thus
    \begin{align*}
        \phi^{-1}(N) &= \{g \in G \vert \phi(g) = N\}\\
        &= \{g \in G \vert gN = N\}\\
        &= \{g \in G \vert g \in N\}\\
        &= G \cap N\\
        &= N\\
        &\subseteq A
    \end{align*}
    by assumption. Since $N$ is a group, we know $N \leq A$. Furthermore $N \leq A \leq G$ and $N \unlhd G$, meaning $N \unlhd A$ (since $gN = Ng$ for all $g \in G$, including those in $A$). Hence $A/N$ is a group.
    
    Now clearly $\phi$ is surjective (since for any $gN \in G/N$ we know $\phi(g) = gN$), which means that $\phi(\phi^{-1}(B)) = B$. Since $\phi^{-1}(B) = A$, so $\phi(A) = B$. Finally,
    \begin{align*}
        \phi(A) &= \{\phi(a) \vert a \in A\}\\
        &= \{aN \vert a \in A\}\\
        &= A/N
    \end{align*}
    which means $B = A/N$.
\end{questions}

\section{Symmetric Groups}
\begin{questions}
    \item We work from the right to the left.
    \begin{itemize}
        \item $\gamma \delta$ has cycle notation
        \begin{align*}
            &\begin{pmatrix}1 & 2 & 5\end{pmatrix}\begin{pmatrix}3 & 4\end{pmatrix}\begin{pmatrix}1 & 3 & 2 & 5\end{pmatrix}\\
            &= \begin{pmatrix}1 & 4 & 3 & 5 & 2\end{pmatrix};
        \end{align*}
        \item $\beta \gamma \delta$ has cycle notation
        \begin{align*}
            &\begin{pmatrix}1 & 5 & 2\end{pmatrix}\begin{pmatrix}3 & 4\end{pmatrix}\begin{pmatrix}1 & 4 & 3 & 5 & 2\end{pmatrix}\\
            &= \begin{pmatrix}1 & 3 & 2 & 5\end{pmatrix}\\
            &= \delta;
        \end{align*}
        and
        \item $\alpha \beta \gamma \delta$ has cycle notation
        \begin{align*}
            &\begin{pmatrix}1 & 5 & 2 & 3\end{pmatrix}\begin{pmatrix}1 & 3 & 2 & 5\end{pmatrix}\\
            &= \id,
        \end{align*}
        the identity.
    \end{itemize}

    \item Recall that $D_3$ has presentation
    \[
        \langle r, s \vert r^3 = s^2 = e, rs = sr^2 \rangle.
    \]

    Let the map $\phi: D_3 \to \Sn{3}$ be given such that $r \mapsto \begin{pmatrix}1 & 2 & 3\end{pmatrix}$ and $s \mapsto \begin{pmatrix}1 & 2\end{pmatrix}$. We show that $\begin{pmatrix}1 & 2 & 3\end{pmatrix}$ and $\begin{pmatrix}1 & 2\end{pmatrix}$ satisfy the two rules above. For brevity let $\sigma = \begin{pmatrix}1 & 2 & 3\end{pmatrix}$ and $\tau = \begin{pmatrix}1 & 2\end{pmatrix}$.
    \begin{itemize}
        \item We check that $\phi(r^3) = \phi(s^2) = \phi(e)$.
        \begin{itemize}
            \item $\sigma^2 = \begin{pmatrix}1 & 2 & 3\end{pmatrix}\begin{pmatrix}1 & 2 & 3\end{pmatrix} = \begin{pmatrix}1 & 3 & 2\end{pmatrix} \neq \id$;
            \item $\sigma^3 = \begin{pmatrix}1 & 2 & 3\end{pmatrix}\begin{pmatrix}1 & 3 & 2\end{pmatrix} = \id$; and
            \item $\tau^2 = \begin{pmatrix}1 & 2\end{pmatrix}\begin{pmatrix}1 & 2\end{pmatrix} = \id$.
        \end{itemize}
        \item We check that $\phi(rs) = \phi(sr^2)$.
        \begin{itemize}
            \item $rs \mapsto \sigma\tau = \begin{pmatrix}1 & 2 & 3\end{pmatrix}\begin{pmatrix}1 & 2\end{pmatrix} = \begin{pmatrix}1 & 3\end{pmatrix}$; and
            \item $sr^2 \mapsto \tau\sigma^2 = \begin{pmatrix}1 & 2\end{pmatrix}\begin{pmatrix}1 & 3 & 2\end{pmatrix} = \begin{pmatrix}1 & 3\end{pmatrix}$.
        \end{itemize}
    \end{itemize}
    Thus $\phi$ is an isomorphism and so $D_3 \cong \Sn{3}$.

    \item We note that $|\Sn{4}| = 4! = 24$.
    \begin{partquestions}{\alph*}
        \item Consider $H = \left\langle \begin{pmatrix}1 & 2 & 3 & 4\end{pmatrix} \right\rangle$. For brevity, let $\sigma = \begin{pmatrix}1 & 2 & 3 & 4\end{pmatrix}$. Note that
        \begin{itemize}
            \item $\sigma^2 = \begin{pmatrix}1 & 3\end{pmatrix}\begin{pmatrix}2 & 4\end{pmatrix} \neq \id$;
            \item $\sigma^3 = \begin{pmatrix}1 & 4 & 3 & 2\end{pmatrix} \neq \id$; and
            \item $\sigma^4 = \id$.
        \end{itemize}
        Thus, $|\sigma| = 4$ which means $|H| = 4$. Therefore, $G \cong H \leq \Sn{4}$.

        \item Let $\sigma = \begin{pmatrix}1 & 2\end{pmatrix}$ and $\tau = \begin{pmatrix}3 & 4\end{pmatrix}$. Let $H$ have presentation $\langle \sigma, \tau \rangle$. Notice that
        \begin{itemize}
            \item $\sigma^2 = \id$;
            \item $\tau^2 = \id$; and
            \item $(\sigma\tau)^2 = \id$.
        \end{itemize}
        Therefore $H = \{\id, \sigma, \tau, \sigma\tau\}$, so $G \cong H \leq \Sn{4}$.
    \end{partquestions}
\end{questions}

\section{Direct Products of Groups}
\begin{questions}
    \item Let $g_1, g_2 \in G$ and $h_1, h_2 \in H$. Then for $(g_1, h_1), (g_2, h_2) \in G\times H$ we see that
    \begin{align*}
        (g_1, h_1)(g_2, h_2) &= (g_1g_2, h_1h_2)\\
        &= (g_2g_1, h_2h_1)\\
        &= (g_2,h_2)(g_1,h_1)
    \end{align*}
    which means that $G \times H$ is abelian.

    \item Let the map $\phi: G\times H \to H \times G, (g, h) \mapsto (h, g)$. We prove that $\phi$ is an isomorphism:
    \begin{itemize}
        \item \textbf{Homomorphism}: Let $(g_1, h_1), (g_2, h_2) \in G \times H$. We note that
        \begin{align*}
            \phi((g_1, h_1)(g_2, h_2)) &= \phi((g_1g_2, h_1h_2))\\
            &= (h_1h_2, g_1g_2)\\
            &= (h_1, g_1)(h_2, g_2)\\
            &= \phi((g_1, h_1))\phi((g_2, h_2))
        \end{align*}
        which proves that $\phi$ is a homomorphism.
        \item \textbf{Injective}: Suppose there exists $(g_1, h_1), (g_2, h_2) \in G \times H$ such that $\phi((g_1, h_1)) = \phi((g_2, h_2))$. Then by definition of $\phi$ we have $(h_1, g_1) = (h_2, g_2)$. Clearly by comparing component parts of each ordered pair, we have $g_1 = g_2$ and $h_1 = h_2$, meaning $(g_1, h_1) = (g_2, h_2)$. Hence $\phi$ is injective.
        \item \textbf{Surjective}: Let $(h, g) \in H \times G$. Clearly $(g, h) \in G \times H$ and $\phi((g, h)) = (h, g)$, meaning that $(h, g)$ has a pre-image of $(g, h)$. Therefore $\phi$ is surjective.
    \end{itemize}
    Therefore $\phi$ is an isomorphism, meaning $G \times H \cong H \times G$.

    \item We claim that $G$ is the internal direct product of $H$ and $K$. We need to check 3 things.
    \begin{itemize}
        \item $\boxed{G = HK}$ We note that
        \begin{align*}
            HK &= \{h \oplus_6 k \vert h \in H, k \in K\}\\
            &= \{0 \oplus_6 0, 0 \oplus_6 3, 2 \oplus_6 0, 2 \oplus_6 3, 4 \oplus_3 0, 4 \oplus_3 3\}\\
            &= \{0, 3, 2, 5, 4, 1\}\\
            &= \mathbb{Z}_6\\
            &= G
        \end{align*}
        so in fact $G = HK$.

        \item $\boxed{H \cap K = \{e\}}$ Clearly $H \cap K = \{0\}$.

        \item $\boxed{hk = kh}$ Since $\oplus_6$ is commutative, thus $h \oplus_6 k = k \oplus_6$.
    \end{itemize}
    Thus $G$ is the internal direct product of $H$ and $K$.

    \item Define the subgroups $H = \{e, a\}$ and $K = \{e, b\}$. We show the $\mathrm{V}$ is the internal direct product of $H$ and $K$.
    \begin{itemize}
        \item $\boxed{\mathrm{V} = HK}$ Observe that
        \begin{align*}
            HK &= \{hk \vert h \in H, k \in K\}\\
            &= \{ee, eb, ae, ab\}\\
            &= \{e, b, a, ab\}\\
            &= \mathrm{V}
        \end{align*}
        so in fact $\mathrm{V} = HK$.

        \item $\boxed{H \cap K = \{e\}}$ Clearly $H \cap K = \{e\}$.

        \item $\boxed{hk = kh}$ Clearly if one of the elements is the identity then result follows. So assume that $h$ and $k$ are both non-identity elements, so $h = a$ and $k = b$. Note
        \begin{align*}
            kh &= ba\\
            &= (ba)\left((ab)(ab)\right) & (\text{since }(ab)^2 = e)\\
            &= (ba ab)(ab)\\
            &= (bb)(ab) & (\text{since }a^2 = e)\\
            &= ab & (\text{since }b^2 = e)\\
            &= hk
        \end{align*}
        so in fact $hk = kh$ for all $h \in H$, $k \in K$.
    \end{itemize}
    Therefore $\mathrm{V}$ is the internal direct product of $H$ and $K$. 
    
    We note $H = \langle a\rangle \cong \mathbb{Z}_2$ and $K = \langle b \rangle \cong \mathbb{Z}_2$. By direct product equivalence (\myref{thrm-direct-product-equivilance}) we know $\mathrm{V} \cong H \times K \cong \mathbb{Z}_2 \times \mathbb{Z}_2 = (\mathbb{Z}_2)^2$.
\end{questions}

\chapter{Further Properties of Homomorphisms}
Earlier in this book, we introduced homomorphisms and isomorphisms, special types of maps that transform elements of one group to another. We look at more properties of such maps in this chapter and describe the uses of these new properties.

\section{Image of a Homomorphism}
As a homomorphism is a mapping between two groups, it is worthy to look at the \textbf{image} of the homomorphism.
\begin{definition}
    The \textbf{image}\index{homomorphism!image} (or \textbf{range}\index{homomorphism!range}) of a homomorphism $\phi: G \to H$ is the set
    \[
        \im\phi = \{\phi(g) \vert g \in G\}.
    \]
\end{definition}
\begin{remark}
    Some authors (e.g. {\cite[Definition 4.2.0]{libretexts_im-and-ker}}) will use the notation $\phi(G)$ for the image of $\phi$. The alternate notation $\mathrm{Im}\;\phi$ may also be used (e.g. by {\cite[Definition I.2.2]{hungerford_1980}} and \cite[\S 66]{clark_1984}).
\end{remark}

\begin{example}
    Consider the homomorphism $f: \Z \to \Z, x \mapsto 0$. Clearly, all possible values of $x$ maps to 0, so $\im f = \{0\}$.
\end{example}
\begin{example}
    The homomorphism $f: \R \to \R$ where $f(x) = |x|$ has an image of $\{x \in \R \vert x \geq 0\}$, i.e. all non-negative real numbers.
\end{example}

\begin{proposition}\label{prop-image-is-subgroup-of-codomain}
    Let $\phi: G \to H$ be a homomorphism. Then $\im\phi \leq H$.
\end{proposition}
\begin{proof}
    Note that $\phi(e_G) = e_H \in \im\phi$, where $e_G$ and $e_H$ are the identities of $G$ and $H$ respectively.
    
    Now suppose $h_1$ and $h_2$ are in the image of $\phi$, meaning that there exists $g_1$ and $g_2$ such that $\phi(g_1) = h_1$ and $\phi(g_2) = h_2$. Note that $\phi(g_2^{-1}) = h_2^{-1}$ by homomorphism property. Hence $\phi(g_1g_2^{-1}) = h_1h_2^{-1} \in \im\phi$.

    Therefore, by subgroup test, $\im\phi \leq H$.
\end{proof}

\begin{exercise}
    Consider the map $\phi: \Z_3 \to \Z_6, n \mapsto 2n$. Determine whether $\phi$ is a homomorphism and, if so, find its image.
\end{exercise}

\section{Kernel of a Homomorphism}
\begin{definition}
    The \textbf{kernel}\index{homomorphism!kernel} of a homomorphism $\phi: G \to H$ is the set of elements in the group $G$ which map to the identity in the group $H$. That is, if the identity of $H$ is $e_H$, then the kernel of $\phi$ is the set
    \[
        \ker\phi = \{x \in G \vert \phi(x) = e_H\}.
    \]
\end{definition}


\begin{remark}
    Some authors (e.g. {\cite[Definition 4.2.0]{libretexts_im-and-ker}}) will use the notation $\phi^{-1}(e_H)$ for the kernel of $\phi$. The alternate notation $\mathrm{Ker}\;\phi$ may also be used by some authors (e.g. by {\cite[Definition I.2.2]{hungerford_1980}} and \cite[\S 65]{clark_1984}).
\end{remark}

\begin{example}
    Let the groups $G = (\Z^2, (+, +))$ and $H = (\Z, +)$. Let the map $\phi: G \to H, (a, b) \mapsto a+b$. Then, $(a, b) \in \ker\phi$ if $\phi((a,b)) = 0$. This means that $a+b = 0$, implying $ b = -a$. Hence the kernel of $\phi$ is $\{(a, -a) \vert a \in \Z\}$.
\end{example}

\begin{exercise}
    Let $i$ be the imaginary unit, that is $i^2 = -1$. Let the group $G$ be the integers under addition and $H = \langle i \rangle$ be under multiplication. Let the map $\phi: G \to H, n \mapsto i^n$.  Show that $\phi$ is a homomorphism and hence find $\ker\phi$.
\end{exercise}

\begin{proposition}\label{prop-kernel-is-normal-subgroup-of-domain}
    Let $\phi: G \to H$ be a homomorphism. Then $\ker\phi \unlhd G$.
\end{proposition}
\begin{proof}
    We will first show $\ker\phi\leq G$. Clearly $e_G \in \ker\phi$ since $\phi(e_G) = e_H$, so $\ker\phi$ is non-empty. Now let $x, y \in \ker\phi$. This means that $\phi(x) = \phi(y) = e_H$. Note
    \begin{align*}
        \phi(xy^{-1}) &= \phi(x)\left(\phi(y)\right)^{-1}\\
        &= e_H(e_H)^{-1}\\
        &= e_H
    \end{align*}
    which means that $xy^{-1}\in\ker\phi$. By subgroup test, $\ker\phi\leq G$.

    Now we prove normality. Let $x \in G$ and $n \in \ker\phi$. We need to show that $xnx^{-1}\in\ker\phi$ to prove normality. Observe that
    \begin{align*}
        \phi(xnx^{-1}) &= \phi(x)\phi(n)\phi(x^{-1})\\
        &= \phi(x)e_H\phi(x)^{-1} & (n \in \ker\phi)\\
        &= \phi(x)\phi(x)^{-1}\\
        &= e_H,
    \end{align*}
    which means that $xnx^{-1} \in \ker\phi$. Hence, $\ker\phi \unlhd G$.
\end{proof}

\begin{exercise}\label{exercise-trivial-kernel-means-injective}
    Prove that a homomorphism $\phi:G\to H$ is injective if and only if $\ker \phi$ is trivial, i.e. $\ker \phi = \{e_G\}$.
\end{exercise}

\section{The Fundamental Homomorphism Theorem}
We are now ready to tackle the three most important theorems regarding homomorphisms. We first state the \textbf{Fundamental Homomorphism Theorem}, which is also sometimes called the \textbf{First Isomorphism Theorem} (e.g. in {\cite[p.~251, Theorem 3]{cohn_1982}}).
\begin{theorem}[Fundamental Homomorphism Theorem]\label{thrm-isomorphism-1}\index{Fundamental Homomorphism Theorem}\index{Isomorphism Theorem!First}
    Let $G$ and $H$ be groups. Let $\phi: G \to H$ be a homomorphism, and let $\pi: G \to G/\ker\phi$ where $g\mapsto g\ker\phi$ be the natural surjective homomorphism. Then there exists a unique isomorphism $\psi: G/\ker\phi \to \im\phi$ such that $\psi\pi = \phi$.
\end{theorem}
\begin{remark}
    Equivalently, the Fundamental Homomorphism Theorem states that
    \[
        G/\ker\phi \cong \im\phi
    \]
    for any homomorphism $\phi$.
\end{remark}

We include the commutativity diagram of the homomorphisms stated to aid clarity:

\begin{figure}[h]
    \centering
    \fbox{\includegraphics[width=0.35\textwidth]{further-homomorphisms/iso-1-comm-diagram.png}}
    \caption{Commutativity Diagram for \myreffigures{thrm-isomorphism-1}}
\end{figure}

In the diagram, $\phi$ sends elements from $G$ to $\im\phi$ and $\pi$ sends elements from $G$ to $G/\ker\phi$. Then the map $\psi$ is a unique map that sends elements from $G/\ker\phi$ to the image of $\phi$.

\begin{proof}
    We know by \myref{prop-image-is-subgroup-of-codomain} that $\im\phi \leq H$. Let $\psi: G/\ker\phi \to \im\phi$ such that $\psi(x\ker\phi) = \phi(x)$. We need to check that $\psi$ is a well-defined isomorphism.
    \begin{itemize}
        \item \textbf{Well-defined}: Suppose $x\ker\phi = y\ker\phi$ where $x, y \in G$. Then $xy^{-1} \in \ker\phi$ by Coset Equality (\myref{lemma-coset-equality}), statements 1 and 5. This means that $\phi(xy^{-1}) = e_H$ by definition of the kernel. Note $\phi(xy^{-1}) = \phi(x)\left(\phi(y)\right)^{-1}$, so $\phi(x)\left(\phi(y)\right)^{-1} = e_H$. Hence $\phi(x) = \phi(y)$. Thus,
        \[
            \psi(x\ker\phi) = \phi(x) = \phi(y) = \psi(y\ker\phi)
        \]
        so $\psi$ is well-defined.

        \item \textbf{Homomorphism}: Note that
        \begin{align*}
            \psi((x\ker\phi)(y\ker\phi)) &= \psi((xy)\ker\phi)\\
            &= \phi(xy)\\
            &= \phi(x)\phi(y)\\
            &= \psi(x\ker\phi)\psi(y\ker\phi)
        \end{align*}
        so $\psi$ is a homomorphism.
        \item \textbf{Injective}: By \myref{exercise-trivial-kernel-means-injective}, we check that $\psi$ is injective by showing that $\ker\psi$ is trivial, i.e. $\ker\psi = \{\ker\phi\}$.

        Suppose $x\ker\phi\in\ker\psi$. Then $\psi(x\ker\phi) = e_H$ by definition of kernel. Hence $\phi(x) = e_H$ by definition of $\psi$, which means $x \in \ker\phi$ by definition of kernel. Thus $x\ker\phi = \ker\phi$ by Element in Coset (\myref{corollary-equivalence-of-element-in-coset}). Therefore $\psi$ is injective.

        \item \textbf{Surjective}: Suppose $y$ is in the image of $\phi$, meaning there exists a $x \in G$ such that $\phi(x) = y$. Note that $\psi(x\ker\phi) = \phi(x) = y$. Thus $\psi$ is surjective.
    \end{itemize}
    Thus $\psi$ is a well-defined isomorphism.

    We now check that $\psi$ satisfies the requirement that $\psi\pi = \phi$. Let $x \in G$. Note that $\pi(x) = x\ker\phi$, and
    \[
        \psi\pi(x) = \psi(x\ker\phi) = \phi(x)
    \]
    for all $x \in G$, so $\psi\pi = \phi$.

    Finally we show that $\psi$ is unique. Suppose $f: G/\ker\phi \to \im\phi$ is an isomorphism satisfying $f\pi=\phi$. Take $x\ker\phi \in G/\ker\phi$. Note that
    \begin{align*}
        f(x\ker\phi) &= f(\pi(x))\\
        &= (f\pi)(x)\\
        &= \phi(x)\\
        &= (\psi\pi)(x)\\
        &= \psi(\pi(x))\\
        &= \psi(x\ker\phi)
    \end{align*}
    for all $x \in G$, meaning that $f = \psi$. Therefore $\psi$ is unique.

    Hence, $\psi$ is a unique isomorphism satisfying $\psi\pi = \phi$.
\end{proof}

\begin{example}
    Let $R = \{x \in \R \vert x > 0\}$, $G = \{x \in \R \vert x \neq 0\}$, and $H = \{1, -1\}$ be groups under multiplication. We show $G / H \cong R$.

    Consider the map $\phi: G \to R$ where $x \mapsto |x|$. We show that $\phi$ is a homomorphism, then find the image of $\phi$, and finally find its kernel.

    \begin{itemize}
        \item \textbf{Homomorphism}: $\phi$ is a homomorphism since $\phi(xy) = |xy| = |x||y| = \phi(x)\phi(y)$.
        \item \textbf{Image}: We find the image of $\phi$.
        \begin{align*}
            \im\phi &= \{\phi(x) \vert x \in G\}\\
            &= \{|x| \vert x \neq 0\}\\
            &= \{x \in \R \vert x > 0\} & (\text{by definition of } |x|)\\
            &= R
        \end{align*}
        which actually means that $\phi$ is surjective.
        \item \textbf{Kernel}: We find the kernel of $\phi$.
        \begin{align*}
            \ker\phi &= \{x \in G \vert \phi(x) = 1\} & (1 \text{ is the identity in } R)\\
            &= \{x \in G \vert |x| = 1\}\\
            &= \{1, -1\}\\
            &= H
        \end{align*}
    \end{itemize}
    Thus $G/H \cong R$ by the Fundamental Homomorphism Theorem (\myref{thrm-isomorphism-1}).
\end{example}

\begin{exercise}
    Let $\phi: G \to H$ be a homomorphism between finite groups $G$ and $H$. Prove that
    \[
        |G| = |\im \phi|\times|\ker \phi|.
    \]
\end{exercise}

\section{The Diamond Isomorphism Theorem}
We now look at the next theorem, called the \textbf{Diamond Isomorphism Theorem} (e.g. in {\cite[Theorem 3.18]{dummit_foote_2004}}) or the \textbf{Second Isomorphism Theorem} (e.g. in {\cite[\S 69]{clark_1984}}).
\begin{theorem}[Diamond Isomorphism Theorem]\label{thrm-isomorphism-2}\index{Diamond Isomorphism Theorem}\index{Isomorphism Theorem!Second}
    Let $G$ be a group and let $H$ and $K$ be subgroups of $G$. Then
    \begin{enumerate}
        \item $H \cap K \leq H$; and
        \item $H \leq HK$.
    \end{enumerate}
    Furthermore, if $N \unlhd G$, then
    \begin{enumerate}[start=3]
        \item $HN \leq G$;
        \item $H \cap N \unlhd H$;
        \item $N \unlhd HN$; and
        \item $H / (H\cap N) \cong HN / N$.
    \end{enumerate}
\end{theorem}

\newpage

We can capture the overall relationships of the subgroups of $G$ using a \textbf{subgroup lattice}.
\begin{figure}[h]
    \centering
    \fbox{\includegraphics[width=0.25\textwidth]{further-homomorphisms/iso-2-subgroup-diagram.png}}
    \caption{Subgroup Lattice for \myreffigures{thrm-isomorphism-2}}
\end{figure}

We only show subgroups that we care about in the diagram. The group $G$ has a (direct) subgroup $HK$; $HK$ has subgroups $H$ and $K$; and $H$ and $K$ has a common subgroup $H\cap K$. The dotted quotient groups are isomorphic to each other if $H \unlhd G$.

\begin{proof}
    We prove each statement in sequence.

    \begin{enumerate}
        \item Clearly $e_G \in H$ and $e_G \in K$ so $e_G \in H \cap K$. Now take $x, y \in H \cap K$, meaning $x, y \in H$ and $x, y \in K$. Since $H, K \leq G$ so $xy^{-1} \in H$ and $xy^{-1} \in K$. Thus $xy^{-1} \in H \cap K$. By the subgroup test, this means that $H \cap K \leq H$.
        
        \item Note that $H = \{he_G \vert h \in H\} \subseteq \{hk \vert h \in H, k \in K\} = HK$, and $H$ is a group (as $H$ is a subgroup). Therefore $H \leq HK$.
        
        \item We note that, because $N$ is normal, hence $hN = Nh$ for all $h \in H \subseteq G$, meaning that $HN = NH$. Therefore by \myref{prop-subgroup-product-is-subgroup}, we have $HN \leq G$.

        \item We know $H \cap N \leq H$ by statement 1, so we only prove normality. Take $x \in H \cap N$. Since $H \leq G$, thus $x \in H \cap N \subseteq H$, meaning for all $g \in H$, $gxg^{-1} \in H$ (where we think of $g$ and $x$ as being in $H$). But since $x \in H \cap N \subseteq N$ and $N \unlhd G$, thus $gxg^{-1} \in N$ (where we think of $g \in H$ and $x \in N$). Therefore $H \cap N \unlhd H$.

        \item We know $N \leq HN$ by statement 2, so we only prove normality. Take $n \in N$ and $x \in HN$ such that $x = h_xn_x$. Then
        \begin{align*}
            xnx^{-1} &= (h_xn_x)n(h_xn_x)^{-1}\\
            &= (h_xn_x)n(n_x^{-1}h_x^{-1}) & (\text{Shoes and Socks})\\
            &= \underbrace{h_x}_{\text{In }G}\underbrace{n_xnn_x^{-1}}_{\text{In }N}\underbrace{h_x^{-1}}_{\text{In G}}\\
            &\in N
        \end{align*}
        since $N \unlhd G$. This proves that $N \unlhd HN$.

        \item This is the main result of this theorem. We define $\phi: H \to HN/N, h \mapsto hN$. We show that $\phi$ is a homomorphism and then find its image and kernel.
        \begin{itemize}
            \item \textbf{Homomorphism}:
            \[
                \phi(xy) = (xy)N = (xN)(yN) = \phi(x)\phi(y)    
            \]

            \item \textbf{Image}: We show that $\phi$ is surjective to show that $\im\phi = HN/N$. Suppose $x \in HN$, meaning $x = hn$ where $h \in H$ and $n \in N$. Thus $xN \in HN/N$, so
            \[
                xN = (hn)N = h(nN) = hN
            \]
            meaning $\phi(h) = hN = xN$. Hence we have found a pre-image of the coset $xN$, meaning $\phi$ is surjective. Thus $\im \phi = HN/N$.

            \item \textbf{Kernel}: We claim that $\ker\phi = H \cap N$.
            
            Note that $\ker\phi = \{h \in H \vert \phi(h) = eN = N\}$ by definition of kernel. This means that if $h \in \ker\phi$ then $\phi(h) = N$. Hence $\phi(h) = hN = N$, which means $h \in N$ by Element in Coset (\myref{corollary-equivalence-of-element-in-coset}). Thus, $h \in H$ and $h \in N$, meaning $h \in H \cap N$. Therefore $\ker \phi \subseteq H \cap N$.
    
            Now suppose $x \in H \cap N$. This means that $x \in N$ necessarily, implying $xN = N$. Thus $\phi(x) = N$ which quickly implies $x \in \ker\phi$. Therefore $H \cap N \subseteq \ker\phi$.
    
            Since $\ker \phi \subseteq H \cap N$ and  $H \cap N \subseteq \ker\phi$ therefore $\ker\phi = H\cap N$.
        \end{itemize}

        By the Fundamental Homomorphism Theorem (\myref{thrm-isomorphism-1}),
        \[
            H / \ker\phi \cong \im \phi,
        \]
        which means
        \[
            H/(H\cap N) \cong HN/N.
        \]
    \end{enumerate}
    This completes the proof of the theorem.
\end{proof}

\begin{corollary}\label{corollary-subgroup-product-is-normal-subgroup-if-subgroups-are-normal}
    Let $G$ be a group with proper subgroups $H$ and $K$. Then $HK \unlhd G$ if $H$ and $K$ are normal subgroups of $G$.
\end{corollary}
\begin{proof}
    Assume that $H, K \lhd G$. By the Diamond Isomorphism Theorem (\myref{thrm-isomorphism-2}), statement 3, we know that $HK \leq G$ since $H \lhd G$. We just need to prove normality. Suppose $hk \in HK$ and take $g \in G$. Then
    \begin{align*}
        g(hk)g^{-1} &= (gh)(kg^{-1})\\
        &= (hg)(g^{-1}k) & (\text{as } H, K \lhd G)\\
        &= h(gg^{-1})k\\
        &= hk \in HK
    \end{align*}
    which means that $HK \unlhd G$.
\end{proof}

We look at two examples using the Diamond Isomorphism Theorem.
\begin{example}
    We say a group $G$ is \textbf{metabelian}\index{metabelian} if and only if there exists $A \unlhd G$ such that $A$ and $G/A$ are both abelian. We will prove that any subgroup of a metabelian group is also metabelian.

    Let $H \leq G$. Then, by the Diamond Isomorphism Theorem (\myref{thrm-isomorphism-2}), we know $H \cap A \unlhd H$ (statement 4) and $H/(H \cap A) \cong HA / A$ (statement 6). We just need to prove that both $H \cap A$ and $H/(H \cap A)$ are abelian.
    \begin{itemize}
        \item Consider any two elements from $H \cap A$, say $x$ and $y$. Then $x \in A$ and $y \in A$, so $xy = yx$ as $A$ is abelian. Hence, elements from $H \cap A$ commute, meaning that $H \cap A$ is abelian.
        \item Consider $H/(H\cap A) \cong HA / A$. Note that $HA \leq G$ since $H \leq G$ and $A \leq G$. Thus, $HA / A \leq G / A$. Note that $G/A$ is abelian by definition of metabelian group. Hence, $H/(H \cap A)$ is also abelian.
    \end{itemize}
    Therefore, we have found a subgroup of $H$ (in particular $H \cap A$) such that both $H \cap A$ and $H/(H\cap A)$ are both abelian. Hence, $H$ is metabelian.
\end{example}

We look at another application of the Diamond Isomorphism Theorem, which has use in Number Theory.
\begin{example}
    We will prove that $\lcm(m,n)\times\gcd(m,n) = mn$ (\myref{prop-product-of-gcd-and-lcm}) by considering the Diamond Isomorphism Theorem. For brevity, let $d = \gcd(m,n)$ and $l = \lcm(m,n)$.

    Consider the groups $G = \Z$, $H = m\Z$, and $N = n\Z$ under addition. By Diamond Isomorphism Theorem (\myref{thrm-isomorphism-2}),
    \[
        m\Z/(m\Z \cap n\Z) \cong (m\Z + n\Z)/(n\Z).
    \]

    Now $m\Z \cap n\Z$ is the set of integers that are both a multiple of $m$ and $n$. Hence, $m\Z \cap n\Z = \lcm(m,n)\Z = l\Z$. On the other hand, $m\Z + n\Z$ is the set of all integers of the form $mx+ny$ where $x$ and $y$ are integers. B\'{e}zout's lemma (\myref{lemma-bezout}) tells us that this set consists of the multiples of $\gcd(m,n)$, i.e. $m\Z + n\Z = \gcd(m,n)\Z = d\Z$. Hence,
    \[
        m\Z/(l\Z) \cong d\Z/(n\Z).
    \]

    We claim that $m\Z / (l\Z) \cong \Z_{\frac lm} \text{ and } d\Z / (n\Z) \cong \Z_{\frac nd}$. This is a specific case of \myref{problem-mZ/nZ-isomorphic-to-Zn/m} which we have left as a problem for later. Hence,
    \[
    \Z_{\frac lm} \cong m\Z/(l\Z) \cong d\Z/(n\Z) \cong \Z_{\frac nd},
    \]
    which means that $\Z_{\frac lm} \cong \Z_{\frac nd}$. We can now finally take orders on both sides:
    \[
        \frac{l}{m} = \frac{n}{d},
    \]
    which means that $ld = mn$. Hence, $\lcm(m,n)\times\gcd(m,n) = mn$.
\end{example}

\begin{exercise}\label{exercise-order-of-subgroup-product}
    Let $G$ be a finite group, $H \leq G$, and $N \lhd G$. Prove that
    \[
        |HN| = \frac{|H||N|}{|H \cap N|}.
    \]
\end{exercise}

\section{The Third Isomorphism Theorem}
We look at the last important theorem regarding homomorphisms and isomorphisms. This is often called the \textbf{Third Isomorphism Theorem} (e.g. in {\cite[Corollary I.5.10]{hungerford_1980}} and {\cite[pp.~253--254, Theorem 5]{cohn_1982}}).

It should be noted that there is no consistency with the numbering of these theorems in books (cf. {\cite[\S 68]{clark_1984}} as ``First Isomorphism Theorem'', {\cite[Theorem 8.16]{humphreys_1996}} as ``Second Isomorphism Theorem''), but the name ``Third Isomorphism Theorem'' is the easiest to research. Hence, we use that name here.

\begin{theorem}[Third Isomorphism Theorem]\label{thrm-isomorphism-3}\index{Isomorphism Theorem!Third}
    Let $G$ be a group. Let $H \unlhd G$ and $N \unlhd G$. Suppose $N \subseteq H$. Then
    \begin{enumerate}
        \item $N \unlhd H$;
        \item $H/N \unlhd G/N$; and
        \item $\frac{G/N}{H/N} \cong G/H$.
    \end{enumerate}
\end{theorem}
\begin{proof}
    We prove the statements in sequence.

    \begin{enumerate}
        \item We note that since $N \subseteq H$ and $N$ is a group (since $N$ is a normal subgroup of $G$) thus $N \leq H$. We just need to prove normality. Since $H$ and $N$ are normal subgroups of $G$, thus for all $g \in G$,
        \[
            gH = Hg \text{ and } gN = Ng.
        \]
        Now since $N \subseteq H \subseteq G$, thus for all $n$ in $N$, $nH = Hn$ (since $n \in G$). This means that $N \unlhd H$.

        \item We first prove that it is a subgroup before proving normality.

        Clearly $N = eN \in H/N$. Let $x$ and $y$ be in $H/N$. Then $x=h_xN$ and $y=h_yN$ for some $h_x, h_y \in H$. Note that $y^{-1} = (h_y^{-1})N$ by group operator on cosets. Hence,
        \begin{align*}
            xy^{-1} &= (h_xN)(h_y^{-1}N)\\
            &= (\underbrace{h_xh_y^{-1}}_{\text{In }H})N\\
            &\in H/N
        \end{align*}
        Hence, by subgroup test, $H/N \leq G/N$.

        Now let $gN \in G/N$ and $hN \in H/N$. We need to show that $(gN)(hN)(gN)^{-1} \in H/N$. Note $(gN)(hN)(gN)^{-1} = (ghg^{-1})N$. Since $H \unlhd G$, thus $ghg^{-1} \in H$ which means that $(ghg^{-1})N \in H/N$.

        Therefore $H/N \unlhd G/N$.

        \item This is the main result of the theorem.

        Define $\phi: G/N \to G/H, gN \mapsto gH$. We check that $\phi$ is a well-defined homomorphism and find its image and kernel.
        \begin{itemize}
            \item \textbf{Well-defined}: Suppose $gN = g'N$. Then $g(g')^{-1} \in N$ by Coset Equality (\myref{lemma-coset-equality}), statements 1 and 5. Since $N \subseteq H$, thus $g(g')^{-1} \in H$ which implies $gH = g'H$, again by Coset Equality, statements 1 and 5. Hence
            \[
                \phi(gN) = gH = g'H = \phi(g'N)
            \]
            so $\phi$ is well-defined.

            \item \textbf{Homomorphism}: Take $gN, g'N \in G/N$. Then
            \begin{align*}
                \phi((gN)(g'N)) &= \phi((gg')N)\\
                &= (gg')H\\
                &= (gH)(g'H)\\
                &= \phi(gN)\phi(g'N)
            \end{align*}
            which means that $\phi$ is a homomorphism.
            
            \item \textbf{Image}: We show $\phi$ is surjective to prove that $\im\phi = G/H$. Suppose $gH \in G/H$. Clearly $\phi(gN) = gH$. Thus $gN$ is a pre-image of $gH$, meaning that $\phi$ is surjective. Hence $\im\phi = G/H$.
            
            \item \textbf{Kernel}: Suppose $gN \in \ker\phi = \{gN \vert \phi(gN) = eH = H\}$. Thus $\phi(gN) = gH = H$, which means $g \in H$. Hence $gN \in H/N$, so $\ker\phi \subseteq H/N$.

            Now assume $hN \in H/N$. Since $H\subseteq G$ (as $H \unlhd G$), thus $h \in G$. Therefore $hN \in G/N$, so $\phi(hN) = hH = H$. Hence $hN \in \ker\phi$ which means $H/N \subseteq \ker\phi$.

            Since $\ker\phi \subseteq H/N$ and $H/N \subseteq \ker\phi$, we must have $\ker\phi = H/N$.
        \end{itemize}

        By the Fundamental Homomorphism Theorem (\myref{thrm-isomorphism-1}), we have $\frac{G/N}{\ker\phi} \cong \im\phi$, which means
        \[
            \frac{G/N}{H/N} \cong G/H,
        \]
        proving statement 3.
    \end{enumerate}
    This proves the theorem.
\end{proof}

\begin{example}
    Take $G = \Z$, $H = m\Z$, and $N = mn\Z$. Note that clearly $H, N \leq G$, and since $G$ is abelian, we must also have $H \unlhd G$ and $N \unlhd G$. By the Third Isomorphism Theorem (\myref{thrm-isomorphism-3}), statement 3,
    \[
        \frac{G/N}{H/N} \cong G/H.
    \]
    Note $G/H = \Z/(m\Z) \cong \Z_m$ by \myref{problem-Zn-isomorphic-to-Z-by-nZ}. Note also
    \[
        \frac{G/N}{H/N} = \frac{\Z/(mn\Z)}{m\Z/(mn\Z)}.
    \]
    Hence we see that
    \[
        \frac{\Z/(mn\Z)}{m\Z/(mn\Z)} \cong \Z/(m\Z) \cong \Z_m.
    \]
\end{example}

\begin{exercise}
    Suppose $x$ and $y$ are positive integers such that $y = mx$ for some integer $m$. Let $H = x\Z$ and $N = y\Z$ be groups under addition.
    \begin{partquestions}{\roman*}
        \item Explain why $N \subseteq H$.
        \item Find a group $G$ such that $H \lhd G$ and $N \lhd G$.
        \item Hence find the order of $H/N$.
    \end{partquestions}
\end{exercise}

\newpage

\section{Problems}
\begin{problem}
    Let $G$ be a group. Prove that $G/G$ is isomorphic to the trivial group.
\end{problem}

\begin{problem}
    Let $R = (\R, +)$. Also let $G = R^2$ and $H = \left\{(r\sqrt2, r\sqrt3) \vert r\in R\right\}$ be groups under component-wise addition. Prove that $G/H \cong R$.
\end{problem}

\begin{problem}\label{problem-subgroup-product-equal-to-subgroup-if-one-is-subgroup-of-another}
    Let $G$ be a group. Let $H$ and $K$ be subgroups of $G$ such that $K \subseteq H$. Prove that $HK = H$.
\end{problem}

\begin{problem}\label{problem-cartesian-product-of-group-by-group-isomorphic-to-group}
    Let $G$ be an abelian group with operation $\ast$. Let $I = \{(g, g^{-1}) \vert g \in G\}$ be a group under component-wise application of $\ast$.
    \begin{partquestions}{\roman*}
        \item Show that $I \cong G$.
        \item Hence prove $G^2/G \cong G$ by considering a suitable homomorphism.
    \end{partquestions}
\end{problem}

\begin{problem}\label{problem-mZ/nZ-isomorphic-to-Zn/m}
    Let $G = m\Z$ and $H = n\Z$ be groups under addition, where $m\vert n$ and $m \neq n$. Let the map $\phi: G \to \Z/({\frac nm}\Z)$ be defined such that
    \[
        \phi(am) = a + \frac nm \Z.
    \]
    Prove that $G/H \cong \Z_{\frac nm}$.
\end{problem}

\section{More Types of Groups}
\subsection*{Exercises}
\begin{questions}
    \item Let $G = \Z_{mn}$ and $H = \{0, n, 2n, \dots, (m-1)n\}$. Clearly $H$ is a subgroup of $G$ of order $m$. By \myref{problem-subgroup-of-cyclic-group-is-cyclic} we know $H$ is normal and cyclic with order $m$ and by \myref{exercise-quotient-group-of-cyclic-group-is-cyclic} we know $G/H$ is cyclic. The order of $G/H$ is $\frac{|G|}{|H|} = \frac{mn}{m} = n$ by Lagrange's theorem (\myref{thrm-lagrange}), meaning that $G/H \cong \Z_n$. Hence, $\Z_{mn}/\Z_m \cong G/H \cong \Z_n$.

    \item Note 0 is the identity in $\Z_n$. By \myref{lemma-order-of-an-element-that-is-equivalent-to-identity} we know that if $12$ is equivalent to the identity in $\Z_n$, then $12 = mn$ for some integer $m$. Since $n > 0$ we restrict $m$ to positive integers. Now $12 = 2^2 \times 3$. Thus the possible values of $n$ are
    \begin{itemize}
        \item $n = 1$ with $m = 12$;
        \item $n = 2$ with $m = 6$;
        \item $n = 3$ with $m = 4$;
        \item $n = 4$ with $m = 3$;
        \item $n = 6$ with $m = 2$; and
        \item $n = 12$ with $m = 1$.
    \end{itemize}

    \item $|10| = \frac{210}{\gcd(10, 210)} = \frac{210}{10} = 21$, $|42| = \frac{210}{\gcd(42, 210)} = \frac{210}{42} = 5$, $|75| = \frac{210}{\gcd(75, 210)} = \frac{210}{15} = 14$, and $|140| = \frac{210}{\gcd(140, 210)} = \frac{210}{70} = 3$.

    \item \begin{partquestions}{\alph*}
        \item Note that $10 = 2 \times 5$. Generators of the group $\Z_{10}$ has to satisfy $\gcd(m, 10) = 1$ by \myref{corollary-element-in-cyclic-group-is-generator-iff-gcd-is-1}. The positive integers that satisfy this requirement (and which are less than 10) are 1, 3, 7, 9. Thus they are the generators of $\Z_{10}$.
        \item Note that 101 is prime. Hence all positive integers from 1 to 100 (inclusive) are generators. (Note that 0 is not a generator of $\Z_{101}$ since 0 is the identity.)
    \end{partquestions}

    \item We show that all subgroups of $\mathrm{Q}$ are, in fact, normal. We consider the first definition of the quaternion group.
    \begin{itemize}
        \item Clearly $\{1\} \lhd \mathrm{Q}$ and $\mathrm{Q} \unlhd \mathrm{Q}$.
        \item The subgroups $\langle i\rangle$, $\langle j\rangle$, and $\langle k\rangle$ have order 4. Therefore, Lagrange's theorem (\myref{thrm-lagrange}) tells us that they have index 2. Hence these subgroups are normal by \myref{problem-subgroup-of-index-2}.
        \item Consider the subgroup $\langle -1 \rangle = \{1, -1\}$. \begin{itemize}
            \item $1\langle -1 \rangle = \langle -1 \rangle1$, since 1 is the identity;
            \item $-1\langle -1 \rangle = \{1, -1\} = \langle -1 \rangle(-1)$;
            \item $i\langle -1 \rangle = \{-i, i\} = \langle -1 \rangle i$;
            \item $-i\langle -1 \rangle = \{i, -i\} = \langle -1 \rangle (-i)$;
            \item $j\langle -1 \rangle = \{-j, j\} = \langle -1 \rangle j$;
            \item $-j\langle -1 \rangle = \{j, -j\} = \langle -1 \rangle (-j)$;
            \item $k\langle -1 \rangle = \{-k, k\} = \langle -1 \rangle k$; and
            \item $-k\langle -1 \rangle = \{k, -k\} = \langle -1 \rangle (-k)$.
        \end{itemize}
        Thus $\langle -1 \rangle$ is normal.
    \end{itemize}
    Hence all subgroups of $\mathrm{Q}$ are normal.

    \item $\begin{pmatrix}2&6\end{pmatrix} = \begin{pmatrix}2&3\end{pmatrix}\begin{pmatrix}3&4\end{pmatrix}\begin{pmatrix}4&5\end{pmatrix}\begin{pmatrix}5&6\end{pmatrix}\begin{pmatrix}4&5\end{pmatrix}\begin{pmatrix}3&4\end{pmatrix}\begin{pmatrix}2&3\end{pmatrix}$.

    \item Note that $\begin{pmatrix}1&3&2&5&4\end{pmatrix} = \begin{pmatrix}1&4\end{pmatrix}\begin{pmatrix}1&5\end{pmatrix}\begin{pmatrix}1&2\end{pmatrix}\begin{pmatrix}1&3\end{pmatrix}$. \myref{thrm-parity-of-permutation} tells us that $\begin{pmatrix}1&3&2&5&4\end{pmatrix}$ is even and thus has a sign of $+1$.

    \item Note that $\An{3}$ has order $\frac{3!}{2} = 3$ so we should expect 3 permutations. Clearly the identity is one such permutation. Looking at \myref{example-symmetric-group-of-degree-3} we can find two more, namely $\begin{pmatrix}1&2&3\end{pmatrix}$ and $\begin{pmatrix}1&3&2\end{pmatrix}$.
    
    For $\An{4}$, note that it has order $\frac{4!}{2} = 12$ so we expect 12 permutations. Again the identity is one of them. Like $\An{3}$ we now find the 3-cycles in $\An{4}$, which are $\begin{pmatrix}1&2&3\end{pmatrix}$, $\begin{pmatrix}1&2&4\end{pmatrix}$, $\begin{pmatrix}1&3&2\end{pmatrix}$, $\begin{pmatrix}1&3&4\end{pmatrix}$, \linebreak $\begin{pmatrix}1&4&2\end{pmatrix}$, $\begin{pmatrix}1&4&3\end{pmatrix}$, $\begin{pmatrix}2&3&4\end{pmatrix}$, and $\begin{pmatrix}2&4&3\end{pmatrix}$. So there are 3 more permutations unaccounted, which are permutations of products of 2-cycles: $\begin{pmatrix}1&2\end{pmatrix}\begin{pmatrix}3&4\end{pmatrix}$, $\begin{pmatrix}1&3\end{pmatrix}\begin{pmatrix}2&4\end{pmatrix}$, and $\begin{pmatrix}1&4\end{pmatrix}\begin{pmatrix}2&3\end{pmatrix}$.

    \item $\Un{10} = \{1, 3, 7, 9\}$.

    \item By a corollary of Lagrange's theorem (\myref{corollary-order-of-group-multiple-of-order-of-element}), the order of $a$ dives the order of the group $\Un{n}$. Now since $|\Un{n}| = \totient(n)$, thus the order of $a$ divides $\totient(n)$.

    \item $\begin{pmatrix}2&1&2\\1&0&1\\2&1&2\end{pmatrix}$

    \item We already proved that $\Inn{G} \leq \Aut{G}$ so we only need to prove normality.

    Let $\phi \in \Aut{G}$ and $\iota_g \in \Inn{G}$. For brevity let $f = \phi\iota_g\phi^{-1}$. We note that $f \in \Aut{G}$; we need to prove that $f \in \Inn{G}$.

    Suppose $x \in G$. Since $\phi$ is an isomorphism, there exists $w \in G$ such that $x = \phi(w)$, i.e. $w = \phi^{-1}(x)$. So
    \begin{align*}
        f(x) &= \left(\phi\iota_g\phi^{-1}\right)(x)\\
        &= \phi(\iota_g(\phi^{-1}(x)))\\
        &= \phi(\iota_g(w))\\
        &= \phi(gwg^{-1})\\
        &= \phi(g)\phi(w)\phi(g^{-1})\\
        &= \phi(g)x\left(\phi(g)\right)^{-1}
    \end{align*}
    which shows that $f \in \Inn{G}$. Hence, $\Inn{G} \unlhd \Aut{G}$.
\end{questions}

\subsection*{Problems}
\begin{questions}
    \item We note that the two questions are equivalent to finding the orders of 3774 and 1870 in the group $\Z_{10101}$. We note that
    \begin{align*}
        1870 &= 2 \times 5 \times 11 \times 17,\\
        3774 &= 2 \times 3 \times 17 \times 37, \text{ and}\\
        10101 &= 3 \times 7 \times 13 \times 37.
    \end{align*}
    Therefore, $\gcd(1870, 10101) = 1$ and $\gcd(3774, 10101) = 3 \times 37 = 111$. Hence $|1870| = 10101$ and $|3774| = \frac{10101}{111} = 91$. Therefore, $a = 10101$ and $b = 91$.

    \item We claim that $\An{n}$ is non-abelian for any $n > 3$. Note that both $\pi = \begin{pmatrix}1 & 2 & 3\end{pmatrix}$ and $\sigma = \begin{pmatrix}2 & 3 & 4\end{pmatrix}$ are even permutations, and hence are in $\An{n}$ for any $n > 3$. We note
    \begin{itemize}
        \item $\pi\sigma = \begin{pmatrix}1 & 2 & 3\end{pmatrix}\begin{pmatrix}2 & 3 & 4\end{pmatrix} = \begin{pmatrix}1 & 2\end{pmatrix}\begin{pmatrix}3 & 4\end{pmatrix}$; and
        \item $\sigma\pi = \begin{pmatrix}2 & 3 & 4\end{pmatrix}\begin{pmatrix}1 & 2 & 3\end{pmatrix} = \begin{pmatrix}1 & 3\end{pmatrix}\begin{pmatrix}2 & 4\end{pmatrix}$.
    \end{itemize}
    So $\pi\sigma \neq \sigma\pi$. Thus $\An{n}$ is non-abelian for any $n > 3$.

    We note that
    \begin{itemize}
        \item $\An{2}$ has order 1 so $\An{2}$ is the trivial group, which is abelian (and cyclic); and
        \item $\An{3}$ has order 3 so $\An{3}$ is cyclic and thus abelian.
    \end{itemize}
    Thus the largest integer $n$ for which $\An{n}$ is abelian is $n = 3$. Furthermore $\An{k}$ is cyclic if $k = 2$ or $k = 3$.

    \item We first note that
    \[
        \totient(2p^k) = 2p^k\left(1-\frac12\right)\left(1-\frac1p\right) = p^k\left(1-\frac1p\right) = \totient(p^k).
    \]
    Now we are given that $r$ is an odd primitive root of $p^k$. Since $r \in \Un{p^k}$, thus $\gcd(r, 2p^k) = 1$ because $\gcd(r, p^k) = 1$. Now as $r$ is odd, thus $r \in \Un{2p^k}$. Let $n$ be the order of $r$ in $\Un{2p^k}$. Then by \myref{exercise-order-of-a-divides-phi-a} we know $n$ divides $\totient(2p^k)$. At the same time, because $r$ is a generator in $\Un{p^k} \cong \Z_{\phi(p^k)}$, so $\totient(p^k) = \totient(2p^k)$ divides $n$ by \myref{lemma-order-of-an-element-that-is-equivalent-to-identity}. Since $n$ divides $\totient(2p^k)$ and $\totient(2p^k)$ divides $n$ simultaneously, therefore $n = \totient(2p^k) = |\Un{2p^k}|$ which means that $r$ is a primitive root modulo $2p^k$.

    \item \begin{partquestions}{\roman*}
        \item The forward direction is clearly true since if $f_1 = f_2$, then $f_1(x) = f_2(x)$ for all $x \in G$, including $g \in G$. For the reverse direction, assume $f_1(g) = f_2(g)$. Note that
        \[
            f_1(g^k) = (f_1(g))^k = (f_2(g))^k = f_2(g^k)
        \]
        for any integer $k$. Since $g$ is a generator, thus we have $f_1(x) = f_2(x)$ for all $x \in G$, meaning $f_1 = f_2$.

        \item We note $f(g) \in G$. Since $g$ is a generator, hence $f(g) = g^k$ for some integer $k$. Hence any homomorphism from $G$ to $G$ is of the form $f(g) = g^{m_f}$ where $0 \leq m_f \leq n-1$, which means $m_f \in \Z_n$.

        \item Suppose the map $f_2: G \to G$ is another homomorphism where $f_2(g) = g^{m_f}$. Then we see
        \[
            f(g) = g^{m_f} = f_2(g)
        \]
        which means $f = f_2$ by \textbf{(i)}. Hence $m_f$ is unique to $f$.

        \item Consider $f_1(f_2(g))$. On one hand,
        \[
            f_1(f_2(g)) = f_1(g^{m_{f_2}}) = (f_1(g))^{m_{f_2}} = g^{m_{f_1}m_{f_2}},
        \]
        while on the other,
        \[
            f_1(f_2(g)) = (f_1 \circ f_2)(g) = g^{m_{f_1\circ f_2}}
        \]
        by definition of $m_f$ as introduced in \textbf{(ii)}. Therefore $m_{f_1\circ f_2} \equiv m_{f_1}m_{f_2} \pmod n$. In other words, $m_{f_1\circ f_2} = m_{f_1} \otimes_n m_{f_2}$.

        \item We prove the forward direction first by assuming that the map $f$ is an automorphism. Hence $f$ is surjective, meaning that there exists $a \in G$ such that $f(a) = g$. Since $a \in G$ thus $a = g^k$ for some $k \in \Z_n$ (we will show $k \in \Un{n}$ later). Observe
        \[
            g = f(a) = f(g^k) = (f(g))^k = g^{m_fk}
        \]
        which means $m_fk \equiv 1 \pmod n$. By \myref{prop-multiplicative-inverse-exists-iff-coprime}, this means that we have $\gcd(m_f, n) = 1$ and $\gcd(k, n) = 1$. Therefore, $m_f$ and $k$ are in $\Un{n}$. Hence $k$ is the multiplicative inverse of $m_f$.

        We now prove the reverse direction. Assume $m_f$ has a multiplicative inverse (say $k$), meaning $m_fk \equiv 1 \pmod n$. As above this means that both $m_f$ and $k$ are in $\Un{n}$. We show that $f$ is a bijection.
        \begin{itemize}
            \item \textbf{Injective}: Suppose $x, y \in G$ such that $f(x) = f(y)$. Since $g$ is a generator we may take $x = g^p$ and $y = g^q$ for some integers $p$ and $q$. Hence we have $g^{m_fp} = g^{m_fq}$. Then
            \[
                \left(g^{m_fp}\right)^k = g^{km_fp} = \left(g^{km_f}\right)^p = g^p
            \]
            and $\left(g^{m_fq}\right)^k = g^q$. Hence this implies $g^p = g^q$ which means $x = y$.
            \item \textbf{Surjective}: Suppose $x \in G$. Since $g$ is a generator we may write $x = g^p$ for some integer $p$. Then $f(g^{kp}) = g^{m_fkp} = g^p = x$.
        \end{itemize}
        Also $f$ is given to be a homomorphism. Hence $f$ is an isomorphism. Since $f: G \to G$, it is thus an automorphism.

        \item We prove that $\phi$ is an isomorphism.
        \begin{itemize}
            \item \textbf{Homomorphism}: Let $f_1, f_2 \in \Aut{G}$. Then
            \begin{align*}
                \phi(f_1\circ f_2) &= m_{f_1\circ f_2} & (\text{definition of } m_f \text{ in }\textbf{(ii)})\\
                &= m_{f_1} \otimes_n m_{f_2} & (\text{by \textbf{(iv)}})\\
                &= \phi(f_1)\otimes_n\phi(f_2),
            \end{align*}
            which means $\phi$ is a homomorphism.

            \item \textbf{Injective}: Suppose we have $f_1, f_2 \in \Aut{G}$ such that $\phi(f_1) = \phi(f_2)$. Thus $m_{f_1} = m_{f_2}$ by definition of $\phi$. However, we know that the value of $m$ uniquely defines a homomorphism from $G$ to $G$ from \textbf{(iii)}. Hence $f_1 = f_2$, which shows that $\phi$ is injective.

            \item \textbf{Surjective}: Suppose $r \in \Un{n}$. Define the map $f: G \to G$ where $f(g) = g^r$. Since $r \in \Un{n}$ it has a multiplicative inverse, which means that $f$ is an automorphism by \textbf{(v)}. Clearly $\phi(f) = r$, so $r$ has a pre-image. So $\phi$ is surjective.
        \end{itemize}
        Hence $\phi$ is an isomorphism, meaning $\Aut{G} \cong \Un{n}$.
    \end{partquestions}
\end{questions}

\section{Group Actions}
\begin{questions}
    \item We prove the two group action axioms.
    \begin{itemize}
        \item \textbf{Identity}: $\alpha(e, x) = exe^{-1} = x$.
        \item \textbf{Compatibility}: Note
        \begin{align*}
            \alpha(g, \alpha(h, x)) &= \alpha(g, hxh^{-1})\\
            &= gh x h^{-1}g^{-1}\\
            &= (gh)x(gh)^{-1}\\
            &= \alpha(gh, x).
        \end{align*}
    \end{itemize}
    Therefore $\alpha$ is a group action of $G$ on $G$.

    \item Recall there are 6 elements in $\Sn{3}$: $\id$, $\begin{pmatrix}1 & 2 & 3\end{pmatrix}$, $\begin{pmatrix}1 & 3 & 2\end{pmatrix}$, $\begin{pmatrix}1 & 2\end{pmatrix}$, $\begin{pmatrix}1 & 3\end{pmatrix}$, and $\begin{pmatrix}2 & 3\end{pmatrix}$. Clearly the identity has all elements of $X$ as fixed points. It is also clear that $\begin{pmatrix}1 & 2 & 3\end{pmatrix}$ and $\begin{pmatrix}1 & 3 & 2\end{pmatrix}$ have no fixed points since they permute all elements. For the rest, the fixed points are the missing element from the cycle notation, i.e. $\begin{pmatrix}1 & 2\end{pmatrix}$ has fixed point 3, $\begin{pmatrix}1 & 3\end{pmatrix}$ has fixed point 2, and $\begin{pmatrix}2 & 3\end{pmatrix}$ has fixed point 1.

    \item For 1, it is $\{\id, \begin{pmatrix}2 & 3\end{pmatrix}\}$. For 2, it is $\{\id, \begin{pmatrix}1 & 3\end{pmatrix}\}$. For 3, it is $\{\id, \begin{pmatrix}1 & 2\end{pmatrix}\}$.

    \item We work from the statement forwards. Note that each of these statements are ``if and only if'' statements.
    \begin{align*}
	    g \cdot x = h \cdot x &\iff g^{-1} \cdot (g \cdot x) = g^{-1} \cdot (h \cdot x)\\
	    &\iff (g^{-1}g) \cdot x = (g^{-1}h) \cdot x\\
	    &\iff e \cdot x = (g^{-1}h) \cdot x\\
	    &\iff x = (g^{-1}h) \cdot x\\
	    &\iff (g^{-1}h) \cdot x = x\\
	    &\iff g^{-1}h \in \Stab{G}{x}
	\end{align*}

	\item \begin{partquestions}{\alph*}
		\item An orbit takes the form $\Orb{G}{x}$. Clearly $e \cdot x = x$ so $x \in \Orb{G}{x}$ and thus $\Orb{G}{x}$ is non-empty.
	    \item Let $x \in X$. Since $e \cdot x = x$, so $x \in \Orb{G}{x}$.
	    \item Suppose $x \in \Orb{G}{x_1} \cap \Orb{G}{x_2}$ (as their intersection is non-empty). Then there exists $g_1, g_2 \in G$ such that $g_1\cdot x_1 = x = g_2\cdot x_2$. Thus,
	    \begin{align*}
	        x_1 &= e \cdot x_1\\
	        &= (g_1^{-1}g_1)\cdot x_1\\
	        &= g_1^{-1} \cdot (g_1 \cdot x_1)\\
	        &= g_1^{-1} \cdot (g_2 \cdot x_2)\\
	        &= (g_1^{-1}g_2) \cdot x_2.
	    \end{align*}
	    Now suppose $y \in \Orb{G}{x_1}$. Then $y = g\cdot x_1$ for some $g \in G$. Hence,
	    \begin{align*}
	        y &= g\cdot x_1 \\
	        &= g \cdot \left((g_1^{-1}g_2) \cdot x_2\right)\\
	        &= (\underbrace{gg_1^{-1}g_2}_{\text{In } G})\cdot x_2\\
	        &\in \Orb{G}{x_2}
	    \end{align*}
	    which means any element in $\Orb{G}{x_1}$ is also in $\Orb{G}{x_2}$. Hence, $\Orb{G}{x_1}$ is a subset of $\Orb{G}{x_2}$. A similar argument can be used to show that $\Orb{G}{x_2}$ is a subset of $\Orb{G}{x_1}$. Hence $\Orb{G}{x_1} = \Orb{G}{x_2}$.
	\end{partquestions}

	\item We prove the forward direction first: suppose the action is transitive. Then there exists $x \in X$ such that $\Orb{G}{x} = X$. Now consider any other element $y \in X$. Since the action is transitive, this means that there exists a $\hat{g} \in G$ such that $\hat{g} \cdot x = y$. Note that $\Orb{G}{y} = \Orb{G}{\hat{g} \cdot x}$, and that $\Orb{G}{x} = \{g \cdot x \vert g \in G\}$. Hence,
	\[
        \Orb{G}{\hat{g} \cdot x} = \{g\cdot (\hat{g} \cdot x) \vert g \in G\} = \{(g\hat{g}) \cdot x \vert g \in G\}.
	\]
	Since $G$ is a group, $g\hat{g} \in G$. In particular, we may pick $g = g'\hat{g}^{-1}$ to obtain any arbitrary element $g' \in G$. Thus, this means that
	\[
        	\{(g\hat{g}) \cdot x \vert g \in G\} = \{g' \cdot x \vert g' \in G \} = \Orb{G}{x} = X.
	\]
	Hence, for any element $y \in X$, $\Orb{G}{y} = \Orb{G}{g \cdot x} = X$.

	The reverse direction is trivial: suppose $\Orb{G}{x} = X$ for all $x \in X$. Then certainly there exists an element $x \in X$ such that $\Orb{G}{x} = X$, meaning that the group action is transitive.

	\item \begin{partquestions}{\alph*}
	    \item Consider $x = n$. The orbit of $n$ is all of $X$. Consider the permutation $\sigma = \begin{pmatrix}k & n\end{pmatrix}$ where $1 \leq k \leq n$. Clearly $\sigma \in \Sn{n}$. Note that $\sigma \cdot n = \sigma(n) = k$. Thus, $\Orb{G}{n} = X$, meaning that the group action ``$\cdot$'' given by $g \cdot x \mapsto g(x)$ is transitive.
	    \item Note that $|X| = n$ and $|\Sn{n}| = n!$. By Orbit-Stabilizer theorem (\myref{thrm-orbit-stabilizer}), the stabilizer of $x$ by $G$ must have order $\frac{n!}{n} = (n-1)!$.
	\end{partquestions}

	\item By the Orbit-Stabilizer theorem (\myref{thrm-orbit-stabilizer}),
	\[
        |\Orb{G}{x}| = \frac{|G|}{|\Stab{G}{x}|} = [G : \Stab{G}{x}]	.
	\]
	Under the group action of conjugation, $\Orb{G}{x} = \Cl{x}$ and $\Stab{G}{x} = \Centralizer{G}{x}$. Hence, $|\Cl{x}| = [G : \Centralizer{G}{x}]$ as required.

	\item \begin{partquestions}{\alph*}
	    \item One sees that $\Z{D_3} = \{e\}$ based on the group table of $D_3$.
	    \item Recall that every element in $D_3$ can be expressed in the form $r^as^b$ where $a \in \{0, 1, 2\}$ and $b \in \{0, 1\}$. One finds that $\Cl{r} = \{r, r^2\}$ and $\Cl{s} = \{s, rs, rs^2\}$.
	    \item The class equation is $6 = 1 + 2 + 3$.
	\end{partquestions}

	\item By Cauchy's Theorem (\myref{thrm-cauchy}) there exists an element (say $x$) with order $p$. Consider $H = \langle x \rangle$. Note that $|H| = p$ and $H \leq G$. Hence we found a subgroup of $G$ of order $p$.
\end{questions}

\section{Sylow Theorems}
\subsection*{Exercises}
\begin{questions}
    \item We note that $12 = 2^2 \times 3$. Thus a Sylow 2-subgroup must have order 4. Clearly $|3| = 4$ so $\langle 3 \rangle = \{0, 3, 6, 9\}$ is the Sylow 2-subgroup of $\Z_{12}$.

    \item Recall that $|\Sn{5}| = 120 = 2^3 \times 3 \times 5$. By a corollary of the First Sylow Theorem (\myref{corollary-sylow-p-subgroup-exists}), $\Syl{p}{G} \neq \emptyset$ if $p$ is 2, 3, or 5.

    \item We prove this by constructing the map $\phi: H \to gHg^{-1}$ where $h \mapsto ghg^{-1}$. We note that $\phi$ is an isomorphism.
    \begin{itemize}
        \item \textbf{Homomorphism}: Let $x, y \in H$. Then
        \[
            \phi(xy) = g(xy)g^{-1} = (gxg^{-1})(gyg^{-1}) = \phi(x)\phi(y)
        \]
        which clearly means that $\phi$ is a homomorphism.
        
        \item \textbf{Injective}: Suppose $x, y \in H$ such that $\phi(x) = \phi(y)$. Then $gxg^{-1} = gyg^{-1}$ which quickly implies $x = y$.
        
        \item \textbf{Surjective}: Suppose $ghg^{-1} \in gHg^{-1}$. Clearly we have $\phi(h) = ghg^{-1}$, so any element in $gHg^{-1}$ has a pre-image inside $H$.
    \end{itemize}
    Hence $H \cong gHg^{-1}$.

    \item By \myref{prop-order-of-conjugate-element-equals-order-of-element} we know that $|xyx^{-1}| = |y|$ for all $x, y \in G$. Substituting $x = g$, and $y = hg$ yields
    \[
        |xyx^{-1}| = |g(hg)g^{-1}| = |gh| \text{ and } |y| = |hg|
    \]
    so the result follows.

    \item Clearly $e \in \N{G}{S}$ since $eSe^{-1} = S$. Consider $x, y \in \N{G}{S}$, meaning that $xSx^{-1} = S$ and $ySy^{-1} = S$. Note that $y^{-1} \in \N{G}{S}$ since
    \begin{align*}
        y^{-1}S\left(y^{-1}\right)^{-1} &= y^{-1}Sy\\
        &= y^{-1}\left(ySy^{-1}\right)y & (\text{since } y \in \N{G}{S})\\
        &= (y^{-1}y)S(y^{-1}y)\\
        &= S.
    \end{align*}
    Therefore
    \begin{align*}
        \left(xy^{-1}\right)S\left(xy^{-1}\right)^{-1} &= \left(xy^{-1}\right)S\left(yx^{-1}\right)\\
        &= x\left(y^{-1}Sy\right)x^{-1}\\
        &= xSx^{-1} & (\text{since } y^{-1} \in \N{G}{S})\\
        &= S & (\text{since } x \in \N{G}{S})
    \end{align*}
    which means that $xy^{-1} \in \N{G}{S}$. Hence, by the subgroup test, we have $\N{G}{S} \leq G$.

    \item By the Second Sylow Theorem (\myref{thrm-sylow-2}), we know that $gHg^{-1} = K$. Since $H \cong gHg^{-1}$ by \myref{exercise-conjugate-subgroup-isomorphic-to-subgroup} thus $H \cong gHg^{-1} = K$ as required.

    \item We note $784 = 2^4 \times 7^2$, so $m = 16$, $p = 7$, and $k = 2$. By the Third Sylow Theorem (\myref{thrm-sylow-3}), we know that
    \begin{itemize}
        \item $n_7 = [G : \N{G}{P}] = \frac{|G|}{|\N{G}{P}|}$;
        \item $n_7 \mid 16$, which implies $n_7 \in \{1, 2, 4, 8, 16\}$; and
        \item $n_7 \equiv 1 \pmod 7$, which implies $n_7 \in \{1, 8, 15, 22, \dots\}$.
    \end{itemize}
    Hence $n_7 = 1$ or $n_7 = 8$. But since $P$ is not a normal subgroup of $G$, by \myref{corollary-sylow-subgroup-is-normal-if-it-is-unique}, $P$ cannot be the only Sylow 7-subgroup, meaning $n_7 \neq 1$. Hence $n_7 = 8$, so
    \[
        8 = n_7 = \frac{|G|}{|\N{G}{P}|} = \frac{784}{|\N{G}{P}|}
    \]
    which means that $|\N{G}{P}| = 98$.

    \item Note $130 = 2 \times 5 \times 13$. Consider the number of Sylow 13-subgroups, $n_{13}$. The Third Sylow Theorem (\myref{thrm-sylow-3}) tells us that
    \begin{itemize}
        \item $n_{13} \mid 2 \times 5 = 10$, so $n_{13} \in \{1, 2, 5, 10\}$, and
        \item $n_{13} \equiv 1 \pmod{13}$ so $n_{13} \in \{1, 14, 27, \dots\}$.
    \end{itemize}
    Hence $n_{13} = 1$. But by \myref{corollary-sylow-subgroup-is-normal-if-it-is-unique} this means that the only Sylow 13-subgroup is normal. Hence a group of order 130 is non-simple.
\end{questions}

\subsection*{Problems}
\begin{questions}
    \item Note $200 = 2^3 \times 5^2$. Note that for $p = 5$ we have $m = 8$ and the factors of 8 are 1, 2, 4, and 8. Furthermore by the Third Sylow Theorem (\myref{thrm-sylow-3}) we must have $n_5 \equiv 1 \pmod 5$. Hence $n_5 = 1$. By a corollary of the Second Sylow Theorem (\myref{corollary-sylow-subgroup-is-normal-if-it-is-unique}) this means that the only Sylow 5-subgroup is normal.

    \item Note $33 = 3 \times 11$,
    \begin{itemize}
        \item when $p = 3$ we have $m = 11$ and the factors of 11 are 1 and 11; and
        \item when $p = 11$ we have $m = 3$ and the factors of 3 are 1 and 3.
    \end{itemize}
    The Third Sylow Theorem (\myref{thrm-sylow-3}) tells us that $n_p \equiv 1 \pmod p$. Hence we must have $n_3 = n_{11} = 1$. A corollary of the Second Sylow Theorem (\myref{corollary-sylow-subgroup-is-normal-if-it-is-unique}) tells us that the only Sylow 3-subgroup and Sylow 11-subgroup are normal.

    \item For brevity let $q = 2^p - 1$, and we are given that $q$ is a prime. By the Third Sylow Theorem (\myref{thrm-sylow-3}), $n_q \mid 2^{p-1}$ and $n_q \equiv 1 \pmod p$. The factors of $2^{p-1}$ are $1, 2, 4, 8, \dots, 2^{p-1}$. We note $2^{p-1} < 2^p - 1 = q$ for any prime $p$ since
    \[
        2^{p-1} + 1 < 2^{p-1} + 2^{p-1} = 2(2^{p-1}) = 2^p
    \]
    which result immediately follows by subtracting 1 on both sides. Hence, the only possible value that satisfies both conditions is $n_q = 1$. By a corollary of the Second Sylow Theorem (\myref{corollary-sylow-subgroup-is-normal-if-it-is-unique}) this means that the only Sylow $q$-subgroup is normal, hence showing that a group with an even perfect number order is non-simple.

    \item \begin{partquestions}{\roman*}
        \item The divisors of $p$ are 1 and $p$ itself. By the Third Sylow Theorem (\myref{thrm-sylow-3}), $n_q$ divides $p$ and $n_q \equiv 1 \pmod q$. Since $p < q$ hence $p \not\equiv 1 \pmod q$ meaning that $n_q = 1$. By a corollary of the Second Sylow Theorem (\myref{corollary-sylow-subgroup-is-normal-if-it-is-unique}) the only Sylow $q$-subgroup is normal.

        \item The divisors of $q$ are 1 and $q$ itself. By the Third Sylow Theorem (\myref{thrm-sylow-3}), $n_p$ divides $q$ and $n_p \equiv 1 \pmod p$. Since $q \not\equiv 1 \pmod p$ by assumption, we must have $n_p = 1$.

        Recall that the order of an element in a group of order $pq$ must divide $pq$ (\myref{corollary-order-of-group-multiple-of-order-of-element}). Hence the possible orders of an element in such a group are 1, $p$, $q$, or $pq$.
        \begin{itemize}
            \item There is only one element of order 1, the identity.
            \item There are $p - 1$ elements of order $p$, all belonging in the single Sylow $p$-subgroup. Note that we subtract 1 because one element in the Sylow $p$-subgroup is the identity.
            \item There are $q - 1$ elements of order $q$, all in the single Sylow $q$-subgroup.
        \end{itemize}
        Hence, since the total number of elements in a group of order $pq$ is $pq$, the number of elements of order $pq$ is
        \begin{align*}
            pq - \left((p-1)+(q-1)+1\right) &= pq - (p+q - 1)\\
            &= pq - p - q + 1\\
            &> 2q - 2 - q + 1\\
            &= 2q - q - 1\\
            &= q - 1\\
            &> 0
        \end{align*}
        which means that there is at least one element of order $pq$. By \myref{thrm-cyclic-group-has-element-with-same-order} this means that such a group is cyclic.
    \end{partquestions}

    \item \begin{partquestions}{\roman*}
        \item Let $P$ be a Sylow $p$-subgroup of $N$. Lagrange's Theorem (\myref{thrm-lagrange}) tells us that $|G| = [G:N]|N|$. Since $p$ does not divide $[G:N]$ we must have $|N| = p^ka$ where $a$ divides $m$. Hence $|P| = p^k$ as $P$ is a Sylow $p$-subgroup of $N$. Since $P$ has order $p^k$ and $P \leq N \leq G$, thus $P$ is also a Sylow $p$-subgroup of $G$.
        \item Let $Q$ be a Sylow $p$-subgroup of $G$. The Second Sylow Theorem (\myref{thrm-sylow-2}) tells us there exist  a $g \in G$ such that $Q = gPg^{-1}$. Recall by definition of normality that $gNg^{-1} = N$ for any $g \in G$. Note also that $P \leq N$. Hence,
        \[
            Q = gPg^{-1} \leq gNg^{-1} = N
        \]
        which means that $Q$ is also a Sylow $p$-subgroup of $N$.
    \end{partquestions}

    \item We note $3325 = 5^2 \times 7 \times 19$. Let the group of order 3325 be $G$. We know that
    \begin{itemize}
        \item for $p = 5$ we have $m = 7 \times 19 = 133$ and so the possible divisors of $m$ are $\{1, 7, 19, 133\}$;
        \item for $p = 7$ we have $m = 5^2 \times 19 = 475$ and so the possible divisors of $m$ are $\{1, 5, 19, 25, 95, 475\}$; and
        \item for $p = 19$ we have $m = 5^2 \times 7 = 175$ and so the possible divisors of $m$ are $\{1, 5, 7, 25, 35, 175\}$.
    \end{itemize}
    The Third Sylow Theorem (\myref{thrm-sylow-3}) tells us that $n_p \equiv 1 \pmod p$. Thus $n_5 = n_7 = n_{19} = 1$. Let $P$, $Q$, and $R$ be the Sylow 5-subgroup, the Sylow 7-subgroup, and the Sylow 19-subgroup respectively. We note that $P$, $Q$, and $R$ are all normal subgroups of $G$ by \myref{corollary-sylow-subgroup-is-normal-if-it-is-unique}.

    Denote the group $QR$ by $H$. Since $Q$ and $R$ are of prime order, their intersection is the identity (\myref{problem-intersection-of-coprime-subgroups}). Furthermore, as $Q$ and $R$ are normal subgroups of $G$, thus they commute by \myref{problem-intersection-of-coprime-subgroups}. Therefore $H$ is the internal direct product of $Q$ and $R$, meaning $H \cong Q \times R$ by \myref{thrm-direct-product-equivalence}. Hence $|H| = |Q||R| = 7 \times 19 = 133$. Now because as $Q$ and $R$ are of prime order, thus $Q$ and $R$ are abelian and so is $H$. Hence $H$ is an abelian group of order 133.

    Now consider the group $PH$. Since 5 and 133 are coprime, thus $P \cap H = \{e\}$. In addition, since $P \lhd G$ thus $PH \leq G$ by Diamond Isomorphism Theorem (\myref{thrm-isomorphism-2}), statement 3. Also,
    \[
        |PH| = \frac{|P||H|}{|P \cap H|} = |P||H| = 5^2 \times 133 = 3325 = |G|
    \]
    which means that $G = PH$. Since $P \lhd G$, thus $ph = hp$ for any element $h \in H$, meaning elements in $P$ and $H$ commute. Hence, $G$ is the internal direct product of $P$ and $H$, meaning $G \cong P \times H$. As the external direct product of two abelian groups is also abelian (\myref{problem-external-direct-product-of-abelian-groups-is-abelian}) thus $G$ is abelian.

    \item Let $P$ be a Sylow $p$-subgroup of $G$. We note that $|G/P| = \frac{p^km}{p^k} = m$. Let $G$ act on the set of cosets $G/P$ by left multiplication, meaning $g\cdot (xP) = (gx)P$. We know by \myref{thrm-group-action-definition-equivalence} that this induces a homomorphism $\phi: G \to \Sn{m}$ where $\phi(g) = \sigma_g$ such that $\sigma_g(xP) = g\cdot (xP) = (gx)P$. By \myref{example-using-kernel-to-show-non-simple}, $\ker\phi = \bigcap_{x \in G}xPx^{-1}$.

    We note $\ker\phi \neq \{e\}$ since otherwise it would imply that $\phi$ is injective (\myref{exercise-trivial-kernel-means-injective}), which is impossible as that would mean $p^km = |G| \leq |\Sn{m}| = m!$ which is a contradiction. Also $\ker\phi \neq G$ as otherwise
    \[
        p^km = |G| = |\ker\phi| = \left|\bigcap_{x \in G} xPx^{-1}\right| \leq |xPx^{-1}| = |P| = p^k,
    \]
    which would mean $m = 1$, a contradiction. Hence $\ker\phi$ is a non-trivial proper subgroup of $G$. We note that $\ker\phi \lhd G$, so we have found a non-trivial proper normal subgroup of $G$, meaning that $G$ is non-simple.

    \item Let $G$ be a group of order 30. Note $30 = 2 \times 3 \times 5$, and consider $n_5$. The Third Sylow Theorem (\myref{thrm-sylow-3}) tells us that
    \begin{itemize}
        \item $6 \vert n_5$, so $n_5 \in \{1, 2, 3, 6\}$; and
        \item $n_5 \equiv 1 \pmod 5$, so $n_5 \in \{1, 6, 11, 16, \dots\}$.
    \end{itemize}
    Hence $n_5$ is 1 or 6. Seeking a contradiction, assume $n_5 = 6$, and let $P_5$ be a Sylow 5-subgroup.

    Since $|P_5| = 5$, which is prime, each non-identity element of $P_5$ is a generator. Hence, no two Sylow 5-subgroups can share any non-identity elements (otherwise they will be the same group), thereby meaning any two Sylow 5-subgroups intersect in the identity only. Thus there exists $6(5-1) = 24$ elements of order 5, meaning there must be 6 elements of order not equal to 5.

    Now consider $n_3$. Note by the Third Sylow Theorem again,
    \begin{itemize}
        \item $10 \vert n_3$, so $n_3 \in \{1, 2, 5, 10\}$; and
        \item $n_3 \equiv 1 \pmod 3$, so $n_3 \in \{1, 4, 7, 10, 13, \dots\}$.
    \end{itemize}
    Thus $n_3$ is 1 or 10. Now if $n_3 = 10$ then there must be $10(3-1) = 20$ elements of order 3, a contradiction to the fact there exists only 6 elements with order not 5. Hence $n_3 = 1$, meaning the only Sylow 3-subgroup (call it $P_3$) is normal in $G$.

    As $P_5 \leq G$ and $P_3 \lhd G$, by the Diamond Isomorphism Theorem (\myref{thrm-isomorphism-2}), statement 3, we have $P_5P_3 \leq G$. Note $P_5 \cap P_3 = \{e\}$ by \myref{problem-intersection-of-coprime-subgroups}. So \myref{exercise-order-of-subgroup-product} tells us
    \[
        |P_5P_3| = \frac{|P_5||P_3|}{|P_5 \cap P_3|} = \frac{5\times3}{1} = 15.
    \]
    One sees that $[G:P_5P_3] = \frac{30}{15} = 2$, so $P_5P_3 \lhd G$ by \myref{problem-subgroup-of-index-2}.

    Now \myref{problem-group-of-order-pq-has-normal-subgroup-of-order-q} tells us there exists a unique $H \lhd P_5P_3$ with $|H| = 5$. But since $P_5P_3 \lhd G$, \myref{problem-normal-subgroup-of-G-contains-all-sylow-p-subgroups} tells us that $P_5P_3$ contains all Sylow 5-subgroups of $G$, meaning $G$ has only 1 Sylow 5-subgroup, i.e. $n_5 = 1$, a contradiction to our assumption that $n_5 = 6$.

    Hence $n_5 = 1$. Therefore the unique Sylow 5-subgroup is a normal subgroup of $G$ by \myref{corollary-sylow-subgroup-is-normal-if-it-is-unique}.

    \item We prove that $G$ has a normal subgroup of order $p$, $q$, or $r$. By \myref{corollary-sylow-subgroup-is-normal-if-it-is-unique}, subgroups of order $p$, $q$, or $r$ are normal if they are unique. By way of contradiction, assume that they are not unique, meaning $n_p, n_q, n_r > 1$.

    By the Third Sylow Theorem (\myref{thrm-sylow-3}), $n_r \equiv 1 \pmod r$ and $n_r \mid pq$. The divisors of $pq$ are 1, $p$, $q$, and $pq$. We note that since both $p$ and $q$ are less than $r$, thus $p \not\equiv 1 \pmod r$ and $q \not\equiv 1 \pmod r$. The only possibility that is left is $n_r = pq$ as we assume $n_r \neq 1$. Similarly, $n_q \equiv 1 \pmod q$ and $n_q \mid pr$. The divisors of $pr$ are 1, $p$, $r$, and $pr$. Since $p < q$ thus $p \not\equiv 1 \pmod q$. Hence $n_q \geq r$ as we assume $n_q \neq 1$. Similarly, $n_p \geq q$.

    We now consider the number of elements with order $p$, $q$, and $r$.
    \begin{itemize}
        \item $\boxed{p}$ With $n_p \geq q$, there are at least $q(p-1)$ elements of order $p$. We minus 1 because one of the elements in a Sylow $p$-subgroup is the identity with order 1.
        \item $\boxed{q}$ With $n_q \geq r$, there are at least $r(q-1)$ elements of order $q$.
        \item $\boxed{r}$ We know $n_r = pq$ so there are exactly $pq(r-1)$ elements of order $r$.
    \end{itemize}
    Since the total number of elements, $pqr$, must be at least the sum of the numbers of these elements, thus
    \begin{align*}
        pqr &\geq q(p-1) + r(q-1) + pq(r-1)\\
        &= pq - q + qr - r + pqr - pq\\
        &= pqr + qr - q - r
    \end{align*}
    which means $qr - q - r \leq 0$. Rearranging, we see
    \[
        q \leq \frac{r}{r-1} = 1 + \frac{1}{r-1}.
    \]
    Since $p < q$ and they are both primes, we must have $q \geq 3$. Hence one sees
    \[
        3 \leq q \leq 1 + \frac{1}{r-1} \leq 2
    \]
    which is a clear contradiction. Hence, at least one of $n_p$, $n_q$, or $n_r$ is 1, meaning that there exists a non-trivial proper normal subgroup in $G$ by \myref{corollary-sylow-subgroup-is-normal-if-it-is-unique}. Therefore $G$ is non-simple.
\end{questions}

\section{Composition Series}
\begin{questions}
    \item \begin{partquestions}{\roman*}
        \item We note $\mathrm{V}$ has order 4. By writing 4 as $2 \times 2$ we know that $\mathrm{V}$ has a subgroup of order 2 (which is cyclic) by Cauchy's Theorem (\myref{thrm-cauchy}). Now $\mathrm{V}$ is abelian (\myref{problem-group-of-order-prime-squared-is-abelian}) which means that the subgroup of order 2 is normal in $\mathrm{V}$ (\myref{prop-subgroup-of-abelian-group-is-normal}). Finally, the only possible order for a proper subgroup of $\mathrm{V}$ is 2 by Lagrange's Theorem (\myref{thrm-lagrange}). Hence, the only composition series for $\mathrm{V}$ is $1 \lhd \Cn{2} \lhd \mathrm{V}$ up to isomorphism.\newline
        (Note that this analysis applies for \textit{any} group of order 4.)
        
        \item Recall that $\mathrm{Q} = \langle \alpha, \beta \vert \alpha^4 = e, \alpha^2 = \beta^2, \text{ and } \beta\alpha = \alpha^3\beta \rangle$. From the solution of \myref{exercise-normal-subgroups-of-quarternion-group}, the maximal subgroups of $\mathrm{Q}$ are $G_1 = \langle \alpha \rangle$, $G_2 = \langle \beta \rangle$, and $G_3 = \langle \alpha\beta \rangle$ (by setting $\alpha = i$ and $\beta = j$). We note the following.
        \begin{itemize}
            \item $G_1 = \{e, \alpha, \alpha^2, \alpha^3\} \cong \Cn{4}$.
            \item $G_2 = \{e, \beta, \beta^2, \beta^3\} = \{e, \beta, \alpha^2, \alpha^2\beta\} \cong \mathrm{V}$ where $a = \alpha^2$ and $b = \beta$.
            \item $G_3 = \{e, \alpha\beta, (\alpha\beta)^2, (\alpha\beta)^3\} = \{e, \alpha\beta, \alpha^2, \alpha^3\beta\} \cong \mathrm{V}$ with $a = \alpha\beta$ and $b = \alpha^2$.
        \end{itemize}
        Also, note that $\Cn{2} \cong \langle \alpha^2 \rangle \lhd G_1$, $\Cn{2} \cong \langle \beta^2 \rangle \lhd G_2$, $\Cn{2} \cong \langle (\alpha\beta)^2 \rangle \lhd G_3$. Hence, the two series up to isomorphism are 
        \begin{align*}
            1 \lhd \Cn{2} \lhd \Cn{4} \lhd \mathrm{Q} & \text{ and }\\
            1 \lhd \Cn{2} \lhd \mathrm{V} \lhd \mathrm{Q}
        \end{align*}
        
        \item By Jordan-H\"older theorem (\myref{thrm-jordan-holder}), the composition factors are isomorphic to each other. We note
        \begin{itemize}
            \item $\Cn{2} / 1 \cong \Cn{2}$;
            \item $\Cn{4} / \Cn{2} \cong \Cn{2}$ by \myref{exercise-Zmn-mod-Zn-cong-Zn}; and
            \item $\mathrm{V} / \Cn{2} \cong (\Cn{2})^2 / \Cn{2} \cong \Cn{2}$ by \myref{problem-cartesian-product-of-group-by-group-isomorphic-to-group}.
        \end{itemize}
        The only unaccounted set of factors is $\mathrm{Q}/\mathrm{V}$ and $\mathrm{Q}/\Cn{4}$. So, either $\mathrm{Q}/\mathrm{V} \cong \Cn{2}$ and $\mathrm{Q}/\Cn{4} \cong \Cn{2}$, or $\mathrm{Q}/\mathrm{V} \cong \mathrm{Q}/\Cn{4}$. Hence $\mathrm{Q}/H \cong \mathrm{Q}/K$.
    \end{partquestions}
    
    \item We know that $\An{4} \lhd \Sn{4}$ by \myref{prop-An-normal-subgroup-of-Sn}. Note $\An{4}$ is a maximal normal subgroup since $|\An{4}| = \frac{4!}2 = 12$ by \myref{prop-order-of-An}, and a subgroup's order must divide the order of the group by Lagrange's Theorem (\myref{thrm-lagrange}).
    
    Now applying that theorem on $\An{4}$, we see that the possible orders of a subgroup of $\An{4}$ are 6, 4, 3, 2, and 1. We claim that a subgroup of order 6 does not exist. Note that $\An{4}$ contains
    \begin{itemize}
        \item 1 element of order 1;
        \item 3 elements of order 2; and
        \item 8 elements of order 3.
    \end{itemize}
    If a subgroup of order 6 exists (say, $H$), then its index would be $\frac{12}{6} = 2$ (Lagrange), meaning $H$ contains all odd order elements (\myref{problem-subgroup-of-index-2}). However, there are $1 + 8 = 9$ odd order elements, meaning that $H$ has an order of at least 9, a contradiction. Hence a subgroup of $\An{4}$ of order 6 is impossible.
    
    Now we note that a subgroup of order $4 = 2^2$ exists by the First Sylow Theorem (\myref{thrm-sylow-1}) as it is a Sylow 2-subgroup. The Third Sylow Theorem (\myref{thrm-sylow-3}) tells us how many Sylow 2-subgroups there are:
    \begin{itemize}
        \item $n_2 \vert 3$, so $n_2$ is 1 or 3; and
        \item $n_2 \equiv 1 \pmod2$, so $n_2 \in \{1, 3, 5, \dots\}$.
    \end{itemize}
    Hence $n_2$ is either 1 or 3. Now if $n_2 = 3$, then the number of elements of order of 1, 2, or 4 is
    \[
        3 \times (4 - 1) + 1 = 10
    \]
    (where the 3 is $n_2$, the $4-1$ is the number of non-identity elements in each Sylow 2-subgroup, and the $+1$ is to add the identity element). However, as noted above, there are only 4 elements of order 1, 2, or 4, a contradiction. Hence $n_2 = 1$, meaning the Sylow 2-subgroup (which is a subgroup of order 4) is normal (\myref{corollary-sylow-subgroup-is-normal-if-it-is-unique}). Therefore the subgroup of order 4 is the maximal normal subgroup of $\An{4}$.
    
    We note that the subgroup of order 4 of $\An{4}$ is not $\Cn{4}$ (as this would imply that $\An{4}$ has an element of order 4, which it does not). Hence, from \myref{problem-smallest-nonabelian-group}, the subgroup of order 4 must be isomorphic to the Klein-4 group, $\mathrm{V}$.
    
    Note that a group of order 4 has a subgroup of order 2 by Cauchy's Theorem (\myref{thrm-cauchy}). Clearly such a subgroup is cyclic (since 2 is prime), and has index $\frac42 = 2$, meaning that it is normal in the group of order 4. Furthermore the trivial group is always a subgroup of any group.
    
    Hence, the composition series for $\Sn{4}$, up to isomorphism, is
    \[
        1 \lhd \Cn{2} \lhd \mathrm{V} \lhd \An{4} \lhd \Sn{4}.   
    \]
    \begin{remark}
        We list the actual subgroups that are isomorphic to the above terms in the composition series here.
        \begin{itemize}
            \item $\Cn{2}$: $\{e, \begin{pmatrix}1&2\end{pmatrix}\begin{pmatrix}3&4\end{pmatrix}\}$
            \item V: $\{e, \begin{pmatrix}1&2\end{pmatrix}\begin{pmatrix}3&4\end{pmatrix}, \begin{pmatrix}1&3\end{pmatrix}\begin{pmatrix}2&4\end{pmatrix}, \begin{pmatrix}1&4\end{pmatrix}\begin{pmatrix}2&3\end{pmatrix}\}$
            \item $\An{4}$ is an actual subgroup of $\Sn{4}$
        \end{itemize}
    \end{remark}
\end{questions}

\chapter{Simple Groups}
Simple groups can be thought of as the `building blocks' of all (finite) groups. The finite simple groups have been completely classified; each belongs to one of 18 infinite families, or is one of 26 sporadic groups that do not follow a specific pattern. We look at the classification of the families of these simple groups here.

\section{Cyclic Groups of Prime Order}
The first infinite family of simple groups we will look at is the family of Cyclic Groups of Prime Order\index{cyclic group!of prime order}.

\begin{lemma}\label{lemma-cyclic-group-simple-iff-order-is-prime}
    $\Cn{n}$ is simple if and only if $n$ is prime.
\end{lemma}
\begin{proof}
    We first prove the forward direction. Suppose $\Cn{n}$ is simple with generator $g$. Then the only normal subgroups of $\Cn{n}$ are the trivial group and the group itself. Seeking a contradiction, assume $n$ is not prime; write $n = ab$ where $a$ and $b$ are positive integers that are both smaller than $n$. Then clearly $\langle g^a\rangle$ is a proper subgroup of $\Cn{n}$. Now $\Cn{n}$ is abelian (\myref{prop-cyclic-group-is-abelian}) which means all subgroups are normal (\myref{prop-subgroup-of-abelian-group-is-normal}). Hence we have found a non-trivial proper normal subgroup of $\Cn{n}$, namely $\langle g^a \rangle$, contradicting that $\Cn{n}$ has no non-trivial proper normal subgroups. Therefore $n$ is prime.

    We now prove the reverse direction. Suppose $n$ is a prime. Then by a corollary of Lagrange's theorem (\myref{corollary-group-with-prime-order-subgroups}), $\Cn{n}$ has no non-trivial proper subgroups. So the only subgroup with order smaller than $n$ is the trivial group, $\{e\}$. Clearly $\Cn{n}$ is normal in itself, and the trivial group is always a normal subgroup. Hence, as the only normal subgroups of $\Cn{n}$ are the trivial group and itself, thus $\Cn{n}$ is simple. Therefore, if $n$ is prime then $\Cn{n}$ is simple.
\end{proof}

In fact, we have a much stronger result which we prove here.

\begin{theorem}\label{thrm-abelian-group-simple-iff-cylic-group-of-prime-order}
    An abelian group is simple if and only if it has prime order.
\end{theorem}
Note that we do not assume that the abelian group is finite; we will show that the group is finite in the proof below.
\begin{proof}
    The reverse direction follows immediately from \myref{lemma-cyclic-group-simple-iff-order-is-prime}, so we prove the forward direction only.

    Suppose $G$ is a simple abelian group; we show that $G$ is finite. Let $g$ a non-identity element of $G$. Then $H = \langle g \rangle$ is a subgroup of $G$. In fact, since $G$ is abelian, $H \unlhd G$ (\myref{prop-subgroup-of-abelian-group-is-normal}). As $G$ is simple, therefore $H = G$, meaning that $g$ is a generator of $G$. Now if $G$ is an infinite group, then one also sees that $\langle g^2 \rangle < G$ which implies $\langle g^2 \rangle \lhd G$, contradicting the fact that $G$ is simple. Hence $G$ is a finite abelian group with generator $g$, meaning $G$ is cyclic. Result follows directly from \myref{lemma-cyclic-group-simple-iff-order-is-prime}.
\end{proof}

From this, we conclude that the only family of simple abelian groups is the family of cyclic groups of prime order.

\section{Alternating Group With Degree $>4$}
The other family of simple groups that is relatively easy to find (and define) is the family of alternating groups with degree above 4\index{alternating group!of degree $>4$}. However, to prove this claim, we need several preliminary results.

\begin{theorem}\label{thrm-group-of-order-60-with->1-sylow-5-subgroup-is-simple}
    Let $G$ be a group of order 60. If $G$ has more than one Sylow 5-subgroup then $G$ is simple.
\end{theorem}
\begin{proof}[Proof (see {\cite[Proposition 4.21]{dummit_foote_2004}})]
    By way of contradiction assume $G$ is a group of order 60 with more than one Sylow 5-subgroup, but has a non-trivial proper normal subgroup $H$. Note $60 = 5 \times 12$, so by the Third Sylow Theorem (\myref{thrm-sylow-3}),
    \begin{itemize}
        \item $12 \vert n_5$, so $n_5 \in \{1, 2, 3, 4, 6, 12\}$; and
        \item $n_5 \equiv 1 \pmod 5$, so $n_5 \in \{1, 6, 11, 16, \dots\}$.
    \end{itemize}
    Therefore $n_5 = 6$ as $n_5 > 1$ (given), i.e. there are 6 Sylow 5-subgroups.

    We note by Lagrange's theorem (\myref{thrm-lagrange}) that the order of $H$ belongs in the set $\{1, 2, 3, 4, 5, 6, 10, 12, 15, 20, 30, 60\}$. As $H$ is a non-trivial proper subgroup of $G$, thus $|H| \neq 1$ and $|H| \neq 60$. That leaves 4 cases which we will deal with separately.
    \begin{enumerate}
        \item $|H| = 6$. Note $6 = 2 \times 3$, so \myref{problem-group-of-order-pq-has-normal-subgroup-of-order-q} tells us that there exists a $N \lhd H$ with $|N| = 3$. Note also $[G:H] = 10$ which is not a multiple of 3, so \myref{problem-normal-subgroup-of-G-contains-all-sylow-p-subgroups} tells us that all Sylow 3-subgroups of $G$ are in $H$. But $N \lhd H$ means that $N$ is the unique Sylow 3-subgroup of $H$ and $G$ (\myref{corollary-sylow-subgroup-is-normal-if-it-is-unique}), so $N \lhd G$ (by the same corollary). Proceed to case 3, using $N$ in place of $H$.

        \item $|H| = 12$. Note $12 = 2^2 \times 3$. Now \myref{exercise-group-of-order-12-has-normal-subgroup-of-3-or-4} (later) tells us that there exists a normal subgroup of $H$ with order 3 or 4. Call that subgroup $N$. If $|N| = 3$ then it is a Sylow 3-subgroup; if $|N| = 4 = 2^2$ it is a Sylow 2-subgroup. As $N \lhd H$, thus $N$ is the unique Sylow 2- or 3- subgroup (\myref{corollary-sylow-subgroup-is-normal-if-it-is-unique}). Since $H \lhd G$, thus $H$ contains all Sylow 2- and 3-subgroups of $G$ (\myref{problem-normal-subgroup-of-G-contains-all-sylow-p-subgroups}), meaning $G$ has only one Sylow 2-subgroup or one Sylow 3-subgroup (or both), in particular $N$. Hence, $N \lhd G$ since a Sylow $p$-subgroup is unique if and only if it is normal (\myref{corollary-sylow-subgroup-is-normal-if-it-is-unique}). Proceed with case 3, using $N$ instead of $H$.

        \item $|H| \in \{2, 3, 4\}$. Since $H \lhd G$, thus $G/H$ is a group. Note $|G/H| \in \{15, 20, 30\}$. We claim that each of these cases produces a new normal subgroup of $G/H$ (call it $\bar{P}$) with order 5. This is proven for the case where $|G/H| = 30$ in \myref{problem-group-of-order-30-has-normal-subgroup-of-order-5}; the other two cases are for \myref{exercise-group-of-order-15-or-20-has-normal-subgroup-of-order-5} (later).

        Now \myref{problem-subgroup-of-quotient-group-is-quotient-group} tells us that $\bar{P}$ has the form $K/H$ where $K < G$ and $H \subseteq K$. Since $\bar{P} = K/H \lhd G/H$, thus for any $g \in G$ and $kH \in \bar{P}$ we have
        \[
            (gH)(kH)(g^{-1}H) = (gkg^{-1})H \in K/H,
        \]
        which means $gkg^{-1} \in K$. Therefore $K \lhd G$ by definition of normality.

        Observe that this means that
        \[
            |K| = |K/H||H| = |\bar{P}||H| = 5|H|,
        \]
        meaning $K$ is a normal subgroup of $G$ with an order that is a multiple of 5. Proceed to case 4, using $K$ in place of $H$.

        \item $|H|$ is a multiple of 5, meaning $H$ has a Sylow 5-subgroup. Note that there are $5-1=4$ non-identity elements in each Sylow 5-subgroup; therefore
        \[
            |H| \geq n_5(5-1) = 24
        \]
        which means that $|H| = 30$. By \myref{problem-group-of-order-30-has-normal-subgroup-of-order-5} again, such a group has only a unique Sylow 5-subgroup.  Note $5 \nmid [G:H]$, so \myref{problem-normal-subgroup-of-G-contains-all-sylow-p-subgroups} implies all Sylow 5-subgroups of $G$ are in $H$. However, right at the start, we concluded that there are 6 Sylow 5-subgroups in $G$, so $H$ must have 6 Sylow 5-subgroups, a contradiction.
    \end{enumerate}
    Hence, $H$ does not exist, and so $G$ is simple.
\end{proof}

\begin{exercise}\label{exercise-group-of-order-12-has-normal-subgroup-of-3-or-4}
    Prove that a group of order 12 either has a normal subgroup of order 3, or a normal subgroup of order 4, or both.
\end{exercise}

\begin{exercise}\label{exercise-group-of-order-15-or-20-has-normal-subgroup-of-order-5}
    Prove that a group of each of the following orders has a normal subgroup of order 5.
    \begin{partquestions}{\alph*}
        \item 15
        \item 20
    \end{partquestions}
\end{exercise}

\begin{corollary}\label{corollary-A5-is-simple}
    The group $\An5$ is simple.
\end{corollary}
\begin{proof}
    \myref{exercise-A5-has-two-distinct-subgroups-of-order-5} (later) gives two distinct subgroups of order 5. Since $|\An{5}| = 60 = 2^2 \times 3 \times 5$, thus subgroups of order 5 are Sylow 5-subgroups. Therefore $\An5$ is simple by \myref{thrm-group-of-order-60-with->1-sylow-5-subgroup-is-simple}.
\end{proof}
\begin{exercise}\label{exercise-A5-has-two-distinct-subgroups-of-order-5}
    Consider the permutation $\sigma = \begin{pmatrix}1&3&2&4&5\end{pmatrix}$.
    \begin{partquestions}{\roman*}
        \item Explain why $\sigma \in \An{5}$.
        \item Find the order of the subgroup $\langle \sigma \rangle$.
        \item Find another subgroup of $\An{5}$ with order 5.
    \end{partquestions}
\end{exercise}

We also state and prove a fairly obvious proposition.

\begin{proposition}\label{prop-An-stabilizer-of-i-is-isomorphic-to-A(n-1)}
    Let the integer $n \geq 3$. Let the set $\{1, 2, 3, \dots, n\}$ be denoted by $\mathcal{N}_n$. Suppose $\An{n}$ acts on $\mathcal{N}_n$ naturally. Then $\Stab{\An{n}}{r} \cong \An{n-1}$.
\end{proposition}
\begin{proof}
    We note that elements of $\Stab{\An{n}}{r}$ are permutations that fix $r$, thereby permuting the $n - 1$ other elements. Therefore the elements of $\Stab{\An{n}}{r}$ are even permutations on $n - 1$ elements, i.e. $\Stab{\An{n}}{r} \cong \An{n-1}$.
\end{proof}

With these results, we are ready to prove the main result of this section.

\begin{theorem}\label{thrm-An-is-simple-for-n>=5}
    The group $\An{n}$ is simple if $n \geq 5$.
\end{theorem}
\begin{proof}[Proof (see {\cite[Theorem 4.24]{dummit_foote_2004}})]
    We induct on $n$. For brevity, let $\mathcal{N}_n = \{1, 2, 3, \dots, n\}$. The base case of $n = 5$ is covered by \myref{corollary-A5-is-simple}. Assume that $\An{k-1}$ is simple for some $k \geq 6$; we will prove that $\An{k}$ is also simple.

    Let $G = \An{k}$ and, seeking a contradiction, assume that $G$ has a non-trivial proper normal subgroup $H$. Let $G$ act on $\mathcal{N}_{k}$ naturally; thus we see that $\Stab{G}{i} \leq G$ with $\Stab{G}{i} \cong \An{k-1}$ (\myref{prop-An-stabilizer-of-i-is-isomorphic-to-A(n-1)}) for any $i \in \mathcal{N}_k$. Note $\An{k-1}$ is simple by the induction hypothesis, so $\Stab{G}{i}$ is simple for each $i \in \mathcal{N}_{k}$.

    Suppose first that there is some non-identity $\pi \in H$ such that $\pi(i) = i$ for some $i \in \mathcal{N}_{k}$. This means that $\pi$ fixes $i$; thus $\pi \in H \cap \Stab{G}{i}$. Note that since $H \lhd G$ and $\Stab{G}{i} \leq G$ thus $H \cap \Stab{G}{i} \lhd \Stab{G}{i}$ by the Second Isomorphism Theorem (\myref{thrm-isomorphism-2}), statement 4. But as $\Stab{G}{i}$ is simple (and non-trivial) we must have $H \cap \Stab{G}{i} = \Stab{G}{i}$. Therefore $\Stab{G}{i} \subseteq H$ which means $\Stab{G}{i} \leq H$. Now by \myref{exercise-conjugate-of-stabilizer} (later), for any $\sigma \in G$, we know that $\sigma\Stab{G}{i}\sigma^{-1} = \Stab{G}{\sigma(i)}$. Therefore we see
    \[
        \sigma\Stab{G}{i}\sigma^{-1} \leq \sigma H\sigma^{-1} = H
    \]
    since $H \lhd G$. Thus, for any $j \in \mathcal{N}_{k+1}$, there exists $\sigma \in G$ where $\sigma(i) = j$ such that
    \[
        \sigma\Stab{G}{i}\sigma^{-1} = \Stab{G}{j} \leq H.
    \]

    Note that any $\lambda \in G$ may be written as a product of an even number of transpositions (\myref{thrm-parity-of-permutation}), say $2t$ transpositions. Thus, we may write $\lambda = \lambda_1\lambda_2\cdots\lambda_t$ where each $\lambda_i$ is a product of two transpositions. Now as $k \geq 5$, each $\lambda_i$ (which could at most consist of two disjoint cycles of 4 elements) must fix at least one element in $\mathcal{N}_{k}$, say $j$. That is, $\lambda_i \in \Stab{G}{j}$ for some $j \in \mathcal{N}_{k}$. Since $\lambda_i \in \Stab{G}{j} \leq H$ for some $j \in \mathcal{N}_{k}$, thus $\lambda_i \in H$. Hence $\lambda = \lambda_1\lambda_2\cdots\lambda_t \in H$. Therefore, any element in $G$ is also in $H$, meaning $G \subseteq H$, contradicting the fact that $H \lhd G$.

    We conclude that for any $\pi \in H$, if $\pi \neq \id$ then $\pi(i) \neq i$ for all $i \in \mathcal{N}_k$. The contrapositive of this statement is that if $\pi(i) = i$ for some $i \in \mathcal{N}_k$ then $\pi = \id$. Now suppose $\pi_1, \pi_2 \in H$ and $\pi_1(i) = \pi_2(i)$ for some $i \in \mathcal{N}_{k}$. Then $\pi_2^{-1}\pi_1(i) = i$, which implies $\pi_2^{-1}\pi_1 = \id$. Hence $\pi_1 = \pi_2$. Therefore, if $\pi_1, \pi_2 \in H$ and $\pi_1(i) = \pi_2(i)$ for some $i \in \mathcal{N}_{k}$, then $\pi_1 = \pi_2$.

    Now suppose a non-identity $\pi_1 \in H$ exists such that the cycle decomposition of $\pi_1$ contains a cycle of length of at least 3, say
    \[
        \pi_1 = \begin{pmatrix}a_1&a_2&a_3&\cdots\end{pmatrix} \begin{pmatrix}b_1&b_2&\cdots\end{pmatrix}\cdots
    \]
    where $a_1$, $a_2$, $a_3$, $b_1$, $b_2$, etc. are distinct (which is possible since $k \geq 5$). We note an element $\sigma \in G$ exists such that $\sigma(a_1) = a_1$, $\sigma(a_2) = a_2$, but $\sigma(a_3) \neq a_3$ because $k \geq 4$ (for example, the permutation $\begin{pmatrix}a_3 & a_4\end{pmatrix}$). Then \myref{exercise-conjugation-of-permutation-by-another} (later) tells us that
    \[
        \sigma\pi_1\sigma^{-1} = \begin{pmatrix}a_1&a_2&\sigma(a_3)&\cdots\end{pmatrix} \begin{pmatrix}\sigma(b_1)&\sigma(b_2)&\cdots\end{pmatrix}\cdots.
    \]
    Set $\pi_2 = \sigma\pi_1\sigma^{-1}$, which is clearly distinct from $\pi_1$. Then we see $\pi_1(a_1) = \pi_2(a_1) = a_2$, contrary to the above observation that $\pi_1(i) = \pi_2(i)$ for any $i \in \mathcal{N}_k$ implies $\pi_1 = \pi_2$. Therefore only 2-cycles can appear in the cycle decomposition of non-identity elements of $H$.

    Let $\pi_1 \in H$ be a non-identity element, so that
    \[
        \pi_1 = \begin{pmatrix}a_1&a_2\end{pmatrix} \begin{pmatrix}a_3&a_4\end{pmatrix} \begin{pmatrix}a_5&a_6\end{pmatrix}\cdots
    \]
    where each $a_i$ is distinct (note that $k \geq 6$ is used above). Consider the permutation $\sigma = \begin{pmatrix}a_1&a_2\end{pmatrix} \begin{pmatrix}a_3&a_5\end{pmatrix}$, which is in $G$ since it is made up of 2 transpositions. Then \myref{exercise-conjugation-of-permutation-by-another} again gives
    \[
        \sigma\pi_1\sigma^{-1} = \begin{pmatrix}a_1&a_2\end{pmatrix} \begin{pmatrix}a_5&a_4\end{pmatrix} \begin{pmatrix}a_3&a_6\end{pmatrix}\cdots.
    \]
    Setting $\pi_2 = \sigma\pi_1\sigma^{-1}$ again gives two distinct permutations $\pi_1$ and $\pi_2$ where $\pi_1(a_1) = \pi_2(a_1) = a_2$, again contrary to the above observation.

    We conclude that such a non-trivial proper normal subgroup $H$ of $\An{k}$ cannot exist. Thus, $\An{k-1}$ being simple implies that $\An{k}$ is also simple.

    By mathematical induction, $\An{n}$ is simple for all $n \geq 5$.
\end{proof}

\begin{exercise}\label{exercise-conjugate-of-stabilizer}
    Let $S$ be a non-empty set and let $G \leq \Sym{S}$ act on $S$. Show that $\sigma\Stab{G}{x}\sigma^{-1} = \Stab{G}{\sigma(x)}$ for any $\sigma \in G$ and $x \in S$.
\end{exercise}

\begin{exercise}\label{exercise-conjugation-of-permutation-by-another}
    Let $\sigma, \pi \in \Sn{n}$. Suppose $\sigma$ has cycle decomposition
    \[
        \begin{pmatrix}a_1&a_2&\cdots&a_{k_1}\end{pmatrix} \begin{pmatrix}b_1&b_2&\cdots&b_{k_2}\end{pmatrix}\cdots,
    \]
    where $a_1, a_2, \dots, a_{k_1}, b_1, b_2, \dots, b_{k_2}, \dots$ are all distinct. Show that
    \[
        \pi\sigma\pi^{-1} = \begin{pmatrix}\pi(a_1)&\cdots&\pi(a_{k_1})\end{pmatrix} \begin{pmatrix}\pi(b_1)&\cdots&\pi(b_{k_2})\end{pmatrix}\cdots,
    \]
    that is, $\pi\sigma\pi^{-1}$ is obtained from $\sigma$ by replacing each entry $i$ by $\pi(i)$.
\end{exercise}

\begin{corollary}
    The group $\An{n}$ is simple for $n \geq 3$ and $n \neq 4$.
\end{corollary}
\begin{proof}
    We note $\An3$ has order $\frac{3!}{2} = 3$ which is prime, so $\An3 \cong \Cn3$ which is simple. Also $\An{n}$ is simple for $n \geq 5$ by \myref{thrm-An-is-simple-for-n>=5}.
\end{proof}

We note that $\An4$ is non-simple by the solution of \myref{problem-S4-composition-series}, in which we found that $\An4$ has a unique composition series of
\[
    1 \lhd \Cn2 \lhd \mathrm{V} \lhd \An4
\]
up to isomorphism.

\section{Groups of Lie Type}
We briefly mention groups of Lie type; we will not prove any significant results here.

Groups of Lie (pronounced ``lee'') type\index{groups of Lie type} usually refers to finite groups that are closely related to the group of rational points of a reductive linear algebraic group with values in a finite field. We will cover finite fields in part III. We briefly mention these groups here.

The list below, taken from \cite{wikipedia_list-of-simple-groups}, is a list of the families of simple groups of Lie type. In what follows, $n$ is a positive integer and $q$ is a positive power of a prime number $p$.
\begin{itemize}
    \item \term{Classical Chevalley groups}\index{Chevalley groups!classical}: there are 4 families of simple groups.
    \begin{itemize}
        \item $A_n(q)$, except for $A_1(2)$ and $A_1(3)$. There are several duplicates, which are
        \begin{itemize}
            \item $A_1(4) \cong A_1(5) \cong \An{5}$;
            \item $A_1(7) \cong A_2(2)$;
            \item $A_1(9) \cong \An{6}$; and
            \item $A_3(2) \cong \An{8}$.
        \end{itemize}
        We note that $\An{n}$ is not the same as $A_n(q)$. We distinguish between the alternating group of degree $n$ ($\An{n}$) and the groups of Lie type $A_n(q)$ by letting the latter be in italics and the former be in `normal' font.

        \item $B_n(q)$ for $n > 1$, except for $B_2(2)$. There are several duplicates, which are
        \begin{itemize}
            \item $B_n(2^m) \cong C_n(2^m)$; and
            \item $B_2(3) \cong {^2A_3(2^2)} = {^2A_3(4)}$, where ${^2A_3(4)}$ is a classical Steinberg group.
        \end{itemize}
        \item $C_n(q)$ for $n > 2$. The only duplicate is $C_n(2^m) \cong B_n(2^m)$ mentioned earlier.
        \item $D_n(q)$ for $n > 3$.
    \end{itemize}

    \item \term{Exceptional Chevalley groups}\index{Chevalley groups!exceptional}: there are 5 families of such groups.
    \begin{itemize}
        \item $E_6(q)$;
        \item $E_7(q)$;
        \item $E_8(q)$;
        \item $F_4(q)$; and
        \item $G_2(q)$, except for $G_2(2)$.
    \end{itemize}

    \item \term{Classical Steinberg groups}\index{Steinberg groups!classical}: there are 2 families of simple groups.
    \begin{itemize}
        \item ${^2A_n(q^2)}$ for $n > 1$, except for ${^2A_2(2^2)} = {^2A_2(4)}$. The only duplicate is ${^2A_3(2^2)} \cong B_2(3)$ mentioned earlier.
        \item ${^2D_n(q^2)}$ for $n > 3$.
    \end{itemize}

    \item \term{Exceptional Steinberg groups}\index{Steinberg groups!exceptional}: there are 2 families of simple groups.
    \begin{itemize}
        \item ${^2E_6(q^2)}$; and
        \item ${^3D_4(q^3)}$.
    \end{itemize}

    \item \term{Suzuki groups}\index{Suzuki groups}: there is 1 family of simple groups, which is ${^2B_2(q)}$ where $q = 2^{2n+1}$ and $n \geq 1$. Such a group has order $q^2(q^2+1)(q-1)$.

    \item \term{Ree groups}\index{Ree groups}: there are 2 families of simple groups.
    \begin{itemize}
        \item $^2F_4(q)$ where $q = 2^{2n+1}$ and $n \geq 1$. The order of such a group is $q^{12}(q^6+1)(q^4-1)(q^3+1)(q-1)$.
        \item $^2G_2(q)$ where $q = 3^{2n+1}$ and $n \geq 1$. The order of such a group is $q^3(q^3+1)(q-1)$.
    \end{itemize}
\end{itemize}

There is also the \term{Tits group}\index{Tits group}, $^2F_4(2)'$, with an order of $17,971,200$. It is the commutator subgroup of $^2F_4(2)$, which is a Lie group but not a simple group. The fact that $^2F_4(2)'$ is linked to Ree groups makes most authors consider it not a sporadic group (see below).

\section{The Sporadic Groups}
Along with the 18 infinite families of simple groups, there are also 26 sporadic simple groups\index{sporadic group} that do not fall within the families (27 if the Tits group is considered a sporadic group).

We first list four categories of sporadic groups.
\begin{table}[h]
    \centering
    \begin{tabular}{|l|l|l|}
        \hline
        \textbf{Name} & \textbf{Symbol} & \textbf{Order} \\ \hline
        \multirow{5}{*}{\term{Mathieu Groups}\index{Mathieu groups}} & $\mathrm{M}_{11}$ & 7,290 \\ \cline{2-3}
        & $\mathrm{M}_{12}$ & 95,040 \\ \cline{2-3}
        & $\mathrm{M}_{22}$ & 443,520 \\ \cline{2-3}
        & $\mathrm{M}_{23}$ & 10,200,960 \\ \cline{2-3}
        & $\mathrm{M}_{24}$ & 244,823,040 \\ \hline
        \multirow{4}{*}{\term{Janko Groups}\index{Janko groups}} & $\mathrm{J}_1$ & 175,560 \\ \cline{2-3}
        & $\mathrm{J}_2$ & 604,800 \\ \cline{2-3}
        & $\mathrm{J}_3$ & 50,232,960 \\ \cline{2-3}
        & $\mathrm{J}_4$ & 86,775,571,046,077,562,880 \\ \hline
        \multirow{3}{*}{\term{Conway Groups}\index{Conway groups}} & $\mathrm{Co}_3$ & 495,766,656,000 \\ \cline{2-3}
        & $\mathrm{Co}_2$ & 42,305,421,312,000 \\ \cline{2-3}
        & $\mathrm{Co}_1$ & 4,157,776,806,543,360,000 \\ \hline
        \multirow{3}{*}{\term{Fischer Groups}\index{Fischer groups}} & $\mathrm{Fi}_{22}$ & 64,561,751,654,400 \\ \cline{2-3}
        & $\mathrm{Fi}_{23}$ & 4,089,470,473,293,004,800 \\ \cline{2-3}
        & $\mathrm{Fi}_{24}$ & 1,255,205,709,190,661,721,292,800 \\ \hline
    \end{tabular}
\end{table}



More sporadic groups are listed below.
\begin{table}[h]
    \centering
    \begin{tabular}{|l|l|l|}
        \hline
        \textbf{Group}        & \textbf{Symbol} & \textbf{Order}  \\ \hline
        \term{Higman-Sims group}\index{Higman-Sims group}     & HS              & 44,352,000      \\ \hline
        \term{McLaughlin group}\index{McLaughlin group}      & McL             & 898,128,000     \\ \hline
        \term{Held group}\index{Held group}            & He              & 4,030,387,200   \\ \hline
        \term{Rudvalis group}\index{Rudvalis group}        & Ru              & 145,926,144,000 \\ \hline
        \term{Suzuki sporadic group}\index{Suzuki sporadic group} & Suz             & 448,345,497,600 \\ \hline
        \term{O'Nan group}\index{O'Nan group}           & $\mathrm{O'N}$  & 460,815,505,920 \\ \hline
        \term{Harada-Norton group}\index{Harada-Norton group}   & HN              & 273,030,912,000,000    \\ \hline
        \term{Lyons group}\index{Lyons group}           & Ly              & 51,765,179,004,000,000 \\ \hline
        \term{Thompson group}\index{Thompson group}        & Th              & 90,745,943,887,872,000 \\ \hline
    \end{tabular}
\end{table}

The remaining 2 sporadic groups are special in that they have extremely large order.
\begin{itemize}
    \item The \term{Baby Monster group}\index{Baby Monster group}, usually denoted $\mathrm{B}$, has order
    \begin{align*}
        &2^{41} \times 3^{13} \times 5^6 \times 7^2 \times 11 \times 13 \times 17 \times 19 \times 23 \times 31 \times 47\\
        &= 4,154,781,481,226,426,191,177,580,544,000,000.
    \end{align*}
    \item The \term{Monster group}\index{Monster group}, usually denoted $\mathrm{M}$, has order $2^{46} \times 3^{20} \times 5^9 \times 7^6 \times 11^{2} \times 13^3 \times 17 \times 19 \times 23 \times 29 \times 31 \times 41 \times 47 \times 59 \times 71$ which equals 808,017,424,794,512,875,886,459,904,961,710,757,005,754,368,\linebreak000,000,000. It is the largest sporadic group.
\end{itemize}

\section{The Classification Theorem of Finite Simple Groups}
One might rightly wonder what the importance of listing out all of these different types of simple groups are. It turns out that, amazingly, that these results provide a complete classification of what a finite simple group can really be. This is captured in the Classification Theorem of Finite Simple Groups\index{Classification Theorem of Finite Simple Groups}, which is sometimes called the enormous theorem\index{Enormous Theorem}.

\begin{theorem}[Classification Theorem]
    Every finite simple group is isomorphic to either
    \begin{itemize}
        \item a cyclic group of prime order;
        \item an alternating group with degree of at least 5;
        \item a group in the 16 infinite families of groups of Lie type, or the Tits group; or
        \item one of 26 sporadic groups.
    \end{itemize}
\end{theorem}

The proof of this theorem required tens of thousands of pages in hundreds of articles, written by a large number of authors that were  published mostly between 1955 to 2004. The longest paper, and the last paper needed to fill in the gap for quasithin groups, was published in 2004 by Aschbacher and Smith and spanned in a 1221 pages. But after all that work, mathematicians had a complete classification of all finite simple groups.


\chapter{Image Acknowledgements}
Unless otherwise stated, all images are the author's own work, and are released under the same licence as this book.

\section{Cover Page}
Image of the icosahedron was created by user \code{Watchduck} on Wikimedia. It is licensed under the Creative Commons Attribution 4.0 International licence. The original file is available \href{https://commons.wikimedia.org/wiki/File:Icosahedron_with_colored_vertices.png}{here}.

\section{Introduction to Groups}
\begin{itemize}
    \item Image of the circular decorative knot with twelve crossings is released into the public domain by user \code{AnonMoos} on Wikimedia. The original file is available \href{https://commons.wikimedia.org/wiki/File:Circular-cross-decorative-knot-12crossings.svg}{here}.
\end{itemize}

\section{Basics of Groups}
\begin{itemize}
    \item Images of symmetries of an equilateral triangle and a square taken from \cite[p. 13]{milne_2021}.
\end{itemize}

\printbibliography[heading=bibintoc, title={References and Bibliography}]
\printindex


\end{document}
