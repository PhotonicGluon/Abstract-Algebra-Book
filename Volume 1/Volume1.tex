\documentclass[
    a5paper,
    pagesize,
    11pt,
    bibtotoc,
    normalheadings,
    twoside,
    openany,
    chapterprefix,
    DIV=9
]{scrbook}

\usepackage[utf8]{inputenc}
\usepackage{tocloft}
\usepackage{mathtools}
\usepackage{amsfonts}
\usepackage{enumitem}
\usepackage{amsmath}
\usepackage{amsthm}
\usepackage{amssymb}
\usepackage[hmargin=2cm, vmargin=2.5cm]{geometry}
\usepackage{graphicx}
\usepackage{wrapfig}
\usepackage{parskip}
\usepackage{framed}
\usepackage{fancyhdr}
\usepackage{emptypage}
\usepackage{multicol}
\usepackage{imakeidx}
\usepackage[breaklinks]{hyperref}
\usepackage[capitalise, nameinlink]{cleveref}
\usepackage[x11names]{xcolor}
\usepackage{crossreftools}

\usepackage[
    backend=bibtex,
    style=alphabetic,
    sorting=ynt
]{biblatex}

%=========== Path to images ==============
\graphicspath{{./images/}}

%============== Resources ================
\addbibresource{../AbstractAlgebra.bib}

%============ Redefinitions ==============
\let\oldemptyset\emptyset
\let\emptyset\varnothing

\let\totient\varphi

\renewcommand{\vert}{ \ \vline \ }
\newcommand{\vertalt}{ \ | \ }

\newcommand{\myref}[1]{\textbf{\crthypercref{#1}}}
\newcommand{\myreffigures}[1]{\textbf{\cref{#1}}}

\renewcommand{\qedsymbol}{\ensuremath{\blacksquare}}

%=========== Theorem Styles ==============
\newtheoremstyle{theorem-style}
    {-12pt}      % Space above
    {-5pt}       % Space below
    {}           % Font to use in the theorem
    {0pt}        % Measure of space to indent
    {\bfseries}  % Name of the head font
    {.}          % Punctuation between head and body
    { }          % Space after theorem head; " " = normal inter-word space
    {\thmname{#1}\thmnumber{ #2}\textit{\thmnote{ (#3)}}}

\newtheoremstyle{definition-style}
    {-12pt}      % Space above
    {-5pt}       % Space below
    {}           % Font to use in the definition
    {0pt}        % Measure of space to indent
    {\bfseries}  % Name of the head font
    {.}          % Punctuation between head and body
    { }          % Space after theorem head; " " = normal inter-word space
    {\thmname{#1}\thmnumber{ #2}\textnormal{\thmnote{ (#3)}}}

\newtheoremstyle{exercise-style}
    {-5pt}       % Space above
    {\topsep}    % Space below
    {}           % Font to use in the exercise
    {0pt}        % Measure of space to indent
    {\bfseries}  % Name of the head font
    {.}          % Punctuation between head and body
    { }          % Space after theorem head; " " = normal inter-word space
    {\thmname{#1}\thmnumber{ #2}\textnormal{\thmnote{ (#3)}}}

%======== Theorem-Like Things ============
\theoremstyle{theorem-style}\newtheorem{theoremhidden}{Theorem}[section]
\renewcommand{\thetheoremhidden}{\Roman{part}.\arabic{chapter}.\arabic{section}.\arabic{theoremhidden}}

\theoremstyle{theorem-style}\newtheorem{lemmahidden}[theoremhidden]{Lemma}

\theoremstyle{theorem-style}\newtheorem{propositionhidden}[theoremhidden]{Proposition}

\theoremstyle{theorem-style}\newtheorem{corollaryhidden}[theoremhidden]{Corollary}

\theoremstyle{definition-style}\newtheorem{definitionhidden}[theoremhidden]{Definition}

\theoremstyle{exercise-style}\newtheorem{exercisehidden}{Exercise}[chapter]
\renewcommand{\theexercisehidden}{\Roman{part}.\arabic{chapter}.\arabic{exercisehidden}}

\theoremstyle{definition}\newtheorem{problem}{Problem}[chapter]
\renewcommand{\theproblem}{\Roman{part}.\arabic{chapter}.\arabic{problem}}

\theoremstyle{definition}\newtheorem*{remark}{Remark}
\theoremstyle{definition}\newtheorem{example}[theoremhidden]{Example}

%============ Environments ===============
\newenvironment{theorem}
{\definecolor{shadecolor}{named}{DarkSeaGreen2}\begin{shaded}\noindent\begin{theoremhidden}}
{\end{theoremhidden}\end{shaded}}

\newenvironment{lemma}
{\definecolor{shadecolor}{named}{Honeydew2}\begin{shaded}\noindent\begin{lemmahidden}}
{\end{lemmahidden}\end{shaded}}

\newenvironment{proposition}
{\definecolor{shadecolor}{named}{Honeydew1}\begin{shaded}\noindent\begin{propositionhidden}}
{\end{propositionhidden}\end{shaded}}

\newenvironment{corollary}
{\definecolor{shadecolor}{named}{DarkSeaGreen1}\begin{shaded}\noindent\begin{corollaryhidden}}
{\end{corollaryhidden}\end{shaded}}

\newenvironment{definition}
{\definecolor{shadecolor}{named}{LightCyan1}\begin{shaded}\noindent\begin{definitionhidden}}
{\end{definitionhidden}\end{shaded}}

\newenvironment{exercise}
{\begin{framed}\noindent\begin{exercisehidden}}
{\end{exercisehidden}\end{framed}}

%=========== Custom Commands =============
\newcommand{\code}[1]{\texttt{#1}}  % Code block
\makeatletter\newcommand*{\rom}[1]{\Ifstr{#1}{0}{0}{\expandafter\@slowromancap\romannumeral #1@}}\makeatother  % Roman numeral

\newcommand{\lcm}{\mathrm{lcm}}  % Lowest common multiple function
\newcommand{\sgn}{\mathrm{sgn}}  % Signum function

\newcommand{\im}{\mathrm{im}\;}  % Image of a function
\newcommand{\id}{\mathrm{id}}    % Identity function

%======== Custom Chapter Styling =========
\makeatletter
\renewcommand{\chaptermark}[1]{
    \markboth{\if@mainmatter\chapapp~\thechapter.\ \fi#1}{}
}

\renewcommand*{\chapterformat}{
  \MakeUppercase{\chapapp\nobreakspace\thechapter}
}

\renewcommand*{\chapterlineswithprefixformat}[3]{
    \Ifstr{#1}{chapter}{
        \vspace{-60px}
        \Ifstr{#2}{\empty}{\vspace{40px}}{\raggedleft#2}
        \vspace{-15px}
        \rule{\linewidth}{1pt}\par\nobreak
        \centering{#3}
        \vspace{-10px}
        \rule{\linewidth}{1pt}\par\nobreak
        \vspace{-10px}
    }{#2#3}
}
\makeatother

%======== Figure Caption Format ==========
\usepackage[labelfont=bf]{caption}
\DeclareCaptionLabelFormat{custom}{#1 \Roman{part}.#2.}
\captionsetup{labelformat=custom,labelsep=space}

%============ Custom Header ==============
\fancypagestyle{plain}{\fancyhf{}\renewcommand{\headrulewidth}{0pt}}  % To clear page numbers from footer, and header line at the start of every chapter

\pagestyle{fancy}
\fancyhf{}  % Clear header/footer

\fancyhead[LE,RO]{\thepage}
\fancyhead[LO,RE]{\textit{\nouppercase\leftmark}}

%========= Customise TOC Heading =========
\makeatletter
\def\createtoc{
    \renewcommand\tableofcontents{
        \chapter*{\contentsname}
        \@starttoc{toc}
    }
    \tableofcontents
}
\makeatother

%======= Customise Draft Watermark =======
\newcommand{\setasdraft}{
    \usepackage{draftwatermark}
    \SetWatermarkLightness{0.95}
    \SetWatermarkScale{5}
}

%========= Front Matter Pages ============
\def\volumetitle{Volume \rom{\volumenumber}: \volumename}

\def\frontmatterpages{
    \frontmatter  % Use lowercase roman numerals for page numbers

    % Title page
    \begin{titlepage}
        \centering{
            \selectfont
            \Huge
            \textbf{Abstract Algebra}\\
            \vspace{-0.2cm}
            
            \Large
            \textbf{A Simple Introduction}\\
            \vspace{0.5cm}
            
            \LARGE
            \volumetitle
            \vspace{2cm}
        }\\
        \centering{\Large{Overwrite}}
        \vspace{\fill}

        \includegraphics[width=5cm]{\volumeimage}
        \vspace{\fill}

        \centering \small{\textit{Version \version}}
    \end{titlepage}

    \newpage{}

    % Edition notice
    \clearpage\null\vfill
    \thispagestyle{empty}
    \begin{minipage}[b]{0.9\textwidth}
        \footnotesize\raggedright
        \setlength{\parskip}{0.5\baselineskip}

        Published by Kan Onn Kit\\
        Singapore
        \vspace{5cm}

        \textbf{Abstract Algebra: A Simple Introduction -- \volumetitle}\par
        Version \version
        \vspace{0.3cm}

        Copyright \copyright \ 2022 -- \the\year\ by Kan Onn Kit\par
        This work is licensed under a
        Creative Commons Attribution-NonCommercial-ShareAlike 4.0 International Licence.\par
        \includegraphics[width=2.5cm]{../Images/CC BY-NC-SA 4.0.png}\\  % With reference to the volumes' folders
        The full licence text is available at \url{http://creativecommons.org/licenses/by-nc-sa/4.0/}.\par    
        The source files for the project are available \href{https://github.com/PhotonicGluon/Abstract-Algebra-Book}{here}.
        \vspace{0.3cm}

        Typeset in 11pt Computer Modern Roman using PDF\LaTeX.
    \end{minipage}

    \vspace*{2\baselineskip}
    \cleardoublepage

    % "Quote" page
    \thispagestyle{empty}
    \vspace*{2cm}

    \begin{center}
        \Large{\parbox{10cm}{
            \begin{raggedright}
                \Large
                \quotepagetext
                \vspace{0.3cm}
                
                \hfill
                --- \quotepageattribution\\
                \vspace{-0.25cm}
                
                \hfill
                \normalsize
                (\quotepagecitation)
            \end{raggedright}
        }
    }
    \end{center}

    \newpage

    % Table of contents
    \createtoc
    \setcounter{part}{\volumenumber}

    % Acknowledgements
    \chapter{Acknowledgements}
    Undertaking such a monumental project is new to me, and I am indebted to the people who accompanied me on this journey.

    I am eternally grateful to my parents, who have spent countless hours and an ungodly amount of effort to raise me into who I am today. Their omnipresent kindness, patience, and love for me are something I certainly do not deserve, and I thank them for taking care of me.
    
    I would like to thank my tutor Leong Chong Ming, who got me interested in abstract algebra in the first place. His enthusiasm and eagerness in sharing his knowledge on the subject is the driving force behind my decision to write these books.

    I am grateful for the help of my friend Low Ji Yuan, who has assisted me with countless revisions of the content in these books and given me another pair of eyes in the vetting of content.

    I also sincerely appreciate the support from my mathematics tutors, Loke Weng Heng, Siow Yun Jie, and Teng Yen Ping, who has been there through my junior college years inspiring me with the wonders of mathematics. I am indebted to them for allowing me to excel in my final examinations.

    My close friends, Aidan Tay, Gabriel Fong, and Low Ji Yuan, accompanied me through two years of schooling (and math jokes). I offer infinite thanks to them for sticking with me and for encouraging this math nerd to pursue his wacky projects.

    A thousand thanks go out to my teachers at the School of Science and Technology, Singapore, and specifically my form teacher Lee Tsi Yew Samuel, who instilled important character values into me so I can excel in my future endeavours.

    % Preface
    \chapter{Preface}
    Although algebra has a long history, it has undergone quite striking changes in the past few decades. Abstract (or modern) algebra is widely recognised as an essential element of higher mathematical education. The results that it showcases, however, are often hard to grasp and understand without prerequisite knowledge or with a heavy background in mathematics. Most books on this subject are crafted for undergraduates at universities. They are not for a general mathematics enthusiast or one who seeks to understand more about the inner structure of algebra that mathematicians encounter frequently.

    The exploration of such structures is fundamental to the current underpinning of scientific inquiries. For example, groups are important as they describe the symmetries which the laws of physics seem to obey. Finite fields are also used in coding theory and combinatorics. I hope this series of books will inspire more people to learn more about abstract algebra, beyond the simple introduction presented here.

    This series of books serves to achieve several goals.
    \begin{itemize}
        \item Provide a step-by-step explanation of core results from abstract algebra, without ambiguity of the results discussed.
        \item Demystify the core steps that many textbooks skip over when writing proofs.
        \item Ensure that results from abstract algebra are as accessible, as approachable, and as understandable for as many people as possible.
    \end{itemize}
    I hope that these books can accomplish these goals and let readers enjoy the wonders of abstract algebra.

    \hfill{\textit{22 March, 2023}}

    \section*{Preface for Volume \rom{\volumenumber}}
    \prefacevolumetext
    
    \hfill{\textit{\prefacevolumedate}}

    % Suggestions on the use of this book
    \chapter{Suggestions on the Use of This Book}
    \section*{General Information}
    \begin{itemize}
        \item For most volumes, we include both exercises and problems.
        \begin{itemize}
            \item An exercise can be thought of as a simple ``self-review'' question. Exercises ensure that the content of a particular section is understood and should not be too hard to answer.
            \item A problem is a more holistic version of an exercise. Generally, solutions to problems require a thorough understanding of the current chapter and may require results from other chapters.
        \end{itemize}
        \item A consistent labelling system for all the results within and between volumes is necessary for a project as long as this one.
        \begin{itemize}
            \item All definitions, examples, lemmas, theorems, propositions, and corollaries are consecutively numbered, using the format
            \begin{quote}
                \code{[VOLUME].[CHAPTER].[SECTION].[NUMBER]}
            \end{quote}
            For example, the fourth statement in Volume I, chapter 2, section 3 is labelled \textbf{I.2.3.4}.
            \item Exercises and problems are also numbered consecutively, using the format
            \begin{quote}
                \code{[VOLUME].[CHAPTER].[NUMBER]}
            \end{quote}
            For example, the third exercise in Volume I, chapter 2 is labelled \textbf{I.2.3}. Likewise, the fourth exercise in Volume II, chapter 3 is labelled \textbf{II.3.4}.
        \end{itemize}
        \item Volume numbers are always written in Roman numerals, except for Volume 0 which will be written as a zero.
        \item The symbol ``$\qedsymbol$'' marks the end of a proof.
    \end{itemize}

    \section*{Chapter Interdependence}
    The diagram on the next page shows chapter interdependence. It should be used in conjunction with the table of contents and notes listed.

    \newpage
    \includegraphics[width=\linewidth]{Interdependence.png}
    
    \newpage

    \textbf{Notes}:
    \interdependencenotes

    \mainmatter  % Now use arabic numerals for page numbers
}

%============= Index Pages ===============
\usepackage[
    totoc,
    columnsep=20pt,
    hangindent=8pt,
    subindent=20pt,
    subsubindent=30pt
]{idxlayout}

\makeindex[options= -s ../index-style.ist]

%======= Bibliography Formatting =========
% These two lines are here to ensure that URLs do not exceed the page by too much
\setcounter{biburllcpenalty}{7000}
\setcounter{biburlucpenalty}{8000}

\usepackage{xr}

%=========== Global Variables ============
\newcommand{\version}{0.16}
\newcommand{\volumenumber}{1}
\newcommand{\volumename}{Groups}
\newcommand{\volumeimage}{cover/Icosahedron.png}

%============= Formatting ================
\linespread{1.05}

%============== Resources ================
\externaldocument{../Volume 0/Volume0}

%=========== Custom Commands =============
\newcommand{\An}[1]{\mathrm{A}_{#1}}                  % Alternating group of degree n
\newcommand{\Aut}[1]{\mathrm{Aut}(#1)}                % Group of automorphisms of G
\newcommand{\C}[2]{\mathrm{C}_{#1}(#2)}               % Centralizer of an element in G
\newcommand{\Cl}[1]{\mathrm{Cl}(#1)}                  % Conjugacy class of the element x
\newcommand{\Cn}[1]{\mathrm{C}_{#1}}                  % Cyclic group of order n
\newcommand{\GL}[2]{\mathrm{GL}_{#1}\left(#2\right)}  % General Linear Group of degree n
\newcommand{\Inn}[1]{\mathrm{Inn}(#1)}                % Group of inner automorphisms of G
\newcommand{\N}[2]{\mathrm{N}_{#1}(#2)}               % Normalizer of S in G
\newcommand{\Out}[1]{\mathrm{Out}(#1)}                % Group of outer automorphisms of G
\newcommand{\SL}[2]{\mathrm{SL}_{#1}\left(#2\right)}  % Special Linear Group of degree n
\newcommand{\Sn}[1]{\mathrm{S}_{#1}}                  % Symmetric group of degree n
\newcommand{\Syl}[2]{\mathrm{Syl}_{#1}(#2)}           % Set of Sylow p-groups of G
\newcommand{\Sym}[1]{\mathrm{Sym}(#1)}                % Symmetric group of a set
\newcommand{\Un}[1]{\mathcal{U}_{#1}}                 % Group of units modulo n
\newcommand{\Z}[1]{\mathrm{Z}(#1)}                    % Center of a group G

\newcommand{\Stab}[2]{\mathrm{Stab}_{#1}(#2)}         % Stabilizer of x by G
\newcommand{\Fix}[2]{\mathrm{Fix}_{#1}(#2)}           % Set of all elements in X which is fixed by g
\newcommand{\Orb}[2]{\mathrm{Orb}_{#1}(#2)}           % Orbit of x under G

%========= Front Matter Pages ============
% Quote page
\newcommand{\quotepagetext}{
    [The] axioms for a group are short and natural... [yet] somehow hidden behind these axioms is the monster simple group, a huge and extraordinary mathematical object, which appears to rely on numerous bizarre coincidences to exist. The axioms for groups give no obvious hint that anything like this exists.
}
\newcommand{\quotepageattribution}{Richard Borcherds, 2009}
\newcommand{\quotepagecitation}{\cite{cook_borcherds_2009}}

% Preface
\newcommand{\prefacevolumetext}{
    Groups are one of the most fundamental structures in abstract algebra. They underpin the ideas of symmetry and allow us to explore the relationships between symmetrical objects. An analysis of all the different ways an object can be symmetric is also possible with groups. It would be an understatement to say that groups are important in abstract algebra; without them, there can be no further and more in-depth exploration of the other structures.

    Volume I is a simple introduction to the world of groups. As with most books on this topic, we concentrate on abstract groups, and, in particular, on finite groups. We also discuss and explore some crucial results about the structure of groups. The content covered in this volume should be ample for one to understand the fundamentals of group theory, and appreciate the wonders of groups and symmetry.
}
\newcommand{\prefacevolumedate}{22 March, 2023}

% Suggestions of use
\newcommand{\interdependencenotes}{
    \begin{itemize}
        \item This volume assumes understanding of the prerequisites in volume 0.
        \item Chapter 1 is essentially independent from the rest of the other chapters. It provides motivation for the axioms of groups, but readers who want to skip this introduction can move straight to chapter 2.
        \item Chapters 2, 3, and 4 are considered to be the essentials of group theory.
        \item Chapter 4 is required reading for chapters 5, 6, 7, and 9.
        \item Chapter 7 requires a small result from chapter 6 (specifically, \myref{prop-subgroup-product-is-subgroup}); otherwise these two chapters are relatively independent.
        \item Chapter 8, on more types of groups, assumes knowledge of chapter 7. Chapter 8 also assumes knowledge of permutations and the symmetric group from chapter 5.
        \item Chapter 9 could be read after chapter 4.
        \item Chapter 10 only requires chapters 7 and 9.
        \item Chapter 11 only require results from chapter 7, except for one problem (\myref{problem-S4-composition-series}) which uses the alternating group introduced in chapter 8.
        \item Chapter 12 assumes full knowledge of chapter 7; minor results from chapter 8 (specifically, the alternating group), chapter 10 (the Third Sylow Theorem), and chapter 11 (\myref{problem-S4-composition-series}) are required.
    \end{itemize}
}

%==== Include only relevant chapters =====
\IfFileExists{\jobname.run.xml}
{
    \includeonly{
        % Main chapters
        introduction-to-groups,
        basics-of-groups,
        subgroups,
        homomorphisms-and-isomorphisms,
        symmetry-groups,
        products,
        further-homomorphisms,
        more-groups,
        group-actions,
        sylow-theorems,
        composition-series,
        simple-groups,
        % Appendix
        exercise-solutions,
        problem-solutions,
        bib-and-index
    }
}
{
    % Do a full document initially to generate all the aux files
}

%=========================================
\begin{document}
\frontmatterpages

%=========================================
\include{introduction-to-groups}
\section{Basics of Groups}
\subsection*{Exercises}
\begin{questions}
    \item The Cayley table of $(\Z_6, \otimes_6)$ is as follows:
    \begin{table}[H]
        \centering
        \begin{tabular}{|l|l|l|l|l|l|l|}
        \hline
        \textbf{$\otimes_n$} & \textbf{0} & \textbf{1} & \textbf{2} & \textbf{3} & \textbf{4} & \textbf{5} \\ \hline
        \textbf{0}       & 0          & 0          & 0          & 0          & 0          & 0          \\ \hline
        \textbf{1}       & 0          & 1          & 2          & 3          & 4          & 5          \\ \hline
        \textbf{2}       & 0          & 2          & 4          & 0          & 2          & 4          \\ \hline
        \textbf{3}       & 0          & 3          & 0          & 3          & 0          & 3          \\ \hline
        \textbf{4}       & 0          & 4          & 2          & 0          & 4          & 2          \\ \hline
        \textbf{5}       & 0          & 5          & 4          & 3          & 2          & 1          \\ \hline
        \end{tabular}
    \end{table}

    Since the identity is $1$, and the row (and column) of 0 does not have a $1$, thus $0$ does not have an inverse. Therefore $(\Z_6, \oplus_6)$ is not a group.

    \item Note that $(xx^{-1})^{-1} = (x^{-1})^{-1}x^{-1}$ by Shoes and Socks and $(xx^{-1})^{-1} = e^{-1} = e$. Thus $(x^{-1})^{-1}x^{-1} = e$. Multiplying both sides on the right by $x$ yields $(x^{-1})^{-1} = ex = x$, i.e. $(x^{-1})^{-1} = x$.

    \item We consider a proof by induction via inducting on $n$.

    The base case of $n = 0$ clearly holds true since
    \begin{align*}
        (x^{-1})^0 &= e & (\text{definition of }g^0 \text{ for any }g\in G)\\
        &= e^{-1} & (\myref{prop-inverse-of-identity-is-identity})\\
        &= (x^0)^{-1}. & (\text{definition of }x^0)
    \end{align*}

    Now assume that the statement holds for a non-negative integer $k$, i.e. $(x^{-1})^k = (x^k)^{-1}$. We are to show that the statement holds for $k+1$, i.e. $(x^{-1})^{k+1} = (x^{k+1})^{-1}$.

    Observe that
    \begin{align*}
        (x^{-1})^{k+1} &= (x^{-1})^k \ast x^{-1} & (\text{by statement 1})\\
        &= (x^k)^{-1} \ast x^{-1} & (\text{by hypothesis})\\
        &= (x\ast x^k)^{-1} & (\text{by Shoes and Socks})\\
        &= (x^{k+1})^{-1} & (\text{by statement 1})
    \end{align*}
    so the statement is true for $k+1$.

    Thus, by induction, we have $(x^{-1})^n = (x^n)^{-1}$ for any non-negative integer $n$.

    \item \begin{partquestions}{\roman*}
        \item The identity is $1$ since:
        \begin{itemize}
            \item $1 \times 1 = 1$;
            \item $1 \times (-1) = (-1) \times 1 = -1$;
            \item $1 \times i = i \times 1 = i$; and
            \item $1 \times (-i) = (-i) \times 1 = -i$.
        \end{itemize}
        \item The order of the identity $1$ is 1, so we look at the other elements:
        \begin{itemize}
            \item $|-1| = 2$ since $-1 \neq 1$ and $(-1)^2 = -1 \times -1 = 1$.
            \item $|i| = 4$ since $i \neq 1$, $i^2 = -1 \neq 1$, $i^3 = -i \neq 1$, but $i^4 = 1$.
            \item $|-i| = 4$ since $-i \neq 1$, $(-i)^2 = -1 \neq 1$, $(-i)^3 = i \neq 1$, but $(-i)^4 = 1$.
        \end{itemize}
    \end{partquestions}

    \item $-i$ is the other generator since $(-i)^1 = -i$, $(-i)^2 = -1$, $(-i)^3 = i$, and $(-i)^4 = 1$.

    \item We work slowly:
    \begin{align*}
        rsr^4sr^3 &= r(sr^4)(sr^3)\\
        &= r(r^2s)(r^3s)\\
        &= r^3sr^3s\\
        &= r^3(sr^3)s\\
        &= r^3(r^3s)s\\
        &= r^6s^2\\
        &= e
    \end{align*}
\end{questions}

\subsection*{Problems}
\begin{questions}
    \item The group table of $D_4$ is given as follows.
    \begin{table}[H]
        \centering
        \begin{tabular}{|l|l|l|l|l|l|l|l|l|}
        \hline
        $\ast$ & $e$    & $r$    & $r^2$  & $r^3$  & $s$    & $rs$   & $r^2s$ & $r^3s$ \\ \hline
        $e$    & $e$    & $r$    & $r^2$  & $r^3$  & $s$    & $rs$   & $r^2s$ & $r^3s$ \\ \hline
        $r$    & $r$    & $r^2$  & $r^3$  & $e$    & $rs$   & $r^2s$ & $r^3s$ & $s$    \\ \hline
        $r^2$  & $r^2$  & $r^3$  & $e$    & $r$    & $r^2s$ & $r^3s$ & $s$    & $rs$   \\ \hline
        $r^3$  & $r^3$  & $e$    & $r$    & $r^2$  & $r^3s$ & $s$    & $rs$   & $r^2s$ \\ \hline
        $s$    & $s$    & $r^3s$ & $r^2s$ & $rs$   & $e$    & $r^3$  & $r^2$  & $r$    \\ \hline
        $rs$   & $rs$   & $s$    & $r^3s$ & $r^2s$ & $r$    & $e$    & $r^3$  & $r^2$  \\ \hline
        $r^2s$ & $r^2s$ & $rs$   & $s$    & $r^3s$ & $r^2$  & $r$    & $e$    & $r^3$  \\ \hline
        $r^3s$ & $r^3s$ & $r^2s$ & $rs$   & $s$    & $r^3$  & $r^2$  & $r$    & $e$    \\ \hline
        \end{tabular}
    \end{table}
    \begin{partquestions}{\alph*}
        \item $D_4$ is not abelian because $rs \neq sr = r^3s$.
        \item We simplify $r^3srsr^3sr^3sr^2$.
        \begin{align*}
            r^3 sr sr^3 sr^3 sr^2 &= r^3srs(r^3s)(r^3s)r^2\\
            &= r^3 srs(e)r^2\\
            &= r^3 sr sr^2\\
            &= r^2(rs rs)r^2\\
            &= r^2(e)r^2\\
            &= r^4\\
            &= e
        \end{align*}
    \end{partquestions}

    \item We need to prove each of the group axioms in order to prove that $(\Q, +)$ is indeed a group.
    \begin{itemize}
        \item \textbf{Closure}: Let $\frac ab$ and $\frac cd$ be rational numbers where $b, d \neq 0$. Their sum is $\frac{ad+bc}{bd}$, which is also rational. Therefore $\Q$ is closed under addition.

        \item \textbf{Associativity}: Addition is associative by \myref{axiom-addition-is-associative}.

        \item \textbf{Identity}: 0 is the identity since
        \[
            0 + \frac ab = \frac ab + 0 = \frac ab
        \]
        for any rational number $\frac ab$ (with $b \neq 0$).

        \item \textbf{Inverse}: For any rational number $\frac ab$, its inverse is $-\frac ab$ since
        \[
            \frac ab + \left(-\frac ab\right) = \left(-\frac ab\right) + \frac ab = 0
        \]
        for any rational number $\frac ab$ (with $b \neq 0$).
    \end{itemize}
    Furthermore addition is assumed to be commutative by \myref{axiom-addition-is-commutative}. Therefore $(\Q, +)$ is an abelian group.

    \item If every element in $G$ is its own inverse, then for every element $g$ in $G$, $g^{-1} = g$. Consider $(gh)^{-1}$ where $g$ and $h$ are elements in $g$. On one hand, by Shoes and Socks, $(gh)^{-1} = h^{-1}g^{-1} = hg$ since each element is its own inverse. On the other hand, since $gh$ is an element in $G$, thus $(gh)^{-1} = gh$. Thus $gh = hg$ which means $G$ is abelian.

    \item Recall that $n = |x|$ is the smallest positive integer that satisfies $x^n = e$.

    We prove the forward direction first. Suppose $m$ is a multiple of $n$, say $m = qn$ for some integer $q$. Then
    \[
        x^m = x^{qn} = \left(x^n\right)^q = e^q = e
    \]
    which means $x^m = e$.

    We now prove the reverse direction. Suppose $x^m = e$. Using Euclid's division lemma (\myref{lemma-euclid-division}), we write $m = qn + r$ where $q$ and $r$ are integers with $0 \leq r < n$. Hence
    \[
        x^m = x^{qn + r} = x^{qn}x^r = \left(x^n\right)^qx^r = e^qx^r = x^r.
    \]
    Note that for all integers $k$ where $1 \leq k < n$, we have $x^k \neq e$ since $n$ is the smallest positive integer such that $x^n = e$. Hence, if $x^r = e$, we conclude $r = 0$. Therefore $m = qn$, meaning $m$ is a multiple of $n$.

    \item \begin{partquestions}{\alph*}
        \item Note that $(gh)^2 = ghgh$. Given that $(gh)^2 = g^2h^2 = gghh$. By cancellation law, $hg = gh$ which means $G$ is abelian.
        \item Suppose $G$ is abelian. Clearly $(gh)^1 = gh$. Suppose $(gh)^{k} = g^kh^k$ for some positive integer $k$. Then
        \begin{align*}
            (gh)^{k+1} &= (gh)(gh)^k\\
            &= (gh)(g^kh^k) & (\text{by assumption})\\
            &= ghg^kh^k\\
            &= g(hg^k)h^k\\
            &= g(g^kh)h^k & (\text{since } G \text{ is abelian})\\
            &= gg^khh^k\\
            &= g^{k+1}h^{k+1}
        \end{align*}
        so $(gh)^{k+1} = g^{k+1}h^{k+1}$ assuming $(gh)^k = g^kh^k$. Thus the claim is proven by mathematical induction.
    \end{partquestions}

    \item Note that $|1| = n$ since $1^2 = 1 \oplus_n 1 = 2$, $1^3 = 1 \oplus_n 1 \oplus_n 1 = 3$, $1^4 = 4$, ..., $1^{n-1} = n-1$ and $1^n = 0$ which is the identity. Since the group $(\Z_n, \oplus_n)$ has an element with the same order as the group, it is thus cyclic with order $n$ and generator 1.

    \item We show that $(A, \circ)$ is a group.
    \begin{itemize}
            \item \textbf{Closure}: Function composition is closed by definition.
            \item \textbf{Associativity}: Function composition is associative.
            \item \textbf{Identity}: By performing brute-force computation, we find that $T^6(x, y) = (x, y)$. Hence $T^6$ is the identity of $A$.
            \item \textbf{Inverse}: If $r = 6$ then $T^r$ is its own inverse. Otherwise, $T^{6-r}$ is the inverse of $T^r$.
    \end{itemize}
    Thus, $(A, \circ)$ is a group, with order 6.
\end{questions}

\section{Subgroups}
\begin{questions}
    \item We note that $G$ contains $\{e, r, r^2, r^3, s, rs, r^2s, r^3s\}$.
    \begin{partquestions}{\alph*}
        \item Yes, this is the trivial subgroup.
        \item No, it is not closed. ($rs$ can be generated by $r \ast s$ but is not in the set)
        \item No, the identity $e$ is missing.
        \item Yes, $\{r, r^3, r^4, r^6\} = \{r, r^3, e, r^2\} = \langle r \rangle$.
    \end{partquestions}

    \item \begin{partquestions}{\alph*}
        \item Clearly $e \in K$ since $e^2 = e \in H$.
        
        Let $x, y \in K$, so $x^2 \in H$ and $y^2 \in H$. We note $y^{-1} \in K$ since $(y^{-1})^2 = (y^2)^{-1} \in H$. Therefore $(xy^{-1})^2 = xy^{-1}xy^{-1} \in H$, so $xy^{-1} \in K$. Hence $K \leq G$ by subgroup test.

        \item Note that the identity of $K$, $e$, is also the identity of $G$. Since $H \leq G$, thus $e \in H$. Since $H$ is a subgroup of $G$, thus for any $x$ and $y$ in $H$ we have $xy^{-1} \in H$. Hence, by subgroup test, $H \leq K$.
    \end{partquestions}

    \item \begin{partquestions}{\alph*}
        \item We have proved that $\Z{G} \leq G$ so we only prove normality. Let $g$ and $z$ be arbitrary elements from $G$ and $\mathrm{Z}(G)$ respectively. Then
        \begin{align*}
            gzg^{-1} &= g(zg^{-1})\\
            &= g(g^{-1}z) & (\text{since }z \in \mathrm{Z}(G))\\
            &= (gg^{-1})z \\
            &= z\\
            &\in \mathrm{Z}(G)
        \end{align*}
        which proves that $\mathrm{Z}(G) \unlhd G$.

        \item We first work in the forward direction by assuming $G = \Z{G}$. Then for all $z \in \Z{G} = G$ we have $gz = zg$ for any $g \in G$ by definition, which means that $G$ is abelian.

        We now work in the reverse direction by assuming that $G$ is abelian. Note $\mathrm{Z}(G) = \{z \in G \vert gz = zg \text{ for all } g \in G\}$. But since $G$ is abelian, $gh = hg$ for all $g$ and $h$ in $G$. Thus every element in $G$ satisfies the condition to be in the center of $G$, meaning $\mathrm{Z}(G) = G$.

        \item We note that $D_4 = \{e, r, r^2, r^3, s, rs, r^2s, r^3s\}$. Since $\mathrm{Z}(D_4)$ is a subgroup of $D_4$ it has a maximum order of 2, by Lagrange's theorem (\myref{thrm-lagrange}). Since 2 is prime the subgroups must be cyclic. Thus the non-trivial subgroups of $D_4$ are $\{e, r^2\}$ and $\{e, s\}$ (since $|r^2| = |s| = 2$). Now like how we proved that $\langle s \rangle = \{e, s\}$ is not a normal subgroup in $D_3$ in \myref{example-normal-subgroups-of-d3}, $\{e, s\}$ is not a normal subgroup of $D_4$. One verifies easily that $\{e, r^2\} = \langle r^2 \rangle$ is a normal subgroup of $D_4$. Thus $\mathrm{Z}(D_4) = \langle r^2 \rangle$ since $\mathrm{Z}(D_4)$ must be a normal subgroup of $D_4$ with order not exceeding 2.
    \end{partquestions}

    \item \begin{partquestions}{\alph*}
        \item We will prove this statement.

        Clearly $e \in H \cap K$ since $e \in H$ and $e \in K$ as both are subgroups of $G$.

        Let $x$ and $y$ be in $H \cap K$, meaning that $x, y \in H$ and $x, y \in K$. Thus $xy^{-1} \in H$ and $xy^{-1} \in K$ as both are subgroups of $G$. Hence $xy^{-1} \in H \cap K$.
        By subgroup test, $H \cap K \leq G$.

        \item We will prove this statement. One sees that $H \cap K \subseteq H$. Since $H \cap K \leq G$, it is thus a group. Hence $H \cap K \leq H$ by definition of a subgroup.

        \item We will disprove this statement. Consider:
        \begin{align*}
            &G = \mathbb{Z}_6 \text{ under }\oplus_6,\\
            &H = \{0, 2, 4\},\text{ and}\\
            &K = \{0, 3\}.
        \end{align*}
        Clearly $H \leq G$ and $K \leq G$. Note $H \cup K = \{0, 2, 3, 4\}$. But $H \cup K$ is not closed since $2 \oplus_6 3 = 5 \not \in H \cup K$. Hence $H \cup K \not\leq G$.

        \item We will disprove this statement. Since $H \cup K$ is not closed it is not a group, meaning it cannot be a subgroup.
    \end{partquestions}

    \item By Lagrange's Theorem (\myref{thrm-lagrange}), the order of a subgroup must divide the order of the group. Since $H \leq G$ is non-trivial, and since $1024 = 2^{10}$, the largest order that $H$ can be is $512$ with $[G:H] = 2$. An example is $G = \mathbb{Z}_{1024}$ and $H = \langle 2 \rangle$, since $|H| = |2| = 512$ as $2 \times 512 = 1024 \equiv 0 \pmod{1024}$.

    \item Let $|G| = 2n$. The identity is its own inverse, leaving an odd number of non-identity elements.
    
    Suppose $x$ is an element of $G$ with $|x| > 2$; we cannot have $x^{-1} = x$ (otherwise $x^2 = e$). Thus $x^{-1}$ and $x$ are distinct. Pair every one of these $x$'s with its inverse $x^{-1}$.

    Remember that there is an odd number of non-identity elements. Hence, there must be at least one element which has not been paired off with any of the others, which is therefore its own self inverse.

    Since this element is not the identity, thus it has to have order 2 (as $g^{-1} = g$ implies $g^2 = e$).

    \item Suppose $G = \langle g \rangle$ and $H \leq G$. Then any element in $H$ is of the form $g^a$ where $a$ is an integer. Suppose $m$ is the smallest positive integer $m$ such that $g^m \in H$. Suppose now $g^n \in H$ for some $n$. By Euclid's division lemma (\myref{lemma-euclid-division}), $n = mq + r$ where $q$ and $r$ are non-negative integers such that $0 \leq r < m$. Hence,
    \[
        g^n = g^{mq}g^r = (g^m)^q g^r.
    \]
    Now, $m$ is the smallest positive integer such that $g^m \in H$. This means that if $r \neq 0$, $g^r \not\in H$ as $0 \leq r < m$. Hence, $r = 0$, which means
    \[
        g^n = (g^m)^q.
    \]
    Thus, every element in the subgroup $H$ can be formed by applying $g^m$ a certain number of times, meaning $H$ is cyclic with generator $g^m$.

    \item \begin{partquestions}{\roman*}
        \item Let $xH$ be a coset in $G$. Since cosets partition $G$, either $xH = H$ or $xH = G \setminus H$ (since there are only two distinct cosets).
    \begin{itemize}
            \item If $xH = H$, then $x \in H$, meaning $xH = H = Hx$.
            \item If $xH \neq H$, then $x \in G \setminus H$. Hence $xH = G \setminus H = Hx$.
    \end{itemize}
    Therefore $H$ is a normal subgroup of $G$, i.e. $H \lhd G$.
        \item Note that $G/H$ is a group since $H \lhd G$. Also since $[G:H] = 2$ thus $|G/H| = 2$.
        
        If $x \in G$ then $x^2H = (xH)^2 = H$ since the order of $G/H$ is 2, meaning that any non-identity element inside it (like $xH$) has an order of at most 2. Since $x^2H = H$, therefore by Coset Equality (\myref{lemma-coset-equality}), statements 1 and 4, we must have $x^2 \in H$.
        
        \item Suppose $x$ has odd order. Write $|x| = 2k - 1$ where $k$ is a positive integer. Hence $x^{2k-1} = e \in H$ since $H < G$. Therefore
        \[
            x = x^{2k} = \left(x^k\right)^2 \in H        
        \]
        which means $x \in H$.
    \end{partquestions}

    \item \begin{partquestions}{\alph*}
        \item Suppose we have an element $x \in H \cap K$, meaning that $x \in H$ and $x \in K$. By a corollary of Lagrange's Theorem (\myref{corollary-order-of-group-multiple-of-order-of-element}), the order of $x$ must divide the order of its group. Hence, $|x|$ divides $|H|$ and $|x|$ divides $|K|$ simultaneously, meaning that $|x| = \gcd(|H|, |K|)$. But the GCD of the orders of both subgroups is 1. Hence, $|x| = 1$, meaning the only element in the intersection $H \cap K$ is the identity $e$.
        
        \item Consider $hkh^{-1}k^{-1}$.
        \begin{itemize}
            \item On one hand, note that $hkh^{-1}k^{-1} = h(kh^{-1}k^{-1})$. Clearly $h \in H$ and $kh^{-1}k^{-1} \in H$ by normality of $H$. Therefore $hkh^{-1}k^{-1} \in H$.
            \item On another hand, $hkh^{-1}k^{-1} = (hkh^{-1})k^{-1}$. Note $hkh^{-1} \in K$ by normality of $K$ and $k^{-1} \in K$, so $hkh^{-1}k^{-1} \in K$.
        \end{itemize}
        Therefore $hkh^{-1}k^{-1} \in H \cap K$. But by \textbf{(a)}, the only element in $H \cap K$ is the identity. Thus, $hkh^{-1}k^{-1} = e$ which the result follows quickly.
    \end{partquestions}
    
    \item \begin{partquestions}{\alph*}
        \item $m = 6$.
        \item We first prove that all groups of order less than 6 are abelian, and then find a non-abelian group of order 6.

        We note that a group of order 1 is the trivial group which is abelian. The groups of order 2, 3, and 5 are groups of prime order, meaning that they are cyclic and hence abelian. We are left with a group of order 4.

        We note that the order of an element of a group of order 4 must divide 4 (\myref{corollary-order-of-group-multiple-of-order-of-element}). Hence the possible orders of an element in such a group is 1, 2, or 4. An element of order 1 is the identity. If an element with order 4 exists, then the group is cyclic and hence abelian. So we assume that all elements are either order 1 or order 2 (in fact, the orders must be 1, 2, 2, 2). This is precisely the group
        \[
            D_2 = \langle r, s \vert r^2 = s^2 = e, rs = sr\rangle
        \]
        which is abelian. Hence all groups of order 4 are abelian.

        We now show that a group of order 6 can be non-abelian. We note that the group
        \[
            D_3 =  \langle r, s \vert r^3 = s^2 = e, rs = sr^2\rangle
        \]
        has order 6 and because $rs = sr^2 \neq sr$, thus $D_3$ is non-abelian. Hence $m = 6$.

        \item For all even $n \geq 6$, the group $D_{\frac n2}$ has $n$ elements and $rs = sr^{\frac n2 - 1} \neq sr$, so $D_{\frac n2}$ is non-abelian.
    \end{partquestions}

    \item Suppose $G / \Z{G}$ is cyclic. Then by definition, $G / \Z{G} = \langle g\Z{G}\rangle$ for some $g \in G$, and any element in $G/\Z{G}$ is of the form $g^n\Z{G}$.

    Now take $x, y \in G$. By \myref{lemma-left-coset-partition}, left cosets partition the group, so we may assume $x \in g^m\Z{G}$ and $y \in g^n\Z{G}$, meaning $x = g^mz_1$ and $y = g^nz_2$ for some $z_1, z_2 \in \Z{G}$. We note
    \begin{align*}
        xy &= (g^mz_1)(g^nz_2)\\
        &= g^m(z_1g^n)z_2\\
        &= g^m(g^nz_1)z_2 & (\text{since }z_1 \in \Z{G})\\
        &= (g^mg^n)(z_1z_2)\\
        &= g^{m+n}z_1z_2\\
        &= g^{n+m}z_2z_1\\
        &= g^ng^mz_2z_1\\
        &= g^n(g^mz_2)z_1\\
        &= g^n(z_2g^m)z_1\\
        &= (g^nz_2)(g^mz_1)\\
        &= yx
    \end{align*}
    which means that $xy = yx$ for any $x, y \in G$. Hence $G$ is abelian.
\end{questions}

\section{Homomorphisms and Isomorphisms}
\begin{questions}
    \item We will prove that $f$ is a homomorphism, is injective, and is surjective.
    \begin{itemize}
        \item \textbf{Homomorphism}: Let $x, y \in G$. Then
        \begin{align*}
            f(xy) &= g(xy)g^{-1}\\
            &= (gxg^{-1})(gyg^{-1})\\
            &= f(x)f(y)
        \end{align*}
        which means that $f$ is a homomorphism.
        \item \textbf{Injective}: Let $x, y \in G$ be such that $f(x) = f(y)$. Then $gxg^{-1} = gyg^{-1}$. By cancellation law, $x = y$.
        \item \textbf{Surjective}: Suppose $y \in G$. Set $x = g^{-1}yg$. Since $G$ is closed, thus $x \in G$. Note $f(x) = g(g^{-1}yg)g^{-1} = y$. Hence $y$ has a pre-image of $x = g^{-1}yg$ in $G$.
    \end{itemize}
    Therefore $f$ is an isomorphism.

    \item Suppose on the contrary there exists an isomorphism $\phi: G \to H$. Since $\phi$ is an isomorphism, it is surjective. Hence, there must exists a rational number $r \in G$ such that $\phi(r) = 2$. As $r$ is rational, so is $\frac r2$.

    Now consider $\phi\left(\frac r2 + \frac r2\right)$. On one hand, $\phi\left(\frac r2 + \frac r2\right) = \phi(r) = 2$. On another hand, $\phi(\frac r2 + \frac r2) = \left(\phi\left(\frac r2\right)\right)^2$ as $\phi$ is a homomorphism. Therefore, $\left(\phi\left(\frac r2\right)\right)^2 = 2$ which quickly implies $\phi\left(\frac r2\right) = \sqrt 2$ since $\phi\left(\frac r2\right)$ must be positive. However, $\sqrt 2 \notin H$ while $\phi\left(\frac r2\right) \in H$, a contradiction.

    Hence, $G \not\cong H$.

    \item \begin{partquestions}{\alph*}
        \item Let $m, n \in G$. Then
        \[
            \phi(m + n) = 2(m + n) = 2m + 2n = \phi(m) + \phi(n)
        \]
        which means $\phi$ is a homomorphism.

        \item Suppose $m, n \in G$ such that $\phi(m) = \phi(n)$. Then $2m = 2n$. Clearly this means that $m = n$. Thus $\phi$ is injective.

        \item Suppose on the contrary there existed a homomorphism $\psi: H \to G$ such that $\psi(\phi(n)) = n$. Then $\psi(2n) = n$ by definition of $\phi$. Note that
        \[
            \psi(2n) = \psi(n + n) = \psi(n) + \psi(n) = 2\psi(n)
        \]
        since $\psi$ is a homomorphism. Hence $2\psi(n) = n$ which implies that $\psi(n) = \frac n2$. But for the case of $n = 1$, $\psi(1) = \frac 12 \notin G$. Hence $\psi$ does not exist.
    \end{partquestions}

    \item We prove the forward direction first: assume that $G$ is abelian. Then $f$ is a homomorphism since
    \[
        f(gh) = (gh)^{-1} = h^{-1}g^{-1} = g^{-1}h^{-1} = f(g)h(g).
    \]

    We now prove the reverse direction: assume that $f$ is a homomorphism, meaning $f(gh) = f(g)f(h) = g^{-1}h^{-1}$. But $f(gh) = (gh)^{-1} = h^{-1}g^{-1}$. Therefore we have $g^{-1}h^{-1} = h^{-1}g^{-1}$ which clearly shows that the group is abelian.

    \item Suppose $\phi: G \to H$ is a surjective homomorphism and $G$ is abelian. Since $\phi$ is surjective, thus $\im \phi = H$. Let $g_1, g_2 \in G$ and $h_1, h_2 \in H$ such that $\phi(g_1) = h_1$ and $\phi(g_2) = h_2$. Consider $\phi(g_1g_2)$.
    \begin{itemize}
        \item On one hand, $\phi(g_1g_2) = \phi(g_1)\phi(g_2) = h_1h_2$.
        \item On another hand, $\phi(g_1g_2) = \phi(g_2g_1) = \phi(g_2)\phi(g_1) = h_2h_1$.
    \end{itemize}
    Hence $h_1h_2 = h_2h_1$ which means that $H$ is abelian.

    \item We first prove $\phi(N)$ is a subgroup of $H$ by using subgroup test before proving normality.

    Note that $e_H \in \phi(N)$ since $e_G \in N$ and $\phi(e_G) = e_H$. Now let $x, y \in \phi(N)$. As $\phi$ is surjective, we know that there exists $n_x, n_y \in N$ where $\phi(n_x) = x$ and $\phi(n_y) = y$. Note that $\phi(n_y^{-1}) = y^{-1}$ and $n_xn_y^{-1} \in N$. Hence, $xy^{-1} = \phi(n_xn_y^{-1}) \in \phi(N)$. By subgroup test, $\phi(N) \leq H$.

    We now show that $\phi(N)$ is a normal subgroup of $H$. Take $g \in G$, $h \in H$, $n \in N$, and $x \in \phi(N)$, such that $\phi(g) = h$ and $\phi(n) = x$. Note that since $N \unlhd G$, thus $gng^{-1} \in N$. Therefore,
    \begin{align*}
        hxh^{-1} &= \phi(g)\phi(n)\phi(g^{-1})\\
        &= \phi(\underbrace{gng^{-1}}_{\text{In }N})\\
        &\in \phi(N)
    \end{align*}
    which means that $\phi(N) \unlhd H$.

    \item Consider the map $\phi: G \to H, a \mapsto a + n\mathbb{Z}$. We show that $\phi$ is an isomorphism:
    \begin{itemize}
        \item \textbf{Homomorphism}: Let $a$ and $b$ be in $G$. Then
        \begin{align*}
            \phi(a\oplus_n b) &= (a\oplus_n b) + n\mathbb{Z}\\
            &= \{(a \oplus_n b) + pn \vert p \in \mathbb{Z}\}\\
            &= \{a+b + pn \vert p \in \mathbb{Z}\}\\
            &= \{a+b + pn + qn\vert p, q \in \mathbb{Z}\}\\
            &= a+b+n\mathbb{Z} + n\mathbb{Z}\\
            &= (a+n\mathbb{Z}) + (b + n\mathbb{Z})\\
            &= \phi(a) + \phi(b).
        \end{align*}
        \item \textbf{Injective}: Let $a$ and $b$ be in $G$ such that $\phi(a) = \phi(b)$. Thus
        \[
            \{a + pn \vert p \in \mathbb{Z} \} = \ \{b + qn \vert q \in \mathbb{Z} \}
        \]
        by definition of $\phi$. Hence $a \equiv b \pmod n$. But since $0 \leq a, b < n$, we must have $a = b$.
        \item \textbf{Surjective}: Let $x + n\mathbb{Z} \in H$. We use Euclid's division lemma (\myref{lemma-euclid-division}) on $x$ to yield
        \[
            x = qn + r, \text{ where } 0 \leq r < n.
        \]
        Note that
        \begin{align*}
            x + n\mathbb{Z} &= \{x + kn \vert k \in \mathbb{Z}\}\\
            &= \{(qn + r) + kn \vert k \in \mathbb{Z}\}\\
            &= \{r + n(\underbrace{q + k}_{\text{In }\mathbb{Z}}) \vertalt k \in \mathbb{Z} \}\\
            &= r + n\mathbb{Z}
        \end{align*}
        with $0 \leq r < n$, meaning $r \in G$. Now observe $\phi(r) = r+n\mathbb{Z} = x+n\mathbb{Z}$ which means that there is a pre-image for every element in $H$, hence proving that $\phi$ is surjective.
    \end{itemize}
    Therefore $\phi$ is an isomorphism, proving $G \cong H$.
    
    \item Consider the map $\phi: G \to G/N$ such that $g \mapsto gN$. We note that $\phi$ is a homomorphism as
    \[
        \phi(gh) = (gh)N = (gN)(hN) = \phi(g)\phi(H).
    \]
    We note by \myref{prop-homomorphism-inverse-is-subgroup} that $A = \phi^{-1}(B) \leq G$. Thus
    \begin{align*}
        \phi^{-1}(N) &= \{g \in G \vert \phi(g) = N\}\\
        &= \{g \in G \vert gN = N\}\\
        &= \{g \in G \vert g \in N\}\\
        &= G \cap N\\
        &= N\\
        &\subseteq A
    \end{align*}
    by assumption. Since $N$ is a group, we know $N \leq A$. Furthermore $N \leq A \leq G$ and $N \unlhd G$, meaning $N \unlhd A$ (since $gN = Ng$ for all $g \in G$, including those in $A$). Hence $A/N$ is a group.
    
    Now clearly $\phi$ is surjective (since for any $gN \in G/N$ we know $\phi(g) = gN$), which means that $\phi(\phi^{-1}(B)) = B$. Since $\phi^{-1}(B) = A$, so $\phi(A) = B$. Finally,
    \begin{align*}
        \phi(A) &= \{\phi(a) \vert a \in A\}\\
        &= \{aN \vert a \in A\}\\
        &= A/N
    \end{align*}
    which means $B = A/N$.
\end{questions}

\section{Symmetric Groups}
\begin{questions}
    \item We work from the right to the left.
    \begin{itemize}
        \item $\gamma \delta$ has cycle notation
        \begin{align*}
            &\begin{pmatrix}1 & 2 & 5\end{pmatrix}\begin{pmatrix}3 & 4\end{pmatrix}\begin{pmatrix}1 & 3 & 2 & 5\end{pmatrix}\\
            &= \begin{pmatrix}1 & 4 & 3 & 5 & 2\end{pmatrix};
        \end{align*}
        \item $\beta \gamma \delta$ has cycle notation
        \begin{align*}
            &\begin{pmatrix}1 & 5 & 2\end{pmatrix}\begin{pmatrix}3 & 4\end{pmatrix}\begin{pmatrix}1 & 4 & 3 & 5 & 2\end{pmatrix}\\
            &= \begin{pmatrix}1 & 3 & 2 & 5\end{pmatrix}\\
            &= \delta;
        \end{align*}
        and
        \item $\alpha \beta \gamma \delta$ has cycle notation
        \begin{align*}
            &\begin{pmatrix}1 & 5 & 2 & 3\end{pmatrix}\begin{pmatrix}1 & 3 & 2 & 5\end{pmatrix}\\
            &= \id,
        \end{align*}
        the identity.
    \end{itemize}

    \item Recall that $D_3$ has presentation
    \[
        \langle r, s \vert r^3 = s^2 = e, rs = sr^2 \rangle.
    \]

    Let the map $\phi: D_3 \to \Sn{3}$ be given such that $r \mapsto \begin{pmatrix}1 & 2 & 3\end{pmatrix}$ and $s \mapsto \begin{pmatrix}1 & 2\end{pmatrix}$. We show that $\begin{pmatrix}1 & 2 & 3\end{pmatrix}$ and $\begin{pmatrix}1 & 2\end{pmatrix}$ satisfy the two rules above. For brevity let $\sigma = \begin{pmatrix}1 & 2 & 3\end{pmatrix}$ and $\tau = \begin{pmatrix}1 & 2\end{pmatrix}$.
    \begin{itemize}
        \item We check that $\phi(r^3) = \phi(s^2) = \phi(e)$.
        \begin{itemize}
            \item $\sigma^2 = \begin{pmatrix}1 & 2 & 3\end{pmatrix}\begin{pmatrix}1 & 2 & 3\end{pmatrix} = \begin{pmatrix}1 & 3 & 2\end{pmatrix} \neq \id$;
            \item $\sigma^3 = \begin{pmatrix}1 & 2 & 3\end{pmatrix}\begin{pmatrix}1 & 3 & 2\end{pmatrix} = \id$; and
            \item $\tau^2 = \begin{pmatrix}1 & 2\end{pmatrix}\begin{pmatrix}1 & 2\end{pmatrix} = \id$.
        \end{itemize}
        \item We check that $\phi(rs) = \phi(sr^2)$.
        \begin{itemize}
            \item $rs \mapsto \sigma\tau = \begin{pmatrix}1 & 2 & 3\end{pmatrix}\begin{pmatrix}1 & 2\end{pmatrix} = \begin{pmatrix}1 & 3\end{pmatrix}$; and
            \item $sr^2 \mapsto \tau\sigma^2 = \begin{pmatrix}1 & 2\end{pmatrix}\begin{pmatrix}1 & 3 & 2\end{pmatrix} = \begin{pmatrix}1 & 3\end{pmatrix}$.
        \end{itemize}
    \end{itemize}
    Thus $\phi$ is an isomorphism and so $D_3 \cong \Sn{3}$.

    \item We note that $|\Sn{4}| = 4! = 24$.
    \begin{partquestions}{\alph*}
        \item Consider $H = \left\langle \begin{pmatrix}1 & 2 & 3 & 4\end{pmatrix} \right\rangle$. For brevity, let $\sigma = \begin{pmatrix}1 & 2 & 3 & 4\end{pmatrix}$. Note that
        \begin{itemize}
            \item $\sigma^2 = \begin{pmatrix}1 & 3\end{pmatrix}\begin{pmatrix}2 & 4\end{pmatrix} \neq \id$;
            \item $\sigma^3 = \begin{pmatrix}1 & 4 & 3 & 2\end{pmatrix} \neq \id$; and
            \item $\sigma^4 = \id$.
        \end{itemize}
        Thus, $|\sigma| = 4$ which means $|H| = 4$. Therefore, $G \cong H \leq \Sn{4}$.

        \item Let $\sigma = \begin{pmatrix}1 & 2\end{pmatrix}$ and $\tau = \begin{pmatrix}3 & 4\end{pmatrix}$. Let $H$ have presentation $\langle \sigma, \tau \rangle$. Notice that
        \begin{itemize}
            \item $\sigma^2 = \id$;
            \item $\tau^2 = \id$; and
            \item $(\sigma\tau)^2 = \id$.
        \end{itemize}
        Therefore $H = \{\id, \sigma, \tau, \sigma\tau\}$, so $G \cong H \leq \Sn{4}$.
    \end{partquestions}
\end{questions}

\section{Direct Products of Groups}
\begin{questions}
    \item Let $g_1, g_2 \in G$ and $h_1, h_2 \in H$. Then for $(g_1, h_1), (g_2, h_2) \in G\times H$ we see that
    \begin{align*}
        (g_1, h_1)(g_2, h_2) &= (g_1g_2, h_1h_2)\\
        &= (g_2g_1, h_2h_1)\\
        &= (g_2,h_2)(g_1,h_1)
    \end{align*}
    which means that $G \times H$ is abelian.

    \item Let the map $\phi: G\times H \to H \times G, (g, h) \mapsto (h, g)$. We prove that $\phi$ is an isomorphism:
    \begin{itemize}
        \item \textbf{Homomorphism}: Let $(g_1, h_1), (g_2, h_2) \in G \times H$. We note that
        \begin{align*}
            \phi((g_1, h_1)(g_2, h_2)) &= \phi((g_1g_2, h_1h_2))\\
            &= (h_1h_2, g_1g_2)\\
            &= (h_1, g_1)(h_2, g_2)\\
            &= \phi((g_1, h_1))\phi((g_2, h_2))
        \end{align*}
        which proves that $\phi$ is a homomorphism.
        \item \textbf{Injective}: Suppose there exists $(g_1, h_1), (g_2, h_2) \in G \times H$ such that $\phi((g_1, h_1)) = \phi((g_2, h_2))$. Then by definition of $\phi$ we have $(h_1, g_1) = (h_2, g_2)$. Clearly by comparing component parts of each ordered pair, we have $g_1 = g_2$ and $h_1 = h_2$, meaning $(g_1, h_1) = (g_2, h_2)$. Hence $\phi$ is injective.
        \item \textbf{Surjective}: Let $(h, g) \in H \times G$. Clearly $(g, h) \in G \times H$ and $\phi((g, h)) = (h, g)$, meaning that $(h, g)$ has a pre-image of $(g, h)$. Therefore $\phi$ is surjective.
    \end{itemize}
    Therefore $\phi$ is an isomorphism, meaning $G \times H \cong H \times G$.

    \item We claim that $G$ is the internal direct product of $H$ and $K$. We need to check 3 things.
    \begin{itemize}
        \item $\boxed{G = HK}$ We note that
        \begin{align*}
            HK &= \{h \oplus_6 k \vert h \in H, k \in K\}\\
            &= \{0 \oplus_6 0, 0 \oplus_6 3, 2 \oplus_6 0, 2 \oplus_6 3, 4 \oplus_3 0, 4 \oplus_3 3\}\\
            &= \{0, 3, 2, 5, 4, 1\}\\
            &= \mathbb{Z}_6\\
            &= G
        \end{align*}
        so in fact $G = HK$.

        \item $\boxed{H \cap K = \{e\}}$ Clearly $H \cap K = \{0\}$.

        \item $\boxed{hk = kh}$ Since $\oplus_6$ is commutative, thus $h \oplus_6 k = k \oplus_6$.
    \end{itemize}
    Thus $G$ is the internal direct product of $H$ and $K$.

    \item Define the subgroups $H = \{e, a\}$ and $K = \{e, b\}$. We show the $\mathrm{V}$ is the internal direct product of $H$ and $K$.
    \begin{itemize}
        \item $\boxed{\mathrm{V} = HK}$ Observe that
        \begin{align*}
            HK &= \{hk \vert h \in H, k \in K\}\\
            &= \{ee, eb, ae, ab\}\\
            &= \{e, b, a, ab\}\\
            &= \mathrm{V}
        \end{align*}
        so in fact $\mathrm{V} = HK$.

        \item $\boxed{H \cap K = \{e\}}$ Clearly $H \cap K = \{e\}$.

        \item $\boxed{hk = kh}$ Clearly if one of the elements is the identity then result follows. So assume that $h$ and $k$ are both non-identity elements, so $h = a$ and $k = b$. Note
        \begin{align*}
            kh &= ba\\
            &= (ba)\left((ab)(ab)\right) & (\text{since }(ab)^2 = e)\\
            &= (ba ab)(ab)\\
            &= (bb)(ab) & (\text{since }a^2 = e)\\
            &= ab & (\text{since }b^2 = e)\\
            &= hk
        \end{align*}
        so in fact $hk = kh$ for all $h \in H$, $k \in K$.
    \end{itemize}
    Therefore $\mathrm{V}$ is the internal direct product of $H$ and $K$. 
    
    We note $H = \langle a\rangle \cong \mathbb{Z}_2$ and $K = \langle b \rangle \cong \mathbb{Z}_2$. By direct product equivalence (\myref{thrm-direct-product-equivilance}) we know $\mathrm{V} \cong H \times K \cong \mathbb{Z}_2 \times \mathbb{Z}_2 = (\mathbb{Z}_2)^2$.
\end{questions}

\chapter{Further Properties of Homomorphisms}
Earlier in this book, we introduced homomorphisms and isomorphisms, special types of maps that transform elements of one group to another. We look at more properties of such maps in this chapter and describe the uses of these new properties.

\section{Image of a Homomorphism}
As a homomorphism is a mapping between two groups, it is worthy to look at the \textbf{image} of the homomorphism.
\begin{definition}
    The \textbf{image}\index{homomorphism!image} (or \textbf{range}\index{homomorphism!range}) of a homomorphism $\phi: G \to H$ is the set
    \[
        \im\phi = \{\phi(g) \vert g \in G\}.
    \]
\end{definition}
\begin{remark}
    Some authors (e.g. {\cite[Definition 4.2.0]{libretexts_im-and-ker}}) will use the notation $\phi(G)$ for the image of $\phi$. The alternate notation $\mathrm{Im}\;\phi$ may also be used (e.g. by {\cite[Definition I.2.2]{hungerford_1980}} and \cite[\S 66]{clark_1984}).
\end{remark}

\begin{example}
    Consider the homomorphism $f: \Z \to \Z, x \mapsto 0$. Clearly, all possible values of $x$ maps to 0, so $\im f = \{0\}$.
\end{example}
\begin{example}
    The homomorphism $f: \R \to \R$ where $f(x) = |x|$ has an image of $\{x \in \R \vert x \geq 0\}$, i.e. all non-negative real numbers.
\end{example}

\begin{proposition}\label{prop-image-is-subgroup-of-codomain}
    Let $\phi: G \to H$ be a homomorphism. Then $\im\phi \leq H$.
\end{proposition}
\begin{proof}
    Note that $\phi(e_G) = e_H \in \im\phi$, where $e_G$ and $e_H$ are the identities of $G$ and $H$ respectively.
    
    Now suppose $h_1$ and $h_2$ are in the image of $\phi$, meaning that there exists $g_1$ and $g_2$ such that $\phi(g_1) = h_1$ and $\phi(g_2) = h_2$. Note that $\phi(g_2^{-1}) = h_2^{-1}$ by homomorphism property. Hence $\phi(g_1g_2^{-1}) = h_1h_2^{-1} \in \im\phi$.

    Therefore, by subgroup test, $\im\phi \leq H$.
\end{proof}

\begin{exercise}
    Consider the map $\phi: \Z_3 \to \Z_6, n \mapsto 2n$. Determine whether $\phi$ is a homomorphism and, if so, find its image.
\end{exercise}

\section{Kernel of a Homomorphism}
\begin{definition}
    The \textbf{kernel}\index{homomorphism!kernel} of a homomorphism $\phi: G \to H$ is the set of elements in the group $G$ which map to the identity in the group $H$. That is, if the identity of $H$ is $e_H$, then the kernel of $\phi$ is the set
    \[
        \ker\phi = \{x \in G \vert \phi(x) = e_H\}.
    \]
\end{definition}


\begin{remark}
    Some authors (e.g. {\cite[Definition 4.2.0]{libretexts_im-and-ker}}) will use the notation $\phi^{-1}(e_H)$ for the kernel of $\phi$. The alternate notation $\mathrm{Ker}\;\phi$ may also be used by some authors (e.g. by {\cite[Definition I.2.2]{hungerford_1980}} and \cite[\S 65]{clark_1984}).
\end{remark}

\begin{example}
    Let the groups $G = (\Z^2, (+, +))$ and $H = (\Z, +)$. Let the map $\phi: G \to H, (a, b) \mapsto a+b$. Then, $(a, b) \in \ker\phi$ if $\phi((a,b)) = 0$. This means that $a+b = 0$, implying $ b = -a$. Hence the kernel of $\phi$ is $\{(a, -a) \vert a \in \Z\}$.
\end{example}

\begin{exercise}
    Let $i$ be the imaginary unit, that is $i^2 = -1$. Let the group $G$ be the integers under addition and $H = \langle i \rangle$ be under multiplication. Let the map $\phi: G \to H, n \mapsto i^n$.  Show that $\phi$ is a homomorphism and hence find $\ker\phi$.
\end{exercise}

\begin{proposition}\label{prop-kernel-is-normal-subgroup-of-domain}
    Let $\phi: G \to H$ be a homomorphism. Then $\ker\phi \unlhd G$.
\end{proposition}
\begin{proof}
    We will first show $\ker\phi\leq G$. Clearly $e_G \in \ker\phi$ since $\phi(e_G) = e_H$, so $\ker\phi$ is non-empty. Now let $x, y \in \ker\phi$. This means that $\phi(x) = \phi(y) = e_H$. Note
    \begin{align*}
        \phi(xy^{-1}) &= \phi(x)\left(\phi(y)\right)^{-1}\\
        &= e_H(e_H)^{-1}\\
        &= e_H
    \end{align*}
    which means that $xy^{-1}\in\ker\phi$. By subgroup test, $\ker\phi\leq G$.

    Now we prove normality. Let $x \in G$ and $n \in \ker\phi$. We need to show that $xnx^{-1}\in\ker\phi$ to prove normality. Observe that
    \begin{align*}
        \phi(xnx^{-1}) &= \phi(x)\phi(n)\phi(x^{-1})\\
        &= \phi(x)e_H\phi(x)^{-1} & (n \in \ker\phi)\\
        &= \phi(x)\phi(x)^{-1}\\
        &= e_H,
    \end{align*}
    which means that $xnx^{-1} \in \ker\phi$. Hence, $\ker\phi \unlhd G$.
\end{proof}

\begin{exercise}\label{exercise-trivial-kernel-means-injective}
    Prove that a homomorphism $\phi:G\to H$ is injective if and only if $\ker \phi$ is trivial, i.e. $\ker \phi = \{e_G\}$.
\end{exercise}

\section{The Fundamental Homomorphism Theorem}
We are now ready to tackle the three most important theorems regarding homomorphisms. We first state the \textbf{Fundamental Homomorphism Theorem}, which is also sometimes called the \textbf{First Isomorphism Theorem} (e.g. in {\cite[p.~251, Theorem 3]{cohn_1982}}).
\begin{theorem}[Fundamental Homomorphism Theorem]\label{thrm-isomorphism-1}\index{Fundamental Homomorphism Theorem}\index{Isomorphism Theorem!First}
    Let $G$ and $H$ be groups. Let $\phi: G \to H$ be a homomorphism, and let $\pi: G \to G/\ker\phi$ where $g\mapsto g\ker\phi$ be the natural surjective homomorphism. Then there exists a unique isomorphism $\psi: G/\ker\phi \to \im\phi$ such that $\psi\pi = \phi$.
\end{theorem}
\begin{remark}
    Equivalently, the Fundamental Homomorphism Theorem states that
    \[
        G/\ker\phi \cong \im\phi
    \]
    for any homomorphism $\phi$.
\end{remark}

We include the commutativity diagram of the homomorphisms stated to aid clarity:

\begin{figure}[h]
    \centering
    \fbox{\includegraphics[width=0.35\textwidth]{further-homomorphisms/iso-1-comm-diagram.png}}
    \caption{Commutativity Diagram for \myreffigures{thrm-isomorphism-1}}
\end{figure}

In the diagram, $\phi$ sends elements from $G$ to $\im\phi$ and $\pi$ sends elements from $G$ to $G/\ker\phi$. Then the map $\psi$ is a unique map that sends elements from $G/\ker\phi$ to the image of $\phi$.

\begin{proof}
    We know by \myref{prop-image-is-subgroup-of-codomain} that $\im\phi \leq H$. Let $\psi: G/\ker\phi \to \im\phi$ such that $\psi(x\ker\phi) = \phi(x)$. We need to check that $\psi$ is a well-defined isomorphism.
    \begin{itemize}
        \item \textbf{Well-defined}: Suppose $x\ker\phi = y\ker\phi$ where $x, y \in G$. Then $xy^{-1} \in \ker\phi$ by Coset Equality (\myref{lemma-coset-equality}), statements 1 and 5. This means that $\phi(xy^{-1}) = e_H$ by definition of the kernel. Note $\phi(xy^{-1}) = \phi(x)\left(\phi(y)\right)^{-1}$, so $\phi(x)\left(\phi(y)\right)^{-1} = e_H$. Hence $\phi(x) = \phi(y)$. Thus,
        \[
            \psi(x\ker\phi) = \phi(x) = \phi(y) = \psi(y\ker\phi)
        \]
        so $\psi$ is well-defined.

        \item \textbf{Homomorphism}: Note that
        \begin{align*}
            \psi((x\ker\phi)(y\ker\phi)) &= \psi((xy)\ker\phi)\\
            &= \phi(xy)\\
            &= \phi(x)\phi(y)\\
            &= \psi(x\ker\phi)\psi(y\ker\phi)
        \end{align*}
        so $\psi$ is a homomorphism.
        \item \textbf{Injective}: By \myref{exercise-trivial-kernel-means-injective}, we check that $\psi$ is injective by showing that $\ker\psi$ is trivial, i.e. $\ker\psi = \{\ker\phi\}$.

        Suppose $x\ker\phi\in\ker\psi$. Then $\psi(x\ker\phi) = e_H$ by definition of kernel. Hence $\phi(x) = e_H$ by definition of $\psi$, which means $x \in \ker\phi$ by definition of kernel. Thus $x\ker\phi = \ker\phi$ by Element in Coset (\myref{corollary-equivalence-of-element-in-coset}). Therefore $\psi$ is injective.

        \item \textbf{Surjective}: Suppose $y$ is in the image of $\phi$, meaning there exists a $x \in G$ such that $\phi(x) = y$. Note that $\psi(x\ker\phi) = \phi(x) = y$. Thus $\psi$ is surjective.
    \end{itemize}
    Thus $\psi$ is a well-defined isomorphism.

    We now check that $\psi$ satisfies the requirement that $\psi\pi = \phi$. Let $x \in G$. Note that $\pi(x) = x\ker\phi$, and
    \[
        \psi\pi(x) = \psi(x\ker\phi) = \phi(x)
    \]
    for all $x \in G$, so $\psi\pi = \phi$.

    Finally we show that $\psi$ is unique. Suppose $f: G/\ker\phi \to \im\phi$ is an isomorphism satisfying $f\pi=\phi$. Take $x\ker\phi \in G/\ker\phi$. Note that
    \begin{align*}
        f(x\ker\phi) &= f(\pi(x))\\
        &= (f\pi)(x)\\
        &= \phi(x)\\
        &= (\psi\pi)(x)\\
        &= \psi(\pi(x))\\
        &= \psi(x\ker\phi)
    \end{align*}
    for all $x \in G$, meaning that $f = \psi$. Therefore $\psi$ is unique.

    Hence, $\psi$ is a unique isomorphism satisfying $\psi\pi = \phi$.
\end{proof}

\begin{example}
    Let $R = \{x \in \R \vert x > 0\}$, $G = \{x \in \R \vert x \neq 0\}$, and $H = \{1, -1\}$ be groups under multiplication. We show $G / H \cong R$.

    Consider the map $\phi: G \to R$ where $x \mapsto |x|$. We show that $\phi$ is a homomorphism, then find the image of $\phi$, and finally find its kernel.

    \begin{itemize}
        \item \textbf{Homomorphism}: $\phi$ is a homomorphism since $\phi(xy) = |xy| = |x||y| = \phi(x)\phi(y)$.
        \item \textbf{Image}: We find the image of $\phi$.
        \begin{align*}
            \im\phi &= \{\phi(x) \vert x \in G\}\\
            &= \{|x| \vert x \neq 0\}\\
            &= \{x \in \R \vert x > 0\} & (\text{by definition of } |x|)\\
            &= R
        \end{align*}
        which actually means that $\phi$ is surjective.
        \item \textbf{Kernel}: We find the kernel of $\phi$.
        \begin{align*}
            \ker\phi &= \{x \in G \vert \phi(x) = 1\} & (1 \text{ is the identity in } R)\\
            &= \{x \in G \vert |x| = 1\}\\
            &= \{1, -1\}\\
            &= H
        \end{align*}
    \end{itemize}
    Thus $G/H \cong R$ by the Fundamental Homomorphism Theorem (\myref{thrm-isomorphism-1}).
\end{example}

\begin{exercise}
    Let $\phi: G \to H$ be a homomorphism between finite groups $G$ and $H$. Prove that
    \[
        |G| = |\im \phi|\times|\ker \phi|.
    \]
\end{exercise}

\section{The Diamond Isomorphism Theorem}
We now look at the next theorem, called the \textbf{Diamond Isomorphism Theorem} (e.g. in {\cite[Theorem 3.18]{dummit_foote_2004}}) or the \textbf{Second Isomorphism Theorem} (e.g. in {\cite[\S 69]{clark_1984}}).
\begin{theorem}[Diamond Isomorphism Theorem]\label{thrm-isomorphism-2}\index{Diamond Isomorphism Theorem}\index{Isomorphism Theorem!Second}
    Let $G$ be a group and let $H$ and $K$ be subgroups of $G$. Then
    \begin{enumerate}
        \item $H \cap K \leq H$; and
        \item $H \leq HK$.
    \end{enumerate}
    Furthermore, if $N \unlhd G$, then
    \begin{enumerate}[start=3]
        \item $HN \leq G$;
        \item $H \cap N \unlhd H$;
        \item $N \unlhd HN$; and
        \item $H / (H\cap N) \cong HN / N$.
    \end{enumerate}
\end{theorem}

\newpage

We can capture the overall relationships of the subgroups of $G$ using a \textbf{subgroup lattice}.
\begin{figure}[h]
    \centering
    \fbox{\includegraphics[width=0.25\textwidth]{further-homomorphisms/iso-2-subgroup-diagram.png}}
    \caption{Subgroup Lattice for \myreffigures{thrm-isomorphism-2}}
\end{figure}

We only show subgroups that we care about in the diagram. The group $G$ has a (direct) subgroup $HK$; $HK$ has subgroups $H$ and $K$; and $H$ and $K$ has a common subgroup $H\cap K$. The dotted quotient groups are isomorphic to each other if $H \unlhd G$.

\begin{proof}
    We prove each statement in sequence.

    \begin{enumerate}
        \item Clearly $e_G \in H$ and $e_G \in K$ so $e_G \in H \cap K$. Now take $x, y \in H \cap K$, meaning $x, y \in H$ and $x, y \in K$. Since $H, K \leq G$ so $xy^{-1} \in H$ and $xy^{-1} \in K$. Thus $xy^{-1} \in H \cap K$. By the subgroup test, this means that $H \cap K \leq H$.
        
        \item Note that $H = \{he_G \vert h \in H\} \subseteq \{hk \vert h \in H, k \in K\} = HK$, and $H$ is a group (as $H$ is a subgroup). Therefore $H \leq HK$.
        
        \item We note that, because $N$ is normal, hence $hN = Nh$ for all $h \in H \subseteq G$, meaning that $HN = NH$. Therefore by \myref{prop-subgroup-product-is-subgroup}, we have $HN \leq G$.

        \item We know $H \cap N \leq H$ by statement 1, so we only prove normality. Take $x \in H \cap N$. Since $H \leq G$, thus $x \in H \cap N \subseteq H$, meaning for all $g \in H$, $gxg^{-1} \in H$ (where we think of $g$ and $x$ as being in $H$). But since $x \in H \cap N \subseteq N$ and $N \unlhd G$, thus $gxg^{-1} \in N$ (where we think of $g \in H$ and $x \in N$). Therefore $H \cap N \unlhd H$.

        \item We know $N \leq HN$ by statement 2, so we only prove normality. Take $n \in N$ and $x \in HN$ such that $x = h_xn_x$. Then
        \begin{align*}
            xnx^{-1} &= (h_xn_x)n(h_xn_x)^{-1}\\
            &= (h_xn_x)n(n_x^{-1}h_x^{-1}) & (\text{Shoes and Socks})\\
            &= \underbrace{h_x}_{\text{In }G}\underbrace{n_xnn_x^{-1}}_{\text{In }N}\underbrace{h_x^{-1}}_{\text{In G}}\\
            &\in N
        \end{align*}
        since $N \unlhd G$. This proves that $N \unlhd HN$.

        \item This is the main result of this theorem. We define $\phi: H \to HN/N, h \mapsto hN$. We show that $\phi$ is a homomorphism and then find its image and kernel.
        \begin{itemize}
            \item \textbf{Homomorphism}:
            \[
                \phi(xy) = (xy)N = (xN)(yN) = \phi(x)\phi(y)    
            \]

            \item \textbf{Image}: We show that $\phi$ is surjective to show that $\im\phi = HN/N$. Suppose $x \in HN$, meaning $x = hn$ where $h \in H$ and $n \in N$. Thus $xN \in HN/N$, so
            \[
                xN = (hn)N = h(nN) = hN
            \]
            meaning $\phi(h) = hN = xN$. Hence we have found a pre-image of the coset $xN$, meaning $\phi$ is surjective. Thus $\im \phi = HN/N$.

            \item \textbf{Kernel}: We claim that $\ker\phi = H \cap N$.
            
            Note that $\ker\phi = \{h \in H \vert \phi(h) = eN = N\}$ by definition of kernel. This means that if $h \in \ker\phi$ then $\phi(h) = N$. Hence $\phi(h) = hN = N$, which means $h \in N$ by Element in Coset (\myref{corollary-equivalence-of-element-in-coset}). Thus, $h \in H$ and $h \in N$, meaning $h \in H \cap N$. Therefore $\ker \phi \subseteq H \cap N$.
    
            Now suppose $x \in H \cap N$. This means that $x \in N$ necessarily, implying $xN = N$. Thus $\phi(x) = N$ which quickly implies $x \in \ker\phi$. Therefore $H \cap N \subseteq \ker\phi$.
    
            Since $\ker \phi \subseteq H \cap N$ and  $H \cap N \subseteq \ker\phi$ therefore $\ker\phi = H\cap N$.
        \end{itemize}

        By the Fundamental Homomorphism Theorem (\myref{thrm-isomorphism-1}),
        \[
            H / \ker\phi \cong \im \phi,
        \]
        which means
        \[
            H/(H\cap N) \cong HN/N.
        \]
    \end{enumerate}
    This completes the proof of the theorem.
\end{proof}

\begin{corollary}\label{corollary-subgroup-product-is-normal-subgroup-if-subgroups-are-normal}
    Let $G$ be a group with proper subgroups $H$ and $K$. Then $HK \unlhd G$ if $H$ and $K$ are normal subgroups of $G$.
\end{corollary}
\begin{proof}
    Assume that $H, K \lhd G$. By the Diamond Isomorphism Theorem (\myref{thrm-isomorphism-2}), statement 3, we know that $HK \leq G$ since $H \lhd G$. We just need to prove normality. Suppose $hk \in HK$ and take $g \in G$. Then
    \begin{align*}
        g(hk)g^{-1} &= (gh)(kg^{-1})\\
        &= (hg)(g^{-1}k) & (\text{as } H, K \lhd G)\\
        &= h(gg^{-1})k\\
        &= hk \in HK
    \end{align*}
    which means that $HK \unlhd G$.
\end{proof}

We look at two examples using the Diamond Isomorphism Theorem.
\begin{example}
    We say a group $G$ is \textbf{metabelian}\index{metabelian} if and only if there exists $A \unlhd G$ such that $A$ and $G/A$ are both abelian. We will prove that any subgroup of a metabelian group is also metabelian.

    Let $H \leq G$. Then, by the Diamond Isomorphism Theorem (\myref{thrm-isomorphism-2}), we know $H \cap A \unlhd H$ (statement 4) and $H/(H \cap A) \cong HA / A$ (statement 6). We just need to prove that both $H \cap A$ and $H/(H \cap A)$ are abelian.
    \begin{itemize}
        \item Consider any two elements from $H \cap A$, say $x$ and $y$. Then $x \in A$ and $y \in A$, so $xy = yx$ as $A$ is abelian. Hence, elements from $H \cap A$ commute, meaning that $H \cap A$ is abelian.
        \item Consider $H/(H\cap A) \cong HA / A$. Note that $HA \leq G$ since $H \leq G$ and $A \leq G$. Thus, $HA / A \leq G / A$. Note that $G/A$ is abelian by definition of metabelian group. Hence, $H/(H \cap A)$ is also abelian.
    \end{itemize}
    Therefore, we have found a subgroup of $H$ (in particular $H \cap A$) such that both $H \cap A$ and $H/(H\cap A)$ are both abelian. Hence, $H$ is metabelian.
\end{example}

We look at another application of the Diamond Isomorphism Theorem, which has use in Number Theory.
\begin{example}
    We will prove that $\lcm(m,n)\times\gcd(m,n) = mn$ (\myref{prop-product-of-gcd-and-lcm}) by considering the Diamond Isomorphism Theorem. For brevity, let $d = \gcd(m,n)$ and $l = \lcm(m,n)$.

    Consider the groups $G = \Z$, $H = m\Z$, and $N = n\Z$ under addition. By Diamond Isomorphism Theorem (\myref{thrm-isomorphism-2}),
    \[
        m\Z/(m\Z \cap n\Z) \cong (m\Z + n\Z)/(n\Z).
    \]

    Now $m\Z \cap n\Z$ is the set of integers that are both a multiple of $m$ and $n$. Hence, $m\Z \cap n\Z = \lcm(m,n)\Z = l\Z$. On the other hand, $m\Z + n\Z$ is the set of all integers of the form $mx+ny$ where $x$ and $y$ are integers. B\'{e}zout's lemma (\myref{lemma-bezout}) tells us that this set consists of the multiples of $\gcd(m,n)$, i.e. $m\Z + n\Z = \gcd(m,n)\Z = d\Z$. Hence,
    \[
        m\Z/(l\Z) \cong d\Z/(n\Z).
    \]

    We claim that $m\Z / (l\Z) \cong \Z_{\frac lm} \text{ and } d\Z / (n\Z) \cong \Z_{\frac nd}$. This is a specific case of \myref{problem-mZ/nZ-isomorphic-to-Zn/m} which we have left as a problem for later. Hence,
    \[
    \Z_{\frac lm} \cong m\Z/(l\Z) \cong d\Z/(n\Z) \cong \Z_{\frac nd},
    \]
    which means that $\Z_{\frac lm} \cong \Z_{\frac nd}$. We can now finally take orders on both sides:
    \[
        \frac{l}{m} = \frac{n}{d},
    \]
    which means that $ld = mn$. Hence, $\lcm(m,n)\times\gcd(m,n) = mn$.
\end{example}

\begin{exercise}\label{exercise-order-of-subgroup-product}
    Let $G$ be a finite group, $H \leq G$, and $N \lhd G$. Prove that
    \[
        |HN| = \frac{|H||N|}{|H \cap N|}.
    \]
\end{exercise}

\section{The Third Isomorphism Theorem}
We look at the last important theorem regarding homomorphisms and isomorphisms. This is often called the \textbf{Third Isomorphism Theorem} (e.g. in {\cite[Corollary I.5.10]{hungerford_1980}} and {\cite[pp.~253--254, Theorem 5]{cohn_1982}}).

It should be noted that there is no consistency with the numbering of these theorems in books (cf. {\cite[\S 68]{clark_1984}} as ``First Isomorphism Theorem'', {\cite[Theorem 8.16]{humphreys_1996}} as ``Second Isomorphism Theorem''), but the name ``Third Isomorphism Theorem'' is the easiest to research. Hence, we use that name here.

\begin{theorem}[Third Isomorphism Theorem]\label{thrm-isomorphism-3}\index{Isomorphism Theorem!Third}
    Let $G$ be a group. Let $H \unlhd G$ and $N \unlhd G$. Suppose $N \subseteq H$. Then
    \begin{enumerate}
        \item $N \unlhd H$;
        \item $H/N \unlhd G/N$; and
        \item $\frac{G/N}{H/N} \cong G/H$.
    \end{enumerate}
\end{theorem}
\begin{proof}
    We prove the statements in sequence.

    \begin{enumerate}
        \item We note that since $N \subseteq H$ and $N$ is a group (since $N$ is a normal subgroup of $G$) thus $N \leq H$. We just need to prove normality. Since $H$ and $N$ are normal subgroups of $G$, thus for all $g \in G$,
        \[
            gH = Hg \text{ and } gN = Ng.
        \]
        Now since $N \subseteq H \subseteq G$, thus for all $n$ in $N$, $nH = Hn$ (since $n \in G$). This means that $N \unlhd H$.

        \item We first prove that it is a subgroup before proving normality.

        Clearly $N = eN \in H/N$. Let $x$ and $y$ be in $H/N$. Then $x=h_xN$ and $y=h_yN$ for some $h_x, h_y \in H$. Note that $y^{-1} = (h_y^{-1})N$ by group operator on cosets. Hence,
        \begin{align*}
            xy^{-1} &= (h_xN)(h_y^{-1}N)\\
            &= (\underbrace{h_xh_y^{-1}}_{\text{In }H})N\\
            &\in H/N
        \end{align*}
        Hence, by subgroup test, $H/N \leq G/N$.

        Now let $gN \in G/N$ and $hN \in H/N$. We need to show that $(gN)(hN)(gN)^{-1} \in H/N$. Note $(gN)(hN)(gN)^{-1} = (ghg^{-1})N$. Since $H \unlhd G$, thus $ghg^{-1} \in H$ which means that $(ghg^{-1})N \in H/N$.

        Therefore $H/N \unlhd G/N$.

        \item This is the main result of the theorem.

        Define $\phi: G/N \to G/H, gN \mapsto gH$. We check that $\phi$ is a well-defined homomorphism and find its image and kernel.
        \begin{itemize}
            \item \textbf{Well-defined}: Suppose $gN = g'N$. Then $g(g')^{-1} \in N$ by Coset Equality (\myref{lemma-coset-equality}), statements 1 and 5. Since $N \subseteq H$, thus $g(g')^{-1} \in H$ which implies $gH = g'H$, again by Coset Equality, statements 1 and 5. Hence
            \[
                \phi(gN) = gH = g'H = \phi(g'N)
            \]
            so $\phi$ is well-defined.

            \item \textbf{Homomorphism}: Take $gN, g'N \in G/N$. Then
            \begin{align*}
                \phi((gN)(g'N)) &= \phi((gg')N)\\
                &= (gg')H\\
                &= (gH)(g'H)\\
                &= \phi(gN)\phi(g'N)
            \end{align*}
            which means that $\phi$ is a homomorphism.
            
            \item \textbf{Image}: We show $\phi$ is surjective to prove that $\im\phi = G/H$. Suppose $gH \in G/H$. Clearly $\phi(gN) = gH$. Thus $gN$ is a pre-image of $gH$, meaning that $\phi$ is surjective. Hence $\im\phi = G/H$.
            
            \item \textbf{Kernel}: Suppose $gN \in \ker\phi = \{gN \vert \phi(gN) = eH = H\}$. Thus $\phi(gN) = gH = H$, which means $g \in H$. Hence $gN \in H/N$, so $\ker\phi \subseteq H/N$.

            Now assume $hN \in H/N$. Since $H\subseteq G$ (as $H \unlhd G$), thus $h \in G$. Therefore $hN \in G/N$, so $\phi(hN) = hH = H$. Hence $hN \in \ker\phi$ which means $H/N \subseteq \ker\phi$.

            Since $\ker\phi \subseteq H/N$ and $H/N \subseteq \ker\phi$, we must have $\ker\phi = H/N$.
        \end{itemize}

        By the Fundamental Homomorphism Theorem (\myref{thrm-isomorphism-1}), we have $\frac{G/N}{\ker\phi} \cong \im\phi$, which means
        \[
            \frac{G/N}{H/N} \cong G/H,
        \]
        proving statement 3.
    \end{enumerate}
    This proves the theorem.
\end{proof}

\begin{example}
    Take $G = \Z$, $H = m\Z$, and $N = mn\Z$. Note that clearly $H, N \leq G$, and since $G$ is abelian, we must also have $H \unlhd G$ and $N \unlhd G$. By the Third Isomorphism Theorem (\myref{thrm-isomorphism-3}), statement 3,
    \[
        \frac{G/N}{H/N} \cong G/H.
    \]
    Note $G/H = \Z/(m\Z) \cong \Z_m$ by \myref{problem-Zn-isomorphic-to-Z-by-nZ}. Note also
    \[
        \frac{G/N}{H/N} = \frac{\Z/(mn\Z)}{m\Z/(mn\Z)}.
    \]
    Hence we see that
    \[
        \frac{\Z/(mn\Z)}{m\Z/(mn\Z)} \cong \Z/(m\Z) \cong \Z_m.
    \]
\end{example}

\begin{exercise}
    Suppose $x$ and $y$ are positive integers such that $y = mx$ for some integer $m$. Let $H = x\Z$ and $N = y\Z$ be groups under addition.
    \begin{partquestions}{\roman*}
        \item Explain why $N \subseteq H$.
        \item Find a group $G$ such that $H \lhd G$ and $N \lhd G$.
        \item Hence find the order of $H/N$.
    \end{partquestions}
\end{exercise}

\newpage

\section{Problems}
\begin{problem}
    Let $G$ be a group. Prove that $G/G$ is isomorphic to the trivial group.
\end{problem}

\begin{problem}
    Let $R = (\R, +)$. Also let $G = R^2$ and $H = \left\{(r\sqrt2, r\sqrt3) \vert r\in R\right\}$ be groups under component-wise addition. Prove that $G/H \cong R$.
\end{problem}

\begin{problem}\label{problem-subgroup-product-equal-to-subgroup-if-one-is-subgroup-of-another}
    Let $G$ be a group. Let $H$ and $K$ be subgroups of $G$ such that $K \subseteq H$. Prove that $HK = H$.
\end{problem}

\begin{problem}\label{problem-cartesian-product-of-group-by-group-isomorphic-to-group}
    Let $G$ be an abelian group with operation $\ast$. Let $I = \{(g, g^{-1}) \vert g \in G\}$ be a group under component-wise application of $\ast$.
    \begin{partquestions}{\roman*}
        \item Show that $I \cong G$.
        \item Hence prove $G^2/G \cong G$ by considering a suitable homomorphism.
    \end{partquestions}
\end{problem}

\begin{problem}\label{problem-mZ/nZ-isomorphic-to-Zn/m}
    Let $G = m\Z$ and $H = n\Z$ be groups under addition, where $m\vert n$ and $m \neq n$. Let the map $\phi: G \to \Z/({\frac nm}\Z)$ be defined such that
    \[
        \phi(am) = a + \frac nm \Z.
    \]
    Prove that $G/H \cong \Z_{\frac nm}$.
\end{problem}

\section{More Types of Groups}
\subsection*{Exercises}
\begin{questions}
    \item Let $G = \Z_{mn}$ and $H = \{0, n, 2n, \dots, (m-1)n\}$. Clearly $H$ is a subgroup of $G$ of order $m$. By \myref{problem-subgroup-of-cyclic-group-is-cyclic} we know $H$ is normal and cyclic with order $m$ and by \myref{exercise-quotient-group-of-cyclic-group-is-cyclic} we know $G/H$ is cyclic. The order of $G/H$ is $\frac{|G|}{|H|} = \frac{mn}{m} = n$ by Lagrange's theorem (\myref{thrm-lagrange}), meaning that $G/H \cong \Z_n$. Hence, $\Z_{mn}/\Z_m \cong G/H \cong \Z_n$.

    \item Note 0 is the identity in $\Z_n$. By \myref{lemma-order-of-an-element-that-is-equivalent-to-identity} we know that if $12$ is equivalent to the identity in $\Z_n$, then $12 = mn$ for some integer $m$. Since $n > 0$ we restrict $m$ to positive integers. Now $12 = 2^2 \times 3$. Thus the possible values of $n$ are
    \begin{itemize}
        \item $n = 1$ with $m = 12$;
        \item $n = 2$ with $m = 6$;
        \item $n = 3$ with $m = 4$;
        \item $n = 4$ with $m = 3$;
        \item $n = 6$ with $m = 2$; and
        \item $n = 12$ with $m = 1$.
    \end{itemize}

    \item $|10| = \frac{210}{\gcd(10, 210)} = \frac{210}{10} = 21$, $|42| = \frac{210}{\gcd(42, 210)} = \frac{210}{42} = 5$, $|75| = \frac{210}{\gcd(75, 210)} = \frac{210}{15} = 14$, and $|140| = \frac{210}{\gcd(140, 210)} = \frac{210}{70} = 3$.

    \item \begin{partquestions}{\alph*}
        \item Note that $10 = 2 \times 5$. Generators of the group $\Z_{10}$ has to satisfy $\gcd(m, 10) = 1$ by \myref{corollary-element-in-cyclic-group-is-generator-iff-gcd-is-1}. The positive integers that satisfy this requirement (and which are less than 10) are 1, 3, 7, 9. Thus they are the generators of $\Z_{10}$.
        \item Note that 101 is prime. Hence all positive integers from 1 to 100 (inclusive) are generators. (Note that 0 is not a generator of $\Z_{101}$ since 0 is the identity.)
    \end{partquestions}

    \item We show that all subgroups of $\mathrm{Q}$ are, in fact, normal. We consider the first definition of the quaternion group.
    \begin{itemize}
        \item Clearly $\{1\} \lhd \mathrm{Q}$ and $\mathrm{Q} \unlhd \mathrm{Q}$.
        \item The subgroups $\langle i\rangle$, $\langle j\rangle$, and $\langle k\rangle$ have order 4. Therefore, Lagrange's theorem (\myref{thrm-lagrange}) tells us that they have index 2. Hence these subgroups are normal by \myref{problem-subgroup-of-index-2}.
        \item Consider the subgroup $\langle -1 \rangle = \{1, -1\}$. \begin{itemize}
            \item $1\langle -1 \rangle = \langle -1 \rangle1$, since 1 is the identity;
            \item $-1\langle -1 \rangle = \{1, -1\} = \langle -1 \rangle(-1)$;
            \item $i\langle -1 \rangle = \{-i, i\} = \langle -1 \rangle i$;
            \item $-i\langle -1 \rangle = \{i, -i\} = \langle -1 \rangle (-i)$;
            \item $j\langle -1 \rangle = \{-j, j\} = \langle -1 \rangle j$;
            \item $-j\langle -1 \rangle = \{j, -j\} = \langle -1 \rangle (-j)$;
            \item $k\langle -1 \rangle = \{-k, k\} = \langle -1 \rangle k$; and
            \item $-k\langle -1 \rangle = \{k, -k\} = \langle -1 \rangle (-k)$.
        \end{itemize}
        Thus $\langle -1 \rangle$ is normal.
    \end{itemize}
    Hence all subgroups of $\mathrm{Q}$ are normal.

    \item $\begin{pmatrix}2&6\end{pmatrix} = \begin{pmatrix}2&3\end{pmatrix}\begin{pmatrix}3&4\end{pmatrix}\begin{pmatrix}4&5\end{pmatrix}\begin{pmatrix}5&6\end{pmatrix}\begin{pmatrix}4&5\end{pmatrix}\begin{pmatrix}3&4\end{pmatrix}\begin{pmatrix}2&3\end{pmatrix}$.

    \item Note that $\begin{pmatrix}1&3&2&5&4\end{pmatrix} = \begin{pmatrix}1&4\end{pmatrix}\begin{pmatrix}1&5\end{pmatrix}\begin{pmatrix}1&2\end{pmatrix}\begin{pmatrix}1&3\end{pmatrix}$. \myref{thrm-parity-of-permutation} tells us that $\begin{pmatrix}1&3&2&5&4\end{pmatrix}$ is even and thus has a sign of $+1$.

    \item Note that $\An{3}$ has order $\frac{3!}{2} = 3$ so we should expect 3 permutations. Clearly the identity is one such permutation. Looking at \myref{example-symmetric-group-of-degree-3} we can find two more, namely $\begin{pmatrix}1&2&3\end{pmatrix}$ and $\begin{pmatrix}1&3&2\end{pmatrix}$.
    
    For $\An{4}$, note that it has order $\frac{4!}{2} = 12$ so we expect 12 permutations. Again the identity is one of them. Like $\An{3}$ we now find the 3-cycles in $\An{4}$, which are $\begin{pmatrix}1&2&3\end{pmatrix}$, $\begin{pmatrix}1&2&4\end{pmatrix}$, $\begin{pmatrix}1&3&2\end{pmatrix}$, $\begin{pmatrix}1&3&4\end{pmatrix}$, \linebreak $\begin{pmatrix}1&4&2\end{pmatrix}$, $\begin{pmatrix}1&4&3\end{pmatrix}$, $\begin{pmatrix}2&3&4\end{pmatrix}$, and $\begin{pmatrix}2&4&3\end{pmatrix}$. So there are 3 more permutations unaccounted, which are permutations of products of 2-cycles: $\begin{pmatrix}1&2\end{pmatrix}\begin{pmatrix}3&4\end{pmatrix}$, $\begin{pmatrix}1&3\end{pmatrix}\begin{pmatrix}2&4\end{pmatrix}$, and $\begin{pmatrix}1&4\end{pmatrix}\begin{pmatrix}2&3\end{pmatrix}$.

    \item $\Un{10} = \{1, 3, 7, 9\}$.

    \item By a corollary of Lagrange's theorem (\myref{corollary-order-of-group-multiple-of-order-of-element}), the order of $a$ dives the order of the group $\Un{n}$. Now since $|\Un{n}| = \totient(n)$, thus the order of $a$ divides $\totient(n)$.

    \item $\begin{pmatrix}2&1&2\\1&0&1\\2&1&2\end{pmatrix}$

    \item We already proved that $\Inn{G} \leq \Aut{G}$ so we only need to prove normality.

    Let $\phi \in \Aut{G}$ and $\iota_g \in \Inn{G}$. For brevity let $f = \phi\iota_g\phi^{-1}$. We note that $f \in \Aut{G}$; we need to prove that $f \in \Inn{G}$.

    Suppose $x \in G$. Since $\phi$ is an isomorphism, there exists $w \in G$ such that $x = \phi(w)$, i.e. $w = \phi^{-1}(x)$. So
    \begin{align*}
        f(x) &= \left(\phi\iota_g\phi^{-1}\right)(x)\\
        &= \phi(\iota_g(\phi^{-1}(x)))\\
        &= \phi(\iota_g(w))\\
        &= \phi(gwg^{-1})\\
        &= \phi(g)\phi(w)\phi(g^{-1})\\
        &= \phi(g)x\left(\phi(g)\right)^{-1}
    \end{align*}
    which shows that $f \in \Inn{G}$. Hence, $\Inn{G} \unlhd \Aut{G}$.
\end{questions}

\subsection*{Problems}
\begin{questions}
    \item We note that the two questions are equivalent to finding the orders of 3774 and 1870 in the group $\Z_{10101}$. We note that
    \begin{align*}
        1870 &= 2 \times 5 \times 11 \times 17,\\
        3774 &= 2 \times 3 \times 17 \times 37, \text{ and}\\
        10101 &= 3 \times 7 \times 13 \times 37.
    \end{align*}
    Therefore, $\gcd(1870, 10101) = 1$ and $\gcd(3774, 10101) = 3 \times 37 = 111$. Hence $|1870| = 10101$ and $|3774| = \frac{10101}{111} = 91$. Therefore, $a = 10101$ and $b = 91$.

    \item We claim that $\An{n}$ is non-abelian for any $n > 3$. Note that both $\pi = \begin{pmatrix}1 & 2 & 3\end{pmatrix}$ and $\sigma = \begin{pmatrix}2 & 3 & 4\end{pmatrix}$ are even permutations, and hence are in $\An{n}$ for any $n > 3$. We note
    \begin{itemize}
        \item $\pi\sigma = \begin{pmatrix}1 & 2 & 3\end{pmatrix}\begin{pmatrix}2 & 3 & 4\end{pmatrix} = \begin{pmatrix}1 & 2\end{pmatrix}\begin{pmatrix}3 & 4\end{pmatrix}$; and
        \item $\sigma\pi = \begin{pmatrix}2 & 3 & 4\end{pmatrix}\begin{pmatrix}1 & 2 & 3\end{pmatrix} = \begin{pmatrix}1 & 3\end{pmatrix}\begin{pmatrix}2 & 4\end{pmatrix}$.
    \end{itemize}
    So $\pi\sigma \neq \sigma\pi$. Thus $\An{n}$ is non-abelian for any $n > 3$.

    We note that
    \begin{itemize}
        \item $\An{2}$ has order 1 so $\An{2}$ is the trivial group, which is abelian (and cyclic); and
        \item $\An{3}$ has order 3 so $\An{3}$ is cyclic and thus abelian.
    \end{itemize}
    Thus the largest integer $n$ for which $\An{n}$ is abelian is $n = 3$. Furthermore $\An{k}$ is cyclic if $k = 2$ or $k = 3$.

    \item We first note that
    \[
        \totient(2p^k) = 2p^k\left(1-\frac12\right)\left(1-\frac1p\right) = p^k\left(1-\frac1p\right) = \totient(p^k).
    \]
    Now we are given that $r$ is an odd primitive root of $p^k$. Since $r \in \Un{p^k}$, thus $\gcd(r, 2p^k) = 1$ because $\gcd(r, p^k) = 1$. Now as $r$ is odd, thus $r \in \Un{2p^k}$. Let $n$ be the order of $r$ in $\Un{2p^k}$. Then by \myref{exercise-order-of-a-divides-phi-a} we know $n$ divides $\totient(2p^k)$. At the same time, because $r$ is a generator in $\Un{p^k} \cong \Z_{\phi(p^k)}$, so $\totient(p^k) = \totient(2p^k)$ divides $n$ by \myref{lemma-order-of-an-element-that-is-equivalent-to-identity}. Since $n$ divides $\totient(2p^k)$ and $\totient(2p^k)$ divides $n$ simultaneously, therefore $n = \totient(2p^k) = |\Un{2p^k}|$ which means that $r$ is a primitive root modulo $2p^k$.

    \item \begin{partquestions}{\roman*}
        \item The forward direction is clearly true since if $f_1 = f_2$, then $f_1(x) = f_2(x)$ for all $x \in G$, including $g \in G$. For the reverse direction, assume $f_1(g) = f_2(g)$. Note that
        \[
            f_1(g^k) = (f_1(g))^k = (f_2(g))^k = f_2(g^k)
        \]
        for any integer $k$. Since $g$ is a generator, thus we have $f_1(x) = f_2(x)$ for all $x \in G$, meaning $f_1 = f_2$.

        \item We note $f(g) \in G$. Since $g$ is a generator, hence $f(g) = g^k$ for some integer $k$. Hence any homomorphism from $G$ to $G$ is of the form $f(g) = g^{m_f}$ where $0 \leq m_f \leq n-1$, which means $m_f \in \Z_n$.

        \item Suppose the map $f_2: G \to G$ is another homomorphism where $f_2(g) = g^{m_f}$. Then we see
        \[
            f(g) = g^{m_f} = f_2(g)
        \]
        which means $f = f_2$ by \textbf{(i)}. Hence $m_f$ is unique to $f$.

        \item Consider $f_1(f_2(g))$. On one hand,
        \[
            f_1(f_2(g)) = f_1(g^{m_{f_2}}) = (f_1(g))^{m_{f_2}} = g^{m_{f_1}m_{f_2}},
        \]
        while on the other,
        \[
            f_1(f_2(g)) = (f_1 \circ f_2)(g) = g^{m_{f_1\circ f_2}}
        \]
        by definition of $m_f$ as introduced in \textbf{(ii)}. Therefore $m_{f_1\circ f_2} \equiv m_{f_1}m_{f_2} \pmod n$. In other words, $m_{f_1\circ f_2} = m_{f_1} \otimes_n m_{f_2}$.

        \item We prove the forward direction first by assuming that the map $f$ is an automorphism. Hence $f$ is surjective, meaning that there exists $a \in G$ such that $f(a) = g$. Since $a \in G$ thus $a = g^k$ for some $k \in \Z_n$ (we will show $k \in \Un{n}$ later). Observe
        \[
            g = f(a) = f(g^k) = (f(g))^k = g^{m_fk}
        \]
        which means $m_fk \equiv 1 \pmod n$. By \myref{prop-multiplicative-inverse-exists-iff-coprime}, this means that we have $\gcd(m_f, n) = 1$ and $\gcd(k, n) = 1$. Therefore, $m_f$ and $k$ are in $\Un{n}$. Hence $k$ is the multiplicative inverse of $m_f$.

        We now prove the reverse direction. Assume $m_f$ has a multiplicative inverse (say $k$), meaning $m_fk \equiv 1 \pmod n$. As above this means that both $m_f$ and $k$ are in $\Un{n}$. We show that $f$ is a bijection.
        \begin{itemize}
            \item \textbf{Injective}: Suppose $x, y \in G$ such that $f(x) = f(y)$. Since $g$ is a generator we may take $x = g^p$ and $y = g^q$ for some integers $p$ and $q$. Hence we have $g^{m_fp} = g^{m_fq}$. Then
            \[
                \left(g^{m_fp}\right)^k = g^{km_fp} = \left(g^{km_f}\right)^p = g^p
            \]
            and $\left(g^{m_fq}\right)^k = g^q$. Hence this implies $g^p = g^q$ which means $x = y$.
            \item \textbf{Surjective}: Suppose $x \in G$. Since $g$ is a generator we may write $x = g^p$ for some integer $p$. Then $f(g^{kp}) = g^{m_fkp} = g^p = x$.
        \end{itemize}
        Also $f$ is given to be a homomorphism. Hence $f$ is an isomorphism. Since $f: G \to G$, it is thus an automorphism.

        \item We prove that $\phi$ is an isomorphism.
        \begin{itemize}
            \item \textbf{Homomorphism}: Let $f_1, f_2 \in \Aut{G}$. Then
            \begin{align*}
                \phi(f_1\circ f_2) &= m_{f_1\circ f_2} & (\text{definition of } m_f \text{ in }\textbf{(ii)})\\
                &= m_{f_1} \otimes_n m_{f_2} & (\text{by \textbf{(iv)}})\\
                &= \phi(f_1)\otimes_n\phi(f_2),
            \end{align*}
            which means $\phi$ is a homomorphism.

            \item \textbf{Injective}: Suppose we have $f_1, f_2 \in \Aut{G}$ such that $\phi(f_1) = \phi(f_2)$. Thus $m_{f_1} = m_{f_2}$ by definition of $\phi$. However, we know that the value of $m$ uniquely defines a homomorphism from $G$ to $G$ from \textbf{(iii)}. Hence $f_1 = f_2$, which shows that $\phi$ is injective.

            \item \textbf{Surjective}: Suppose $r \in \Un{n}$. Define the map $f: G \to G$ where $f(g) = g^r$. Since $r \in \Un{n}$ it has a multiplicative inverse, which means that $f$ is an automorphism by \textbf{(v)}. Clearly $\phi(f) = r$, so $r$ has a pre-image. So $\phi$ is surjective.
        \end{itemize}
        Hence $\phi$ is an isomorphism, meaning $\Aut{G} \cong \Un{n}$.
    \end{partquestions}
\end{questions}

\section{Group Actions}
\begin{questions}
    \item We prove the two group action axioms.
    \begin{itemize}
        \item \textbf{Identity}: $\alpha(e, x) = exe^{-1} = x$.
        \item \textbf{Compatibility}: Note
        \begin{align*}
            \alpha(g, \alpha(h, x)) &= \alpha(g, hxh^{-1})\\
            &= gh x h^{-1}g^{-1}\\
            &= (gh)x(gh)^{-1}\\
            &= \alpha(gh, x).
        \end{align*}
    \end{itemize}
    Therefore $\alpha$ is a group action of $G$ on $G$.

    \item Recall there are 6 elements in $\Sn{3}$: $\id$, $\begin{pmatrix}1 & 2 & 3\end{pmatrix}$, $\begin{pmatrix}1 & 3 & 2\end{pmatrix}$, $\begin{pmatrix}1 & 2\end{pmatrix}$, $\begin{pmatrix}1 & 3\end{pmatrix}$, and $\begin{pmatrix}2 & 3\end{pmatrix}$. Clearly the identity has all elements of $X$ as fixed points. It is also clear that $\begin{pmatrix}1 & 2 & 3\end{pmatrix}$ and $\begin{pmatrix}1 & 3 & 2\end{pmatrix}$ have no fixed points since they permute all elements. For the rest, the fixed points are the missing element from the cycle notation, i.e. $\begin{pmatrix}1 & 2\end{pmatrix}$ has fixed point 3, $\begin{pmatrix}1 & 3\end{pmatrix}$ has fixed point 2, and $\begin{pmatrix}2 & 3\end{pmatrix}$ has fixed point 1.

    \item For 1, it is $\{\id, \begin{pmatrix}2 & 3\end{pmatrix}\}$. For 2, it is $\{\id, \begin{pmatrix}1 & 3\end{pmatrix}\}$. For 3, it is $\{\id, \begin{pmatrix}1 & 2\end{pmatrix}\}$.

    \item We work from the statement forwards. Note that each of these statements are ``if and only if'' statements.
    \begin{align*}
	    g \cdot x = h \cdot x &\iff g^{-1} \cdot (g \cdot x) = g^{-1} \cdot (h \cdot x)\\
	    &\iff (g^{-1}g) \cdot x = (g^{-1}h) \cdot x\\
	    &\iff e \cdot x = (g^{-1}h) \cdot x\\
	    &\iff x = (g^{-1}h) \cdot x\\
	    &\iff (g^{-1}h) \cdot x = x\\
	    &\iff g^{-1}h \in \Stab{G}{x}
	\end{align*}

	\item \begin{partquestions}{\alph*}
		\item An orbit takes the form $\Orb{G}{x}$. Clearly $e \cdot x = x$ so $x \in \Orb{G}{x}$ and thus $\Orb{G}{x}$ is non-empty.
	    \item Let $x \in X$. Since $e \cdot x = x$, so $x \in \Orb{G}{x}$.
	    \item Suppose $x \in \Orb{G}{x_1} \cap \Orb{G}{x_2}$ (as their intersection is non-empty). Then there exists $g_1, g_2 \in G$ such that $g_1\cdot x_1 = x = g_2\cdot x_2$. Thus,
	    \begin{align*}
	        x_1 &= e \cdot x_1\\
	        &= (g_1^{-1}g_1)\cdot x_1\\
	        &= g_1^{-1} \cdot (g_1 \cdot x_1)\\
	        &= g_1^{-1} \cdot (g_2 \cdot x_2)\\
	        &= (g_1^{-1}g_2) \cdot x_2.
	    \end{align*}
	    Now suppose $y \in \Orb{G}{x_1}$. Then $y = g\cdot x_1$ for some $g \in G$. Hence,
	    \begin{align*}
	        y &= g\cdot x_1 \\
	        &= g \cdot \left((g_1^{-1}g_2) \cdot x_2\right)\\
	        &= (\underbrace{gg_1^{-1}g_2}_{\text{In } G})\cdot x_2\\
	        &\in \Orb{G}{x_2}
	    \end{align*}
	    which means any element in $\Orb{G}{x_1}$ is also in $\Orb{G}{x_2}$. Hence, $\Orb{G}{x_1}$ is a subset of $\Orb{G}{x_2}$. A similar argument can be used to show that $\Orb{G}{x_2}$ is a subset of $\Orb{G}{x_1}$. Hence $\Orb{G}{x_1} = \Orb{G}{x_2}$.
	\end{partquestions}

	\item We prove the forward direction first: suppose the action is transitive. Then there exists $x \in X$ such that $\Orb{G}{x} = X$. Now consider any other element $y \in X$. Since the action is transitive, this means that there exists a $\hat{g} \in G$ such that $\hat{g} \cdot x = y$. Note that $\Orb{G}{y} = \Orb{G}{\hat{g} \cdot x}$, and that $\Orb{G}{x} = \{g \cdot x \vert g \in G\}$. Hence,
	\[
        \Orb{G}{\hat{g} \cdot x} = \{g\cdot (\hat{g} \cdot x) \vert g \in G\} = \{(g\hat{g}) \cdot x \vert g \in G\}.
	\]
	Since $G$ is a group, $g\hat{g} \in G$. In particular, we may pick $g = g'\hat{g}^{-1}$ to obtain any arbitrary element $g' \in G$. Thus, this means that
	\[
        	\{(g\hat{g}) \cdot x \vert g \in G\} = \{g' \cdot x \vert g' \in G \} = \Orb{G}{x} = X.
	\]
	Hence, for any element $y \in X$, $\Orb{G}{y} = \Orb{G}{g \cdot x} = X$.

	The reverse direction is trivial: suppose $\Orb{G}{x} = X$ for all $x \in X$. Then certainly there exists an element $x \in X$ such that $\Orb{G}{x} = X$, meaning that the group action is transitive.

	\item \begin{partquestions}{\alph*}
	    \item Consider $x = n$. The orbit of $n$ is all of $X$. Consider the permutation $\sigma = \begin{pmatrix}k & n\end{pmatrix}$ where $1 \leq k \leq n$. Clearly $\sigma \in \Sn{n}$. Note that $\sigma \cdot n = \sigma(n) = k$. Thus, $\Orb{G}{n} = X$, meaning that the group action ``$\cdot$'' given by $g \cdot x \mapsto g(x)$ is transitive.
	    \item Note that $|X| = n$ and $|\Sn{n}| = n!$. By Orbit-Stabilizer theorem (\myref{thrm-orbit-stabilizer}), the stabilizer of $x$ by $G$ must have order $\frac{n!}{n} = (n-1)!$.
	\end{partquestions}

	\item By the Orbit-Stabilizer theorem (\myref{thrm-orbit-stabilizer}),
	\[
        |\Orb{G}{x}| = \frac{|G|}{|\Stab{G}{x}|} = [G : \Stab{G}{x}]	.
	\]
	Under the group action of conjugation, $\Orb{G}{x} = \Cl{x}$ and $\Stab{G}{x} = \Centralizer{G}{x}$. Hence, $|\Cl{x}| = [G : \Centralizer{G}{x}]$ as required.

	\item \begin{partquestions}{\alph*}
	    \item One sees that $\Z{D_3} = \{e\}$ based on the group table of $D_3$.
	    \item Recall that every element in $D_3$ can be expressed in the form $r^as^b$ where $a \in \{0, 1, 2\}$ and $b \in \{0, 1\}$. One finds that $\Cl{r} = \{r, r^2\}$ and $\Cl{s} = \{s, rs, rs^2\}$.
	    \item The class equation is $6 = 1 + 2 + 3$.
	\end{partquestions}

	\item By Cauchy's Theorem (\myref{thrm-cauchy}) there exists an element (say $x$) with order $p$. Consider $H = \langle x \rangle$. Note that $|H| = p$ and $H \leq G$. Hence we found a subgroup of $G$ of order $p$.
\end{questions}

\section{Sylow Theorems}
\subsection*{Exercises}
\begin{questions}
    \item We note that $12 = 2^2 \times 3$. Thus a Sylow 2-subgroup must have order 4. Clearly $|3| = 4$ so $\langle 3 \rangle = \{0, 3, 6, 9\}$ is the Sylow 2-subgroup of $\Z_{12}$.

    \item Recall that $|\Sn{5}| = 120 = 2^3 \times 3 \times 5$. By a corollary of the First Sylow Theorem (\myref{corollary-sylow-p-subgroup-exists}), $\Syl{p}{G} \neq \emptyset$ if $p$ is 2, 3, or 5.

    \item We prove this by constructing the map $\phi: H \to gHg^{-1}$ where $h \mapsto ghg^{-1}$. We note that $\phi$ is an isomorphism.
    \begin{itemize}
        \item \textbf{Homomorphism}: Let $x, y \in H$. Then
        \[
            \phi(xy) = g(xy)g^{-1} = (gxg^{-1})(gyg^{-1}) = \phi(x)\phi(y)
        \]
        which clearly means that $\phi$ is a homomorphism.
        
        \item \textbf{Injective}: Suppose $x, y \in H$ such that $\phi(x) = \phi(y)$. Then $gxg^{-1} = gyg^{-1}$ which quickly implies $x = y$.
        
        \item \textbf{Surjective}: Suppose $ghg^{-1} \in gHg^{-1}$. Clearly we have $\phi(h) = ghg^{-1}$, so any element in $gHg^{-1}$ has a pre-image inside $H$.
    \end{itemize}
    Hence $H \cong gHg^{-1}$.

    \item By \myref{prop-order-of-conjugate-element-equals-order-of-element} we know that $|xyx^{-1}| = |y|$ for all $x, y \in G$. Substituting $x = g$, and $y = hg$ yields
    \[
        |xyx^{-1}| = |g(hg)g^{-1}| = |gh| \text{ and } |y| = |hg|
    \]
    so the result follows.

    \item Clearly $e \in \N{G}{S}$ since $eSe^{-1} = S$. Consider $x, y \in \N{G}{S}$, meaning that $xSx^{-1} = S$ and $ySy^{-1} = S$. Note that $y^{-1} \in \N{G}{S}$ since
    \begin{align*}
        y^{-1}S\left(y^{-1}\right)^{-1} &= y^{-1}Sy\\
        &= y^{-1}\left(ySy^{-1}\right)y & (\text{since } y \in \N{G}{S})\\
        &= (y^{-1}y)S(y^{-1}y)\\
        &= S.
    \end{align*}
    Therefore
    \begin{align*}
        \left(xy^{-1}\right)S\left(xy^{-1}\right)^{-1} &= \left(xy^{-1}\right)S\left(yx^{-1}\right)\\
        &= x\left(y^{-1}Sy\right)x^{-1}\\
        &= xSx^{-1} & (\text{since } y^{-1} \in \N{G}{S})\\
        &= S & (\text{since } x \in \N{G}{S})
    \end{align*}
    which means that $xy^{-1} \in \N{G}{S}$. Hence, by the subgroup test, we have $\N{G}{S} \leq G$.

    \item By the Second Sylow Theorem (\myref{thrm-sylow-2}), we know that $gHg^{-1} = K$. Since $H \cong gHg^{-1}$ by \myref{exercise-conjugate-subgroup-isomorphic-to-subgroup} thus $H \cong gHg^{-1} = K$ as required.

    \item We note $784 = 2^4 \times 7^2$, so $m = 16$, $p = 7$, and $k = 2$. By the Third Sylow Theorem (\myref{thrm-sylow-3}), we know that
    \begin{itemize}
        \item $n_7 = [G : \N{G}{P}] = \frac{|G|}{|\N{G}{P}|}$;
        \item $n_7 \mid 16$, which implies $n_7 \in \{1, 2, 4, 8, 16\}$; and
        \item $n_7 \equiv 1 \pmod 7$, which implies $n_7 \in \{1, 8, 15, 22, \dots\}$.
    \end{itemize}
    Hence $n_7 = 1$ or $n_7 = 8$. But since $P$ is not a normal subgroup of $G$, by \myref{corollary-sylow-subgroup-is-normal-if-it-is-unique}, $P$ cannot be the only Sylow 7-subgroup, meaning $n_7 \neq 1$. Hence $n_7 = 8$, so
    \[
        8 = n_7 = \frac{|G|}{|\N{G}{P}|} = \frac{784}{|\N{G}{P}|}
    \]
    which means that $|\N{G}{P}| = 98$.

    \item Note $130 = 2 \times 5 \times 13$. Consider the number of Sylow 13-subgroups, $n_{13}$. The Third Sylow Theorem (\myref{thrm-sylow-3}) tells us that
    \begin{itemize}
        \item $n_{13} \mid 2 \times 5 = 10$, so $n_{13} \in \{1, 2, 5, 10\}$, and
        \item $n_{13} \equiv 1 \pmod{13}$ so $n_{13} \in \{1, 14, 27, \dots\}$.
    \end{itemize}
    Hence $n_{13} = 1$. But by \myref{corollary-sylow-subgroup-is-normal-if-it-is-unique} this means that the only Sylow 13-subgroup is normal. Hence a group of order 130 is non-simple.
\end{questions}

\subsection*{Problems}
\begin{questions}
    \item Note $200 = 2^3 \times 5^2$. Note that for $p = 5$ we have $m = 8$ and the factors of 8 are 1, 2, 4, and 8. Furthermore by the Third Sylow Theorem (\myref{thrm-sylow-3}) we must have $n_5 \equiv 1 \pmod 5$. Hence $n_5 = 1$. By a corollary of the Second Sylow Theorem (\myref{corollary-sylow-subgroup-is-normal-if-it-is-unique}) this means that the only Sylow 5-subgroup is normal.

    \item Note $33 = 3 \times 11$,
    \begin{itemize}
        \item when $p = 3$ we have $m = 11$ and the factors of 11 are 1 and 11; and
        \item when $p = 11$ we have $m = 3$ and the factors of 3 are 1 and 3.
    \end{itemize}
    The Third Sylow Theorem (\myref{thrm-sylow-3}) tells us that $n_p \equiv 1 \pmod p$. Hence we must have $n_3 = n_{11} = 1$. A corollary of the Second Sylow Theorem (\myref{corollary-sylow-subgroup-is-normal-if-it-is-unique}) tells us that the only Sylow 3-subgroup and Sylow 11-subgroup are normal.

    \item For brevity let $q = 2^p - 1$, and we are given that $q$ is a prime. By the Third Sylow Theorem (\myref{thrm-sylow-3}), $n_q \mid 2^{p-1}$ and $n_q \equiv 1 \pmod p$. The factors of $2^{p-1}$ are $1, 2, 4, 8, \dots, 2^{p-1}$. We note $2^{p-1} < 2^p - 1 = q$ for any prime $p$ since
    \[
        2^{p-1} + 1 < 2^{p-1} + 2^{p-1} = 2(2^{p-1}) = 2^p
    \]
    which result immediately follows by subtracting 1 on both sides. Hence, the only possible value that satisfies both conditions is $n_q = 1$. By a corollary of the Second Sylow Theorem (\myref{corollary-sylow-subgroup-is-normal-if-it-is-unique}) this means that the only Sylow $q$-subgroup is normal, hence showing that a group with an even perfect number order is non-simple.

    \item \begin{partquestions}{\roman*}
        \item The divisors of $p$ are 1 and $p$ itself. By the Third Sylow Theorem (\myref{thrm-sylow-3}), $n_q$ divides $p$ and $n_q \equiv 1 \pmod q$. Since $p < q$ hence $p \not\equiv 1 \pmod q$ meaning that $n_q = 1$. By a corollary of the Second Sylow Theorem (\myref{corollary-sylow-subgroup-is-normal-if-it-is-unique}) the only Sylow $q$-subgroup is normal.

        \item The divisors of $q$ are 1 and $q$ itself. By the Third Sylow Theorem (\myref{thrm-sylow-3}), $n_p$ divides $q$ and $n_p \equiv 1 \pmod p$. Since $q \not\equiv 1 \pmod p$ by assumption, we must have $n_p = 1$.

        Recall that the order of an element in a group of order $pq$ must divide $pq$ (\myref{corollary-order-of-group-multiple-of-order-of-element}). Hence the possible orders of an element in such a group are 1, $p$, $q$, or $pq$.
        \begin{itemize}
            \item There is only one element of order 1, the identity.
            \item There are $p - 1$ elements of order $p$, all belonging in the single Sylow $p$-subgroup. Note that we subtract 1 because one element in the Sylow $p$-subgroup is the identity.
            \item There are $q - 1$ elements of order $q$, all in the single Sylow $q$-subgroup.
        \end{itemize}
        Hence, since the total number of elements in a group of order $pq$ is $pq$, the number of elements of order $pq$ is
        \begin{align*}
            pq - \left((p-1)+(q-1)+1\right) &= pq - (p+q - 1)\\
            &= pq - p - q + 1\\
            &> 2q - 2 - q + 1\\
            &= 2q - q - 1\\
            &= q - 1\\
            &> 0
        \end{align*}
        which means that there is at least one element of order $pq$. By \myref{thrm-cyclic-group-has-element-with-same-order} this means that such a group is cyclic.
    \end{partquestions}

    \item \begin{partquestions}{\roman*}
        \item Let $P$ be a Sylow $p$-subgroup of $N$. Lagrange's Theorem (\myref{thrm-lagrange}) tells us that $|G| = [G:N]|N|$. Since $p$ does not divide $[G:N]$ we must have $|N| = p^ka$ where $a$ divides $m$. Hence $|P| = p^k$ as $P$ is a Sylow $p$-subgroup of $N$. Since $P$ has order $p^k$ and $P \leq N \leq G$, thus $P$ is also a Sylow $p$-subgroup of $G$.
        \item Let $Q$ be a Sylow $p$-subgroup of $G$. The Second Sylow Theorem (\myref{thrm-sylow-2}) tells us there exist  a $g \in G$ such that $Q = gPg^{-1}$. Recall by definition of normality that $gNg^{-1} = N$ for any $g \in G$. Note also that $P \leq N$. Hence,
        \[
            Q = gPg^{-1} \leq gNg^{-1} = N
        \]
        which means that $Q$ is also a Sylow $p$-subgroup of $N$.
    \end{partquestions}

    \item We note $3325 = 5^2 \times 7 \times 19$. Let the group of order 3325 be $G$. We know that
    \begin{itemize}
        \item for $p = 5$ we have $m = 7 \times 19 = 133$ and so the possible divisors of $m$ are $\{1, 7, 19, 133\}$;
        \item for $p = 7$ we have $m = 5^2 \times 19 = 475$ and so the possible divisors of $m$ are $\{1, 5, 19, 25, 95, 475\}$; and
        \item for $p = 19$ we have $m = 5^2 \times 7 = 175$ and so the possible divisors of $m$ are $\{1, 5, 7, 25, 35, 175\}$.
    \end{itemize}
    The Third Sylow Theorem (\myref{thrm-sylow-3}) tells us that $n_p \equiv 1 \pmod p$. Thus $n_5 = n_7 = n_{19} = 1$. Let $P$, $Q$, and $R$ be the Sylow 5-subgroup, the Sylow 7-subgroup, and the Sylow 19-subgroup respectively. We note that $P$, $Q$, and $R$ are all normal subgroups of $G$ by \myref{corollary-sylow-subgroup-is-normal-if-it-is-unique}.

    Denote the group $QR$ by $H$. Since $Q$ and $R$ are of prime order, their intersection is the identity (\myref{problem-intersection-of-coprime-subgroups}). Furthermore, as $Q$ and $R$ are normal subgroups of $G$, thus they commute by \myref{problem-intersection-of-coprime-subgroups}. Therefore $H$ is the internal direct product of $Q$ and $R$, meaning $H \cong Q \times R$ by \myref{thrm-direct-product-equivalence}. Hence $|H| = |Q||R| = 7 \times 19 = 133$. Now because as $Q$ and $R$ are of prime order, thus $Q$ and $R$ are abelian and so is $H$. Hence $H$ is an abelian group of order 133.

    Now consider the group $PH$. Since 5 and 133 are coprime, thus $P \cap H = \{e\}$. In addition, since $P \lhd G$ thus $PH \leq G$ by Diamond Isomorphism Theorem (\myref{thrm-isomorphism-2}), statement 3. Also,
    \[
        |PH| = \frac{|P||H|}{|P \cap H|} = |P||H| = 5^2 \times 133 = 3325 = |G|
    \]
    which means that $G = PH$. Since $P \lhd G$, thus $ph = hp$ for any element $h \in H$, meaning elements in $P$ and $H$ commute. Hence, $G$ is the internal direct product of $P$ and $H$, meaning $G \cong P \times H$. As the external direct product of two abelian groups is also abelian (\myref{problem-external-direct-product-of-abelian-groups-is-abelian}) thus $G$ is abelian.

    \item Let $P$ be a Sylow $p$-subgroup of $G$. We note that $|G/P| = \frac{p^km}{p^k} = m$. Let $G$ act on the set of cosets $G/P$ by left multiplication, meaning $g\cdot (xP) = (gx)P$. We know by \myref{thrm-group-action-definition-equivalence} that this induces a homomorphism $\phi: G \to \Sn{m}$ where $\phi(g) = \sigma_g$ such that $\sigma_g(xP) = g\cdot (xP) = (gx)P$. By \myref{example-using-kernel-to-show-non-simple}, $\ker\phi = \bigcap_{x \in G}xPx^{-1}$.

    We note $\ker\phi \neq \{e\}$ since otherwise it would imply that $\phi$ is injective (\myref{exercise-trivial-kernel-means-injective}), which is impossible as that would mean $p^km = |G| \leq |\Sn{m}| = m!$ which is a contradiction. Also $\ker\phi \neq G$ as otherwise
    \[
        p^km = |G| = |\ker\phi| = \left|\bigcap_{x \in G} xPx^{-1}\right| \leq |xPx^{-1}| = |P| = p^k,
    \]
    which would mean $m = 1$, a contradiction. Hence $\ker\phi$ is a non-trivial proper subgroup of $G$. We note that $\ker\phi \lhd G$, so we have found a non-trivial proper normal subgroup of $G$, meaning that $G$ is non-simple.

    \item Let $G$ be a group of order 30. Note $30 = 2 \times 3 \times 5$, and consider $n_5$. The Third Sylow Theorem (\myref{thrm-sylow-3}) tells us that
    \begin{itemize}
        \item $6 \vert n_5$, so $n_5 \in \{1, 2, 3, 6\}$; and
        \item $n_5 \equiv 1 \pmod 5$, so $n_5 \in \{1, 6, 11, 16, \dots\}$.
    \end{itemize}
    Hence $n_5$ is 1 or 6. Seeking a contradiction, assume $n_5 = 6$, and let $P_5$ be a Sylow 5-subgroup.

    Since $|P_5| = 5$, which is prime, each non-identity element of $P_5$ is a generator. Hence, no two Sylow 5-subgroups can share any non-identity elements (otherwise they will be the same group), thereby meaning any two Sylow 5-subgroups intersect in the identity only. Thus there exists $6(5-1) = 24$ elements of order 5, meaning there must be 6 elements of order not equal to 5.

    Now consider $n_3$. Note by the Third Sylow Theorem again,
    \begin{itemize}
        \item $10 \vert n_3$, so $n_3 \in \{1, 2, 5, 10\}$; and
        \item $n_3 \equiv 1 \pmod 3$, so $n_3 \in \{1, 4, 7, 10, 13, \dots\}$.
    \end{itemize}
    Thus $n_3$ is 1 or 10. Now if $n_3 = 10$ then there must be $10(3-1) = 20$ elements of order 3, a contradiction to the fact there exists only 6 elements with order not 5. Hence $n_3 = 1$, meaning the only Sylow 3-subgroup (call it $P_3$) is normal in $G$.

    As $P_5 \leq G$ and $P_3 \lhd G$, by the Diamond Isomorphism Theorem (\myref{thrm-isomorphism-2}), statement 3, we have $P_5P_3 \leq G$. Note $P_5 \cap P_3 = \{e\}$ by \myref{problem-intersection-of-coprime-subgroups}. So \myref{exercise-order-of-subgroup-product} tells us
    \[
        |P_5P_3| = \frac{|P_5||P_3|}{|P_5 \cap P_3|} = \frac{5\times3}{1} = 15.
    \]
    One sees that $[G:P_5P_3] = \frac{30}{15} = 2$, so $P_5P_3 \lhd G$ by \myref{problem-subgroup-of-index-2}.

    Now \myref{problem-group-of-order-pq-has-normal-subgroup-of-order-q} tells us there exists a unique $H \lhd P_5P_3$ with $|H| = 5$. But since $P_5P_3 \lhd G$, \myref{problem-normal-subgroup-of-G-contains-all-sylow-p-subgroups} tells us that $P_5P_3$ contains all Sylow 5-subgroups of $G$, meaning $G$ has only 1 Sylow 5-subgroup, i.e. $n_5 = 1$, a contradiction to our assumption that $n_5 = 6$.

    Hence $n_5 = 1$. Therefore the unique Sylow 5-subgroup is a normal subgroup of $G$ by \myref{corollary-sylow-subgroup-is-normal-if-it-is-unique}.

    \item We prove that $G$ has a normal subgroup of order $p$, $q$, or $r$. By \myref{corollary-sylow-subgroup-is-normal-if-it-is-unique}, subgroups of order $p$, $q$, or $r$ are normal if they are unique. By way of contradiction, assume that they are not unique, meaning $n_p, n_q, n_r > 1$.

    By the Third Sylow Theorem (\myref{thrm-sylow-3}), $n_r \equiv 1 \pmod r$ and $n_r \mid pq$. The divisors of $pq$ are 1, $p$, $q$, and $pq$. We note that since both $p$ and $q$ are less than $r$, thus $p \not\equiv 1 \pmod r$ and $q \not\equiv 1 \pmod r$. The only possibility that is left is $n_r = pq$ as we assume $n_r \neq 1$. Similarly, $n_q \equiv 1 \pmod q$ and $n_q \mid pr$. The divisors of $pr$ are 1, $p$, $r$, and $pr$. Since $p < q$ thus $p \not\equiv 1 \pmod q$. Hence $n_q \geq r$ as we assume $n_q \neq 1$. Similarly, $n_p \geq q$.

    We now consider the number of elements with order $p$, $q$, and $r$.
    \begin{itemize}
        \item $\boxed{p}$ With $n_p \geq q$, there are at least $q(p-1)$ elements of order $p$. We minus 1 because one of the elements in a Sylow $p$-subgroup is the identity with order 1.
        \item $\boxed{q}$ With $n_q \geq r$, there are at least $r(q-1)$ elements of order $q$.
        \item $\boxed{r}$ We know $n_r = pq$ so there are exactly $pq(r-1)$ elements of order $r$.
    \end{itemize}
    Since the total number of elements, $pqr$, must be at least the sum of the numbers of these elements, thus
    \begin{align*}
        pqr &\geq q(p-1) + r(q-1) + pq(r-1)\\
        &= pq - q + qr - r + pqr - pq\\
        &= pqr + qr - q - r
    \end{align*}
    which means $qr - q - r \leq 0$. Rearranging, we see
    \[
        q \leq \frac{r}{r-1} = 1 + \frac{1}{r-1}.
    \]
    Since $p < q$ and they are both primes, we must have $q \geq 3$. Hence one sees
    \[
        3 \leq q \leq 1 + \frac{1}{r-1} \leq 2
    \]
    which is a clear contradiction. Hence, at least one of $n_p$, $n_q$, or $n_r$ is 1, meaning that there exists a non-trivial proper normal subgroup in $G$ by \myref{corollary-sylow-subgroup-is-normal-if-it-is-unique}. Therefore $G$ is non-simple.
\end{questions}

\section{Composition Series}
\subsection*{Exercises}
\begin{questions}
    \item \begin{partquestions}{\roman*}
        \item One sees clearly that $\{0, 2\}$ is the only non-trivial proper normal subgroup of $G$, so the subnormal series of length 2 is $1 \lhd \{0, 2\} \lhd G$.
        \item There are 2 factor groups of the above subnormal series. The first is $\{0, 2\} / 1 \cong \Z_2$ and the second is
        \begin{align*}
            G / \{0, 2\} &= \{g \oplus_4 \{0, 2\} \vert g \in G\}\\
            &= \{\{0, 2\}, \{1, 3\}, \{2, 0\}, \{3, 1\}\}\\
            &= \{\{0, 2\}, \{1, 3\}\}\\
            &= \langle \{1, 3\} \rangle\\
            &\cong \Z_2.
        \end{align*}
        \item Since $1 \lhd G$ and $\{0, 2\} \lhd G$ thus the subnormal series in \textbf{(i)} is also a normal series of $G$.
    \end{partquestions}

    \item By Lagrange's theorem (\myref{thrm-lagrange}) we know that the order of a subgroup must divide the order of the group. Furthermore $\Z_{120}$ is abelian, so any subgroup of it is normal. Now the subgroup $N = \{0, 2, 4, \dots, 118\}$ has 60 elements which is the maximum possible guaranteed by Lagrange. Hence $N$ is the maximal normal subgroup of $\Z_{120}$, which has order 60.

    \item $\Cn{6}$ has
    \begin{align*}
        &1 \lhd \Cn{2} \lhd \Cn{6} \text{ and }\\
        &1 \lhd \Cn{3} \lhd \Cn{6}
    \end{align*}
    as composition series up to isomorphism. In both cases, their composition length is 2. Their respective composition factors are
    \begin{itemize}
        \item $\Cn{2} / 1 \cong \Cn{2}$ and $\Cn{6} / \Cn{2} \cong \Cn{3}$ by \myref{exercise-Zmn-mod-Zn-cong-Zn}; and
        \item $\Cn{3} / 1 \cong \Cn{3}$ and $\Cn{6} / \Cn{3} \cong \Cn{2}$ by \myref{exercise-Zmn-mod-Zn-cong-Zn},
    \end{itemize}
    up to isomorphism.

    \item Let the group in question be $G$. We know by Cauchy's theorem (\myref{thrm-cauchy}) and \myref{exercise-group-of-order-multiple-of-prime-has-subgroup-of-prime-order}, and by writing $p^2$ as $p \times p$, that $G$ has a subgroup of order $p$. Let this subgroup be $H$.

    Lagrange's theorem (\myref{thrm-lagrange}) tells us that the possible orders of the subgroups of $G$ are 1, $p$, and $p^2$. These subgroups are $\{e\}$, $H$, and $G$ respectively. Furthermore, by \myref{problem-group-of-order-prime-squared-is-abelian}, $G$ must be abelian, therefore its subgroups are all normal (\myref{prop-subgroup-of-abelian-group-is-normal}). Finally, a corollary of Lagrange's theorem (\myref{corollary-group-with-prime-order-subgroups}) says that the only subgroups of $H$ are the trivial group and the group itself. Hence, $G$ has only one composition series, namely $1 \lhd H \lhd G$.
\end{questions}

\subsection*{Problems}
\begin{questions}
    \item \begin{partquestions}{\roman*}
        \item We note $\mathrm{V}$ has order 4. As $4 = 2 \times 2$, thus we know that $\mathrm{V}$ has a subgroup of order 2 (which is cyclic) by Cauchy's theorem (\myref{thrm-cauchy}). Now $\mathrm{V}$ is abelian (\myref{problem-group-of-order-prime-squared-is-abelian}) which means that the subgroup of order 2 is normal (\myref{prop-subgroup-of-abelian-group-is-normal}). Finally, the only possible order for a non-trivial proper subgroup of $\mathrm{V}$ is 2 by Lagrange's theorem (\myref{thrm-lagrange}). Hence, the only composition series for $\mathrm{V}$ is $1 \lhd \Cn{2} \lhd \mathrm{V}$ up to isomorphism.\newline
        (Note that this analysis applies for any group of order 4.)

        \item Recall that $\mathrm{Q} = \langle \alpha, \beta \vert \alpha^4 = e, \alpha^2 = \beta^2, \text{ and } \beta\alpha = \alpha^3\beta \rangle$. From the solution of \myref{exercise-normal-subgroups-of-quarternion-group}, the maximal subgroups of $\mathrm{Q}$ are $G_1 = \langle \alpha \rangle$, $G_2 = \langle \beta \rangle$, and $G_3 = \langle \alpha\beta \rangle$ (by setting $\alpha = i$ and $\beta = j$). We note the following.
        \begin{itemize}
            \item $G_1 = \{e, \alpha, \alpha^2, \alpha^3\} \cong \Cn{4}$.
            \item $G_2 = \{e, \beta, \beta^2, \beta^3\} = \{e, \beta, \alpha^2, \alpha^2\beta\} \cong \mathrm{V}$ where $a = \alpha^2$ and $b = \beta$.
            \item $G_3 = \{e, \alpha\beta, (\alpha\beta)^2, (\alpha\beta)^3\} = \{e, \alpha\beta, \alpha^2, \alpha^3\beta\} \cong \mathrm{V}$ with $a = \alpha\beta$ and $b = \alpha^2$.
        \end{itemize}
        Also, note that $\Cn{2} \cong \langle \alpha^2 \rangle \lhd G_1$, $\Cn{2} \cong \langle \beta^2 \rangle \lhd G_2$, and $\Cn{2} \cong \langle (\alpha\beta)^2 \rangle \lhd G_3$. Hence, the two series up to isomorphism are
        \begin{align*}
            1 \lhd \Cn{2} \lhd \Cn{4} \lhd \mathrm{Q} & \text{ and }\\
            1 \lhd \Cn{2} \lhd \mathrm{V} \lhd \mathrm{Q}.
        \end{align*}

        \item By the Jordan-H\"older theorem (\myref{thrm-jordan-holder}), the composition factors are isomorphic to each other. We note
        \begin{itemize}
            \item $\Cn{2} / 1 \cong \Cn{2}$;
            \item $\Cn{4} / \Cn{2} \cong \Cn{2}$ by \myref{exercise-Zmn-mod-Zn-cong-Zn}; and
            \item $\mathrm{V} / \Cn{2} \cong (\Cn{2})^2 / \Cn{2} \cong \Cn{2}$ by \myref{problem-cartesian-product-of-group-by-group-isomorphic-to-group}.
        \end{itemize}
        The only unaccounted set of factors is $\mathrm{Q}/\mathrm{V}$ and $\mathrm{Q}/\Cn{4}$. So, either $\mathrm{Q}/\mathrm{V} \cong \Cn{2}$ and $\mathrm{Q}/\Cn{4} \cong \Cn{2}$, or $\mathrm{Q}/\mathrm{V} \cong \mathrm{Q}/\Cn{4}$. Hence $\mathrm{Q}/H \cong \mathrm{Q}/K$.
    \end{partquestions}

    \item We know that $\An{4} \lhd \Sn{4}$ by \myref{prop-An-normal-subgroup-of-Sn}. Note $\An{4}$ is a maximal normal subgroup since $|\An{4}| = \frac{4!}2 = 12$ by \myref{prop-order-of-An}, and a subgroup's order must divide the order of the group by Lagrange's theorem (\myref{thrm-lagrange}).

    Now applying that theorem on $\An{4}$, we see that the possible orders of a subgroup of $\An{4}$ are 6, 4, 3, 2, and 1. We claim that a subgroup of order 6 does not exist. Note that $\An{4}$ contains
    \begin{itemize}
        \item 1 element of order 1;
        \item 3 elements of order 2; and
        \item 8 elements of order 3.
    \end{itemize}
    If a subgroup of order 6 exists (say, $H$), then its index would be $\frac{12}{6} = 2$ by Lagrange, meaning $H$ contains all odd order elements (\myref{problem-subgroup-of-index-2}). However, there are $1 + 8 = 9$ odd order elements, meaning that $H$ has an order of at least 9, a contradiction. Hence a subgroup with $\An{4}$ of order 6 is impossible.

    Now we note that a subgroup of order $4 = 2^2$ exists by a corollary of the First Sylow Theorem (\myref{corollary-sylow-p-subgroup-exists}) as it is a Sylow 2-subgroup. The Third Sylow Theorem (\myref{thrm-sylow-3}) tells us how many Sylow 2-subgroups there are, in particular
    \begin{itemize}
        \item $n_2 \vert 3$, so $n_2$ is 1 or 3; and
        \item $n_2 \equiv 1 \pmod2$, so $n_2 \in \{1, 3, 5, \dots\}$.
    \end{itemize}
    Hence $n_2 = 1$ or $n_2 = 3$. Now if $n_2 = 3$, then the number of elements of order of 1, 2, or 4 is
    \[
        3 \times (4 - 1) + 1 = 10,
    \]
    where the 3 is $n_2$, the $4-1$ is the number of non-identity elements in each Sylow 2-subgroup, and the $+1$ is to add the identity element. However, as noted above, there are only 4 elements of order 1, 2, or 4, a contradiction. Hence $n_2 = 1$, meaning the Sylow 2-subgroup (which is a subgroup of order 4) is normal (\myref{corollary-sylow-subgroup-is-normal-if-it-is-unique}). Therefore the subgroup of order 4 is the maximal normal subgroup of $\An{4}$.

    We note that the subgroup of order 4 of $\An{4}$ is not $\Cn{4}$ (as this would imply that $\An{4}$ has an element of order 4, which it does not). Hence, from \myref{problem-smallest-nonabelian-group}, the subgroup of order 4 must be isomorphic to the Klein-4 group, $\mathrm{V}$.

    Note that a group of order 4 has a subgroup of order 2 by Cauchy's theorem (\myref{thrm-cauchy}). Clearly such a subgroup is cyclic (since 2 is prime), and has index $\frac42 = 2$, meaning that it is normal in the group of order 4. Furthermore the trivial group is always a subgroup of any group.

    Hence, the composition series for $\Sn{4}$, up to isomorphism, is
    \[
        1 \lhd \Cn{2} \lhd \mathrm{V} \lhd \An{4} \lhd \Sn{4}.
    \]
    \begin{remark}
        We list the actual subgroups that are isomorphic to the above terms in the composition series here.
        \begin{itemize}
            \item $\Cn{2}$: $\{e, \begin{pmatrix}1&2\end{pmatrix}\begin{pmatrix}3&4\end{pmatrix}\}$
            \item V: $\{e, \begin{pmatrix}1&2\end{pmatrix}\begin{pmatrix}3&4\end{pmatrix}, \begin{pmatrix}1&3\end{pmatrix}\begin{pmatrix}2&4\end{pmatrix}, \begin{pmatrix}1&4\end{pmatrix}\begin{pmatrix}2&3\end{pmatrix}\}$
            \item $\An{4}$ is an actual subgroup of $\Sn{4}$
        \end{itemize}
    \end{remark}
\end{questions}

\chapter{Simple Groups}
Simple groups can be thought of as the `building blocks' of all (finite) groups. The finite simple groups have been completely classified; each belongs to one of 18 infinite families, or is one of 26 sporadic groups that do not follow a specific pattern. We look at the classification of the families of these simple groups here.

\section{Cyclic Groups of Prime Order}
The first infinite family of simple groups we will look at is the family of Cyclic Groups of Prime Order\index{cyclic group!of prime order}.

\begin{lemma}\label{lemma-cyclic-group-simple-iff-order-is-prime}
    $\Cn{n}$ is simple if and only if $n$ is prime.
\end{lemma}
\begin{proof}
    We first prove the forward direction. Suppose $\Cn{n}$ is simple with generator $g$. Then the only normal subgroups of $\Cn{n}$ are the trivial group and the group itself. Seeking a contradiction, assume $n$ is not prime; write $n = ab$ where $a$ and $b$ are positive integers that are both smaller than $n$. Then clearly $\langle g^a\rangle$ is a proper subgroup of $\Cn{n}$. Now $\Cn{n}$ is abelian (\myref{prop-cyclic-group-is-abelian}) which means all subgroups are normal (\myref{prop-subgroup-of-abelian-group-is-normal}). Hence we have found a non-trivial proper normal subgroup of $\Cn{n}$, namely $\langle g^a \rangle$, contradicting that $\Cn{n}$ has no non-trivial proper normal subgroups. Therefore $n$ is prime.

    We now prove the reverse direction. Suppose $n$ is a prime. Then by a corollary of Lagrange's theorem (\myref{corollary-group-with-prime-order-subgroups}), $\Cn{n}$ has no non-trivial proper subgroups. So the only subgroup with order smaller than $n$ is the trivial group, $\{e\}$. Clearly $\Cn{n}$ is normal in itself, and the trivial group is always a normal subgroup. Hence, as the only normal subgroups of $\Cn{n}$ are the trivial group and itself, thus $\Cn{n}$ is simple. Therefore, if $n$ is prime then $\Cn{n}$ is simple.
\end{proof}

In fact, we have a much stronger result which we prove here.

\begin{theorem}\label{thrm-abelian-group-simple-iff-cylic-group-of-prime-order}
    An abelian group is simple if and only if it has prime order.
\end{theorem}
Note that we do not assume that the abelian group is finite; we will show that the group is finite in the proof below.
\begin{proof}
    The reverse direction follows immediately from \myref{lemma-cyclic-group-simple-iff-order-is-prime}, so we prove the forward direction only.

    Suppose $G$ is a simple abelian group; we show that $G$ is finite. Let $g$ a non-identity element of $G$. Then $H = \langle g \rangle$ is a subgroup of $G$. In fact, since $G$ is abelian, $H \unlhd G$ (\myref{prop-subgroup-of-abelian-group-is-normal}). As $G$ is simple, therefore $H = G$, meaning that $g$ is a generator of $G$. Now if $G$ is an infinite group, then one also sees that $\langle g^2 \rangle < G$ which implies $\langle g^2 \rangle \lhd G$, contradicting the fact that $G$ is simple. Hence $G$ is a finite abelian group with generator $g$, meaning $G$ is cyclic. Result follows directly from \myref{lemma-cyclic-group-simple-iff-order-is-prime}.
\end{proof}

From this, we conclude that the only family of simple abelian groups is the family of cyclic groups of prime order.

\section{Alternating Group With Degree $>4$}
The other family of simple groups that is relatively easy to find (and define) is the family of alternating groups with degree above 4\index{alternating group!of degree $>4$}. However, to prove this claim, we need several preliminary results.

\begin{theorem}\label{thrm-group-of-order-60-with->1-sylow-5-subgroup-is-simple}
    Let $G$ be a group of order 60. If $G$ has more than one Sylow 5-subgroup then $G$ is simple.
\end{theorem}
\begin{proof}[Proof (see {\cite[Proposition 4.21]{dummit_foote_2004}})]
    By way of contradiction assume $G$ is a group of order 60 with more than one Sylow 5-subgroup, but has a non-trivial proper normal subgroup $H$. Note $60 = 5 \times 12$, so by the Third Sylow Theorem (\myref{thrm-sylow-3}),
    \begin{itemize}
        \item $12 \vert n_5$, so $n_5 \in \{1, 2, 3, 4, 6, 12\}$; and
        \item $n_5 \equiv 1 \pmod 5$, so $n_5 \in \{1, 6, 11, 16, \dots\}$.
    \end{itemize}
    Therefore $n_5 = 6$ as $n_5 > 1$ (given), i.e. there are 6 Sylow 5-subgroups.

    We note by Lagrange's theorem (\myref{thrm-lagrange}) that the order of $H$ belongs in the set $\{1, 2, 3, 4, 5, 6, 10, 12, 15, 20, 30, 60\}$. As $H$ is a non-trivial proper subgroup of $G$, thus $|H| \neq 1$ and $|H| \neq 60$. That leaves 4 cases which we will deal with separately.
    \begin{enumerate}
        \item $|H| = 6$. Note $6 = 2 \times 3$, so \myref{problem-group-of-order-pq-has-normal-subgroup-of-order-q} tells us that there exists a $N \lhd H$ with $|N| = 3$. Note also $[G:H] = 10$ which is not a multiple of 3, so \myref{problem-normal-subgroup-of-G-contains-all-sylow-p-subgroups} tells us that all Sylow 3-subgroups of $G$ are in $H$. But $N \lhd H$ means that $N$ is the unique Sylow 3-subgroup of $H$ and $G$ (\myref{corollary-sylow-subgroup-is-normal-if-it-is-unique}), so $N \lhd G$ (by the same corollary). Proceed to case 3, using $N$ in place of $H$.

        \item $|H| = 12$. Note $12 = 2^2 \times 3$. Now \myref{exercise-group-of-order-12-has-normal-subgroup-of-3-or-4} (later) tells us that there exists a normal subgroup of $H$ with order 3 or 4. Call that subgroup $N$. If $|N| = 3$ then it is a Sylow 3-subgroup; if $|N| = 4 = 2^2$ it is a Sylow 2-subgroup. As $N \lhd H$, thus $N$ is the unique Sylow 2- or 3- subgroup (\myref{corollary-sylow-subgroup-is-normal-if-it-is-unique}). Since $H \lhd G$, thus $H$ contains all Sylow 2- and 3-subgroups of $G$ (\myref{problem-normal-subgroup-of-G-contains-all-sylow-p-subgroups}), meaning $G$ has only one Sylow 2-subgroup or one Sylow 3-subgroup (or both), in particular $N$. Hence, $N \lhd G$ since a Sylow $p$-subgroup is unique if and only if it is normal (\myref{corollary-sylow-subgroup-is-normal-if-it-is-unique}). Proceed with case 3, using $N$ instead of $H$.

        \item $|H| \in \{2, 3, 4\}$. Since $H \lhd G$, thus $G/H$ is a group. Note $|G/H| \in \{15, 20, 30\}$. We claim that each of these cases produces a new normal subgroup of $G/H$ (call it $\bar{P}$) with order 5. This is proven for the case where $|G/H| = 30$ in \myref{problem-group-of-order-30-has-normal-subgroup-of-order-5}; the other two cases are for \myref{exercise-group-of-order-15-or-20-has-normal-subgroup-of-order-5} (later).

        Now \myref{problem-subgroup-of-quotient-group-is-quotient-group} tells us that $\bar{P}$ has the form $K/H$ where $K < G$ and $H \subseteq K$. Since $\bar{P} = K/H \lhd G/H$, thus for any $g \in G$ and $kH \in \bar{P}$ we have
        \[
            (gH)(kH)(g^{-1}H) = (gkg^{-1})H \in K/H,
        \]
        which means $gkg^{-1} \in K$. Therefore $K \lhd G$ by definition of normality.

        Observe that this means that
        \[
            |K| = |K/H||H| = |\bar{P}||H| = 5|H|,
        \]
        meaning $K$ is a normal subgroup of $G$ with an order that is a multiple of 5. Proceed to case 4, using $K$ in place of $H$.

        \item $|H|$ is a multiple of 5, meaning $H$ has a Sylow 5-subgroup. Note that there are $5-1=4$ non-identity elements in each Sylow 5-subgroup; therefore
        \[
            |H| \geq n_5(5-1) = 24
        \]
        which means that $|H| = 30$. By \myref{problem-group-of-order-30-has-normal-subgroup-of-order-5} again, such a group has only a unique Sylow 5-subgroup.  Note $5 \nmid [G:H]$, so \myref{problem-normal-subgroup-of-G-contains-all-sylow-p-subgroups} implies all Sylow 5-subgroups of $G$ are in $H$. However, right at the start, we concluded that there are 6 Sylow 5-subgroups in $G$, so $H$ must have 6 Sylow 5-subgroups, a contradiction.
    \end{enumerate}
    Hence, $H$ does not exist, and so $G$ is simple.
\end{proof}

\begin{exercise}\label{exercise-group-of-order-12-has-normal-subgroup-of-3-or-4}
    Prove that a group of order 12 either has a normal subgroup of order 3, or a normal subgroup of order 4, or both.
\end{exercise}

\begin{exercise}\label{exercise-group-of-order-15-or-20-has-normal-subgroup-of-order-5}
    Prove that a group of each of the following orders has a normal subgroup of order 5.
    \begin{partquestions}{\alph*}
        \item 15
        \item 20
    \end{partquestions}
\end{exercise}

\begin{corollary}\label{corollary-A5-is-simple}
    The group $\An5$ is simple.
\end{corollary}
\begin{proof}
    \myref{exercise-A5-has-two-distinct-subgroups-of-order-5} (later) gives two distinct subgroups of order 5. Since $|\An{5}| = 60 = 2^2 \times 3 \times 5$, thus subgroups of order 5 are Sylow 5-subgroups. Therefore $\An5$ is simple by \myref{thrm-group-of-order-60-with->1-sylow-5-subgroup-is-simple}.
\end{proof}
\begin{exercise}\label{exercise-A5-has-two-distinct-subgroups-of-order-5}
    Consider the permutation $\sigma = \begin{pmatrix}1&3&2&4&5\end{pmatrix}$.
    \begin{partquestions}{\roman*}
        \item Explain why $\sigma \in \An{5}$.
        \item Find the order of the subgroup $\langle \sigma \rangle$.
        \item Find another subgroup of $\An{5}$ with order 5.
    \end{partquestions}
\end{exercise}

We also state and prove a fairly obvious proposition.

\begin{proposition}\label{prop-An-stabilizer-of-i-is-isomorphic-to-A(n-1)}
    Let the integer $n \geq 3$. Let the set $\{1, 2, 3, \dots, n\}$ be denoted by $\mathcal{N}_n$. Suppose $\An{n}$ acts on $\mathcal{N}_n$ naturally. Then $\Stab{\An{n}}{r} \cong \An{n-1}$.
\end{proposition}
\begin{proof}
    We note that elements of $\Stab{\An{n}}{r}$ are permutations that fix $r$, thereby permuting the $n - 1$ other elements. Therefore the elements of $\Stab{\An{n}}{r}$ are even permutations on $n - 1$ elements, i.e. $\Stab{\An{n}}{r} \cong \An{n-1}$.
\end{proof}

With these results, we are ready to prove the main result of this section.

\begin{theorem}\label{thrm-An-is-simple-for-n>=5}
    The group $\An{n}$ is simple if $n \geq 5$.
\end{theorem}
\begin{proof}[Proof (see {\cite[Theorem 4.24]{dummit_foote_2004}})]
    We induct on $n$. For brevity, let $\mathcal{N}_n = \{1, 2, 3, \dots, n\}$. The base case of $n = 5$ is covered by \myref{corollary-A5-is-simple}. Assume that $\An{k-1}$ is simple for some $k \geq 6$; we will prove that $\An{k}$ is also simple.

    Let $G = \An{k}$ and, seeking a contradiction, assume that $G$ has a non-trivial proper normal subgroup $H$. Let $G$ act on $\mathcal{N}_{k}$ naturally; thus we see that $\Stab{G}{i} \leq G$ with $\Stab{G}{i} \cong \An{k-1}$ (\myref{prop-An-stabilizer-of-i-is-isomorphic-to-A(n-1)}) for any $i \in \mathcal{N}_k$. Note $\An{k-1}$ is simple by the induction hypothesis, so $\Stab{G}{i}$ is simple for each $i \in \mathcal{N}_{k}$.

    Suppose first that there is some non-identity $\pi \in H$ such that $\pi(i) = i$ for some $i \in \mathcal{N}_{k}$. This means that $\pi$ fixes $i$; thus $\pi \in H \cap \Stab{G}{i}$. Note that since $H \lhd G$ and $\Stab{G}{i} \leq G$ thus $H \cap \Stab{G}{i} \lhd \Stab{G}{i}$ by the Second Isomorphism Theorem (\myref{thrm-isomorphism-2}), statement 4. But as $\Stab{G}{i}$ is simple (and non-trivial) we must have $H \cap \Stab{G}{i} = \Stab{G}{i}$. Therefore $\Stab{G}{i} \subseteq H$ which means $\Stab{G}{i} \leq H$. Now by \myref{exercise-conjugate-of-stabilizer} (later), for any $\sigma \in G$, we know that $\sigma\Stab{G}{i}\sigma^{-1} = \Stab{G}{\sigma(i)}$. Therefore we see
    \[
        \sigma\Stab{G}{i}\sigma^{-1} \leq \sigma H\sigma^{-1} = H
    \]
    since $H \lhd G$. Thus, for any $j \in \mathcal{N}_{k+1}$, there exists $\sigma \in G$ where $\sigma(i) = j$ such that
    \[
        \sigma\Stab{G}{i}\sigma^{-1} = \Stab{G}{j} \leq H.
    \]

    Note that any $\lambda \in G$ may be written as a product of an even number of transpositions (\myref{thrm-parity-of-permutation}), say $2t$ transpositions. Thus, we may write $\lambda = \lambda_1\lambda_2\cdots\lambda_t$ where each $\lambda_i$ is a product of two transpositions. Now as $k \geq 5$, each $\lambda_i$ (which could at most consist of two disjoint cycles of 4 elements) must fix at least one element in $\mathcal{N}_{k}$, say $j$. That is, $\lambda_i \in \Stab{G}{j}$ for some $j \in \mathcal{N}_{k}$. Since $\lambda_i \in \Stab{G}{j} \leq H$ for some $j \in \mathcal{N}_{k}$, thus $\lambda_i \in H$. Hence $\lambda = \lambda_1\lambda_2\cdots\lambda_t \in H$. Therefore, any element in $G$ is also in $H$, meaning $G \subseteq H$, contradicting the fact that $H \lhd G$.

    We conclude that for any $\pi \in H$, if $\pi \neq \id$ then $\pi(i) \neq i$ for all $i \in \mathcal{N}_k$. The contrapositive of this statement is that if $\pi(i) = i$ for some $i \in \mathcal{N}_k$ then $\pi = \id$. Now suppose $\pi_1, \pi_2 \in H$ and $\pi_1(i) = \pi_2(i)$ for some $i \in \mathcal{N}_{k}$. Then $\pi_2^{-1}\pi_1(i) = i$, which implies $\pi_2^{-1}\pi_1 = \id$. Hence $\pi_1 = \pi_2$. Therefore, if $\pi_1, \pi_2 \in H$ and $\pi_1(i) = \pi_2(i)$ for some $i \in \mathcal{N}_{k}$, then $\pi_1 = \pi_2$.

    Now suppose a non-identity $\pi_1 \in H$ exists such that the cycle decomposition of $\pi_1$ contains a cycle of length of at least 3, say
    \[
        \pi_1 = \begin{pmatrix}a_1&a_2&a_3&\cdots\end{pmatrix} \begin{pmatrix}b_1&b_2&\cdots\end{pmatrix}\cdots
    \]
    where $a_1$, $a_2$, $a_3$, $b_1$, $b_2$, etc. are distinct (which is possible since $k \geq 5$). We note an element $\sigma \in G$ exists such that $\sigma(a_1) = a_1$, $\sigma(a_2) = a_2$, but $\sigma(a_3) \neq a_3$ because $k \geq 4$ (for example, the permutation $\begin{pmatrix}a_3 & a_4\end{pmatrix}$). Then \myref{exercise-conjugation-of-permutation-by-another} (later) tells us that
    \[
        \sigma\pi_1\sigma^{-1} = \begin{pmatrix}a_1&a_2&\sigma(a_3)&\cdots\end{pmatrix} \begin{pmatrix}\sigma(b_1)&\sigma(b_2)&\cdots\end{pmatrix}\cdots.
    \]
    Set $\pi_2 = \sigma\pi_1\sigma^{-1}$, which is clearly distinct from $\pi_1$. Then we see $\pi_1(a_1) = \pi_2(a_1) = a_2$, contrary to the above observation that $\pi_1(i) = \pi_2(i)$ for any $i \in \mathcal{N}_k$ implies $\pi_1 = \pi_2$. Therefore only 2-cycles can appear in the cycle decomposition of non-identity elements of $H$.

    Let $\pi_1 \in H$ be a non-identity element, so that
    \[
        \pi_1 = \begin{pmatrix}a_1&a_2\end{pmatrix} \begin{pmatrix}a_3&a_4\end{pmatrix} \begin{pmatrix}a_5&a_6\end{pmatrix}\cdots
    \]
    where each $a_i$ is distinct (note that $k \geq 6$ is used above). Consider the permutation $\sigma = \begin{pmatrix}a_1&a_2\end{pmatrix} \begin{pmatrix}a_3&a_5\end{pmatrix}$, which is in $G$ since it is made up of 2 transpositions. Then \myref{exercise-conjugation-of-permutation-by-another} again gives
    \[
        \sigma\pi_1\sigma^{-1} = \begin{pmatrix}a_1&a_2\end{pmatrix} \begin{pmatrix}a_5&a_4\end{pmatrix} \begin{pmatrix}a_3&a_6\end{pmatrix}\cdots.
    \]
    Setting $\pi_2 = \sigma\pi_1\sigma^{-1}$ again gives two distinct permutations $\pi_1$ and $\pi_2$ where $\pi_1(a_1) = \pi_2(a_1) = a_2$, again contrary to the above observation.

    We conclude that such a non-trivial proper normal subgroup $H$ of $\An{k}$ cannot exist. Thus, $\An{k-1}$ being simple implies that $\An{k}$ is also simple.

    By mathematical induction, $\An{n}$ is simple for all $n \geq 5$.
\end{proof}

\begin{exercise}\label{exercise-conjugate-of-stabilizer}
    Let $S$ be a non-empty set and let $G \leq \Sym{S}$ act on $S$. Show that $\sigma\Stab{G}{x}\sigma^{-1} = \Stab{G}{\sigma(x)}$ for any $\sigma \in G$ and $x \in S$.
\end{exercise}

\begin{exercise}\label{exercise-conjugation-of-permutation-by-another}
    Let $\sigma, \pi \in \Sn{n}$. Suppose $\sigma$ has cycle decomposition
    \[
        \begin{pmatrix}a_1&a_2&\cdots&a_{k_1}\end{pmatrix} \begin{pmatrix}b_1&b_2&\cdots&b_{k_2}\end{pmatrix}\cdots,
    \]
    where $a_1, a_2, \dots, a_{k_1}, b_1, b_2, \dots, b_{k_2}, \dots$ are all distinct. Show that
    \[
        \pi\sigma\pi^{-1} = \begin{pmatrix}\pi(a_1)&\cdots&\pi(a_{k_1})\end{pmatrix} \begin{pmatrix}\pi(b_1)&\cdots&\pi(b_{k_2})\end{pmatrix}\cdots,
    \]
    that is, $\pi\sigma\pi^{-1}$ is obtained from $\sigma$ by replacing each entry $i$ by $\pi(i)$.
\end{exercise}

\begin{corollary}
    The group $\An{n}$ is simple for $n \geq 3$ and $n \neq 4$.
\end{corollary}
\begin{proof}
    We note $\An3$ has order $\frac{3!}{2} = 3$ which is prime, so $\An3 \cong \Cn3$ which is simple. Also $\An{n}$ is simple for $n \geq 5$ by \myref{thrm-An-is-simple-for-n>=5}.
\end{proof}

We note that $\An4$ is non-simple by the solution of \myref{problem-S4-composition-series}, in which we found that $\An4$ has a unique composition series of
\[
    1 \lhd \Cn2 \lhd \mathrm{V} \lhd \An4
\]
up to isomorphism.

\section{Groups of Lie Type}
We briefly mention groups of Lie type; we will not prove any significant results here.

Groups of Lie (pronounced ``lee'') type\index{groups of Lie type} usually refers to finite groups that are closely related to the group of rational points of a reductive linear algebraic group with values in a finite field. We will cover finite fields in part III. We briefly mention these groups here.

The list below, taken from \cite{wikipedia_list-of-simple-groups}, is a list of the families of simple groups of Lie type. In what follows, $n$ is a positive integer and $q$ is a positive power of a prime number $p$.
\begin{itemize}
    \item \term{Classical Chevalley groups}\index{Chevalley groups!classical}: there are 4 families of simple groups.
    \begin{itemize}
        \item $A_n(q)$, except for $A_1(2)$ and $A_1(3)$. There are several duplicates, which are
        \begin{itemize}
            \item $A_1(4) \cong A_1(5) \cong \An{5}$;
            \item $A_1(7) \cong A_2(2)$;
            \item $A_1(9) \cong \An{6}$; and
            \item $A_3(2) \cong \An{8}$.
        \end{itemize}
        We note that $\An{n}$ is not the same as $A_n(q)$. We distinguish between the alternating group of degree $n$ ($\An{n}$) and the groups of Lie type $A_n(q)$ by letting the latter be in italics and the former be in `normal' font.

        \item $B_n(q)$ for $n > 1$, except for $B_2(2)$. There are several duplicates, which are
        \begin{itemize}
            \item $B_n(2^m) \cong C_n(2^m)$; and
            \item $B_2(3) \cong {^2A_3(2^2)} = {^2A_3(4)}$, where ${^2A_3(4)}$ is a classical Steinberg group.
        \end{itemize}
        \item $C_n(q)$ for $n > 2$. The only duplicate is $C_n(2^m) \cong B_n(2^m)$ mentioned earlier.
        \item $D_n(q)$ for $n > 3$.
    \end{itemize}

    \item \term{Exceptional Chevalley groups}\index{Chevalley groups!exceptional}: there are 5 families of such groups.
    \begin{itemize}
        \item $E_6(q)$;
        \item $E_7(q)$;
        \item $E_8(q)$;
        \item $F_4(q)$; and
        \item $G_2(q)$, except for $G_2(2)$.
    \end{itemize}

    \item \term{Classical Steinberg groups}\index{Steinberg groups!classical}: there are 2 families of simple groups.
    \begin{itemize}
        \item ${^2A_n(q^2)}$ for $n > 1$, except for ${^2A_2(2^2)} = {^2A_2(4)}$. The only duplicate is ${^2A_3(2^2)} \cong B_2(3)$ mentioned earlier.
        \item ${^2D_n(q^2)}$ for $n > 3$.
    \end{itemize}

    \item \term{Exceptional Steinberg groups}\index{Steinberg groups!exceptional}: there are 2 families of simple groups.
    \begin{itemize}
        \item ${^2E_6(q^2)}$; and
        \item ${^3D_4(q^3)}$.
    \end{itemize}

    \item \term{Suzuki groups}\index{Suzuki groups}: there is 1 family of simple groups, which is ${^2B_2(q)}$ where $q = 2^{2n+1}$ and $n \geq 1$. Such a group has order $q^2(q^2+1)(q-1)$.

    \item \term{Ree groups}\index{Ree groups}: there are 2 families of simple groups.
    \begin{itemize}
        \item $^2F_4(q)$ where $q = 2^{2n+1}$ and $n \geq 1$. The order of such a group is $q^{12}(q^6+1)(q^4-1)(q^3+1)(q-1)$.
        \item $^2G_2(q)$ where $q = 3^{2n+1}$ and $n \geq 1$. The order of such a group is $q^3(q^3+1)(q-1)$.
    \end{itemize}
\end{itemize}

There is also the \term{Tits group}\index{Tits group}, $^2F_4(2)'$, with an order of $17,971,200$. It is the commutator subgroup of $^2F_4(2)$, which is a Lie group but not a simple group. The fact that $^2F_4(2)'$ is linked to Ree groups makes most authors consider it not a sporadic group (see below).

\section{The Sporadic Groups}
Along with the 18 infinite families of simple groups, there are also 26 sporadic simple groups\index{sporadic group} that do not fall within the families (27 if the Tits group is considered a sporadic group).

We first list four categories of sporadic groups.
\begin{table}[h]
    \centering
    \begin{tabular}{|l|l|l|}
        \hline
        \textbf{Name} & \textbf{Symbol} & \textbf{Order} \\ \hline
        \multirow{5}{*}{\term{Mathieu Groups}\index{Mathieu groups}} & $\mathrm{M}_{11}$ & 7,290 \\ \cline{2-3}
        & $\mathrm{M}_{12}$ & 95,040 \\ \cline{2-3}
        & $\mathrm{M}_{22}$ & 443,520 \\ \cline{2-3}
        & $\mathrm{M}_{23}$ & 10,200,960 \\ \cline{2-3}
        & $\mathrm{M}_{24}$ & 244,823,040 \\ \hline
        \multirow{4}{*}{\term{Janko Groups}\index{Janko groups}} & $\mathrm{J}_1$ & 175,560 \\ \cline{2-3}
        & $\mathrm{J}_2$ & 604,800 \\ \cline{2-3}
        & $\mathrm{J}_3$ & 50,232,960 \\ \cline{2-3}
        & $\mathrm{J}_4$ & 86,775,571,046,077,562,880 \\ \hline
        \multirow{3}{*}{\term{Conway Groups}\index{Conway groups}} & $\mathrm{Co}_3$ & 495,766,656,000 \\ \cline{2-3}
        & $\mathrm{Co}_2$ & 42,305,421,312,000 \\ \cline{2-3}
        & $\mathrm{Co}_1$ & 4,157,776,806,543,360,000 \\ \hline
        \multirow{3}{*}{\term{Fischer Groups}\index{Fischer groups}} & $\mathrm{Fi}_{22}$ & 64,561,751,654,400 \\ \cline{2-3}
        & $\mathrm{Fi}_{23}$ & 4,089,470,473,293,004,800 \\ \cline{2-3}
        & $\mathrm{Fi}_{24}$ & 1,255,205,709,190,661,721,292,800 \\ \hline
    \end{tabular}
\end{table}



More sporadic groups are listed below.
\begin{table}[h]
    \centering
    \begin{tabular}{|l|l|l|}
        \hline
        \textbf{Group}        & \textbf{Symbol} & \textbf{Order}  \\ \hline
        \term{Higman-Sims group}\index{Higman-Sims group}     & HS              & 44,352,000      \\ \hline
        \term{McLaughlin group}\index{McLaughlin group}      & McL             & 898,128,000     \\ \hline
        \term{Held group}\index{Held group}            & He              & 4,030,387,200   \\ \hline
        \term{Rudvalis group}\index{Rudvalis group}        & Ru              & 145,926,144,000 \\ \hline
        \term{Suzuki sporadic group}\index{Suzuki sporadic group} & Suz             & 448,345,497,600 \\ \hline
        \term{O'Nan group}\index{O'Nan group}           & $\mathrm{O'N}$  & 460,815,505,920 \\ \hline
        \term{Harada-Norton group}\index{Harada-Norton group}   & HN              & 273,030,912,000,000    \\ \hline
        \term{Lyons group}\index{Lyons group}           & Ly              & 51,765,179,004,000,000 \\ \hline
        \term{Thompson group}\index{Thompson group}        & Th              & 90,745,943,887,872,000 \\ \hline
    \end{tabular}
\end{table}

The remaining 2 sporadic groups are special in that they have extremely large order.
\begin{itemize}
    \item The \term{Baby Monster group}\index{Baby Monster group}, usually denoted $\mathrm{B}$, has order
    \begin{align*}
        &2^{41} \times 3^{13} \times 5^6 \times 7^2 \times 11 \times 13 \times 17 \times 19 \times 23 \times 31 \times 47\\
        &= 4,154,781,481,226,426,191,177,580,544,000,000.
    \end{align*}
    \item The \term{Monster group}\index{Monster group}, usually denoted $\mathrm{M}$, has order $2^{46} \times 3^{20} \times 5^9 \times 7^6 \times 11^{2} \times 13^3 \times 17 \times 19 \times 23 \times 29 \times 31 \times 41 \times 47 \times 59 \times 71$ which equals 808,017,424,794,512,875,886,459,904,961,710,757,005,754,368,\linebreak000,000,000. It is the largest sporadic group.
\end{itemize}

\section{The Classification Theorem of Finite Simple Groups}
One might rightly wonder what the importance of listing out all of these different types of simple groups are. It turns out that, amazingly, that these results provide a complete classification of what a finite simple group can really be. This is captured in the Classification Theorem of Finite Simple Groups\index{Classification Theorem of Finite Simple Groups}, which is sometimes called the enormous theorem\index{Enormous Theorem}.

\begin{theorem}[Classification Theorem]
    Every finite simple group is isomorphic to either
    \begin{itemize}
        \item a cyclic group of prime order;
        \item an alternating group with degree of at least 5;
        \item a group in the 16 infinite families of groups of Lie type, or the Tits group; or
        \item one of 26 sporadic groups.
    \end{itemize}
\end{theorem}

The proof of this theorem required tens of thousands of pages in hundreds of articles, written by a large number of authors that were  published mostly between 1955 to 2004. The longest paper, and the last paper needed to fill in the gap for quasithin groups, was published in 2004 by Aschbacher and Smith and spanned in a 1221 pages. But after all that work, mathematicians had a complete classification of all finite simple groups.


%=========================================
\appendix
\section{Galois Theory}
\begin{questions}
    \item \begin{partquestions}{\roman*}
        \item We note that $[\C:\R] = 2$ since $\C = \R(i)$ and $i$ is a zero of the irreducible polynomial $x^2 + 1$ over $\R$. Therefore, by \myref{thrm-order-of-galois-group-is-degree-of-field-extension}, we see $|\Gal{\C/\R}| = [\C:\R] = 2$.
        
        \item Certainly $\id \in \Gal{\C/\R}$. We claim that the other element in $\Gal{\C/\R}$ is $\phi: \C \to \C$ where $\phi(a+bi) = a-bi$ for all $a+bi \in \C$. We first need to check that $\phi$ is an automorphism.
        \begin{itemize}
            \item \textbf{Homomorphism}: One sees clearly for any $a+bi, c+di \in \C$ that
            \begin{align*}
                \phi((a+bi)+(c+di)) &= \phi((a+c)+(b+d)i)\\
                &= (a+c)-(b+d)i\\
                &= (a-bi) + (c-di)\\
                &= \phi(a+bi) + \phi(c+di)
            \end{align*}
            and
            \begin{align*}
                \phi((a+bi)(c+di)) &= \phi((ac-bd) + (ad+bc)i)\\
                &= (ac-bd) - (ad+bc)i\\
                &= (a-bi)(c-di)\\
                &= \phi(a+bi)\phi(c+di)
            \end{align*}
            which proves that $\phi$ is indeed a homomorphism.

            \item \textbf{Injective}: Since $\phi$ is non-trivial it is thus injective by \myref{thrm-homomorphism-from-field-is-injective-or-trivial}.
            
            \item \textbf{Surjective}: For any $a + bi \in \C$ we note that $a - bi \in \C$ and that $\phi(a - bi) = a - (-b)i = a+bi$, proving that $\phi$ is surjective.
        \end{itemize}
        Therefore $\phi$ is a bijective homomorphism from $\C$ to $\C$, i.e. an automorphism. One also sees that $\phi(r) = r$ for all $r \in \R$, so $\phi$ fixes $\R$. Thus $\phi \in \Gal{\C/\R}$. But as $\Gal{\C/\R}$ has order 2, thus $\id$ and $\phi$ are the only two elements in $\Gal{\C/\R}$.
    \end{partquestions}
\end{questions}


\chapter{Problem Solutions}
\section{Introduction to Rings}
\subsection*{Exercises}
\begin{questions}
    \item We prove the ring axioms.
    \begin{itemize}
        \item \textbf{Addition-Abelian}: $(\{0\}, +)$ is an abelian group since this is just the trivial group.
        \item \textbf{Multiplication-Semigroup}: $(\{0\}, \cdot)$ is an abelian group since this is, again, just the trivial group. So $(\{0\}, \cdot)$ is a semigroup.
        \item \textbf{Distributive}: We know $+$ and $\cdot$ distribute.
    \end{itemize}
    Hence $(\{0\}, +, \cdot)$ is a commutative ring with identity.
\end{questions}

\subsection*{Problems}
No problems.

\section{Basics of Rings}
\begin{questions}
    \item We note the following.
    \begin{itemize}
        \item \textbf{Addition-Abelian}: $(\Z, +)$ is an abelian group.
        \item \textbf{Multiplication-Semigroup}: $(\Z, \times)$ is a semigroup since
        \begin{itemize}
            \item multiplying two integers always results in an integer, so $\Z$ is closed under $\times$; and
            \item $\times$ is associative.
        \end{itemize}
        \item \textbf{Distributive}: We know $+$ and $\times$ distribute.
    \end{itemize}
    Hence $(\Z, +, \times)$ is a ring.

    \item Consider $(-a)(-b) + (-ab)$ and note
    \begin{align*}
        &(-a)(-b) + (-ab)\\
        &= (-a)(-b) + (-a)b & (\text{\myref{prop-product-of-element-and-additive-inverse-is-additive-inverse-of-product}})\\
        &= (-a)(-b + b) & (\text{by \textbf{Distributive} axiom})\\
        &= (-a)0\\
        &= 0 & (\myref{prop-multiplying-by-zero-is-zero})
    \end{align*}
    which means $(-a)(-b) = -(-ab) = ab$ as required.

    \item The ring $\Mn{2}{\mathbb{R}}$ indeed has zero divisors, as $\begin{pmatrix}0&1\\0&0\end{pmatrix} \neq \begin{pmatrix}0&0\\0&0\end{pmatrix}$ but $\begin{pmatrix}0&1\\0&0\end{pmatrix}^2 = \begin{pmatrix}0&0\\0&0\end{pmatrix}$ which means that $\begin{pmatrix}0&1\\0&0\end{pmatrix}$ is a zero divisor.

    \item \begin{partquestions}{\alph*}
        \item $\Z$ is not a field. Note that the multiplicative inverse of 2 is $\frac12$ which is not an integer. Hence not all non-zero elements in $\Z$ has a multiplicative inverse, meaning that not all non-zero elements are units.

        \item $\Q$ is a field. Note for any rational number $\frac ab$ (where $b \neq 0$) it has an inverse of $\frac ba$. Thus any non-zero rational number is a unit, which means $\Q$ is a division ring. Since $\Q$ is also a commutative ring, therefore $\Q$ is a field.
    \end{partquestions}

    \item Let $u$ and $v$ be units, meaning that $u^{-1}$ and $v^{-1}$ exist. Then one sees that $(uv)(v^{-1}u^{-1}) = (v^{-1}u^{-1})(uv) = 1$, which means that $uv$ is also a unit.

    \item We first show that $(R, +) \leq (\Mn{2}{\R}, +)$.
    \begin{itemize}
        \item Clearly the identity of $(\Mn{2}{\R}, +)$, the zero matrix $\begin{pmatrix}0&0\\0&0\end{pmatrix}$, is inside $R$.
        \item Consider $\begin{pmatrix}a&a\\a&a\end{pmatrix}, \begin{pmatrix}b&b\\b&b\end{pmatrix} \in R$. The additive inverse of the matrix $\begin{pmatrix}b&b\\b&b\end{pmatrix}$ is the matrix $\begin{pmatrix}-b&-b\\-b&-b\end{pmatrix}$, and so their sum is
        \[
            \begin{pmatrix}a&a\\a&a\end{pmatrix} + \begin{pmatrix}-b&-b\\-b&-b\end{pmatrix} = \begin{pmatrix}a-b&a-b\\a-b&a-b\end{pmatrix} \in R
        \]
        which means $R$ is closed under addition.
    \end{itemize}
    Hence $(R, +) \leq (\Mn{2}{\R}, +)$ by subgroup test.

    We now show that $R$ is closed under multiplication. Some calculation yields that
    \[
        \begin{pmatrix}a&a\\a&a\end{pmatrix}\begin{pmatrix}b&b\\b&b\end{pmatrix} = \begin{pmatrix}2ab&2ab\\2ab&2ab\end{pmatrix}
    \]
    which is clearly in $R$. Therefore $R$ is a subring of $\Mn{2}{\R}$.
\end{questions}

\section{Integral Domains}
\begin{questions}
    \item To find a $a+bi \in \Z_5[i]$ such that there exists a $c+di \in \Z_5[i]$ where $(a+bi)(c+di) = 0$ but both $a+bi$ and $c+di$ are non-zero. Expanding $(a+bi)(c+di)$ yields $(ac-bd)+(ad+bc)i = 0$. Therefore we must have $ac-bd = 0$ and $ad+bc = 0$. For simplicity let's choose $a=c=1$. Using second equation we have $d+b = 0$ which means $d = -b$. Hence $(1 - b(-b))+(-b + b)i = 1+b^2 = 0$. Therefore choosing $b = 2$ would make it work. Therefore one solution is $a = 1, b = 2, c = 1, d = -2 = 3$; i.e. two zero divisors are $1+2i$ and $1+3i$.
    
    \item Take $w, z \in \Z[i]$ such that $w \neq 0$ and $wz = 0$. We want to show that $z = 0$. Let $z = a+bi$ and $w = c+di$. Since $w \neq 0$ we must have $c^2+d^2 \neq 0$. Now
    \[
        (a+bi)(c+di) = (ac-bd)+(ad+bc)i = 0
    \]
    which means $ac - bd = 0$ and $ad+bc = 0$. Multiplying first equation by $d$ yields $acd - bd^2 = 0$; multiplying second equation by $c$ yields $acd + bc^2 = 0$. Now summing them up yields $bc^2+bd^2 = b(c^2+d^2) = 0$ which hence means $b = 0$ since $c^2+d^2 \neq 0$. Therefore $ac - 0d = 0$ implies $ac = 0$ and $ad+0c = 0$ implies $ad = 0$. Squaring both equations and adding them up yields $a^2c^2 + a^2d^2 = a^2(c^2+d^2) = 0$ which hence means $a^2$ (and thus $a$) is zero. Therefore we have shown $z = 0$, meaning that there are no zero divisors in $\Z[i]$, so $\Z[i]$ is an integral domain.

    \item \begin{partquestions}{\alph*}
        \item Note that multiplication is commutative with identity $1 = 1 + 0\sqrt{n} \in R$. We just need to show that there are no zero divisors in $R$.
        
        Take $a+b\sqrt n, c+d\sqrt n \in R$ such that $a+b\sqrt n \neq 0$ but $(a+b\sqrt n)(c+d\sqrt n) = 0$. We want to show $c = d = 0$. Consider
        \[
            \left((a+b\sqrt n)(\underbrace{a-b\sqrt n}_{\neq 0})\right)\left((c+d\sqrt n)(\underbrace{c-d\sqrt n}_{\neq 0})\right) = 0.
        \]
        This means that $(a^2-nb^2)(c^2-nd^2) = 0$, so either $a^2-nb^2 = 0$ or $c^2-nd^2 = 0$.

        Now if $n < 0$ then clearly we have to have $c = d = 0$. Otherwise we have $a = b\sqrt n$ or $c = d\sqrt n$. But $\sqrt n$ is not an integer, so the only way for equality is if $c = d = 0$. Thus $\Z[\sqrt n]$ has no zero divisors, meaning $\Z[\sqrt n]$ is an integral domain.

        \item Consider $2 + \sqrt 2 \in \Z[\sqrt 2]$. Its multiplicative inverse is
        \begin{align*}
            \frac{1}{2+\sqrt2} &= \frac{2-\sqrt2}{(2+\sqrt2)(2-\sqrt2)}\\
            &= \frac{2-\sqrt2}{4-2}\\
            &= 1 - \frac12\sqrt2 \notin \Z[\sqrt2].
        \end{align*}
        This means that $2+\sqrt2$, a non-zero element in $\Z[\sqrt2]$, does not have an inverse in $\Z[\sqrt2]$. Therefore $\Z[\sqrt2]$ is not a field, meaning $R$ is not a field in the general case.
    \end{partquestions}

    \newpage

    \item For brevity let O$ = \begin{pmatrix}0&0\\0&0\end{pmatrix}$, I$ = \begin{pmatrix}1&0\\0&1\end{pmatrix}$, A$ = \begin{pmatrix}1&1\\1&0\end{pmatrix}$, and B$ = \begin{pmatrix}0&1\\1&1\end{pmatrix}$.
    
    \begin{partquestions}{\roman*}
        \item Clearly one sees that $R$ is a subset of $\Mn{2}{\Z_2}$.
        \begin{itemize}
            \item We show $(R, +)\leq(\Mn{2}{\Z_2},+)$.
            \begin{table}[h]
                \centering
                \begin{tabular}{|l|l|l|l|l|}
                    \hline
                    \textbf{+} & \textbf{O} & \textbf{I} & \textbf{A} & \textbf{B} \\ \hline
                    \textbf{O} & O          & I          & A          & B          \\ \hline
                    \textbf{I} & I          & O          & B          & A          \\ \hline
                    \textbf{A} & A          & B          & O          & I          \\ \hline
                    \textbf{B} & B          & A          & I          & O          \\ \hline
                \end{tabular}
            \end{table}
            
            From the Cayley table, clearly the identity of the ring $\Mn{2}{\Z_2}$ is in $R$ and $R$ is closed under addition. Hence $(R, +)\leq(\Mn{2}{\Z_2},+)$

            \item We show $R$ is closed under multiplication.
            \begin{table}[h]
                \centering
                \begin{tabular}{|l|l|l|l|l|}
                    \hline
                    $\boldsymbol{\cdot}$ & \textbf{O} & \textbf{I} & \textbf{A} & \textbf{B} \\ \hline
                    \textbf{O}           & O          & O          & O          & O          \\ \hline
                    \textbf{I}           & O          & I          & A          & B          \\ \hline
                    \textbf{A}           & O          & A          & B          & I          \\ \hline
                    \textbf{B}           & O          & B          & I          & A          \\ \hline
                \end{tabular}
            \end{table}
            
            From the Cayley table, clearly $R$ is closed under multiplication.
        \end{itemize}
        Therefore $R$ is a subring of $\Mn{2}{\Z_2}$.

        \item Since $R$ is a subring of $\Mn{2}{\Z_2}$, it is a ring. Furthermore, by the Cayley table of $(R, \cdot)$, we see that $R$ is commutative with identity I. Finally, one sees that $\mathrm{A}^{-1} = \mathrm{B}$, $\mathrm{B}^{-1} = \mathrm{A}$, and $\mathrm{I}^{-1} = \mathrm{I}$. Therefore all non-zero elements of $R$ have inverses. Hence $R$ is a field.
    \end{partquestions}
\end{questions}

\section{Ideals and Quotient Rings}
\begin{questions}
    \item Note that $36 = 2^2 \times 3^2$. So
    \begin{align*}
        \Ann{\Z_{36}}{\{15\}} &= \{r \in \Z_{36} \vert 15r = 0\}\\
        &= \{r \in \Z_{15} \vert 3(5r) = 0\}\\
        &= \{r \in \Z_{15} \vert r \text{ is a multiple of }2^2\times3 = 12\}\\
        &= \{0,12,24\}.
    \end{align*}

    \item We first show that $S$ is a subring of $\Z[i]$.
    \begin{itemize}
        \item The identity of $\Z[i]$ is $0 = 0 + 2(0)i \in S$.
        \item For any $a+2bi, c+2di \in S$, clearly $a+2bi + (-(c + 2di)) = (a-c) + 2(b-d)i \in S$.
        \item For any $a+2bi, c+2di \in S$, one sees that
        \begin{align*}
            (a+2bi)(c+2di) &= ac + 2adi + 2bci + 4bdi^2\\
            &= (ac-4bd) + 2(ad+bc)i\\
            &\in S.
        \end{align*}
    \end{itemize}
    Therefore $S$ is a subring of $\Z[i]$.

    We now show that $S$ is not an ideal of $\Z[i]$. Consider $1+2i \in \S$ and $1+i \in \Z[i]$. Then
    \begin{align*}
        (1+2i)(1+i) &= 1+i+2i+2i^2\\
        &= -1 + 3i\\
        &\notin S
    \end{align*}
    so there exists a $s \in S$ and a $r \in \Z[i]$ such that $rs\notin S$, meaning that $S$ is not a left ideal (and hence is not an ideal).

    \item We consider the test for ideal (\myref{thrm-test-for-ideal}).
    \begin{itemize}
        \item Note that $\begin{pmatrix}0&0\\0&0\end{pmatrix}=\begin{pmatrix}2(0)&2(0)\\2(0)&(0)\end{pmatrix}$ is in $I$ so $I$ is non-empty.
        \item $\begin{pmatrix}2a&2b\\2c&2d\end{pmatrix}-\begin{pmatrix}2e&2f\\2g&2h\end{pmatrix} = \begin{pmatrix}2(a-e)&2(b-f)\\2(c-g)&2(d-h)\end{pmatrix} \in I$.
        \item To show left ideal, take $\begin{pmatrix}2a&2b\\2c&2d\end{pmatrix} \in I$ and $\begin{pmatrix}e&f\\g&h\end{pmatrix} \in \Mn{2}{\Z}$. Then
        \begin{align*}
            \begin{pmatrix}2a&2b\\2c&2d\end{pmatrix}\begin{pmatrix}e&f\\g&h\end{pmatrix} &= \begin{pmatrix}2ae+2bg&2af+2bh\\2ce+2dg&2cf+2dh\end{pmatrix}\\
            &= \begin{pmatrix}2(ae+bg)&2(af+bh)\\2(ce+dg)&2(cf+dh)\end{pmatrix}\\
            &\in I
        \end{align*}
        so $I$ is a left ideal of $\Mn{2}{\Z}$.
        \item To show right ideal, take $\begin{pmatrix}a&b\\c&d\end{pmatrix} \in \Mn{2}{\Z}$ and $\begin{pmatrix}2e&2f\\2g&2h\end{pmatrix} \in I$. Then
        \begin{align*}
            \begin{pmatrix}a&b\\c&d\end{pmatrix}\begin{pmatrix}2e&2f\\2g&2h\end{pmatrix} &= \begin{pmatrix}2ae+2bg&2af+2bh\\2ce+2dg&2cf+2dh\end{pmatrix}\\
            &= \begin{pmatrix}2(ae+bg)&2(af+bh)\\2(ce+dg)&2(cf+dh)\end{pmatrix}\\
            &\in I
        \end{align*}
        so $I$ is a right ideal of $\Mn{2}{\Z}$.
    \end{itemize}
    Therefore by the test for ideal we have $I$ is an ideal of $\Mn{2}{\Z}$.

    \item \begin{partquestions}{\alph*}
        \item Suppose $I$ is not the trivial ring; we want to show that $I = R$. Since $I$ is non-trivial there there exists a non-zero element $a$ in $I$. Note that $a^{-1}$ exists since $R$ is a field, so $a$ is a unit. By \myref{exercise-ideal-containing-1-is-whole-ring} this means $I = R$. Note that $\{0\} = \princ{0}$ and $R = \princ{1}$ by \myref{exercise-trivial-ideal-and-whole-ring-are-principal-ideals}, so $R$ is indeed a PID.

        \item Take a non-zero $x \in R$ and note that $\princ{x}$ is a non-trivial ideal. Since there are no proper ideals in $R$, thus $\princ{x} = R$. This means that $1 \in \princ{x}$ (since $\princ{x} = R$ is a ring with identity), meaning that there exists an element $r \in R$ such that $xr = 1$. Therefore $x$ is a unit.
        
        Since $x$ is an arbitrary non-zero element in $R$, this thus shows that all non-zero elements of the ring $R$ are units, meaning $R$ is a division ring.

        Finally, because $R$ is commutative, thus $R$ is a field.
    \end{partquestions}

    \item \begin{partquestions}{\alph*}
        \item Suppose $r \in \sqrt{\sqrt{I}}$, meaning that $r^m \in \sqrt{I}$ for some positive integer $m$, further meaning that $(r^m)^n \in I$ for some positive integer $n$. Note $(r^m)^n = r^{mn} \in I$, so $r \in \sqrt{I}$. Therefore $\sqrt{\sqrt{I}} \subseteq \sqrt{I}$.
        
        Now suppose $r \in \sqrt{I}$, meaning that $r^n \in I$ for some positive integer $n$. Note that $r = r^1 \in \sqrt{I}$, so $r \in \sqrt{\sqrt{I}}$. Hence $\sqrt{I} \subseteq \sqrt{\sqrt{I}}$.

        Therefore, since $\sqrt{\sqrt{I}} \subseteq \sqrt{I}$ and $\sqrt{I} \subseteq \sqrt{\sqrt{I}}$, thus $\sqrt{\sqrt{I}} = \sqrt{I}$.

        \item Suppose $r \in \sqrt{I\cap J}$, so $r^n \in I \cap J$ for some positive integer $n$. This means that $r^n \in I$ and $r^n \in J$. Hence $r \in \sqrt{I}$ and $r \in \sqrt{J}$ by definition of the radical, so $r \in \sqrt{I}\cap\sqrt{J}$. Thus $\sqrt{I\cap J} \subseteq \sqrt{I}\cap\sqrt{J}$.
        
        Now suppose $r \in \sqrt{I}\cap\sqrt{J}$, meaning that $r \in \sqrt{I}$ and $r \in \sqrt{J}$. Thus $r^m \in I$ and $r^n \in J$ for some positive integers $m$ and $n$. Note that
        \[
            (\underbrace{r^m}_{\text{In }I})^n \in I \text{ and } (\underbrace{r^n}_{\text{In }J})^m \in J
        \]
        so $r^{mn} \in I$ and $r^{mn} \in J$, meaning $r^{mn} \in I \cap J$. Thus $r \in \sqrt{I \cap J}$, showing that $\sqrt{I}\cap\sqrt{J} \subseteq \sqrt{I\cap J}$.

        Therefore $\sqrt{I}\cap\sqrt{J} = \sqrt{I\cap J}$.
    \end{partquestions}

    \item \begin{partquestions}{\alph*}
        \item Suppose $a \in m\Z\cap n\Z$. Thus $a \in m\Z$ and $a \in n\Z$, meaning that $a = mx = ny$ for some integers $x$ and $y$. Therefore $a = \lcm(m,n)z = lz$ for some integer $z$, meaning $a \in l\Z$. Hence $m\Z \cap n\Z \subseteq l\Z$.
        
        Now suppose $a \in l\Z$, so $a = lx$ for some integer $x$. Write $l = m\alpha = n\beta$ for some integers $\alpha$ and $\beta$. Note that
        \begin{align*}
            a &= (m\alpha)x = m(\alpha x) \in m\Z\\
            a &= (n\beta)x = n(\beta x) \in n\Z
        \end{align*}
        so $a \in m\Z \cap n\Z$. Thus $l\Z \subseteq m\Z \cap n\Z$.

        Therefore $m\Z\cap n\Z = l\Z$.

        \item Suppose $a \in m\Z + n\Z$, meaning that there exist integers $x$ and $y$ such that $a = mx + ny$. By definition of the GCD, write $m = d\alpha$ and $n = d\beta$ for some integers $\alpha$ and $\beta$. Hence
        \begin{align*}
            a &= (d\alpha)x + (d\beta)y\\
            &= d(\alpha x + \beta y)\\
            &\in d\Z
        \end{align*}
        so $m\Z + n\Z \subseteq d\Z$.

        On the other hand, suppose $a \in d\Z$, meaning $a = dt$ for some integer $t$. By B\'{e}zout's Lemma (\myref{lemma-bezout}), we may write $d = mx + ny$ for some integers $x$ and $y$. Hence
        \begin{align*}
            a &= (mx + ny)t\\
            &= m(xt) + n(yt)\\
            &\in m\Z + n\Z
        \end{align*}
        which means $d\Z \subseteq m\Z + n\Z$.

        Therefore $m\Z + n\Z = d\Z$.
    \end{partquestions}

    \item Let $r \in R$, and suppose $x = r + \Nilr{R} \in R/\Nilr{R}$ is nilpotent, i.e. there is a positive integer $n$ such that
    \[
        x^n = (r + \Nilr{R})^n = r^n + \Nilr{R} = 0 + \Nilr{R}.
    \]
    Coset Equality (\myref{lemma-coset-equality}) thus tells us that $r^n \in \Nilr{R}$. Note that $\Nilr{R}$ contains all the nilpotents of $R$. Thus $r^n$ is a nilpotent of $R$, i.e. there exists a positive integer $m$ such that $(r^n)^m = 0$. But clearly $(r^n)^m = r^{mn} = 0$, so $r$ is nilpotent, meaning $r \in \Nilr{R}$. Hence $x = r + \Nilr{R} = 0 + \Nilr{R}$, meaning that the only nilpotent of $R/\Nilr{R}$ is the zero element. Therefore $R/\Nilr{R}$ has no non-zero nilpotents.

    \item Suppose $R$ is a PID and $I$ is a non-zero prime ideal. Let $J$ be an ideal such that $I \subseteq J \subseteq R$. Since $R$ is a PID, write $I = \princ{a}$ and $J = \princ{b}$ for some elements $a$ and $b$ in $R$. Note $a \in \princ{a} = I \subseteq J = \princ{b}$, so there exists an $r \in R$ such that $a = rb$. Now since $a = rb \in \princ{a} = I$ and $I$ is prime, therefore $r \in I$ or $b \in I$.
    \begin{itemize}
        \item If $r \in I$, write $r = sa$ for some $s \in R$. Then
        \[
            a = rb = (sa)b = (as)b = a(sb)
        \]
        since an integral domain is commutative. Thus $a - a(sb) = a(1-sb) = 0$. Now as $R$ is an integral domain thus either $a = 0$ (impossible since $a \neq 0$) or $1-sb = 0$. So $1-sb = 0$, meaning $sb = 1 \in J$ since $b \in J$. By \myref{exercise-ideal-containing-1-is-whole-ring} we have $J = R$.
        \item If instead $b \in I$, take any $x \in J = \princ{b}$, so $x = rb$ for some $r \in R$. Thus $x = rb \in I$ since $b \in I$, so $J \subseteq I$. But $I \subseteq J$, so $J = I$.
    \end{itemize}
    Therefore we have shown that $I$ is maximal.

    \item First we work in the forward direction. Suppose $\princ{a} = \princ{b}$. As $a \in \princ{a} = \princ{b}$, thus $a = bx$ for some $x \in R$. Also, as $b \in \princ{b} = \princ{a}$, thus $b = ay$ for some $y \in R$. Therefore
    \[
        b = ay = (bx)y = b(xy)
    \]
    which means $xy = 1$. Thus $x$ and $y$ are units, meaning $a = bx$ with $x$ being a unit.

    Now we work in the reverse direction; suppose $a = bu$ for some unit $u$ in $D$.
    \begin{itemize}
        \item Take $r \in \princ{a}$, so $r = ax$ for some $x$ in $D$. Thus $r = (bu)x = b(ux) \in \princ{b}$, so $\princ{a} \subseteq \princ{b}$.
        \item Note $b = au^{-1}$ since $u$ is a unit. Take $s \in \princ{b}$, so $s = by$ for some $y$ in $D$. But as $b = au^{-1}$, hence $s = (au^{-1})y = a(u^{-1}y) \in \princ{a}$, so $\princ{b} \subseteq \princ{a}$.
    \end{itemize}
    Therefore we see that $\princ{a} = \princ{b}$.
\end{questions}

\chapter{Ring Homomorphisms and Isomorphisms}
Like with groups, rings too have homomorphisms and isomorphisms, although they are defined slightly differently than in groups. Similar to how group homomorphisms preserve some structure between the two groups, ring homomorphisms and isomorphisms also preserve structure between rings.

\section{Ring Homomorphisms and Isomorphisms}
\begin{definition}
    Let $(R_1, +, \cdot)$ and $(R_2, \oplus, \otimes)$ be rings. A map $\phi: R_1 \to R_2$ is a \term{ring homomorphism}\index{homomorphism!ring} if and only if for all $a, b \in R_1$, we have
    \begin{align*}
        \phi(a+b) &= \phi(a) \oplus \phi(b) \text{ and}\\
        \phi(a\cdot b) &= \phi(a)\otimes\phi(b).
    \end{align*}
\end{definition}
\begin{remark}
    Like with group homomorphisms, we usually use ``$+$'' for both addition operations and suppress the multiplication operation. That is, the ring homomorphism conditions become
    \begin{align*}
        \phi(a+b) &= \phi(a) + \phi(b) \text{ and}\\
        \phi(ab) &= \phi(a)\phi(b).
    \end{align*}
\end{remark}

\begin{example}
    We show that the map $\phi: \Z \to \Z/n\Z, x \mapsto x + n\Z$ is a ring homomorphism. Let $a, b \in \Z$. Note
    \begin{align*}
        \phi(a+b) &= (a+b) + n\Z\\
        &= (a + n\Z) + (b + n\Z) & (\text{Definition of coset addition})\\
        &=\phi(a)+\phi(b)
    \end{align*}
    and
    \begin{align*}
        \phi(ab) &= ab + n\Z\\
        &= (a + n\Z)(b + n\Z) & (\text{Definition of coset multiplication})\\
        &= \phi(a)\phi(b)
    \end{align*}
    so $\phi$ is a homomorphism.
\end{example}

\begin{example}
    Consider the ring
    \[
        R = \left\{\begin{pmatrix}a&b\\0&c\end{pmatrix}\vert a,b,c\in\Z\right\}.
    \]
    The map $\phi: R \to \Z^2, \begin{pmatrix}a&b\\0&c\end{pmatrix} \mapsto (a,c)$ is a ring homomorphism since, for any $\begin{pmatrix}a&b\\0&c\end{pmatrix}, \begin{pmatrix}x&y\\0&z\end{pmatrix} \in R$, we have
    \begin{align*}
        \phi\left(\begin{pmatrix}a&b\\0&c\end{pmatrix} + \begin{pmatrix}x&y\\0&z\end{pmatrix}\right) &= \phi\left(\begin{pmatrix}a+x&b+y\\0&c+z\end{pmatrix}\right)\\
        &= (a+x,c+z)\\
        &= (a,c) + (x,z)\\
        &= \phi\left(\begin{pmatrix}a&b\\0&c\end{pmatrix}\right) + \phi\left(\begin{pmatrix}x&y\\0&z\end{pmatrix}\right)
    \end{align*}
    and
    \begin{align*}
        \phi\left(\begin{pmatrix}a&b\\0&c\end{pmatrix}\begin{pmatrix}x&y\\0&z\end{pmatrix}\right) &= \phi\left(\begin{pmatrix}ax&ay+bz\\0&cz\end{pmatrix}\right)\\
        &= (ax, cz)\\
        &= (a,c)(x,z)\\
        &= \phi\left(\begin{pmatrix}a&b\\0&c\end{pmatrix}\right)\phi\left(\begin{pmatrix}x&y\\0&z\end{pmatrix}\right).
    \end{align*}
\end{example}
\begin{exercise}
    Let the function $\phi: \Mn{2}{\Z} \to \Z$ be defined such that
    \[
        \phi\left(\begin{pmatrix}a&b\\c&d\end{pmatrix}\right) = a+d.
    \]
    Is $\phi$ a ring homomorphism?
\end{exercise}
\begin{exercise}
    Let $R$ and $S$ be rings with additive identities $0_R$ and $0_S$ respectively. Show that the \term{trivial homomorphism}\index{homomorphism!trivial} $\phi: R \to S, r \mapsto 0_S$ is, indeed, a ring homomorphism.
\end{exercise}

\pagebreak

An endomorphism is a specific type of homomorphism.
\begin{definition}
    A \term{ring endomorphism}\index{endomorphism!ring} of a ring $R$ is a homomorphism $\phi: R \to R$.
\end{definition}
\begin{example}
    Let $R$ be a commutative ring with prime characteristic $p$. The \term{Frobenius endomorphism}\index{Frobenius endomorphism} $\phi: R \to R$ is such that $\phi(r) = r^p$. We show that $\phi$ is a ring endomorphism.

    Note that for any $a, b \in R$ that
    \begin{align*}
        \phi(a+b) &= (a+b)^p\\
        &= a^p + pa^{p-1}b + {p \choose 2}a^{p-2}b^2 + \cdots + pab^{p-1} + b^p.
    \end{align*}
    Note that the binomial coefficients ${p \choose k}$ where $1 \leq k \leq p-1$ are all multiples of $p$ (\myref{prop-binomial-coefficient-multiple-of-p}). As the characteristic of the ring $R$ is $p$, thus $px = 0$ for any $x \in R$. Therefore,
    \begin{align*}
        \phi(a+b) = &a^p + pa^{p-1}b + {p \choose 2}a^{p-2}b^2 + \cdots + pab^{p-1} + b^p\\
        &= a^p + 0 + 0 + \cdots + 0 + b^p\\
        &= a^p + b^p\\
        &=\phi(a) + \phi(b).
    \end{align*}

    Also,
    \[
        \phi(ab) = (ab)^p = a^pb^p = \phi(a)\phi(b).
    \]
    Therefore $\phi$ is a ring endomorphism.
\end{example}

\begin{exercise}
    Let $R$ be a ring. Show that the \term{identity homomorphism}\index{homomorphism!identity} $\id: R \to R, r \mapsto r$ is a ring endomorphism.
\end{exercise}

The definition of ring isomorphisms is analogous to that of group isomorphisms.

\begin{definition}
    A \term{ring isomorphism}\index{isomorphism!ring} is a bijective ring homomorphism.
\end{definition}

Similar to groups, we write $R_1 \cong R_2$ if and only if $R_1$ and $R_2$ are (ring) isomorphic to each other.

\begin{example}\label{example-Zn-ring-isomorphic-to-Z/nZ}
    We show that $\Z_n \cong \Z/n\Z$. Consider the map $\phi:\Z_n \to \Z/n\Z$ where $m \mapsto m + n\Z$. We show that $\phi$ is an isomorphism.
    \begin{itemize}
        \item \textbf{Homomorphism}: For any $a, b \in \Z_n$ we see
        \[
            \phi(a+b) = (a+b) + n\Z = (a + n\Z) + (b + n\Z) = \phi(a) + \phi(b)
        \]
        and
        \[
            \phi(ab) = (ab) + n\Z = (a+n\Z)(b+n\Z) = \phi(a)\phi(b).
        \]

        \item \textbf{Injective}: Suppose $a, b \in \Z_n$ such that $\phi(a) = \phi(b)$. This means $a + n\Z = b + n\Z$, i.e. $a \equiv b \pmod n$. Now note that $0 \leq a,b < n$ so we have $a = b$.

        \item \textbf{Surjective}: Suppose $m + n\Z \in \Z/n\Z$. Applying Euclid's division lemma (\myref{lemma-euclid-division}) on $m$ we have $m = nq + r$ with $0 \leq r < n$. One sees that
        \begin{align*}
            \phi(r) &= r + n\Z\\
            &= r + (nq + n\Z)\\
            &= (r + nq) + n\Z\\
            &= m + n\Z
        \end{align*}
        so $m + n\Z$ has a pre-image of $r$ in $\Z_n$.
    \end{itemize}
    Since $\phi$ is a bijective ring homomorphism, thus $\phi$ is an isomorphism, meaning $\Z_n \cong \Z/n\Z$ as rings.
\end{example}
\begin{example}
    We show that $\Z \not\cong 2\Z$ as rings. Suppose $\phi: \Z \to 2\Z$ is a ring isomorphism. Set $a = \phi(1) = 2\Z$. Note that
    \[
        a = \phi(1) = \phi(1\times1) = (\phi(1))^2 = a^2
    \]
    so $a^2 = a$, which means $a = 0$ or $a = 1$. But as $a \in 2\Z$, thus $a \neq 1$ which means $a = 0$.

    But notice for any $n \in \Z$ we have
    \begin{align*}
        \phi(n) &= \phi(n1)\\
        &= \phi(n)\phi(1)\\
        &= \phi(n) \times 0\\
        &= 0.
    \end{align*}
    Thus one sees that $\phi(0) = \phi(1) = 0$ which means $\phi$ is not injective, a contradiction.
\end{example}

\begin{definition}
    A bijective ring endomorphism is called a \term{ring automorphism}\index{automorphism!ring}.
\end{definition}

\begin{exercise}\label{exercise-identity-homomorphism-is-an-isomorphism}
    Show that the identity homomorphism is actually an automorphism.
\end{exercise}

\section{Properties of Ring Homomorphisms}
For the following, let $R_1$ and $R_2$ be rings with additive identities $0_1$ and $0_2$ respectively. Also let $\phi: R_1 \to R_2$ be a ring homomorphism.

\begin{proposition}\label{prop-ring-image-of-additive-identity-is-additive-identity}
    $\phi(0_1) = 0_2$.
\end{proposition}
\begin{proof}
    See \myref{exercise-ring-image-of-identity-is-identity} (later).
\end{proof}

\begin{proposition}
    If $R_1$ and $R_2$ are division rings, then $\phi(1_1) = 1_2$ where $1_1$ and $1_2$ are the multiplicative identities of $R_1$ and $R_2$ respectively.
\end{proposition}
\begin{proof}
    See \myref{exercise-ring-image-of-identity-is-identity} (later).
\end{proof}

\begin{proposition}
    $\phi(-x) = -\phi(x)$ for all $x \in R_1$.
\end{proposition}
\begin{proof}
    See \myref{exercise-ring-image-of-inverse-is-inverse} (later).
\end{proof}

\begin{proposition}\label{prop-inverse-under-ring-homomorphism}
    If $R_1$ and $R_2$ are both division rings, then $\phi(x^{-1}) = (\phi(x))^{-1}$ for all $x \in R_1$.
\end{proposition}
\begin{proof}
    See \myref{exercise-ring-image-of-inverse-is-inverse} (later).
\end{proof}

\begin{proposition}\label{prop-homomorphism-on-subring-is-subring}
    If $S$ is a subring of $R_1$, then
    \[
        \phi(S) = \{\phi(s) | s \in S\}
    \]
    is a subring of $R_2$.
\end{proposition}
\begin{proof}
    Let $S$ be a subring of $R_1$. Take $a, b \in \phi(S)$, which means that there exist $s_a, s_b\in S$ such that $\phi(s_a) = a$ and $\phi(s_b) = b$.
    \begin{itemize}
        \item We show that $(\phi(S), +) \leq (R_2, +)$.
        \begin{itemize}
            \item Note that $\phi(S) \neq \emptyset$ since $\phi(0_1) = 0_2 \in \phi(S)$.
            \item Also note that $a - b = \phi(s_a) - \phi(s_b) = \phi(s_a-s_b) \in \phi(S)$.
        \end{itemize}

        \item One also sees that
        \[
            ab = \phi(s_a)\phi(s_b) = \phi(s_as_b) \in \phi(S).
        \]
    \end{itemize}
    Therefore $\phi(S)$ is a subring of $R_2$.
\end{proof}

\begin{proposition}
    If $\phi$ is surjective and $I$ is an ideal of $R_1$, then $\phi(I)$ is an ideal of $R_2$.
\end{proposition}
\begin{proof}
    From previous proposition $\phi(I)$ is a subring of $R_2$. We just need to show that $\phi(I)$ is an ideal of $R_2$.

    Take $a \in \phi(I)$ and $r_2 \in R_2$. As $\phi$ is surjective, we can find a $r_1 \in R_1$ such that $\phi(r_1) = r_2$. Also, let $a = \phi(i)$ for an $i \in I$.

    Note
    \begin{align*}
        ar_2 = \phi(i)\phi(r_1) = \phi(ir_1) \in \phi(I)\\
        r_2a = \phi(r_1)\phi(i) = \phi(r_1i) \in \phi(I)
    \end{align*}
    so $\phi(I)$ is an ideal of $R_2$.
\end{proof}

\begin{proposition}\label{prop-inverse-homomorphism-on-ideal-is-ideal}
    Let $J$ be an ideal of $R_2$. Then
    \[
        \phi^{-1}(J) = \{r \in R_1 \vert \phi(r) \in J\}
    \]
    is an ideal of $R_1$.
\end{proposition}
\begin{proof}
    Suppose $J$ is an ideal of $R_2$. We consider the test for ideal (\myref{thrm-test-for-ideal}) to show $\phi^{-1}(J)$ is an ideal of $R_1$.

    One sees that $\phi^{-1}(J) \neq \emptyset$ since $\phi(0_1) = 0_2 \in J$, so $0_1 \in \phi^{-1}(J)$.

    Let $a, b \in \phi^{-1}(J)$, so $\phi(a), \phi(b) \in J$. Note that
    \[
        \phi(a-b) = \phi(a) - \phi(b) \in J
    \]
    so $a-b \in \phi^{-1}(J)$ for all $a,b \in J$.

    Let $r \in R_1$ and $a \in \phi^{-1}(J)$. Note that $\phi(a) \in J$ and $\phi(r) \in R_2$, so $\underbrace{\phi(a)}_{\text{In }J}\underbrace{\phi(r)}_{\text{In }R_2} \in J$ and $\phi(r)\phi(a) \in J$. Note $\phi(a)\phi(r) = \phi(ar) \in J$, so $ar \in \phi^{-1}(J)$, and similarly we have $\phi(r)\phi(a) = \phi(ra) \in J$, so $ra \in \phi^{-1}(J)$.

    Therefore, by the test for ideal, $\phi^{-1}(J)$ is an ideal of $R_1$.
\end{proof}

\begin{exercise}\label{exercise-ring-image-of-identity-is-identity}
    Let $R_1$ and $R_2$ be rings, and $\phi: R_1 \to R_2$ be a ring homomorphism.
    \begin{partquestions}{\alph*}
        \item Show that $\phi(0_1) = 0_2$, where $0_1$ and $0_2$ are the additive identities of $R_1$ and $R_2$ respectively.
        \item If $R_1$ and $R_2$ are division rings, then show that $\phi(1_1) = 1_2$, where $1_1$ and $1_2$ are the multiplicative identities of $R_1$ and $R_2$ respectively.
    \end{partquestions}
\end{exercise}

\begin{exercise}\label{exercise-ring-image-of-inverse-is-inverse}
    Let $R_1$ and $R_2$ be rings, $x \in R_1$, and $\phi: R_1 \to R_2$ be a ring homomorphism.
    \begin{partquestions}{\alph*}
        \item Show that $\phi(-x) = -\phi(x)$.
        \item If $R_1$ and $R_2$ are division rings, show that $\phi(x^{-1}) = (\phi(x))^{-1}$.
    \end{partquestions}
\end{exercise}

\section{Image and Kernel}
Similar to group homomorphisms, ring homomorphisms too have a image and kernel.
\begin{definition}
    The \term{image}\index{image} of a ring homomorphism $\phi: R_1 \to R_2$ is
    \[
        \im\phi = \{\phi(r) \vert r \in R_1\}.
    \]
\end{definition}
\begin{definition}
    The \term{kernel}\index{kernel} of a ring homomorphism $\phi:R_1 \to R_2$ is
    \[
        \ker\phi = \{r \in R_1 \vert \phi(r) = 0\}.
    \]
\end{definition}

\begin{example}\label{example-homomorphism-on-upper-triangle-matrices}
    Consider the ring
    \[
        R = \left\{\begin{pmatrix}a&b\\0&c\end{pmatrix}\vert a,b,c\in\Z\right\}
    \]
    and the homomorphism $\phi: R \to \Z^2, \begin{pmatrix}a&b\\0&c\end{pmatrix} \mapsto (a,c)$.

    We note that $\phi$ is surjective; for any $(x,y)\in\Z^2$, we see that 
    \[
        \phi\left(\begin{pmatrix}x&0\\0&y\end{pmatrix}\right) = (x,y)
    \]
    so any $(x,y)$ has a pre-image in $R$. Therefore $\im \phi = \Z^2$.

    We now find the kernel of $\phi$.
    \begin{align*}
        \ker\phi &= \left\{\begin{pmatrix}a&b\\0&c\end{pmatrix} \in R \vert \phi\left(\begin{pmatrix}a&b\\0&c\end{pmatrix}\right) = (0,0)\right\}\\
        &= \left\{\begin{pmatrix}a&b\\0&c\end{pmatrix} \in R \vert (a,c) = (0,0)\right\}\\
        &= \left\{\begin{pmatrix}0&n\\0&0\end{pmatrix} \vert n \in \Z\right\}.
    \end{align*}
\end{example}

We look at some results regarding the image and kernel of a ring homomorphism. These results may look familiar to those in part I.
\begin{proposition}\label{prop-image-is-a-subring}
    Let $R_1$ and $R_2$ be rings, and let $\phi: R_1 \to R_2$ be a ring homomorphism. Then $\im\phi$ is a subring of $R_2$.
\end{proposition}
\begin{proof}
    \myref{prop-homomorphism-on-subring-is-subring} tells us $\im\phi = \phi(R_1)$ is a subring of $R_2$.
\end{proof}

\begin{proposition}\label{prop-kernel-is-an-ideal}
    Let $R_1$ and $R_2$ be rings, and let $\phi: R_1 \to R_2$ be a ring homomorphism. Then $\ker\phi$ is an ideal of $R_1$.
\end{proposition}
\begin{proof}
    See \myref{exercise-kernel-is-an-ideal} (later).
\end{proof}

\begin{proposition}
    Let $R_1$ and $R_2$ be rings, and let $\phi: R_1 \to R_2$ be a ring homomorphism. Then $\phi$ is injective if and only if $\ker\phi = \{0_1\}$.
\end{proposition}
\begin{proof}
    We first prove the forward direction; suppose $\phi$ is injective and let $a \in \ker\phi$. By definition of the kernel we have $\phi(a) = 0_2$. But by \myref{prop-ring-image-of-additive-identity-is-additive-identity}, we have $\phi(0_1) = 0_2$. Since $\phi$ is injective, therefore $a = 0_1$, meaning $\ker\phi = \{0_1\}$.

    We now prove the reverse direction; suppose $\ker\phi = \{0_1\}$. Now let $a,b \in R_1$ such that $\phi(a) = \phi(b)$. Therefore $\phi(a) - \phi(b) = \phi(a-b) = 0_2$. Therefore $a-b \in \ker\phi$ by definition of the kernel. However $\ker\phi = \{0_1\}$ which means that $a - b = 0_1$. Therefore $a = b$, meaning $\phi$ is injective.
\end{proof}

\begin{exercise}\label{exercise-kernel-is-an-ideal}
    Let $R_1$ and $R_2$ be rings, and let $\phi: R_1 \to R_2$ be a ring homomorphism. Prove that $\ker\phi$ is an ideal of $R_1$.
\end{exercise}

\section{The Ring Isomorphism Theorems}
Similar to group theory, there are three main ring isomorphism theorems. However, we will only explicitly prove the first ring isomorphism theorem; the other two will be left as problems.

\begin{theorem}[First Ring Isomorphism Theorem (FRIT)]\label{thrm-ring-isomorphism-1}\index{isomorphism theorem!ring!first}\index{FRIT}
    Let $R$ and $R'$ be rings. Let $\phi: R \to R'$ be a ring homomorphism, and let $\pi: R \to R/\ker\phi$, $r\mapsto r + \ker\phi$ be the natural surjective homomorphism. Then there exists a unique ring isomorphism $\psi: R / \ker\phi \to \im\phi$ such that $\psi\pi = \phi$.
\end{theorem}
\begin{remark}
    Equivalently, the FRIT states that
    \[
        R / \ker\phi \cong \im\phi
    \]
    for any ring homomorphism $\phi$. This means that the set of pre-images that have an image of $\phi(x)$ is $x\ker\phi$.
\end{remark}

We include the commutativity diagram of the stated maps for clarity.
\begin{figure}[h]
    \centering
    \pdfteximgframed[12pt]{0.3\textwidth}{part2/images/ring-homomorphisms/ring-iso-1-commutativity.pdf_tex}
    \caption{Commutativity Diagram for \myreffigures{thrm-ring-isomorphism-1}}
\end{figure}

In the diagram, $\phi$ sends elements from $R$ to $\im\phi$ and $\pi$ sends elements from $R$ to $R/\ker\phi$. Then the map $\psi$ is a unique map that sends elements from $R/\ker\phi$ to the image of $\phi$.

\begin{proof}[Proof (cf. {\cite[p.~302, Factor Theorem For Rings]{cohn_1982}})]
    Let the map $\psi$ be defined such that $\psi(r + \ker\phi) = \phi(r)$. We first show that $\psi$ is a well-defined ring isomorphism.
    \begin{itemize}
        \item \textbf{Well-Defined}: Suppose $a + \ker\phi$ and $b + \ker\phi$ are in $R/\ker\phi$ such that $a + \ker\phi = b+\ker\phi$. This means that $a - b \in \ker\phi$, i.e. $\phi(a-b) = 0$ by definition of the kernel. Hence $\phi(a) - \phi(b) = 0$ which means $\phi(a) = \phi(b)$. Therefore, we see that
        \[
            \psi(a + \ker\phi) = \phi(a) = \phi(b) = \psi(b + \ker\phi)
        \]
        which means $\psi$ is well-defined.

        \item \textbf{Homomorphism}: Let $a + \ker\phi, b + \ker\phi \in R/\ker\phi$. Then note
        \begin{align*}
            &\psi((a + \ker\phi)+(b+\ker\phi))\\
            &= \psi((a+b)+\ker\phi)\\
            &= \phi(a+b)\\
            &= \phi(a) + \phi(b)\\
            &= \psi(a + \ker\phi) + \psi(b + \ker\phi)
        \end{align*}
        and
        \begin{align*}
            \psi((a + \ker\phi)(b+\ker\phi)) &= \psi((ab)+\ker\phi)\\
            &= \phi(ab)\\
            &= \phi(a)\phi(b)\\
            &= \psi(a + \ker\phi)\psi(b + \ker\phi),
        \end{align*}
        so $\psi$ is a ring homomorphism.

        \item \textbf{Injective}: Suppose $a + \ker\phi, b + \ker\phi \in R/\ker\phi$ such that $\psi(a+\ker\phi) = \psi(b+\ker\phi)$. So,
        \begin{align*}
            \phi(a) &= \phi(b) & (\text{definition of }\psi)\\
            \phi(a) - \phi(b) &= 0\\
            \phi(a-b) &= 0 & (\phi \text{ is a ring homomorphism})\\
            a - b &\in \ker\phi & (\text{definition of kernel})\\
            a + \ker\phi &= b + \ker\phi.
        \end{align*}
        Therefore if $\psi(a+\ker\phi) = \psi(b+\ker\phi)$ then $a+\ker\phi = b+\ker\phi$, which means $\psi$ is injective.

        \item \textbf{Surjective}: Suppose $s \in \im\phi$, so there is an $r \in R$ such that $s = \phi(r)$. Clearly $\psi(r + \ker\phi) = \phi(r) = s$, so $s$ has a pre-image of $r + \ker\phi$, i.e. $\psi$ is surjective.
    \end{itemize}
    Therefore, $\psi$ is a well-defined bijective ring homomorphism, i.e. $\psi$ is a well-defined ring isomorphism.

    We now check that $\psi$ satisfies the requirement that $\psi\pi = \phi$. Note that $\pi(x) = x + \ker\phi$ and
    \[
        \psi\pi(x) = \psi(x + \ker\phi) = \phi(x)
    \]
    for all $x \in R$, so $\psi\pi = \phi$.

    Finally, we show that $\psi$ is unique. Suppose $f: R/\ker\phi \to \im\phi$ is an isomorphism satisfying $f\pi=\phi$. Note that
    \begin{align*}
        f(x + \ker\phi) &= f(\pi(x))\\
        &= (f\pi)(x)\\
        &= \phi(x)\\
        &= (\psi\pi)(x)\\
        &= \psi(\pi(x))\\
        &= \psi(x + \ker\phi)
    \end{align*}
    for all $x \in R$, meaning that $f = \psi$. Therefore $\psi$ is unique.

    Hence, $\psi$ is a unique ring isomorphism satisfying $\psi\pi = \phi$.
\end{proof}

\begin{example}
    Consider the ring
    \[
        R = \left\{\begin{pmatrix}a&b\\0&c\end{pmatrix}\vert a,b,c\in\Z\right\}
    \]
    and the homomorphism $\phi: R \to \Z^2, \begin{pmatrix}a&b\\0&c\end{pmatrix} \mapsto (a,c)$. We found in \myref{example-homomorphism-on-upper-triangle-matrices} that $\phi$ is surjective (i.e., $\im\phi = \Z^2$) with kernel
    \[
        \left\{\begin{pmatrix}0&n\\0&0\end{pmatrix} \vert n \in \Z\right\}
    \]
    which, for brevity, we shall denote by $I$. Thus the FRIT (\myref{thrm-ring-isomorphism-1}) tells us that
    \[
        R/I \cong \Z^2.
    \]
\end{example}
\begin{exercise}
    Show that $\Z_n \cong \Z/n\Z$ by considering the ring homomorphism $\phi: \Z \to \Z_n$ where $m \mapsto m \mod n$ and by using the FRIT (\myref{thrm-ring-isomorphism-1}).
\end{exercise}

We briefly mention the other two main ring isomorphism theorems, although the proof of them will be left as problems. They are much less used than the FRIT, so we make only a passing mention of them.

\newpage

\begin{theorem}[Second Ring Isomorphism Theorem]\label{thrm-ring-isomorphism-2}\index{isomorphism theorem!ring!second}
    Let $R$ be a ring with subring $S$ and ideal $I$. Then
    \begin{enumerate}
        \item $S+I = \{s+i \vert s\in S,\;i\in I\}$ is a subring of $R$;
        \item $S \cap I$ is an ideal of $S$; and
        \item $(S+I)/I \cong S/(S\cap I)$.
    \end{enumerate}
\end{theorem}
\begin{proof}
    See \myref{problem-ring-isomorphism-2} (later).
\end{proof}

\begin{theorem}[Third Ring Isomorphism Theorem]\label{thrm-ring-isomorphism-3}\index{isomorphism theorem!ring!third}
    Let $R$ be a ring with ideals $I$ and $J$ such that $I$ is a subset of $J$. Then
    \begin{enumerate}
        \item $J/I$ is an ideal of $R/I$; and
        \item $\frac{R/I}{J/I} \cong R/J$.
    \end{enumerate}
\end{theorem}
\begin{proof}
    See \myref{problem-ring-isomorphism-3} (later).
\end{proof}

\section{Restrictiveness of Ring Homomorphisms}
Although ring homomorphisms appear to be quite general, we explore how restricted they really are when dealing with certain rings.

\begin{example}\label{example-endomorphisms-of-Z}
    We find all ring endomorphisms of $\Z$.

    Let $\phi:\Z\to\Z$ be a ring endomorphism. Set $a = \phi(1)$. Note that
    \[
        a = \phi(1) = \phi(1\times1) = \phi(1)\phi(1) = a^2
    \]
    so $a^2 = a$. Thus $a = 0$ or $a = 1$ in $\Z$.

    If $a = 0$, then for any $n \in \Z$ we have
    \[
        \phi(n) = \phi(1n) = \phi(1)\phi(n) = 0\phi(n) = 0
    \]
    so $\phi(n) = 0$ for all $n \in \Z$, which is the trivial homomorphism.

    Now consider the case that $a = 1$. We claim that $\phi(n) = n$ for all $n \in \Z$. We leave the proof that $\phi(n) = n$ for all \textit{positive} integers $n$ for \myref{exercise-homomorphism-maps-n-to-n-if-n-is-positive} (later). Furthermore $\phi(0) = 0$ by the properties of ring homomorphism (specifically \myref{prop-ring-image-of-additive-identity-is-additive-identity}). Finally, note that for any non-negative integer $n$,
    \begin{align*}
        0 &= \phi(0)\\
        &= \phi(n - n)\\
        &= \phi(n) + \phi(-n)\\
        &= n + \phi(-n)
    \end{align*}
    which means $\phi(-n) = -n$. Thus $\phi(n) = n$ for all integers $n$, which is the identity endomorphism.

    Therefore, the only ring endomorphisms $\phi:\Z\to\Z$ are the trivial homomorphism and the identity endomorphism.
\end{example}
\begin{exercise}\label{exercise-homomorphism-maps-n-to-n-if-n-is-positive}
    For the map $\phi$ in \myref{example-endomorphisms-of-Z}, show that $\phi(n) = n$ for all positive integers $n$.
\end{exercise}
Note that in \myref{example-endomorphisms-of-Z} we started the entire computation with the observation that $\phi(1) = \phi(1)^2$. This means that $\phi(1)$ is an idempotent.
\begin{definition}
    Let $R$ be a ring. Then an element $x \in R$ is an \term{idempotent}\index{idempotent} if and only if $x^2 = x$.
\end{definition}
\begin{proposition}\label{prop-homomorphism-on-multiplicative-identity-is-idempotent}
    Let $R$ and $R'$ be rings, and let $\phi: R \to R'$ be a ring homomorphism. If $R$ is a ring with identity, then $\phi(1)$ is an idempotent.
\end{proposition}
\begin{proof}
    Note
    \[
        \phi(1) = \phi(1 \times 1) = \phi(1) \times \phi(1) = \left(\phi(1)\right)^2
    \]
    which means $\phi(1)$ is an idempotent.
\end{proof}

In \myref{example-endomorphisms-of-Z} we used the fact that the only idempotents of $\Z$ are 0 and 1. However, this is not true for a general ring.

\begin{example}\label{example-homomorphisms-from-Z12-to-Z28}
    We find all ring homomorphisms $\phi: \Z_{12} \to \Z_{28}$. Note that $\Z_{12}$ is a ring with an identity of 1.

    \myref{prop-homomorphism-on-multiplicative-identity-is-idempotent} tells us that $\phi(1)$ is an idempotent, so we need to find all idempotents of $\Z_{28}$. However, we cannot just assume that 0 and 1 are the \textbf{only} idempotents in $\Z_{28}$; we need to check for them exhaustively.

    By exhaustion, we see that $0^2 = 0$, $1^2 = 1$, $8^2 = 64 = 2 \times 28 + 8 = 8$, $21^2 = 441 = 15 \times 28 + 21 = 21$. So the idempotents in $\Z_{28}$ are 0, 1, 8, and 21. This is not enough to narrow down the possible values of $\phi(1)$, so we need to use more facts.

    Recall from part I that $|\phi(1)|_+$ divides $|1|_+$ by \myref{exercise-order-of-homomorphism-divides-order}. Therefore $|\phi(1)|_+$ divides 12. Furthermore, \myref{thrm-order-of-element-in-cyclic-group} tells us that the additive order of an element $k$ in the group $(\Z_n, +)$ is $\frac{n}{\gcd(k,n)}$. So we must exhaust all idempotents in $\Z_{28}$ to check whether they are valid values for $\phi(1)$.

    % \begin{itemize}
    %     \item $|0|_+ = 1$ which divides 12, so 0 is a valid value of $\phi(1)$.
    %     \item $|1|_+ = 28$ which does not divide 12, so 1 is not a valid value of $\phi(1)$. Note that this is different from the previous example, where 1 was a possible value of $\phi(1)$.
    %     \item $|8|_+ = \frac{28}{\gcd(8,28)} = \frac{28}4 = 7$ which does not divide 12, so 8 is not a valid value of $\phi(1)$.
    %     \item $|21|_+ = \frac{28}{\gcd(21,28)} = \frac{28}7 = 4$ which divides 12, so 21 is a valid value of $\phi(1)$.
    % \end{itemize}
    % Hence $\phi(1) = 0$ or $\phi(1) = 21$.

    \begin{multicols}{2}
        \begin{itemize}
            \item $|0|_+ = \frac{28}{\gcd(0,28)} = \frac{28}{28} = 1$
            \item $|1|_+ = \frac{28}{\gcd(1,28)} = \frac{28}{1} = 28$
            \item $|8|_+ = \frac{28}{\gcd(8,28)} = \frac{28}{4} = 7$
            \item $|21|_+ = \frac{28}{\gcd(21,28)} = \frac{28}{7}= 4$
        \end{itemize}
    \end{multicols}

    Of the four idempotents, only $|0|_+ = 1$ and $|21|_+ = 4$ divides 12, which means $\phi(1) = 0$ or $\phi(1) = 21$.

    If $\phi(1) = 0$, then for any $n \in \Z_{12}$,
    \begin{align*}
        \phi(n) &= \phi(\underbrace{1 + 1 + \cdot + 1}_{n \text{ times}})\\
        &= \underbrace{\phi(1) + \phi(1) + \cdot + \phi(1)}_{n \text{ times}}\\
        &= \underbrace{0 + 0 + \cdots + 0}_{n \text{ times}}\\
        &= 0
    \end{align*}
    which means that $\phi(n) = 0$ for all $n \in \Z_{12}$, i.e. $\phi$ is trivial.

    If instead $\phi(1) = 21$, then
    \begin{align*}
        \phi(n) &= \underbrace{\phi(1) + \phi(1) + \cdot + \phi(1)}_{n \text{ times}}\\
        &= \underbrace{21 + 21 + \cdots + 21}_{n \text{ times}}\\
        &= 21n
    \end{align*}
    which means $\phi(n) = 21n$ for all $n \in \Z_{12}$.

    Thus the only homomorphisms $\phi: \Z_{12} \to \Z_{28}$ are $\phi(n) = 0$ and $\phi(n) = 21n$ for all $n \in \Z_{12}$.
\end{example}

\begin{exercise}\label{exercise-homomorphism-over-Q-fixes-elements-of-Q}
    Suppose $R$ and $R'$ are rings such that $\Q$ is a subring of both $R$ and $R'$. Let $\phi: R \to R'$ be a ring homomorphism such that $\phi(1) = 1$. Show that for any $q \in \Q$ we have $\phi(q) = q$.
\end{exercise}

\newpage

\section{Problems}
\begin{problem}\label{problem-integral-domain-iff-trivial-ideal-is-prime}
    Let $R$ be a ring.
    \begin{partquestions}{\roman*}
        \item Show that $R/\{0\} \cong R$.
        \item Prove that $R$ is an integral domain if and only if $\{0\}$ is a prime ideal.
        \item Prove that $R$ is a field if and only if $\{0\}$ is a maximal ideal.
    \end{partquestions}
\end{problem}

\begin{problem}
    Find all ring endomorphisms of $\Q$.
\end{problem}

\begin{problem}
    Show $\Z^2 \not\cong \Q$.
\end{problem}

\begin{problem}
    Show $\Q[\sqrt2] \not\cong \Q[\sqrt3]$.
\end{problem}

\begin{problem}
    Consider the subring
    \[
        R = \left\{\begin{pmatrix}a&0\\0&b\end{pmatrix}\vert a,b \in \Z\right\}
    \]
    of $\Mn{2}{\Z}$. Show that $R \cong \Z^2$.
\end{problem}

\begin{problem}\label{problem-properties-of-ring-isomorphism}
    Let $R$ and $R'$ be rings, and let $\phi: R \to R'$ be a ring isomorphism. Prove or disprove the following statements.
    \begin{partquestions}{\alph*}
        \item $\phi^{-1}: R' \to R$ is a ring isomorphism.
        \item If $R$ has a subring with $n$ elements, then so does $R'$.
        \item If $R$ has an ideal, then so does $R'$.
    \end{partquestions}
\end{problem}

\begin{problem}
    Find all ring endomorphisms of $\Z_{10}$. Hence find all ring automorphisms $\psi$ of $\Z_{10}$.
\end{problem}

\begin{problem}
    Find all ring endomorphisms of $\Q[\sqrt3]$. Hence find all ring automorphisms $\psi$ of $\Q[\sqrt3]$.
\end{problem}

\begin{problem}
    Let $R$ and $R'$ be commutative rings, $I$ be an ideal of $R$, and $\phi: R\to R'$ be a ring homomorphism.
    \begin{partquestions}{\roman*}
        \item Show that $\phi(\sqrt I) \subseteq \sqrt{\phi(I)}$.
        \item If $\phi$ is surjective with $\ker\phi \subseteq I$, prove that $\phi(\sqrt I) = \sqrt{\phi(I)}$.
    \end{partquestions}
\end{problem}

\pagebreak

\begin{problem}\label{problem-ring-isomorphism-2}
    Let $R$ be a ring with a subring $S$ and ideal $I$. Prove that
    \begin{partquestions}{\roman*}
        \item $S+I$ is a subring of $R$;
        \item $S \cap I$ is an ideal of $S$; and
        \item $S/(S\cap I)\cong (S+I)/I$.
    \end{partquestions}
\end{problem}

\begin{problem}\label{problem-ring-isomorphism-3}
    Let $R$ be a ring with ideals $I$ and $J$ such that $I$ is a subset of $J$.
    \begin{partquestions}{\roman*}
        \item Prove that $J/I$ is an ideal of $R/I$.
        \item Prove that $\frac{R/I}{J/I} \cong R/J$.\newline
        (\textit{Note: remember to prove that the map is well-defined.})
    \end{partquestions}
\end{problem}

\section{Polynomial Rings}
\begin{questions}
    \item For brevity let $I = \princ{x} = \{xP(x) \vert P(x) \in \Z[x]\}$. This means that $I$ is the set of polynomials with integer coefficients and with constant term 0. Now suppose $f(x), g(x) \in \Z[x]$; write
    \begin{align*}
        f(x) &= a_0 + a_1x + \cdots + a_mx^m\\
        g(x) &= b_0 + b_1x + \cdots + b_nx^n
    \end{align*}
    where $a_i, b_i \in \Z$ and $m$ and $n$ are positive integers. Note that
    \[
        f(x)g(x) = a_0b_0 + (a_1b_0+a_0b_1)x + \cdots.
    \]
    Now if $f(x)g(x) \in I$, this means that $a_0b_0 = 0$. Hence either $a_0 = 0$ or $b_0 = 0$, meaning that either $f(x)$ has zero constant term (so $f(x) \in I$) or $g(x)$ has zero constant term (so $g(x) \in I$). Thus $I$ is prime.

    \item \begin{partquestions}{\roman*}
        \item Let $f(x), g(x) \in \Z[x]$. Then
        \[
            \phi(f(x) + g(x)) = f(-2) + g(-2) = \phi(f(x)) + \phi(g(x))
        \]
        and
        \[
            \phi(f(x)g(x)) = f(-2)g(-2) = \phi(f(x))\phi(g(x))
        \]
        so $\phi$ is a ring homomorphism.

        \item Note that
        \begin{align*}
            \ker\phi &= \{f(x) \in \Z[x] \vert \phi(f(x)) = 0\}\\
            &= \{f(x) \in \Z[x] \vert f(-2) = 0\}\\
            &= I.
        \end{align*}
        \myref{prop-kernel-is-an-ideal} tells us that $\ker\phi$ is an ideal of $\Z[x]$, so $I$ is an ideal of $\Z[x]$.

        \item We first show that $\phi$ is surjective. Let $n \in \Z$, note that $n$ is a degree zero polynomial, so $n \in \Z[x]$. Clearly $\phi(n) = n$ so $n$ is its own pre-image. Therefore $\im\phi = \Z$.
        
        By FRIT (\myref{thrm-ring-isomorphism-1}),
        \[
            \Z[x]/I \cong \Z.
        \]
        Note that $\Z$ is an integral domain but not a field. Thus $I$ is prime but not maximal.
    \end{partquestions}
\end{questions}


\printbibliography[heading=bibintoc, title={References and Bibliography}]
\printindex


\end{document}
