\section{Polynomial Rings}
\begin{questions}
    \item Let $p(x), q(x) \in R$. Note
    \[
        \phi_a(p(x)+q(x)) = p(a) + q(a) = \phi_a(p(x)) + \phi_a(q(x))
    \]
    and
    \[
        \phi_a(p(x)q(x)) = p(a)q(a) = \phi_a(p(x))\phi_a(q(x))
    \]
    so $\phi_a$ is indeed a homomorphism.

    \item For simplicity let
    \begin{align*}
        f(x)(g(x)h(x)) &= \sum_{k=0}^{m+n+l}u_kx^k\\
        (f(x)g(x))h(x) &= \sum_{k=0}^{m+n+l}v_kx^k
    \end{align*}
    for some $u_k, v_k \in \R$.
    \begin{itemize}
        \item On one hand,
        \begin{align*}
            u_k &= \sum_{r+s=k}\left(a_r\left(\sum_{p+q=s}b_pc_q\right)\right) & (\text{Definition of polynomial multiplication})\\
            &= \sum_{r+s=k}\left(\sum_{p+q=s}a_rb_pc_q\right)\\
            &= \sum_{p+q+r=k}a_rb_pc_q\\
            &= \sum_{p+q+r=k}a_pb_qc_r & (\text{Order of }p,\;q,\;r \text{ is arbitrary})
        \end{align*}

        \item On another hand,
        \begin{align*}
            v_k &= \sum_{r+s=k}\left(\left(\sum_{p+q=r}a_pb_q\right)c_s\right) & (\text{Definition of polynomial multiplication})\\
            &= \sum_{r+s=k}\left(\sum_{p+q=r}a_pb_qc_s\right)\\
            &= \sum_{p+q+s=k}a_pb_qc_s\\
            &= \sum_{p+q+r=k}a_pb_qc_r & (s\text{ is dummy variable})
        \end{align*}
    \end{itemize}
    Therefore $u_k = v_k$ for all $k$, meaning that $f(x)(g(x)h(x)) = (f(x)g(x))h(x)$.

    \item We note
    \begin{align*}
        \Q[\sqrt[3]{2}] &= \left\{a_0 + a_1\sqrt[3]{2} + a_2\left(\sqrt[3]{2}\right)^2 + a_3\left(\sqrt[3]{2}\right)^3 + \cdots + a_n\left(\sqrt[3]{2}\right)^n \vert a_i \in \Q\right\}\\
        &= \left\{a_0 + a_1\sqrt[3]{2} + a_2\sqrt[3]{4} + 2a_3 + \cdots + a_n\left(\sqrt[3]{2}\right)^n \vert a_i \in \Q\right\}\\
        &= \left\{(a_0 + 2a_3 + \cdots) + (a_1 + 2a_4 + \cdots)\sqrt[3]{2} + (a_2 + 2a_5 + \cdots)\sqrt[3]{4} \vert a_i \in \Q\right\}\\
        &= \left\{a + b\sqrt[3]{2} + c\sqrt[3]{4} \vert a,b,c \in \Q\right\}
    \end{align*}
    which establishes the required result.

    \item One example would be $x^5$. Essentially any polynomial where the highest term is $x^5$ would work.

    \item Let $f(x) = x^2 - 1 \in \Z_4[x]$. Note that
    \begin{align*}
        f(0) &= 0^2 - 1 = -1 = 3 \neq 0,\\
        f(1) &= 1^2 - 1 = 0,\\
        f(2) &= 2^2 - 1 = 3 \neq 0, \text{ and}\\
        f(3) &= 3^3 - 1 = 8 = 0
    \end{align*}
    so the zeroes of $f(x)$ are 1 and 3.

    \item \begin{partquestions}{\alph*}
        \item Suppose $r \in R$. If $r = 0$, then it is immediately a constant polynomial of $R[x]$ by definition. Otherwise, may interpret $r$ as a degree 0 polynomial in $R[x]$, which means $r$ is a constant polynomial of $R[x]$.

        \item Suppose $f(x)$ is a constant polynomial in $R[x]$. If $f(x) = 0$ then clearly $f(x) \in R$. Otherwise it takes the form $f(x) = a_0$ for some $a_0 \in R$. Thus clearly $f(x) \in R$.
    \end{partquestions}

    \item \begin{partquestions}{\alph*}
        \item We first suppose $D$ is a ring with identity 1. We may see 1 as a degree 0 polynomial in $D[x]$. Now for any polynomial $f(x) = a_0+a_1x+a_2x^2+\cdots+a_nx^n$ in $D[x]$ we have
        \begin{align*}
            (1)(f(x)) &= (1)(a_0+a_1x+\cdots+a_nx^n)\\
            &= (1a_0)+(1a_1)x+\cdots+(1a_n)x^n\\
            &= a_0+a_1x+\cdots+a_nx^n\\
            &= f(x)
        \end{align*}
        and
        \begin{align*}
            (f(x))(1) &= (a_0+a_1x+\cdots+a_nx^n)(1)\\
            &= (a_{0}1)+(a_{1}1)x+\cdots+(a_{n}1)x^n\\
            &= a_0+a_1x+\cdots+a_nx^n\\
            &= f(x)
        \end{align*}
        so 1 is the identity in $D[x]$.

        Now suppose $D[x]$ is a ring with identity $\id(x)$. We see that $\id(x)f(x) = f(x)\id(x) = f(x)$ for any $f(x) \in D[x]$, meaning $\deg(\id(x)f(x)) = \deg(f(x))$. We note that $\deg(\id(x)f(x)) = \deg(\id(x)) + \deg(f(x))$, this means that $\id$ has degree 0, meaning we may write $\id(x) = e$ for some $e \in D$. Now if $f(x) = a$ for some $a \in D$, we must have $\id(x)f(x) = ae = a$ and $f(x)\id(x) = ea = a$, meaning that $e$ is the identity of $D$.

        \item Suppose first that $D$ is a commutative ring. Let
        \[
            f(x) = \sum_{i=0}^ma_ix^i \text{ and } g(x) = \sum_{j=0}^nb_jx^j
        \]
        be polynomials in $D[x]$. Then
        \begin{align*}
            f(x)g(x) &= \left(\sum_{i=0}^ma_ix^i\right)\left(\sum_{j=0}^nb_jx^j\right)\\
            &= \sum_{k=0}^{m+n}\left(\sum_{i=0}^k a_{i}b_{k-i}\right)x^k\\
            &= \sum_{k=0}^{m+n}\left(\sum_{i=0}^k b_{k-i}a_{i}\right)x^k\\
            &= \sum_{k=0}^{m+n}\left(\sum_{i=0}^k b_{i}a_{k-i}\right)x^k\\
            &= \left(\sum_{j=0}^nb_jx^j\right)\left(\sum_{i=0}^ma_ix^i\right)\\
            &= g(x)f(x)
        \end{align*}
        which therefore means that $D[x]$ is commutative.

        Now suppose $D[x]$ is commutative. Consider the polynomials $f(x) = a$ and $g(x) = b$ where $a$ and $b$ are non-zero. We thus have $ab = f(x)g(x) = g(x)f(x) = ba$ for all $a,b \in D$ which means $D$ is commutative.
    \end{partquestions}

    \item For brevity, let
    \begin{align*}
        f(x) &= \sum_{i=0}^ma_ix^i,\\
        g(x) &= \sum_{i=0}^nb_ix^i
    \end{align*}
    be polynomials in $R[x]$. Without loss of generality assume $m \geq n$, and define $b_i = 0$ for $i > n$. Then
    \begin{align*}
        \phi(f(x) + g(x)) &= \phi\left(\sum_{i=0}^m (a_i+b_i)x^i\right)\\
        &= \sum_{i=0}^m (a_i+b_i + I)x^i\\
        &= \sum_{i=0}^m ((a_i + I) + (b_i + I))x^i\\
        &= \sum_{i=0}^m (a_i + I)x^i + \sum_{i=0}^m (b_i + I)x^i\\
        &= \sum_{i=0}^m (a_i + I)x^i + \sum_{i=0}^n (b_i + I)x^i & (\text{since } b_i = 0\text{ for } i > n)\\
        &= \phi(f(x)) + \phi(g(x))
    \end{align*}
    and
    \begin{align*}
        \phi(f(x)g(x)) &= \phi\left(\sum_{i=0}^{m+n}\left(\sum_{j=0}^i a_jb_{i-j}\right)x^i\right)\\
        &= \sum_{i=0}^{m+n}\left(\left(\sum_{j=0}^i a_jb_{i-j}\right) + I\right)x^i\\
        &= \sum_{i=0}^{m+n}\left(\sum_{j=0}^i (a_jb_{i-j} + I)\right)x^i\\
        &= \sum_{i=0}^{m+n}\left(\sum_{j=0}^i ((a_j+I)(b_{i-j}+I))\right)x^i\\
        &= \left(\sum_{i=0}^m(a_i+I)x^i\right)\left(\sum_{i=0}^n(b_i+I)x^i\right)\\
        &= \phi(f(x))\phi(g(x))
    \end{align*}
    so $\phi$ is indeed a ring homomorphism.

    \item We note
    \begin{align*}
        &P \text{ is a prime ideal of }R\\
        \iff&R/P \text{ is an integral domain} & (\myref{thrm-prime-ideal-iff-quotient-ring-is-integral-domain})\\
        \iff&(R/P)[x] \text{ is an integral domain} & (\myref{thrm-integral-domain-iff-polynomial-ring-is-integral-domain})\\
        \iff&R[x]/P[x] \text{ is an integral domain} & (\myref{prop-polynomial-ring-quotient-ideal-polynomial-ring-cong-quotient-polynomial-ring})\\
        \iff&P[x]\text{ is a prime ideal of }R[x] & (\myref{thrm-prime-ideal-iff-quotient-ring-is-integral-domain})
    \end{align*}
    proving the theorem.

    \item Note that
    \begin{align*}
        3x^4 + 3x^3 + 4x^2 + 3x + 3 &= 3x^2(x^2+2x+3) - 3x(x^2+2x+3)\\
        &\quad\quad+ 1(x^2+2x+3) + 10x\\
        &= (3x^2-3x+1)(x^2+2x+3) + 10x\\
        &= (3x^2+2x+1)(x^2+2x+3) + 0x\\
        &= (3x^2+2x+1)(x^2+2x+3)
    \end{align*}
    so dividing $3x^4 + 3x^3 + 4x^2 + 3x + 3$ by $x^2+2x+3$ yields $3x^2+2x+1$. Thus two factors of $3x^4 + 3x^3 + 4x^2 + 3x + 3$ are $x^2+2x+3$ and $3x^2+2x+1$.
\end{questions}
