\section{Rings and Encryption}
\begin{questions}
    \item We claim that $x+2$ is its own inverse when $N = 2$ and $p = 4$, since
    \begin{align*}
        (x+2)^2 &= x^2 + 4x + 4\\
        &= x^0 + 4x + 4 & (\text{reduce powers modulo } N = 2)\\
        &= 4x + 5\\
        &= 1 & (\text{reduce coefficients modulo } p = 4)
    \end{align*}

    \item Suppose $f(x) \in \mathcal{L}(k,k)$. Then $f(x)$ has exactly $k$ coefficients of 1 and $k$ coefficients of -1. Therefore, evaluating $f(1)$ yields a result of 0.
    
    The equivalent requirement for $f(x)$ to have an inverse modulo $p$, say $F_p(x)$, is that $F_p(x)f(x) = 1$ within the ring $\Z_p[x]$. However $f(1) = 0$, meaning that $f(x)$ has a zero in $\Z_p$, while the constant polynomial 1 does not. Therefore, there cannot be a polynomial $F_p(x)$ such that $F_p(x)f(x) = 1$ in $\Z_p[x]$, i.e. $f(x)$ does not have an inverse.
\end{questions}
