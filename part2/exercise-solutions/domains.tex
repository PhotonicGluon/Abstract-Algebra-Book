\section{Domains and Factorization}
\begin{questions}
    \item For the forward direction, suppose $a \vert b$ and $b \vert a$. Then there exist elements $u, v \in D$ such that $b = au$ and $a = bv$. So
    \[
        a = bv = (au)v = auv
    \]
    which means $uv = 1$ by cancellation law (\myref{prop-domain-cancellation-law}). Thus $u$ (and also $v$) is a unit, which shows that $a$ and $b$ are associates.

    For the reverse direction, if $a$ and $b$ are associates, then there exists a unit $u \in D$ such that $a = bu$, which also means that $b = au^{-1}$. Therefore one clearly sees that $a \vert b$ and $b \vert a$.

    \item Suppose $r$ is an irreducible and $q$ an associate of $r$, meaning there exists a unit $u$ in the integral domain (say $D$) such that $q = ru$. Suppose on the contrary that $q$ is reducible, meaning $q = ab$ where neither $a$ nor $b$ are units in $D$. Then $ru = ab$ which means $r = (u^{-1}a)b$. As $r$ is irreducible we know that either $b$ is a unit (contradicting the assumption) or $u^{-1}a$ is a unit. Since the product of units is a unit, thus $u(u^{-1}a) = a$ is a unit, contradicting the assumption. Therefore $q = ru$ is irreducible.

    \item \begin{partquestions}{\alph*}
        \item Note $2 = (1+i)(1-i)$ and $5 = (2+i)(2-i)$, and one can easily verify that each of these factors do not have a norm of $\pm1$. Therefore 2 and 5 are reducible in $\Z[i]$, so they cannot be prime (contrapositive of \myref{thrm-in-integral-domain-primes-are-irreducibles}).

        \item We first note $N(x) \geq 0$ for all $x \in \Z[i]$.
        \begin{partquestions}{\roman*}
            \item Without loss of generality we consider only $a$. We note that
            \begin{itemize}
                \item if $a \equiv 0 \pmod4$ then $a^2 \equiv 0 \pmod4$;
                \item if $a \equiv 1 \pmod4$ then $a^2 \equiv 1 \pmod4$;
                \item if $a \equiv 2 \pmod4$ then $a^2 \equiv 4 \equiv 0 \pmod4$; and
                \item if $a \equiv 3 \pmod4$ then $a^2 \equiv 9 \equiv 1 \pmod4$.
            \end{itemize}
            So the sum of two squares modulo 4 can only be 0, 1, or 2. Thus it is impossible for $a^2+b^2 \equiv 3\pmod4$.

            \item Seeking a contradiction, suppose $3 = pq$ where $p,q\in\Z[i]$ are both non-units. Then
            \[
                9 = N(3) = N(pq) = N(p)N(q).
            \]
            Since $p$ and $q$ are non-units we must have $N(p) = N(q) = 3$ (since otherwise $N(p) = 1$ which means $p$ is a unit by \myref{prop-properties-of-quadratic-integer-norm}). So if $p = a+bi$ then $a^2+b^2 = 3$. But we just proved in \textbf{(b)(i)} that this is impossible. Thus 3 is irreducible.

            \item Let $x = m+ni$, so $N(x) = m^2 + n^2$. Given $9 \vert N(x)$, we know $m^2 + n^2 \equiv 0 \pmod9$. We note the possibilities of squares modulo 9.
            \begin{table}[H]
                \centering
                \begin{tabular}{|l|l|l|}
                    \hline
                    $k\mod9$ & $k^2$ & $k^2\mod9$ \\ \hline
                    0 & 0 & 0 \\ \hline
                    1 & 1 & 1 \\ \hline
                    2 & 4 & 4 \\ \hline
                    3 & 9 & 0 \\ \hline
                    4 & 16 & 7 \\ \hline
                    5 & 25 & 7 \\ \hline
                    6 & 36 & 0 \\ \hline
                    7 & 49 & 4 \\ \hline
                    8 & 64 & 1 \\ \hline
                \end{tabular}
            \end{table}
            One may then verify that the set of pairs of $(m,n)$ that results in $m^2 + n^2 \equiv 0 \pmod9$ is
            \[
                \left\{(0,0),(0,3),(0,6),(3,0),(3,3),(3,6),(6,0),(6,3),(6,6)\right\}.
            \]
            In each case, both $m$ and $n$ are divisible by 3. Say $m = 3p$ and $n = 3q$ for some integers $p$ and $q$. Then $x = m + ni = 3p + (3q)i = 3(p+qi)$ which therefore means $3 \vert x$.

            \item Suppose $3 \vert uv$ for some $u,v\in D$. This means $uv = 3w$ for some $w \in D$. Note
            \[
                N(u)N(v) = N(uv) = N(3w) = 9N(w)
            \]
            and $9 = 3\times3$. We have 3 cases.
            \begin{enumerate}[label=\arabic*.]
                \item If $9 \vert N(u)$ we may use \textbf{(b)(iii)} to see that $3 \vert u$.
                \item Otherwise, if $9 \vert N(v)$ we may again use \textbf{(b)(iii)} to see that $3 \vert v$.
                \item Otherwise, we must have $3 \vert N(u)$ and $3 \vert N(v)$. Let's focus on $3 \vert N(u)$ only; let $u = x+yi$ for some integers $x$ and $y$. Then $3 \vert x^2+y^2$. Note for any integer $n$,
                \begin{itemize}
                    \item $n \equiv 0 \pmod3$ means $n^2 \equiv 0 \pmod3$;
                    \item $n \equiv 1 \pmod3$ means $n^2 \equiv 1 \pmod3$; and
                    \item $n \equiv 2 \pmod3$ means $n^2 \equiv 4 \equiv 1 \pmod3$.
                \end{itemize}
                So for $x^2+y^2$ to be divisible by 3, both $x$ and $y$ must be multiples of 3. Let $x = 3m$ and $y = 3n$ for some integers $n$; we see
                \[
                    N(u) = x^2+y^2 = (3m)^2 + (3n)^2 = 9(m^2+n^2)
                \]
                which means $9 \vert N(u)$. Finally use \textbf{(b)(iii)} to see that $3 \vert u$.
            \end{enumerate}
            Hence in any case, $3 \vert uv$ means $3 \vert u$ or $3 \vert v$, meaning 3 is prime.
        \end{partquestions}
    \end{partquestions}

    \item Suppose $N(x) = p$ where $p$ is a prime number, and $x = ab$ where $a,b \in \Z[\sqrt{d}]$. Taking norms yields $N(x) = N(ab) = N(a)N(b) = p$. But since $p$ is prime, one of $N(a)$ or $N(b)$ is 1, which by \myref{prop-properties-of-quadratic-integer-norm}, statement 3, we know that $a$ or $b$ is a unit. Therefore $x$ is irreducible.

    \item Suppose $\frac{a_1}{b_1} = \frac{a_2}{b_2}$ and $\frac{c_1}{d_1} = \frac{c_2}{d_2}$. We show that $\frac{a_1}{b_1} \times \frac{c_1}{d_1} = \frac{a_2}{b_2} \times \frac{c_2}{d_2}$, i.e. $\frac{a_1c_1}{b_1d_1} = \frac{a_2c_2}{b_2c_2}$, which is equivalent to showing that $(a_1c_1, b_1d_1) \mathrel{\sim} (a_2c_2, b_2d_2)$ by \myref{thrm-equivalence-class-equivalence}, i.e.
    \[
        a_1c_1b_2d_2 = b_1d_1a_2c_2.
    \]
    As $\frac{a_1}{b_1} = \frac{a_2}{b_2}$, we know that $(a_1, b_1) \mathrel{\sim} (a_2, b_2)$ (\myref{thrm-equivalence-class-equivalence}) which means $a_1b_2 = b_1a_2$. Similarly, because $\frac{c_1}{d_1} = \frac{c_2}{d_2}$ thus $c_1d_2 = d_1c_2$. Hence
    \begin{align*}
        a_1c_1b_2d_2 &= a_1b_2c_1d_2\\
        &= (a_1b_2)(c_1d_2)\\
        &= (b_1a_2)(d_1c_2)\\
        &= b_1d_1a_2c_2.
    \end{align*}
    so we have shown that $\frac{a_1}{b_1} \times \frac{c_1}{d_1} = \frac{a_2}{b_2} \times \frac{c_2}{d_2}$, meaning multiplication is well-defined.

    We now show that multiplication is commutative. Let $\frac ab, \frac cd \in \Frac{D}$. Then
    \begin{align*}
        \frac ab \times \frac cd &= \frac{ac}{bd}\\
        &= \frac{ca}{db} & (\text{since } D \text{ is commutative})\\
        &= \frac cd \times \frac ab
    \end{align*}
    so multiplication is commutative.

    \item \begin{partquestions}{\alph*}
        \item Disprove. One sees $1 \in \Z[x]$ and $x \in \Z[x]$ but $\frac1x \notin \Q[x]$.

        \item Prove. Define $\phi: \Frac{k\Z} \to \Q$ by $\phi([(ka, kb)]) = \frac ab$. We show that this is a well-defined isomorphism.
        \begin{itemize}
            \item \textbf{Well-defined}: Suppose $[(ka, kb)], [(kc, kd)] \in \Frac{k\Z}$ where $[(ka,kb)]=[(kc,kd)]$. Then $(ka, kb) \mathrel{\sim} (kc, kd)$ by \myref{thrm-equivalence-class-equivalence}, i.e. $k^2ad = k^2bc$. Hence one sees clearly that $\frac ab = \frac cd$. Therefore,
            \[
                \phi([(ka,kb)]) = \frac ab = \frac cd = \phi([(kc, kd)])
            \]
            and so $\phi$ is well-defined.

            \item \textbf{Homomorphism}: Let $[(ka, kb)], [(kc, kd)] \in \Frac{k\Z}$. Note
            \begin{align*}
                \phi([(ka,kb)] + [(kc,kd)]) &= \phi([(k^2(ad+bc), k^2bd)])\\
                &= \frac{k(ad+bc)}{kbd}\\
                &= \frac{k^2(ad+bc)}{k^2bd}\\
                &= \frac {ka}{kb} + \frac {kc}{kd}\\
                &= \phi([(ka,kb)]) + \phi([(kc,kd)])
            \end{align*}
            and
            \begin{align*}
                \phi([(ka,kb)] \times [(kc,kd)]) &= \phi([(k^2ac, k^2bd)])\\
                &= \frac{kac}{kbd}\\
                &= \frac{k^2ac}{k^2bd}\\
                &= \frac {ka}{kb} \times \frac {kc}{kd}\\
                &= \phi([(ka,kb)]) \times \phi([(kc,kd)])
            \end{align*}
            which shows that $\phi$ is a homomorphism.

            \item \textbf{Injective}: Let $[(ka, kb)], [(kc, kd)] \in \Frac{k\Z}$ such that $\phi([(ka, kb)]) = \phi([(kc, kd)])$. Then $\frac ab = \frac cd$, meaning $ad = bc$. Hence $(ka, kb) \mathrel{\sim} (kc, kd)$ by definition of the equivalence relation on $\Frac{k\Z}$, and so $[(ka, kb)] = [(kc, kd)]$ by \myref{thrm-equivalence-class-equivalence} again. Hence $\phi$ is injective.

            \item \textbf{Surjective}: Suppose $\frac ab \in \Q$ where $a, b \in \Z$ such that $b \neq 0$. Then note that $[(ka, kb)]$ is its pre-image since
            \[
                \phi([(ka, kb)]) = \frac ab
            \]
            which proves that $\phi$ is surjective.
        \end{itemize}
        Therefore $\phi$ is a well-defined ring isomorphism, meaning $\Frac{k\Z} \cong \Q$.
    \end{partquestions}

    \item For the forward direction, we prove by contrapositive. If $a$ is a unit then $\princ{b} = \princ{ab}$ by \myref{prop-principal-ideals-equal-iff-associates}, so $\princ{ab} \not\subset \princ{b}$.

    For the reverse direction, we again consider contrapositive. Assume $\princ{ab}$ is not a proper subset of $\princ{b}$, meaning $\princ{b} \subseteq \princ{ab}$. But one clearly sees that $\princ{ab} \subseteq \princ{b}$. Thus $\princ{b} = \princ{ab}$ which means $a$ is a unit, again by \myref{prop-principal-ideals-equal-iff-associates}.

    \item We prove the contrapositive of the forward direction. Suppose $f(x)$ is reducible in $D[x]$. Then, since $f(x)$ is primitive, $f(x) = p(x)q(x)$ for some non-unit polynomials $p(x), q(x) \in D[x]$. Therefore $p(x), q(x) \in \Frac{D}$, which are non-units in $\Frac{D}$. Thus $f(x) = p(x)q(x)$ for some non-unit polynomials in $p(x), q(x) \in \Frac{D}$, so $f(x)$ is reducible in $\Frac{D}$.

    We now prove the contrapositive of the reverse direction. Suppose $f(x)$ is reducible in $\Frac{D}$, so there exists non-constant polynomials $p(x), q(x) \in \Frac{D}$ such that $f(x) = p(x)q(x)$. Then there exists polynomials $P(x), Q(x) \in D[x]$ such that $f(x) = P(x)Q(x)$ and $\deg P(x) = \deg p(x)$ and $\deg Q(x) = \deg q(x)$ by \myref{lemma-reducible-in-field-of-fractions-means-reducible-in-UFD}. Hence $f(x)$ is reducible in $D[x]$.

    \item For a field $F$, choose the norm to be $N(x) = 1$ for any $x \in F$. Clearly \textbf{EF1} is satisfied since $1 = N(x) \leq N(xy) = 1$ for any $x,y\in F$. For \textbf{EF2}, for any $n, d \in F$, we can write $n = (nd^{-1})d + 0$. Note $nd^{-1}$ exists since we are in a field, so \textbf{EF2} is satisfied. Therefore $F$ is a Euclidean domain.

    \item We note
    \begin{itemize}
        \item all fields are Euclidean domains by \myref{thrm-field-is-euclidean-domain};
        \item all Euclidean domains are PIDs by \myref{thrm-euclidean-domain-is-PID}; and
        \item all PIDs are UFDs by \myref{thrm-PID-is-UFD}.
    \end{itemize}
\end{questions}
