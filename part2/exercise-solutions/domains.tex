\section{Domains and Factorization}
\begin{questions}
    \item For the forward direction, suppose $a \vert b$ and $b \vert a$. Then there exist elements $u, v \in D$ such that $b = au$ and $a = bv$. So
    \[
        a = bv = (au)v = auv
    \]
    which means $uv = 1$ by cancellation law (\myref{prop-domain-cancellation-law}). Thus $u$ (and also $v$) is a unit, which shows that $a$ and $b$ are associates.
    
    For the reverse direction, if $a$ and $b$ are associates, then there exists a unit $u \in D$ such that $a = bu$, which also means that $b = au^{-1}$. Therefore one clearly sees that $a \vert b$ and $b \vert a$.

    \item Suppose $r$ is an irreducible and $q$ an associate of $r$, meaning there exists a unit $u$ in the integral domain (say $D$) such that $q = ru$. Suppose on the contrary that $q$ is reducible, meaning $q = ab$ where neither $a$ nor $b$ are units in $D$. Then $ru = ab$ which means $r = (u^{-1}a)b$. As $r$ is irreducible we know that either $b$ is a unit (contradicting the assumption) or $u^{-1}a$ is a unit. Since the product of units is a unit, thus $u(u^{-1}a) = a$ is a unit, contradicting the assumption. Therefore $q = ru$ is irreducible.
    
    \item % TODO: Add solution

    \item Suppose $N(x) = \pm p$ where $p$ is a prime number, and $x = ab$. Taking norms yields $N(x) = N(ab) = N(a)N(b) = \pm p$. But since $p$ is prime, one of $N(a)$ or $N(b)$ is 1 (or -1), which by \myref{prop-properties-of-quadratic-integer-norm}, statement 3, we know that $a$ or $b$ is a unit. Therefore $x$ is irreducible.
\end{questions}
