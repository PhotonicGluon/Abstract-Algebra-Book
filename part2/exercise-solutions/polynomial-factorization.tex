\section{Factorization of Polynomials}
\begin{questions}
    \item \begin{partquestions}{\alph*}
        \item Not irreducible since $3x^2 - 6 = 3(x^2-2)$ and both 3 and $x^2-2$ are non-units in $\Z$.
        \item Irreducible in $\Q$ since we cannot express $3x^2-6$ as a product of two polynomial of smaller degree that are in $\Q[x]$ (by \myref{thrm-irreducible-iff-not-expressable-as-product-of-smaller-polynomials}).
        \item Reducible in $\R$ since $3x^2 - 6 = (3x-3\sqrt2)(x+\sqrt2)$ and both $3x-3\sqrt2, x+\sqrt2 \in \R[x]$.
    \end{partquestions}
    
    \item For brevity we compute all possible values of $f(x)$ for $x \in \Z_7$ and then check for zeroes.
    \begin{table}[H]
        \centering
        \begin{tabular}{|l|l|l|l|l|l|}
            \hline
            $\boldsymbol{x}$ & $\boldsymbol{f(x)}$ & $\boldsymbol{f(x) \mod2}$ & $\boldsymbol{f(x) \mod3}$ & $\boldsymbol{f(x) \mod5}$ & $\boldsymbol{f(x) \mod7}$ \\ \hline
            \textbf{0} & 9 & 1 & 0 & 4 & 2 \\ \hline
            \textbf{1} & 15 & 1 & 0 & 0 & 1 \\ \hline
            \textbf{2} & 33 &  & 0 & 3 & 5 \\ \hline
            \textbf{3} & 75 &  &  & 0 & 5 \\ \hline
            \textbf{4} & 153 &  &  & 3 & 6 \\ \hline
            \textbf{5} & 279 &  &  &  & 6 \\ \hline
            \textbf{6} & 465 &  &  &  & 3 \\ \hline
        \end{tabular}
    \end{table}
    We therefore see that $f(x)$ has zeroes in $\Z_3$ and $\Z_5$ and no zeroes in $\Z_2$ and $\Z_7$. Thus $f(x)$ is irreducible in $\Z_2$ and $\Z_7$ and reducible in $\Z_3$ and $\Z_5$.

    \item \begin{partquestions}{\alph*}
        \item Statement is the converse of \myref{thrm-reducible-over-Q-means-reducible-over-Z}, so it is true.
        \item Disprove. $2(x^2+1)$ has no zeroes in $\Q$ and so is irreducible in $\Q$ (\myref{thrm-degree-2-or-3-reducible-iff-has-zero}). But $2(x^2+1)$ is reducible in $\Z$ as $2 \times (x^2 + 1)$ where 2 and $x^2 + 1$ are both non-units.
    \end{partquestions}
\end{questions}
