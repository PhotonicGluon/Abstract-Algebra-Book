\section{Factorization of Polynomials}
\begin{questions}
    \item \begin{partquestions}{\alph*}
        \item Not irreducible since $3x^2 - 6 = 3(x^2-2)$ and both 3 and $x^2-2$ are non-units in $\Z$.
        \item Irreducible in $\Q$ since we cannot express $3x^2-6$ as a product of two polynomial of smaller degree that are in $\Q[x]$ (by \myref{thrm-irreducible-iff-not-expressable-as-product-of-smaller-polynomials}).
        \item Reducible in $\R$ since $3x^2 - 6 = (3x-3\sqrt2)(x+\sqrt2)$ and both $3x-3\sqrt2, x+\sqrt2 \in \R[x]$.
    \end{partquestions}

    \item For brevity we compute all possible values of $f(x)$ for $x \in \Z_7$ and then check for zeroes.
    \begin{table}[H]
        \centering
        \begin{tabular}{|l|l|l|l|l|l|}
            \hline
            $\boldsymbol{x}$ & $\boldsymbol{f(x)}$ & $\boldsymbol{f(x) \mod2}$ & $\boldsymbol{f(x) \mod3}$ & $\boldsymbol{f(x) \mod5}$ & $\boldsymbol{f(x) \mod7}$ \\ \hline
            \textbf{0} & 9 & 1 & 0 & 4 & 2 \\ \hline
            \textbf{1} & 15 & 1 & 0 & 0 & 1 \\ \hline
            \textbf{2} & 33 &  & 0 & 3 & 5 \\ \hline
            \textbf{3} & 75 &  &  & 0 & 5 \\ \hline
            \textbf{4} & 153 &  &  & 3 & 6 \\ \hline
            \textbf{5} & 279 &  &  &  & 6 \\ \hline
            \textbf{6} & 465 &  &  &  & 3 \\ \hline
        \end{tabular}
    \end{table}
    We therefore see that $f(x)$ has zeroes in $\Z_3$ and $\Z_5$ and no zeroes in $\Z_2$ and $\Z_7$. Thus $f(x)$ is irreducible over $\Z_2$ and $\Z_7$ and reducible in $\Z_3$ and $\Z_5$.

    \item \begin{partquestions}{\alph*}
        \item Statement is the converse of \myref{thrm-irreducible-over-Z-means-irreducible-over-Q}, so it is true.
        \item Disprove. $2(x^2+1)$ has no zeroes in $\Q$ and so is irreducible over $\Q$ (\myref{thrm-degree-2-or-3-irreducible-iff-has-no-zeroes}). But $2(x^2+1)$ is reducible in $\Z$ as $2 \times (x^2 + 1)$ where 2 and $x^2 + 1$ are both non-units.
    \end{partquestions}

    \item Clearly $1^2 + 2(1) + 1 = 4 \equiv 0 \pmod2$, so we consider $p > 3$. Consider the polynomial $f(x) = x^2 + 2x + 1$. Clearly $f(-1) = 0$ so $f(x)$ is reducible in $\Q$ (\myref{thrm-degree-2-or-3-irreducible-iff-has-no-zeroes}). Thus the converse of the Mod $p$ Irreducibility Test (\myref{thrm-mod-p-irreducibility-test}) tells us that $f(x)$ is reducible in $\Z_p$ (since reducing the coefficients modulo $p$ yields the same polynomial). Therefore $f(x) \in \Z_p[x]$ must have a zero (\myref{thrm-degree-2-or-3-irreducible-iff-has-no-zeroes}), i.e. there is an integer $n \in \Z_p$ such that $f(n) = n^2 + 2n + 1 = 0$ when working modulo $p$. That same $n$ satisfies $n^2 + 2n + 1 \equiv 0 \pmod{p}$ as required.

    \item We claim that $f(x) = x^n + 2$ is irreducible over $\Q$ for any $n \geq 1$. We note that 2 does not divide 1, 2 divides 2, and $2^2 = 4$ does not divide 2. Thus $f(x)$ is irreducible over $\Q$ by Eisenstein's Criterion (\myref{thrm-eisenstein-criterion}) with the prime 2.

    \item \begin{partquestions}{\roman*}
        \item Choose the prime 3. Clearly 3 does not divide 1, 3 divides 3, but $3^2 = 9$ does not divide 3. Thus Eisenstein's Criterion (\myref{thrm-eisenstein-criterion}) tells us that $f(x)$ is irreducible over $\Q$.

        \item One may use Eisenstein's Criterion (with the prime 2) on the given polynomial. We instead note that
        \[
            x^4 + 4x^3 + 6x^2 + 4x + 4 = (x+1)^4 + 3
        \]
        and since $x^4 + 3$ is irreducible thus the above polynomial is also irreducible over $\Q$. Note that the given polynomial is primitive, so it is also irreducible over $\Z$ (\myref{thrm-irreducible-over-Z-means-irreducible-over-Q}). Hence $x^4 + 4x^3 + 6x^2 + 4x + 4$ has no zeroes in $\Z$ by contrapositive of \myref{thrm-degree-above-1-reducible-if-has-zero}.

        \item We note we cannot use Eisenstein's Criterion directly, since the only prime that works on the constant term is 19, and $19 \nmid 32$. We need to consider a substitution.

        One may see that, after some trial and error, that
        \begin{align*}
            x^4 - 8x^3 + 24x^2 - 32x + 19 &= (x^4 - 8x^3 + 24x^2 - 32x + 16) + 3\\
            &= (2-x)^4 + 3
        \end{align*}
        and since $x^4 + 3$ is irreducible by \textbf{(i)}, thus $x^4 - 8x^3 + 24x^2 - 32x + 19$ is irreducible by transformation.
    \end{partquestions}

    \item We prove the three requirements for a ring isomorphism.
    \begin{itemize}
        \item \textbf{Homomorphism}: One sees that
        \begin{align*}
            \phi(f(x) + g(x)) &= \phi\left(\sum_{i=0}^m(a_i+b_i)x^i\right)\\
            &= \sum_{i=0}^m(a_i+b_i)(kx)^i\\
            &= \left(\sum_{i=0}^ma_i(kx)^i\right) + \left(\sum_{i=0}^mb_i(kx)^i\right)\\
            &= \left(\sum_{i=0}^ma_i(kx)^i\right) + \left(\sum_{i=0}^nb_i(kx)^i\right) & (\because b_i = 0 \text{ for } i > n)\\
            &= \phi(f(x)) + \phi(g(x))
        \end{align*}
        and
        \begin{align*}
            \phi(f(x)g(x)) &= \phi\left(\sum_{r=0}^{m+n}\left(\sum_{i=0}^ra_ib_{r-i}\right)x^r\right)\\
            &= \sum_{r=0}^{m+n}\left(\sum_{i=0}^ra_ib_{r-i}\right)(kx)^r\\
            &= \left(\sum_{r=0}^ma_r(kx)^r\right)\left(\sum_{r=0}^nb_r(kx)^r\right)\\
            &= \phi(f(x))\phi(g(x))
        \end{align*}
        so $\phi$ is a ring homomorphism.

        \item \textbf{Injective}: Suppose $\phi(f(x)) = \phi(g(x))$. Then
        \begin{align*}
            \sum_{i=0}^m(a_ik^i)x^i &= \sum_{i=0}^ma_i(kx)^i\\
            &= \sum_{i=0}^nb_i(kx)^i\\
            &= \sum_{i=0}^n(b_ik^i)x^i
        \end{align*}
        which, by comparing coefficients, we see $a_i = b_i$ for all $0 \leq i \leq m$. Therefore $f(x) = g(x)$, meaning $\phi$ is injective.

        \item \textbf{Surjective}: Let $p(x) = c_0 + c_1x + \cdots + c_rx^r$ be a polynomial in $D[x]$. Since $k$ is a unit, therefore $k^{-1}$ exists. Note that $q(x) = c_0 + c_1(k^{-1}x) + \cdots + c_r(k^{-1}x)^r$ is also a polynomial in $D[x]$. Observe
        \begin{align*}
            \phi(q(x)) &= c_0 + c_1(k(k^{-1}x)) + \cdots + c_r(k(k^{-1}x))^r\\
            &= c_0 + c_1x + \cdots + c_rx^r\\
            &= p(x)
        \end{align*}
        so any $p(x) \in D[x]$ has a pre-image under $\phi$.
    \end{itemize}
    Therefore $\phi$ is an isomorphism.

    \item Since $p(x)$ is irreducible we know $D[x]/\princ{p(x)}$ is a field (\myref{corollary-polynomial-quotient-by-principal-ideal-is-field-iff-polynomial-irreducible}), which is a integral domain (\myref{prop-field-is-integral-domain}), and so $\princ{p(x)}$ is a prime ideal (\myref{thrm-prime-ideal-iff-quotient-ring-is-integral-domain}). As $p(x) = a(x)b(x)$ therefore $a(x)b(x) \in \princ{p(x)}$. Hence $a(x) \in \princ{p(x)}$ or $b(x) \in \princ{p(x)}$ by definition of prime ideal. So $a(x) = k(x)p(x)$ or $b(x) = k(x)p(x)$ for some $k(x) \in D[x]$, meaning $p(x) \vert a(x)$ or $p(x) \vert b(x)$.

    \item \begin{partquestions}{\roman*}
        \item Let $p(x) = x^2 + 2 \in \Z_5[x]$. One sees that
        \begin{itemize}
            \item $p(0) = 0^2 + 2 = 2 \neq 0$;
            \item $p(1) = 1^2 + 2 = 3 \neq 0$;
            \item $p(2) = 2^2 + 2 = 6 = 1 \neq 0$;
            \item $p(3) = 3^2 + 2 = 11 = 1 \neq 0$; and
            \item $p(4) = 4^2 + 2 = 18 = 3 \neq 0$,
        \end{itemize}
        so $p(x)$ is irreducible in $\Z_5$ by \myref{thrm-degree-2-or-3-irreducible-iff-has-no-zeroes}.

        \item $F = Z_5[x]/\princ{p(x)}$ is a field of 25 elements.

        \item \begin{partquestions}{\alph*}
            \item We see
            \begin{align*}
                \left(2x+3 + I\right) + \left(4x^2+3x+2 + I\right) &= (4x^2+5x+5) + I\\
                &= 4(x^2+2) + 5x - 3 + I\\
                &= 5x - 3 + I\\
                &= 0x + 2 + I
            \end{align*}
            so $(f(x) + I) + (g(x) + I) = 2 + I$.

            \item We note
            \begin{align*}
                (2x+3 + I)(4x^2+3x+2 + I) &= (8 x^3 + 18 x^2 + 13 x + 6) + I\\
                &= \left((8x + 18)(x^2+2) - 3x - 30\right) + I\\
                &= -3x - 30 + I\\
                &= 2x + I
            \end{align*}
            so $(f(x) + I)(g(x) + I) = 2x + I$.
        \end{partquestions}
    \end{partquestions}

    \item \begin{partquestions}{\alph*}
        \item $x \times x = x^2 = 1(x^2+1) - 1 = -1 = 2$.

        \item We note
        \begin{align*}
            (x+1)(2x+1) &= 2x^2 + 3x + 1\\
            &= 2x^2 + 0x + 1\\
            &= 2x^2 + 1\\
            &= 2(2) + 1 & (\text{since } x^2 = 2)\\
            &= 5\\
            &= 2.
        \end{align*}

        \item We see
        \begin{align*}
            (2x+2)(x+2) &= 2\left((x+1)(x+2)\right)\\
            &= 2(1) & (\text{since} (x+1)(x+2) = 1 \text{ from table})\\
            &= 2.
        \end{align*}

        \item Note
        \begin{align*}
            (2x+2)(2x+1) &= 2(x+1)(2x+1)\\
            &= 2(2) & (\text{from }\textbf{(b)})\\
            &= 4\\
            &= 1.
        \end{align*}
    \end{partquestions}
\end{questions}
