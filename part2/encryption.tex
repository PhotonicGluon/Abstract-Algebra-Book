\chapter{Rings and Encryption}
Rings have practical use in the real world. Recently, with the threat of quantum computers breaking traditional encryption such as the Rivest-Shamir-Adleman (RSA) encryption scheme, post-quantum encryption schemes were developed. We look at the NTRU encryption scheme as described in \cite{hoffstein_pipher_silverman_1996}, a polynomial-based and allegedly post-quantum resistant encryption scheme.

\section{Cryptosystems}\index{cryptosystem}
The study and use of methods for secure information sharing when third parties are present is known as cryptography. Within cryptography, there are different cryptosystems, which are systems that has their own encryption, decryption, and key generation methods.
\begin{itemize}
    \item Encryption refers to the process of converting information or data into a usually random sequence of data, in order to prevent unauthorized parties from accessing the information.
    \item Decryption refers to the process of undoing encryption.
    \item Key generation is the process of creating keys that are used to encrypt and decrypt whatever data is being sent.
\end{itemize}

\begin{example}
    The easiest example to illustrate a cryptosystem is the Caesar Cipher cryptosystem. The message $M = \code{HELLO WORLD}$ can be `encrypted' into $E = \code{JGNNQ YQTNF}$ by using an `encryption key' of $k_e = 2$, i.e. we shift each letter in $M$ by two places. To `decrypt', we reverse the process by shifting each letter in $E$ to letters behind to reveal $M$ again. The `decryption key' in this case is $k_d = 2$, since we are shifting the letters backwards 2 spaces. We note that $0 \leq k_e \leq 25$ (with $k_e = 0$ meaning that no encryption takes place), and $k_d = k_e$.
\end{example}

The Caesar Cipher cryptosystem is what is known as a \textbf{private key cryptosystem}\index{cryptosystem!private key}, meaning that the key that was generated for encryption is also used for that of decryption. As seen above, once one knows the encryption key, the decryption key can be retrieved very easily. Therefore, for a secure transmission, it is essential that both encryption and decryption keys are kept secret. This is why this is called a private key cryptosystem.

In contrast, most encryption algorithms that form the backbone of communication on the internet are \textbf{public key cryptosystems}\index{cryptosystem!public key}. In such a system, both an encryption key and decryption key are generated. The encryption key is known to everyone who wants to communicate, while the decryption key is kept secret. Hence, the encryption key is called the \textbf{public key}\index{public key} and the decryption key is called the \textbf{private key}\index{private key}. It is often the case that the public key is \textit{different} to that of the private key, which is why most public key systems are also called \textbf{asymmetric cryptography}\index{asymmetric cryptography}. NTRU is an example of such a cryptosystem.

\section{Truncated Polynomial Rings}
NTRU (as described in \cite[\S 1.1]{hoffstein_pipher_silverman_1998}) uses the truncated polynomial ring
\[
    R = \Z[x] / \princ{x^N - 1}
\]
where $N$ is a positive integer. An element in $R$ takes the form
\[
    \sum_{i=0}^{N-1}a_ix^i + \princ{x^N - 1} = a_0 + a_1x + a_2x^2 + \cdots + a_{N-1}x^{N-1} + \princ{x^N - 1}.
\]
However, since all polynomials will have the trailing $\princ{x^N - 1}$ at the end, we instead call an element in $R$ a `polynomial' (although it is not strictly a polynomial) and write an element $f(x) \in R$ by
\[
    f(x) = \sum_{i=0}^{N-1}f_ix^i
\]
where we follow the notation of \cite{hoffstein_pipher_silverman_1996,hoffstein_pipher_silverman_1998} and let the coefficients be written like above.

\begin{example}
    Let us work through an example of `polynomial addition' within $R$ for $N = 5$. Let $f(x) = x^3 + x + 2$ and $g(x) = 2x^4 + 5x^2 + 8x$ be `polynomials' in $R$. Then
    \begin{align*}
        f(x) + g(x) &= (x^3 + x + 2) + (2x^4 + 5x^2 + 8x)\\
        &= 2x^4 + x^3 + 5x^2 + 9x + 2
    \end{align*}
    which is an element in $R$.
    
    The trouble comes with `polynomial multiplication', where we see
    \begin{align*}
        f(x)g(x) &= (x^3 + x + 2)(2x^4 + 5x^2 + 8x)\\
        &= (2x^7 + 5x^5 + 8x^4) + (2x^5 + 5x^3 + 8x^2) + (4x^4 + 10x^2 + 16x)\\
        &= 2x^7 + 7x^5 + 12x^4 + 5x^3 + 18x^2 + 16x
    \end{align*}
    which, if we are using normal polynomial multiplication, is \textit{not} an element of $R$.

    However, recall that every polynomial in $R$ has a `hidden' $\princ{x^N - 1}$ at their end. In this case, the hidden ideal is $I = \princ{x^5-1}$; so in fact we can reduce their product $f(x)g(x)$,
    \begin{align*}
        f(x)g(x) &= 2x^7 + 7x^5 + 12x^4 + 5x^3 + 18x^2 + 16x + I\\
        &= (2x^2(x^5-1) + 7(x^5-1) + 2x^2 + 7) + (12x^4 + 5x^3 + 18x^2 + 16x) + I\\
        &= (2x^2+7)(x^5-1) + (12x^4 + 5x^3 + (18 + 2)x^2 + 16x + 7) + I\\
        &= (12x^4 + 5x^3 + 20x^2 + 16x + 7) + I,
    \end{align*}
    which means that in $R$ our original product is `equal' to $12x^4 + 5x^3 + 20x^2 + 16x + 7$.
\end{example}

Such polynomials in $R$ shall henceforth be called \textbf{truncated polynomials}\index{truncated polynomials}.

In summary, the rules for `polynomial addition' and `polynomial multiplication' are as follows.
\begin{itemize}
    \item `Polynomial addition' is the same as regular polynomial addition.
    \item `Polynomial multiplication' is almost the same as regular polynomial multiplication, except that for each integer $m$ greater than $N - 1$, we replace $x^m$ with $x^{m \mod N}$.
\end{itemize}

For NTRU to work, the polynomials that we choose must have inverses. In particular, NTRU will involve reducing the \textit{coefficients} modulo certain integers. Suppose $p$ is a positive integer that we want to reduce the coefficients modulo of (note that $p$ may not be prime; this is just the notation used in \cite[\S~1.1]{hoffstein_pipher_silverman_1998}). Then an inverse of $f(x) \in R$ is an $F_p(x) \in R$ such that
\[
    F_p(x)f(x) \equiv 1 \pmod p
\]
Again, we follow the notation used in \cite{hoffstein_pipher_silverman_1998}. Note that $f(x)$ may not have an inverse; this problem is not significant when we are actually performing the encryption, since we may just choose another $f(x)$.

\begin{example}
    We claim that $F_3(x) = 2x^4 + 2x^3 + x^2 + x + 2$ is an inverse of $f(x) = x^2 + 1$ in $R$ when $N = 5$ and $p = 3$. Note
    \begin{align*}
        F_3(x)f(x) &= (2x^4 + 2x^3 + x^2 + x + 2)(x^2+1)\\
        &= 2x^6 + 2x^5 + 3x^4 + 3x^3 + 3x^2 + x + 2\\
        &= 2x + 2 + 3x^4 + 3x^3 + 3x^2 + x + 2 & (\text{reduce powers modulo }N = 5)\\
        &= 3x^4 + 3x^3 + 3x^2 + 3x + 4\\
        &= 1 & (\text{reduce coefficients modulo } p = 3)
    \end{align*}
    so $F_3(x)$ is an inverse of $f(x)$ when $N = 5$ and $p = 3$.
\end{example}

\begin{exercise}
    Find an inverse of $f(x) = x + 2$ when $N = 2$ and $p = 4$.
\end{exercise}

\section{NTRU}
NTRU fits in the general framework of probabilistic cryptosystems, meaning that any message that one wishes to encrypt may have many different possible encrypted values, since a hint of randomness is included. When compared to the more famous RSA encryption scheme, for a message of length $N$, it takes only the order of $N^2$ operations, compared to RSA which takes the order of $N^3$ operations \cite[p.~268]{hoffstein_pipher_silverman_1998}.

\subsection{Parameters}
An NTRU cryptosystem uses 3 integer parameters.
\begin{itemize}
    \item $N$, the number of terms present in a `polynomial' in the ring $R$ defined above. Again, we remark that the maximum degree that a `polynomial' in $R$ is $N - 1$.
    \item $p$, a so-called small modulus, which is usually an odd number. This may not be prime.
    \item $q$, a so-called large modulus. This is usually a power of 2, but has to be considerably larger than and coprime to $p$.
\end{itemize}

As defined in the previous section, NTRU works with `polynomials' within the ring
\[
    R = \Z[x] / \princ{x^N - 1}.
\]
NTRU also uses 4 sub\textit{sets} of $R$.
\begin{itemize}
    \item $\mathcal{L}_f$, the set of possible private keys \cite[\S 1.2]{hoffstein_pipher_silverman_1998}.
    \item $\mathcal{L}_g$, the set of `adjustment' polynomials. It should be noted that NTRU does not explicitly name what $\mathcal{L}_g$ is; we call it `adjustment' polynomials here since they adjust the private keys slightly, as we will see later.
    \item $\mathcal{L}_\phi$, the set of `encoding fuzz' polynomials \cite[\S 1.3]{hoffstein_pipher_silverman_1996}.
    \item $\mathcal{L}_m$, the set of possible message polynomials \cite[\S 1.3]{hoffstein_pipher_silverman_1998}.
\end{itemize}

For ease of computation of the multiplication of polynomials later, the coefficients of these polynomials would be small. We follow \cite[\S~2.2]{hoffstein_pipher_silverman_1998} on the description on the polynomials in the above sets.
\begin{itemize}
    \item For $\mathcal{L}_m$, let the coefficients of the polynomials lie within the interval $\left[-\frac{p-1}2, \frac{p-1}2\right]$, assuming $p$ is odd.
    \item For the rest, first define $\mathcal{L}(d_1, d_2)$ to be the set of polynomials in $R$ with $d_1$ coefficients equalling 1, $d_2$ coefficients equalling -1, and the rest equal to 0. Then, by choosing positive integers $d_f$, $d_g$, and $d_\phi$, set
    \begin{align*}
        \mathcal{L}_f &= \mathcal{L}(d_f, d_f-1),\\
        \mathcal{L}_g &= \mathcal{L}(d_g, d_g), \text{ and}\\
        \mathcal{L}_\phi &= \mathcal{L}(d, d).
    \end{align*}
\end{itemize}

A notable property of polynomials in $\mathcal{L}_g$ and $\mathcal{L}_\phi$ is that they all do not have inverses.
\begin{exercise}
    Show that a polynomial in $\mathcal{L}(k,k)$, where $k$ is some positive integer smaller than $\frac N2$, does not have an inverse modulo $p$ for any $p$.
\end{exercise}
Thus, to ensure that $f(x)$ has a chance to be invertible, we choose $\mathcal{L}_f$ to use $ \mathcal{L}(d_f, d_f-1)$, so that it has one more coefficient of 1 than than it does for -1.

\subsection{Key Creation}
Suppose Bob wants to send Alice a message. Before that, Alice needs to create her encryption and decryption key within the NTRU cryptosystem.

Alice chooses two polynomials $f(x) \in \mathcal{L}_f$ and $g(x) \in \mathcal{L}_g$, which will have `small' coefficients (which, in this case, meaning that the coefficients are either 0 or $\pm1$). Now $g(x)$ can be anything, while $f(x)$ has to satisfy the additional requirement that it has inverses modulo $p$ and $q$. That is, there must exist polynomials $F_p(x), F_q(x) \in R$ such that
\[
    F_p(x)f(x) \equiv 1 \pmod p \text{ and } F_q(x)f(x) \equiv 1 \pmod q.
\]
For most choices of $f(x)$ this will be true; if not, just randomly pick another $f(x)$ from $\mathcal{L}_f$. Alice then computes the polynomial $h(x)$ given by
\[
    h(x) = F_q(x)g(x) \mod q,
\]
that is, $h(x)$ is $F_q(x)g(x)$ and then reducing the coefficients modulo $q$. Alice's public key is $h(x)$; her private key is $f(x)$, although in practice she will also store $F_p(x)$.

\begin{example}
    Suppose $N = 5$, $p = 3$, $q = 8$. Suppose Alice chooses $f(x) = x^4 + x^3 - x$ and $g(x) = x^3 - x^2 + x - 1$. We note that
    \[
        F_3(x) = 2x^4 + x^3 + 1
    \]
    and
    \[
        F_8(x) = 7x^4 + x^3 + 1
    \]
    So we see that $h(x)$ is
    \begin{align*}
        h(x) &= F_8(x)g(x)\\
        &= (7x^4 + x^3 + 1)(x^3 - x^2 + x - 1)\\
        &= 7x^7 - 6x^6 + 6x^5 - 6x^4 - x^2 + x - 1\\
        &= 7x^2 - 6x + 6 - 6x^4 - x^2 + x - 1\\
        &= -6x^4 + 6x^2 - 5x + 5\\
        &\equiv 2x^4 + 6x^2 + 3x + 5 \pmod8
    \end{align*}
    which is Alice's public key. Alice would then keep $f(x) = x^4 + x^3 - x$ and $F_3(x) = 2x^4 + x^3 + 1$ secret as her private keys.
\end{example}
\begin{example}
    Suppose $N = 7$, $p = 3$, $q = 32$. Suppose Alice chooses $f(x) = x^6 - x^3 + 1$ and $g(x) = x^5 + x^4 - x^2 - 1$, and note
    \begin{align*}
        F_3(x) &= 2x^6 + x^4 + 2x^2 + x + 1, \text{ and}\\
        F_{32}(x) &= 31x^6 + x^4 + 31x^2 + x + 1.
    \end{align*}
    So Alice's public key is
    \begin{align*}
        h(x) &= F_{32}(x)g(x)\\
        &= (31x^6 + x^4 + 31x^2 + x + 1)(x^5 + x^4 - x^2 - 1)\\
        &= 31x^{11} + 31x^{10} + x^9 - 30x^8 + 31x^7 + 2x^5 - 31x^4 - x^3 - 32x^2 - x - 1\\
        &= 31x^4 + 31x^3 + x^2 - 30x + 31 + 2x^4 - 31x^4 - x^3 - 32x^2 - x - 1\\
        &= 2x^4 + 30x^3 - 31x^2 - 31x + 30\\
        &\equiv 2x^4 + 30x^3 + x^2 + x + 30 \pmod{32}.
    \end{align*}
    Alice would then keep $f(x) = x^6 - x^3 + 1$ and $F_3(x) = 2x^6 + x^4 + 2x^2 + x + 1$ secret as her private keys.
\end{example}

\begin{exercise}
    Let $N = 6$, $p = 5$, and $q = 16$, and suppose Alice chooses $f(x) = x^3 + x^2 - 1$ and $g(x) = x^4 - x^3 - x + 1$. Given
    \begin{align*}
        F_5(x) &= 3x^5 + 4x^4 + x^3 + 2x^2 + 2x + 4,\\
        F_{16}(x) &= 7x^5 + 5x^4 + 14x^3 + 9x^2 + 12x + 2,
    \end{align*}
    what is Alice's public key?
\end{exercise}

\subsection{Encryption}
%TODO: Add


\subsection{Decryption}
%TODO: Add

\section{Security Analysis}
%TODO: Add
