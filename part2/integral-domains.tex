\chapter{Integral Domains}
We explored the absolute basics of rings in the previous chapter. When working in abstract algebra, we usually prefer rings with slightly more structure as they allow us to generate more useful and applicable results. Integral domains are an important category of rings, which we explore in this chapter.

\section{What is an Integral Domain?}
\begin{definition}
    An \textbf{integral domain}\index{integral domain} is a commutative ring with identity and contains no zero divisors.
\end{definition}
\begin{remark}
    Recall that an element $a \neq 0$ is a zero divisor if there exists a $b \neq 0$ such that $ab = 0$. So, equivalently, if $a \neq 0$ and $b \neq 0$ then $ab \neq 0$ for any $a$ and $b$ in the integral domain.
\end{remark}
\begin{remark}
    What this property grants us is a ``cancellation law'', similar to that for groups. We will prove this in \myref{prop-integral-domain-cancellation-law}. However, there may not be inverses when performing multiplication. This is similar to how the integers do not have multiplicative inverses in general, hence the name of ``integral domain''.
\end{remark}

\begin{example}
    We show that the integers, $\Z$, the namesake for integral domains, is indeed an integral domain. We note that $\Z$ is a commutative ring as multiplication is commutative. Furthermore, 1 is the multiplicative identity in $\Z$. All that remains is to show that there does not exist any zero divisors in $\Z$.

    Suppose $a$ and $b$ are non-zero elements in $\Z$. We show that $ab \neq 0$ to prove that there does not exist any zero divisors. Let's first consider the case when $b > 0$. We may write
    \[
        ab = \underbrace{a + a + \cdots + a}_{b \text{ times}}
    \]
    which clearly is not zero since $a \neq 0$. Now if $b < 0$ then note
    \[
        ab = (-a)(-b) = \underbrace{(-a) + (-a) + \cdots + (-a)}_{(-b) \text{times}}
    \]
    which is also not zero as $-a \neq 0$. Hence there are no zero divisors in $\Z$.

    Therefore $\Z$ is a commutative ring with identity without any zero divisors, meaning that $\Z$ is an integral domain.
\end{example}

\begin{example}
    We show that the ring $\Z[i]$, the gaussian integers, is an integral domain. Similar to $\Z$, we note multiplication is commutative with identity 1.

    Suppose $w$ and $z$ are non-zero elements in $\Z[i]$; write $w = a+bi$ and $z = c+di$. Note that $a^2+b^2 \neq 0$ and $c^2 + d^2 \neq 0$. By definition of complex multiplication we have
    \[
        wz = (a+bi)(c+di) = (ac-bd) + (ad+bc)i.
    \]
    We just need to show that $ac-bd \neq 0$ and $ad+bc \neq 0$. Note that for any real numbers $x$ and $y$, we have $x^2 + y^2 \neq 0$ if and only if both $x$ and $y$ are non-zero. So we see
    \begin{align*}
        (ac-bd)^2 + (ad+bc)^2 &= (a^2c^2 - 2abcd + b^2d^2) + (a^2d^2 + 2abcd + b^2c^2)\\
        &= a^2c^2 + a^2d^2 + b^2c^2 + b^2d^2\\
        &= (a^2 + b^2)(c^2 + d^2)\\
        &\neq 0
    \end{align*}
    which hence means that $ac-bd \neq 0$ and $ad+bc \neq 0$. Therefore $wz \neq 0$, meaning that $\Z[i]$ has no zero divisors. Thus $\Z[i]$ is indeed an integral domain.
\end{example}

We note an interesting result regarding fields here.
\begin{proposition}\label{prop-field-is-integral-domain}
    Any field is an integral domain.
\end{proposition}
\begin{proof}
    Any non-zero element in a field is a unit, i.e. has an inverse. Now by \myref{prop-zero-divisors-have-no-inverses}, any zero divisor must not have an inverse; the converse being that any element that has an inverse cannot be a zero divisor. Hence, there are no zero divisors in a field. Now a field is commutative, therefore meaning that the field is an integral domain.
\end{proof}

Let's look at some non-examples of integral domains.
\begin{example}
    The ring $\Mn{n}{R}$ where $R$ is any ring is not an integral domain since $\Mn{n}{R}$ is not commutative.
\end{example}
\begin{example}
    The ring $2\Z$ is not an integral domain as it does not contain a multiplicative identity (namely, 1).
\end{example}
\begin{example}
    The ring $\Z \times \Z$ under pairwise addition and multiplication is not an integral domain as $(0,1) \neq (0, 0)$ and $(1, 0) \neq (0, 0)$ but $(0,1)(1,0) = (0, 0)$, meaning that $\Z \times \Z$ contains at least one zero divisor.
\end{example}

\begin{exercise}
    Prove that the ring
    \[
        \Z[\sqrt2] = \left\{a + b\sqrt2 \vert a,b \in \Z\right\}
    \]
    is an integral domain.
\end{exercise}
\begin{exercise}
    Explain why all $\Z_n$ where $n$ is a positive composite number are not integral domains.
\end{exercise}

\section{General Results}
With an introduction of integral domains out of the way, we introduce some general results applicable to integral domains.
\begin{proposition}[Cancellation Law for Integral Domains]\index{integral domain!cancellation law}\label{prop-integral-domain-cancellation-law}
    Let $R$ be an integral domain, $r, x, y \in R$, and $r \neq 0$. Then the following statements are equivalent.
    \begin{enumerate}[label=(\arabic*)]
        \item $x = y$
        \item $rx = ry$
        \item $xr = yr$
    \end{enumerate}
\end{proposition}
\begin{proof}
    We prove the statements in order.
    \begin{itemize}
        \item $\boxed{(1) \implies (2)}$ Given $x = y$, this means $x - y = 0$. Multiplying $r$ on the left on both sides yields $r(x-y) = r0 = 0$. Distributing yields $rx - ry = 0$ which hence means $rx = ry$.
        
        \item $\boxed{(2) \implies (3)}$ Given $rx = ry$. Multiplying $r$ on the right on both sides yields $rxr = ryr$, thereby meaning $rxr - ryr = 0$. Factoring yields $r(xr - yr) = 0$. As $R$ is an integral domain, the only way for this to occur is if $r = 0$ (which is impossible) or $xr - yr = 0$. Therefore $xr = yr$.
        
        \item $\boxed{(3) \implies (1)}$ Given $xr = yr$ we may write $xr - yr = (x-y)r = 0$. Since $R$ is an integral domain we must have $x - y = 0$ or $r = 0$ (impossible as $r \neq 0$). Hence $x = y$.
    \end{itemize}
    This proves the proposition.
\end{proof}

\begin{theorem}\label{thrm-finite-integral-domain-is-field}
    Every finite integral domain is a field.
\end{theorem}
\begin{proof}
    A finite integral domain is a commutative ring with identity; all that remains is to prove that every non-zero element has an inverse.

    Suppose $R$ is a finite integral domain and take a non-zero element $r$ in $R$. Consider
    \begin{align*}
        A &= \{r^k \vert k \in \Z,\;k\geq 0\}\\
        &= \{1, r, r^2, r^3, \dots\}\\
        &\subseteq R.
    \end{align*}
    Note that since $R$ is finite, $A$ must be finite as well. Therefore there must exist positive integers $m$ and $n$ with $m > n$ such that $r^m = r^n$, meaning $r^m - r^n = 0$. As $m > n$, thus $r^n\left(r^{m-n}-1\right) = 0$. Hence, as there are no zero divisors in $R$, either $r^n = 0$ or $r^{m-n} - 1 = 0$.

    Now if $r^n = 0$ then $n > 1$ (since $r^1 = r \neq 0$ by assumption). Note $r^n = rr^{n-1} = 0$, which hence means $r^{n-1} = 0$ as $r \neq 0$ and $R$ has no zero divisors. But we may repeat this argument on $n - 1$ to eventually terminate at $r = 0$, which is a contradiction. Hence we must conclude that $r^{m-n} - 1 = 0$. We may write $r^{m-n}-1 = 0$ as $rr^{m-n-1} = 1$ which means that $r$ is a unit (as $r^{m-n-1}$ is its inverse). Therefore every non-zero element has an inverse.

    Hence $R$ is a field.
\end{proof}

\begin{remark}
    We note that finite fields are also known as \textbf{Galois fields}\index{Galois Field}, so-named in honour of \'Evariste Galois. We explore more properties of fields and Galois fields in part III.
\end{remark}

\begin{example}
    Let $p$ be a prime and consider the ring $\Z_p$. Clearly multiplication is commutative with identity 1. Also, as the only way to factor $p$ is $1 \times p$ and since $p \notin \Z_p$, thus there are no zero divisors in $\Z_p$. Hence $\Z_p$ is an integral domain. Now $\Z_p$ only has $p$ elements (namely the integers 0 to $p - 1$), so it is finite. By \myref{thrm-finite-integral-domain-is-field} we know that $\Z_p$ is a field.
\end{example}

\begin{example}
    We show that $\Z_3[i] = \{a+bi \vert a,b \in \Z_3\}$ is a finite integral domain in order to show that it is a field. Clearly multiplication is commutative with identity 1, so we just need to show that there are no zero divisors in $\Z_3[i]$.
    \begin{table}[h]
        \centering
        \resizebox{\textwidth}{!}{
            \begin{tabular}{|l|l|l|l|l|l|l|l|l|}
            \hline
            $\boldsymbol{\times}$ & $\boldsymbol{1}$ & $\boldsymbol{2}$ & $\boldsymbol{i}$ & $\boldsymbol{1+i}$ & $\boldsymbol{2+i}$ & $\boldsymbol{2i}$ & $\boldsymbol{1+2i}$ & $\boldsymbol{2+2i}$ \\ \hline
            $\boldsymbol{1}$    & 1      & 2      & $i$    & $1+i$  & $2+i$  & $2i$   & $1+2i$ & $2+2i$ \\ \hline
            $\boldsymbol{2}$    & 2      & 1      & $2i$   & $2+2i$ & $1+2i$ & $i$    & $2+i$  & $1+i$  \\ \hline
            $\boldsymbol{i}$    & $i$    & $2i$   & 2      & $2+i$  & $2+2i$ & 1      & $1+i$  & $1+2i$ \\ \hline
            $\boldsymbol{1+i}$  & $1+i$  & $2+2i$ & $2+i$  & $2i$   & 1      & $1+2i$ & 2      & $i$    \\ \hline
            $\boldsymbol{2+i}$  & $2+i$  & $1+2i$ & $2+2i$ & 1      & $i$    & $1+i$  & $2i$   & 2      \\ \hline
            $\boldsymbol{2i}$   & $2i$   & $i$    & 1      & $1+2i$ & $1+i$  & 2      & $2+2i$ & $2+i$  \\ \hline
            $\boldsymbol{1+2i}$ & $1+2i$ & $2+i$  & $1+i$  & 2      & $2i$   & $2+2i$ & $i$    & 1      \\ \hline
            $\boldsymbol{2+2i}$ & $2+2i$ & $1+i$  & $1+2i$ & $i$    & 2      & $2+i$  & 1      & $2i$   \\ \hline
            \end{tabular}
        }
    \end{table}
    
    From the table, we have shown that no two non-zero elements in $\Z_3[i]$ multiply to zero. Therefore $\Z_3[i]$ has no zero divisors, which means $\Z_3[i]$ is an integral domain. Furthermore, as $\Z_3[i]$ is finite, therefore we know that it is a finite field by \myref{thrm-finite-integral-domain-is-field}.
\end{example}

\begin{exercise}\label{exercise-Zn2[alpha]}
    Is the ring
    \[
        \Z_2[\alpha] = \{a+b\alpha \vert a,b\in\Z_2\}
    \]
    where $\alpha^2 = 1 + \alpha$ under regular addition and multiplication a field?
\end{exercise}

\section{Characteristic of a Ring}
Before we introduce the characteristic of a ring, we clarify the meaning of the order of an element in the ring.
\begin{definition}
    Let $R$ be a ring and $x$ an element of the ring $R$.
    \begin{itemize}
        \item The \textbf{additive order}\index{order!additive} of $x$ is the order of $x$ in the group $(R, +)$. We denote the additive order of $x$ by $|x|_+$.
        \item If $R$ is a group with identity 1, the \textbf{multiplicative order}\index{order!multiplicative} of $x$ is the smallest positive integer $n$ such that $x^n = 1$ and is denoted $|x|_\times$, if it exists.
    \end{itemize}
\end{definition}

\begin{definition}
    Let $R$ be a ring. We say that \textbf{characteristic}\index{characteristic} of $R$ is $n$ if $n$ is the smallest positive integer such that
    \[
        nr = \underbrace{r+r+\cdots+r}_{n \text{ times}} = 0
    \]
    for all $r \in R$. We then write $\Char{R} = n$. If no such $n$ exists we say $\Char{R} = 0$.
\end{definition}

We look at two properties about the characteristic.
\begin{proposition}
    If $R$ is a ring with identity and $\Char{R} \neq 0$, then $\Char{R}$ is equal to $|1|_+$.
\end{proposition}
\begin{proof}
    Let $\Char{R} = n$ and $|1|_+ = m$. Our goal is to show that $m = n$.

    On one hand, note that
    \[
        n1 = \underbrace{1+1+1+\cdots+1}_{n \text{ times}} = 0
    \]
    by definition of the characteristic of a ring. Note also that $m1 = \overbrace{1+1+1+\cdots+1}^{m \text{ times}} = 0$ by definition of the order of an element in the group $(R, +)$. By \myref{problem-element-to-power-of-multiple-of-order-is-identity}, this means that $n$ is a multiple of $m$, which thus means $m \leq n$.

    On another hand, for any $r \in R$, we know that
    \begin{align*}
        \underbrace{r + r + r + \cdots + r}_{m \text{ times}} &= r(\underbrace{1+1+1+\cdots+1}_{m \text{ times}})\\
        &= r0 & (\text{since } |1|_+ = m)\\
        &= 0.
    \end{align*}
    The minimality of the characteristic $n$ means that $m$ has to be at least $n$, i.e. $m \geq n$.

    Since $m \leq n$ and $m \geq n$, thus $m = n$.
\end{proof}

\begin{proposition}\label{prop-zero-of-prime-characteristic-if-integral-domain}
    If $R$ is an integral domain, then either $\Char{R} = 0$ or $\Char{R} = p$ where $p$ is a prime.
\end{proposition}
\begin{proof}
    If $\Char{R} = 0$ we are done, so assume that $\Char{R} = n \neq 0$. Furthermore assume $n = ab$ with $1 \leq a,b \leq n$.

    Note that
    \[
        0 = \underbrace{1 + 1 + \cdots + 1}_{n \text{ times}} = (\underbrace{1+1+\cdots+1}_{a \text{ times}})(\underbrace{1+1+\cdots+1}_{b \text{ times}}).
    \]
    As $R$ is an integral domain, this means that either
    \[
        \underbrace{1+1+\cdots+1}_{a \text{ times}} = 0 \text{ or } \underbrace{1+1+\cdots+1}_{b \text{ times}} = 0.
    \]
    Without loss of generality, assume that $\underbrace{1+1+\cdots+1}_{a \text{ times}} = 0$. By minimality of $\Char{R} = n$, this means that $a \geq n$. But we assumed that $a \leq n$, so we must have $a = n$. Therefore $a = n$ and $b = 1$ which is exactly what we need to show that $n$ is a prime.
\end{proof}

\begin{exercise}\label{exercise-trivial-ring-is-not-an-integral-domain}
    Is the trivial ring an integral domain?
\end{exercise}
\begin{exercise}
    What is the characteristic of the ring $\Z_2[\alpha]$ (as defined in \myref{exercise-Zn2[alpha]})?
\end{exercise} 

\newpage

\section{Problems}
\begin{problem}
    Find two zero divisors in the ring
    \[
        \Z_5[i] = \{a + bi \vert a,b\in\Z_5\}.
    \]
\end{problem}

\begin{problem}
    Show that the ring of gaussian integers, $\Z[i]$, is an integral domain.
\end{problem}

\begin{problem}
    Let the integer $n$ be such that $\sqrt{|n|}$ is not an integer. Let the ring
    \[
        R = \Z[\sqrt{n}] = \{a + b\sqrt{n} \vert a,b\in \Z\}.
    \]
    \begin{partquestions}{\alph*}
        \item Show that $R$ is an integral domain.
        \item Is $R$ a field for all integers $n$?
    \end{partquestions}
\end{problem}

\begin{problem}
    Show that
    \[
        R = \left\{\begin{pmatrix}0&0\\0&0\end{pmatrix},\begin{pmatrix}1&0\\0&1\end{pmatrix},\begin{pmatrix}1&1\\1&0\end{pmatrix},\begin{pmatrix}0&1\\1&1\end{pmatrix}\right\}
    \]
    with entries in $\Z_2$ is
    \begin{partquestions}{\roman*}
        \item a subring of $\Mn{2}{\Z_2}$; and
        \item a field.
    \end{partquestions}
\end{problem}
