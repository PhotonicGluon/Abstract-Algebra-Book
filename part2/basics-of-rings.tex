\chapter{Basics of Rings}
With an intuition and definition of rings out of the way, we are now ready to tackle the basics in this chapter.

\section{Obvious Rings}
Before we introduce some examples of rings, we make some remarks for the notation that is used in Ring Theory.
\begin{itemize}
    \item The multiplication symbol $\cdot$ is usually omitted, so $x \cdot y$ is written as $xy$.
    \item The additive identity of $R$ will always be denoted by 0 and the multiplicative identity of $R$ (if it exists) will always be denoted by 1.
    \item The additive inverse of the element $x$ will be denoted by $-x$ and the multiplicative inverse of $x$ (if it exists) will be denoted by $x^{-1}$.
    \item $n$ applications of $+$ on an element $x$ will be denoted $nx$ (and will be denoted $-nx$ if the element is $-x$), while $n$ applications of $\cdot$ on an element $x$ will be denoted $x^n$ (and will be denoted $x^{-n}$ if the element is $x^{-1}$ and if it exists).
\end{itemize}

Let's look at some examples of rings.
\begin{definition}
    The \textbf{ring of integers}\index{ring!of integers} is the set $\Z$ together with integer addition and multiplication.
\end{definition}
\begin{remark}
    We denote the ring of integers by $\Z$.
\end{remark}
\begin{proposition}
    $\Z$ is a commutative ring with identity.
\end{proposition}
\begin{proof}
    \myref{exercise-ring-of-integers-is-a-ring} (later) shows that $\Z$ is a ring. In addition, multiplication is commutative (\myref{axiom-multiplication-is-commutative}), and 1 is the multiplicative identity. Thus $\Z$ is a commutative ring with identity.
\end{proof}

\begin{definition}
    Let the integer $n > 2$. The \textbf{ring of integers modulo $n$}\index{ring!of integers!modulo $n$} is $(\Z_n, \oplus_n, \otimes_n)$, where $\oplus_n$ and $\otimes_n$ denote addition and multiplication modulo $n$ respectively.
\end{definition}
\begin{remark}
    We denote the ring of integers modulo $n$ by $\Z_n$.
\end{remark}
\begin{proposition}
    $\Z_n$ is a commutative ring with identity.
\end{proposition}
\begin{proof}
    We first prove the ring axioms before showing that it is commutative with a multiplicative identity.
    \begin{itemize}
        \item \textbf{Addition-Abelian}: We know $(\Z_n, \oplus_n)$ is an abelian group by \myref{prop-Zn-is-abelian-group}
        \item \textbf{Multiplication-Semigroup}: We can see that $(\Z_n, \otimes_n)$ is a semigroup as
        \begin{itemize}
            \item $\Z_n$ is closed under $\otimes_n$ because $a \otimes_n b \in \{0, 1, 2, \dots, n-1\} = \Z_n$; and
            \item multiplication is associative (\myref{axiom-multiplication-is-associative}), so multiplication modulo $n$ is associative.
        \end{itemize}
        \item \textbf{Distributive}: Since multiplication distributes over addition (\myref{axiom-distributivity}), thus multiplication modulo $n$ (i.e. $\otimes_n$) distributes over addition modulo $n$ (i.e. $\oplus_n$).
    \end{itemize}
    Hence $(\Z_n, \oplus_n, \otimes_n)$ is a ring.
    
    Furthermore, multiplication is commutative (\myref{axiom-multiplication-is-commutative}), so $\otimes_n$ is commutative. Also $\otimes_n$ has an identity of 1. Therefore $(\Z_n, \oplus_n, \otimes_n)$ is a commutative ring with identity.
\end{proof}

\begin{definition}
    The \textbf{ring of rational numbers}\index{ring!of rational numbers} is $(\Q, +, \times)$, where $+$ and $\times$ denote normal addition and multiplication.
\end{definition}
\begin{remark}
    We denote the ring of rational numbers by $(\Q, +, \times)$.
\end{remark}
\begin{proposition}
    $\Q$ is a commutative ring with identity.
\end{proposition}
\begin{proof}
    We first show that $\Q$ satisfies the ring axioms.
    \begin{itemize}
        \item \textbf{Addition-Abelian}: We know that $(\Q, +)$ is an abelian group from \myref{problem-Q-is-abelian-group-under-addition}.
        \item \textbf{Multiplication-Semigroup}: We note that $(\Q, \times)$ is a semigroup as
        \begin{itemize}
            \item $\Q$ is closed under $\times$ because multiplying two rational numbers together produce a rational number; and
            \item multiplication is associative (\myref{axiom-multiplication-is-associative}).
        \end{itemize}
        \item \textbf{Distributive}: Multiplication distributes over addition by \myref{axiom-distributivity}.
    \end{itemize}
    Hence $\Q$ is a ring. Furthermore, $\times$ has an identity of 1 and is commutative (\myref{axiom-multiplication-is-commutative}). So $\Q$ is a commutative ring with identity.
\end{proof}

\begin{definition}
    The \textbf{ring of real numbers}\index{ring!of real numbers} is the ring $(\R, +, \times)$ where $+$ and $\times$ denotes regular addition and multiplication respectively.
\end{definition}
\begin{remark}
    We denote the ring of real numbers by $(\R, +, \times)$.
\end{remark}
\begin{proposition}
    $\R$ is a commutative ring with identity.
\end{proposition}
\begin{proof}
    Replace $(\Q, +)$ with $(\R, +)$ and $(\Q, \times)$ with $(\R, \times)$ in the previous proof.
\end{proof}

We end this section by looking at the ring of complex numbers.
\begin{definition}
    Let the set of \textbf{complex numbers}\index{complex numbers}
    \[
        \C = \{a + bi \vert a, b \in \R\}
    \]
    where $i = \sqrt{-1}$ is known as the \textbf{imaginary unit}\index{imaginary unit}, where $i^2 = -1$. Define complex addition and multiplication by
    \begin{align*}
        (a+bi) + (c+di) &= (a+c) + (b+d)i,\\
        (a+bi) \cdot (c+di) &= (ac-bd) + (ad+bc)i.
    \end{align*}
    Then $\C$ under complex addition and multiplication is the \textbf{ring of complex numbers}\index{ring!of complex numbers}.
\end{definition}
\begin{remark}
    We denote the ring of complex numbers by $\C$.
\end{remark}
\begin{proposition}
    $\C$ is a commutative ring with identity.
\end{proposition}
\begin{proof}
    We first show that $\C$ satisfies the ring axioms.
    \begin{itemize}
        \item \textbf{Addition-Abelian}: We show that $(\C, +)$ satisfies the group axioms, and then show that $(\C, +)$ is commutative.
        \begin{itemize}
            \item \textbf{Closure}: Clearly for all real numbers $a$, $b$, $c$, and $d$ we have $a + c \in \R$ and $b+d \in \R$. Thus $(a+bi) + (c+di) = (a+c) + (b+d)i \in \C$, meaning $\C$ is closed under complex addition.
            
            \item \textbf{Associativity}: Let $a+bi, c+di, e+fi \in \C$. Then note that
            \begin{align*}
                &(a+bi) + ((c+di) + (e+fi))\\
                &= (a+bi) + ((c+e) + (d+f)i)\\
                &= (a+(c+e)) + (b+(d+f))i\\
                &= ((a+c)+e) + ((b+d)+f)i & (+ \text{ is associative, }\myref{axiom-addition-is-associative})\\
                &= ((a+c) + (b+d)i) + (e+fi)\\
                &= ((a+bi) + (c+di)) + (e+fi)
            \end{align*}
            so complex addition is associative.
            
            \item \textbf{Identity}: The identity in $\C$ is $0 + 0i = 0$ since
            \[
                (0+0i) + (a+bi) = (0+a) + (0+b)i = a+bi
            \]
            and complex addition is commutative (to be proved later), so $(\C,+)$ has an additive identity.
            
            \item \textbf{Inverse}: Let $a+bi \in \C$. Clearly $-a, -b \in \R$ and that
            \[
                (a+bi) + (-a+(-b)i)= (a+(-a)) + (b+(-b))i = 0
            \]
            and complex addition is commutative (to be proved later), so any $a+bi\in C$ has an additive inverse of $-a-bi \in \C$.

            \item \textbf{Commutative}: Let $a+bi, c+di \in \C$. Then
            \begin{align*}
                (a+bi) + (c+di) &= (a+c) + (b+d)i\\
                &= (c+a) + (d+b)i & (+\text{ is commutative, } \myref{axiom-addition-is-commutative})\\
                &= (c+di) + (a+bi)
            \end{align*}
            so complex addition is commutative.
        \end{itemize}
        
        \item \textbf{Multiplication-Semigroup}: We show that $(\C, \times)$ is a semigroup.
        \begin{itemize}
            \item \textbf{Closure}: Clearly for all real numbers $a$, $b$, $c$, and $d$ we have $ac, bd, ad, bc \in \R$, so $ac - bd, ad + bc \in \R$. Therefore
            \[
                (a+bi)(c+di) = (ac-bd) + (ad+bc)i \in \C
            \]
            which means $\C$ is closed under multiplication.

            \item \textbf{Associativity}: Let $a+bi, c+di, e+fi \in \C$. Note that
            \begin{align*}
                &(a+bi)((c+di)(e+fi))\\
                &= (a+bi)((ce-df)+(cf+de)i)\\
                &= (a(ce-df) - b(cf+de)) + (a(cf+de) + b(ce-df))i\\
                &= (ace - adf - bcf - bde) + (acf + ade + bce - bdf)i\\
                &= (ace - bde - adf - bcf) + (acf - bdf + ade + bce)i\\
                &= ((ac-bd)e - (ad+bc)f) + ((ac-bd)f + (ad+bc)e)i\\
                &= ((ac-bd)+(ad+bc)i)(e+fi)\\
                &= ((a+bi)(c+di))(e+fi)
            \end{align*}
            so complex multiplication is associative.
        \end{itemize}
        
        \item \textbf{Distributive}: We only prove left distributivity because we will show that complex multiplication is commutative later. Let $a+bi, c+di, e+fi \in \C$. Note that
        \begin{align*}
            &(a+bi)((c+di) + (e+fi))\\
            &= (a+bi)((c+e) + (d+f)i)\\
            &= (a(c+e)-b(d+f)) + (a(d+f) + b(c+e))i\\
            &= (ac+ae-bd-bf) + (ad+af+bc+be)i\\
            &= (ac-bd+ae-bf) + (ad+bc+af+be)i & (+ \text{ is associative})\\
            &= ((ac-bd) + (ad+bc)i) + ((ae - bf) + (af + be)i)\\
            &= (a+bi)(c+di) + (a+bi)(e+fi)
        \end{align*}
        so complex multiplication distributes over complex addition.
    \end{itemize}
    Hence $\C$ is a ring.
    
    \newpage
     
    We now show that complex multiplication is commutative. Let $a+bi, c+di \in C$. Then we see
    \begin{align*}
        (a+bi)(c+di) &= (ac-bd) + (ad+bc)i\\
        &= (ca-db) + (da+cb)i & (\times\text{ is commutative, } \myref{axiom-multiplication-is-commutative})\\
        &= (c+di)(a+bi)
    \end{align*}
    so complex multiplication is commutative.

    Finally we show that complex multiplication has an identity. Consider $1 + 0i \in \C$. Note that
    \[
        (1+0i)(a+bi) = (1a-0b) + (1b+0a)i = a+bi,
    \]
    and since complex multiplication is commutative, therefore $1+0i$ is the multiplicative identity in $\C$.
    
    Therefore $\C$ is a commutative ring with identity.
\end{proof}

These are just some examples of rings; we explore more later in this chapter.
\begin{exercise}\label{exercise-ring-of-integers-is-a-ring}
    Prove that $\Z$ is a ring under regular addition and multiplication.\newline
    (\textit{You do \textbf{not} need to prove the \textbf{Distributive} axiom.})
\end{exercise}

\section{General Properties of Rings}
We list some properties of rings here. For each of the propositions, assume $R$ is a ring.

\begin{proposition}\label{prop-multiplying-by-zero-is-zero}
    $0x = x0 = 0$ for all $x \in R$.
\end{proposition}
\begin{proof}
    We note that
    \begin{align*}
        0x &= (0 + 0)x & (0 \text{ is additive inverse})\\
        &= 0x + 0x & (\text{by \textbf{Distributive} axiom})
    \end{align*}
    so by `subtracting' $0x$ on both sides (i.e., adding $-0x$ on both sides) we see $0 = 0x$.
    
    Also
    \begin{align*}
        x0 &= x(0 + 0) & (0 \text{ is additive inverse})\\
        &= x0 + x0 & (\text{by \textbf{Distributive} axiom})
    \end{align*}
    so by `subtracting' $x0$ on both sides we see $0 = x0$.
    
    Therefore $0x = x0 = 0$ for all $x \in R$.
\end{proof}

\begin{proposition}\label{prop-product-of-element-and-additive-inverse-is-additive-inverse-of-product}
    $(-a)b = a(-b) = -(ab)$ for any $a$ and $b$ in $R$.
\end{proposition}
\begin{proof}
    We show that $(-a)b = -(ab)$ and $a(-b) = -(ab)$ to complete the proof.
    \begin{itemize}
        \item Note $(-a)b + ab = (-a + a)b = 0b = 0$ by \textbf{Distributive} axiom. Hence by subtracting $ab$ on both sides we see $(-a)b = -(ab)$.
        \item Note also $a(-b) + ab = a(-b + b) = a0 = 0$ by \textbf{Distributive} axiom. Hence by subtracting $ab$ on both sides we see $a(-b) = -(ab)$.
    \end{itemize}
    Result follows.
\end{proof}

\begin{proposition}
    $(-a)(-b) = ab$ for any $a$ and $b$ in $R$.
\end{proposition}
\begin{proof}
    See \myref{exercise-product-of-additive-inverses} (later).
\end{proof}

\begin{proposition}
    If $R$ has an identity, it is unique.
\end{proposition}
\begin{proof}
    Suppose 1 and $1'$ are identities, and consider the sum $1 + 1'$. Then
    \begin{align*}
        1 + 1' &= 1\times(1+1') & (\text{multiplying by identity }1)\\
        &= 1\times1 + 1\times1' & (\text{by \textbf{Distributive} axiom})\\
        &= 1 + 1. & (1 \text{ and } 1' \text{ are identities})
    \end{align*}
    Subtracting 1 on both sides yields $1 = 1'$, meaning that the identity is unique.
\end{proof}

\begin{exercise}\label{exercise-product-of-additive-inverses}
    Show that $(-a)(-b) = ab$ for any $a$ and $b$ in $R$.
\end{exercise}

\section{Matrix Rings}
The rings that we explored in previous sections can be thought of as the `obvious' rings, since they are number systems. As rings were made to generalize number systems, they should clearly be rings. However, there are less obvious rings.

We looked at matrices in the context of the General/Special Linear Group of matrices. Here we see that matrices in fact form rings, known as matrix rings. Before that though, we need to define the operations within that ring.

\begin{definition}[Matrix Addition]\index{matrix addition}
    For any two matrices $\textbf{A}$ and $\textbf{B}$ with $n$ rows and columns and entries in the ring $(R, \oplus, \otimes)$, their sum is the matrix $\textbf{C} = \textbf{A} + \textbf{B}$ with $n$ rows and columns such that
    \[
        c_{i,j} = a_{i,j} \oplus b_{i,j}
    \]
    for all $i,j \in \{1, 2, \dots, n\}$.
\end{definition}
\begin{definition}[Matrix Multiplication]\index{matrix multiplication}
    For any two matrices $\textbf{A}$ and $\textbf{B}$ with $n$ rows and columns and entries in the ring $(R, \oplus, \otimes)$, their product is the matrix $\textbf{C} = \textbf{AB}$ with $n$ rows and columns such that, for all $i,j \in \{1, 2, \dots, n\}$, we have
    \begin{align*}
        c_{i,j} &= (a_{i,1}\otimes b_{1,j}) \oplus (a_{i,2}\otimes b_{2,j}) \oplus \cdots \oplus (a_{i,n}\otimes b_{n,j})\\
        &= \bigoplus_{k=1}^n (a_{i,k}\otimes b_{k,j}).
    \end{align*}
\end{definition}

We also define two matrices that will become useful when we work with matrix rings.
\begin{definition}
    The \textbf{zero matrix}\index{zero matrix} with $n$ rows and columns is
    \[
        \ZeroM{n} = 
        \begin{pmatrix}
            0 & 0 & 0 & \cdots & 0 \\
            0 & 0 & 0 & \cdots & 0 \\
            0 & 0 & 0 & \cdots & 0 \\
            \vdots & \vdots & \vdots & \ddots & \vdots \\
            0 & 0 & 0 & \cdots & 0 \\
        \end{pmatrix}
    \]
    where 0 is the additive identity (i.e. zero) in the ring $(R, \oplus, \otimes)$.
\end{definition}
\begin{definition}
    The \textbf{identity matrix}\index{identity matrix} with $n$ rows and columns is
    \[
        \IdentityM{n} = 
        \begin{pmatrix}
            1 & 0 & 0 & \cdots & 0 \\
            0 & 1 & 0 & \cdots & 0 \\
            0 & 0 & 1 & \cdots & 0 \\
            \vdots & \vdots & \vdots & \ddots & \vdots \\
            0 & 0 & 0 & \cdots & 1 \\
        \end{pmatrix}
    \]
    where 0 and 1 are the additive and multiplicative identities (i.e. zero and one) in the ring $(R, \oplus, \otimes)$ respectively. That is, the identity matrix is the matrix with 1s in the leading diagonal.
\end{definition}

We can now define what is a matrix ring.
\begin{definition}
    Let $(R, \oplus, \otimes)$ be a ring and $n$ be a positive integer. Then $\Mn{n}{R}$ under matrix addition and multiplication is a ring, known as the \textbf{matrix ring}\index{matrix ring} with elements in $(R, \oplus, \otimes)$.
\end{definition}
\begin{proposition}
    $\Mn{n}{R}$ is a ring with identity.
\end{proposition}
\begin{proof}
    We need to prove that the ring axioms hold.
    \begin{itemize}
        \item \textbf{Addition-Abelian}: We first prove that $(\Mn{n}{R}, +)$ is indeed an abelian group.
        \begin{itemize}
            \item \textbf{Closure}: Clearly the sum of any two matrices in $\Mn{n}{R}$ is also a square matrix with $n$ rows with elements inside $R$, meaning that $\Mn{n}{R}$ is closed under matrix addition.

            \item \textbf{Associativity}: Let the matrices $\textbf{A}$, $\textbf{B}$, and $\textbf{C}$ belong inside $\Mn{n}{R}$. Let $\textbf{P} = \textbf{A} + (\textbf{B} + \textbf{C})$ and $\textbf{Q} = (\textbf{A} + \textbf{B}) + \textbf{C}$. We note that $\textbf{P} = \textbf{Q}$ as
            \[
                p_{i,j} = a_{i,j} \oplus (b_{i,j} \oplus c_{i,j}) = (a_{i,j} \oplus b_{i,j}) \oplus c_{i,j} = q_{i,j}
            \]
            by associativity of $\oplus$, which proves that matrix addition is associative.
    
            \item \textbf{Identity}: We show that $\ZeroM{n}$ is the additive identity in $\Mn{n}{R}$. Let $\textbf{M} \in \Mn{n}{R}$; let $\textbf{N} = \textbf{M} + \ZeroM{n}$. Note that $n_{i,j} = m_{i,j} \oplus 0 = m_{i,j}$ so $\textbf{M} + \ZeroM{n} = \textbf{M}$. Therefore $\textbf{M} + \ZeroM{n} = \textbf{M}$ for any matrix in $\Mn{n}{R}$.
            
            \item \textbf{Inverse}: Let $\textbf{A} \in \Mn{n}{R}$. Define the matrix $\textbf{B} = -\textbf{A}$ such that $b_{i,j} = -a_{i,j}$. That is, $b_{i,j}$ contains the additive inverse of $a_{i,j}$ in the ring $R$. Then one sees that $\textbf{A} + \textbf{B} = \ZeroM{n}$. (We denote the additive inverse of a matrix $\textbf{M}$ by $-\textbf{M}$).

            \item \textbf{Commutative}: Let $\textbf{A}, \textbf{B} \in \Mn{n}{R}$. Set $\textbf{C} = \textbf{A} + \textbf{B}$ and $\textbf{D} = \textbf{B} + \textbf{C}$. Consider $c_{i,j} = a_{i,j} \oplus b_{i,j}$. Since $\oplus$ is commutative, thus $a_{i,j} \oplus b_{i,j} = b_{i,j} \oplus a_{i,j}$. But $d_{i,j} = b_{i,j} \oplus a_{i,j}$, so we have $c_{i,j} = d_{i,j}$. Therefore $\textbf{C} = \textbf{D}$.
        \end{itemize}

        \item \textbf{Multiplication-Semigroup}: We show that $(\Mn{n}{R}, \cdot)$ is a semigroup.
        \begin{itemize}
            \item \textbf{Closure}: In \myref{subsection-intro-to-matrices} we showed that matrix multiplication produces another $n \times n$ matrix. Furthermore the entries of the new matrix are elements of $R$. Hence $\Mn{n}{R}$ is closed under matrix multiplication.
        
            \item \textbf{Associativity}: We proved that matrix multiplication is associative in \myref{subsection-GLR-matrix-group}.
        \end{itemize}
        
        \item \textbf{Distributive}: We prove only $\textbf{A}(\textbf{B} + \textbf{C}) = (\textbf{AB}) + (\textbf{AC})$ as the other case is proven similarly. Let $\textbf{R} = \textbf{A}(\textbf{B} + \textbf{C})$, $\textbf{G} = \textbf{AB}$, and $\textbf{H} = \textbf{AC}$. We note
        \begin{align*}
            r_{i,j} &= \bigoplus_{k=1}^n \left(a_{i,k} \otimes \left(b_{k,j} \oplus c_{k,j}\right)\right)\\
            &= \bigoplus_{k=1}^n \left((a_{i,k} \otimes b_{k,j}) \oplus (a_{i,k} \otimes c_{k,j})\right)\\
            &= \left(\bigoplus_{k=1}^n (a_{i,k} \otimes b_{k,j})\right) \oplus \left(\bigoplus_{k=1}^n (a_{i,k} \otimes c_{k,j})\right)\\
            &= g_{i,j}\oplus h_{i,j}
        \end{align*}
        which means $\textbf{R} = \textbf{G} + \textbf{H}$.
    \end{itemize}
    As all the ring axioms are satisfied, thus $\Mn{n}{R}$ is a ring.

    We now show that $\Mn{n}{R}$ has a multiplicative identity, namely the identity matrix $\IdentityM{n}$. Let $\textbf{A} \in \Mn{n}{R}$ and let $\textbf{B} = \IdentityM{n}$. Note that $b_{i,j} = 1$ if and only if $i = j$.
    \begin{itemize}
        \item Let $\textbf{C} = \textbf{AB}$ and we see
        \begin{align*}
            &c_{i,j}\\
            &= \bigoplus_{k=1}^n(a_{i,k}\otimes b_{k,j})\\
            &= (a_{i,1}\otimes b_{1,j}) \oplus \cdots \oplus (a_{i,j-1}\otimes b_{j-1,j}) \oplus (a_{i,j}\otimes b_{j,j})\\
            &\quad\quad\oplus (a_{i,{j+1}}\otimes b_{j+1,j}) \oplus \cdots \oplus (a_{i,n}\otimes b_{n,j})\\
            &= (a_{i,1}\otimes 0) \oplus \cdots \oplus (a_{i,{j-1}}\otimes 0)\oplus (a_{i,j}\otimes 1) \oplus (a_{i,{j+1}}\otimes 0) \oplus \cdots \oplus (a_{i,n}\otimes 0)\\
            &= 0 \oplus \cdots \oplus 0 \oplus a_{i,j} \oplus 0 \oplus \cdots \oplus 0\\
            &= a_{i,j}
        \end{align*}
        so $\textbf{A}\IdentityM{n} = \textbf{A}$.

        \item Now let $\textbf{D} = \textbf{BA}$ and we also see
        \begin{align*}
            &d_{i,j}\\
            &= \bigoplus_{k=1}^n(b_{i,k}\otimes a_{k,j})\\
            &= (b_{i,1}\otimes a_{1,j}) \oplus \cdots \oplus (b_{i,i-1}\otimes b_{i-1,j}) \oplus (b_{i,i}\otimes a_{i,j})\\
            &\quad\quad\oplus (b_{i,{i+1}}\otimes a_{i+1,j}) \oplus \cdots \oplus (b_{i,n}\otimes a_{n,j})\\
            &= (0 \otimes a_{1,j}) \oplus \cdots \oplus (0\otimes a_{i-1,j})\oplus (1\otimes a_{i,j}) \oplus (0\otimes a_{i+1,j}) \oplus \cdots \oplus (0\otimes a_{n,j})\\
            &= 0 \oplus \cdots \oplus 0 \oplus a_{i,j} \oplus 0 \oplus \cdots \oplus 0\\
            &= a_{i,j}
        \end{align*}
        so $\IdentityM{n}\textbf{A} = \textbf{A}$.
    \end{itemize}
    Therefore the identity matrix $\IdentityM{n}$ is the multiplicative identity.

    Hence $\Mn{n}{R}$ is a ring with identity.
\end{proof}

\section{More Definitions}
Suppose $R$ is a ring.
\begin{definition}
    We say that $a \neq 0$ is a \textbf{zero divisor}\index{zero divisor} in $R$ if there exists $b \neq 0$ such that $ab = 0$.
\end{definition}
\begin{example}
    Consider the ring $\Z_{12}$. Clearly 4 and 6 are in $\Z_{12}$, and their product is $24 = 2 \times 12 = 0$ in $\Z_{12}$. Hence 4 and 6 are zero divisors in $\Z_{12}$.
\end{example}
\begin{example}
    Let $R$ be the ring of functions with domain and codomain $[0, 1]$. We claim that $R$ has zero divisors. Consider the functions
    \begin{align*}
        f:[0,1]\to[0,1], x &\mapsto x\\
        g:[0,1]\to[0,1], x &\mapsto \begin{cases}
            0 & \text{ if } x \neq 0\\
            1 & \text{ if } x = 0
        \end{cases}
    \end{align*}
    Clearly neither of them are the zero function. However, consider $f(x)g(x)$.
    \begin{itemize}
        \item If $x \neq 0$, then $g(x) = 0$ which means $f(x)g(x) = 0$.
        \item If $x = 0$, then $f(x) = 0$ which means $f(x)g(x) = 0$.
    \end{itemize}
    Hence their product is the zero function, meaning that $R$ has zero divisors $f$ and $g$.
\end{example}
\begin{exercise}
    Does the ring $\Mn{2}{\mathbb{R}}$ have zero divisors?
\end{exercise}

We note one property about zero divisors, which will be used in future chapters.

\begin{proposition}\label{prop-zero-divisors-have-no-inverses}
    Zero divisors do not have inverses.
\end{proposition}
\begin{proof}
    Assume $a \neq 0$ and $b \neq 0$ are zero divisors in the ring $R$, so $ab = 0$. Seeking a contradiction, assume $a$ has an inverse, so
    \[
        b = (a^{-1}a)b = a^{-1}(ab) = a^{-1}0 = 0    
    \]
    which contradicts $b \neq 0$. Hence a zero divisor has no inverse.
\end{proof}

\begin{definition}
    Suppose $R$ is a ring with identity such that $0 \neq 1$. An element $u \in R$ is called a \textbf{unit}\index{unit} if there exists a $v \in R$ such that $uv=vu=1$. Equivalently, $u$ is a unit if it has a multiplicative inverse.
\end{definition}
\begin{example}
    3 and 7 are units in $\Z_{10}$ since $3 \times 7 = 7 \times 3 = 21 = 1$ in $\Z_{10}$.
\end{example}

\begin{proposition}\label{prop-product-of-units-is-unit}
    The product of two units is a unit.
\end{proposition}
\begin{proof}
    See \myref{exercise-product-of-units-is-unit} (later).
\end{proof}

\begin{definition}
    Suppose $R$ is a ring with identity such that $0 \neq 1$. If every non-zero element $x \in R$ is a unit, then $R$ is said to be a \textbf{division ring}\index{division ring}.
\end{definition}

\begin{definition}
    A commutative division ring is called a \textbf{field}\index{field}.
\end{definition}

\begin{example}
    Earlier, we shown that $\R$ is a commutative ring. We now show that $\R$ is actually a field by noting that every non-zero $x \in \R$ has a reciprocal $\frac1x$ that is a real number (\myref{axiom-reciprocal}) such that $x\left(\frac1x\right) = \left(\frac1x\right)x = 1$. Thus every non-zero $x$ in $\R$ is a unit, meaning that $\R$ is a division ring. Coupled with the fact that $\R$ is a commutative ring means that $\R$ is a field.
\end{example}
\begin{example}
    We also shown earlier that $\C$ is a commutative ring. We note that any non-zero complex number $z = a + bi$ has a multiplicative inverse given by
    \[
        w = \frac{a}{a^2+b^2} - \frac{b}{a^2+b^2}i
    \]
    since
    \begin{align*}
        zw &= (a+bi)\left(\frac{a}{a^2+b^2} - \frac{b}{a^2+b^2}i\right)\\
        &= \frac{(a+bi)(a-bi)}{a^2+b^2}\\
        &= \frac{a^2 - b^2i^2}{a^2+b^2}\\
        &= \frac{a^2+b^2}{a^2+b^2} & (i^2 = -1)\\
        &= 1.
    \end{align*}
    Thus any non-zero complex number is a unit, meaning that $\C$ is a division ring. As $\C$ is also a commutative ring, this means that $\C$ is a field.
\end{example}

We look at fields in more detail in part III.

\begin{exercise}
    Which of the following rings, if any, are fields?
    \begin{partquestions}{\alph*}
        \item $\Z$
        \item $\Q$
    \end{partquestions}
\end{exercise}

\begin{exercise}\label{exercise-product-of-units-is-unit}
    Prove \myref{prop-product-of-units-is-unit}.
\end{exercise}

\section{Subrings}
We end this chapter off with an exploration about subrings.

\begin{definition}
    Let $R$ be a ring and $S$ be a subset of $R$. Then $S$ is a \textbf{subring}\index{subring} of $R$ if $S$ is also a ring under the same operations of addition and multiplication as $R$.
\end{definition}
\begin{remark}
    Equivalently, a subset $S$ of $R$ is a subring of $R$ if is satisfies the following conditions.
    \begin{itemize}
        \item $(S, +) \leq (R, +)$, that is, the subset $S$ under addition is a subgroup of $R$ under addition; and
        \item for all $a$ and $b$ in $S$ we have $ab \in S$, i.e. $S$ is closed under multiplication.
    \end{itemize}
\end{remark}

\begin{example}
    We know that $\Z$ and $\Q$ are rings, and clearly $\Z \subseteq \Q$. Hence $\Z$ is a subring of $\Q$. Similarly, since $\Q \subseteq \R$ and $\R \subseteq \C$, thus $\Q$ is a subring of $\R$ and $\R$ is a subring of $\C$.
\end{example}

\begin{example}
    Consider the set of \textbf{Gaussian integers}\index{Gaussian integers}
    \[
        \Z[i] = \{a + bi \vert a,b\in\mathbb{Z}\},
    \]
    read as ``$\Z$ adjoin $i$''. We first show that $\Z[i]$ is a subring of $\C$.
    \begin{proof}
        Clearly $\Z[i] \subseteq \C$. We show that $(\Z[i], +) \leq (\C, +)$.
        \begin{itemize}
            \item Clearly the identity of $(\C, +)$, which is 0, is inside $(\Z[i], +)$ as $0 = 0 + 0i$.
            \item For any $a + bi, c+di \in \Z[i]$, we have
            \[
                (a+bi) + (-(c+di)) = (a-c) + (b-d)i \in \Z[i].
            \]
        \end{itemize}
        Thus the subgroup test (\myref{thrm-subgroup-test}) tells us that $(\Z[i], +) \leq (\C, +)$.

        Now we show that $\Z[i]$ is closed under multiplication. Let $a + bi, c+di \in \Z[i]$. Note that $ac, bd, ad, bc \in \Z$; the product of the two Gaussian integers is
        \[
            (a+bi)(c+di) = (ac-bd) + (ad+bc)i \in \Z[i]
        \]
        so $\Z[i]$ is closed under multiplication.
        
        Therefore $\Z[i]$ is a subring of $\C$.
    \end{proof}
\end{example}
\begin{exercise}
    Show that
    \[
        R = \left\{\begin{pmatrix}a&a\\a&a\end{pmatrix} \vert a \in \R\right\}
    \]
    is a subring of $\Mn{2}{\mathbb{R}}$.
\end{exercise}

\newpage

\section{Problems}
\begin{problem}
    Let $R$ be a ring. Prove that if $u \in R$ is a unit then so is $-u$.
\end{problem}

\begin{problem}
    Prove that the trivial ring is the unique ring with identity in which $0 = 1$.
\end{problem}

\begin{problem}
    Let $R$ be a set with an operation $\ast$ such that for all elements $x$ and $y$ in $R$ we have $x \ast y \in R$. If $(R, \ast, \ast)$ is a ring, describe the elements in $R$.
\end{problem}

\begin{problem}
    Prove that it is impossible for an element of a ring $R$ to be both a zero divisor and a unit.
\end{problem}

\begin{problem}
    Let $R$ be a ring with identity 1, and let $x$ be an element from that ring.
    \begin{partquestions}{\roman*}
        \item Find \textbf{four} closed forms for the geometric series $1 + x + x^2 + x^3 + \cdots + x^n$.
        \item What are the condition(s) such that the closed forms are valid?
        \item Evaluate 112 in the ring $\Z_{37}$.
        \item Hence, using the result(s) above, evaluate
        \[
            1 + 2^3 + 2^6 + 2^9 + \cdots + 2^{72}
        \]
        in the ring $\Z_{37}$.
    \end{partquestions}
\end{problem}

\begin{problem}
    Show that
    \[
        \Q[\sqrt2] = \{a + b\sqrt2 \vert a,b \in \Q\}
    \]
    is a ring. Hence show it is a field.
\end{problem}

\begin{problem}
    Let
    \[
        R = \left\{\begin{pmatrix}a&b\\0&0\end{pmatrix} \vert a,b \in \R\right\}    
    \]
    be a ring under matrix addition and multiplication.
    \begin{partquestions}{\roman*}
        \item Show that $R$ has no identity.
        \item Show that $R$ contains a non-trivial subring $S$ with identity.
    \end{partquestions}
\end{problem}

\begin{problem}
    A ring $R$ is called a \textbf{Boolean ring}\index{Boolean ring} if $r^2 = r$ for all $r \in R$.
    \begin{partquestions}{\roman*}
        \item Show that $r = -r$ for all $r \in R$.
        \item Prove that every Boolean ring is commutative.
    \end{partquestions}
\end{problem}

\begin{problem}
    Let $R$ be a commutative ring with identity. We say that an element $x$ in $R$ is \textbf{nilpotent}\index{nilpotent} if there exists a positive integer $n$ such that $x^n = 0$.
    \begin{partquestions}{\roman*}
        \item Let $u \in R$ be a unit and $x \in R$ be nilpotent. Show that $ux$ is nilpotent.
        \item Show that $u - x$ is a unit.
    \end{partquestions}
\end{problem}
