\section{Integral Domains}
\begin{questions}
    \item To find a $a+bi \in \Z_5[i]$ such that there exists a $c+di \in \Z_5[i]$ where $(a+bi)(c+di) = 0$ but both $a+bi$ and $c+di$ are non-zero. Expanding $(a+bi)(c+di)$ yields $(ac-bd)+(ad+bc)i = 0$. Therefore we must have $ac-bd = 0$ and $ad+bc = 0$. For simplicity let's choose $a=c=1$. Using second equation we have $d+b = 0$ which means $d = -b$. Hence $(1 - b(-b))+(-b + b)i = 1+b^2 = 0$. Therefore choosing $b = 2$ would make it work. Therefore one solution is $a = 1, b = 2, c = 1, d = -2 = 3$; i.e. two zero divisors are $1+2i$ and $1+3i$.
    
    \item Take $w, z \in \Z[i]$ such that $w \neq 0$ and $wz = 0$. We want to show that $z = 0$. Let $z = a+bi$ and $w = c+di$. Since $w \neq 0$ we must have $c^2+d^2 \neq 0$. Now
    \[
        (a+bi)(c+di) = (ac-bd)+(ad+bc)i = 0
    \]
    which means $ac - bd = 0$ and $ad+bc = 0$. Multiplying first equation by $d$ yields $acd - bd^2 = 0$; multiplying second equation by $c$ yields $acd + bc^2 = 0$. Now summing them up yields $bc^2+bd^2 = b(c^2+d^2) = 0$ which hence means $b = 0$ since $c^2+d^2 \neq 0$. Therefore $ac - 0d = 0$ implies $ac = 0$ and $ad+0c = 0$ implies $ad = 0$. Squaring both equations and adding them up yields $a^2c^2 + a^2d^2 = a^2(c^2+d^2) = 0$ which hence means $a^2$ (and thus $a$) is zero. Therefore we have shown $z = 0$, meaning that there are no zero divisors in $\Z[i]$, so $\Z[i]$ is an integral domain.

    \item \begin{partquestions}{\alph*}
        \item Note that multiplication is commutative with identity $1 = 1 + 0\sqrt{n} \in R$. We just need to show that there are no zero divisors in $R$.
        
        Take $a+b\sqrt n, c+d\sqrt n \in R$ such that $a+b\sqrt n \neq 0$ but $(a+b\sqrt n)(c+d\sqrt n) = 0$. We want to show $c = d = 0$. Consider
        \[
            \left((a+b\sqrt n)(\underbrace{a-b\sqrt n}_{\neq 0})\right)\left((c+d\sqrt n)(\underbrace{c-d\sqrt n}_{\neq 0})\right) = 0.
        \]
        This means that $(a^2-nb^2)(c^2-nd^2) = 0$, so either $a^2-nb^2 = 0$ or $c^2-nd^2 = 0$.

        Now if $n < 0$ then clearly we have to have $c = d = 0$. Otherwise we have $a = b\sqrt n$ or $c = d\sqrt n$. But $\sqrt n$ is not an integer, so the only way for equality is if $c = d = 0$. Thus $\Z[\sqrt n]$ has no zero divisors, meaning $\Z[\sqrt n]$ is an integral domain.

        \item Consider $2 + \sqrt 2 \in \Z[\sqrt 2]$. Its multiplicative inverse is
        \begin{align*}
            \frac{1}{2+\sqrt2} &= \frac{2-\sqrt2}{(2+\sqrt2)(2-\sqrt2)}\\
            &= \frac{2-\sqrt2}{4-2}\\
            &= 1 - \frac12\sqrt2 \notin \Z[\sqrt2].
        \end{align*}
        This means that $2+\sqrt2$, a non-zero element in $\Z[\sqrt2]$, does not have an inverse in $\Z[\sqrt2]$. Therefore $\Z[\sqrt2]$ is not a field, meaning $R$ is not a field in the general case.
    \end{partquestions}

    \item For brevity let O$ = \begin{pmatrix}0&0\\0&0\end{pmatrix}$, I$ = \begin{pmatrix}1&0\\0&1\end{pmatrix}$, A$ = \begin{pmatrix}1&1\\1&0\end{pmatrix}$, and B$ = \begin{pmatrix}0&1\\1&1\end{pmatrix}$.
    
    \begin{partquestions}{\roman*}
        \item Clearly one sees that $R$ is a subset of $\Mn{2}{\Z_2}$.
        \begin{itemize}
            \item We show $(R, +)\leq(\Mn{2}{\Z_2},+)$.
            \begin{table}[h]
                \centering
                \begin{tabular}{|l|l|l|l|l|}
                    \hline
                    \textbf{+} & \textbf{O} & \textbf{I} & \textbf{A} & \textbf{B} \\ \hline
                    \textbf{O} & O          & I          & A          & B          \\ \hline
                    \textbf{I} & I          & O          & B          & A          \\ \hline
                    \textbf{A} & A          & B          & O          & I          \\ \hline
                    \textbf{B} & B          & A          & I          & O          \\ \hline
                \end{tabular}
            \end{table}
            
            From the Cayley table, clearly the identity of the ring $\Mn{2}{\Z_2}$ is in $R$ and $R$ is closed under addition. Hence $(R, +)\leq(\Mn{2}{\Z_2},+)$

            \item We show $R$ is closed under multiplication.
            \begin{table}[h]
                \centering
                \begin{tabular}{|l|l|l|l|l|}
                    \hline
                    $\boldsymbol{\cdot}$ & \textbf{O} & \textbf{I} & \textbf{A} & \textbf{B} \\ \hline
                    \textbf{O}           & O          & O          & O          & O          \\ \hline
                    \textbf{I}           & O          & I          & A          & B          \\ \hline
                    \textbf{A}           & O          & A          & B          & I          \\ \hline
                    \textbf{B}           & O          & B          & I          & A          \\ \hline
                \end{tabular}
            \end{table}
            
            From the Cayley table, clearly $R$ is closed under multiplication.
        \end{itemize}
        Therefore $R$ is a subring of $\Mn{2}{\Z_2}$.

        \item Since $R$ is a subring of $\Mn{2}{\Z_2}$, it is a ring. Furthermore, by the Cayley table of $(R, \cdot)$, we see that $R$ is commutative with identity I. Finally, one sees that $\mathrm{A}^{-1} = \mathrm{B}$, $\mathrm{B}^{-1} = \mathrm{A}$, and $\mathrm{I}^{-1} = \mathrm{I}$. Therefore all non-zero elements of $R$ have inverses. Hence $R$ is a field.
    \end{partquestions}
\end{questions}
