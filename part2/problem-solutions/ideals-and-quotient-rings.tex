\section{Ideals and Quotient Rings}
\begin{questions}
    \item Note that $36 = 2^2 \times 3^2$. So
    \begin{align*}
        \Ann{\Z_{36}}{\{15\}} &= \{r \in \Z_{36} \vert 15r = 0\}\\
        &= \{r \in \Z_{15} \vert 3(5r) = 0\}\\
        &= \{r \in \Z_{15} \vert r \text{ is a multiple of }2^2\times3 = 12\}\\
        &= \{0,12,24\}.
    \end{align*}

    \item We first show that $S$ is a subring of $\Z[i]$.
    \begin{itemize}
        \item The identity of $\Z[i]$ is $0 = 0 + 2(0)i \in S$.
        \item For any $a+2bi, c+2di \in S$, clearly $a+2bi + (-(c + 2di)) = (a-c) + 2(b-d)i \in S$.
        \item For any $a+2bi, c+2di \in S$, one sees that
        \begin{align*}
            (a+2bi)(c+2di) &= ac + 2adi + 2bci + 4bdi^2\\
            &= (ac-4bd) + 2(ad+bc)i\\
            &\in S.
        \end{align*}
    \end{itemize}
    Therefore $S$ is a subring of $\Z[i]$.

    We now show that $S$ is not an ideal of $\Z[i]$. Consider $1+2i \in \S$ and $1+i \in \Z[i]$. Then
    \begin{align*}
        (1+2i)(1+i) &= 1+i+2i+2i^2\\
        &= -1 + 3i\\
        &\notin S
    \end{align*}
    so there exists a $s \in S$ and a $r \in \Z[i]$ such that $rs\notin S$, meaning that $S$ is not a left ideal (and hence is not an ideal).

    \item We consider the test for ideal (\myref{thrm-test-for-ideal}).
    \begin{itemize}
        \item Note that $\begin{pmatrix}0&0\\0&0\end{pmatrix}=\begin{pmatrix}2(0)&2(0)\\2(0)&(0)\end{pmatrix}$ is in $I$ so $I$ is non-empty.
        \item $\begin{pmatrix}2a&2b\\2c&2d\end{pmatrix}-\begin{pmatrix}2e&2f\\2g&2h\end{pmatrix} = \begin{pmatrix}2(a-e)&2(b-f)\\2(c-g)&2(d-h)\end{pmatrix} \in I$.
        \item To show left ideal, take $\begin{pmatrix}2a&2b\\2c&2d\end{pmatrix} \in I$ and $\begin{pmatrix}e&f\\g&h\end{pmatrix} \in \Mn{2}{\Z}$. Then
        \begin{align*}
            \begin{pmatrix}2a&2b\\2c&2d\end{pmatrix}\begin{pmatrix}e&f\\g&h\end{pmatrix} &= \begin{pmatrix}2ae+2bg&2af+2bh\\2ce+2dg&2cf+2dh\end{pmatrix}\\
            &= \begin{pmatrix}2(ae+bg)&2(af+bh)\\2(ce+dg)&2(cf+dh)\end{pmatrix}\\
            &\in I
        \end{align*}
        so $I$ is a left ideal of $\Mn{2}{\Z}$.
        \item To show right ideal, take $\begin{pmatrix}a&b\\c&d\end{pmatrix} \in \Mn{2}{\Z}$ and $\begin{pmatrix}2e&2f\\2g&2h\end{pmatrix} \in I$. Then
        \begin{align*}
            \begin{pmatrix}a&b\\c&d\end{pmatrix}\begin{pmatrix}2e&2f\\2g&2h\end{pmatrix} &= \begin{pmatrix}2ae+2bg&2af+2bh\\2ce+2dg&2cf+2dh\end{pmatrix}\\
            &= \begin{pmatrix}2(ae+bg)&2(af+bh)\\2(ce+dg)&2(cf+dh)\end{pmatrix}\\
            &\in I
        \end{align*}
        so $I$ is a right ideal of $\Mn{2}{\Z}$.
    \end{itemize}
    Therefore by the test for ideal we have $I$ is an ideal of $\Mn{2}{\Z}$.

    \item \begin{partquestions}{\alph*}
        \item Suppose $I$ is not the trivial ring; we want to show that $I = R$. Since $I$ is non-trivial there there exists a non-zero element $a$ in $I$. Note that $a^{-1}$ exists since $R$ is a field, so $a$ is a unit. By \myref{prop-ideal-contains-unit-iff-ideal-is-whole-ring} this means $I = R$. Note that $\{0\} = \princ{0}$ and $R = \princ{1}$ by \myref{exercise-trivial-ideal-and-whole-ring-are-principal-ideals}, so $R$ is indeed a PID.

        \item Take a non-zero $x \in R$ and note that $\princ{x}$ is a non-trivial ideal. Since there are no proper ideals in $R$, thus $\princ{x} = R$. This means that $1 \in \princ{x}$ (since $\princ{x} = R$ is a ring with identity), meaning that there exists an element $r \in R$ such that $xr = 1$. Therefore $x$ is a unit.
        
        Since $x$ is an arbitrary non-zero element in $R$, this thus shows that all non-zero elements of the ring $R$ are units, meaning $R$ is a division ring.

        Finally, because $R$ is commutative, thus $R$ is a field.
    \end{partquestions}

    \item \begin{partquestions}{\alph*}
        \item Suppose $r \in \sqrt{\sqrt{I}}$, meaning that $r^m \in \sqrt{I}$ for some positive integer $m$, further meaning that $(r^m)^n \in I$ for some positive integer $n$. Note $(r^m)^n = r^{mn} \in I$, so $r \in \sqrt{I}$. Therefore $\sqrt{\sqrt{I}} \subseteq \sqrt{I}$.
        
        Now suppose $r \in \sqrt{I}$, meaning that $r^n \in I$ for some positive integer $n$. Note that $r = r^1 \in \sqrt{I}$, so $r \in \sqrt{\sqrt{I}}$. Hence $\sqrt{I} \subseteq \sqrt{\sqrt{I}}$.

        Therefore, since $\sqrt{\sqrt{I}} \subseteq \sqrt{I}$ and $\sqrt{I} \subseteq \sqrt{\sqrt{I}}$, thus $\sqrt{\sqrt{I}} = \sqrt{I}$.

        \item Suppose $r \in \sqrt{I\cap J}$, so $r^n \in I \cap J$ for some positive integer $n$. This means that $r^n \in I$ and $r^n \in J$. Hence $r \in \sqrt{I}$ and $r \in \sqrt{J}$ by definition of the radical, so $r \in \sqrt{I}\cap\sqrt{J}$. Thus $\sqrt{I\cap J} \subseteq \sqrt{I}\cap\sqrt{J}$.
        
        Now suppose $r \in \sqrt{I}\cap\sqrt{J}$, meaning that $r \in \sqrt{I}$ and $r \in \sqrt{J}$. Thus $r^m \in I$ and $r^n \in J$ for some positive integers $m$ and $n$. Note that
        \[
            (\underbrace{r^m}_{\text{In }I})^n \in I \text{ and } (\underbrace{r^n}_{\text{In }J})^m \in J
        \]
        so $r^{mn} \in I$ and $r^{mn} \in J$, meaning $r^{mn} \in I \cap J$. Thus $r \in \sqrt{I \cap J}$, showing that $\sqrt{I}\cap\sqrt{J} \subseteq \sqrt{I\cap J}$.

        Therefore $\sqrt{I}\cap\sqrt{J} = \sqrt{I\cap J}$.
    \end{partquestions}

    \item \begin{partquestions}{\alph*}
        \item Suppose $a \in m\Z\cap n\Z$. Thus $a \in m\Z$ and $a \in n\Z$, meaning that $a = mx = ny$ for some integers $x$ and $y$. Therefore $a = \lcm(m,n)z = lz$ for some integer $z$, meaning $a \in l\Z$. Hence $m\Z \cap n\Z \subseteq l\Z$.
        
        Now suppose $a \in l\Z$, so $a = lx$ for some integer $x$. Write $l = m\alpha = n\beta$ for some integers $\alpha$ and $\beta$. Note that
        \begin{align*}
            a &= (m\alpha)x = m(\alpha x) \in m\Z\\
            a &= (n\beta)x = n(\beta x) \in n\Z
        \end{align*}
        so $a \in m\Z \cap n\Z$. Thus $l\Z \subseteq m\Z \cap n\Z$.

        Therefore $m\Z\cap n\Z = l\Z$.

        \item Suppose $a \in m\Z + n\Z$, meaning that there exist integers $x$ and $y$ such that $a = mx + ny$. By definition of the GCD, write $m = d\alpha$ and $n = d\beta$ for some integers $\alpha$ and $\beta$. Hence
        \begin{align*}
            a &= (d\alpha)x + (d\beta)y\\
            &= d(\alpha x + \beta y)\\
            &\in d\Z
        \end{align*}
        so $m\Z + n\Z \subseteq d\Z$.

        On the other hand, suppose $a \in d\Z$, meaning $a = dt$ for some integer $t$. By B\'{e}zout's Lemma (\myref{lemma-bezout}), we may write $d = mx + ny$ for some integers $x$ and $y$. Hence
        \begin{align*}
            a &= (mx + ny)t\\
            &= m(xt) + n(yt)\\
            &\in m\Z + n\Z
        \end{align*}
        which means $d\Z \subseteq m\Z + n\Z$.

        Therefore $m\Z + n\Z = d\Z$.
    \end{partquestions}

    \item Let $r \in R$, and suppose $x = r + \Nilr{R} \in R/\Nilr{R}$ is nilpotent, i.e. there is a positive integer $n$ such that
    \[
        x^n = (r + \Nilr{R})^n = r^n + \Nilr{R} = 0 + \Nilr{R}.
    \]
    Coset Equality (\myref{lemma-coset-equality}) thus tells us that $r^n \in \Nilr{R}$. Note that $\Nilr{R}$ contains all the nilpotents of $R$. Thus $r^n$ is a nilpotent of $R$, i.e. there exists a positive integer $m$ such that $(r^n)^m = 0$. But clearly $(r^n)^m = r^{mn} = 0$, so $r$ is nilpotent, meaning $r \in \Nilr{R}$. Hence $x = r + \Nilr{R} = 0 + \Nilr{R}$, meaning that the only nilpotent of $R/\Nilr{R}$ is the zero element. Therefore $R/\Nilr{R}$ has no non-zero nilpotents.

    \item Suppose $R$ is a PID and $I$ is a non-zero prime ideal. Let $J$ be an ideal such that $I \subseteq J \subseteq R$. Since $R$ is a PID, write $I = \princ{a}$ and $J = \princ{b}$ for some elements $a$ and $b$ in $R$. Note $a \in \princ{a} = I \subseteq J = \princ{b}$, so there exists an $r \in R$ such that $a = rb$. Now since $a = rb \in \princ{a} = I$ and $I$ is prime, therefore $r \in I$ or $b \in I$.
    \begin{itemize}
        \item If $r \in I$, write $r = sa$ for some $s \in R$. Then
        \[
            a = rb = (sa)b = (as)b = a(sb)
        \]
        since an integral domain is commutative. Thus $a - a(sb) = a(1-sb) = 0$. Now as $R$ is an integral domain thus either $a = 0$ (impossible since $a \neq 0$) or $1-sb = 0$. So $1-sb = 0$, meaning $sb = 1 \in J$ since $b \in J$. By \myref{prop-ideal-contains-unit-iff-ideal-is-whole-ring} we have $J = R$.
        \item If instead $b \in I$, take any $x \in J = \princ{b}$, so $x = rb$ for some $r \in R$. Thus $x = rb \in I$ since $b \in I$, so $J \subseteq I$. But $I \subseteq J$, so $J = I$.
    \end{itemize}
    Therefore we have shown that $I$ is maximal.

    \item First we work in the forward direction. Suppose $\princ{a} = \princ{b}$. As $a \in \princ{a} = \princ{b}$, thus $a = bx$ for some $x \in R$. Also, as $b \in \princ{b} = \princ{a}$, thus $b = ay$ for some $y \in R$. Therefore
    \[
        b = ay = (bx)y = b(xy)
    \]
    which means $xy = 1$. Thus $x$ and $y$ are units, meaning $a = bx$ with $x$ being a unit.

    Now we work in the reverse direction; suppose $a = bu$ for some unit $u$ in $D$.
    \begin{itemize}
        \item Take $r \in \princ{a}$, so $r = ax$ for some $x$ in $D$. Thus $r = (bu)x = b(ux) \in \princ{b}$, so $\princ{a} \subseteq \princ{b}$.
        \item Note $b = au^{-1}$ since $u$ is a unit. Take $s \in \princ{b}$, so $s = by$ for some $y$ in $D$. But as $b = au^{-1}$, hence $s = (au^{-1})y = a(u^{-1}y) \in \princ{a}$, so $\princ{b} \subseteq \princ{a}$.
    \end{itemize}
    Therefore we see that $\princ{a} = \princ{b}$.
\end{questions}
