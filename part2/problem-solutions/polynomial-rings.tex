\section{Polynomial Rings}
\begin{questions}
    \item Note
    \begin{align*}
        1 - 2x + 3x^2 - 4x^3 + 5x^4 - 6x^5 &= 1 + x + 0x^2 + 2x^3 + 2x^4 + 0x^5\\
        &= 1 + x + 2x^3 + 2x^4
    \end{align*}
    so a degree 4 polynomial that is equal to the given polynomial is $1 + x + 2x^3 + 2x^4$.

    \item We work step-by-step.
    \begin{align*}
        &\left((x + 3) + I\right)\left((2x^2 + 3x - 1) + I\right)\\
        &= \left((x + 3)(2x^2+3x-1)\right) + I\\
        &= \left(2x^3 + 9x^2 + 8x - 3\right) + I\\
        &= \left((2x+3)\underbrace{(x^2+3x-1)}_{\text{In }I} + x\right) + I\\
        &= x + I
    \end{align*}

    \item \begin{partquestions}{\alph*}
        \item False. $2x-1$ is a polynomial with integer coefficients with no integer zeroes.

        \item False. By definition, zeroes must always come from the ground ring, so if $f(x) \in \Z[x]$ has zero(es) they must be integers.

        \item True. $2x-1 \in \R[x]$ is a polynomial with integer coefficients and has a non-integer zero of $\frac12$.
    \end{partquestions}

    \item \begin{partquestions}{\alph*}
        \item We guess the form of $f(x)$ for $4x^2 + 2x + 1$.

        Guess first that $f(x)$ is a degree 0 polynomial, i.e. $f(x) = a$. Then $4ax^2 + 2ax + a = 1$, meaning $a = 1$. But that is the same as the original polynomial, so we conclude $f(x)$ cannot be a polynomial of degree 0.

        Guess next that $f(x)$ is a degree 1 polynomial, i.e. $f(x) = ax + b$. Then $(4x^2+2x+1)(ax+b) = 1$, meaning $4ax^3 + 2(a+2b)x^2 + (a+2b)x + b = 1$. Thus $b = 1$, otherwise it would not be possible for its product to be 1. Thus it reduces to $4ax^3 + 2(a+2)x^2 + (a+2)x + 1$, which means that $a+2 = 0$. Therefore $a = -2 = 6$. We check that
        \begin{align*}
            (4x^2+2x+1)(6x+1) &= 24x^3 + 16x^2 + 8x + 1\\
            &= 0x^3 + 0x^2 + 0x + 1\\
            &= 1
        \end{align*}
        so $f(x) = 6x+1$ works.

        \item We note $\Z_7$ is a field (\myref{example-Zp-is-field}) and so $\Z_7$ is an integral domain. This means $\Z_7[x]$ is an integral domain (\myref{thrm-integral-domain-iff-polynomial-ring-is-integral-domain}). So a unit of $\Z_7[x]$ is a unit of $\Z_7$ (\myref{prop-unit-of-ring-iff-unit-of-polynomial-ring}), which means only constants can be units. Thus $x^2 + 2x + 1$ has no corresponding $f(x)$ such that $(x^2+2x+1)f(x) = 1$.
    \end{partquestions}

    \item Using polynomial long division (\myref{thrm-polynomial-long-division}) we write
    \[
        f(x) = q(x)(x-a) + r(x) \text{ with } r(x) \text{ or } \deg r(x) < \deg(x-a) = 1.
    \]
    Thus $r(x) = b \in F$. Evaluating $f(x)$ at a yields
    \[
        f(a) = q(x)(a-a) + b = b
    \]
    which means $f(a)$ is the remainder of the division of $f(x)$ by $x-a$.

    \item \begin{partquestions}{\roman*}
        \item Note that
        \begin{align*}
            \ker\phi &= \{f(x) \in \Z[x] \vert \phi(f(x)) = 0\}\\
            &= \{f(x) \in \Z[x] \vert f(-2) = 0\}\\
            &= I.
        \end{align*}
        \myref{prop-kernel-is-an-ideal} tells us that $\ker\phi$ is an ideal of $\Z[x]$, so $I$ is an ideal of $\Z[x]$.

        \item We first show that $\phi$ is surjective. Let $n \in \Z$, note that $n$ is a degree zero polynomial, so $n \in \Z[x]$. Clearly $\phi(n) = n$ so $n$ is its own pre-image. Therefore $\im\phi = \Z$.

        By FRIT (\myref{thrm-ring-isomorphism-1}),
        \[
            \Z[x]/I \cong \Z.
        \]
        Note that $\Z$ is an integral domain but not a field. Thus $I$ is prime but not maximal.
    \end{partquestions}

    \item For brevity let $I = \princ{x} = \{xP(x) \vert P(x) \in \Z[x]\}$. This means that $I$ is the set of polynomials with integer coefficients and with constant term 0. Now suppose $f(x), g(x) \in \Z[x]$; write
    \begin{align*}
        f(x) &= a_0 + a_1x + \cdots + a_mx^m\\
        g(x) &= b_0 + b_1x + \cdots + b_nx^n
    \end{align*}
    where $a_i, b_i \in \Z$ and $m$ and $n$ are positive integers. Note that
    \[
        f(x)g(x) = a_0b_0 + (a_1b_0+a_0b_1)x + \cdots.
    \]
    Now if $f(x)g(x) \in I$, this means that $a_0b_0 = 0$. Hence either $a_0 = 0$ or $b_0 = 0$, meaning that either $f(x)$ has zero constant term (so $f(x) \in I$) or $g(x)$ has zero constant term (so $g(x) \in I$). Thus $I$ is prime.

    \item We show that $\Z[x] / \princ{x} \cong \Z$ by using the FRIT.

    Let $\phi: \Z[x] \to \Z$ be defined such that $p(x) \mapsto p(0)$.

    We claim $\phi$ is surjective. Let $n \in \Z$. Note that
    \[
        n = n + 0x + 0x^2 + \cdots \in \Z[x]
    \]
    so
    \begin{align*}
        \phi(n) &= \phi(n + 0x + 0x^2 + \cdots)\\
        &= n + 0(0) + 0(0)^2 + \cdots\\
        &= n
    \end{align*}
    which means that $n$ is its own pre-image. Thus $\im\phi = \Z$.

    Now we find the kernel of $\phi$. Suppose $p(x) \in \ker\phi$, i.e. $\phi(p(x)) = 0$. This means that $p(0) = 0$. The general form for a univariate polynomial $p(x)$ is $a_0 + a_1x + a_2x^2 + \cdots$, so if $p(0) = 0$ then $a_0 = 0$. Thus,
    \begin{align*}
        p(x) &= 0 + a_1x + a_2x^2 + \cdots\\
        &= x(a_1 + a_2x + \cdots)\\
        &= xq(x)
    \end{align*}
    where $q(x) \in \Z[x]$. Therefore
    \[
        \ker\phi = \{xq(x) \vert q(x) \in \Z[x]\} = \princ{x}.
    \]

    Thus, the FRIT (\myref{thrm-ring-isomorphism-1}) tells us that
    \[
        \Z[x] / \princ{x} \cong \Z.
    \]

    \item We first note that the polynomial functions $f: x \mapsto x$ and $g: x \mapsto x^5$ are equal since
    \begin{align*}
        g(0) &= 0^5 = 0 = f(0),\\
        g(1) &= 1^5 = 1 = f(1),\\
        g(1) &= 2^5 = 32 = 2 = f(2),\\
        g(3) &= 3^5 = 243 = 3 = f(3), \text{ and}\\
        g(4) &= 4^5 = 1024 = 4 = f(4).
    \end{align*}
    One sees thereafter that the polynomial $h(x) = x^5 + 4x$ has the entirety of $\Z_5$ as zeroes, since
    \begin{align*}
        h(0) &= 0^5 + 4(0) = 0,\\
        h(1) &= 1^5 + 4(1) = 5 = 0,\\
        h(2) &= 2^5 + 4(2) = 40 = 0,\\
        h(3) &= 3^5 + 4(3) = 255 = 0, \text{ and}\\
        h(4) &= 4^5 + 4(4) + 1040 = 0.
    \end{align*}
    We note that the polynomial $(x^n)(x^5 + 4x)$ where $n$ is a non-negative integer would also have the entirety of $\Z_5$ as zeroes. Therefore a set $S$ that works is
    \[
        S = \{x^n(x^5+4x) \vert n \geq 0\}
    \]
    and we note that this is an infinite set where distinct polynomials in $S$ have different degrees.

    \item \begin{partquestions}{\roman*}
        \item Suppose $g(x) \in I$ is a non-zero polynomial of minimum degree.
        \begin{itemize}
            \item Since $g(x) \in I$ thus $\princ{g(x)} \subseteq I$.
            \item Now suppose $f(x) \in I$. By polynomial long division (\myref{thrm-polynomial-long-division}) write $f(x) = q(x)g(x) + r(x)$ where $r(x) = 0$ or $\deg r(x) < \deg g(x)$. Note $r(x) = f(x) - \underbrace{g(x)q(x)}_{\text{In }I} \in I$, so the minimality of $\deg g(x)$ means that $\deg r(x) \not< \deg g(x)$, i.e. $r(x) = 0$. Thus $f(x) = q(x)g(x) \in \princ{g(x)}$, meaning $I \subseteq \princ{g(x)}$.
        \end{itemize}
        This shows that $I = \princ{g(x)}$.

        \item If $g(x) = 0$ then $I = \princ{0} = \{0\}$, which is the zero ideal. Thus $g(x) \neq 0$. Let $g(x)$ have degree $n$. Now suppose there exists a $h(x) \in I$ that has a smaller degree than $g(x)$. By \textbf{(i)} we know $I = \princ{h(x)} = \{f(x)h(x) \vert f(x) \in F[x]\}$. Let $h(x)$ have degree $m$, so $g(x) = h(x) + x^{n-m}k(x)$ for some polynomial $k(x)$ with $\deg k(x) < m$ (since if $\deg k(x) = m$ this means $g(x)$ also has degree $m$, a contradiction). We also know
        \begin{align*}
            I &= \princ{g(x)}\\
            &= \{f(x)g(x) \vert f(x) \in F[x]\}\\
            &= \{f(x)(h(x) + x^{n-m}k(x)) \vert f(x) \in F[x]\}\\
            &= \{f(x)h(x) + x^{n-m}f(x)k(x) \vert f(x) \in F[x]\}\\
            &\subset \{f(x)h(x) \vert f(x) \in F[x]\}\\
            &= \princ{h(x)}\\
            &= I,
        \end{align*}
        a contradiction.
    \end{partquestions}

    \item Consider the map $\phi: R[x] \to R[x^k]$ where
    \[
        \sum_{i=0}^n a_ix^i \mapsto \sum_{i=0}^na_ix^{ki}.
    \]
    We show that $\phi$ is a ring isomorphism. For brevity let $f(x), g(x) \in R[x]$ where, without loss of generality, assume $m \geq n$,
    \begin{align*}
        f(x) &= \sum_{i=0}^m a_ix^i,\\
        g(x) &= \sum_{i=0}^n b_ix^i,
    \end{align*}
    and set $b_i = 0$ for any $i > n$.
    \begin{itemize}
        \item \textbf{Homomorphism}: Note that
        \begin{align*}
            \phi(f(x) + g(x)) &= \phi\left(\sum_{i=0}^m(a_i+b_i)x^i\right)\\
            &= \sum_{i=0}^m(a_i+b_i)x^{ki}\\
            &= \sum_{i=0}^ma_ix^{ki} + \sum_{i=0}^mb_ix^{ki}\\
            &= \sum_{i=0}^ma_ix^{ki} + \sum_{i=0}^mb_ix^{ki} & (\text{since } b_i = 0 \text{ for } i > n)\\
            &= \phi(f(x)) + \phi(g(x))
        \end{align*}
        and
        \begin{align*}
            \phi(f(x)g(x)) &= \phi\left(\sum_{r=0}^{m+n}\left(\sum_{i=0}^ra_ib_{r-i}\right)x^r\right)\\
            &= \sum_{r=0}^{m+n}\left(\sum_{i=0}^ra_ib_{r-i}\right)x^{kr}\\
            &= \left(\sum_{i=0}^ma_ix_{ki}\right)\left(\sum_{i=0}^nb_ix_{ki}\right)\\
            &= \phi(f(x))\phi(g(x))
        \end{align*}
        so $\phi$ is a ring homomorphism.

        \item \textbf{Injective}: Suppose $\phi(f(x)) = \phi(g(x))$. Thus
        \[
            \sum_{i=0}^m a_ix^{ki} = \sum_{i=0}^n b_ix^{ki}
        \]
        which clearly means that $a_i = b_i$ for all $0 \leq i \leq m$ (where, again, $b_i = 0$ if $i > n$). Thus
        \[
            f(x) = \sum_{i=0}^m a_ix^i = \sum_{i=0}^n b_ix^i = g(x)
        \]
        which shows that $\phi$ is injective.

        \item \textbf{Surjective}: Suppose $h(x) = c_0 + c_kx^k + c_{2k}x^{2k} + \cdots + c_{pk}x^{pk} \in R[x^k]$. Set $a_i = c_{ki}$ for all $0 \leq i \leq p$ and one sees that
        \[
            \phi\left(\sum_{i=0}^pa_ix^i\right) = \sum_{i=0}^pa_ix^{ki} = \sum_{i=0}^pc_{ki}x^{ki} = h(x)
        \]
        which means that any $h(x) \in R[x^k]$ has a pre-image in $R[x]$. Thus $\phi$ is surjective.
    \end{itemize}
    Hence $\phi$ is a ring isomorphism, meaning $R[x] \cong R[x^k]$.

    \item \begin{partquestions}{\roman*}
        \item Let $f(x), g(x) \in R[x]$ where, without loss of generality, assume $m \geq n$,
        \begin{align*}
            f(x) &= \sum_{i=0}^m a_ix^i,\\
            g(x) &= \sum_{i=0}^n b_ix^i,
        \end{align*}
        and set $b_i = 0$ for any $i > n$. Note that
        \begin{align*}
            \psi(f(x) + g(x)) &= \psi\left(\sum_{i=0}^m(a_i+b_i)x_i\right)\\
            &= \sum_{i=0}^m\phi(a_i + b_i)x_i\\
            &= \sum_{i=0}^m\left(\phi(a_i) + \phi(b_i)\right)x_i\\
            &= \sum_{i=0}^m\phi(a_i)x_i + \sum_{i=0}^m\phi(b_i)x_i\\
            &= \sum_{i=0}^m\phi(a_i)x_i + \sum_{i=0}^n\phi(b_i)x_i & (\text{since } b_i = 0 \text{ for } i > n)\\
            &= \psi(f(x)) + \psi(g(x))
        \end{align*}
        and
        \begin{align*}
            \psi(f(x)g(x)) &= \psi\left(\sum_{k=0}^{m+n}\left(\sum_{i=0}^ka_ib_{k-i}\right)x^k\right)\\
            &= \sum_{k=0}^{m+n}\left(\phi\left(\sum_{i=0}^ka_ib_{k-i}\right)x^k\right)\\
            &= \sum_{k=0}^{m+n}\left(\left(\sum_{i=0}^k\phi(a_ib_{k-i})\right)x^k\right)\\
            &= \sum_{k=0}^{m+n}\left(\left(\sum_{i=0}^k\phi(a_i)\phi(b_{k-i})\right)x^k\right)\\
            &= \left(\sum_{i=0}^m\phi(a_i)x_i\right)\left(\sum_{i=0}^n\phi(b_i)x_i\right)\\
            &= \psi(f(x))\psi(g(x))
        \end{align*}
        so $\psi$ is a ring homomorphism.

        \item We prove that $\psi$ is both injective and surjective. We use the same functions and notation as in \textbf{(i)}.
        \begin{itemize}
            \item \textbf{Injective}: Suppose $\psi(f(x)) = \psi(g(x))$. Thus
            \[
                \sum_{i=0}^m \phi(a_i)x^i = \sum_{i=0}^n \phi(b_i)x^i
            \]
            by definition of $\psi$, which means $\phi(a_i) = \phi(b_i)$ for all $0 \leq i \leq m$ (where $b_i = 0$ for $i > n$). Now as $\phi$ is an isomorphism, thus we see $a_i = b_i$ for all $0 \leq i \leq m$ (where, again, $b_i = 0$ for $i > n$). Therefore
            \[
                \sum_{i=0}^m a_ix^i = \sum_{i=0}^n b_ix^i
            \]
            which therefore means $f(x) = g(x)$. So $\psi$ is an injective function.

            \item \textbf{Surjective}: Suppose $F(x) = a_0 + a_1x + a_2x^2 + \cdots + a_mx^m \in S[x]$. Let $G(x) = \phi^{-1}(a_0) + \phi^{-1}(a_1)x + \cdots + \phi^{-1}(a_m)x^m \in R[x]$. Then note
            \[
                \psi(G(x)) = \sum_{i=0}^m\phi(\phi^{-1}(a_i))x_i = \sum_{i=0}^ma_ix_i = F(x)
            \]
            so any $F(x) \in S[x]$ has a pre-image in $R[x]$. Therefore $\psi$ is a surjective function.
        \end{itemize}

        So $\psi$ is a bijection. Furthermore by \textbf{(i)} we know that $\psi$ is a homomorphism. Therefore $\psi$ is a ring isomorphism, meaning $R[x] \cong S[x]$.
    \end{partquestions}

    \item Let $\ideal{a} = \princ{2}$ and $\ideal{b} = \princ{x}$ be principal ideals. Note that
    \[
        I = \ideal{a} + \ideal{b} = \{2f(x) + xg(x) \vert f(x),g(x) \in \Z[x]\}
    \]
    is a sum of ideals and thus is an ideal (\myref{prop-sum-of-ideals-is-ideal}). We show that $I$ is not a principal ideal.

    Suppose on the contrary that $I = \princ{F(x)}$ for some $F(x) \in \Z[x]$. Then $2 \in I$, meaning that $2 = f(x)F(x)$ for some $f(x) \in \Z[x]$. We note that $\Z$ is an integral domain, so $\Z[x]$ is an integral domain (\myref{thrm-integral-domain-iff-polynomial-ring-is-integral-domain}) which means that
    \[
        \deg(f(x)F(x)) = \deg f(x) + \deg F(x) = \deg 2 = 0,
    \]
    so $f(x)$ and $F(x)$ are both constant polynomials. Thus $f(x) = m \in \Z$ and $F(x) = n \in \Z$. Actually, since $2 \in I$, one sees that the only possibilities for $n$ are $\pm1$ and $\pm2$.

    Note that an element of $I$ takes the form $2a_0 + a_1x + a_2x^2 + \cdots + a_nx^n$.
    \begin{itemize}
        \item We note $\princ{1} = \princ{-1}$. But this means that $1 \in I$ which is impossible (since the constant term must be an even integer).
        \item We note $\princ{2} = \princ{-2}$. But this means that $I = \princ{2}$ is the set of all polynomials of even coefficients. But $2 + x \in I$ although $2 + x \notin \princ{2}$, a contradiction.
    \end{itemize}

    Therefore $I$ is non-principal, which means that $\Z[x]$ is not a PID.
\end{questions}
