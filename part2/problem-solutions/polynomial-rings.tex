\section{Polynomial Rings}
\begin{questions}
    \item We work step-by-step.
    \begin{align*}
        &\left((x + 3) + I\right)\left((2x^2 + 3x - 1) + I\right)\\
        &= \left((x + 3)(2x^2+3x-1)\right) + I\\
        &= \left(2x^3 + 9x^2 + 8x - 3\right) + I\\
        &= \left((2x+3)\underbrace{(x^2+3x-1)}_{\text{In }I} + x\right) + I\\
        &= x + I
    \end{align*}
    
    \item For brevity let $I = \princ{x} = \{xP(x) \vert P(x) \in \Z[x]\}$. This means that $I$ is the set of polynomials with integer coefficients and with constant term 0. Now suppose $f(x), g(x) \in \Z[x]$; write
    \begin{align*}
        f(x) &= a_0 + a_1x + \cdots + a_mx^m\\
        g(x) &= b_0 + b_1x + \cdots + b_nx^n
    \end{align*}
    where $a_i, b_i \in \Z$ and $m$ and $n$ are positive integers. Note that
    \[
        f(x)g(x) = a_0b_0 + (a_1b_0+a_0b_1)x + \cdots.
    \]
    Now if $f(x)g(x) \in I$, this means that $a_0b_0 = 0$. Hence either $a_0 = 0$ or $b_0 = 0$, meaning that either $f(x)$ has zero constant term (so $f(x) \in I$) or $g(x)$ has zero constant term (so $g(x) \in I$). Thus $I$ is prime.

    \item \begin{partquestions}{\roman*}
        \item Let $f(x), g(x) \in \Z[x]$. Then
        \[
            \phi(f(x) + g(x)) = f(-2) + g(-2) = \phi(f(x)) + \phi(g(x))
        \]
        and
        \[
            \phi(f(x)g(x)) = f(-2)g(-2) = \phi(f(x))\phi(g(x))
        \]
        so $\phi$ is a ring homomorphism.

        \item Note that
        \begin{align*}
            \ker\phi &= \{f(x) \in \Z[x] \vert \phi(f(x)) = 0\}\\
            &= \{f(x) \in \Z[x] \vert f(-2) = 0\}\\
            &= I.
        \end{align*}
        \myref{prop-kernel-is-an-ideal} tells us that $\ker\phi$ is an ideal of $\Z[x]$, so $I$ is an ideal of $\Z[x]$.

        \item We first show that $\phi$ is surjective. Let $n \in \Z$, note that $n$ is a degree zero polynomial, so $n \in \Z[x]$. Clearly $\phi(n) = n$ so $n$ is its own pre-image. Therefore $\im\phi = \Z$.
        
        By FRIT (\myref{thrm-ring-isomorphism-1}),
        \[
            \Z[x]/I \cong \Z.
        \]
        Note that $\Z$ is an integral domain but not a field. Thus $I$ is prime but not maximal.
    \end{partquestions}

    \item We show that $\Z[x] / \princ{x} \cong \Z$ by using the FRIT.

    Let $\phi: \Z[x] \to \Z$ be defined such that $p(x) \mapsto p(0)$.

    We claim $\phi$ is surjective. Let $n \in \Z$. Note that
    \[
        n = n + 0x + 0x^2 + \cdots \in \Z[x]
    \]
    so
    \begin{align*}
        \phi(n) &= \phi(n + 0x + 0x^2 + \cdots)\\
        &= n + 0(0) + 0(0)^2 + \cdots\\
        &= n
    \end{align*}
    which means that $n$ is its own pre-image. Thus $\im\phi = \Z$.

    Now we find the kernel of $\phi$. Suppose $p(x) \in \ker\phi$, i.e. $\phi(p(x)) = 0$. This means that $p(0) = 0$. The general form for a univariate polynomial $p(x)$ is $a_0 + a_1x + a_2x^2 + \cdots$, so if $p(0) = 0$ then $a_0 = 0$. Thus,
    \begin{align*}
        p(x) &= 0 + a_1x + a_2x^2 + \cdots\\
        &= x(a_1 + a_2x + \cdots)\\
        &= xq(x)
    \end{align*}
    where $q(x) \in \Z[x]$. Therefore
    \[
        \ker\phi = \{xq(x) \vert q(x) \in \Z[x]\} = \princ{x}.
    \]

    Thus, the FRIT (\myref{thrm-ring-isomorphism-1}) tells us that
    \[
        \Z[x] / \princ{x} \cong \Z.
    \]

    \item Using polynomial long division (\myref{thrm-polynomial-long-division}) we write
    \[
        f(x) = q(x)(x-a) + r(x) \text{ with } r(x) \text{ or } \deg r(x) < \deg(x-a) = 1.
    \]
    Thus $r(x) = b \in F$. Evaluating $f(x)$ at a yields
    \[
        f(a) = q(x)(a-a) + b = b
    \]
    which means $f(a)$ is the remainder of the division of $f(x)$ by $x-a$.

    \item \begin{partquestions}{\roman*}
        \item Suppose $g(x) \in I$ is a non-zero polynomial of minimum degree.
        \begin{itemize}
            \item Since $g(x) \in I$ thus $\princ{g(x)} \subseteq I$.
            \item Now suppose $f(x) \in I$. By polynomial long division (\myref{thrm-polynomial-long-division}) write $f(x) = q(x)g(x) + r(x)$ where $r(x) = 0$ or $\deg r(x) < \deg g(x)$. Note $r(x) = f(x) - \underbrace{g(x)q(x)}_{\text{In }I} \in I$, so the minimality of $\deg g(x)$ means that $\deg r(x) \not< \deg g(x)$, i.e. $r(x) = 0$. Thus $f(x) = q(x)g(x) \in \princ{g(x)}$, meaning $I \subseteq \princ{g(x)}$.
        \end{itemize}
        This shows that $I = \princ{g(x)}$.

        \item If $g(x) = 0$ then $I = \princ{0} = \{0\}$, which is the zero ideal. Thus $g(x) \neq 0$. Let $g(x)$ have degree $n$. Now suppose there exists a $h(x) \in I$ that has a smaller degree than $g(x)$. By \textbf{(a)} we know $I = \princ{h(x)} = \{f(x)h(x) \vert f(x) \in F[x]\}$. Let $h(x)$ have degree $m$, so $g(x) = h(x) + x^{n-m}k(x)$ for some polynomial $k(x)$ with $\deg k(x) < m$ (since if $\deg k(x) = m$ this means $g(x)$ also has degree $m$, a contradiction). We also know
        \begin{align*}
            I &= \princ{g(x)}\\
            &= \{f(x)g(x) \vert f(x) \in F[x]\}\\
            &= \{f(x)(h(x) + x^{n-m}k(x)) \vert f(x) \in F[x]\}\\
            &= \{f(x)h(x) + x^{n-m}f(x)k(x) \vert f(x) \in F[x]\}\\
            &\subset \{f(x)h(x) \vert f(x) \in F[x]\}\\
            & \princ{h(x)}\\
            &= I,
        \end{align*}
        a contradiction.
    \end{partquestions}
\end{questions}
