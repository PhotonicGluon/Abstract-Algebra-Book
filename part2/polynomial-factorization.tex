\chapter{Factorization of Polynomials}
One of the underappreciated facts about working with polynomials with real coefficients, like in high school, is that they can be factored simply. The idea of an `irreducible' polynomial is simple to understand within the reals -- it simply cannot be `broken down' into `smaller' polynomials without `breaking the rules'. We explore how polynomials are factored in a more general setting -- polynomial rings.

\section{(Ir)reducible Polynomials}
We formally define what (ir)reducible polynomials are in a polynomial ring that is an integral domain.
\begin{definition}
    Let $D$ be an integral domain and $f(x) \in D[x]$ be a non-zero, non-unit polynomial. Then $f(x)$ is \textbf{irreducible over $D$}\index{polynomial!irreducible} if, whenever we write $f(x) = p(x)q(x)$ for some polynomials $p(x)$ and $q(x)$ in $D[x]$, then either $p(x)$ is a unit or $q(x)$ is a unit in $D[x]$.

    A non-zero, non-unit polynomial that is not irreducible over $D$ is said to be \textbf{reducible over $D$}\index{polynomial!reducible}.
\end{definition}

\begin{example}
    The polynomial $x + 1$ is irreducible over $\Z$ since we are unable to express $x+1$ as a product of non-unit polynomials in $\Z[x]$.
\end{example}

\begin{example}
    The polynomial $2x + 2$ is reducible over $\Z$ since $2x+2 = 2(x+1)$ and both 2 and $x+1$ are not units in $\Z[x]$.
\end{example}

It is often more useful to talk about irreducible polynomials over a field. In this case, we have this useful theorem to simplify the task of determining the irreducibility of a polynomial.
\begin{theorem}\label{thrm-irreducible-iff-not-expressable-as-product-of-smaller-polynomials}
    Let $F$ be a field. A non-constant polynomial $f(x) \in F[x]$ with degree $n$ is irreducible if and only if $f(x)$ cannot be expressed as a product of two polynomials each of degree lower than $n$.
\end{theorem}
\begin{proof}
    For the forward direction, assume that a non-constant polynomial $f(x)$ with degree $n$ is irreducible. So there does not exist two non-unit polynomials $p(x)$ and $q(x)$ such that $f(x) = p(x)q(x)$, which thus means that it cannot be expressed as two polynomials of degree lower than $n$.

    For the reverse direction, assume that $f(x)$ with degree $n$ cannot be expressed as a product of two polynomials of degree lower than $n$. This means that it is possible for $f(x) = p(x)q(x)$ where at least one of the polynomials $p(x)$ and $q(x)$ have degree $n$. Without loss of generality assume $p(x)$ has degree $n$, thereby meaning $q(x)$ has degree $n - n = 0$ (see \myref{thrm-polynomial-degree-properties}). Thus $q(x)$ is a constant polynomial, so $q(x) = u \in F$ (\myref{prop-constant-polynomial-iff-ring-element}). Note $u \neq 0$ since otherwise $f(x) = 0$, a constant polynomial (a contradiction). But every non-zero element in $F$ is a unit. Therefore the only factorization of $f(x)$ is as a product of a unit and a polynomial of equal degree. Thus $f(x)$ is irreducible.
\end{proof}

\begin{example}
    One sees that $2x^2 + 4$ is reducible over $\Z$ since $2x^2 + 4 = 2(x^2 + 2)$ and neither 2 nor $x^2 + 2$ are units in $\Z[x]$. However, in $\Q$, the given polynomial is irreducible by \myref{thrm-irreducible-iff-not-expressable-as-product-of-smaller-polynomials}.
\end{example}

\begin{example}
    The polynomial $2x^2 + 4$ is irreducible in $\Q$ and $\R$, but reducible in $\C$ since $2x^2 + 4 = (2x-4i)(x+2i)$.
\end{example}

\begin{exercise}
    Let the polynomial $f(x) = 3x^2 - 6$. Which of the following integral domains is $f(x)$ irreducible?
    \begin{partquestions}{\alph*}
        \item $\Z$
        \item $\Q$
        \item $\R$
    \end{partquestions} 
\end{exercise}

\section{Tests for (Ir)reducibility}
Trying to see if a polynomial is (ir)reducible by definition is not a trivial task. For example, it is hard to see that $x^2 + 1$ is irreducible over $\Z_3$ but reducible over $\Z_5$ by just using the definition directly. Therefore we develop some tests for (ir)reducibility to more easily see which polynomials are reducible or irreducible.

\subsection{Zeroes Of A Polynomial}
The first test is a consequence of the factor theorem.
\begin{theorem}\label{thrm-degree-above-1-reducible-if-has-zero}
    Let $F$ be a field and $f(x) \in F[x]$ have degree above 1. If $f(x)$ has a zero then $f(x)$ is reducible.
\end{theorem}
\begin{proof}
    Since $f(x)$ has a zero, say $\alpha \in F$, we may write $f(x) = (x-\alpha)q(x)$ (\myref{corollary-factor-theorem}). Note $\deg f(x) = \deg(x-\alpha) + \deg q(x) = 1 + \deg q(x)$ (\myref{thrm-polynomial-degree-properties}), and clearly $1 \leq \deg q(x) < \deg f(x)$ (since degree of $f(x)$ is above 1), so $f(x)$ is expressible as a product of two polynomials of lower degree in $F[x]$. Thus $f(x)$ is reducible in $F$. 
\end{proof}

The next test provides a partial converse to the above theorem for polynomials of degree 2 or 3.
\begin{theorem}\label{thrm-degree-2-or-3-irreducible-iff-has-no-zeroes}
    Let $F$ be a field and $f(x) \in F[x]$ have degree 2 or 3. Then $f(x)$ is irreducible over $F$ if and only if $f(x)$ has no zeroes.
\end{theorem}
\begin{proof}
    The forward direction follows immediately as it is the contrapositive of \myref{thrm-degree-above-1-reducible-if-has-zero}, so we only work in the reverse direction.

    We consider a contrapositive proof. Assume that $f(x) \in F[x]$ has degree $n \in \{2, 3\}$ and is reducible. Then $f(x) = p(x)q(x)$ where both $p(x)$ and $q(x)$ have degrees less than $n$ (\myref{thrm-irreducible-iff-not-expressable-as-product-of-smaller-polynomials}). Note $n = \deg p(x) + \deg q(x)$, if $n = 2$ then both $p(x)$ and $q(x)$ must have degree 1, and if $n = 3$ then at one of $p(x)$ and $q(x)$ has degree 1. So we know that at least one of $p(x)$ or $q(x)$ has degree 1. Without loss of generality assume $p(x)$ has degree 1; set $p(x) = ax + b$ where $a,b \in F$. Note $a^{-1} \in F$; setting $\alpha = -a^{-1}b$ we see
    \begin{align*}
        f(\alpha) &= (a\alpha+b)q(\alpha)\\
        &= \left(a\left(-a^{-1}b\right) + b\right)q(\alpha)\\
        &= \left(-aa^{-1}b + b\right)q(\alpha)\\
        &= (-b+b)q(\alpha)\\
        &= 0.
    \end{align*}
    Since $\alpha \in F$ thus $\alpha$ is a zero of $f(x)$. Hence $f(x)$ has a zero in $F$.
\end{proof}

\myref{thrm-degree-2-or-3-irreducible-iff-has-no-zeroes} is particularly useful for finite fields, since we just have to check all elements of the field for a zero.

\begin{example}
    Consider the opening example: the polynomial $f(x) = x^2 + 1$. In $\Z_3$, we note that
    \begin{itemize}
        \item $f(0) = 0^2 + 1 = 1 \neq 0$;
        \item $f(1) = 1^2 + 1 = 2 \neq 0$; and
        \item $f(2) = 2^2 + 1 = 5 = 2 \neq 0$,
    \end{itemize}
    so $f(x)$ is not reducible (i.e., irreducible) over $\Z_3$ by \myref{thrm-degree-2-or-3-irreducible-iff-has-no-zeroes}. However, note that $f(2) = 5 = 0$ in $\Z_5$, meaning $f(x)$ has a zero in $\Z_5$. Thus $x^2 + 1$ is reducible over $\Z_5$.
\end{example}

\begin{example}
    Consider the polynomial $f(x) = x^2 - 2$. One sees that $f(3) = 3^2 - 2 = 7 = 0$ in $\Z_7$, so $f(x)$ is reducible in the field $\Z_7$. However $f(x)$ is not reducible in $\Q$ since it has no zeroes in $\Q$. But $f(x)$ is reducible in $\R$ since $f(\sqrt2) = 0$.
\end{example}

\begin{example}
    The theorem fails for polynomials with degree of at least 4. For example, $f(x) = x^4 + 2x^2 + 1$ is reducible in $\Q$ since $x^4 + 2x^2 + 1 = (x^2+1)^2$, but $f(x)$ has no zeroes in $\Q$.
\end{example}

\begin{exercise}
    Let the polynomial $f(x) = 2x^3 + 4x + 9$. For which field(s) listed below is $f(x)$ irreducible?
    \begin{multicols}{4}
        \begin{partquestions}{\alph*}
            \item $\Z_2$
            \item $\Z_3$
            \item $\Z_5$
            \item $\Z_7$
        \end{partquestions} 
    \end{multicols}
\end{exercise}

\subsection{(Ir)reducibility In $\Q$}
Before we introduce the next test, which involves polynomials with integer coefficients, we first introduce some terminology and a lemma to ease the proof of the test.

\begin{definition}
    The \textbf{content}\index{polynomial!content} of a non-zero polynomial $f(x) \in \Z[x]$ is the GCD of its coefficients.
\end{definition}

\begin{definition}
    A \textbf{primitive polynomial}\index{polynomial!primitive} is a non-zero polynomial $f(x) \in \Z[x]$ with content 1.
\end{definition}

\begin{lemma}[Gauss]\label{lemma-gauss}\index{Gauss's Lemma}
    The product of two primitive polynomials is primitive.
\end{lemma}
\begin{proof}[Proof (see {\cite[p.~291]{gallian_2016}})]
    Seeking a contradiction, suppose $f(x)$ and $g(x)$ are primitive polynomials with a non-primitive product $h(x) = f(x)g(x)$. Therefore the content of $h(x)$ is not 1; in fact it is at least 2, meaning the content of $h(x)$ is either prime or composite.
    
    Let $p$ be a prime divisor of the content of $h(x)$, and let $\bar{f}(x)$, $\bar{g}(x)$, and $\bar{h}(x)$ be polynomials obtained from $f(x)$, $g(x)$, and $h(x)$ by reducing their coefficients modulo $p$. Since the content of $h(x)$ is not 1, thus coefficients of $h(x)$ are multiples of $p$; accordingly when they are reduced modulo $p$, we see $\bar{h}(x) = 0$. Also, we may interpret $\bar{f}(x)$ and $\bar{g}(x)$ as polynomials belonging in $\Z_p[x]$ where $(\bar{f}(x))(\bar{g}(x)) = \bar{h}(x) = 0$. Since $\Z_p[x]$ is an integral domain (\myref{thrm-integral-domain-iff-polynomial-ring-is-integral-domain}), this means that $\bar{f}(x) = 0$ or $\bar{g}(x) = 0$. Without loss of generality assume $\bar{f}(x) = 0$. Therefore $p$ divides every coefficient of $f(x)$, meaning that the content of $f(x)$ is at least $p$, i.e. $f(x)$ is not primitive, a contradiction.

    The contradiction means that a product of two primitive polynomials must be primitive.
\end{proof}

\begin{theorem}\label{thrm-irreducible-over-Z-means-irreducible-over-Q}
    Let $f(x) \in \Z[x]$.
    \begin{itemize}
        \item If $f(x)$ is irreducible in $\Z$ then it is irreducible in $\Q$.
        \item If $f(x)$ is a primitive polynomial and is irreducible in $\Q$ then it is irreducible in $\Z$.
    \end{itemize}
    Furthermore, if $f(x)$ is reducible in $\Z$ with $f(x) = p(x)q(x)$ for some non-unit polynomials $p(x)$ and $q(x)$ in $\Z[x]$, then neither $p(x)$ nor $q(x)$ are constant polynomials.
\end{theorem}
\begin{proof}
    We prove the converse of the first statement. Suppose $f(x) \in \Z[x]$ is reducible over $\Q$, meaning that $f(x) = p(x)q(x)$ where $p(x), q(x) \in \Q[x]$. Without loss of generality, assume the content of $f(x)$ is 1, since otherwise we can just divide $p(x)$ and $q(x)$ by the content of $f(x)$ to obtain the same result. Thus $f(x)$ is primitive.

    Suppose $a$ is the LCM of the denominators of the coefficients of $p(x)$, and likewise $b$ is the LCM of the denominators of the coefficients of $q(x)$. Then
    \[
        ab\times f(x) = (a\times p(x))(b\times q(x)),
    \]
    noting that $a\times p(x)$ and $b\times q(x)$ are now polynomials with integer coefficients. Let $C_p$ and $C_q$ denote the contents of $a \times p(x)$ and $b \times q(x)$ respectively; write $a\times p(x) = C_p\times P(x)$ and $b \times q(x) = C_q\times Q(x)$ where one sees clearly that $P(x)$ and $Q(x)$ are polynomials with content 1 (i.e., primitive polynomials). One also sees that
    \[
        ab \times f(x) = C_pC_q \times P(x)Q(x).
    \]
    As $f(x)$ is primitive, thus the content of $ab \times f(x)$ is $ab$. The product of primitive polynomials is primitive (\myref{lemma-gauss}), so $P(x)Q(x)$ is also primitive; the content of $C_pC_q \times P(x)Q(x)$ is therefore $C_pC_q$. So $ab = C_pC_q$.
    
    Dividing both sides by $ab = C_pC_q$ leaves $f(x) = P(x)Q(x)$, where both $P(x)$ and $Q(x)$ are polynomials with integer coefficients. Note that
    \begin{align*}
        \deg p(x) &= \deg (a\times p(x)) & (\text{multiplying by constant does not change degree})\\
        &= \deg(C_p\times P(x)) & (\text{as } a\times p(x) = C_p\times P(x))\\
        &= \deg P(x)
    \end{align*}
    and likewise $\deg q(x) = \deg Q(x)$. Since $f(x) = p(x)q(x)$ and $\Q$ is a field, thus we see that $\deg f(x) = \deg p(x) + \deg q(x)$ (\myref{thrm-polynomial-degree-properties}), meaning $\deg p(x) = \deg P(x) < \deg f(x)$, and likewise $\deg Q(x) < \deg f(x)$, So we have expressed $f(x)$, a polynomial with integer coefficients, as a product of two polynomials $P(x)$ and $Q(x)$ with integer coefficients and of smaller degree than $f(x)$, showing that $f(x)$ is reducible in $\Z$.

    Before we prove the second statement, we note that the final assertion made in the theorem is readily proven by noticing that $P(x)$ and $Q(x)$ are non-constant polynomials of degree smaller than $f(x)$.

    We now prove the converse of the second statement. If $f(x)$ is reducible in $\Z$, then $f(x) = p(x)q(x)$ where $p(x)$ and $q(x)$ are non-unit polynomials. We note that $p(x)$ and $q(x)$ cannot be constant polynomials, since otherwise that would mean that $f(x)$ has a content that is not 1. Thus $p(x)$ and $q(x)$ are non-constant polynomials of degree smaller than $f(x)$. As all integers are also rational numbers, these same polynomials appear in $\Q[x]$ and so $f(x)$ is reducible in $\Q$.
\end{proof}

\begin{example}
    Consider the polynomial $f(x) = 6x^2 + 5x - 4$. One sees that $f(x)$ is reducible in $\Q$ since $6x^2 + 5x - 4 = \left(2x + \frac83\right)\left(3x - \frac32\right)$. Set $p(x) = 2x + \frac83$ and $q(x) = 3x - \frac32$. In this case, and using the notation in \myref{thrm-irreducible-over-Z-means-irreducible-over-Q}, we see $a = 3$, $b = 2$, $C_p = 2$, $C_q = 3$. So $P(x) = 3x + 4$ and $Q(x) = 2x - 1$, which means
    \[
        (3\times2)\times (6x^2 + 5x - 4) = (2\times3)(3x+4)(2x-1),
    \]
    so $6x^2 + 5x - 4 = (3x+4)(2x-1)$, i.e. $f(x)$ is reducible over $\Z$.
\end{example}

\begin{example}
    Consider the polynomial $f(x) = 144x^4 + 168x^3 + 73x^2 + 14x + 1$ and note $f(x) = \left(16x^2 + \frac{32}3x + \frac{16}9\right)\left(9x^2 + \frac92x + \frac9{16}\right)$. Set $p(x) = 16x^2 + \frac{32}3x + \frac{16}9$ and $q(x) = 9x^2 + \frac92x + \frac9{16}$ and using the notation in \myref{thrm-irreducible-over-Z-means-irreducible-over-Q} we see $a = 9$, $b = 16$, $C_p = 16$, and $C_q = 9$. So $P(x) = 9x^2+6x+1$ and $Q(x) = 16x^2 + 8x + 9$, which means
    \[
        (9 \times 16)\left(144x^4 + 168x^3 + 73x^2 + 14x + 1\right) = (16 \times 9)(9x^2+6x+1)(16x^2 + 8x + 9)
    \]
    so $144x^4 + 168x^3 + 73x^2 + 14x + 1 = (9x^2+6x+1)(16x^2 + 8x + 9)$, i.e. $f(x)$ is reducible over $\Z$.
\end{example}

\begin{exercise}
    Let $f(x) \in \Z[x]$. Prove or disprove the following statements.
    \begin{partquestions}{\alph*}
        \item If $f(x)$ is irreducible in $\Z$ then it is irreducible in $\Q$.
        \item If $f(x)$ is irreducible in $\Q$ then it is irreducible in $\Z$.
    \end{partquestions}
\end{exercise}

\subsection{Mod $p$ Irreducibility Test}
We covered some reducibility tests in the previous section; each of their contrapositives can be considered as irreducibility tests, as listed below:
\begin{itemize}
    \item For a field $F$ and a degree 2 or 3 polynomial $f(x) \in F[x]$, $f(x)$ is irreducible if and only if $f(x)$ has no zeroes. (\myref{thrm-degree-2-or-3-irreducible-iff-has-no-zeroes})
    \item If $f(x) \in \Z[x]$ is irreducible in $\Z$ then it is irreducible in $\Q$. (\myref{thrm-irreducible-over-Z-means-irreducible-over-Q})
\end{itemize}
We look at more irreducibility tests, starting with one that is related to \myref{thrm-irreducible-over-Z-means-irreducible-over-Q}.

\begin{theorem}[Mod $p$ Irreducibility Test]\label{thrm-mod-p-irreducibility-test}\index{Mod $p$ Irreducibility Test}
    Let $p$ be a prime and suppose $f(x) \in \Z[x]$ is a non-constant polynomial. Let $\bar{f}(x) \in \Z_p[x]$ be the polynomial obtained after reducing all coefficients of $f(x)$ modulo $p$. If $\bar{f}(x)$ is irreducible in $\Z_p$ and $\deg \bar{f}(x) = \deg f(x)$ then $f(x)$ is irreducible over $\Q$.
\end{theorem}
\begin{proof}
    Seeking a contrapositive proof, suppose $f(x)$ is reducible over $\Q$. Then by \myref{thrm-irreducible-over-Z-means-irreducible-over-Q} there exist non-constant polynomials $p(x), q(x) \in \Z[x]$, both with degree smaller than that of $f(x)$, such that $f(x) = p(x)q(x)$. Let $\bar{f}(x), \bar{p}(x), \bar{q}(x) \in \Z_p[x]$ be the polynomials after reducing the coefficients of $f(x)$, $p(x)$, and $q(x)$ modulo $p$ respectively. Since $\deg \bar{f}(x) = \deg f(x)$ by assumption, we have
    \begin{align*}
        \deg \bar{p}(x) &\leq \deg p(x) < \deg f(x),\\
        \deg \bar{q}(x) &\leq \deg q(x) < \deg f(x),
    \end{align*}
    and $\bar{f}(x) = \bar{p}(x)\bar{q}(x)$. Thus $\bar{f}(x)$ is reducible in $\Z_p$.
\end{proof}

\begin{example}
    Consider the polynomial $f(x) = 7x^3 + x^2 + 2x + 5$. Reducing coefficients modulo 2 yields $\bar{f}(x) = x^3 + x^2 + 1$. Note that
    \begin{itemize}
        \item $\bar{f}(0) = 0^3 + 0^2 + 1 = 1 \neq 0$; and
        \item $\bar{f}(1) = 1^3 + 1^2 + 1 = 3 = 1 \neq 0$,
    \end{itemize}
    so $\bar{f}(x)$ has no zeroes in $\Z_2$. Thus $\bar{f}(x)$ is irreducible in $\Z_2$ (\myref{thrm-degree-2-or-3-irreducible-iff-has-no-zeroes}) and so it is irreducible in $\Q$ by Mod 2 Irreducibility Test (\myref{thrm-mod-p-irreducibility-test}).
\end{example}
\begin{example}
    Consider now $f(x) = 2x^2 + 3x + 4$.
    
    We note that reducing $f(x)$ modulo 2 yields $3x$, but we cannot use \myref{thrm-mod-p-irreducibility-test} on it since $3x$ is of smaller degree than $f(x)$.

    If instead we reduce $f(x)$ modulo 3, we get the polynomial $\bar{f}(x) = 2x^2 + 1$. But one sees clearly that $\bar{f}(1) = 2(1)^2 + 1 = 3 = 0$ in $\Z_3$, so $\bar{f}(x)$ is reducible modulo 3 (\myref{thrm-degree-2-or-3-irreducible-iff-has-no-zeroes}).

    Let's try reducing $f(x)$ modulo 5, yielding a polynomial with the same coefficients in $\Z_5[x]$. Observe that, when evaluating in $\Z_5$, we have
    \begin{itemize}
        \item $f(0) = 2(0)^2 + 3(0) + 4 = 4 \neq 0$;
        \item $f(1) = 2(1)^2 + 3(1) + 4 = 9 = 4 \neq 0$;
        \item $f(2) = 2(2)^2 + 3(2) + 4 = 18 = 3 \neq 0$;
        \item $f(3) = 2(3)^2 + 3(3) + 4 = 31 = 1 \neq 0$; and
        \item $f(4) = 2(4)^2 + 3(4) + 4 = 48 = 3 \neq 0$,
    \end{itemize}
    so $f(x)$ is irreducible modulo 5 (\myref{thrm-degree-2-or-3-irreducible-iff-has-no-zeroes}). Therefore $f(x)$ is irreducible in $\Q$ by Mod 5 Irreducibility Test.
\end{example}

It is important to know that not all values of $p$ in the Mod $p$ Irreducibility Test (\myref{thrm-mod-p-irreducibility-test}) will result in a polynomial that is irreducible in $\Z_p$, as seen in the above example. As long as \textit{one} prime $p$ makes $f(x)$ irreducible in $\Z_p$, then the Mod $p$ Irreducibility Test holds and we may conclude that $f(x)$ is irreducible in $\Q$.

\begin{exercise}
    Prove that, for all prime numbers $p$, there exists a non-negative integer $n < p$ such that $n^2 + 2n + 1 \equiv 0 \pmod{p}$.
\end{exercise}

The Mod $p$ Irreducibility Test can also be used to check for irreducibility of polynomials of degree greater than 3, though more thought needs to be executed in order to come up with a rigorous proof.

\begin{example}
    We will show that $f(x) = \frac15x^4 + \frac8{15}x^3 + \frac45x^2 + \frac13x + 3$ is irreducible in $\Q$. First, let $F(x) = 15f(x) = 3x^4 + 8x^3 + 12x^2 + 5x + 45$, which is a polynomial with integer coefficients. Reducing coefficients of $F(x)$ modulo 2 yields $\bar{F}(x) = x^4 + x + 1$. One sees that
    \begin{itemize}
        \item $\bar{F}(0) = 0^4 + 0 + 1 = 1 \neq 0$; and
        \item $\bar{F}(1) = 1^4 + 1 + 1 = 3 = 1 \neq 0$
    \end{itemize}
    so $\bar{F}(x)$ has no zeroes in $\Z_2$. Now we \textit{cannot} conclude that $F(x)$ is irreducible at this juncture as $\bar{F}(x)$ is a degree 4 polynomial, and \myref{thrm-degree-2-or-3-irreducible-iff-has-no-zeroes} only applies to degree 2 or 3 polynomials. We need to be more creative.

    Seeking a contradiction, suppose that $\bar{F}(x) = p(x)q(x)$ where $p(x), q(x) \in \Z_2[x]$ are polynomials with lower degree than $\bar{F}(x)$. Note neither $p(x)$ nor $q(x)$ can be linear, as this would imply that $\bar{F}(x)$ has a zero in $\Z_2$ (which it does not). So both $p(x)$ and $q(x)$ are quadratic factors. There are 4 quadratic factors in $\Z_2[x]$:
    \begin{enumerate}
        \item $x^2$;
        \item $x^2 + 1$;
        \item $x^2 + x$; and
        \item $x^2 + x + 1$,
    \end{enumerate}
    and clearly options 1, 2, and 3 are not possible factors of $\bar{F}(x)$ since they clearly have zeroes in $\Z_2$. That leaves $x^2 + x + 1$ being the only possible quadratic factor; since $p(x)$ and $q(x)$ are both quadratic this means $p(x) = q(x) = x^2 + x + 1$. But one sees that
    \begin{align*}
        (x^2 + x + 1)^2 &= (x^4 + x^3 + x^2) + (x^3 + x^2 + x) + (x^2 + x + 1)\\
        &= x^4 + 2x^3 + 3x^2 + 2x + 1\\
        &= x^4 + x^2 + 1
    \end{align*}
    which is not $\bar{F}(x)$, a contradiction. Thus $\bar{F}(x)$ is irreducible in $\Z_2$.

    Since $\bar{F}(x)$ is irreducible, thus $F(x)$ is also irreducible by Mod 2 Irreducibility Test (\myref{thrm-mod-p-irreducibility-test}). By contrapositive of \myref{thrm-irreducible-over-Z-means-irreducible-over-Q} we know $F(x)$ is irreducible over $\Q$, thus $f(x)$ is irreducible over $\Q$.
\end{example}

\subsection{Eisenstein's Criterion}
We now look at another irreducibility test, given by Gotthold Eisenstein in 1850.

\begin{theorem}[Eisenstein's Criterion]\label{thrm-eisenstein-criterion}\index{Eisenstein's Criterion}
    Let the polynomial
    \[
        f(x) = a_0 + a_1x + \cdots + a_{n-1}x^{n-1} + a_nx^n
    \]
    be in $\Z[x]$. If there is a prime number $p$ such that
    \begin{itemize}
        \item $p \nmid a_n$,
        \item $p \vert a_i$ for all $0 \leq i < n$, and
        \item $p^2 \nmid a_0$,
    \end{itemize}
    then $f(x)$ is irreducible over $\Q$.
\end{theorem}
\begin{proof}
    Seeking a contradiction, suppose that $f(x)$ is reducible over $\Q$, so $f(x)$ is also reducible over $\Z$. By contrapositive of \myref{thrm-irreducible-over-Z-means-irreducible-over-Q} this means there exist non-constant polynomials $g(x),h(x) \in \Z[x]$ with degree smaller than that of $f(x)$ such that $f(x) = g(x)h(x)$.

    Write
    \begin{align*}
        g(x) &= b_0 + b_1x + \cdots + b_rx^r\\
        h(x) &= c_0 + c_1x + \cdots + c_sx^s
    \end{align*}
    where $0 < r, s < n$. Note $a_0 = b_0c_0$. Since $p \vert a_0$ but $p^2 \nmid a_0$ thus $p$ divides exactly one of $b_0$ and $c_0$. Without loss of generality assume $p \vert b_0$ and $p \nmid c_0$. Note also that $a_n = b_rc_s$ and $p \nmid a_n$. Thus $p \nmid b_r$. So there must exist a smallest integer $t$, where $0 < t \leq r$, such that $p \nmid b_t$.

    Consider the coefficient of the degree $t$ term in $f(x)$, which is
    \[
        a_t = \sum_{i=0}^tb_ic_{t-i} = b_0c_t + b_1c_{t-1} + \cdots + b_tc_0.
    \]
    By assumption, $p \vert a_t$ since $0 < t \leq r < n$; also $p \vert b_i$ for all $0 \leq i < t$, so every term except for the last is divisible by $p$. This means that
    \[
        p \vert (a_t - (b_0c_t + b_1c_{t-1} + \cdots + b_{t-1}c_1)).
    \]
    However $a_t - (b_0c_t + b_1c_{t-1} + \cdots + b_{t-1}c_1) = b_tc_0$, and neither $b_t$ nor $c_0$ is divisible by $p$, a contradiction.

    Hence $f(x)$ is irreducible over $\Q$.
\end{proof}

\begin{example}
    The polynomial $f(x) = 3x^4 + 15x^2 + 10$ is irreducible over $\Q$ by Eisenstein's Criterion (\myref{thrm-eisenstein-criterion}) since the prime 5 divides 10 and 15 but does not divide 3, and also that $5^2 = 25$ does not divide 10.
\end{example}

\begin{exercise}
    Find a polynomial $f(x) \in \Z[x]$ of degree $n \geq 1$ that is irreducible in $n$.
\end{exercise}

\section{Irreducibility via Transformation}
Sometimes, the tests may not be applied directly. In these cases, appropriate substitutions or transformations are needed to convert polynomials into a form that is suitable for testing its (ir)reducibility.

\begin{theorem}[Transformation Rule]\label{thrm-transformation-rule-for-irreducibility}
    Let $D[x]$ be an integral domain and $f(x) \in D[x]$ be a non-zero, non-unit polynomial. Suppose $\phi: D[x] \to D[x]$ is an automorphism. Then $f(x)$ is irreducible if and only if $\phi(f(x))$ is irreducible.
\end{theorem}
\begin{proof}
    We prove the reverse direction first, i.e. we are proving that if $\phi(f(x))$ is irreducible then $f(x)$ is irreducible, via contrapositive. Let $f(x) \in D[x]$ be reducible, meaning there exist non-unit $p(x), q(x) \in D[x]$ such that $f(x) = p(x)q(x)$. Then
    \[
        \phi(f(x)) = \phi(p(x)q(x)) = \phi(p(x))\phi(q(x)).
    \]
    Note that, for a polynomial $u(x) \in D[x]$,
    \begin{align*}
        &u(x) \text{ is a unit}\\
        \iff&u(x)v(x) = 1 \text{ for some }v(x) \in D[x] & (\text{Definition of unit})\\
        \iff&\phi(u(x))\phi(v(x))  = \phi(u(x)v(x)) = \phi(1) = 1 & (\text{Ring homomorphism properties})\\
        \iff&\phi(u(x)) \text{ is a unit.} & (\text{Definition of unit})
    \end{align*}
    Note that the second if and only if statement holds because $\phi$ is an automorphism (which is an isomorphism). So we know that neither $\phi(p(x))$ nor $\phi(q(x))$ are units. Therefore $\phi(f(x))$ is also reducible.

    We now show the forward direction, again via contrapositive. Suppose $\phi(f(x))$ is reducible. Note $\phi^{-1}$ is also a ring isomorphism (\myref{problem-properties-of-ring-isomorphism}) so it is also a ring automorphism, so use the reverse direction's result to conclude that $f(x)$ is also reducible.
\end{proof}

With this general result proven, we come up with a few rules for substitution.

\begin{corollary}\label{coro-irreducible-iff-translation-is-irreducible}
    Let $D$ be an integral domain, $k \in D$, and let $f(x) \in D[x]$. Then $f(x)$ is irreducible if and only if $f(x + k)$ is irreducible.
\end{corollary}
\begin{proof}
    We consider the map $\phi: D[x] \to D[x]$ where
    \[
        \sum_{i=0}^n a_ix^i \mapsto \sum_{i=0}^na_i(x+k)^i.  
    \]
    We show that this is a ring automorphism. For brevity let
    \begin{align*}
        f(x) &= a_0 + a_1x + \cdots + a_mx^m\\
        g(x) &= b_0 + b_1x + \cdots + b_nx^n
    \end{align*}
    be polynomials in $D[x]$, where we assume, without loss of generality, that $m \geq n$, and set $b_i = 0$ for $i > n$.
    \begin{itemize}
        \item \textbf{Homomorphism}: One sees that
        \begin{align*}
            \phi(f(x) + g(x)) &= \phi\left(\sum_{i=0}^m(a_i+b_i)x^i\right)\\
            &= \sum_{i=0}^m(a_i+b_i)(x+k)^i\\
            &= \left(\sum_{i=0}^ma_i(x+k)^i\right) + \left(\sum_{i=0}^mb_i(x+k)^i\right)\\
            &= \left(\sum_{i=0}^ma_i(x+k)^i\right) + \left(\sum_{i=0}^nb_i(x+k)^i\right) & (\because b_i = 0 \text{ for } i > n)\\
            &= \phi(f(x)) + \phi(g(x))
        \end{align*}
        and
        \begin{align*}
            \phi(f(x)g(x)) &= \phi\left(\sum_{r=0}^{m+n}\left(\sum_{i=0}^ra_ib_{r-i}\right)x^r\right)\\
            &= \sum_{r=0}^{m+n}\left(\sum_{i=0}^ra_ib_{r-i}\right)(x+k)^r\\
            &= \left(\sum_{r=0}^ma_r(x+k)^r\right)\left(\sum_{r=0}^nb_r(x+k)^r\right)\\
            &= \phi(f(x))\phi(g(x))
        \end{align*}
        so $\phi$ is a ring homomorphism.
        
        \item \textbf{Injective}: Suppose $\phi(f(x)) = \phi(g(x))$. Then
        \[
            \sum_{i=0}^ma_i(x+k)^i = \sum_{i=0}^nb_i(x+k)^i
        \]
        which, by comparing coefficients, we see $a_i = b_i$ for all $0 \leq i \leq m$. Therefore $f(x) = g(x)$ originally, meaning $\phi$ is injective.
        
        \item \textbf{Surjective}: Let $p(x) = c_0 + c_1x + \cdots + c_rx^r$ be a polynomial in $D[x]$. Note that $q(x) = c_0 + c_1(x-k) + \cdots + c_r(x-k)^r$ is also a polynomial in $D[x]$; observe
        \begin{align*}
            \phi(q(x)) &= c_0 + c_1((x-k)+k) + \cdots + c_r((x-k)+k)^r\\
            &= c_0 + c_1x + \cdots + c_rx^r\\
            &= p(x)
        \end{align*}
        so any $p(x) \in D[x]$ has a pre-image under $\phi$.
    \end{itemize}
    Therefore $\phi$ is an isomorphism (and so it is an automorphism), thus $f(x)$ is irreducible if and only if $f(x+k)$ is irreducible by \myref{thrm-transformation-rule-for-irreducibility}.
\end{proof}

\begin{corollary}\label{coro-irreducible-iff-constant-factor-multiple-is-irreducible}
    Let $D$ be an integral domain, $k \in D$ be a unit, and let $f(x) \in D[x]$. Then $f(x)$ is irreducible if and only if $f(kx)$ is irreducible.
\end{corollary}
\begin{proof}
    Consider the map $\phi: D[x] \to D[x]$ where
    \[
        \sum_{i=0}^n a_ix^i \mapsto \sum_{i=0}^na_i(kx)^i.  
    \]
    \myref{exercise-substitution-by-constant-factor-multiple-map} (later) shows that this is a ring isomorphism (and so is an automorphism), so this means that $f(x)$ is irreducible if and only if $f(kx)$ is irreducible.
\end{proof}

\begin{exercise}\label{exercise-substitution-by-constant-factor-multiple-map}
    Show that the map $\phi$ given in \myref{coro-irreducible-iff-constant-factor-multiple-is-irreducible} is indeed a ring automorphism.
\end{exercise}

\begin{example}
    Since $x^2 + 1$ is irreducible in $\Z$, we know that $(x+2)^2 + 1 = x^2 + 4x + 5$ is also irreducible in $\Z$.
\end{example}

\begin{example}
    We note that $x^2 + x + 2$ is irreducible in $\Z_3$ since
    \begin{align*}
        x^2 + x + 2 &= x^2 + 4x + 5\\
        &= (x^2 + 4x + 4) + 1\\
        &= (x+2)^2 + 1
    \end{align*}
    and $x^2 + 1$ is irreducible in $\Z_3$.
\end{example}

\begin{example}
    We note that $x^3 + 1$ has no zeroes in $\Q$ so $x^3 + 1$ is irreducible in $\Q$. So we also know that
    \begin{itemize}
        \item $(x+1)^3 + 1 = x^3 + 3x^2 + 3x + 2$;
        \item $(\frac12x)^3 + 1 = \frac18x^3 + 1$; and
        \item $(\frac12x + 3)^3 + 1 = \frac18x^3 + \frac94x^2 + \frac{27}2x + 28$
    \end{itemize}
    are all irreducible polynomials in $\Q$.
\end{example}

\begin{example}
    We can show that $f(x) = x^2 + x + 2$ is irreducible in $\Q$. Since the coefficient of $x$, 1, is not divisible by any prime, we cannot directly apply Eisenstein's Criterion (\myref{thrm-eisenstein-criterion}). But using the substitution $x + 3$, we see $f(x+3) = (x+3)^2 + (x+3) + 2 = x^2 + 7x + 14$ which is irreducible by Eisenstein's Criterion using the prime 7. Therefore $f(x)$ is irreducible.
\end{example}

\begin{exercise}
    Let $f(x) = x^4 + 3$.
    \begin{partquestions}{\roman*}
        \item Show that $f(x)$ is irreducible in $\Q$.
        \item Explain why $x^4 + 4x^3 + 6x^2 + 4x + 4$ has no integer zeroes.
        \item Show that $x^4 - 8x^3 + 24x^2 - 32x + 19$ is irreducible in $\Q$.
    \end{partquestions}
\end{exercise}

\section{Uses of Irreducible Polynomials}
%TODO: Add

\section{Unique Factorization in $\Z[x]$}
%TODO: Add

\newpage

\section{Problems}
%TODO: Add
