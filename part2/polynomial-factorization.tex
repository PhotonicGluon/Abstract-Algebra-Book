\chapter{Factorization of Polynomials}
One of the underappreciated facts about working with polynomials with real coefficients, like in high school, is that they can be factored simply. The idea of an `irreducible' polynomial is simple to understand within the reals -- it simply cannot be `broken down' into `smaller' polynomials without `breaking the rules'. We explore how polynomials are factored in a more general setting -- polynomial rings.

\section{(Ir)reducible Polynomials}
We formally define what (ir)reducible polynomials are in a polynomial ring that is an integral domain.
\begin{definition}
    Let $D$ be an integral domain and $f(x) \in D[x]$ be a non-zero, non-unit polynomial. Then $f(x)$ is \textbf{irreducible over $D$}\index{polynomial!irreducible} if, whenever we write $f(x) = p(x)q(x)$ for some polynomials $p(x)$ and $q(x)$ in $D[x]$, then either $p(x)$ is a unit or $q(x)$ is a unit in $D[x]$.

    A non-zero, non-unit polynomial that is not irreducible over $D$ is said to be \textbf{reducible over $D$}\index{polynomial!reducible}.
\end{definition}

\begin{example}
    The polynomial $x + 1$ is irreducible over $\Z$ since we are unable to express $x+1$ as a product of non-unit polynomials in $\Z[x]$.
\end{example}

\begin{example}
    The polynomial $2x + 2$ is reducible over $\Z$ since $2x+2 = 2(x+1)$ and both 2 and $x+1$ are not units in $\Z[x]$.
\end{example}

It is often more useful to talk about irreducible polynomials over a field. In this case, we have this useful theorem to simplify the task of determining the irreducibility of a polynomial.
\begin{theorem}\label{thrm-irreducible-iff-not-expressable-as-product-of-smaller-polynomials}
    Let $F$ be a field. A non-constant polynomial $f(x) \in F[x]$ with degree $n$ is irreducible if and only if $f(x)$ cannot be expressed as a product of two polynomials each of degree lower than $n$.
\end{theorem}
\begin{proof}
    For the forward direction, assume that a non-constant polynomial $f(x)$ with degree $n$ is irreducible. So there does not exist two non-unit polynomials $p(x)$ and $q(x)$ such that $f(x) = p(x)q(x)$, which thus means that it cannot be expressed as two polynomials of degree lower than $n$.

    For the reverse direction, assume that $f(x)$ with degree $n$ cannot be expressed as a product of two polynomials of degree lower than $n$. This means that it is possible for $f(x) = p(x)q(x)$ where at least one of the polynomials $p(x)$ and $q(x)$ have degree $n$. Without loss of generality assume $p(x)$ has degree $n$, thereby meaning $q(x)$ has degree $n - n = 0$ (see \myref{thrm-polynomial-degree-properties}). Thus $q(x)$ is a constant polynomial, so $q(x) = u \in F$ (\myref{prop-constant-polynomial-iff-ring-element}). Note $u \neq 0$ since otherwise $f(x) = 0$, a constant polynomial (a contradiction). But every non-zero element in $F$ is a unit. Therefore the only factorization of $f(x)$ is as a product of a unit and a polynomial of equal degree. Thus $f(x)$ is irreducible.
\end{proof}

\begin{example}
    One sees that $2x^2 + 4$ is reducible over $\Z$ since $2x^2 + 4 = 2(x^2 + 2)$ and neither 2 nor $x^2 + 2$ are units in $\Z[x]$. However, in $\Q$, the given polynomial is irreducible by \myref{thrm-irreducible-iff-not-expressable-as-product-of-smaller-polynomials}.
\end{example}

\begin{example}
    The polynomial $2x^2 + 4$ is irreducible in $\Q$ and $\R$, but reducible in $\C$ since $2x^2 + 4 = (2x-4i)(x+2i)$.
\end{example}

\begin{exercise}
    Let the polynomial $f(x) = 3x^2 - 6$. Which of the following integral domains is $f(x)$ irreducible?
    \begin{partquestions}{\alph*}
        \item $\Z$
        \item $\Q$
        \item $\R$
    \end{partquestions} 
\end{exercise}

\section{Reducibility Tests}
Trying to see if a polynomial is (ir)reducible by definition is not a trivial task. For example, it is hard to see that $x^2 + 1$ is irreducible over $\Z_3$ but reducible over $\Z_5$ by just using the definition directly. Therefore we develop some tests for (ir)reducibility to more easily see which polynomials are reducible or irreducible.

The first test is a consequence of the factor theorem.
\begin{theorem}\label{thrm-degree-above-1-reducible-if-has-zero}
    Let $F$ be a field and $f(x) \in F[x]$ have degree above 1. If $f(x)$ has a zero then $f(x)$ is reducible.
\end{theorem}
\begin{proof}
    Since $f(x)$ has a zero, say $\alpha \in F$, we may write $f(x) = (x-\alpha)q(x)$ (\myref{corollary-factor-theorem}). Note $\deg f(x) = \deg(x-\alpha) + \deg q(x) = 1 + \deg q(x)$ (\myref{thrm-polynomial-degree-properties}), and clearly $1 \leq \deg q(x) < \deg f(x)$ (since degree of $f(x)$ is above 1), so $f(x)$ is expressible as a product of two polynomials of lower degree in $F[x]$. Thus $f(x)$ is reducible in $F$. 
\end{proof}

The next test provides a partial converse to the above theorem for polynomials of degree 2 or 3.
\begin{theorem}\label{thrm-degree-2-or-3-reducible-iff-has-zero}
    Let $F$ be a field and $f(x) \in F[x]$ have degree 2 or 3. Then $f(x)$ is reducible over $F$ if and only if $f(x)$ has a zero.
\end{theorem}
\begin{proof}
    The reverse direction follows immediately from \myref{thrm-degree-above-1-reducible-if-has-zero} so we only work in the forward direction.

    Assume that $f(x) \in F[x]$ has degree $n \in \{2, 3\}$ and is reducible. Then $f(x) = p(x)q(x)$ where both $p(x)$ and $q(x)$ have degrees less than $n$ (\myref{thrm-irreducible-iff-not-expressable-as-product-of-smaller-polynomials}). Note $n = \deg p(x) + \deg q(x)$, if $n = 2$ then both $p(x)$ and $q(x)$ must have degree 1, and if $n = 3$ then at one of $p(x)$ and $q(x)$ has degree 1. So we know that at least one of $p(x)$ or $q(x)$ has degree 1. Without loss of generality assume $p(x)$ has degree 1; set $p(x) = ax + b$ where $a,b \in F$. Note $a^{-1} \in F$; setting $\alpha = -a^{-1}b$ we see
    \begin{align*}
        f(\alpha) &= (a\alpha+b)q(\alpha)\\
        &= \left(a\left(-a^{-1}b\right) + b\right)q(\alpha)\\
        &= \left(-aa^{-1}b + b\right)q(\alpha)\\
        &= (-b+b)q(\alpha)\\
        &= 0.
    \end{align*}
    Since $\alpha \in F$ thus $\alpha$ is a zero of $f(x)$. Hence $f(x)$ has a zero in $F$.
\end{proof}

\myref{thrm-degree-2-or-3-reducible-iff-has-zero} is particularly useful for finite fields, since we just have to check all elements of the field for a zero.

\begin{example}
    Consider the opening example: the polynomial $f(x) = x^2 + 1$. In $\Z_3$, we note that
    \begin{align*}
        f(0) &= 0^2 + 1 = 1 \neq 0,\\
        f(1) &= 1^2 + 1 = 2 \neq 0, \text{ and}\\
        f(2) &= 2^2 + 1 = 5 = 2 \neq 0,
    \end{align*}
    so $f(x)$ is not reducible (i.e., irreducible) over $\Z_3$ by \myref{thrm-degree-2-or-3-reducible-iff-has-zero}. However, note that $f(2) = 5 = 0$ in $\Z_5$, meaning $f(x)$ has a zero in $\Z_5$. Thus $x^2 + 1$ is reducible over $\Z_5$.
\end{example}

\begin{example}
    Consider the polynomial $f(x) = x^2 - 2$. One sees that $f(3) = 3^2 - 2 = 7 = 0$ in $\Z_7$, so $f(x)$ is reducible in the field $\Z_7$. However $f(x)$ is not reducible in $\Q$ since it has no zeroes in $\Q$. But $f(x)$ is reducible in $\R$ since $f(\sqrt2) = 0$.
\end{example}

\begin{example}
    The theorem fails for polynomials with degree of at least 4. For example, $f(x) = x^4 + 2x^2 + 1$ is reducible in $\Q$ since $x^4 + 2x^2 + 1 = (x^2+1)^2$, but $f(x)$ has no zeroes in $\Q$.
\end{example}

\begin{exercise}
    Let the polynomial $f(x) = 2x^3 + 4x + 9$. For which field(s) listed below is $f(x)$ irreducible?
    \begin{partquestions}{\alph*}
        \item $\Z_2$
        \item $\Z_3$
        \item $\Z_5$
        \item $\Z_7$
    \end{partquestions}
\end{exercise}

Before we introduce the next test for reducibility, which involves polynomials with integer coefficients, we first introduce some terminology and a result to ease the proof of the test.

\begin{definition}
    The \textbf{content}\index{polynomial!content} of a non-zero polynomial $f(x) \in \Z[x]$ is the GCD of its coefficients.
\end{definition}

\begin{definition}
    A \textbf{primitive polynomial}\index{polynomial!primitive} is a non-zero polynomial $f(x) \in \Z[x]$ with content 1.
\end{definition}

\begin{lemma}[Gauss]\label{lemma-gauss}\index{Gauss's Lemma}
    The product of two primitive polynomials is primitive.
\end{lemma}
\begin{proof}[Proof (see {\cite[p.~291]{gallian_2016}})]
    Seeking a contradiction, suppose $f(x)$ and $g(x)$ are primitive polynomials with a non-primitive product $h(x) = f(x)g(x)$. Therefore the content of $h(x)$ is not 1; in fact it is at least 2, meaning the content of $h(x)$ is either prime or composite.
    
    Let $p$ be a prime divisor of the content of $h(x)$, and let $\bar{f}(x)$, $\bar{g}(x)$, and $\bar{h}(x)$ be polynomials obtained from $f(x)$, $g(x)$, and $h(x)$ by reducing their coefficients modulo $p$. Since the content of $h(x)$ is not 1, thus coefficients of $h(x)$ are multiples of $p$; accordingly when they are reduced modulo $p$, we see $\bar{h}(x) = 0$. Also, we may interpret $\bar{f}(x)$ and $\bar{g}(x)$ as polynomials belonging in $\Z_p[x]$ where $(\bar{f}(x))(\bar{g}(x)) = \bar{h}(x) = 0$. Since $\Z_p[x]$ is an integral domain (\myref{thrm-integral-domain-iff-polynomial-ring-is-integral-domain}), this means that $\bar{f}(x) = 0$ or $\bar{g}(x) = 0$. Without loss of generality assume $\bar{f}(x) = 0$. Therefore $p$ divides every coefficient of $f(x)$, meaning that the content of $f(x)$ is at least $p$, i.e. $f(x)$ is not primitive, a contradiction.

    The contradiction means that a product of two primitive polynomials must be primitive.
\end{proof}

\begin{theorem}\label{thrm-reducible-over-Q-means-reducible-over-Z}
    If $f(x) \in \Z[x]$ is reducible over $\Q$ then it is reducible over $\Z$.
\end{theorem}
\begin{proof}
    Suppose $f(x) \in \Z[x]$ is reducible over $\Q$, meaning that $f(x) = p(x)q(x)$ where $p(x), q(x) \in \Q[x]$. Without loss of generality, assume the content of $f(x)$ is 1, since otherwise we can just divide $p(x)$ and $q(x)$ by the content of $f(x)$ to obtain the same result. Thus $f(x)$ is primitive.

    Suppose $a$ is the LCM of the denominators of the coefficients of $p(x)$, and likewise $b$ is the LCM of the denominators of the coefficients of $q(x)$. Then
    \[
        ab\times f(x) = (a\times p(x))(b\times q(x)),
    \]
    noting that $a\times p(x)$ and $b\times q(x)$ are now polynomials with integer coefficients. Let $C_p$ and $C_q$ denote the contents of $a \times p(x)$ and $b \times q(x)$ respectively; write $a\times p(x) = C_p\times P(x)$ and $b \times q(x) = C_q\times Q(x)$ where one sees clearly that $P(x)$ and $Q(x)$ are polynomials with content 1 (i.e., primitive polynomials). One also sees that
    \[
        ab \times f(x) = C_pC_q \times P(x)Q(x).
    \]
    As $f(x)$ is primitive, thus the content of $ab \times f(x)$ is $ab$. The product of primitive polynomials is primitive (\myref{lemma-gauss}), so $P(x)Q(x)$ is also primitive; the content of $C_pC_q \times P(x)Q(x)$ is therefore $C_pC_q$. So $ab = C_pC_q$.
    
    Dividing both sides by $ab = C_pC_q$ leaves $f(x) = P(x)Q(x)$, where both $P(x)$ and $Q(x)$ are polynomials with integer coefficients. Note that
    \begin{align*}
        \deg p(x) &= \deg (a\times p(x)) & (\text{multiplying by constant does not change degree})\\
        &= \deg(C_p\times P(x)) & (\text{as } a\times p(x) = C_p\times P(x))\\
        &= \deg P(x)
    \end{align*}
    and likewise $\deg q(x) = \deg Q(x)$. Since $f(x) = p(x)q(x)$ and $\Q$ is a field, thus we see that $\deg f(x) = \deg p(x) + \deg q(x)$ (\myref{thrm-polynomial-degree-properties}), meaning $\deg p(x) = \deg P(x) < \deg f(x)$, and likewise $\deg Q(x) < \deg f(x)$, So we have expressed $f(x)$, a polynomial with integer coefficients, as a product of two polynomials $P(x)$ and $Q(x)$ with integer coefficients and of smaller degree than $f(x)$, showing that $f(x)$ is reducible in $\Z$.
\end{proof}

\begin{example}
    Consider the polynomial $f(x) = 6x^2 + 5x - 4$. One sees that $f(x)$ is reducible in $\Q$ since $6x^2 + 5x - 4 = \left(2x + \frac83\right)\left(3x - \frac32\right)$. Set $p(x) = 2x + \frac83$ and $q(x) = 3x - \frac32$. In this case, and using the notation in \myref{thrm-reducible-over-Q-means-reducible-over-Z}, we see $a = 3$, $b = 2$, $C_p = 2$, $C_q = 3$. So $P(x) = 3x + 4$ and $Q(x) = 2x - 1$, which means
    \[
        (3\times2)\times (6x^2 + 5x - 4) = (2\times3)(3x+4)(2x-1),
    \]
    so $6x^2 + 5x - 4 = (3x+4)(2x-1)$, i.e. $f(x)$ is reducible over $\Z$.
\end{example}

\begin{example}
    Consider the polynomial $f(x) = 144x^4 + 168x^3 + 73x^2 + 14x + 1$ and note $f(x) = \left(16x^2 + \frac{32}3x + \frac{16}9\right)\left(9x^2 + \frac92x + \frac9{16}\right)$. Set $p(x) = 16x^2 + \frac{32}3x + \frac{16}9$ and $q(x) = 9x^2 + \frac92x + \frac9{16}$ and using the notation in \myref{thrm-reducible-over-Q-means-reducible-over-Z} we see $a = 9$, $b = 16$, $C_p = 16$, and $C_q = 9$. So $P(x) = 9x^2+6x+1$ and $Q(x) = 16x^2 + 8x + 9$, which means
    \[
        (9 \times 16)\left(144x^4 + 168x^3 + 73x^2 + 14x + 1\right) = (16 \times 9)(9x^2+6x+1)(16x^2 + 8x + 9)
    \]
    so $144x^4 + 168x^3 + 73x^2 + 14x + 1 = (9x^2+6x+1)(16x^2 + 8x + 9)$, i.e. $f(x)$ is reducible over $\Z$.
\end{example}

\begin{exercise}
    Let $f(x) \in \Z[x]$. Prove or disprove the following statements.
    \begin{partquestions}{\alph*}
        \item If $f(x)$ is irreducible in $\Z$ then it is irreducible in $\Q$.
        \item If $f(x)$ is irreducible in $\Q$ then it is irreducible in $\Z$.
    \end{partquestions}
\end{exercise}

\section{Irreducibility Tests}
%TODO: Add

\section{Unique Factorization in $\Z[x]$}
%TODO: Add

\newpage

\section{Problems}
%TODO: Add
