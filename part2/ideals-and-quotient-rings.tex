\chapter{Ideals and Quotient Rings}
Ideals are a special type of subring. They help us define the idea of quotient rings in ring theory, similar to how quotient ring are defined in group theory. Ideals form a fundamental part of ring theory, and we will see them appearing repeatedly in the following chapters.

\section{History Behind Ideals}
The term ``ideal'' originated from Ernst Kummer, who was looking to study factorization of numbers. For instance, we may factor $45$ in two ``different'' ways, $45 = 5 \times 9$ and $45 = 3 \times 15$. However, these numbers have not been `factored enough', as we can reduce the factorization down to the primes, $45 = 3^3 \times 5$. This is a unique factorization. However, over algebraic rings, such as in $\Z[\sqrt{-3}]$, this idea of unique factorization down to the irreducible factors fails, since
\[
    4 = 2 \times 2 = (1+\sqrt{-3})(1-\sqrt{-3})
\]
and all the factors here are irreducible.

Kummer's idea was that these irreducible factors were not `reduced enough'; that there were better, ``ideal'' factors for which the ideal factorization of numbers hold. However, such a construction requires that one prove the existence of such ideals, and then show that they are indeed ideals -- too tedious for practical use.

Richard Dedekind came up with an alternate definition for ``ideals'', defining not the numbers themselves, but the set of numbers that they divide. So instead of talking about the ``ideal number'' 2, we talk about the set of numbers that are divided by 2, namely $\{\dots, -4, -2, 0, 2, 4 \dots\}$, which is what we now call the principal ideal generated by 2. This way, sums and products of ideals become easier to handle, and more results can be created from their use.

The modern formulation of ideals are much weaker than the original ideals proposed by Dedekind; their motivation in modern times stems from the desire to create ``quotient rings'', similar to that of ``quotient groups'' in group theory. Ideals are the ``normal subgroups'' of ring theory. In the coming sections, we look at ideals, before working our way back to Dedekind's original principal ideal definition.

\newpage

\section{What is an Ideal?}
Rings have two operations defined on them (addition and multiplication). Therefore, we need to define which operation for which a coset will apply to.
\begin{definition}
    Given a ring $R$ and a subring $S$ of $R$, the \textbf{coset}\index{coset!for rings} of $S$ in $R$ with \textbf{representative}\index{coset!representative} $r \in R$ is
    \[
        r + S = \{r+s \vert s \in S\}.
    \]
\end{definition}
\begin{remark}
    As $(R,+)$ is an abelian group, thus, as groups, $(S,+)$ is a normal subgroup of $(R,+)$ and so $(R/S,+)$ is well-defined.
\end{remark}

At this point, we only know that $(R/S,+)$ is a subgroup of $(R,+)$. What conditions do we need on $S$ such that $R/S$ is a ring? Well, we need a ``well-defined'' multiplication operation on the cosets, specifically, for any two elements $a$ and $b$ in $R$, we want
\[
    (a+S)(b+S) = ab + S.
\]
We now try and find the condition(s) required for this operation to be well-defined.

Suppose $a+S = c+S$ and $b+S = d+S$ for some other elements $c$ and $d$ in $R$; our goal is to show $ab+S = cd+S$. Now by Coset Equality, statements 1 and 5, we have $a-c \in S$ and $b-d \in S$. Set $a-c = s_1 \in S$ and $b-d = s_2 \in S$, so $a = c+s_1$ and $b = d+s_2$. Hence,
\[
    ab = (c+s_1)(d+s_2) = cd + cs_2 + s_1d + s_1s_2.
\]
Now for $ab + S = cd+S$ we need to have $ab-cd \in S$, again by Coset Equality statements 1 and 5. Therefore, we need to have $ab-cd = cs_2+s_1d+s_1s_2 \in S$. We note that $s_1s_2 \in S$ (as $S$ is a subring), so we just need both $cs_2$ and $s_1d$ to be in $S$ for $R/S$ to be a ring.

We are now ready to present the definition of an ideal.
\begin{definition}
    Let $R$ be a ring and a subset $I$ of $R$. Suppose $(I,+) \leq (R,+)$. Then $I$ is an \textbf{ideal}\index{ideal} (or \textbf{two-sided ideal}\index{ideal!two-sided}) if, for every $r \in R$ and $i \in I$, both $ri$ and $ir$ are in $I$.
\end{definition}
\begin{exercise}\label{exercise-ideal-is-a-subring}
    Let $I$ be an ideal of a ring $R$. Show that $I$ is a subring of $R$.
\end{exercise}
\begin{example}\label{example-nZ-ideal-of-Z}
    We show that the subring
    \[
        n\Z = \{\dots, -2n, -n, 0, n, 2n, \dots\}
    \]
    is an ideal of $\Z$.

    Suppose $m \in \Z$ and $an \in n\Z$.  Note that
    \[
        (m)(an) = (ma)n \in n\Z
    \]
    and
    \[
        (an)(m) = a(nm) = a(mn) = (am)n \in n\Z,
    \]
    with swapping of $m$ and $n$ possible since $\Z$ is a commutative ring. Therefore $n\Z$ is an ideal of $\Z$.
\end{example}

\begin{example}
    The subset $\{0\}$ is an ideal of any ring, called the \textbf{trivial ideal}\index{ideal!trivial} (or \textbf{zero ideal}\index{ideal!zero}). An ideal that is not $\{0\}$ is called a \textbf{non-trivial ideal}\index{ideal!non-trivial} (or \textbf{non-zero ideal}\index{ideal!non-zero}).
\end{example}
\begin{example}
    For any ring $R$, the ring itself is an ideal of $R$. An ideal $I$ that is a proper subset of the ring $R$ is a \textbf{proper ideal}\index{ideal!proper} of $R$.
\end{example}

We are almost ready to define the quotient ring; we just need to define the operations within it.
\begin{definition}[Coset Addition]
    The sum of the cosets $r+I$ and $s+I$ is $(r+s)+I$.
\end{definition}
\begin{definition}[Coset Multiplication]
    The product of $r+I$ and $s+I$ is $rs+I$.
\end{definition}

We can now define the quotient ring.
\begin{definition}
    Given a ring $R$ and an ideal $I$ of $R$, the \textbf{quotient ring}\index{quotient ring} of $R$ by $I$ is
    \[
        R/I = \{r + I \vert r \in R\},
    \]
    under coset addition and multiplication.
\end{definition}
\begin{remark}
    Some authors (e.g. \cite[p.~243]{dummit_foote_2004}) may choose to represent $r + I$ as $\overline{r}$. With this notation, addition and multiplication in the quotient ring becomes $\overline{r}+\overline{s} = \overline{r+s}$ and $\overline{r}\,\overline{s} = \overline{rs}$ respectively.
\end{remark}
\begin{remark}
    Like in group theory, $R/I$ is read as ``$R$ mod $I$''.
\end{remark}

\begin{proposition}
    If $R$ is a ring with ideal $I$, then $R/I$ is indeed a ring under coset addition and multiplication.
\end{proposition}
\begin{proof}
    We prove this using the ring axioms.
    \begin{itemize}
        \item \textbf{Addition-Abelian}: We show that $(R/I,+)$ is an abelian group.
        \begin{itemize}
            \item \textbf{Group}: As $(R,+)$ is an abelian group and $(I,+)$ is a normal subgroup of $(R,+)$, therefore $(R/I,+)$ is a well-defined quotient group.
            \item \textbf{Commutative}: All that remains is to show that coset addition is commutative. Let $r+I, s+I \in R/I$. Then
            \begin{align*}
                (r+I) + (s+I) &= (r+s)+I\\
                &= (s+r) + I & (\text{since } + \text{ is commutative})\\
                &= (s+I) + (r+I)
            \end{align*}
            so coset addition is commutative.
        \end{itemize}
        \item \textbf{Multiplication-Semigroup}: We need to show that $R/I$ is a semigroup under coset multiplication.
        \begin{itemize}
            \item \textbf{Closure}: Let $r+I, s+I \in R/I$. Note that $rs \in R$ by closure of multiplication. Therefore
            \[
                (r+I)(s+I) = rs+I \in R/I
            \]
            which means that $R/I$ is closed under coset multiplication.

            \item \textbf{Associativity}: Let $r+I$, $s+I$ and $t+I$ be in $R/I$. Then
            \begin{align*}
                (r+I)((s+I)(t+I)) &= (r+I)(st+I)\\
                &= (r(st))+I\\
                &= ((rs)t) + I & (\text{since } \times \text{ is associative})\\
                &= ((rs)+I)(t+I)\\
                &= ((r+I)(s+I))(t+I)
            \end{align*}
            which means that coset multiplication is associative.

        \end{itemize}
        \item \textbf{Distributive}: Lastly, we need to show that coset multiplication distributes over coset addition. For the following, let the cosets $r+I, s+I, t+I \in R/I$.
        \begin{itemize}
            \item We first show left distribution.
            \begin{align*}
                (r+I)((s+I)+(t+I)) &= (r+I)((s+t)+I)\\
                &= (r(s+t))+I\\
                &= (rs+rt)+I\\
                &= (rs+I) + (rt+I)\\
                &=(r+I)(s+I) + (r+I)(t+I).
            \end{align*}
            Therefore the left distributive property has been shown.

            \item We now show right distribution.
            \begin{align*}
                ((r+I)+(s+I))(t+I) &= ((r+s)+I)(t+I)\\
                &= ((r+s)t)+I\\
                &= (rt+st)+I\\
                &= (rt+I) + (st+I)\\
                &= (r+I)(t+I) + (s+I)(t+I).
            \end{align*}
            Therefore the right distributive property has been shown.
        \end{itemize}
    \end{itemize}
    Therefore $R/I$ is a ring.
\end{proof}

\begin{example}
    Consider the subring $6\Z$. From \myref{example-nZ-ideal-of-Z} we know that $6\Z$ is an ideal of $\Z$. Thus $\Z/6\Z$ is a quotient ring; it contains
    \[
        \Z/6\Z = \{0 + 6\Z, 1+6\Z, 2+6\Z, 3+6\Z, 4+6\Z, 5+6\Z\}.
    \]

    We explore two possible products in this quotient ring.
    \begin{itemize}
        \item $(3+6\Z)(4+6\Z) = 12+6\Z = 2\times 6 + 6\Z = 0 + 6\Z$. Thus we have found a pair of zero divisors in $\Z/6\Z$, namely $(3+6\Z)$ and $(4+6\Z)$.
        \item $(5+6\Z)^2 = 25+6\Z = (1 + 6\times4) + 6\Z = 1 + 6\Z$.
    \end{itemize}
\end{example}

\begin{exercise}
    Let the sets
    \begin{align*}
        R &= \left\{\begin{pmatrix}x&y\\0&z\end{pmatrix} \vert x,y,z \in \R\right\},\\
        I &= \left\{\begin{pmatrix}x&y\\0&0\end{pmatrix} \vert x,y \in \R\right\}.
    \end{align*}
    It is given that $R$ is a subring of $\Mn{2}{\R}$.
    \begin{partquestions}{\roman*}
        \item Show that $I$ is a subring of $R$.
        \item Show that $I$ is an ideal of $R$.
        \item Simplify
        \[
            \left(\begin{pmatrix}1&2\\0&1\end{pmatrix} + I\right)\left(\begin{pmatrix}1&-2\\0&1\end{pmatrix} + I\right)
        \]
        in $R/I$.
    \end{partquestions}
\end{exercise}

To save us the hassle of testing whether a subset of a ring is an ideal, we introduce the test for ideal\index{test for ideal}.
\begin{theorem}[Test for Ideal]\label{thrm-test-for-ideal}
    Let $R$ be a ring and $I$ be a non-empty subset of $R$. Then $I$ is an ideal of $R$ if and only if
    \begin{enumerate}
        \item $x - y \in I$ for all $x, y \in I$; and 
        \item for all $i \in I$ and $r \in R$ we have $ri, ir \in I$.
    \end{enumerate}
\end{theorem}
\begin{proof}
    The forward direction is trivial to prove; if $I$ is an ideal then $I$ is a subring by \myref{exercise-ideal-is-a-subring}, meaning statement 1 holds. Statement 2 holds because that is the definition of an ideal.

    We now work in the reverse direction by assuming the two statements hold. First and foremost we know that $(I,+) \leq (R,+)$ by virtue of statement 1 and by applying the subgroup test (\myref{thrm-subgroup-test}). Now statement 2 tells us that $ri \in I$ for any $r \in R$, in particular we may choose an $r \in I$ so that $ri \in I$. Therefore $I$ is closed under multiplication, and so $I$ is a subring of $R$. Finally, statement 2 is exactly the condition for $I$ to be an ideal.
\end{proof}
\begin{remark}
    Like with the subgroup test (\myref{thrm-subgroup-test}), we can check whether $I$ is non-empty by seeing if the additive identity 0 is in $I$.
\end{remark}

We end this section by noting two results with regards to ideals.
\begin{proposition}\label{prop-ideal-contains-unit-iff-ideal-is-whole-ring}
    Let $R$ be a ring with identity and $I$ an ideal of $R$. Then $I$ contains a unit if and only if $I = R$.
\end{proposition}
\begin{proof}
    See \myref{exercise-ideal-contains-unit-iff-ideal-is-whole-ring} (later).
\end{proof}

\begin{proposition}\label{prop-ring-is-field-iff-no-proper-ideals}
    Let $R$ be a commutative ring with identity. Then $R$ is a field if and only if $R$ has no proper ideals.
\end{proposition}
\begin{proof}
    See \myref{problem-ring-is-field-iff-no-proper-ideals} (later).
\end{proof}

\begin{exercise}\label{exercise-ideal-contains-unit-iff-ideal-is-whole-ring}
    Let $R$ be a ring with identity 1, and let $I$ be an ideal of $R$.
    \begin{partquestions}{\roman*}
        \item If $1 \in I$, what does this imply about $I$?
        \item Prove $I$ contains a unit if and only if $I = R$.
    \end{partquestions}
\end{exercise}

\section{Ideal Operations}
We first look at the sum of ideals.

\begin{definition}\index{ideal!sum}
    Let $R$ be a ring and let $\ideal{a}$ and $\ideal{b}$ be ideals of $R$. Then the sum of the ideals $\ideal{a}$ and $\ideal{b}$ is
    \[
        \ideal{a} + \ideal{b} = \{a + b \vert a\in\ideal{a},\;b\in\ideal{b}\}.
    \]
\end{definition}

\begin{example}
    Consider the ring $\Z$ and the ideals $I = \{2n \vert n \in \Z\}$ and $J = \{3n \vert n \in \Z\}$. Then their sum is
    \[
        I + J = \{2a + 3b \vert a,b \in \Z\}.
    \]
    which is just $\Z$ by B\'ezout's lemma (\myref{lemma-bezout}).
\end{example}

\begin{proposition}\label{prop-sum-of-ideals-is-ideal}
    Let $R$ be a ring and let $\ideal{a}$ and $\ideal{b}$ be ideals of $R$. Then $\ideal{a} + \ideal{b}$ is an ideal of $R$.
\end{proposition}
\begin{proof}
    We note that since $\ideal{a}$ and $\ideal{b}$ are ideals, they are thus non-empty and so $\ideal{a}+\ideal{b}$ is non-empty.

    Let $a_1, a_2 \in \ideal{a}$ and $b_1, b_2 \in \ideal{b}$. Then we know $a_1 - a_2 \in \ideal{a}$ and $b_1 - b_2 \in \ideal{b}$ by the test for ideal (\myref{thrm-test-for-ideal}). Therefore $(a_1 - a_2) + (b_1 - b_2) = (a_1 + b_1) - (a_2 + b_2) \in \ideal{a} + \ideal{b}$, satisfying the first requirement.

    Now take any $r \in R$ and $a+b \in \ideal{a}+\ideal{b}$.
    \begin{itemize}
        \item $r(a+b) = ra + rb \in \ideal{a}+\ideal{b}$ since $ra \in \ideal{a}$ (because $\ideal{a}$ is an ideal) and $rb \in \ideal{b}$ (because $\ideal{b}$ is an ideal).
        \item $(a+b)r = ar + br \in \ideal{a}+\ideal{b}$ since $ar \in \ideal{a}$ (because $\ideal{a}$ is an ideal) and $br \in \ideal{b}$ (because $\ideal{b}$ is an ideal).
    \end{itemize}
    Therefore $\ideal{a} + \ideal{b}$ is an ideal by the test for ideal.
\end{proof}

\newpage

We now look at the product of ideals.
\begin{definition}\index{ideal!product}
    Let $R$ be a ring and let $\ideal{a}$ and $\ideal{b}$ be ideals of $R$. Then the product of the ideals $\ideal{a}$ and $\ideal{b}$ is
    \[
        \ideal{ab} = \{a_1b_1 + a_2b_2 + \cdots + a_nb_n \vert a_i\in\ideal{a},\;b_i\in\ideal{b}\}.
    \]
\end{definition}

\begin{example}
    Consider the ring $\Z$ and the ideals $I = \{2n \vert n \in \Z\}$ and $J = \{3n \vert n \in \Z\}$. Then their product is
    \begin{align*}
        IJ &= \{(2a_1)(3b_1) + (2a_2)(3b_2) + \cdots + (2a_n)(3b_n) \vert a_i,b_i \in \Z\}\\
        &= \{6(a_1b_1 + a_2b_2 + \cdots + a_nb_n) \vert a_i,b_i \in \Z\}\\
        &= \{6k \vert k \in \Z\}.
    \end{align*}
    which is $6\Z$.
\end{example}

\begin{proposition}\label{prop-product-of-ideals-is-ideal}
    Let $R$ be a ring and let $\ideal{a}$ and $\ideal{b}$ be ideals of $R$. Then $\ideal{ab}$ is an ideal of $R$.
\end{proposition}
\begin{proof}
    Since $\ideal{a}$ and $\ideal{b}$ are non-empty as they are ideals, therefore $\ideal{ab}$ is non-empty.

    Let $a_1,a_2 \in \ideal{a}$ and $b_1,b_2 \in \ideal{b}$. Since $\ideal{a}$ is an ideal and is hence a subring, so $-a_2 \in \ideal{a}$. Thus $a_1b_1, (-a_2)(b_2) \in \ideal{ab}$. Clearly $a_1b_1 + (-a_2)(b_2) = a_1b_1 - a_2b_2 \in \ideal{ab}$ so this satisfies the first requirement in the test for ideal (\myref{thrm-test-for-ideal}).

    Now take any $r \in R$ and let $a_1b_1 + \cdots + a_nb_n \in \ideal{ab}$.
    \begin{itemize}
        \item $r(a_1b_1 + \cdots + a_nb_n) = (ra_1)b_1 + \cdots + (ra_n)b_n$. Note $ra_i \in \ideal{a}$ since $\ideal{a}$ is an ideal, and $b_i \in \ideal{b}$. Hence $(ra_1)b_1 + \cdots + (ra_n)b_n \in \ideal{ab}$.
        \item $(a_1b_1 + \cdots + a_nb_n)r = a_1(b_1r) + \cdots + a_n(b_nr)$. Note $b_ir \in \ideal{b}$ since $\ideal{b}$ is an ideal, and $a_i \in \ideal{a}$. Hence $a_1(b_1r) + \cdots + a_n(b_nr) \in \ideal{ab}$.
    \end{itemize}

    Therefore $\ideal{ab}$ is an ideal by the test for ideal.
\end{proof}

\begin{exercise}
    Let $R$ be a ring and let $\ideal{a}$ and $\ideal{b}$ be ideals of $R$. Prove that $\ideal{a} \cap \ideal{b}$ is an ideal of $R$.
\end{exercise}

\section{Principal Ideals}
We return to the original definition for ideals by Dedekind, which is what we now call principal ideals. They can be thought of as ideals that are `generated' by an element from the ring.
\begin{definition}
    Let $R$ be a commutative ring with identity and let $a \in R$. Then the \textbf{principal ideal generated by $a$}\index{ideal!principal} is
    \[
        \princ{a} = aR = \{ar \vert r \in R\}.
    \]
\end{definition}
\begin{remark}
    Most authors (e.g. \cite[p.~123, Definition III.2.4]{hungerford_1980}, \cite[\S 158]{clark_1984}, \cite[p.~251]{dummit_foote_2004}) choose to denote the principal ideal generated by $a$ by $(a)$. However, to avoid ambiguity with normal parentheses, we follow \cite[p.~250, Example 3]{gallian_2016} and \cite[Example 16.24]{judson_beezer_2022} and choose to denote it by $\princ{a}$ instead. Although one may be concerned with this being confused with the cyclic (sub)group generated by $a$, the meaning of this notation should be clear from context.
\end{remark}
\begin{remark}[see {\cite[p.~251]{dummit_foote_2004}}]
    If $R$ is a non-commutative ring, or if $R$ does not contain an identity, then the situation becomes more complicated. In particular,
    \begin{align*}
        \princ{a} &= \text{Smallest ideal of } R \text{ containing }a\\
        &= \bigcap_{\substack{\text{all ideals }I \\ \text{with } a \in I}}I
    \end{align*}
\end{remark}

We restrict the discussion of principal ideals to commutative rings with identity.

\begin{proposition}
    All principal ideals are ideals.
\end{proposition}
\begin{proof}
    See \myref{exercise-principal-ideal-is-ideal} (later).
\end{proof}

\begin{proposition}\label{prop-principal-ideals-equal-iff-associates}
    Let $D$ be an integral domain, and let $a,b\in D$. Then $\princ{a} = \princ{b}$ if and only if $a = bu$ for some unit $u$ in $D$.
\end{proposition}
\begin{proof}
    See \myref{problem-principal-ideals-equal-iff-associates} (later).
\end{proof}

\begin{example}
    The ideal $n\Z$ is a principal ideal of $\Z$ since
    \[
        n\Z = \{\dots, -2n, -n, 0, n, 2n, \dots\} = \princ{n}.
    \]
\end{example}
\begin{remark}
    In the context of $\Z$, we usually write the principal ideal $\princ{n}$ as $n\Z$.
\end{remark}

\begin{exercise}\label{exercise-principal-ideal-is-ideal}
    Show that all principal ideals are ideals.
\end{exercise}

\begin{exercise}\label{exercise-trivial-ideal-and-whole-ring-are-principal-ideals}
    Let $R$ be a commutative ring with identity. Show that the following ideals are principal in $R$.
    \begin{partquestions}{\alph*}
        \item The trivial ideal.
        \item The ring itself.
    \end{partquestions}
\end{exercise}

We look at two related definitions to the principal ideal before we look at an interesting proposition.
\begin{definition}
    A commutative ring where every ideal is principal is called a \textbf{principal ideal ring}\index{principal ideal ring}.
\end{definition}
\begin{definition}
    A principal ideal ring that is an integral domain is called a \textbf{principal ideal domain}\index{principal ideal domain} (\textbf{PID}\index{PID}).
\end{definition}

We look at one example of a PID.
\begin{proposition}\label{prop-Z-is-PID}
    $\Z$ is a PID.
\end{proposition}
\begin{proof}
    We note that $\Z$ is an integral domain, so all that remains to be proven is that every ideal of $\Z$ is principal.

    Let $I$ be an ideal of $\Z$, and suppose $I$ is not the trivial ideal. This means that $I$ must contain both positive and negative numbers, as otherwise elements may not have an additive inverse.

    Set $n = \min(I \cap \mathbb{N})$, i.e. $n$ is the smallest positive integer that is in $I$. Observe that since $n \in I$ we have $mn \in I$ for all $m \in \Z$. Therefore
    \[
        \princ{n} = \{mn \vert m \in \Z\} = n\Z \subseteq I.
    \]

    Now we want to show that $I \subseteq \princ{n}$. Suppose $a \in I$. By Euclid's division lemma (\myref{lemma-euclid-division}), we see that
    \[
        a = nq + r \text{ with } q,r\in\Z \text{ and } 0 \leq r < n,
    \]
    meaning that $r = a - nq \in I$. Note that $a$ and $n$ are in $I$, so $nq \in I$ and therefore $a - nq = r \in I$. If $r$ is positive, then we have found a smaller positive integer than $n$ in $I$, a contradiction since $n$ is the smallest positive integer in $I$. Hence $r = 0$, which means $a = nq \in \princ{n}$. Therefore $I \subseteq \princ{n} = n\Z$.

    As $n\Z \subseteq I$ and $I \subseteq n\Z$, therefore $I = n\Z$, meaning that any arbitrary ideal in $\Z$ is principal. Hence $\Z$ is a PID.
\end{proof}

\section{Prime and Maximal Ideals}\label{section-prime-and-maximal-ideals}
Let's first look at the definition of a prime ideal.
\begin{definition}
    Let $R$ be a commutative ring with identity. An proper ideal $P$ of $R$ is called a \textbf{prime ideal}\index{ideal!prime}\index{prime!ideal} if whenever $ab \in P$ we have $a \in P$ or $b \in P$.
\end{definition}
This definition may seem weird, but is completely natural when we consider the properties of primes in the positive integers. Recall that Euclid's lemma (\myref{corollary-euclid}) tells us that if a prime $p$ divides $ab$, then either $p$ divides $a$ or $p$ divides $b$. Similarly, if the product $ab$ belongs within the prime ideal $P$, then either $a$ belongs in $P$ or $b$ belongs in $P$. However, prime ideals does not necessarily imply unique `factorization' like what occurs in the integers via the Fundamental Theorem of Arithmetic (\myref{thrm-fundamental-theorem-of-arithmetic}); we will explore the uniqueness criteria in a later chapter.

We look at the connection between this definition of a prime ideal and prime numbers in the integers.
\begin{proposition}\label{prop-ideals-of-Z}
    The prime ideals of $\Z$ are the trivial ideal and $p\Z$, where $p$ is a prime number.
\end{proposition}
\begin{proof}
    First let's assume the ideal in question is the trivial ideal. Suppose $ab \in \{0\}$, meaning $ab = 0$. As $\Z$ is an integral domain, this means either $a = 0$ or $b = 0$, which therefore means either $a \in \{0\}$ or $b \in \{0\}$. Hence the trivial ideal is a prime ideal in $\Z$.

    Now suppose we have a non-trivial prime ideal $P$ of $\Z$. As $\Z$ is a PID, we may write $P = n\Z$ where $n \geq 2$ (note that if $n = 1$ we will get $P = \Z$). Furthermore write $n = ab$ where $1 \leq a,b \leq n$. Since $ab = n \in P$, therefore $a \in P$ or $b \in P$. Without loss of generality assume $a \in P$. We note $P = n\Z = \{\dots, -2n, -n, 0, n, 2n, \dots\}$ and $1 \leq a \leq n$, so we must have $a = n$. Therefore $a = n$ and $b = 1$, which shows that $n$ is prime. Hence, the prime ideal $P = n\Z$, where $n$ is prime.
\end{proof}
\begin{exercise}
    Is the principal ideal $\princ{2} = \{0, 2, 4, 6\}$ prime in $\Z_8$?
\end{exercise}

We now look at the definition of a maximal ideal.
\begin{definition}
    Let $R$ be a commutative ring with identity. A proper ideal $M$ of $R$ is called a \textbf{maximal ideal}\index{ideal!maximal} if whenever an ideal $I$ is such that $M \subseteq I \subseteq R$ we have $I = M$ or $I = R$.
\end{definition}
\begin{remark}
    This means that the only ideal that properly contains a maximal ideal is the entire ring.
\end{remark}
\begin{remark}
    To show maximality of an ideal, we usually assume that $I \neq M$ and show that $I = R$.
\end{remark}

\begin{example}
    We show that $M = \princ{2} = \{0, 2, 4, \dots, 32, 34\}$ is a maximal ideal of $\Z_{36}$. Suppose we have an ideal $I$ of $\Z_{36}$ such that $M \subset I \subseteq \Z_{36}$. We show that $I = \Z_{36}$.

    Take an $n \in I \setminus M$. Then Euclid's division lemma (\myref{lemma-euclid-division}) tells us that we may write
    \[
        n = 2q + r \text{ with } 0 \leq r < 2
    \]
    Now since $n \notin M$, therefore $n$ is not a multiple of 2. Thus there must be a remainder, i.e. $1 \leq r < 2$, meaning that $r = 1$. Furthermore, one sees that $r = 1 = n - 2q$, and because $n \in I$ and $2q \in I$ (since $2q \in M$ and $M \subset I$), therefore $1 \in I$. Hence, by \myref{prop-ideal-contains-unit-iff-ideal-is-whole-ring}, $I = \Z_{36}$, which means that $\princ{2}$ is a maximal ideal of $\Z_{36}$.
\end{example}
\begin{exercise}
    What conditions must be placed on the positive integer $n$ so that $n\Z$ is a maximal ideal in $\Z$?
\end{exercise}

We note that there are much easier ways to determine whether an ideal is prime, maximal, or both. We state two important results here.

\begin{theorem}\label{thrm-prime-ideal-iff-quotient-ring-is-integral-domain}
    Let $R$ be a commutative ring with identity, and let $P$ be an ideal of $R$. Then $P$ is prime if and only if $R/P$ is an integral domain.
\end{theorem}
\begin{proof}
    We first prove the forward direction. Suppose $P$ is prime and $a+P, b+P \in R/P$ such that $(a+P)(b+P) = 0+P$. This means that $ab + P = 0 + P$, so $ab \in P$. Now as $P$ is prime, this thus means that either $a \in P$ or $b \in P$. Hence, $a + P = 0 + P$ or $b + P = 0 + P$. So we have shown that if two elements in $R/P$ multiply to zero, then one of the elements must be zero, meaning that $R/P$ has no zero divisors.

    Now clearly $(a+P)(b+P) = ab + P = ba + P = (b+P)(a+P)$ since $R$ is commutative, so $R/P$ is commutative. Furthermore one sees that $1 + P$ is the identity in $R/P$. Therefore, $R/P$ is an integral domain.

    Now we prove the reverse direction. Suppose $R/P$ is an integral domain. Take $a,b \in R$ such that $ab \in P$. This means that $ab + P = (a+P)(b+P) = 0 + P$. Therefore $a+P = 0 + P$ or $b + P = 0 + P$ since $R/P$ is an integral domain with no zero divisors. Hence $a \in P$ or $b \in P$.

    Now if $P = R$ then $R/P = \{0 + P\}$ which is the trivial quotient ring. But by \myref{exercise-trivial-ring-is-not-an-integral-domain}, the trivial ring is not an integral domain. Therefore $P \neq R$, meaning that $P$ is a prime ideal.
\end{proof}
\begin{exercise}
    Is $\princ{3 - i}$ a prime ideal in the Gaussian integers?
\end{exercise}

\begin{theorem}\label{thrm-maximal-ideal-iff-quotient-ring-is-field}
    Let $R$ be a commutative ring with identity, and let $M$ be an ideal of $R$. Then $M$ is maximal if and only if $R/M$ is a field.
\end{theorem}
\begin{proof}
    We first work in the forward direction. Suppose $M$ is a maximal ideal, and take $a + M \in R/M$ such that $a \neq 0$ and $a \notin M$ (which means $a + M$ is non-zero). Observe
    \[
        M \subset \princ{a} + M \subseteq R,
    \]
    with strict subset achieved because $a \notin M$, and the sum of ideals is an ideal (\myref{prop-sum-of-ideals-is-ideal}). As $M$ is maximal, this means that $\princ{a} + M = R$. Therefore $1 \in \princ{a} + M$; write $1 = ar + m$ for some $r \in R$ and $m \in M$. Note $1-ar = m \in M$, so $ar + M = 1 + M$. As $ar + M = (a+M)(r+M) = 1 + M$, so $(a+M)^{-1} = r+M$, meaning that any non-zero element in $R/M$ has an inverse. A similar argument to what is shown in \myref{thrm-prime-ideal-iff-quotient-ring-is-integral-domain} shows that $R/M$ is commutative with identity $1 + M$, so $R/M$ is a field.

    Now we work in the reverse direction; assume that $R/M$ is a field. Suppose $I$ is an ideal such that $M \subset I \subseteq R$. We want to show that $I = R$.

    Take $a \in I \setminus M$ with $a \neq 0$, so $a + M \in R/M$ is non-zero. As $R/M$ is a field, there exists a $b + M \in R/M$ such that $(a+M)(b+M) = 1 + M$, meaning $ab+M = 1+M$, thus $ab - 1 \in M \subset I$. Since $ab \in I$ and $ab - 1 \in I$, the only way for this to happen is if $1 \in I$. By \myref{prop-ideal-contains-unit-iff-ideal-is-whole-ring} this means $I = R$.

    Finally, if $M = R$, then $R/M = \{0 + M\}$, the trivial ring. But in the trivial ring, the additive and multiplicative inverses is the same element, so $R/M$ is not a field by definition. Hence $M \neq R$, so $M \subset R$, meaning $M$ is maximal.
\end{proof}

\begin{corollary}\label{corollary-all-maximal-ideals-are-prime-ideals}
    All maximal ideals are prime ideals.
\end{corollary}
\begin{proof}
    If $M$ is a maximal ideal in a commutative ring with identity $R$, then $R/M$ is a field by \myref{thrm-maximal-ideal-iff-quotient-ring-is-field}. Note that any field is an integral domain by \myref{prop-field-is-integral-domain}. Therefore $R/M$ is an integral domain, meaning $M$ is prime by \myref{thrm-prime-ideal-iff-quotient-ring-is-integral-domain}.
\end{proof}

\begin{corollary}\label{corollary-prime-ideal-is-maximal-in-finite-commutative-ring-with-identity}
    All prime ideals in a finite commutative ring with identity are maximal.
\end{corollary}
\begin{proof}
    See \myref{exercise-prime-ideal-is-maximal-in-finite-commutative-ring-with-identity} (later).
\end{proof}
\begin{exercise}\label{exercise-prime-ideal-is-maximal-in-finite-commutative-ring-with-identity}
    Let $R$ be a finite commutative ring with identity, and let $P$ be a prime ideal in $R$. Show that $P$ is maximal in $R$.
\end{exercise}

\section{The Annihilator and Radical}
To end this chapter, we look at two constructs relating to a ring.
\begin{definition}
    Let $R$ be a commutative ring and $A$ a non-empty subset of $R$. The \textbf{annihilator}\index{annihilator} of $A$ is
    \[
        \Ann{R}{A} = \{r \in R \vert ra = 0 \text{ for all } a \in A\}.
    \]
\end{definition}
\begin{example}
    $\Ann{\Z}{\Z} = \{0\}$.
\end{example}
\begin{proposition}
    Let $R$ be a commutative ring and $A$ a non-empty subset of $R$. Then $\Ann{R}{A}$ is an ideal of $R$.
\end{proposition}
\begin{proof}
    See \myref{exercise-annihilator-is-an-ideal} (later).
\end{proof}

\begin{definition}
    Let $R$ be a commutative ring and $I$ an ideal of $R$. The \textbf{radical}\index{radical} of $I$ is
    \[
        \sqrt I = \{r \in R \vert r^n \in I \text{ for some } n \in \mathbb{N}\}.
    \]
\end{definition}
\begin{definition}
    The radical of the trivial ideal is called the \textbf{nilradical}\index{nilradical} and is given by
    \[
        \Nilr{R} = \sqrt{\{0\}} = \{r \in R \vert r^n = 0 \text{ for some } n \in \mathbb{N}\}.
    \]
    That is, the nilradical of a commutative ring $R$ is the set of all nilpotents in $R$.
\end{definition}
\begin{example}
    We find the nilradical of the ring $\Z_{12}$.
    \begin{align*}
        \Nilr{\Z_{12}} &= \{r \in \Z_{12} \vert r^n = 0 \text{ for some } n \in \mathbb{N}\}\\
        &= \{0, 6\}.
    \end{align*}
\end{example}
\begin{example}
    In the integers, we show that $\sqrt{4\Z} = 2\Z$.

    Suppose $r \in \sqrt{4\Z}$, so $r^n = 4m$ for some $m \in \Z$ and $n \in \mathbb{N}$. Clearly $r^n = 4m = 2(2m) \in 2\Z$, so $r \in 2\Z$. Thus this means that $\sqrt{4\Z} \subseteq 2\Z$.

    On another hand, suppose $r \in 2\Z$, meaning $r = 2m$ for some $m \in \Z$. Note that $r^2 = (2m)^2 = 4m^2 \in 4\Z$, so $r \in \sqrt{4\Z}$. Thus $2\Z \subseteq \sqrt{4\Z}$.

    Therefore $\sqrt{4\Z} = 2\Z$.
\end{example}

\begin{proposition}
    Let $R$ be a commutative ring and $I$ be an ideal of $R$. Then $\sqrt{I}$ is an ideal of $R$.
\end{proposition}
\begin{proof}
    We again consider the test for ideal (\myref{thrm-test-for-ideal}) to prove this. Note that $0 \in \sqrt{I}$ since $0^1 = 0 \in I$, so $\sqrt{I}$ is non-empty.

    First let $r, s \in \sqrt{I}$, meaning $r^m \in I$ and $s^n \in I$ for some positive integers $m$ and $n$. Without loss of generality, assume $m \geq n$. Consider $(r-s)^{mn}$. The Binomial Theorem (\myref{thrm-binomial}) tells us that
    \[
        (r-s)^{mn} = \sum_{k=0}^{mn}(-1)^k{mn \choose k}r^{mn-k}s^k.
    \]
    Observe that at any one point, either $mn - k \geq m$ or $k \geq m$, so at any point either $r^{mn-k} \in I$ or $s^k \in I$, meaning that $(-1)^k{mn \choose k}r^{mn-k} \in I$. Therefore, $(r-s)^{mn} \in I$, which in turn means $r-s \in \sqrt{I}$.

    Next suppose $x \in R$ and $r \in \sqrt{I}$, meaning $r^n \in I$ for some positive integer $n$. Note that $(rx)^n = (xr)^n = x^nr^n \in I$ since $R$ is commutative and $r^n \in I$. Therefore $rx, xr \in \sqrt{I}$.

    By the test for ideal, we have $\sqrt{I}$ is an ideal of $R$.
\end{proof}

\begin{exercise}\label{exercise-annihilator-is-an-ideal}
    Let $R$ be a commutative ring and $A$ a non-empty subset of $R$. Show that $\Ann{R}{A}$ is an ideal of $R$.
\end{exercise}

\newpage

\section{Problems}
\begin{problem}
    Find $\Ann{\Z_{36}}{\{15\}}$.
\end{problem}

\begin{problem}
    Let $S = \{a + 2bi \vert a, b \in \Z\}$. Show that $S$ is a subring of $\Z[i]$ but not an ideal of $\Z[i]$.
\end{problem}

\begin{problem}
    Consider
    \[
        I = \left\{\begin{pmatrix}2a&2b\\2c&2d\end{pmatrix} \vert a,b,c,d \in \Z\right\}.
    \]
    Show that $I$ is an ideal of $\Mn{2}{\Z}$.
\end{problem}

\begin{problem}\label{problem-ring-is-field-iff-no-proper-ideals}
    Let $R$ be a commutative ring with identity and $I$ be an ideal of $R$.
    \begin{partquestions}{\alph*}
        \item Prove that if $R$ is a field then $I$ is either the trivial ring or $R$ (i.e., $R$ has no proper ideals). Hence prove that any field is a PID.
        \item Prove that if $R$ has no proper ideals, then $R$ is a field.
    \end{partquestions}
\end{problem}

\begin{problem}
    Let $R$ be a commutative ring and let $I$ and $J$ be ideals in $R$. Prove that
    \begin{partquestions}{\alph*}
        \item $\sqrt{\sqrt{I}} = \sqrt{I}$; and
        \item $\sqrt{I \cap J} = \sqrt{I} \cap \sqrt{J}$.
    \end{partquestions}
\end{problem}

\begin{problem}
    Let $m$ and $n$ be positive integers, and let $d = \gcd(m,n)$ and $l = \lcm(m,n)$. Prove the following.
    \begin{partquestions}{\alph*}
        \item $m\Z \cap n\Z = l\Z$
        \item $m\Z + n\Z = d\Z$
    \end{partquestions}
\end{problem}

\begin{problem}
    Let $R$ be a commutative ring. Prove that $R / \Nilr{R}$ has no non-zero nilpotents.
\end{problem}

\begin{problem}\label{problem-non-trivial-prime-ideal-is-maximal-in-PID}
    Prove that every non-trivial prime ideal is a maximal ideal in a PID.
\end{problem}

\begin{problem}\label{problem-principal-ideals-equal-iff-associates}
    Prove \myref{prop-principal-ideals-equal-iff-associates}.
\end{problem}
