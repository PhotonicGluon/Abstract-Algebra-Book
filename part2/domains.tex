\chapter{Domains and Factorization}
Earlier, we introduced domains, before focusing our attention mainly on integral domains. We then explored concepts of factorization, but restricted the discussion to polynomial rings. We explore more types of domains here, and explore the factorization available in such domains.

\section{Factorization in Domains}
Before we define some constructs present in integral domains, we first look at what is means for an element to be a factor/divisor of another element in that domain.

\begin{definition}
    Let $D$ be an integral domain and $a,b \in D$. Then \textbf{$a$ divides $b$}\index{divides!for integral domains} (or that $a$ is a \textbf{factor}\index{factor!for integral domains} of $b$) if there is an element $k \in D$ such that $ak = b$. This is denoted $a \vert b$.
\end{definition}
\begin{example}
    One sees that 2 divides 6 in $\Z$ since $2 \times 3 = 6$. Similarly 3 divides 6 since $3 \times 2 = 6$.
\end{example}
\begin{example}
    In $\Z_7$, we see that 3 divides 5 since $3 \times 4 = 12 = 5$. We also see that 5 divides 3 since $5 \times 2 = 10 = 3$.
\end{example}

\begin{definition}
    Let $D$ be an integral domain, and $a,b \in D$. Then $a$ and $b$ are said to be \textbf{associates}\index{associates} if and only if there exists a unit $u \in D$ such that $a = ub$.
\end{definition}

\begin{example}
    In $\Z$, 2 and -2 are associates since $-2 = (-1)(2)$ and -1 is a unit in $\Z$.
\end{example}

\begin{example}
    In $\Z_7$, 3 and 5 are associates since $5 = 12 = (4)(3)$. In fact, any two non-zero elements in $\Z_7$ are associates of each other, since every non-zero element of $\Z_7$ is a unit.
\end{example}

\begin{example}
    In $\Q$, any two non-zero elements are associates of each other, since every non-zero element in $\Q$ is a unit. As an example, $\frac{12}{35}$ and $\frac{71}{25}$ are associates since $\frac{71}{25} = \left(\frac{35}{12} \times \frac{71}{25}\right)\left(\frac{12}{35}\right)$.
\end{example}

From the previous two examples, we can come up with the following proposition.

\begin{proposition}
    Any two non-zero elements in a field $F$ are associates.
\end{proposition}
\begin{proof}
    Let $a, b \in F$ be non-zero elements. Note $a^{-1}$ exists since $F$ is a field, and $a^{-1}b$ is an element of the field by closure. Since a field is a division ring (by definition), $a^{-1}b$ has an inverse, so it is a unit. Therefore one sees $b = (a^{-1}b)a$ which means $a$ and $b$ are associates.
\end{proof}

Thus, in a sense, it is only meaningful to talk about associates when we restrict or view to that of integral domains that are \textit{not} fields.

\begin{exercise}
    Let $D$ be an integral domain and $a,b \in D$. Prove that $a \vert b$ and $b \vert a$ if and only if $a$ and $b$ are associates.
\end{exercise}

We now look at irreducibles and primes.
\begin{definition}
    Let $D$ be an integral domain and $x \in D$. Then $x$ is an \textbf{irreducible}\index{irreducible!element} if and only if, when $x$ is written as $x = pq$ where $p, q \in D$, then either $p$ is a unit or $q$ is a unit.
\end{definition}
\begin{definition}
    Let $D$ be an integral domain and $p \in D$. Then $p$ is a \textbf{prime}\index{prime} if $p$ is not a unit and, for all $a, b \in D$, we have $p \vert ab$ implies $p \vert a$ or $p \vert b$.
\end{definition}

We note that the above definition of irreducibles is similar to that of irreducible polynomials, and the definition of primes is similarly defined to that of prime ideals.

The principal confusion one may have with the definition of a prime above is that, in the integers, we defined prime numbers to satisfy the definition of an irreducible (\myref{definition-prime-number}), and that primes also satisfy the primality definition above (\myref{corollary-euclid}). The reason for this confusion is that, over the integers, irreducibles are primes. However, in general, they are \textit{not} the same. We will look at such examples in a while.

To explore the differences between irreducibles and primes, we look at quadratic integers, with a more detailed exploration done in the next chapter.
\begin{definition}
    A \textbf{square-free integer}\index{square-free integer} is an integer that is not divisible by any square number other than 1.
\end{definition}

Let $d$ be a square-free integer not equal to 1. We note that the set 
\[
    \Z[\sqrt{d}] = \left\{a + b\sqrt{d} \vert a,b \in \Z\right\}
\]
forms an integral domain. We will leave proving that this is indeed an integral domain to the next section. We also define the norm, the map $N: \Z[\sqrt{d}] \to \Z$ where
\[
    N(a+b\sqrt{d}) = a^2 - db^2.
\]
For now, we note that $N(xy) = N(x)N(y)$ for all $x, y \in \Z[\sqrt{d}]$. We will prove this and other properties of the norm later.

\begin{example}
    We show that it is possible for an irreducible in $D = \Z[\sqrt{-3}]$ to not be prime. Note that $N(a+b\sqrt{-3}) = a^2 + 3b^2$. Let $r = 1 + \sqrt{-3}$.
    \begin{itemize}
        \item We claim that $r$ is irreducible. Seeking a contradiction, suppose $r = xy$ where neither $x$ nor $y$ is a unit. Then $N(r) = N(x)N(y) = N(1+\sqrt{-3}) = 4$. It follows that, since $x$ and $y$ are not units, that $N(x) = N(y) = 2$. But, there are no integers $a$ and $b$ such that $a^2 + 3b^2 = 2$ required for $x$ (and $y$), a contradiction. Therefore $x$ or $y$ is a unit, meaning that $r$ is irreducible.

        \item We claim that $r$ is \textit{not} a prime. Observe that
        \[
            (1+\sqrt{-3})(1-\sqrt{-3}) = 1-(-3) = 4 = 2 \times 2
        \]
        so $1 + \sqrt{-3}$ divides $2 \times 2$. However, for integers $a$ and $b$ to exist so that $2 = (1+\sqrt{-3})(a+b\sqrt{-3}) = (a-3b) + (a+b)\sqrt{-3}$, we must have $a - 3b = 2$ and $a + b = 0$, which has a solution $a = \frac12$ and $b = -\frac12$ which is not an integer. Therefore $r$ is not prime.
    \end{itemize}
    Hence it is possible for an irreducible to not be a prime.
\end{example}

We note one property of irreducibles.
\begin{proposition}\label{prop-associates-of-irreducible-is-irreducible}
    In an integral domain, associates of an irreducible are irreducible.
\end{proposition}
\begin{proof}
    See \myref{exercise-associates-of-irreducible-is-irreducible} (later).
\end{proof}

One might wonder whether it is possible for a prime to not be an irreducible. It turns out that it is not possible.
\begin{theorem}\label{thrm-in-integral-domain-primes-are-irreducibles}
    In an integral domain $D$, all primes are irreducibles.
\end{theorem}
\begin{proof}
    Suppose $p \in D$ is a prime. Suppose $p = xy$ where $x, y\in D$. Then clearly $p = (1)(xy)$ and so $p \vert xy$, meaning $p \vert x$ or $p \vert y$. Without loss of generality, assume $p \vert x$, meaning $x = kp$ for some $k \in D$. Then one sees
    \begin{align*}
        1x &= x\\
        &= kp\\
        &= k(xy)\\
        &= (ky)x & (\text{Integral domain is commutative})
    \end{align*}
    which means $ky = 1$ by cancellation law (\myref{prop-domain-cancellation-law}). Hence $y$ is a unit; $p$ is an irreducible.
\end{proof}

If we are working in a principal ideal domain then we have a stronger result.
\begin{theorem}\label{thrm-in-PID-prime-iff-irreducible}
    Let $D$ be a PID. Then an element is a prime if and only if it is an irreducible.
\end{theorem}
\begin{proof}
    The forward direction is proven by \myref{thrm-in-integral-domain-primes-are-irreducibles}, so we only work in the reverse direction: proving that an element is prime if it is irreducible in $D$.

    Suppose $r$ is an irreducible in $D$ and suppose $r \vert xy$ where $x, y \in D$. Consider the set
    \[
        I = \{\lambda r + \mu x \vert \lambda, \mu \in D\}.
    \]
    We know that $I$ is an ideal since it is the sum of principal ideals $\princ{r}$ and $\princ{x}$ (\myref{prop-sum-of-ideals-is-ideal}). Since $D$ is a PID let $I = \princ{d}$ for some $d \in D$. As $r \in I$, write $r = kd$ for some $k \in D$. Note that because $r$ is irreducible we know that either $k$ is a unit or $d$ is a unit.
    \begin{itemize}
        \item If $k$ is a unit, then $\princ{r} = \princ{kd} = \princ{d} = I$ (\myref{prop-principal-ideals-equal-iff-associates}). Since $x \in I$, there is a $t \in D$ such that $x = rt$, meaning $r \vert x$.
        \item Otherwise, if $d$ is a unit, then $I$ contains a unit and so $I = D$ (\myref{prop-ideal-contains-unit-iff-ideal-is-whole-ring}), meaning $1 \in I$. Write $1 = \lambda r + \mu x$, so $y = \lambda ry + \mu xy$. Since $r \vert \lambda ry$ and $r \vert xy$ (by assumption), thus $r \vert (\lambda ry + \mu xy)$, meaning $r \vert y$.
    \end{itemize}
    Therefore, if $r \vert xy$ then either $r \vert x$ or $r \vert y$, meaning $r$ is prime.
\end{proof}
\begin{exercise}\label{exercise-associates-of-irreducible-is-irreducible}
    Prove \myref{prop-associates-of-irreducible-is-irreducible}.
\end{exercise}

\section{Quadratic Field and Quadratic Integers}
We alluded to quadratic integers in the previous section; we make it formal what they are here.

\begin{definition}
    Let $d$ be a square-free integer not equal to 1. The \textbf{quadratic field}\index{quadratic field} of $d$ is
    \[
        \Q[\sqrt{d}] = \{a + b\sqrt{d} \vert a,b\in\Q\}.
    \]
\end{definition}
\begin{proposition}
    The quadratic field $\Q[\sqrt{d}]$, where $d$ is a square-free integer not equal to 1, is indeed a field.
\end{proposition}
\begin{proof}
    We note that the arithmetic was not specified explicitly; this is because we are adapting the definitions of polynomial addition and multiplication to our current context. In particular,
    \[
        (a+b\sqrt{d}) + (x + y\sqrt{d}) = (a+x) + (b+y)\sqrt{d}
    \]
    and
    \[
        (a+b\sqrt{d})(x+y\sqrt{d}) = (ax+(by)d) + (ay + bx)\sqrt{d}.
    \]

    Note we do not need to prove that $(\Q[\sqrt{d}], +)$ is an abelian group as this is inherited from our adaptation of the polynomial ring $\Q[x]$, substituting $x$ with $\sqrt{d}$. We, however, need to prove that multiplication has identity and inverses, since the rest is also inherited from polynomial multiplication.

    \begin{itemize}
        \item \textbf{Identity}: The multiplicative identity is $1 = 1 + 0\sqrt{d}$, since one sees clearly that
        \[
            (a+b\sqrt{d})(1+0\sqrt{d}) = ((a\times1) + (b\times0)d) + ((a\times0) + (b \times 1))\sqrt{d} = a + b\sqrt{d}
        \]

        \item \textbf{Inverses}: We first note that since $d$ is square-free, therefore $\sqrt{d}$ is not an integer. Thus $a^2 - db^2$ is non-zero. We note that for any element $a+b\sqrt{d} \in \Q[\sqrt{d}]$, its inverse is $\frac{a}{a^2-db^2} - \frac{b}{a^2-db^2}\sqrt{d}$ since
        \begin{align*}
            &(a+b\sqrt{d})\left(\frac{a}{a^2-db^2} - \frac{b}{a^2-db^2}\sqrt{d}\right)\\
            &= \left(a\left(\frac{a}{a^2-db^2}\right) + b\left(-\frac{b}{a^2-db^2}\right)d\right) + \left(a\left(-\frac{b}{a^2-db^2}\right) + b\left(\frac{a}{a^2-db^2}\right)\right)\sqrt{d}\\
            &=\left(\frac{a^2}{a^2-db^2} -\frac{b^2d}{a^2-db^2}\right) + \left(-\frac{ab}{a^2-db^2} + \frac{ab}{a^2-db^2}\right)\sqrt{d}\\
            &=1.
        \end{align*}
    \end{itemize}

    Therefore $\Q[\sqrt{d}]$ is a field.
\end{proof}

As discussed in the previous section, there is a function, called the norm, that is of interest.
\begin{definition}
    The \textbf{norm}\index{quadratic integer!norm} of an element $a+b\sqrt{d} \in \Z[\sqrt{d}]$, where $d$ is a square-free integer not equal to 1, is given by the map $N: \Z[\sqrt{d}] \to \Z$ where
    \[
        N(a+b\sqrt{d}) = a^2-db^2.
    \]
\end{definition}

\begin{proposition}\label{prop-properties-of-quadratic-integer-norm}
    Let $x,y \in \Z[\sqrt{d}]$, where $d$ is a square-free integer not equal to 1. Then
    \begin{enumerate}
        \item $x = 0$ if and only if $N(x) = 0$;
        \item $N(xy) = N(x)N(y)$ for all $x$ and $y$;
        \item $x$ is a unit if and only if $N(x) = \pm1$; and
        \item if $N(x) = \pm p$ where $p$ is a prime number, then $x$ is irreducible.
    \end{enumerate}
\end{proposition}
\begin{proof}
    We prove the statements in order.
    \begin{enumerate}
        \item For the forward direction, note that if $x = 0$ then clearly $N(0) = 0^2 - d(0)^2 = 0$. For the reverse direction suppose $N(x) = a^2 - db^2 = 0$ where $a,b \in \Z$. Then $a^2 = db^2$ which means $a = \pm\sqrt{d}b$. Now since $d$ is a square-free integer, therefore the only way for equality on both sides (with the left hand side being an integer and the right being irrational) is for $a = b = 0$. Therefore $x = 0$.
        
        \item Let $x = p + q\sqrt{d}$ and $y = r + s\sqrt{d}$ where $p,q,r,s \in \Z$. Then
        \begin{align*}
            N(xy) &= N((p+q\sqrt{d})(r+s\sqrt{d}))\\
            &= N((pr + qsd) + (ps + qr)\sqrt{d})\\
            &= (pr+qsd)^2 - d(ps+qr)^2\\
            &= ((pr)^2 + 2(pqrsd) + (qsd)^2) - d((ps)^2 + 2(pqrs) + (qr)^2)\\
            &= p^2r^2 + 2pqrsd + q^2s^2d^2 - p^2s^2d - 2pqrsd - q^2r^2d\\
            &= p^2r^2 + q^2s^2d^2 - p^2s^2d - q^2r^2d\\
            &= p^2r^2 - p^2s^2d - q^2r^2d + q^2s^2d^2\\
            &= (p^2 - dq^2)(r^2 - ds^2)\\
            &= N(p+q\sqrt{d})N(r+s\sqrt{d})\\
            &= N(x)N(y)
        \end{align*}
        which proves this statement.

        \item For the forward direction, if $x$ is a unit then there exists a $x^{-1}$ such that $xx^{-1} = 1$. Note $N(xx^{-1}) = N(x)N(x^{-1}) = N(1) = 1$. Therefore, since we are working in the integers, either both $N(x)$ and $N(x^{-1})$ are 1 or both $N(x)$ and $N(x^{-1})$ are -1.
        
        For the reverse direction, consider $x = p+q\sqrt{d}$ and $\bar{x} = p-q\sqrt{d}$. Note $x\bar{x} = p^2-dq^2 = N(x)$. Therefore if $N(x) = 1$ we see $x\bar{x} = 1$, meaning that $x$ is a unit. Otherwise if $N(x) = -1$ one sees $x(-\bar{x}) = 1$, so again $x$ is a unit.

        \item See \myref{exercise-quadratic-integer-irreducible-if-norm-is-prime} (later).
    \end{enumerate}
\end{proof}

We are now ready to define the ring of quadratic integers.
\begin{definition}
    Let $d$ be a square-free integer not equal to 1. The \textbf{ring of quadratic integers}\index{quadratic integer!ring} is the set
    \[
        \Z[\sqrt{d}] = \{a + b\sqrt{d} \vert a,b\in\Z\}
    \]
    under the same operations of addition and multiplication as $\Q[\sqrt{d}]$.

    An element in $\Z[\sqrt{d}]$ is called a \textbf{quadratic integer}\index{quadratic integer}.
\end{definition}
\begin{proposition}
    For a square-free integer $d \neq 1$, $\Z[\sqrt{d}]$ is a subring of $\Q[\sqrt{d}]$. Furthermore $\Z[\sqrt{d}]$ is an integral domain.
\end{proposition}
\begin{proof}
    % TODO: Add
\end{proof}

\begin{exercise}\label{exercise-quadratic-integer-irreducible-if-norm-is-prime}
    Let $d$ be a square-free integer that is not 1. Prove that $x \in \Z[\sqrt{d}]$ is irreducible if $N(x) = \pm p$, where $p$ is a prime number.
\end{exercise}

\begin{exercise}
    Recall that the Gaussian integers is the set $\Z[i] = \Z[\sqrt{-1}]$.
    \begin{partquestions}{\alph*}
        \item Show that the numbers 2 and 5, which are primes in the positive integers, are no longer primes in the Gaussian integers.
        \item \begin{partquestions}{\roman*}
            \item Show that, for any two integers $a$ and $b$, it is impossible to have $a^2 + b^2 \equiv 3 \pmod4$.
            \item Prove that 3 is irreducible in the Gaussian integers.
            \item Let $x \in \Z[i]$. Show that if $9 \vert N(x)$ then $3 \vert x$.\newline
            (\textit{Hint: you may need to consider all possibilities of squares modulo 9.})
            \item Prove that 3 is a prime in the Gaussian integers, which is called a \textbf{Gaussian prime}\index{Gaussian prime}.
        \end{partquestions}
    \end{partquestions}
    % TODO: Add solution
\end{exercise}

%TODO: Continue

\section{Unique Factorization Domains (UFDs)}
% TODO: Add
% TODO: Add facts about Z[x] being a UFD here

\section{Euclidean Domains}
% TODO: Add

\newpage

\section{Problems}
% TODO: Add
