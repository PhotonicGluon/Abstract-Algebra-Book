%=========== Custom Commands =============
% Roman numerals
\makeatletter
\newcommand*{\rom}[1]{\Ifstr{#1}{0}{0}{\expandafter\@slowromancap\romannumeral #1@}}
\makeatother

% Text for writing the edition
\newcommand*{\editiontext}{
    \editionnumber~Edition~\Ifstr{\buildnumber}{}{}{(Build \buildnumber)}
}

%============= Front Matter ==============
% Half title page
\thispagestyle{empty}
\null\vspace{4cm}
\begin{raggedleft}
    {\fontsize{24pt}{0pt}\selectfont \textbf{A Complete Introduction To Abstract Algebra}}\\
\end{raggedleft}

% Title page
\begin{titlepage}
    \null\vspace{4cm}
    \begin{raggedleft}
        {\fontsize{20pt}{0pt}\selectfont \textbf{A Complete}\\\textbf{Introduction To}}\\

        \vspace{0.4cm}
        {\fontsize{48pt}{0pt}\selectfont \textbf{ABSTRACT}}\\
        \vspace{0.15cm}
        {\fontsize{48pt}{0pt}\selectfont \textbf{ALGEBRA}}\\

        \vspace{0.5cm}
        {\fontsize{16pt}{0pt}\selectfont \editiontext}\\
        
        \vspace{1.25cm}
        {\fontsize{20pt}{0pt}\selectfont Kan Onn Kit}\\
    \end{raggedleft}
    \vspace*{\fill}
\end{titlepage}

\newpage{}

% Edition notice
\clearpage\null\vfill
\thispagestyle{empty}
\begin{minipage}[b]{0.9\textwidth}
    \footnotesize\raggedright
    \setlength{\parskip}{0.5\baselineskip}

    Published by Kan Onn Kit\\
    Singapore
    \vspace{5cm}

    \textbf{A Complete Introduction To Abstract Algebra}\par
    \editiontext
    \vspace{0.3cm}

    Copyright \copyright \ 2022 -- \the\year\ by Kan Onn Kit\par
    This work is licensed under a
    Creative Commons Attribution-NonCommercial-ShareAlike 4.0 International Licence.\par
    \pdfteximg{2.5cm}{images/CC_BY-NC-SA_4.0.pdf_tex}\\
    The full licence text is available at \url{http://creativecommons.org/licenses/by-nc-sa/4.0/}.\par    
    The source files for the project are available at \url{https://github.com/PhotonicGluon/Abstract-Algebra-Book}.
    \vspace{0.3cm}

    Typeset in 10pt \TeX~Gyre Pagella using PDF\LaTeX.
\end{minipage}

\vspace*{2\baselineskip}
\cleardoublepage

% Dedication page
\thispagestyle{empty}
\vspace*{1cm}

\begin{center}
    {\fontsize{18pt}{0}\selectfont \textit{For my friends}}\\
\end{center}

\vspace*{1.5cm}

\begin{center}
    \Large{\parbox{10cm}{
        \begin{raggedright}
            \Large
            Symmetry is a vast subject, significant in art and nature. Mathematics lies at its root, and it would be hard to find a better one on which to demonstrate the working of the mathematical intellect.
            \vspace{0.3cm}
            
            \hfill
            --- Hermann Weyl, 1952\\
            \vspace{-0.25cm}
            
            \hfill
            \normalsize
            ({\cite[p.~145]{weyl_1952}})
        \end{raggedright}
    }
}
\end{center}

\vspace*{\fill}

% "Content for no one... where you hide something so deeply that you don't care if anyone finds it [because] you just like knowing that it's there"
\begin{center}
    \fontsize{8pt}{8pt}\selectfont
    \code{3202 1503 2406 0000 0801 1109 0903 1601 2905}
\end{center}

\newpage

% Table of contents
\createtoc

% Acknowledgements
\chapter{Acknowledgements}
Undertaking such a monumental project is new to me, and I am indebted to the people who accompanied me on this journey.

I am eternally grateful to my parents, who have spent countless hours and an ungodly amount of effort to raise me into who I am today. Their omnipresent kindness, patience, and love for me are something I certainly do not deserve, and I thank them for taking care of me.

I would like to thank my tutor Leong Chong Ming, who got me interested in abstract algebra in the first place. His enthusiasm and eagerness to share his knowledge on the subject is the driving force behind my decision to write this book.

I am grateful for the help of my friend Low Ji Yuan, who has assisted me with countless revisions of the content in this book and given me another pair of eyes in the vetting of content. Without her, many errors would slip by unnoticed in this book.

My utmost appreciation, gratitude, and thanks go to Joshua Foo Yong Yi, who helped proofread countless drafts of this book since its inception. He has also personally sat through lessons about abstract algebra with me using this book, which allowed me to fix countless errors that I would not have otherwise spotted. I cannot thank him enough.

I also sincerely appreciate the continued support from my mathematics tutors, Loke Weng Heng, Siow Yun Jie, and Teng Yen Ping, who have been there through my junior college years inspiring me with the wonders of mathematics. I am indebted to them for allowing me to excel in my final examinations.

My close friends, Aidan Tay, Gabriel Fong, and Low Ji Yuan, have accompanied me through two years of schooling (and maths jokes). I offer infinite thanks to them for sticking with me and for encouraging this maths nerd to pursue his wacky projects.

A thousand thanks go out to my teachers at the School of Science and Technology, Singapore, and specifically my form teacher Lee Tsi Yew Samuel, who instilled important character values into me so I can excel in my future endeavours.

And, last but not least, thank you for picking up this book. As one abstract algebra book out of countless others, thank you for choosing this one.

% Preface
\chapter{Preface}
Although algebra has a long history, it has undergone quite striking changes in the past few decades. Abstract (or modern) algebra is widely recognised as an essential element of higher mathematical education. However, the results that it showcases are often hard to grasp and understand without prerequisite knowledge or a heavy background in mathematics. Most books on this subject are crafted for undergraduates at universities. They are not for a general mathematics enthusiast or one who seeks to understand more about the inner structure of algebra that mathematicians encounter frequently.

The exploration of such structures is fundamental to the current underpinning of scientific inquiries. For example, groups are important as they describe the symmetries which the laws of physics seem to obey. Finite fields are also used in coding theory and combinatorics. I hope this book will inspire more people to learn more about abstract algebra, beyond the simple introduction presented here.

In addition, I find that, in most textbooks, important details are left for the reader to figure out independently, without providing any additional guidance or help along the way. Exercises and problems are provided for topics taught in abstract algebra, but only a handful have written solutions provided, leaving readers unsure of the correctness of their answers. I believe that the completeness of a textbook is essential; no claim made should be without justification (unless necessary). This book offers a more complete picture of abstract algebra by providing full-worked solutions to all exercises and problems posed.

This book serves to achieve several goals. First, I hope this book adequately provides a step-by-step explanation of the core results from abstract algebra, without assuming any sophisticated prior knowledge such as number theory or modular arithmetic. I hope that this book can serve as a self-contained reference to anyone looking to pick up abstract algebra out of interest, and I hope that anyone who wishes to does not have to look up any information online to supplement missing details.

In addition, I hope this book can demystify the core steps that many textbooks gloss over when providing important results or when writing solutions to problems and exercises. In those textbooks, the reader is expected to fill in the missing details, and possible errors that could be committed when attempting to complete solutions may go unnoticed without guidance. This book, by providing written solutions to the results put forth, would hopefully allay any issues with determining the correctness of one's solution to exercises and problems posed.

Finally, I hope that this book can make the results of this wondrous field as accessible, approachable, and understandable as possible. The field of abstract algebra, although not immediately apparent, underpins most discussions of mathematics today, and one cannot escape from its influence in its entirety. Therefore, a thorough understanding of at least the basics of such a fundamental subject needs to be properly conveyed, lest we risk this beautiful field being resigned to exclusivity by those who can understand it.

\newpage

In closing, I recall two famous quotes from two mathematicians. The first by Godfrey Harold Hardy:
\quoteattr[0.1cm]{
    \normalshape
    We have concluded that the trivial mathematics is, on the whole, useful, and that the real mathematics, on the whole, is not.
    }
    {G.H. Hardy}
    {\cite[p.~43]{hardy_snow_1969}}

The second by Paul Lockheart:
\quoteattr[0.1cm]{
    \normalshape
    Mathematics is \textit{the music of reason}. To do mathematics is to engage in an act of discovery and conjecture, intuition and inspiration; to be in a state of confusion -- not because it makes no sense to you, but because you gave it sense and you still don't understand what your creation is up to; to have a breakthrough idea; to be frustrated as an artist; to be awed and overwhelmed by an almost painful beauty; to be \textit{alive}, damn it.
    }
    {Paul Lockheart}
    {\cite[p.~8]{lockheart_2002}, emphasis his}

I hope that this book can accomplish these goals and let readers enjoy the wonders of abstract algebra.
\hfill{\textit{21 October, 2023}}

% Suggestions on the use of this book
\chapter{Suggestions on the Use of This Book}
\section*{General Information}
\begin{itemize}
    \item For most parts, we include both exercises and problems.
    \begin{itemize}
        \item An exercise can be thought of as a simple ``self-review'' question. Exercises ensure that the content of a particular section is understood, and is relatively simple to answer.
        \item A problem is a more holistic version of an exercise. Generally, solutions to problems require a thorough understanding of the current chapter and may require results from other chapters.        
    \end{itemize}
    Full solutions to both exercises and problems are included. Note that these are just suggested answers; other approaches may be valid.

    \item Results and questions have differing numbering systems.
    \begin{itemize}
        \item All definitions, axioms, examples, lemmas, theorems, propositions, and corollaries are consecutively numbered using the format
        \begin{quote}
            \code{[CHAPTER].[SECTION].[NUMBER]}
        \end{quote}
        For example, the fourth statement in chapter 2, section 3 is labelled \textbf{2.3.4}.
        \item Exercises and problems are also numbered consecutively using the format
        \begin{quote}
            \code{[CHAPTER].[NUMBER]}
        \end{quote}
        For example, the third exercise in chapter 2 is labelled \textbf{2.3}. Likewise, the fourth exercise in chapter 3 is labelled \textbf{3.4}.
    \end{itemize}
    \item For more technical proofs, we include a sketch of the proof along with the full proof. The symbol ``$\qedsketch$'' marks the end of the sketch of a proof, while ``$\qedproof$'' marks the end of a proof.
\end{itemize}

\section*{Chapter Interdependence}
The diagram on the next page shows chapter interdependence. It should be used in conjunction with the table of contents and notes listed.

\newpage
\begin{center}
    \pdfteximg{\linewidth}{images/interdependence.pdf_tex}
\end{center}

\newpage

\textbf{Notes}:
\begin{itemize}
    \item Each part is largely independent from the other parts. However, part 0 is required reading for future parts of the book.
    \item Part I on groups can be thought of as the fundamentals of abstract algebra.
    \begin{itemize}
        \item Chapter 7 is essentially independent from the rest of the other chapters. It provides motivation for the axioms of groups, but readers who want to skip this introduction can move straight to chapter 9.
        \item Chapters 8, 9, and 10 are considered to be the essentials of group theory.
        \item Chapter 10 is required reading for chapters 11, 12, 13, and 15.
        \item Chapter 13 only requires knowledge of the subgroup product from chapter 12 (specifically \myref{definition-subgroup-product} and \myref{prop-subgroup-product-is-subgroup}); otherwise these two chapters are relatively independent.
        \item Chapter 14 assumes knowledge of chapter 13, and knowledge of permutations and the symmetric group from chapter 11.
        \item Usually, group actions (chapter 15) would be read after the essentials of group theory; therefore chapter 15 could be read after chapter 10.
        \item Chapter 16 only requires chapters 13 and 15.
        \item Chapter 17 only require results from chapter 13, except for \myref{problem-S4-composition-series} which uses the alternating group introduced in chapter 14.
        \item Chapter 18 assumes full knowledge of chapter 13; minor results from chapter 14 (specifically, the alternating group), chapter 16 (the Third Sylow Theorem, \myref{thrm-sylow-3}), and chapter 17 (\myref{problem-S4-composition-series}) are required.
    \end{itemize}
    \item Part II on rings builds on ideas from part I. Unlike part I, the flow of content for rings is very linear.
    \begin{itemize}
        \item Before starting on chapter 19, on the introduction to rings, it is recommended to read chapters 8, 9, 10, 13, and 14 from group theory. This way, a firm foundation of the terminology used in chapter 19 onwards would be built. However, not all content from chapter 14 is relevant for ring theory. The critical sections are \myref{section-more-about-cyclic-groups} and \myref{section-groups-of-matrices}. Additionally, \myref{section-group-of-units-mod-n} would only be used for one problem (\myref{problem-failure-case-of-mod-p-irreducibility-test}) in chapter 25.
        \item Chapter 19 motivates the definition of rings, and provides the axioms for rings.
        \item Chapter 20 builds on chapter 19 by defining more terms and properties found within rings.
        \item Chapter 21 focuses on a specific type of rings; readers should be familiar with the terminology and results from chapter 20.
        \item Chapter 22 could be read after chapter 20; only \myref{section-prime-and-maximal-ideals} on prime and maximal ideals and \myref{problem-non-trivial-prime-ideal-is-maximal-in-PID} requires knowledge from chapter 21.
        \item Chapter 23 requires chapter 22, but mostly does not require chapter 21; only \myref{problem-integral-domain-iff-trivial-ideal-is-prime} requires knowledge of chapter 21 and knowledge of \myref{section-prime-and-maximal-ideals}, which requires chapter 21 as well.
        \item Chapter 24 assumes knowledge of chapter 23, and requires the whole of chapter 21.
        \item Chapter 25 requires chapter 24. As mentioned before, \myref{problem-failure-case-of-mod-p-irreducibility-test} would require knowledge of \myref{section-group-of-units-mod-n}.
        \item Chapter 26 assumes familiarity with all content in ring theory (chapters 19 to 25), since many connections that are established within chapter 26 will only be evident after reading previous chapters.
        \item Chapter 27 can be read after chapter 24, since it only requires familiarity with polynomial rings. The content of chapter 27 is relatively independent from the rest of ring theory, so readers may also choose to skip it.
    \end{itemize}
\end{itemize}
