%=========== Custom Commands =============
% Roman numerals
\makeatletter
\newcommand*{\rom}[1]{\Ifstr{#1}{0}{0}{\expandafter\@slowromancap\romannumeral #1@}}
\makeatother

% Text for writing the edition
\newcommand*{\editiontext}{
    \editionnumber~Edition~\Ifstr{\buildnumber}{}{}{(Build \buildnumber)}
}

%============= Front Matter ==============
% Half title page
\thispagestyle{empty}
\null\vspace{4cm}
\begin{raggedleft}
    {\fontsize{24pt}{0pt}\selectfont \textbf{A Complete Introduction To Abstract Algebra}}\\
\end{raggedleft}

% Title page
\begin{titlepage}
    \null\vspace{4cm}
    \begin{raggedleft}
        {\fontsize{20pt}{0pt}\selectfont \textbf{A Complete}\\\textbf{Introduction To}}\\

        \vspace{0.4cm}
        {\fontsize{48pt}{0pt}\selectfont \textbf{ABSTRACT}}\\
        \vspace{0.15cm}
        {\fontsize{48pt}{0pt}\selectfont \textbf{ALGEBRA}}\\

        \vspace{0.5cm}
        {\fontsize{16pt}{0pt}\selectfont \editiontext}\\
        
        \vspace{1.25cm}
        {\fontsize{20pt}{0pt}\selectfont Kan Onn Kit}\\
    \end{raggedleft}
    \vspace*{\fill}
\end{titlepage}

\newpage{}

% Edition notice
\clearpage\null\vfill
\thispagestyle{empty}
\begin{minipage}[b]{0.9\textwidth}
    \footnotesize\raggedright
    \setlength{\parskip}{0.5\baselineskip}

    Published by Kan Onn Kit\\
    Singapore
    \vspace{5cm}

    \textbf{A Complete Introduction To Abstract Algebra}\par
    \editiontext
    \vspace{0.3cm}

    Copyright \copyright \ 2022 -- \the\year\ by Kan Onn Kit\par
    This work is licensed under a
    Creative Commons Attribution-NonCommercial-ShareAlike 4.0 International Licence.\par
    \pdfteximg{2.5cm}{images/CC_BY-NC-SA_4.0.pdf_tex}\\
    The full licence text is available at \url{http://creativecommons.org/licenses/by-nc-sa/4.0/}.\par    
    The source files for the project are available at \url{https://github.com/PhotonicGluon/Abstract-Algebra-Book}.
    \vspace{0.3cm}

    Typeset in 10pt \TeX~Gyre Pagella using PDF\LaTeX.
\end{minipage}

\vspace*{2\baselineskip}
\cleardoublepage

% Dedication page
\thispagestyle{empty}
\vspace*{1cm}

\begin{center}
    {\fontsize{18pt}{0}\selectfont \textit{For my friends}}\\
\end{center}

\vspace*{1.5cm}

\begin{center}
    \Large{\parbox{10cm}{
        \begin{raggedright}
            \Large
            Symmetry is a vast subject, significant in art and nature. Mathematics lies at its root, and it would be hard to find a better one on which to demonstrate the working of the mathematical intellect.
            \vspace{0.3cm}
            
            \hfill
            --- Hermann Weyl, 1952\\
            \vspace{-0.25cm}
            
            \hfill
            \normalsize
            ({\cite[p.~145]{weyl_1952}})
        \end{raggedright}
    }
}
\end{center}

\vspace*{\fill}

% "Content for no one... where you hide something so deeply that you don't care if anyone finds it [because] you just like knowing that it's there"
\begin{center}
    \fontsize{8pt}{8pt}\selectfont
    \code{3202 1503 2406 0000 0801 1109 0903 1601 2905}
\end{center}

\newpage

% Table of contents
\createtoc

% Acknowledgements
\chapter{Acknowledgements}
Undertaking such a monumental project is new to me, and I am indebted to the people who accompanied me on this journey.

I am eternally grateful to my parents, who have spent countless hours and an ungodly amount of effort to raise me into who I am today. Their omnipresent kindness, patience, and love for me are something I certainly do not deserve, and I thank them for taking care of me.

I would like to thank my tutor Leong Chong Ming, who got me interested in abstract algebra in the first place. His enthusiasm and eagerness to share his knowledge on the subject is the driving force behind my decision to write this book.

I am grateful for the help of my friend Low Ji Yuan, who has assisted me with countless revisions of the content in this book and given me another pair of eyes in the vetting of content.

I also sincerely appreciate the continued support from my mathematics tutors, Loke Weng Heng, Siow Yun Jie, and Teng Yen Ping, who have been there through my junior college years inspiring me with the wonders of mathematics. I am indebted to them for allowing me to excel in my final examinations.

My close friends, Aidan Tay, Gabriel Fong, and Low Ji Yuan, have accompanied me through two years of schooling (and math jokes). I offer infinite thanks to them for sticking with me and for encouraging this math nerd to pursue his wacky projects.

A thousand thanks go out to my teachers at the School of Science and Technology, Singapore, and specifically my form teacher Lee Tsi Yew Samuel, who instilled important character values into me so I can excel in my future endeavours.

% Preface
\chapter{Preface}
Although algebra has a long history, it has undergone quite striking changes in the past few decades. Abstract (or modern) algebra is widely recognised as an essential element of higher mathematical education. However, the results that it showcases are often hard to grasp and understand without prerequisite knowledge or a heavy background in mathematics. Most books on this subject are crafted for undergraduates at universities. They are not for a general mathematics enthusiast or one who seeks to understand more about the inner structure of algebra that mathematicians encounter frequently.

The exploration of such structures is fundamental to the current underpinning of scientific inquiries. For example, groups are important as they describe the symmetries which the laws of physics seem to obey. Finite fields are also used in coding theory and combinatorics. I hope this book will inspire more people to learn more about abstract algebra, beyond the simple introduction presented here.

In addition, I find that, in most textbooks, important details are left for the reader to figure out independently, without providing any additional guidance or help along the way. Exercises and problems are provided for topics taught in abstract algebra, but only a handful have written solutions provided, leaving readers unsure of the correctness of their answers. I believe that the completeness of a textbook is essential; no claim made should be without justification (unless necessary). This book offers a more complete picture of abstract algebra by providing full-worked solutions to all exercises and problems posed.

This book serves to achieve several goals.
\begin{itemize}
    \item Provide a step-by-step explanation of core results from abstract algebra, without ambiguity of the results discussed.
    \item Demystify the core steps that many textbooks gloss over when proving results or when writing the solutions to problems/exercises.
    \item Ensure that results from abstract algebra are as accessible, as approachable, and as understandable for as many people as possible.
\end{itemize}
I hope that this book can accomplish these goals and let readers enjoy the wonders of abstract algebra.

\hfill{\textit{2 October, 2023}}

% Suggestions on the use of this book
\chapter{Suggestions on the Use of This Book}
\section*{General Information}
\begin{itemize}
    \item For most parts, we include both exercises and problems.
    \begin{itemize}
        \item An exercise can be thought of as a simple ``self-review'' question. Exercises ensure that the content of a particular section is understood, and is relatively simple to answer.
        \item A problem is a more holistic version of an exercise. Generally, solutions to problems require a thorough understanding of the current chapter and may require results from other chapters.        
    \end{itemize}
    Full solutions to both exercises and problems are included. Note that these are just suggested answers; other approaches may be valid.

    \item Results and questions have differing numbering systems.
    \begin{itemize}
        \item All definitions, axioms, examples, lemmas, theorems, propositions, and corollaries are consecutively numbered using the format
        \begin{quote}
            \code{[CHAPTER].[SECTION].[NUMBER]}
        \end{quote}
        For example, the fourth statement in chapter 2, section 3 is labelled \textbf{2.3.4}.
        \item Exercises and problems are also numbered consecutively using the format
        \begin{quote}
            \code{[CHAPTER].[NUMBER]}
        \end{quote}
        For example, the third exercise in chapter 2 is labelled \textbf{2.3}. Likewise, the fourth exercise in chapter 3 is labelled \textbf{3.4}.
    \end{itemize}
    \item For more technical proofs, we include a sketch of the proof along with the full proof. The symbol ``$\qedsketch$'' marks the end of the sketch of a proof, while ``$\qedproof$'' marks the end of a proof.
\end{itemize}

\section*{Chapter Interdependence}
The diagram on the next page shows chapter interdependence. It should be used in conjunction with the table of contents and notes listed.

\newpage
\begin{center}
    \pdfteximg{\linewidth}{images/interdependence.pdf_tex}
\end{center}

\newpage

\textbf{Notes}:
\begin{itemize}
    \item Each part is largely independent from the other parts. However, part 0 is required reading for future parts of the book.
    \item Part 1 on group theory can be thought of as the fundamentals of abstract algebra.
    \begin{itemize}
        \item Chapter 8 is essentially independent from the rest of the other chapters. It provides motivation for the axioms of groups, but readers who want to skip this introduction can move straight to chapter 9.
        \item Chapters 9, 10, and 11 are considered to be the essentials of group theory.
        \item Chapter 11 is required reading for chapters 12, 13, 14, and 16.
        \item Chapter 14 only requires knowledge of the subgroup product from chapter 13 (specifically \myref{definition-subgroup-product} and \myref{prop-subgroup-product-is-subgroup}); otherwise these two chapters are relatively independent.
        \item Chapter 15 assumes knowledge of chapter 14, and knowledge of permutations and the symmetric group from chapter 12.
        \item Usually, group actions (chapter 16) would be read after the essentials of group theory; therefore chapter 16 could be read after chapter 11.
        \item Chapter 17 only requires chapters 14 and 16.
        \item Chapter 18 only require results from chapter 14, except for \myref{problem-S4-composition-series} which uses the alternating group introduced in chapter 15.
        \item Chapter 19 assumes full knowledge of chapter 14; minor results from chapter 15 (specifically, the alternating group), chapter 17 (the Third Sylow Theorem, \myref{thrm-sylow-3}), and chapter 18 (\myref{problem-S4-composition-series}) are required.
    \end{itemize}
    \item Part 2 on rings builds on ideas of group theory.
    \begin{itemize}
        \item Chapters 9 and 10 from group theory are required reading before continuing with ring theory.
        \item TODO: Add ring theory interdependence.
    \end{itemize}
\end{itemize}
