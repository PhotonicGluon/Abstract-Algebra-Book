\chapter{Algebraic Extensions}
Elements in field extensions are one of two types. The first are elements that are a zero of a specific polynomial within the base field. The second are the rest. Elements that are a zero of a polynomial is of particular interest to us, and we will investigate the extensions that create all possible zeroes of polynomials.

\section{Algebraic Elements}
\begin{definition}
    Let $E/F$ be a field extension. Then $\alpha \in E$ is \textbf{algebraic}\index{algebraic}\index{algebraic!element} over $F$ if and only if there exists some non-constant polynomial $f(x) \in F[x]$ such that $\alpha$ is a zero of $f(x)$.

    An element that is not algebraic over $F$ is called \textbf{transcendental}\index{transcendental} over $F$.
\end{definition}

\begin{example}
    We know that both $\sqrt2$ and $i = \sqrt{-1}$ are algebraic over $\Q$ since they are the zeroes of the polynomials $x^2 - 2$ and $x^2 + 1$ respectively.
\end{example}

\begin{example}
    Both $\pi$ and $e$ are algebraic over $\R$ since they are the zeroes of the polynomials $x - \pi$ and $x - e$ respectively. However, it is non-trivial to show that $\pi$ and $e$ are not algebraic (i.e., transcendental) over $\Q$.
\end{example}

For the case when the field in question is $\Q$, we have a more commonplace definition for algebraic and transcendental elements.

\begin{definition}
    An $\alpha \in \C$ that is algebraic over $\Q$ is called an \textbf{algebraic number}\index{algebraic!number}. Otherwise it is called a \textbf{transcendental number}\index{transcendental!number}.
\end{definition}

\begin{example}
    We show that $\sqrt{2+\sqrt3}$ is an algebraic number. Let $\alpha = \sqrt{2+\sqrt3}$; one sees $\alpha^2 = 2 + \sqrt3$. Therefore $\alpha^2 - 2 = \sqrt3$ and so $(\alpha^2 - 2)^2 = 3$, which means that
    \[
        (\alpha^2 - 2)^2 - 3 = 0.
    \]
    One may then expand the left hand side to yield $\alpha^4 - 4\alpha^2 + 1$. Therefore $\alpha$ is a zero of the polynomial $x^4 - 4x^2 + 1 \in \Q[x]$, meaning that $\alpha$ is an algebraic number.
\end{example}

\begin{exercise}
    Is $\sqrt{2 - \sqrt{i}}$ a transcendental number?
\end{exercise}

\section{Characterising Extensions}
% TODO: Add

\section{Finite Extensions}
% TODO: Add

\section{Properties of Extensions}
% TODO: Add

\newpage

\section{Problems}
% TODO: Add