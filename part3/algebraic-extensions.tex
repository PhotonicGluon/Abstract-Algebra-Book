\chapter{Algebraic Extensions}
Elements in field extensions are one of two types. The first are elements that are a zero of a specific polynomial within the base field. The second are the rest. Elements that are a zero of a polynomial is of particular interest to us, and we will investigate the extensions that create all possible zeroes of polynomials.

\section{Algebraic Elements}
\begin{definition}
    Let $E/F$ be a field extension. Then $\alpha \in E$ is \textbf{algebraic}\index{algebraic}\index{algebraic!element} over $F$ if and only if there exists some non-constant polynomial $f(x) \in F[x]$ such that $\alpha$ is a zero of $f(x)$.

    An element that is not algebraic over $F$ is called \textbf{transcendental}\index{transcendental} over $F$.
\end{definition}

\begin{example}
    We know that both $\sqrt2$ and $i = \sqrt{-1}$ are algebraic over $\Q$ since they are the zeroes of the polynomials $x^2 - 2$ and $x^2 + 1$ respectively.
\end{example}

\begin{example}
    Both $\pi$ and $e$ are algebraic over $\R$ since they are the zeroes of the polynomials $x - \pi$ and $x - e$ respectively. However, it is non-trivial to show that $\pi$ and $e$ are not algebraic (i.e., transcendental) over $\Q$.
\end{example}

For the case when the field in question is $\Q$, we have a more commonplace definition for algebraic and transcendental elements.

\begin{definition}
    An $\alpha \in \C$ that is algebraic over $\Q$ is called an \textbf{algebraic number}\index{algebraic!number}. Otherwise it is called a \textbf{transcendental number}\index{transcendental!number}.
\end{definition}

\begin{example}
    We show that $\sqrt{2+\sqrt3}$ is an algebraic number. Let $\alpha = \sqrt{2+\sqrt3}$; one sees $\alpha^2 = 2 + \sqrt3$. Therefore $\alpha^2 - 2 = \sqrt3$ and so $(\alpha^2 - 2)^2 = 3$, which means that
    \[
        (\alpha^2 - 2)^2 - 3 = 0.
    \]
    One may then expand the left hand side to yield $\alpha^4 - 4\alpha^2 + 1$. Therefore $\alpha$ is a zero of the polynomial $x^4 - 4x^2 + 1 \in \Q[x]$, meaning that $\alpha$ is an algebraic number.
\end{example}

\begin{exercise}
    Is $\sqrt{2 - \sqrt{i}}$ a transcendental number?
\end{exercise}

\newpage

\section{Characterising Extensions}
With an understanding of what algebraic and transcendental elements are, we link these definitions back to the idea of field extensions.

\begin{definition}
    Let $F$ be a field. A field extension $E/F$ is called an \textbf{algebraic extension}\index{extension!algebraic} if and only if every element of $E$ is algebraic over $F$. Otherwise $E$ is called a \textbf{transcendental extension}\index{extension!transcendental} of $F$.
\end{definition}

We expand upon \myref{thrm-simple-extension-isomorphism} and provide a characterisation of simple extensions of both algebraic and transcendental elements.

\begin{theorem}\label{thrm-characterisation-of-extensions}
    Let $F$ be a field, $E/F$ a field extension, and $\alpha \in E$.
    \begin{itemize}
        \item If $\alpha$ is algebraic then $F(\alpha) \cong F[x]/\princ{p(x)}$, where $p(x) \in F[x]$ is a polynomial in $F[x]$ of minimum degree such that $\alpha$ is a zero. Furthermore $p(x)$ is irreducible over $F$.
        \item If $\alpha$ is transcendental over $F$ then $F(\alpha) \cong \Frac{F[x]}$.
    \end{itemize}
\end{theorem}
\begin{proof}
    Consider the map $\phi: F[x] \to F(\alpha)$ where $f(x) \mapsto f(\alpha)$. Recall from \myref{thrm-simple-extension-isomorphism} that $\phi$ is a homomorphism.

    If $\alpha$ is algebraic over $F$, then by definition of an algebraic element, thus there is a polynomial in $F[x]$, say $f(x)$, that $\alpha$ is a zero of, i.e. $f(\alpha) = 0$. Hence $\ker\phi$ is non-trivial; by \myref{thrm-criterion-for-principal-ideal-in-polynomial-field} there is a polynomial $p(x) \in F[x]$ such that $\ker\phi = \princ{p(x)}$ and where $p(x)$ has minimum degree among all non-zero elements of $\ker\phi$.

    We now show that $p(x)$ is irreducible. Suppose not, that $p(x) = f(x)g(x)$ where $f(x), g(x) \in F[x]$ are non-constant polynomials. Hence $f(x)$ and $g(x)$ are polynomials of degree smaller than $p(x)$. However, since $\alpha$ is a zero of $p(x)$, this means that $f(\alpha) = 0$ or $g(\alpha) = 0$. But this contradicts the minimality of the degree of $p(x)$; therefore $p(x)$ is irreducible.

    On the other hand, if $\alpha$ is transcendental, then there does not exist a polynomial in $F[x]$ such that $\alpha$ is a zero of. Therefore $\ker\phi$ is trivial. We extend $\phi$ to become the map $\psi: \Frac{F[x]} \to F(\alpha)$ where $\psi\left(\frac{f(x)}{g(x)}\right) = f(\alpha)(g(\alpha))^{-1}$. We prove that $\psi$ is a well-defined isomorphism.
    \begin{itemize}
        \item \textbf{Well-Defined}: Suppose $\frac{f(x)}{g(x)} = \frac{p(x)}{q(x)}$ for polynomials $f(x), g(x), p(x), q(x) \in F[x]$. Thus $f(x)q(x) = p(x)g(x)$ by definition of equality of equivalence classes in the field of fractions of $F[x]$. So we see $f(\alpha)q(\alpha) = p(\alpha)g(\alpha)$ which quickly means $f(\alpha)(g(\alpha))^{-1} = p(\alpha)(q(\alpha))^{-1}$. Therefore
        \[
            \psi\left(\frac{f(x)}{g(x)}\right) = f(\alpha)(g(\alpha))^{-1} = p(\alpha)(q(\alpha))^{-1} = \psi\left(\frac{p(x)}{q(x)}\right)
        \]
        which shows that $\psi$ is a well-defined map.

        \item \textbf{Homomorphism}: Let $\frac{f(x)}{g(x)}, \frac{p(x)}{q(x)} \in \Frac{F[x]}$. Then we see
        \begin{align*}
            \psi\left(\frac{f(x)}{g(x)} + \frac{p(x)}{q(x)}\right) &= \psi\left(\frac{f(x)q(x) + g(x)p(x)}{g(x)q(x)}\right)\\
            &= \left(f(\alpha)q(\alpha) + g(\alpha)p(\alpha)\right)\left(g(\alpha)q(\alpha)\right)^{-1}\\
            &= f(\alpha)q(\alpha)\left(g(\alpha)q(\alpha)\right)^{-1} + g(\alpha)p(\alpha)\left(g(\alpha)q(\alpha)\right)^{-1}\\
            &= f(\alpha)(g(\alpha))^{-1} + p(\alpha)(q(\alpha))^{-1}\\
            &= \psi\left(\frac{f(x)}{g(x)}\right) + \psi\left(\frac{p(x)}{q(x)}\right)
        \end{align*}
        and
        \begin{align*}
            \psi\left(\frac{f(x)}{g(x)}\times \frac{p(x)}{q(x)}\right) &= \psi\left(\frac{f(x)p(x)}{g(x)q(x)}\right)\\
            &= f(\alpha)p(\alpha)\left(g(\alpha)q(\alpha)\right)^{-1}\\
            &= f(\alpha)(g(\alpha))^{-1}p(\alpha)(q(\alpha))^{-1}\\
            &= \psi\left(\frac{f(x)}{p(x)}\right)\psi\left(\frac{p(x)}{q(x)}\right)
        \end{align*}
        so $\psi$ is indeed a homomorphism.
        
        \item \textbf{Injective}: Since $\psi$ is non-trivial it is thus injective by \myref{thrm-homomorphism-from-field-is-injective-or-trivial}.
        
        \item \textbf{Surjective}: Suppose $r \in F(\alpha)$. Note that both  1 and $r$ are constant polynomials in $F[x]$, so $\frac r1 \in \Frac{F[x]}$. We therefore see $\psi\left(\frac r1\right) = r1^{-1} = r$ and hence every element in $F(\alpha)$ has a pre-image.
    \end{itemize}
    Hence $\psi$ is a well-defined isomorphism, meaning $F(\alpha) \cong \Frac{F[x]}$.
\end{proof}

For an algebraic element $\alpha$, \myref{thrm-characterisation-of-extensions} gives us an irreducible polynomial $p(x)$ of minimum degree which $\alpha$ is a root of. If $p(x)$ is monic we have a slightly stronger result.

\begin{corollary}\label{corollary-unique-minimal-polynomial}
    Let $F$ be a field and $E/F$ be a field extension. If $\alpha \in E$ is algebraic over $F$, then there is a unique monic irreducible polynomial $p(x) \in F[x]$ such that $\alpha$ is a zero of $p(x)$.
\end{corollary}
\begin{proof}
    Consider the map $\phi: F[x] \to F(\alpha)$ where $f(x) \mapsto f(\alpha)$. From \myref{thrm-characterisation-of-extensions} we know that $\ker\phi \neq \emptyset$ since $\alpha$ is algebraic over $F$. We also know that there exists a irreducible polynomial of minimal degree in $F[x]$ such that $\alpha$ is a zero of it. For brevity, let $J_\alpha = \ker\phi$.

    Now suppose $f(x), g(x) \in F[x]$ are both monic irreducible polynomials of minimal degree in $F[x]$ with $\alpha$ as a zero. Hence $f(x), g(x) \in J_\alpha$. Let $r(x) = f(x) - g(x)$; note $r(x) \in J_\alpha$. 
    
    Seeking a contradiction, suppose $r(x) \neq 0$. Then $\deg r(x) < \deg f(x)$ since both $f(x)$ and $g(x)$ are minimal degree (which means that they are of the same degree). Let $m = \deg r(x)$ and $c_m$ be the leading coefficient of $r(x)$. Now observe that $c_m^{-1}r(x)$ is a monic polynomial with coefficients in $F$, and that $c_m^{-1}r(x) \in J_\alpha$. Furthermore $\alpha$ is a zero of $c_m^{-1}r(x)$; so we have found a polynomial of smaller degree than $f(x)$ (and $g(x)$) with $\alpha$ as a zero, contradicting the assumption that $f(x)$ (and $g(x)$) are of minimal degree.

    Hence $r(x) = 0$, meaning that $f(x) - g(x) = 0$ and thus $f(x) = g(x)$, proving the uniqueness of the monic irreducible polynomial in $F[x]$ such that $\alpha$ is a zero.
\end{proof}

There is a name for the unique monic irreducible polynomial in \myref{corollary-unique-minimal-polynomial}.

\begin{definition}
    Let $F$ be a field, $E/F$ be a field extension, and $\alpha \in E$ be algebraic over $F$. The monic irreducible polynomial $p(x) \in F[x]$ which $\alpha$ is a root of is called the \textbf{minimal polynomial for $\alpha$ over $F$}\index{extension!algebraic!minimal polynomial}.
\end{definition}

We note one more corollary regarding the characterisation of extensions.

\begin{corollary}\label{corollary-minimal-polynomial-divides-polynomial-with-same-root}
    Let $F$ be a field, $E/F$ be a field extension, $\alpha \in E$ be algebraic over $F$, and $p(x) \in F[x]$ be the minimal polynomial for $\alpha$ over $F$. If $f(x) \in F[x]$ has a zero of $\alpha$ then $p(x)$ divides $f(x)$.
\end{corollary}
\begin{proof}
    Let $p(x), f(x) \in F[x]$ be as above. Using polynomial long division with divisor $p(x)$ we see
    \[
        f(x) = q(x)p(x) + r(x)
    \]
    where $r(x) = 0$ or $\deg r(x) < \deg p(x)$. If the latter case applies, then we see
    \[
        f(\alpha) = q(\alpha)p(\alpha) + r(\alpha).
    \]
    Since $\alpha$ is a zero of both $f(x)$ and $p(x)$ we thus see $0 = 0 + r(\alpha)$ which means $\alpha$ is a zero of $r(x)$ too; but $\deg r(x) < \deg p(x)$ contradicts the minimality of the degree of $p(x)$. Therefore $r(x) = 0$ which means $f(x) = q(x)p(x)$, i.e. $p(x)$ divides $f(x)$.
\end{proof}

\section{Finite Extensions}
Recall from \myref{thrm-field-is-vector-space} that a field is a vector space over a subfield. Similarly, an extension field is a vector space over the base field. With the idea of vector spaces, we can define the degree of a particular field extension.

\begin{definition}
    Let $F$ be a field and $E/F$ be a field extension. We say that $E$ has \textbf{degree $n$ over $F$}\index{extension field!degree}\index{degree!extension field} and write $[E:F] = n$ if and only if $\dim E = n$ when viewed as a vector space over $F$.

    If $[E:F]$ is finite, then $E$ is called a \textbf{finite extension}\index{extension!finite} of $F$. Otherwise $E$ is called an \textbf{infinite extension}\index{extension!infinite} of $F$.
\end{definition}

We can depict the degree of a field extension using a diagram.
\begin{figure}[H]
    \centering
    \pdfteximg{0.2\textwidth}{part3/images/algebraic-extensions/degree-of-extension.pdf_tex}
    \caption{Field Extension Diagram}
\end{figure}

Of course, such a diagram is meaningless without context, so let us look at some examples.

\begin{examplewithcutout}{right}{0}{0.7\textwidth}{0pt}{7}{
    \begin{figure}[H]
        \centering
        \pdfteximg{0.6\linewidth}{part3/images/algebraic-extensions/Q-sqrt2-and-Q.pdf_tex}
    \end{figure}
}
    Consider the simple extension $\Q(\sqrt2)$. Viewed as a vector space over $\Q$, we see that $\Q(\sqrt 2)$ has dimension 2, as a possible basis is $\{1, \sqrt2\}$. Hence, we write $[\Q(\sqrt2):\Q] = 2$ and produce the field extension diagram as shown on the right.

    We note that $\Q(\sqrt2)$ is a finite extension over $\Q$ since its degree is finite.
\end{examplewithcutout}

\begin{examplewithcutout}{right}{0}{0.7\textwidth}{0pt}{7}{
    \begin{figure}[H]
        \centering
        \pdfteximg{0.6\linewidth}{part3/images/algebraic-extensions/Q-cbrt2-and-Q.pdf_tex}
    \end{figure}
}
    Consider the simple extension $\Q(\sqrt[3]2)$. We note $\Q(\sqrt[3]2)$ has dimension 3 when viewed as a vector space over $\Q$, and a possible basis for $\Q(\sqrt[3]2)$ is $\{1, \sqrt[3]2, \sqrt[3]4\}$. Hence, we write $[\Q(\sqrt[3]2):\Q] = 3$ and produce the field extension diagram as shown on the right. Like with the previous example, $\Q(\sqrt[3]2)$ is a finite extension since its degree, 3, is finite.
\end{examplewithcutout}

\begin{examplewithcutout}{right}{0}{0.7\textwidth}{0pt}{7}{
    \begin{figure}[H]
        \centering
        \pdfteximg{0.6\linewidth}{part3/images/algebraic-extensions/C-and-R.pdf_tex}
    \end{figure}
}
    The field of complex numbers, $\C$, has degree 2 over the reals, $\R$, since $\{1, i\}$ is a basis of $\C$ over $\R$. This produces the field extension diagram on the right.
    
    However, $\C$ over $\Q$ is an infinite extension. In such a case, a field extension diagram is not meaningful.
\end{examplewithcutout}

\begin{examplewithcutout}{right}{0}{0.7\textwidth}{0pt}{7}{
    \begin{figure}[H]
        \centering
        \pdfteximg{0.6\linewidth}{part3/images/algebraic-extensions/simple-algebraic-extension-degree.pdf_tex}
    \end{figure}
}
    Let $F$ be a field and $E$ an extension field of $F$. If $\alpha \in E$ is algebraic over $F$ and its minimal polynomial $p(x) \in F[x]$ has degree $n$, then we know that $\{1, \alpha, \alpha^2, \dots, \alpha^{n-1}\}$ is a basis for the simple extension $F(\alpha)$ over $F$ by \myref{thrm-simple-extension-isomorphism}. Therefore, one sees that $[F(a): F] = n$ and we thus generate a field extension diagram as shown on the right.
\end{examplewithcutout}

The previous example leads us to define the degree of an algebraic element.

\begin{definition}
    Let $F$ be a field and $E/F$ be a field extension. If $\alpha \in E$ is algebraic and $F(\alpha)$ is a finite extension of $F$ of degree $n$, then the element $\alpha$ is said to have \textbf{degree $n$ over $F$}\index{degree!element}\index{algebraic!element!degree}.
\end{definition}

One sees clearly that algebraic elements `produce' finite extensions, but what about the reverse direction? What does a finite extension imply about the elements inside it? It turns out this implies that all elements inside the finite extension are algebraic over the base field.

\begin{theorem}
    If $E$ is a finite extension of a field $F$, then $E$ is an algebraic extension of $F$.
\end{theorem}
\begin{proof}[Proof (see {\cite[Theorem 21.4]{gallian_2016}})]
    Suppose the degree of $E$ over $F$ is $n$ and $\alpha \in E$. Then the set $\{1, \alpha, \alpha^2, \dots, \alpha^{n-1}, \alpha^n\}$ is a linearly dependent set over $F$ since this set has $n + 1$ elements while $\dim E = n$. Hence, by definition of linear dependence, there exist elements $a_0, a_1, \dots, a_n \in F$ that are not all zero such that
    \[
        a_0 + a_1\alpha + a_2\alpha^2 + \cdots + a_n\alpha^n = 0.
    \]
    Clearly this means that $\alpha$ is a zero of the non-zero polynomial
    \[
        f(x) = a_0 + a_1x + a_2x^2 + \cdots + a_nx^n
    \]
    which means that $\alpha$ is algebraic. Since $\alpha$ is arbitrary, therefore every element in $E$ is algebraic; consequently $E$ is an algebraic extension of $F$.
\end{proof}

% TODO: Continue

\section{Properties of Extensions}
% TODO: Add

\newpage

\section{Problems}
% TODO: Add