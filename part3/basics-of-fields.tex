\chapter{Basics of Fields}
In the previous part, we looked at rings. We explored the different types of rings and their properties. Fields are rings with more properties. We look at the basics of fields in this chapter.

\section{What is a Field?}
Recall that a ring is an algebraic structure with two operations, addition, and multiplication, defined on it. We succinctly summarised the ring axioms as follows.
\begin{itemize}
    \item \textbf{Addition-Abelian}: $(R, +)$ is an abelian group.
    \item \textbf{Multiplication-Semigroup}: $(R, \cdot)$ is a semigroup.
    \item \textbf{Distributive}: $\cdot$ is distributive over $+$. That is,
    \begin{itemize}
        \item $a \cdot (b + c) = (a \cdot b) + (b \cdot c)$; and
        \item $(a + b) \cdot c = (a \cdot c) + (b \cdot c)$.
    \end{itemize}
\end{itemize}

Recall that a field is defined to be a commutative division ring; it has an identity, and every non-zero element of a field has a multiplicative inverse. Therefore, an equivalent definition of a field, which we will use in part III, is as follows.
\begin{definition}
    A \textbf{field}\index{field} is a set $F$ with two binary operations $+$ and $\cdot$ satisfying the following axioms.
    \begin{itemize}
        \item \textbf{Addition-Abelian}\index{axiom!field!addition-abelian}: $(F, +)$ is an abelian group.
        \item \textbf{Multiplication-Abelian}\index{axiom!field!multiplication-abelian}: $(F^\ast, \cdot)$ is an abelian group, where $F^\ast$ denotes the set of non-zero elements of $F$.
        \item \textbf{Distributive}\index{axiom!field!distributive}: $\cdot$ is distributive over $+$. That is, $a \cdot (b + c) = (a \cdot b) + (b \cdot c)$.
    \end{itemize}
\end{definition}

As mentioned, fields are a more specific type of ring, so it is not surprising that the field axioms are very similar to that of the ring axioms. The notable difference here is that multiplication now is an abelian group, but only for the non-zero elements of the field.

\begin{example}
    We showed in part II (specifically in \myref{section-rings-more-definitions}) that $\Q$, $\R$, and $\C$ are fields.
\end{example}

\begin{example}
    Let $p$ be a prime. We found in \myref{example-Zp-is-field} that $\Z_p$ is a field.
\end{example}

\begin{definition}
    Let $F$ be a field under $+$ (addition) and $\cdot$ (multiplication).
    \begin{itemize}
        \item The \textbf{additive group}\index{field!additive group} of a field is $(F, +)$.
        \item The \textbf{multiplicative group}\index{field!multiplicative group} of a field is $(F^\ast, +)$.
    \end{itemize}
    Both the additive group and multiplicative group are abelian groups.
\end{definition}

\begin{example}
    The additive group of $\Q$ is $(\Q, +)$. We showed that this is, indeed, an abelian group in \myref{problem-Q-is-abelian-group-under-addition}. The multiplicative group of $\Q$ is $(\Q^\ast, \times)$. Note that $\Q^\ast$ is the set of all rational numbers except 0.
\end{example}

\begin{example}
    The additive group of $\Z_p$ is $(\Z_p, \oplus_p)$, where $\oplus_p$ denotes addition modulo $p$. Again, we have already shown that this is a group in \myref{prop-Zn-is-abelian-group}. The multiplicative group of $\Z_p$ is $(\Z_p^\ast, \otimes_p)$, where $\otimes_p$ denotes multiplication modulo $p$. Since every element of $\Z_p^\ast$ is coprime with $p$, thus $(\Z_p^\ast, \otimes_p)$ is in fact $\Un{p}$, the group of units modulo $p$, which we have explored in chapter \myref{section-group-of-units-mod-n}.
\end{example}

To end this section, we note some notation that we use for fields, which is similar to that used for rings.
\begin{itemize}
    \item The multiplication symbol $\cdot$ is usually omitted, so $x \cdot y$ is written as $xy$.
    \item The additive identity of $F$ will always be denoted by 0 and the multiplicative identity of $F$ will always be denoted by 1.
    \item The additive inverse of the element $x$ will be denoted by $-x$ and the multiplicative inverse of $x$ will be denoted by $x^{-1}$.
    \item $a - b$ means $a + (-b)$ for any two elements $a$ and $b$ in $F$.
\end{itemize}

\section{Basic Properties of Fields}
Most of these properties are inherited from the fact that fields are integral domains. However, we will restate them here in the context of fields for completeness.

For the following, assume $F$ is a field.

\begin{proposition}[Multiplication by Zero]
    $0x = x0 = 0$ for all $x \in F$.
\end{proposition}
\begin{proof}
    Follows from \myref{prop-multiplying-by-zero-is-zero}.
\end{proof}

\begin{proposition}
    $-0 = 0$ and $1^{-1} = 1$.
\end{proposition}
\begin{proof}
    See \myref{exercise-inverse-of-additive-and-multiplicative-identities-are-themselves} (later).
\end{proof}

\begin{proposition}
    $(-a)b = a(-b) = -(ab)$ for any $a$ and $b$ in $F$.
\end{proposition}
\begin{proof}
    Follows from \myref{prop-product-of-element-and-additive-inverse-is-additive-inverse-of-product}.
\end{proof}

\begin{proposition}
    $(-a)(-b) = ab$ for any $a$ and $b$ in $F$.
\end{proposition}
\begin{proof}
    Follows from \myref{prop-product-of-additive-inverses}.
\end{proof}

\begin{proposition}
    $F$ has unique additive and multiplicative identities.
\end{proposition}
\begin{proof}
    Since the additive group and multiplicative group are groups, thus they each contain a single unique identity.
\end{proof}

\begin{proposition}
    For all elements $a$ and $b$ in $F$, $-(a+b) = -a-b$ and $(ab)^{-1} = a^{-1}b^{-1}$.
\end{proposition}
\begin{proof}
    Recall that the Shoes and Socks theorem tells us that $-(a+b) = -b - a$ and $(ab)^{-1} = b^{-1}a^{-1}$. But as the additive and multiplicative groups are abelian, we see $-(a+b) = -a-b$ and $(ab)^{-1} = a^{-1}b^{-1}$.
\end{proof}

\begin{proposition}[Cancellation Law for Fields]\index{cancellation law!fields}
    Let $F$ be a domain, $r, x, y \in F$, and $r \neq 0$. Then the following statements are equivalent.
    \begin{enumerate}[label=(\arabic*)]
        \item $x = y$
        \item $rx = ry$
        \item $xr = yr$
    \end{enumerate}
\end{proposition}
\begin{proof}
    Follows from \myref{prop-domain-cancellation-law}.
\end{proof}

\begin{proposition}
    Let $F$ be a field, $x$ an element of the field, and let $m$ and $n$ be non-negative integers. Then
    \begin{enumerate}
        \item $mx + nx = (m+n)x$ and $(x^m)(x^n) = x^{m+n}$;
        \item $n(mx) = (mn)x$ and $(x^m)^n = x^{mn}$; and
        \item $n(-x) = -(nx)$ and $(x^{-1})^n = (x^n)^{-1}$.
    \end{enumerate}
\end{proposition}
\begin{proof}
    All statements follow by applying \myref{prop-group-laws-of-exponents} on both the additive and multiplicative groups.
\end{proof}

\begin{exercise}\label{exercise-inverse-of-additive-and-multiplicative-identities-are-themselves}
    Prove the following statements \textit{without} using any results from the previous chapters.
    \begin{partquestions}{\alph*}
        \item $-0 = 0$.
        \item $1^{-1} = 1$.
    \end{partquestions}
\end{exercise}

% TODO: Add characteristic

\section{Subfields}
% TODO: Add

\section{What Are the Finite Fields?}
% TODO: Add

\newpage

\section{Problems}
% TODO: Add
