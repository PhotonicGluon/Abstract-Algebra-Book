\section{Extension Fields and Splitting Fields}
\begin{questions}
    \item Note that since $p, q \in F$, thus $pa + q \in F(a)$. Hence $F(pa+q) \subseteq F(a)$. On the other hand, note that because $p \neq 0$ thus $p^{-1}$ exists, so
    \[
        a = p^{-1}(pa+q) - p^{-1}q
    \]
    and thus $a \in F(pa+q)$. Hence $F(a) \subseteq F(pa+q)$ also. Therefore $F(pa+q) = F(a)$.

    \item \begin{partquestions}{\alph*}
        \item One sees that $x^3-3x+2 = (x-1)^2(x+2)$, so $x^3-3x+2$ splits in $\Q$. Hence $\Q$ is the splitting field of $x^3-3x+2$ over $\Q$.

        \item We see $x^3 - 1 = (x-1)(x^2+x+1)$. Note $x^2 + x + 1$ has zeroes of $\frac{1\pm\sqrt{-3}}{2} = \frac12 \pm \frac12\sqrt{-3}$ by the quadratic formula. Thus the splitting field of $x^3 - 1$ over $\Q$ is
        \begin{align*}
            &\Q\left(1, \frac12 + \frac12\sqrt{-3}, \frac12 - \frac12\sqrt{-3}\right)\\
            &=\Q(1)\left(\frac12 + \frac12\sqrt{-3}\right)\left(\frac12 - \frac12\sqrt{-3}\right)\\
            &=\Q(\sqrt{-3})(\sqrt{-3}) & (\myref{problem-simple-extension-absorbs-field-elements})\\
            &=\Q(\sqrt{-3}).
        \end{align*}

        \item Given $x^4 + x^2 + 1 = (x^2 - x + 1)(x^2 + x + 1)$. The zeroes of $x^2 + x + 1$ are $\frac12 \pm \frac12\sqrt{-3}$; the zeroes of $x^2 - x + 1$ are $-\frac12 \pm \frac12\sqrt{-3}$. Therefore the splitting field of $x^4 + x^2 + 1$ over $\Q$ is
        \begin{align*}
            &\Q\left(\frac12 + \frac12\sqrt{-3}, \frac12 - \frac12\sqrt{-3}, -\frac12 + \frac12\sqrt{-3}, -\frac12 - \frac12\sqrt{-3}\right)\\
            &=\Q\left(\frac12 + \frac12\sqrt{-3}\right)\left(\frac12 - \frac12\sqrt{-3}\right)\left(-\frac12 + \frac12\sqrt{-3}\right)\left(-\frac12 - \frac12\sqrt{-3}\right)\\
            &=\Q(\sqrt{-3})(\sqrt{-3})(\sqrt{-3})(\sqrt{-3})\\
            &= \Q(\sqrt{-3}).
        \end{align*}

        \item The zeroes of $x^2 + 2\sqrt2x + 3$ are $\sqrt2 \pm i$ by quadratic formula. Hence the splitting field of $x^2 + 2\sqrt2x + 3$ over $\Q(\sqrt2)$ is
        \begin{align*}
            \Q(\sqrt2)(\sqrt2+i, \sqrt2-i) &= \Q(\sqrt2)(\sqrt2 + i)(\sqrt2 - i)\\
            &= Q(\sqrt2, i).
        \end{align*}
    \end{partquestions}

    \item \begin{partquestions}{\alph*}
        \item Let $f(x) = x^4 + x + 7$. Note that $f(0) = 7 = 1 \neq 0$ and $f(1) = 1 + 1 + 7 = 9 = 1 \neq 0$, so $f(x)$ does not have a zero in $\Z_2$. Hence $f(x)$ does not have a multiple zero.

        \item Let $f(x) = x^{19} + x^8 + 1$, so $f'(x) = 19x^{18} + 8x^7$. In $\Z_3[x]$, $f'(x) = x^{18} + 2x^7$. Note that $f(1) = 1 + 1 + 1 = 3 = 0$ and $f'(1) = 1 + 2 = 3 = 0$, so both $f(x)$ and $f'(x)$ share a common factor of positive degree of $x-1$ by Factor Theorem (\myref{corollary-factor-theorem}). Hence 1 is a multiple zero by \myref{thrm-criterion-for-multiple-zeroes}.

        \item Let $f(x) = 2x^6 + x^4 + 2x^3 + 2$, so $f'(x) = 12x^5 + 4x^3 + 6x^2$. In $\Z_3[x]$, $f'(x) = x^3$. Note that $f(0) = 2 \neq 0$, $f(1) = 2 + 1 + 2 + 2 = 7 = 1 \neq 0$, but $f(2) = 2(2)^6 + 2^4 + 2(2)^3 + 2 = 162 = 0$, so the only zero of $f(x)$ is 2. But $f'(2) = 2^3 = 8 = 2 \neq 0$. Hence $f(x)$ and $f'(x)$ do not share a common factor of positive degree, meaning that $f(x)$ has no multiple zeroes (\myref{thrm-criterion-for-multiple-zeroes}).

        \item Let $f(x) = x^8 + 3x^5 + x^3 + 5$, so $f'(x) = 8x^7 + 15x^4 + 3x^2$. In $\Z_7[x]$, $f'(x) = x^7 + x^4 + 3x^2$. Note $f(4) = 68677 = 0$ and $f'(4) = 16688 = 0$, so $f(x)$ and $f'(x)$ share a common factor of positive degree of $x-4$ by Factor Theorem (\myref{corollary-factor-theorem}). Hence 4 is a multiple zero by \myref{thrm-criterion-for-multiple-zeroes}.
    \end{partquestions}

    \item \begin{partquestions}{\roman*}
        \item Prove. Since $p(x)$ is irreducible, thus $\princ{p(x)}$ is maximal (\myref{thrm-irreducible-iff-principal-ideal-is-maximal}). Therefore $\Z_2[x]/\princ{p(x)}$ is a field by \myref{thrm-maximal-ideal-iff-quotient-ring-is-field}.

        \item Prove. Since $\Z_2[x]/\princ{p(x)}$ is a field, thus $\left(\Z_2[x]/\princ{p(x)}\right)^\ast$ is in fact the multiplicative group of the field.

        \item Prove. Note that $\left|\Z_2[x]/\princ{p(x)}\right| = 2^7 = 128$, so $\left|\left(\Z_2[x]/\princ{p(x)}\right)^\ast\right| = 2^7 - 1 = 127$, since the multiplicative group omits the additive identity. One sees that 127 is prime. Hence every non-identity element of $\left(\Z_2[x]/\princ{p(x)}\right)^\ast$ is a generator (\myref{exercise-prime-order-element}).
    \end{partquestions}

    \item Suppose $\alpha_1, \alpha_2, \dots, \alpha_n$ are the zeroes of $f(x)$. Then $\alpha_1 - a, \alpha_2 - a, \dots, \alpha_n - a$ are the zeroes of $f(x+a)$. Thus the splitting field of $f(x+a)$ over $F$ is
    \begin{align*}
        F(\alpha_1 - a, \alpha_2 - a, \dots, \alpha_n - a) &= F(\alpha_1 - a)(\alpha_2 - a)\dots(\alpha_n - a)\\
        &= F(\alpha_1)(\alpha_2)\dots(\alpha_n) & (\myref{problem-simple-extension-absorbs-field-elements})\\
        &= F(\alpha_1, \alpha_2, \dots, \alpha_n)
    \end{align*}
    which is the splitting field of $f(x)$ over $F$.

    \item By definition of a simple extension, $\Q(\sqrt2)$ contains $\Q$. But $\Q$ is a prime field (\myref{thrm-Q-is-prime-field}), so it contains no smaller subfields. Therefore the only subfields of $\Q(\sqrt2)$ are $\Q$ and $\Q(\sqrt2)$.

    \item Let $f(x) = x^{p^n} - x$. Thus $f'(x) = p^nx^{p^n-1}-1$. Note that since $\Char{F} = p$, therefore $p^nx^{p^n-1} = 0$. Hence $f'(x) = -1$; clearly $f'(x)$ shares no common factor of positive degree with $f(x)$. Hence, there are no multiple zeroes of $f(x)$ (\myref{thrm-criterion-for-multiple-zeroes}), meaning that any zero of $f(x)$ must be simple.

    \item We consider two cases.

    If $f(x)$ has a zero in $F$, say $\alpha$, then that must mean $\alpha^p - a = 0$ by definition of a zero. Hence $a = \alpha^p$, meaning
    \[
        f(x) = x^p - \alpha^p = (x-\alpha)^p
    \]
    by the Freshman's Dream (\myref{prop-freshman-dream}). This clearly means that $f(x)$ splits over $F$.

    Otherwise $f(x)$ has no zeroes in $F$. Let $E/F$ be the splitting field of $f(x)$, and suppose $\beta \in E$ is a zero of $f(x)$. Using the above working we know that $f(x) = (x-\beta)^p$.

    Now, seeking a contradiction, suppose $f(x)$ is reducible over $F$, meaning that there exist non-constant polynomials $g(x), h(x) \in F[x]$ such that $f(x) = g(x)h(x)$. In particular, $g(x) = (x-\beta)^n$ for some positive integer $n$ where $1 \leq n \leq p$. Expanding $g(x)$ using the Binomial Theorem yields
    \[
        g(x) = x^n + n\beta x^{n-1} + {n\choose2}\beta^2x^{n-2} + \cdots + n\beta^{n-1}x + \beta^n \in F[x],
    \]
    and in particular this means that $\beta^n \in F$. Also, since $f(x) = (x-\beta)^p$, we draw a similar conclusion that $\beta^p \in F$.

    Now since $1 \leq n \leq p$ thus $n$ and $p$ are coprime, i.e. $\gcd(n,p) = 1$. By B\'ezout's lemma (\myref{lemma-bezout}) we are able to find integers $s$ and $t$ such that $sn + tp = 1$. In particular, by the closure of fields, we know $\beta^{sn} = \left(\beta^n\right)^s \in F$ and $\beta^{tp} = \left(\beta^p\right)^t \in F$, so
    \[
        \beta = \beta^1 = \beta^{sn+tp} = \beta^{sn}\beta^{tp} \in F.
    \]
    Thus $f(x)$ has a zero, namely $\beta$, in $F$. But this contradicts the fact that $f(x)$ has no zeroes in $F$. Therefore, $f(x)$ is irreducible over $F$.
\end{questions}
