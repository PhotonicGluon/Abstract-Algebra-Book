\chapter{Finite Fields}
Finite fields appear in many applications of algebra (like in the NTRU cryptosystem). We already found out that $\Z_p$ is a field, and that we may construct fields of prime-power order via the use of irreducible polynomials. In this chapter, we discover one of the most surprising and important results relating to these fields.

\section{Classification and Structure of Finite Fields}
\'Evariste Galois first introduced finite fields in his proof of the insolvability of the quintic in 1830. Today, finite fields are immensely useful in many aspects of algebra and computing.

The most surprising and important result regarding finite fields is that there is a unique finite field (up to isomorphism) of order $p^n$ where $p$ is a prime. The existence of such a field was given by Galois and Gauss in the early 19th century, but it took until Dedekind in 1857 and Jordan in 1870 to produce a rigorous proof of the existence of such fields. The uniqueness of such a field was proved by Eliakim Hastings Moore in 1893. We state his theorem below.

\begin{theorem}\label{thrm-finite-field-is-unique}
    For each prime $p$ and every positive integer $n$, there exists a finite field of order $p^n$. Furthermore any field of order $p^n$ is isomorphic to the splitting field of $x^{p^n} - x$ over $\Z_p$.
\end{theorem}
\begin{proof}[Proof (see {\cite[Theorem 22.1]{gallian_2016}} and {\cite[Theorem 22.6]{judson_beezer_2022}})]
    Let $f(x) = x^{p^n} - x$ and let $F$ be the splitting field of $f(x)$ over $\Z_p$. We show that $|F| = p^n$. Since $f(x)$ has degree $p^n$ and splits over $F$, thus there must be $p^n$ zeroes of $f(x)$ in $F$. Also we know that each of these $p^n$ zeroes are simple (\myref{problem-(x^p^n-x)-only-has-simple-zeroes}), meaning that $f(x)$ has $p^n$ distinct zeroes in $F$. We note that \myref{exercise-zeroes-of-polynomial-form-subfield} (later) shows that the set of zeroes of $f(x)$ in $F$, which we will denote by $S$, form a subfield of $F$. Note $|S| = p^n$. But as $S$ contains all zeroes of $f(x)$ in $F$, we know that $f(x)$ splits over $S$. As $S \subseteq F$, thus $S = F$, meaning $|F| = p^n$.

    We now show uniqueness. Suppose $K$ is another field of order $p^n$. To show that $K \cong F$, we need to show that every element of $K$ is a zero of $f(x)$. Certainly $0 \in K$ is a zero of $f(x)$. Let $\alpha \in K$ be a non-zero element. Since the multiplicative group of $K$ has order $p^n - 1$, thus we must have $\alpha^{p^n-1} = 1$, which means $\alpha^{p^n} = \alpha$ and so $\alpha^{p^n}-\alpha = 0$. Thus any non-zero $\alpha \in K$ is a zero of $f(x)$. Since $|K| = p^n$, thus $K$ must be a splitting field of $f(x)$ over $\Z_p$. But by \myref{corollary-splitting-field-unique-up-to-isomorphism}, splitting fields are unique up to isomorphism, meaning $K \cong F$.
\end{proof}

\begin{exercise}\label{exercise-zeroes-of-polynomial-form-subfield}
    In \myref{thrm-finite-field-is-unique}, prove that $S$, the set of zeroes of $f(x)$ in $F$, is a subfield of $F$.
\end{exercise}

As there is only one field (up to isomorphism) that is of order $p^n$, we may unambiguously call such a field \text{the} field of order $p^n$.

\begin{definition}
    Let $p$ be a prime number and $n$ be a positive integer. The field of order $p^n$ is called the \textbf{Galois field of order $p^n$}\index{Galois field} and is denoted by $\GF{p^n}$.
\end{definition}
\begin{remark}
    We call $\GF{p^n}$ the Galois field of order $p^n$ in honour of Galois who first introduced finite fields back in 1830.
\end{remark}

% TODO: Continue

\section{Subfields of Finite Fields}
% TODO: Add

\newpage
\section{Problems}
% TODO: Add
