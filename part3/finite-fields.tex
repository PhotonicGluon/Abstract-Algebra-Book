\chapter{Finite Fields}
Finite fields appear in many applications of algebra (like in the NTRU cryptosystem). We already found out that $\Z_p$ is a field, and that we may construct fields of prime-power order via the use of irreducible polynomials. In this chapter, we discover one of the most surprising and important results relating to these fields.

\section{Classification and Structure of Finite Fields}
\'Evariste Galois first introduced finite fields in his proof of the insolvability of the quintic in 1830. Today, finite fields are immensely useful in many aspects of algebra and computing.

The most surprising and important result regarding finite fields is that there is a unique finite field (up to isomorphism) of order $p^n$ where $p$ is a prime. The existence of such a field was given by Galois and Gauss in the early 19th century, but it took until Dedekind in 1857 and Jordan in 1870 to produce a rigorous proof of the existence of such fields. The uniqueness of such a field was proved by Eliakim Hastings Moore in 1893. We state his theorem below.

\begin{theorem}\label{thrm-finite-field-is-unique}
    For each prime $p$ and every positive integer $n$, there exists a finite field of order $p^n$. Furthermore any field of order $p^n$ is isomorphic to the splitting field of $x^{p^n} - x$ over $\Z_p$.
\end{theorem}
\begin{proof}[Proof (see {\cite[Theorem 22.1]{gallian_2016}} and {\cite[Theorem 22.6]{judson_beezer_2022}})]
    Let $f(x) = x^{p^n} - x$ and let $F$ be the splitting field of $f(x)$ over $\Z_p$. We show that $|F| = p^n$. Since $f(x)$ has degree $p^n$ and splits over $F$, thus there must be $p^n$ zeroes of $f(x)$ in $F$. Also we know that each of these $p^n$ zeroes are simple (\myref{problem-(x^p^n-x)-only-has-simple-zeroes}), meaning that $f(x)$ has $p^n$ distinct zeroes in $F$. We note that \myref{exercise-zeroes-of-polynomial-form-subfield} (later) shows that the set of zeroes of $f(x)$ in $F$, which we will denote by $S$, form a subfield of $F$. Note $|S| = p^n$. But as $S$ contains all zeroes of $f(x)$ in $F$, we know that $f(x)$ splits over $S$. As $S \subseteq F$, thus $S = F$, meaning $|F| = p^n$.

    We now show uniqueness. Suppose $K$ is another field of order $p^n$. To show that $K \cong F$, we need to show that every element of $K$ is a zero of $f(x)$. Certainly $0 \in K$ is a zero of $f(x)$. Let $\alpha \in K$ be a non-zero element. Since the multiplicative group of $K$ has order $p^n - 1$, thus we must have $\alpha^{p^n-1} = 1$, which means $\alpha^{p^n} = \alpha$ and so $\alpha^{p^n}-\alpha = 0$. Thus any non-zero $\alpha \in K$ is a zero of $f(x)$. Since $|K| = p^n$, thus $K$ must be a splitting field of $f(x)$ over $\Z_p$. But by \myref{corollary-splitting-field-unique-up-to-isomorphism}, splitting fields are unique up to isomorphism, meaning $K \cong F$.
\end{proof}

\begin{exercise}\label{exercise-zeroes-of-polynomial-form-subfield}
    In \myref{thrm-finite-field-is-unique}, prove that $S$, the set of zeroes of $f(x)$ in $F$, is a subfield of $F$.
\end{exercise}

As there is only one field (up to isomorphism) that is of order $p^n$, we may unambiguously call such a field \text{the} field of order $p^n$.

\begin{definition}
    Let $p$ be a prime number and $n$ be a positive integer. The field of order $p^n$ is called the \textbf{Galois field of order $p^n$}\index{Galois field} and is denoted by $\GF{p^n}$.
\end{definition}
\begin{remark}
    We call $\GF{p^n}$ the Galois field of order $p^n$ in honour of Galois who first introduced finite fields back in 1830.
\end{remark}

Since there is only one finite field of a given order (up to isomorphism), it is easy to classify the additive and multiplicative group of such a field.

\begin{theorem}\label{thrm-structure-of-finite-field}
    Let $p$ be a prime and $n$ be a non-negative integer. The additive group of $\GF{p^n}$ is isomorphic to $\Cn{p}^n$ and the multiplicative group of $\GF{p^n}$ isomorphic to $\Cn{p^n-1}$ (and is, therefore, cyclic).
\end{theorem}
\begin{proof}
    Since $\GF{p^n}$ has characteristic $p$, the additive order of any element in $\GF{p^n}$ is at most $p$. Thus the cyclic subgroup generated by an arbitrary element in the additive group of $\GF{p^n}$ has at most $p$ elements; in fact there can only be subgroups of $p$ elements or 1 element (the trivial subgroup). Therefore the Fundamental Theorem of Finite Abelian Groups (\myref{thrm-fundamental-theorem-of-finite-abelian-groups}) tells us that
    \[
        \GF{p^n} \cong \underbrace{\Cn{p} \times \Cn{p} \times \cdots \times \Cn{p}}_{n\text{ times}} = \Cn{p}^n
    \]
    when viewed as a group under addition.

    Now note that by the Fundamental Theorem of Finite Abelian Groups again we see that the multiplicative group of $\GF{p^n}$, i.e. $\GF{p^n}^\ast$, is isomorphic to $\Cn{n_1}\times\Cn{n_2}\cdots\times\Cn{n_m}$ where $n_1, n_2, \dots, n_m$ are positive integers with a product of $p^n - 1$. We show that these integers are pairwise coprime via contradiction; suppose that for some $n_i$ and $n_j$ within that list that there is a $d > 1$ that divides both. In particular, a prime $q$ must divide both $n_i$ and $n_j$. By a corollary of Cauchy's Theorem (\myref{corollary-cauchy-theorem-with-subgroup}) there must be a subgroup of order $q$ in both $\Cn{n_i}$ and $\Cn{n_j}$, which means that $\GF{p^n}^\ast$ has two distinct subgroups of order $q$. Call these subgroups $H$ and $K$. Let us focus on $H$ for now. Since $H$ has order $q$, we must have $x^q = 1$ for all $x \in H$, meaning that $x^q - 1 = 0$ for all $x \in H$. We also see that $x^q - 1 = 0$ for all $x \in K$. Since $H$ and $K$ are of order $q$, thus $x^q - 1$ has more than $q$ zeroes in $\GF{p^n}$, contradicting the fact that a polynomial of degree $q$ has at most $q$ zeroes over a field (\myref{thrm-polynomial-of-degree-n-has-at-most-n-zeroes}). Therefore each of the positive integers $n_1, n_2, \dots, n_m$ are pairwise coprime. By \myref{thrm-Zm-cross-Zn-isomorphic-to-Zmn-condition} we therefore see that
    \[
        \Cn{n_1}\times\Cn{n_2}\cdots\times\Cn{n_m} \cong \Cn{n_1n_2\cdots n_m} = \Cn{p^n-1}
    \]
    and so $\GF{p^n}^\ast \cong \Cn{p^n-1}$, proving the second part of the theorem.
\end{proof}

With this theorem, we obtain the following aesthetically pleasing corollary.

\begin{corollary}\label{corollary-degree-of-galois-field-to-prime-power}
    $[\GF{p^n}: \GF{p}] = n$ for any prime $p$ and any positive integer $n$.
\end{corollary}
\begin{proof}
    Note that the additive group of $\GF{p^n}$ is isomorphic to $\Z_p^n$ by \myref{thrm-structure-of-finite-field}, and likewise the additive group of $\GF{p}$ is isomorphic to $\Z_p$. One sees that $\Z_p^n$ is a vector space over $\Z_p$ with the standard basis
    \[
        \left\{(\underbrace{1, 0, 0, \dots, 0, 0}_{n \text{ elements}}), (\underbrace{0, 1, 0, \dots, 0, 0}_{n \text{ elements}}), (\underbrace{0, 0, 1, \dots, 0, 0}_{n \text{ elements}}), \dots, (\underbrace{0, 0, 0, \dots, 1}_{n \text{ elements}})\right\}
    \]
    and so $\dim{\GF{p^n}} = n$ over $\GF{p}$, i.e. $[\GF{p^n}: \GF{p}] = n$.
\end{proof}

\begin{corollary}
    Let $a$ be a generator of $\GF{p^n}^\ast$. Then $a$ is algebraic over $\GF{p}$. Furthermore $a$ has degree $n$ over $\GF{p}$.
\end{corollary}
\begin{proof}
    Since $a$ is a generator of $\GF{p^n}^\ast$, thus the extension field $\GF{p}(a)$ is simply $\GF{p^n}$. Therefore one sees
    \[
        [\GF{p}(a): \GF{p}] = [\GF{p^n}:\GF{p}] = n
    \]
    by \myref{corollary-degree-of-galois-field-to-prime-power} and so $\GF{p}(a)$ is a finite extension over $\GF{p}$. Hence $\GF{p}(a)$ is an algebraic extension over $\GF{p}$ (\myref{thrm-finite-extension-is-algebraic}) and so $a$ is algebraic over $\GF{p}$ with degree $n$.
\end{proof}

Let us examine some finite fields in detail.
\begin{example}\label{example-GF16-analysis}
    Let us examine $\GF{16}$ in detail. In part of \myref{example-x^4+x+1-is-irreducible-over-Z2} we have shown that $x^4 + x + 1$ is irreducible over $\Z_2$. Therefore we know that $\princ{x^4 + x + 1}$ is a maximal ideal in $\Z_2[x]$ (\myref{thrm-irreducible-iff-principal-ideal-is-maximal}) and therefore $\Z_2/\princ{x^4+x+1}$ is a field (\myref{thrm-maximal-ideal-iff-quotient-ring-is-field}). In fact this is a field of order 16, which means that.
    \[
        \GF{16} \cong \Z_2/\princ{x^4+x+1}
    \]
    So we may think of $\GF{16}$ as the set
    \[
        F = \{a_0 + a_1\alpha + a_2\alpha^2 + a_3\alpha^3 \vert a_i \in \Z_2 \text{ and } \alpha^4 + \alpha + 1 = 0\}.
    \]
    Remembering that $\alpha^4 + \alpha + 1 = 0$ we add and multiply elements of $\GF{16}$ exactly as we add and multiply polynomials. The multiplicative group of $\GF{16}$ is isomorphic to $\Cn{15}$ with generator $\alpha$, meaning that we obtain the following conversions.
    \begin{multicols}{3}
        \begin{itemize}
            \item $\alpha^1 = \alpha$
            \item $\alpha^2 = \alpha^2$
            \item $\alpha^3 = \alpha^3$
            \item $\alpha^4 = \alpha + 1$
            \item $\alpha^5 = \alpha^2 + \alpha$
            \item $\alpha^6 = \alpha^3 + \alpha^2$
            \item $\alpha^7 = \alpha^3 + \alpha + 1$
            \item $\alpha^8 = \alpha^2 + 1$
            \item $\alpha^9 = \alpha^3 + \alpha$
            \item $\alpha^{10} = \alpha^2 + \alpha + 1$
            \item $\alpha^{11} = \alpha^3 + \alpha^2 + \alpha$
            \item $\alpha^{12} = \alpha^3 + \alpha^2 + \alpha + 1$
            \item $\alpha^{13} = \alpha^3 + \alpha^2 + 1$
            \item $\alpha^{14} = \alpha^3 + 1$
            \item $\alpha^{15} = 1$
        \end{itemize}
    \end{multicols}
    For example, we see
    \begin{align*}
        \alpha^{10} + \alpha^7 &= (\alpha^2 + \alpha + 1) + (\alpha^3 + \alpha + 1)\\
        &= \alpha^3 + \alpha^2 + 2\alpha + 2\\
        &= \alpha^3 + \alpha^2 & (\text{Evaluate coefficients in }\Z_2)\\
        &= \alpha^6.
    \end{align*}
    Also, we see
    \begin{align*}
        (\alpha^3 + \alpha^2 + 1)(\alpha^3 + \alpha^2 + \alpha + 1) &= \alpha^{13}\alpha^{12}\\
        &= \alpha^{25}\\
        &= \alpha^{15}\alpha^{10}\\
        &= \alpha^{10} & (\text{since }\alpha^{15} = 1)\\
        &= \alpha^2 + \alpha + 1.
    \end{align*}
\end{example}

\begin{example}
    One sees clearly that $x^3 + x^2 + 1$ has no zeroes in $\Z_2$ and so $x^3 + x^2 + 1$ is irreducible over $\Z_2$. Therefore one sees $K = \Z_2/\princ{x^3+x^2+1}$ is a field of 8 elements, i.e. $\GF{8} \cong K$. Note that $f(x)$ has a zero in $F$ by \myref{thrm-fundamental-theorem-of-field-theory}, say $\alpha$. Note that $\GF{8}^\ast$ has order 7, a prime, so $|\alpha|_\times = 7$. Hence we may think of $\GF{8}$ as being the set
    \begin{align*}
        F &= \{0, 1, \alpha, \alpha^2, \alpha^3, \alpha^4, \alpha^5, \alpha^6\}\\
        &= \{0, 1, \alpha, \alpha^2, \alpha^2 + 1, \alpha^2 + \alpha + 1, \alpha + 1, \alpha^2 + \alpha\}\\
        &= \{0, 1, \alpha, \alpha + 1, \alpha^2, \alpha^2 + 1, \alpha^2 + \alpha, \alpha^2 + \alpha + 1\}
    \end{align*}
    where
    \begin{multicols}{2}
        \begin{itemize}
            \item $\alpha^1 = \alpha$
            \item $\alpha^2 = \alpha^2$
            \item $\alpha^3 = \alpha^2 + 1$
            \item $\alpha^4 = \alpha^2 + \alpha + 1$
            \item $\alpha^5 = \alpha + 1$
            \item $\alpha^6 = \alpha^2 + \alpha$
            \item $\alpha^7 = 1$
        \end{itemize}
    \end{multicols}

    Let us find the other zeroes of $f(x)$ over $F$. We see that
    \begin{align*}
        f(\alpha^2) &= (\alpha^2)^3 + (\alpha^2)^2 + 1\\
        &= \alpha^6 + \alpha^4 + 1\\
        &= (\alpha^2 + \alpha) + (\alpha^2 + \alpha + 1) + 1\\
        &= 2\alpha^2 + 2\alpha + 2\\
        &= 0
    \end{align*}
    so $\alpha^2$ is another zero of $f(x)$. Next, we try $\alpha^3$ and we observe
    \begin{align*}
        f(\alpha^3) &= (\alpha^3)^3 + (\alpha^3)^2 + 1\\
        &= \alpha^9 + \alpha^6 + 1\\
        &= \alpha^2 + \alpha^6 + 1\\
        &= \alpha^2 + (\alpha^2 + \alpha) + 1\\
        &= 2\alpha^2 + \alpha + 1\\
        &\neq 0
    \end{align*}
    so $\alpha^3$ is \textit{not} a zero of $f(x)$. Now we check $\alpha^4$ and note
    \begin{align*}
        f(\alpha^4) &= (\alpha^4)^3 + (\alpha^4)^2 + 1\\
        &= \alpha^{12} + \alpha^8 + 1\\
        &= \alpha^5 + \alpha + 1\\
        &= (\alpha + 1) + \alpha + 1\\
        &= 2\alpha + 2\\
        &= 0
    \end{align*}
    so $\alpha^4$ is another zero of $f(x)$ in $F$. Since $f(x)$ has degree 3, this means that $f(x)$ has at most 3 zeroes over $F$; as we have found 3 zeroes that is it.

    Therefore
    \[
        f(x) = (x - \alpha)(x - \alpha^2)(x - \alpha^4) = (x + \alpha)(x + \alpha^2)(x + \alpha^4).
    \]
\end{example}

\begin{exercise}
    Find all the zeroes of $f(x) = x^4 + x + 1$ over the field $F$ described in \myref{example-GF16-analysis}, using $\alpha$ as one possible zero of $f(x)$ in $F$.
\end{exercise}

\section{Subfields of Finite Fields}
% TODO: Add

\newpage

\section{Problems}
% TODO: Add
