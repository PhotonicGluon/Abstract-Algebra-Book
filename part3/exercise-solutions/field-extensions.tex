\section{Extension Fields and Splitting Fields}
\begin{questions}
    \item We need to prove that $\phi$ is a well-defined ring isomorphism.
    \begin{itemize}
        \item \textbf{Well-defined}: Suppose $ax + b + \princ{x^2+1}, cx + d + \princ{x^2+1} \in F$ such that $ax + b + \princ{x^2+1} = cx + d + \princ{x^2+1}$. Then $(a - c)x + (b - d) + \princ{x^2 + 1} = 0 + \princ{x^2+1}$ by Coset Equality, which therefore means that $(a-c)x + (b-d) \in \princ{x^2+1}$. Now elements of $\princ{x^2+1}$ have degree of at least 2, or is the zero polynomial. Since $(a-c)x + (b-d)$ has degree less than 2, we must have $(a-c)x + (b-d) = 0$ which means $a - c = 0$ and $b - d = 0$. Hence $a = c$ and $b = d$, and so
        \begin{align*}
            \phi\left(ax + b + \princ{x^2+1}\right) &= b + ai\\
            &= d + ci\\
            &= \phi\left(cx + d + \princ{x^2+1}\right)
        \end{align*}
        which proves that $\phi$ is well-defined.

        \item \textbf{Homomorphism}. Let $ax + b + \princ{x^2+1}, cx + d + \princ{x^2+1} \in F$. Then
        \begin{align*}
            &\phi\left(\left(ax + b + \princ{x^2+1}\right) + \left(cx + d + \princ{x^2+1}\right)\right)\\
            &= \phi\left((a+c)x + (b+d) + \princ{x^2+1}\right)\\
            &= (b+d) + (a+c)i\\
            &= (b+ai) + (d+ci)\\
            &= \phi\left(ax + b + \princ{x^2+1}\right) + \phi\left(cx + d + \princ{x^2+1}\right)
        \end{align*}
        and
        \begin{align*}
            &\phi\left(\left(ax + b + \princ{x^2+1}\right)\left(cx + d + \princ{x^2+1}\right)\right)\\
            &= \phi\left((acx^2 + (ad + bc)x + bd) + \princ{x^2+1}\right)\\
            &= \phi\left(((ad + bc)x + (bd - ac)) + \princ{x^2+1}\right) & (\text{since } x^2 = -1 \text{ in } F)\\
            &= (bd - ac) + (ad + bc)i\\
            &= (b + ai)(d + ci)\\
            &= \phi\left(ax + b + \princ{x^2+1}\right)\phi\left(cx + d + \princ{x^2+1}\right)
        \end{align*}
        which shows that $\phi$ is a ring homomorphism.

        \item \textbf{Injective}: Suppose $ax + b + \princ{x^2+1}, cx + d + \princ{x^2+1} \in F$ such that
        \[
            \phi\left(ax + b + \princ{x^2+1}\right) = \phi\left(cx + d + \princ{x^2+1}\right).
        \]
        This means that $b + ai = d + ci$. Hence $b = d$ and $a = c$, which shows that $ax + b + \princ{x^2+1} = cx + d + \princ{x^2+1}$. Hence $\phi$ is injective.

        \item \textbf{Surjective}: Suppose $u + vi \in \C$. Then clearly
        \[
            \phi\left(vx + u + \princ{x^2+1}\right) = u + vi
        \]
        which means that $\phi$ is surjective.
    \end{itemize}
    Therefore $\phi$ is a well-defined ring isomorphism, proving that $F \cong \C$.

    \item By way of contradiction, suppose a ring $R$ has an subring that is isomorphic to $\Z_4$, and that $f(x)$ has a zero in such a ring. Let this zero be denoted $\alpha$, which means $2\alpha + 1 = 0$ in $R$. Note that this means
    \begin{align*}
        0 &= 2\times0\\
        &= 2(2\alpha + 1)\\
        &= 2(2\alpha) + 2\\
        &= (2 \times 2)\alpha + 2\\
        &= 0\alpha + 2\\
        &= 2
    \end{align*}
    but $0 \neq 2$ in $\Z_4$.

    \item We note that $x^4 + 3x^2 + 1 = (x^2+1)(x^2+2)$, and that $x^2+1$ and $x^2 + 2$ are irreducible over $\Q$. By the Fundamental Theorem of Field Theory (\myref{thrm-fundamental-theorem-of-field-theory}) we may choose
    \[
        \Q[x]/\princ{x^2+1} \quad\text{or}\quad \Q[x]/\princ{x^2+2}
    \]
    as the possible extension fields.
\end{questions}
