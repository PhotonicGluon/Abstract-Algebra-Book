\section{Extension Fields and Splitting Fields}
\begin{questions}
    \item We need to prove that $\phi$ is a well-defined ring isomorphism.
    \begin{itemize}
        \item \textbf{Well-defined}: Suppose $ax + b + \princ{x^2+1}, cx + d + \princ{x^2+1} \in F$ such that $ax + b + \princ{x^2+1} = cx + d + \princ{x^2+1}$. Then $(a - c)x + (b - d) + \princ{x^2 + 1} = 0 + \princ{x^2+1}$ by Coset Equality, which therefore means that $(a-c)x + (b-d) \in \princ{x^2+1}$. Now elements of $\princ{x^2+1}$ have degree of at least 2, or is the zero polynomial. Since $(a-c)x + (b-d)$ has degree less than 2, we must have $(a-c)x + (b-d) = 0$ which means $a - c = 0$ and $b - d = 0$. Hence $a = c$ and $b = d$, and so
        \begin{align*}
            \phi\left(ax + b + \princ{x^2+1}\right) &= b + ai\\
            &= d + ci\\
            &= \phi\left(cx + d + \princ{x^2+1}\right)
        \end{align*}
        which proves that $\phi$ is well-defined.

        \item \textbf{Homomorphism}. Let $ax + b + \princ{x^2+1}, cx + d + \princ{x^2+1} \in F$. Then
        \begin{align*}
            &\phi\left(\left(ax + b + \princ{x^2+1}\right) + \left(cx + d + \princ{x^2+1}\right)\right)\\
            &= \phi\left((a+c)x + (b+d) + \princ{x^2+1}\right)\\
            &= (b+d) + (a+c)i\\
            &= (b+ai) + (d+ci)\\
            &= \phi\left(ax + b + \princ{x^2+1}\right) + \phi\left(cx + d + \princ{x^2+1}\right)
        \end{align*}
        and
        \begin{align*}
            &\phi\left(\left(ax + b + \princ{x^2+1}\right)\left(cx + d + \princ{x^2+1}\right)\right)\\
            &= \phi\left((acx^2 + (ad + bc)x + bd) + \princ{x^2+1}\right)\\
            &= \phi\left(((ad + bc)x + (bd - ac)) + \princ{x^2+1}\right) & (\text{since } x^2 = -1 \text{ in } F)\\
            &= (bd - ac) + (ad + bc)i\\
            &= (b + ai)(d + ci)\\
            &= \phi\left(ax + b + \princ{x^2+1}\right)\phi\left(cx + d + \princ{x^2+1}\right)
        \end{align*}
        which shows that $\phi$ is a ring homomorphism.

        \item \textbf{Injective}: Suppose $ax + b + \princ{x^2+1}, cx + d + \princ{x^2+1} \in F$ such that
        \[
            \phi\left(ax + b + \princ{x^2+1}\right) = \phi\left(cx + d + \princ{x^2+1}\right).
        \]
        This means that $b + ai = d + ci$. Hence $b = d$ and $a = c$, which shows that $ax + b + \princ{x^2+1} = cx + d + \princ{x^2+1}$. Hence $\phi$ is injective.

        \item \textbf{Surjective}: Suppose $u + vi \in \C$. Then clearly
        \[
            \phi\left(vx + u + \princ{x^2+1}\right) = u + vi
        \]
        which means that $\phi$ is surjective.
    \end{itemize}
    Therefore $\phi$ is a well-defined ring isomorphism, proving that $F \cong \C$.

    \item By way of contradiction, suppose a ring $R$ has an subring that is isomorphic to $\Z_4$, and that $f(x)$ has a zero in such a ring. Let this zero be denoted $\alpha$, which means $2\alpha + 1 = 0$ in $R$. Note that this means
    \begin{align*}
        0 &= 2\times0\\
        &= 2(2\alpha + 1)\\
        &= 2(2\alpha) + 2\\
        &= (2 \times 2)\alpha + 2\\
        &= 0\alpha + 2\\
        &= 2
    \end{align*}
    but $0 \neq 2$ in $\Z_4$.

    \item We note that $x^4 + 3x^2 + 1 = (x^2+1)(x^2+2)$, and that $x^2+1$ and $x^2 + 2$ are irreducible over $\Q$. By the Fundamental Theorem of Field Theory (\myref{thrm-fundamental-theorem-of-field-theory}) we may choose
    \[
        \Q[x]/\princ{x^2+1} \quad\text{or}\quad \Q[x]/\princ{x^2+2}
    \]
    as the possible extension fields.

    \item For brevity let $K = F(a_1, a_2, \dots, a_{n-1})(a_n)$. Note $K$ contains $F(a_1, a_2, \dots, a_{n-1})$ and the element $a_n$, and that $F(a_1, a_2, \dots, a_{n-1})$ contains the field $F$ and the elements $a_1, a_2, \dots, a_{n-1}$. Thus $K$ contains $F$ and the elements $a_1, a_2, \dots, a_{n-1}, a_n$, which means that $K$ is included in the intersection that generates $F(a_1, a_2, \dots, a_n)$, so $K \subseteq F(a_1, a_2, \dots, a_n)$. But $F(a_1, a_2, \dots, a_n)$ is the smallest subfield of $E$ that contains $F$ and the elements $a_1, a_2, \dots, a_n$. Therefore $K = F(a_1, a_2, \dots, a_n)$ as required.
    
    \item \begin{partquestions}{\roman*}
        \item One sees that
        \[
            f(x) = (x-\sqrt2)(x+\sqrt2)(x-\sqrt3)(x+\sqrt3)
        \]
        and so a splitting field of $f(x)$ over $\Q$ is
        \begin{align*}
            &\Q(\sqrt2, -\sqrt2, \sqrt3, -\sqrt3)\\
            &=\Q(\sqrt2, \sqrt3)\\
            &=\Q(\sqrt2)(\sqrt3)\\
            &\cong \Q[\sqrt2][\sqrt3]. & (\myref{thrm-simple-extension-isomorphism})
        \end{align*}

        \item Reducing $p(x)$ modulo 5 yields $\bar{p}(x) = x^4 + 1$. One sees then that
        \begin{itemize}
            \item $\bar{p}(0) = 1 \neq 0$;
            \item $\bar{p}(1) = 2 \neq 0$;
            \item $\bar{p}(2) = 17 = 2 \neq 0$;
            \item $\bar{p}(3) = 82 = 2 \neq 0$; and
            \item $\bar{p}(4) = 257 = 2 \neq 0$.
        \end{itemize}
        Therefore $p(x)$ is irreducible over $\Q$ by Mod 5 Irreducibility Test (\myref{thrm-mod-p-irreducibility-test}). One also sees that
        \begin{align*}
            &p(\sqrt2 + \sqrt3)\\
            &= (\sqrt2 + \sqrt3)^4 - 10(\sqrt2 + \sqrt3)^2 + 1\\
            &= \left((\sqrt2)^4 + 4(\sqrt2)^3(\sqrt3) + 6(\sqrt2)^2(\sqrt3)^2 + 4(\sqrt2)(\sqrt3)^3 + (\sqrt3)^4\right)\\
            &\quad\quad- 10\left((\sqrt2)^2 + 2\sqrt2\sqrt3 + (\sqrt3)^2\right) + 1\\
            &=\left(49 + 20\sqrt6\right) - 10\left(5 + 2\sqrt6\right) + 1\\
            &= 0
        \end{align*}
        so $\sqrt2 + \sqrt3$ is indeed a zero of $p(x)$ in $\R$.

        \item We work backwards. Since $p(x) = x^4 - 10x + 1$ is an irreducible polynomial of degree 4, therefore $\Q[\sqrt2 + \sqrt3]$ has 4 basis vectors. Therefore
        \begin{align*}
            &\Q(\sqrt2 + \sqrt3)\\
            &\cong \Q[\sqrt2 + \sqrt3]\\
            &= \{s + t(\sqrt2 + \sqrt3) + u(\sqrt2 + \sqrt3)^2 + v(\sqrt2 + \sqrt3)^3 \vert s,t,u,v \in \Q\}\\
            &= \{(s + 5u) + (t + 11v)\sqrt2 + (t + 9v)\sqrt3 + 2u\sqrt6 \vert s, t, u, v \in \Q\}\\
            &= \{a + b\sqrt2 + c\sqrt3 + d\sqrt6 \vert a, b, c, d \in \Q\}\\
            &=\{(a+b\sqrt2) + (c+d\sqrt2)\sqrt3 \vert a, b, c, d \in \Q\}\\
            &=\{u + v\sqrt3 \vert u, v \in \Q[\sqrt2]\}\\
            &= \Q[\sqrt2][\sqrt3]
        \end{align*}
        and so there is a splitting field of $f(x)$ over $\Q$ that is isomorphic to $\Q(\sqrt2 + \sqrt3)$.
    \end{partquestions}
\end{questions}
