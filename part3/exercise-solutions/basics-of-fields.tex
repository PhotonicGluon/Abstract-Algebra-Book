\section{Basics of Fields}
\begin{questions}
    \item \begin{partquestions}{\alph*}
        \item Clearly $0 = 0 - 0$ since $x - x = 0$ for all elements $x$ in $F$, including $x = 0$. But since 0 is the additive identity, this means $0 + x = x$ for all $x$, including $x = -0$. Hence $0 = -0$.
        \item Clearly $1 = 1\times 1^{-1}$ since $xx^{-1} = 1$ for all $x \in F^\ast$, including $x = 1$. But since 1 is the multiplicative identity, this means $1x = x$ for all $x$, including $x = 1^{-1}$. So $1 = 1^{-1}$.
    \end{partquestions}

    \item We first note that $T^\ast \neq \emptyset$ since $1 \in S$ (which means $1 \in S^\ast$) and so $1 = 1 \times 1^{-1} \in T^\ast$.
    
    Next, for two elements $ab^{-1}, cd^{-1} \in T$ we see
    \begin{align*}
        ab^{-1} - cd^{-1} &= (ab^{-1})(dd^{-1}) - (cd^{-1})(bb^{-1})\\
        &= adb^{-1}d^{-1} - bcb^{-1}d^{-1}\\
        &= (ad-bc)(bd)^{-1}\\
        \in T
    \end{align*}
    because $bd \neq 0$. Finally, notice for $ab^{-1}, cd^{-1} \in T$ and $cd^{-1} \neq 0$ we have
    \begin{align*}
        (ab^{-1})(cd^{-1})^{-1} &= ab^{-1}c^{-1}d\\
        &= (ad)(bc)^{-1}
        &\in T
    \end{align*}
    which shows that $T$ is a subfield of $F$ by the subfield test (\myref{thrm-subfield-test}).
\end{questions}
