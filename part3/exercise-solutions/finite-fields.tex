\section{Finite Fields}
\begin{questions}
    \item Note for any $\alpha \in S$, because $f(\alpha) = \alpha^{p^n} - \alpha = 0$, thus $\alpha^{p^n} = \alpha$. This fact will be useful later.
    \begin{itemize}
        \item Clearly $f(1) = 1^{p^n} - 1 = 0$ so $1 \in S^\ast$.
        
        \item Now let $\alpha,\beta \in S$. Then we see
        \begin{align*}
            f(\alpha - \beta) &= (\alpha-\beta)^{p^n} - (\alpha - \beta)\\
            &= \alpha^{p^n} + (-1)^{p^n}\beta^{p^n} - \alpha + \beta & (\myref{prop-freshman-dream})\\
            &= (\alpha^{p^n} - \alpha) + ((-1)^{p^n}\beta^{p^n} + \beta)\\
            &= 0 + (-1)^{p^n}\beta^{p^n} + \beta & (\text{as } \alpha \in S)\\
            &= (-1)^{p^n}\beta^{p^n} + \beta.
        \end{align*}
        We further split into two cases.
        \begin{itemize}
            \item If $p$ is odd then $(-1)^{p^n}\beta^{p^n} + \beta = -(\beta^{p^n} - \beta) = 0$.
            \item If $p$ is even then $p = 2$. So $(-1)^{p^n}\beta^{p^n} + \beta = \beta^{2^n} + \beta = 2\beta = 0$ since $\Char{F} = 2$.
        \end{itemize}
        In either case, we see that $f(\alpha-\beta) = 0$, meaning $\alpha - \beta \in S$.

        \item Finally, for $\alpha \in S$ and $\beta \in S^\ast$ we see 
        \begin{align*}
            f(\alpha\beta^{-1}) &= (\alpha\beta^{-1})^{p^n} - (\alpha\beta^{-1})\\
            &= \alpha^{p^n}\left(\beta^{-1}\right)^{p^n} - \alpha\beta^{-1}\\
            &= \alpha^{p^n}\left(\beta^{p^n}\right)^{-1} - \alpha\beta^{-1}\\
            &= \alpha\beta^{-1} - \alpha\beta^{-1}\\
            &= 0
        \end{align*}
        and so $\alpha\beta^{-1} \in S$.
    \end{itemize}
    Therefore by the subfield test (\myref{thrm-subfield-test}) we see that $S$, the set of zeroes of $f(x)$ in $F$, is a subfield of $F$.
\end{questions}
