\section{Vector Spaces}
\begin{questions}
    \item We first need to prove the four group axioms.
    \begin{itemize}
        \item \textbf{Closure}: We clearly see
        \[
            (a_1, a_2, \dots, a_n) + (b_1, b_2, \dots, b_n) = (a_1 + b_1, a_2 + b_2, \dots, a_n + b_n) \in \R^n
        \]
        so $\R^n$ is closed under vector addition.
        
        \item \textbf{Associativity}: Let $(a_1, a_2, \dots, a_n), (b_1, b_2, \dots, b_n), (c_1, c_2, \dots, c_n) \in \R^n$. Note
        \begin{align*}
            &(a_1, a_2, \dots, a_n) + \left((b_1, b_2, \dots, b_n) + (c_1, c_2, \dots, c_n)\right)\\
            &= (a_1, a_2, \dots, a_n) + (b_1 + c_1, b_2 + c_2, \dots, b_n + c_n)\\
            &= (a_1 + (b_1 + c_1), a_2 + (b_2 + c_2), \dots, a_n + (b_n + c_n))\\
            &= ((a_1 + b_1) + c_1, (a_2 + b_2) + c_2, \dots, (a_n + b_n) + c_n)\\
            &= (a_1 + b_1, a_2 + b_2, \dots, a_n + b_n) + (c_1, c_2, \dots, c_n)\\
            &= \left((a_1, a_2, \dots, a_n) + (b_1, b_2, \dots, b_n)\right) + (c_1, c_2, \dots, c_n)
        \end{align*}
        so vector addition is associative.
        
        \item \textbf{Identity}: One sees that $(0, 0, \dots, 0) \in \R^n$ is the identity.
        
        \item \textbf{Inverse}: One sees that $(-a_1, -a_2, \dots, -a_n) \in \R^n$ is the inverse of the element $(a_1, a_2, \dots, a_n)$.
    \end{itemize}

    Thus $(\R^n, +)$ is a group. Also, for any $(a_1, a_2, \dots, a_n), (b_1, b_2, \dots, b_n) \in \R^n$ we see
    \begin{align*}
        (a_1, a_2, \dots, a_n) + (b_1, b_2, \dots, b_n) &= (a_1 + b_1, a_2 + b_2, \dots, a_n + b_n)\\
        &= (b_1+a_1, b_2+a_2, \dots, b_n+a_n)\\
        &= (b_1, b_2, \dots, b_n) + (a_1, a_2, \dots, a_n)
    \end{align*}
    so vector addition is also commutative. Therefore $(\R^n, +)$ is an abelian group.

    \item No. Note $x^5 + 1 \in W$ and $-x^5 + 1 \in W$ but $(x^5+1) + (-x^5+1) = 2 \notin W$ since 2 has a degree of 0.
    
    \item Note
    \[
        a\textbf{0} = a(\textbf{0} + \textbf{0}) = a\textbf{0} + a\textbf{0}
    \]
    by \textbf{Distributivity-Addition}. Then, by adding $-(a\textbf{0})$ on both sides we see
    \begin{align*}
        a\textbf{0} + (-(a\textbf{0})) &= (a\textbf{0} + a\textbf{0}) + (-(a\textbf{0}))\\
        &= a\textbf{0} + (a\textbf{0} + (-(a\textbf{0}))) & (\textbf{Addition-Abelian-Associativity})\\
    \end{align*}
    which means
    \[
        \textbf{0} = a\textbf{0} + \textbf{0},
    \]
    by \textbf{Addition-Abelian-Inverses}, and hence $a\textbf{0} = \textbf{0}$, by \textbf{Addition-Abelian-Identity}.
\end{questions}
