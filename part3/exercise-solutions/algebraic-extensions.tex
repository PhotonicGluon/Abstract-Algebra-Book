\section{Algebraic Extensions}
\begin{questions}
    \item It is not transcendental, i.e. it is algebraic. Let $\alpha = \sqrt{2-\sqrt{i}}$; one sees $\alpha^2 = 2 - \sqrt{i}$. Thus $\alpha^2 - 2 = -\sqrt{i}$ and so $(\alpha^2-2)^2 = i$. Note that $(\alpha^2 - 2)^2 = \alpha^4 - 4\alpha^2 + 4$ and so $\alpha^4 - 4\alpha^2 + 4 = i$. Squaring one more time we see
    \[
        x^8 - 8x^6 + 24x^4 - 32x^2 + 16 = i^2 = -1
    \]
    and so $\alpha$ is a zero of the polynomial $x^8 - 8x^6 + 24x^4 - 32x^2 + 17 \in \Q$. Therefore $\alpha$ is algebraic.

    \item \begin{partquestions}{\alph*}
        \item \begin{partquestions}{\roman*}
            \item We know by \myref{thrm-multiplying-finite-extension-degrees} that
            \begin{align*}
                [\Q(\sqrt2, \sqrt[3]3): \Q] &= [\Q(\sqrt2, \sqrt[3]3): \Q(\sqrt2)][\Q(\sqrt2):\Q]\\
                &= [\Q(\sqrt2, \sqrt[3]3): \Q(\sqrt[3]3)][\Q(\sqrt[3]3):\Q].
            \end{align*}
            Since $[\Q(\sqrt2):\Q] = 2$ and $[\Q(\sqrt[3]3):\Q] = 3$, thus $[\Q(\sqrt2, \sqrt[3]3): \Q]$ is a multiple of $\lcm(2, 3) = 6$. On the other hand, note that $x^3 - 3 \in \Q(\sqrt2)[x]$ has a zero of $\sqrt[3]3$, so $[\Q(\sqrt2, \sqrt[3]3): \Q(\sqrt2)]$ is at most 3. Hence $[\Q(\sqrt2, \sqrt[3]3): \Q] = [\Q(\sqrt2, \sqrt[3]3): \Q(\sqrt2)][\Q(\sqrt2):\Q] \leq 3 \times 2 = 6$. Hence $[\Q(\sqrt2, \sqrt[3]3): \Q] = 6$.
            
            \item Note that $\sqrt[6]{72} = \sqrt2 \times \sqrt[3]3$, so $\sqrt[6]{72} \in \Q(\sqrt2, \sqrt[3]3)$. Consequently, we see that $\Q(\sqrt2, \sqrt[3]3, \sqrt[6]{72}) = \Q(\sqrt2, \sqrt[3]3)$ and so $[\Q(\sqrt2, \sqrt[3]3, \sqrt[6]{72}):\Q] = [\Q(\sqrt2, \sqrt[3]3): \Q] = 6$.
        \end{partquestions}
        
        \item \begin{partquestions}{\roman*}
            \item Let $\bar{p}(x) = x^5 + 2x + 1 \in \Z_3[x]$. Note
            \begin{itemize}
                \item $\bar{p}(0) = 1 \neq 0$;
                \item $\bar{p}(1) = 4 = 1 \neq 0$; and
                \item $\bar{p}(2) = 37 = 1 \neq 0$,
            \end{itemize}
            so $\bar{p}(x)$ has no zeroes in $\Z_3$. Thus $p(x)$ is irreducible over $\Q$ by Mod 3 Irreducibility test (\myref{thrm-mod-p-irreducibility-test}).
            
            \item Suppose otherwise, that $\sqrt2 \in \Q(\alpha)$. So $\Q \subset \Q(\sqrt2) \subseteq \Q(\alpha)$. Thus we see
            \[
                [\Q(\alpha): \Q] = [\Q(\alpha):\Q(sqrt2)][\Q(\sqrt2):\Q].
            \]
            But $[\Q(\alpha): \Q] = 5$ (since $\alpha$ is a zero of the irreducible polynomial $p(x)$) and $[\Q(\sqrt2):\Q] = 2$, so we see that 2 divides 5, a contradiction. Therefore $\sqrt2 \notin \Q(\alpha)$.
        \end{partquestions}
    \end{partquestions}

    \item \begin{partquestions}{\roman*}
        \item One clearly sees $\Q(\sqrt3 + \sqrt5) \subseteq \Q(\sqrt3, \sqrt5)$ as $\sqrt3 + \sqrt5 \in \Q(\sqrt3, \sqrt5)$. Now observe
        \begin{align*}
            \left(\sqrt3+\sqrt5\right)^{-1} &= \frac{\sqrt3-\sqrt5}{\left(\sqrt3+\sqrt5\right)\left(\sqrt3-\sqrt5\right)}\\
            &= -\frac12\left(\sqrt3-\sqrt5\right)\\
        \end{align*}
        so $\sqrt3-\sqrt5 \in \Q(\sqrt3 + \sqrt5)$. It follows then $\frac12\left(\sqrt3 + \sqrt5\right) + \frac12\left(\sqrt3 - \sqrt5\right) = \sqrt3$ and $\frac12\left(\sqrt3 + \sqrt5\right) - \frac12\left(\sqrt3 - \sqrt5\right) = \sqrt5$ and so $\sqrt3,\sqrt5\in\Q(\sqrt3+\sqrt5)$. Hence $\Q(\sqrt3,\sqrt5)\subseteq\Q(\sqrt3+\sqrt5)$, and therefore $\Q(\sqrt3, \sqrt5) = \Q(\sqrt3 + \sqrt5)$.

        \item We see
        \[
            [\Q(\sqrt3, \sqrt5): \Q] = \underbrace{[\Q(\sqrt3, \sqrt5): \Q(\sqrt3)]}_2\underbrace{[\Q(\sqrt3):\Q]}_2 = 4
        \]
        by \myref{thrm-multiplying-finite-extension-degrees}, which means that the minimal polynomial of $\sqrt3+\sqrt5$ has degree 4.
    \end{partquestions}
\end{questions}
