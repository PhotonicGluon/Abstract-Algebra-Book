\chapter{Geometric Constructions}
Geometric constructions were a favourite of the ancient Greeks. They were particularly interested in constructions that could be created with just a compass and an unmarked straightedge. In this chapter, we explore how field theory can be used to answer ancient questions about some geometric constructions.

\section{Basic Geometry}
% TODO: Use Euclid?

To ensure a level playing field, we state some foundational terminology and results from geometry.

\subsection{Basic Definitions}
\begin{definition}
    An \textbf{angle}\index{angle} is the figure formed by two sides and shares a common endpoint. An angle formed by the sides $AB$ and $BC$ with common endpoint $B$ is denoted $\angle ABC$.
\end{definition}

\begin{definition}
    A \textbf{right angle}\index{angle!right} is an angle which measures $\frac\pi2 = 90^\circ$.
\end{definition}

In this chapter, we elect to use radians instead of degrees to measure angles.

\subsection{Polygons}
\begin{definition}
    A \textbf{polygon}\index{polygon} is a plane figure made up of \textbf{sides}\index{polygon!side} connected to form a closed figure. An $n$-sided polygon is called a \textbf{$n$-gon}\index{$n$-gon}.
\end{definition}

\begin{definition}
    A \textbf{triangle}\index{triangle} is a 3-gon.
    \begin{itemize}
        \item An \textbf{equilateral triangle}\index{triangle!equilateral} is a triangle where all 3 sides are equal in length.
        \item An \textbf{isosceles triangle}\index{triangle!isosceles} is a triangle where 2 of the sides are equal in length.
        \item A \textbf{scalene triangle}\index{triangle!scalene} is a triangle where all 3 sides are of different length.
    \end{itemize}
    A triangle with endpoints $A$, $B$, and $C$, read in anticlockwise order, is denoted $\triangle ABC$.
\end{definition}

\begin{definition}
    A \textbf{right-angled triangle}\index{triangle!right-angled} is a triangle where one of its angles is a right angle.
\end{definition}

\begin{definition}
    The longest side of a right-angled triangle is called its \textbf{hypotenuse}\index{triangle!right-angled!hypotenuse}.
\end{definition}

\begin{theorem}[Pythagoras]
    For a right-angled triangle with legs $a$ and $b$, the hypotenuse has length $\sqrt{a^2 + b^2}$.
\end{theorem}

Of special interest to us when we are dealing with geometric constructions are that of similar triangles.

\begin{definition}
    Two triangles $\triangle ABC$ and $\triangle A'B'C'$ are similar if and only if corresponding angles are equal in measure. That is,
    \begin{itemize}
        \item $\angle ABC = \angle A'B'C'$;
        \item $\angle BAC = \angle B'A'C'$; and
        \item $\angle BCA = \angle B'C'A'$.
    \end{itemize}
\end{definition}

\section{Basic Trigonometry}
Before we can introduce constructable numbers, we need to introduce the basics of trigonometry. Specifically, the sine, cosine, and tangent functions.

\begin{definition}
    Let $\triangle AOB$ be a right-angled triangle with hypotenuse $OA$ of length 1 and a right angle of $\angle OBA$. Let $\angle AOB = \theta$ where $\theta \in [0, \frac\pi2]$.
    \begin{itemize}
        \item The \textbf{sine}\index{sine} of $\theta$, denoted $\sin\theta$, is the length of the side $AB$.
        \item The \textbf{cosine}\index{cosine} of $\theta$, denoted $\cos\theta$, is the length of the side $OB$.
        \item The \textbf{tangent}\index{tangent} of $\theta$, denoted $\tan\theta$, is $\frac{\sin\theta}{\cos\theta}$.
    \end{itemize}
\end{definition}

We note some special values of sine, cosine, and tangent for $\theta \in [0, \frac\pi2]$.

\begin{table}[H]
    \centering
    \begin{tabular}{|l|l|l|l|}
        \hline
        $\boldsymbol{\theta}$ & $\boldsymbol{\sin\theta}$ & $\boldsymbol{\cos\theta}$ & $\boldsymbol{\tan\theta}$ \\ \hline
        $\boldsymbol{0}$ & 0 & 1 & 0 \\ \hline
        $\boldsymbol{\frac\pi6}$ & $\frac12$ & $\frac{\sqrt3}2$ & $\frac{\sqrt3}3$ \\ \hline
        $\boldsymbol{\frac\pi4}$ & $\frac{\sqrt2}2$ & $\frac{\sqrt2}2$ & 1 \\ \hline
        $\boldsymbol{\frac\pi3}$ & $\frac{\sqrt3}2$ & $\frac12$ & $\sqrt3$ \\ \hline
        $\boldsymbol{\frac\pi2}$ & 1 & 0 & Undefined \\ \hline
    \end{tabular}
    \caption{Values of $\sin\theta$, $\cos\theta$, and $\tan\theta$ for $\theta \in [0, \frac\pi2]$}
\end{table}

It is natural to extend the definition of sine, cosine, and tangent for values of $\theta$ above $\frac\pi2$ by considering the unit circle.

\begin{definition}
    Let $\theta \in [0, 2\pi)$. Suppose we have a circle of radius 1 centred at a point $O$. Draw a line $OA$ of length 1 that is rotated $\theta$ anticlockwise about $O$. Then $\sin\theta$ is the $y$-coordinate of the point $A$ and  $\cos\theta$ is the $x$-coordinate of the point $A$.
\end{definition}

A diagram would certainly aid this definition.

\begin{figure}[H]
    \centering
    \pdfteximg{0.475\textwidth}{part3/images/geometric-constructions/trigonometry-quadrant-1.pdf_tex}
    \pdfteximg{0.475\textwidth}{part3/images/geometric-constructions/trigonometry-quadrant-2.pdf_tex}
    \pdfteximg{0.475\textwidth}{part3/images/geometric-constructions/trigonometry-quadrant-3.pdf_tex}
    \pdfteximg{0.475\textwidth}{part3/images/geometric-constructions/trigonometry-quadrant-4.pdf_tex}
    \caption{Sine and Cosine In Different Quadrants}
\end{figure}

In particular, we observe the following.
\begin{table}[H]
    \centering
    \begin{tabular}{|l|l|l|l|}
        \hline
        $\boldsymbol{\theta}$ & $\boldsymbol{\sin\theta}$ & $\boldsymbol{\cos\theta}$ & $\boldsymbol{\tan\theta}$ \\ \hline
        $\boldsymbol{[0,\frac\pi2)}$ & $\sin(\theta)$ & $-\cos(\theta)$ & $-\tan(\theta)$ \\ \hline
        $\boldsymbol{[\frac\pi2, \pi)}$ & $\sin(\pi-\theta)$ & $-\cos(\pi-\theta)$ & $-\tan(\pi-\theta)$ \\ \hline
        $\boldsymbol{[\pi,\frac{3\pi}2)}$ & $-\sin(\theta-\pi)$ & $-\cos(\theta-\pi)$ & $\tan(\theta-\pi)$ \\ \hline
        $\boldsymbol{[\frac{3\pi}2, 2\pi)}$ & $-\sin(2\pi-\theta)$ & $\cos(2\pi-\theta)$ & $-\tan(2\pi-\theta)$ \\ \hline
    \end{tabular}
    \caption{Values of Trigonometric Functions in Different Intervals of $\theta$}
\end{table}

\begin{example}
    We note that $\frac{5\pi}6 \in [\frac\pi2, \pi)$. Therefore one sees that $\tan(\frac{5\pi}{6}) = -\tan(\pi-\frac{5\pi}{6}) = -\tan(\frac16\pi) = -\frac{\sqrt3}3$.
\end{example}

\begin{example}
    As $\frac{5\pi}4 \in [\pi, \frac{3\pi}2)$, thus we see $\cos(\frac{5\pi}4) = -\cos(\frac{5\pi}4-\pi) = -\cos(\frac14\pi) = -\frac{\sqrt2}2$.
\end{example}

\begin{example}
    One sees that $\sin(\frac{11\pi}{6}) = -\sin(2\pi-\frac{11\pi}{6}) = -\sin(\frac16\pi) = -\frac12$ because $\frac{11\pi}{6} \in [\frac{3\pi}2, 2\pi)$.
\end{example}

If $\theta$ is outside the interval $[0, 2\pi)$ then we will consult the following definition.

\begin{definition}
    For any real number $\theta$ define $\sin\theta = \sin(\theta + 2\pi)$, $\cos\theta = \cos(\theta + 2\pi)$, and $\tan\theta = \tan(\theta + 2\pi)$.
\end{definition}

% TODO: Continue

\section{Constructable Numbers}
% TODO: Add

\section{Impossibility of Constructions}
% TODO: Add

\subsection{Doubling the Square}
% TODO: Add

\subsection{Squaring the Circle}
% TODO: Add

\subsection{Trisecting an Angle}
% TODO: Add

\subsection{Constructing a Heptagon}
% TODO: Add
