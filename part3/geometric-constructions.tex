\chapter{Geometric Constructions}
Geometric constructions were a favourite of the ancient Greeks. They were particularly interested in constructions that could be created with just a compass and an unmarked straightedge. In this chapter, we explore how field theory can be used to answer ancient questions about some geometric constructions.

\section{Basic Geometry}
% TODO: Use Euclid?

To ensure a level playing field, we state some foundational terminology and results from geometry.

\subsection{Basic Definitions}
\begin{definition}
    An \textbf{angle}\index{angle} is the figure formed by two sides and shares a common endpoint. An angle formed by the sides $AB$ and $BC$ with common endpoint $B$ is denoted $\angle ABC$.
\end{definition}

\begin{definition}
    A \textbf{right angle}\index{angle!right} is an angle which measures $\frac\pi2 = 90^\circ$.
\end{definition}

In this chapter, we elect to use radians instead of degrees to measure angles.

\subsection{Polygons}
\begin{definition}
    A \textbf{polygon}\index{polygon} is a plane figure made up of \textbf{sides}\index{polygon!side} connected to form a closed figure. An $n$-sided polygon is called a \textbf{$n$-gon}\index{$n$-gon}.
\end{definition}

\begin{definition}
    A \textbf{triangle}\index{triangle} is a 3-gon.
    \begin{itemize}
        \item An \textbf{equilateral triangle}\index{triangle!equilateral} is a triangle where all 3 sides are equal in length.
        \item An \textbf{isosceles triangle}\index{triangle!isosceles} is a triangle where 2 of the sides are equal in length.
        \item A \textbf{scalene triangle}\index{triangle!scalene} is a triangle where all 3 sides are of different length.
    \end{itemize}
    A triangle with endpoints $A$, $B$, and $C$, read in anticlockwise order, is denoted $\triangle ABC$.
\end{definition}

\begin{definition}
    A \textbf{right-angled triangle}\index{triangle!right-angled} is a triangle where one of its angles is a right angle.
\end{definition}

\begin{definition}
    The longest side of a right-angled triangle is called its \textbf{hypotenuse}\index{triangle!right-angled!hypotenuse}.
\end{definition}

\begin{theorem}[Pythagoras]
    For a right-angled triangle with legs $a$ and $b$, the hypotenuse has length $\sqrt{a^2 + b^2}$.
\end{theorem}

Of special interest to us when we are dealing with geometric constructions are that of similar triangles.

\begin{definition}
    Two triangles $\triangle ABC$ and $\triangle A'B'C'$ are similar if and only if corresponding angles are equal in measure. That is,
    \begin{itemize}
        \item $\angle ABC = \angle A'B'C'$;
        \item $\angle BAC = \angle B'A'C'$; and
        \item $\angle BCA = \angle B'C'A'$.
    \end{itemize}
\end{definition}


\section{Constructable Numbers}
% TODO: Add

\section{Impossibility of Constructions}
% TODO: Add

\subsection{Doubling the Square}
% TODO: Add

\subsection{Squaring the Circle}
% TODO: Add

\subsection{Trisecting an Angle}
% TODO: Add

\subsection{Constructing a Heptagon}
% TODO: Add
