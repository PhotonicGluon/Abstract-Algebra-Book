%=========== Custom Commands =============
\newcommand{\constructionproof}[6][0.7\textwidth]{
    \vspace{0.1cm}
    \begin{proof}[Construction]
        \renewcommand{\qedsymbol}{}
        \renewcommand\windowpagestuff{#3}
        \opencutright
        \begin{cutout}{0}{#1}{0pt}{#2}
                #4
        \end{cutout}
    \end{proof}
    \textbf{Claim}. {#5}
    \begin{proof}
        #6
    \end{proof}
}

\newcommand{\Con}{\mathfrak{C}}  % Field of all constructible numbers
\newcommand{\con}{\mathcal{C}}  % A field of constructible numbers

%=========================================
\chapter{Geometric Constructions}
Geometric constructions were a favourite of the ancient Greeks. They were particularly interested in constructions that could be created with just a compass and an unmarked straightedge. In this chapter, we explore how field theory can be used to answer ancient questions about some geometric constructions.

\section{Basic Geometry}
To ensure a level playing field, we state some foundational terminology and results from geometry. We will not prove the results here.

\subsection{Angles}
\begin{definition}
    An \textbf{angle}\index{angle} is the figure formed by two sides and shares a common endpoint. An angle formed by the sides $AB$ and $BC$ with common endpoint $B$ is denoted $\angle ABC$.
\end{definition}

\begin{definition}
    The total sum of angles about a point add up to $2\pi = 360^\circ$.
\end{definition}

\begin{definition}
    A \textbf{right angle}\index{angle!right} is an angle which measures $\frac\pi2 = 90^\circ$.
\end{definition}

In this chapter, we elect to use radians instead of degrees to measure angles.

\subsection{Lines}
\begin{definition}
    A \textbf{line}\index{line} is an infinitely long object with no width, depth, or curvature. A \textbf{line segment}\index{line!segment} is part of a line of finite length.
\end{definition}

\begin{remark}
    In this part, we may abuse the definition of lines and line segments and call line segments ``lines''.
\end{remark}

\begin{definition}
    Two lines are \textbf{perpendicular}\index{line!perpendicular} to each other when, produced indefinitely in both directions, they meet each other at a right angle.
\end{definition}

\begin{definition}
    Two lines are \textbf{parallel}\index{line!parallel} to each other when, produced indefinitely in both directions, they do not meet one another in either direction.
\end{definition}

\begin{axiom}[Parallel Postulate]
    If the sum of the interior angles on one side is less than two right angles, the two straight lines, produced indefinitely, meet on that side.
\end{axiom}

\begin{proposition}
    If the sum of interior angles on one side is exactly the sum of two right angles, the two straight lines, produced indefinitely, will never meet.
\end{proposition}

\subsection{Polygons}
\begin{definition}
    A \textbf{polygon}\index{polygon} is a plane figure made up of \textbf{sides}\index{polygon!side} connected to form a closed figure. An $n$-sided polygon (where $n \geq 3$) is called a \textbf{$n$-gon}\index{$n$-gon}.
\end{definition}

\begin{proposition}
    The sum of interior angles in an $n$-gon is $(n-2)\pi$.
\end{proposition}

\begin{definition}
    A polygon where all of the sides are of equal length is called a \textbf{regular polygon}\index{polygon!regular}.
\end{definition}

\begin{definition}
    A \textbf{triangle}\index{triangle} is a 3-gon.
    \begin{itemize}
        \item An \textbf{equilateral triangle}\index{triangle!equilateral} is a triangle where all 3 sides are equal in length. Such a triangle is also called a regular 3-gon.
        \item An \textbf{isosceles triangle}\index{triangle!isosceles} is a triangle where 2 of the sides are equal in length.
        \item A \textbf{scalene triangle}\index{triangle!scalene} is a triangle where all 3 sides are of different length.
    \end{itemize}
    A triangle with endpoints $A$, $B$, and $C$, read in anticlockwise order, is denoted $\triangle ABC$.
\end{definition}

\begin{definition}
    A \textbf{right-angled triangle}\index{triangle!right-angled} is a triangle where one of its angles is a right angle. The longest side of a right-angled triangle is called its \textbf{hypotenuse}\index{triangle!right-angled!hypotenuse}.
\end{definition}

\begin{theorem}[Pythagoras]
    For a right-angled triangle with legs $a$ and $b$, the hypotenuse has length $\sqrt{a^2 + b^2}$.
\end{theorem}

Of special interest to us when we are dealing with geometric constructions are that of similar and congruent triangles.

\begin{definition}
    Two triangles $\triangle ABC$ and $\triangle A'B'C'$ are similar if and only if corresponding angles are equal in measure. That is,
    \begin{itemize}
        \item $\angle ABC = \angle A'B'C'$;
        \item $\angle BAC = \angle B'A'C'$; and
        \item $\angle BCA = \angle B'C'A'$.
    \end{itemize}
\end{definition}

\newpage

\begin{proposition}
    If $\triangle ABC$ and $\triangle PQR$ are similar, then the ratio of the corresponding sides of the two triangles are equal. That is,
    \begin{itemize}
        \item $\frac{AB}{AC} = \frac{PQ}{PR}$;
        \item $\frac{AB}{BC} = \frac{PQ}{QR}$; and
        \item $\frac{AC}{BC} = \frac{PR}{QR}$.
    \end{itemize}
\end{proposition}

\begin{definition}
    Two triangles $\triangle ABC$ and $\triangle A'B'C'$ are congruent if and only if corresponding sides are equal in length. That is,
    \begin{itemize}
        \item $AB = A'B'$;
        \item $AC = A'C'$; and
        \item $BC = B'C'$.
    \end{itemize}
\end{definition}

\subsection{Coordinate Geometry}
\begin{definition}
    A line in $\R^2$ has general form $ax + by + c = 0$ where $a,b,c, \in \R$.
\end{definition}

\begin{proposition}
    A line passing through the points $(x_1, y_1), (x_2, y_2) \in \R^2$ has the cartesian equation
    \[
        y - y_1 = \frac{y_2-y_1}{x_2-x_1}(x - x_1).
    \]
\end{proposition}

\begin{definition}
    A circle in $\R^2$ has general form $x^2 + y^2 + ax + by + c = 0$ where $a,b,c, \in \R$.
\end{definition}

\begin{proposition}
    A circle with center $(x_c, y_c) \in \R^2$ and radius $r \geq 0$ has the cartesian equation
    \[
        (x-x_c)^2 + (y-y_c)^2 = r^2.
    \]
\end{proposition}

\section{Basic Trigonometry}
\begin{definition}
    Let $\triangle AOB$ be a right-angled triangle with hypotenuse $OA$ of length 1 and a right angle of $\angle OBA$. Let $\angle AOB = \theta$ where $\theta \in [0, \frac\pi2]$.
    \begin{itemize}
        \item The \textbf{sine}\index{sine} of $\theta$, denoted $\sin\theta$, is the length of the side $AB$.
        \item The \textbf{cosine}\index{cosine} of $\theta$, denoted $\cos\theta$, is the length of the side $OB$.
        \item The \textbf{tangent}\index{tangent} of $\theta$, denoted $\tan\theta$, is $\frac{\sin\theta}{\cos\theta}$.
    \end{itemize}
\end{definition}

We note some special values of sine, cosine, and tangent for $\theta \in [0, \frac\pi2]$.

\begin{table}[H]
    \centering
    \begin{tabular}{|l|l|l|l|}
        \hline
        $\boldsymbol{\theta}$ & $\boldsymbol{\sin\theta}$ & $\boldsymbol{\cos\theta}$ & $\boldsymbol{\tan\theta}$ \\ \hline
        $\boldsymbol{0}$ & 0 & 1 & 0 \\ \hline
        $\boldsymbol{\frac\pi6}$ & $\frac12$ & $\frac{\sqrt3}2$ & $\frac{\sqrt3}3$ \\ \hline
        $\boldsymbol{\frac\pi4}$ & $\frac{\sqrt2}2$ & $\frac{\sqrt2}2$ & 1 \\ \hline
        $\boldsymbol{\frac\pi3}$ & $\frac{\sqrt3}2$ & $\frac12$ & $\sqrt3$ \\ \hline
        $\boldsymbol{\frac\pi2}$ & 1 & 0 & Undefined \\ \hline
    \end{tabular}
    \caption{Values of $\sin\theta$, $\cos\theta$, and $\tan\theta$ for $\theta \in [0, \frac\pi2]$}
\end{table}

It is natural to extend the definition of sine, cosine, and tangent for values of $\theta$ above $\frac\pi2$ by considering the unit circle.

\begin{definition}
    Let $\theta \in [0, 2\pi)$. Suppose we have a circle of radius 1 centred at a point $O$. Draw a line $OA$ of length 1 that is rotated $\theta$ anticlockwise about $O$. Then $\sin\theta$ is the $y$-coordinate of the point $A$ and  $\cos\theta$ is the $x$-coordinate of the point $A$.
\end{definition}

A diagram would certainly aid this definition.

\begin{figure}[H]
    \centering
    \pdfteximg{0.475\textwidth}{part3/images/geometric-constructions/trigonometry-quadrant-1.pdf_tex}
    \pdfteximg{0.475\textwidth}{part3/images/geometric-constructions/trigonometry-quadrant-2.pdf_tex}
    \pdfteximg{0.475\textwidth}{part3/images/geometric-constructions/trigonometry-quadrant-3.pdf_tex}
    \pdfteximg{0.475\textwidth}{part3/images/geometric-constructions/trigonometry-quadrant-4.pdf_tex}
    \caption{Sine and Cosine In Different Quadrants}
\end{figure}

In particular, we observe the following.
\begin{table}[H]
    \centering
    \begin{tabular}{|l|l|l|l|}
        \hline
        $\boldsymbol{\theta}$ & $\boldsymbol{\sin\theta}$ & $\boldsymbol{\cos\theta}$ & $\boldsymbol{\tan\theta}$ \\ \hline
        $\boldsymbol{[0,\frac\pi2)}$ & $\sin(\theta)$ & $-\cos(\theta)$ & $-\tan(\theta)$ \\ \hline
        $\boldsymbol{[\frac\pi2, \pi)}$ & $\sin(\pi-\theta)$ & $-\cos(\pi-\theta)$ & $-\tan(\pi-\theta)$ \\ \hline
        $\boldsymbol{[\pi,\frac{3\pi}2)}$ & $-\sin(\theta-\pi)$ & $-\cos(\theta-\pi)$ & $\tan(\theta-\pi)$ \\ \hline
        $\boldsymbol{[\frac{3\pi}2, 2\pi)}$ & $-\sin(2\pi-\theta)$ & $\cos(2\pi-\theta)$ & $-\tan(2\pi-\theta)$ \\ \hline
    \end{tabular}
    \caption{Values of Trigonometric Functions in Different Intervals of $\theta$}
\end{table}

If $\theta$ is outside the interval $[0, 2\pi)$ then we will consult the following definition.

\begin{definition}
    For any real number $\theta$ define $\sin\theta = \sin(\theta + 2\pi)$, $\cos\theta = \cos(\theta + 2\pi)$, and $\tan\theta = \tan(\theta + 2\pi)$.
\end{definition}

We note some important results about sine and cosine.
\begin{theorem}
    $(\sin\theta)^2 + (\cos\theta)^2 = 1$ for all $\theta \in \R$.
\end{theorem}
\begin{remark}
    $(\sin\theta)^2$ and $(\cos\theta)^2$ are more commonly written as $\sin^2\theta$ and $\cos^2\theta$.
\end{remark}

\begin{theorem}
    For any $\alpha,\beta \in \R$ we have $\sin(\alpha+\beta) = \sin\alpha\cos\beta + \cos\alpha\sin\beta$.
\end{theorem}
\begin{corollary}
    $\sin2\theta = 2\sin\theta\cos\theta$ for all $\theta \in \R$.
\end{corollary}

\begin{theorem}
    For any $\alpha,\beta \in \R$ we have $\cos(\alpha+\beta) = \cos\alpha\cos\beta - \sin\alpha\sin\beta$.
\end{theorem}
\begin{corollary}
    $\cos2\theta = \cos^2\theta - \sin^2\theta$ for all $\theta \in \R$.
\end{corollary}

\section{Rules of Construction}
We are finally ready to start the discussion on constructible numbers. Before that, we need to introduce the rules of construction.

We are only concerned with straightedge-and-compass constructions. It is important to note that \textit{a straightedge is not a ruler}. We cannot measure arbitrary lengths with a straightedge; it is merely a tool for drawing a line between two points on a line.

Let us look at some examples of construction with these rules.

\begin{proposition}[Perpendicular Bisector]
    Given a line segment $AB$, we can construct a line that bisects $AB$ and is perpendicular to $AB$.
\end{proposition}
\constructionproof{4}{
    \begin{figure}[H]
        \centering
        \pdfteximg{0.7\linewidth}{part3/images/geometric-constructions/perpendicular-bisector.pdf_tex}
    \end{figure}
}{
    Construct a circle centred at $A$ with radius $AB$. Construct another circle centred at $B$ with radius $AB$. Denote the intersection of the two circles by $C$ and $D$ as shown in the diagram. Connect the points $C$ and $D$ using a straightedge.
}{
    $CD$ bisects $AB$ and is perpendicular to $AB$.
}{
    The distances $AC$, $BC$, $AD$, and $BD$ are all the same length since they are all the same length $AB$ by construction.
    
    Observe that $\triangle CAB$ and $\triangle DAB$ are congruent because $AC = AD$, $BC = BD$, and the side $AB$ is shared by both triangles. One also sees that $\triangle CAD$ and $\triangle CBD$ are congruent since $AC = BC$, $AD = BD$, and $CD$ is a common side shared by both triangles.

    Thus we see $\triangle APC$, $\triangle BPC$, $\triangle APD$, and $\triangle BPD$ are all congruent since $AC = BC = AD = BD$ and the included angles are of the same measure. Therefore the angles $\angle CPA$, $\angle CPB$, $\angle DPA$, and $\angle DPB$ are equal. Since the total sum of angles about a point is $2\pi$, therefore $\angle CPA = \frac\pi2$, meaning $CD$ is perpendicular to $AB$. Also $AP = PB$ since $\triangle APC$ and $\triangle BPC$ are congruent, meaning that $CD$ bisects $AB$.

    Hence $CD$ is a perpendicular bisector of $AB$.
}

\begin{proposition}[Parallel Line]
    Given a line segment $AB$, we can construct a line that is parallel to $AB$.
\end{proposition}
\constructionproof{5}{
    \begin{figure}[H]
        \centering
        \pdfteximg{0.7\linewidth}{part3/images/geometric-constructions/parallel-line.pdf_tex}
    \end{figure}
}{
    Extend $AB$ infinitely. Construct the perpendicular bisector of $AB$. Then construct the perpendicular bisector of that perpendicular bisector. Label two points on this newly constructed perpendicular bisector $A'$ and $B'$.
}{
    $A'B'$ is parallel to $AB$.
}{
    The sum of the interior angles is exactly that of two right angles. Therefore $A'B'$ is parallel to $AB$.
}

\begin{proposition}[Angle in Semicircle]\label{prop-angle-in-semicircle}
    Suppose a triangle is inscribed within a circle where one of its sides is the diameter. Then the triangle is a right-angled triangle.
\end{proposition}
\constructionproof{5}{
    \begin{figure}[H]
        \centering
        \pdfteximg{0.7\linewidth}{part3/images/geometric-constructions/angle-in-semicircle.pdf_tex}
    \end{figure}
}{
    Draw a circle with center $O$. Let $AOB$ be a diameter of the circle. Choose a point $P$ on the circumference of the circle. Draw $\triangle APB$ and draw the line $OP$.
}{
    $\triangle APB$ is a right-angled triangle.
}{
    Consider $\triangle AOP$. Since $AO$ and $OP$ are both radii of the circle, therefore $\triangle AOP$ is an isosceles triangle. Similarly, $\triangle BOP$ is an isosceles triangle because $OB$ and $OP$ are radii of the circle.

    Let $\angle PAO = \alpha$ and $\angle PBO = \beta$. Then we see that $\angle APO = \alpha$ and $\angle BPO = \beta$ since $\triangle AOP$ and $\triangle BOP$ are isosceles triangles. Now because the sum of angles in a triangle is $\pi$, therefore within $\triangle APB$ we see that
    \[
        \alpha + \beta + (\alpha + \beta) = \pi
    \]
    which means $\alpha+\beta = \frac\pi2$. But $\alpha+\beta$ is exactly the measure of $\angle APB$, meaning that $\angle APB$ is a right angle. Hence $\triangle APB$ is a right-angled triangle.
}

\section{Constructable Numbers}
With the basics of construction out of the way, we can finally dive into the meat of this chapter -- constructible numbers.

\begin{definition}
    A real number $\alpha$ is \textbf{constructible}\index{constructible number} if and only if we can construct a line segment of length $|\alpha|$ in a finite number of steps from a line segment of length 1, a straightedge, and a compass.
\end{definition}

For brevity we denote the set of all constructible numbers by $\Con$.

\begin{theorem}\label{thrm-constructible-numbers-is-subfield-of-real-numbers}
    The set of all constructible numbers, $\Con$, is a subfield of the real numbers.
\end{theorem}
\begin{proof}
    By definition we know that $\Con^\ast$ is non-empty since 1 is a constructible number.

    Now let $\alpha, \beta \in \Con$. Without loss of generality we may assume $\alpha > \beta$, since we just need to construct $|\alpha - \beta|$.
    \begin{figure}[H]
        \centering
        \pdfteximg{0.5\textwidth}{part3/images/geometric-constructions/difference.pdf_tex}
    \end{figure}
    In the diagram, we first construct a line segment of length $|\alpha|$. Then, from the end, construct a line segment of length $|\beta|$. Whatever remains from the original start to the end of the line segment of length $|\beta|$ is a line segment of length $|\alpha - \beta|$, which shows that $\alpha - \beta$ is constructible.

    Now for any $\alpha,\beta \in \Con$ with $\beta \neq 0$ we show that $\alpha\beta^{-1} = \frac\alpha\beta$ is constructible.
    \begin{figure}[H]
        \centering
        \pdfteximg{0.5\textwidth}{part3/images/geometric-constructions/quotient.pdf_tex}
    \end{figure}
    In the diagram, we first construct $\triangle APQ$ where the length of $AP$ is $|\alpha|$ and the length of $AQ$ is $|\beta|$. The length $PQ$ can be freely chosen. Next we construct a line segment $BC$ that is parallel to $PQ$ and such that the length of $AC$ is 1. Since $BC$ is parallel to $PQ$ we therefore know that $\angle ABC = \angle APQ$ and $\angle ACB = \angle AQP$. Coupled with the fact that $\angle BAC = \angle PAQ$ is a shared angle means that $\triangle ABC$ and $\triangle APQ$ are similar. Hence we obtain the relationship that
    \[
        \frac{AB}{AC} = \frac{AP}{AQ}
    \]
    and therefore $\frac{AB}{1} = \frac{\alpha}{\beta}$, i.e. $AB = \frac\alpha\beta$, proving that $\frac\alpha\beta$ is constructible.

    Therefore by the subfield test (\myref{thrm-subfield-test}) we know that $\Con$ is a subfield of $\R$.
\end{proof}

We also note that the square root of a non-negative constructible number is constructible.
\begin{proposition}
    If $\alpha$ is a constructible number then $\sqrt{|\alpha|}$ is a constructible number.
\end{proposition}
\constructionproof[0.6\textwidth]{5}{
    \vspace{0.75cm}
    \begin{figure}[H]
        \centering
        \pdfteximg{0.75\linewidth}{part3/images/geometric-constructions/sqrt-constructible.pdf_tex}
    \end{figure}
}{
    First draw a line $AC$ of length $1 + |\alpha|$. Draw a circle of diameter $AC$. Mark a point $D$ on $AC$ such that $AD$ is of length 1. Construct a perpendicular line passing through $D$, marking the intersection point of this perpendicular line with the circle as $B$.
}{
    The length of $BD$ is $\sqrt{|\alpha|}$.
}{
    By Angle in Semicircle (\myref{prop-angle-in-semicircle}) we know that $\angle ABC$ is a right angle. Let $\angle BAD = \theta$. Since the sum of angles in a triangle is $\pi$, thus $\angle BCA = \frac\pi2 - \theta$. Consequently we see that $\angle ABD = \frac\pi2 - \theta$ and $\angle DBC = \theta$. Hence we see that $\triangle ABD$ and $\triangle BCD$ are similar. Thus
    \[
        \frac{AD}{BD} = \frac{BD}{CD}
    \]
    which means $\frac{1}{BD} = \frac{BD}{|\alpha|}$. Therefore $(BD)^2 = |\alpha|$, which proves that $BD = \sqrt{|\alpha|}$ as required.
}

\begin{exercise}
    Let $\theta \in [0, \frac\pi2]$. Prove that $\sin\theta$ is constructible if and only if $\cos\theta$ is constructible.
\end{exercise}

From \myref{thrm-constructible-numbers-is-subfield-of-real-numbers} we know that we can locate any point in the plane with rational coordinates. So any field of constructible numbers must be an extension field of the rationals. With this fact, we determine what other points can be constructed just from a straightedge and compass. To do so, we work with coordinates.

\begin{lemma}
    Let $\con$ be a field of constructible numbers. Then the points determined by the intersections of lines and circles in $\con$ lie in the field $\con(\sqrt{\alpha})$ for some positive $\alpha \in \con$.
\end{lemma}
\begin{proof}[Proof (see {\cite[pp.~272-273]{judson_beezer_2022}})]
    Starting with the points in the plane of $\con$, there are only 3 ways to construct new points.
    \begin{enumerate}
        \item Intersect two lines in $\con$. Suppose we have two lines $a_1x + b_1y + c_1 = 0$ and $a_2x + b_2y + c_2 = 0$ in $\con$, where $a_1,a_2,b_1,b_2,c_1,c_2 \in \con$. Solving for $x$ yields
        \[
            x = \left(\frac{b_1b_2}{a_1b_2-a_2b_1}\right)\left(\frac{c_2}{b_2}-\frac{c_1}{b_1}\right)
        \]
        which could clearly already be constructed. So this case does not generate any new points.

        \item Intersect two circles in $\con$. Suppose we have two circles with equations
        \begin{align*}
            x^2 + y^2 + a_1x + b_1y + c_1 &= 0\\
            x^2 + y^2 + a_2x + b_2y + c_2 &= 0
        \end{align*}
        where $a_1,a_2,b_1,b_2,c_1,c_2 \in \con$. Subtracting the two equations yields
        \[
            (a_1-a_2)x + (b_1-b_2)y + (c_1-c_2) = 0
        \]
        which is the equation of a line. Therefore the second case reduces down to solving
        \begin{align*}
            x^2 + y^2 + a_1x + b_1y + c_1 &= 0\\
            (a_1-a_2)x + (b_1-b_2)y + (c_1-c_2) &= 0
        \end{align*}
        which is the intersection of a circle and a line. This will be covered in the third case.

        \item Intersect a line and a circle in $\con$. Suppose we have a line and a circle with equations
        \begin{align*}
            ax + by + c &= 0\\
            x^2 + y^2 + dx + ey + f &= 0
        \end{align*}
        where $a,b,c,d,e,f \in \con$. In the equation of the line, we may solve for $y$, yielding
        \[
            y = \frac1b (-ax - c).    
        \]
        Substituting this into the equation of a circle and simplifying leaves us with an equation of the form $Ax^2 + Bx + C = 0$ where $A,B,C\in\con$. Solving for $x$ leaves us with
        \[
            x = -\frac{B}{2A} \pm \frac{1}{2A}\sqrt{B^2-4AC}.
        \]
        Note that we already know that $-\frac{B}{2A}$ and $\frac{1}{2A}$ are in $\con$; the only uncertain part is $\sqrt{B^2-4AC}$. Since the line and circle do intersect, we know that $B^2-4AC \geq 0$; letting $\alpha = B^2 - 4AC$ we therefore see that $x \in \con(\alpha)$. Note if $\alpha = 0$ then $\con(\alpha) = \con$; otherwise $\con(\alpha)$ is an extension field of $\con$.
    \end{enumerate}
    Therefore the points of intersections of lines and circles within $\con$ lie within the field $\con(\alpha)$ where $\alpha \in \con$ and $\alpha > 0$ as desired.
\end{proof}

\begin{theorem}\label{thrm-condition-for-constructability}
    A real number $\gamma$ is constructible if and only if there exists a sequence of fields
    \[
        \Q = F_0 \subset F_1 \subset F_2 \subset \cdots \subset F_n
    \]
    where $F_i = F_{i-1}(\sqrt{\alpha_i})$ with $\alpha_i \in F_{i-1}$ and $\gamma \in F_n$. In particular, there exists an integer $k \geq 0$ such that $[\Q(\gamma):\Q] = 2^k$.
\end{theorem}
\begin{proof}
    By the lemma we know that each $F_i$ and each $\alpha_i$ exists. By Tower Law (\myref{thrm-tower-law}) we therefore see that
    \[
        [F_n:\Q] = [F_n:F_{n-1}][F_{n-1}:F_{n-2}]\cdots[F_2:F_1][F_1:\Q].
    \]
    Now note that each $[F_i:F_{i-1}]$ is 2 since we defined $F_i = F_{i-1}(\sqrt{\alpha_i})$. Therefore $[F_n:\Q]$ is a power of 2, say $2^r$. In particular we note
    \[
        \underbrace{[F_n:\Q]}_{2^r} = [F_n:\Q(\gamma)][\Q(\gamma):\Q]
    \]
    by Tower Law, which means that $[\Q(\gamma):\Q] = 2^k$ for some integer $k \geq 0$.
\end{proof}

\begin{corollary}\label{corollary-field-of-all-constructible-numbers-is-algebraic-extension}
    $\Con$ is an algebraic extension of $\Q$.
\end{corollary}
\begin{proof}
    See \myref{exercise-field-of-all-constructible-numbers-is-algebraic-extension} (later).
\end{proof}

One sees that $\Con$ is not a finite extension, but is algebraic. This is yet another counterexample to the converse of \myref{thrm-finite-extension-is-algebraic} -- an algebraic field need \textit{not} be finite!

\begin{exercise}\label{exercise-field-of-all-constructible-numbers-is-algebraic-extension}
    Prove \myref{corollary-field-of-all-constructible-numbers-is-algebraic-extension}.
\end{exercise}

\section{Impossibility of Constructions}
We are now ready to investigate the classical problems that have plagued the Greeks.

\subsection{Doubling the Cube}
The Greeks wondered whether, given any cube, another cube of twice the volume could be constructed just by using a straightedge and compass. Using the tools that we have built, we can safely conclude that this task is impossible for any general cube.

Consider a cube of side length 1. We are to create a cube of volume 2, meaning that each side of the cube needs to have a side length of $\sqrt[3]2$. However, $\sqrt[3]2$ is a zero of the irreducible polynomial $x^3 - 2$ over $\Q$ (one can check that $x^3-2$ is indeed irreducible over $\Q$ by using Eisenstein's Criterion (\myref{thrm-eisenstein-criterion}) with the prime 2), which means that $[\Q(\sqrt[3]2):\Q] = 3$. But any constructible number has to create a finite extension with degree that is a power of 2. As 3 is not a power of 2, therefore $\sqrt[3]2$ is \textit{not} constructible (\myref{thrm-condition-for-constructability}); a cube of side length $\sqrt[3]2$ cannot be constructed and, therefore, the general problem of doubling the cube is impossible.

\subsection{Squaring the Circle}
Another problem the Greeks pondered was whether a square with the same area of a given circle can be constructed. Well, consider a circle with radius 1, meaning that the circle has area $\pi$. So we need to construct a square with area $\pi$, meaning that it has side length $\sqrt\pi$.

However, $\pi$ is a transcendental number, and therefore $\sqrt\pi$ is also transcendental. This means that $\sqrt\pi$ is not only not algebraic, it is not constructible; therefore squaring a circle is impossible.

\subsection{Trisecting an Angle}
The bisection of any given angle is easy for the Greeks; trisection was hard. In fact, this is yet another one of the impossible constructions that plagued the Greeks.

First we derive the triple-angle formula for cosine. For any $\theta \in \R$ we see
\begin{align*}
    \cos3\theta &= \cos(\theta + 2\theta)\\
    &= \cos\theta\cos2\theta - \sin\theta\sin2\theta\\
    &= (\cos\theta)(\cos^2\theta-\sin^2\theta) - (\sin\theta)(2\sin\theta\cos\theta)\\
    &= (\cos\theta)(\cos^2\theta - \sin^2\theta - 2\sin^2\theta)\\
    &= (\cos\theta)(\cos^2\theta - 3\sin^2\theta)\\
    &= (\cos\theta)(\cos^2\theta - 3(1-\cos^2\theta))\\
    &= (\cos\theta)(4\cos^2\theta - 3)\\
    &= 4\cos^3\theta - 3\cos\theta.
\end{align*}

Now consider $\theta = \frac{\pi}{9} = 20^\circ$. Using the formula above we see
\[
    \frac12 = \cos\frac\pi3 = \cos(3\times\frac\pi9) = 4\cos^3\frac\pi9 - 3\cos\frac\pi9
\]
which means that $\cos\frac\pi9$ is a zero of the polynomial $8x^3 - 6x - 1$ over $\Q$. However, we note that $8x^3 - 6x - 1$ is irreducible over $\Q$ by \myref{exercise-triple-angle-cosine-polynomial-is-irreducible} (later), so $[\Q(\cos\frac\pi9): \Q] = 3$, which means $\cos\frac\pi9$ is not constructible by \myref{thrm-condition-for-constructability}.

Now why does this mean that an angle of $\frac\pi3$ cannot be trisected? Well, if we were to trisect the angle, that would create a right-angled triangle of sides $\sin\frac\pi9$ and $\cos\frac\pi9$ with hypotenuse 1, which we have shown is not constructible. Therefore an angle of $\frac\pi3$ cannot be trisected, proving the impossibility of trisecting a general angle.

\begin{exercise}\label{exercise-triple-angle-cosine-polynomial-is-irreducible}
    Prove that $8x^3 - 6x - 1$ is irreducible over $\Q$.\newline
    (\textit{Hint: transformation before checking irreducibility.})
\end{exercise}
