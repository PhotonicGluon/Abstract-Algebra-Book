\chapter{Extension Fields and Splitting Fields}
For all of the previous chapters, we have looked at `substructures', subsets of the original structure that are, themselves, of the same type. For groups, we looked at subgroups; for rings, we looked at subrings; for vector spaces, we looked at subspaces. For fields, we can go the `reverse' direction, by finding a larger field that contains the original field, called an extension field.

\section{Extension Fields}
We motivate the definition for extension fields by looking at the following example.

\begin{example}\label{example-R[x]-mod-x^2+1-is-isomorphic-to-C}
    Consider the ring $F = \R[x]/\princ{x^2+1}$. We note that $x^2 + 1$ is irreducible over $\R$; this can be easily seen as $x^2+1$ has no zeroes in $\R$ and is therefore irreducible by \myref{thrm-degree-2-or-3-irreducible-iff-has-no-zeroes}. Therefore $\princ{x^2+1}$ is a maximal ideal in $\R[x]$ (\myref{thrm-irreducible-iff-principal-ideal-is-maximal}) which means that, in fact, $F$ is a field (\myref{thrm-maximal-ideal-iff-quotient-ring-is-field}).

    We find the general form of elements in $F$. Suppose $f(x) \in \R[x]$. Then by polynomial long division (\myref{thrm-polynomial-long-division}) we may write
    \[
        f(x) = q(x)(x^2+1) + r(x)
    \]
    where $r(x) = 0$ or $\deg r(x) < \deg(x^2+1) = 2$. It follows that $r(x) = ax + b$ for some $a, b \in \R$. Note that $q(x)(x^2+1) \in \princ{x^2+1}$, so $f(x) + \princ{x^2+1} = r(x) + \princ{x^2+1}$ in $F$, meaning that
    \[
        F = \left\{ax + b + \princ{x^2+1} \vert a,b\in \R\right\}.
    \]

    Now because
    \[
        (x^2 + 1) + \princ{x^2 + 1} = 0 + \princ{x^2+1}
    \]
    it follows quickly that
    \[
        x^2 + \princ{x^2 + 1} = -1 + \princ{x^2+1},
    \]
    so, in this field, the polynomial $x^2$ is `equivalent' to -1. This means that, in this field, there is a `solution' to the equation $x^2 + 1 = 0$. It should not be too surprising, in light of this observation, that $F \cong \C$, via the map $\phi: F \to \C$ where
    \[
        ax + b + \princ{x^2 + 1} \mapsto b + ai.
    \]
    We leave it to \myref{exercise-R[x]-mod-x^2+1-is-isomorphic-to-C} (later) to verify that $F \cong \C$.
\end{example}

\begin{exercise}\label{exercise-R[x]-mod-x^2+1-is-isomorphic-to-C}
    Verify that the field $F$ in \myref{example-R[x]-mod-x^2+1-is-isomorphic-to-C} is indeed isomorphic to $\C$ by using the map $\phi$ given. For completeness, also verify that $\phi$ is well-defined.
\end{exercise}

The idea that we can `extend' the real numbers to create a solution to $x^2 + 1 = 0$ is critical to our definition of \textbf{extension fields}.

\begin{definition}
    A field $E$ is an \textbf{extension field}\index{extension field} of a field $F$ if and only if $E$ contains a subfield isomorphic to $F$. In this case $F$ is called the \textbf{base field}\index{base field}\index{field!base}.
\end{definition}
\begin{remark}
    This definition differs from most authors' definitions of a field extension (cf. \cite[p.~511]{dummit_foote_2004}, \cite[p.~442]{artin_2011}, \cite[p.~338]{gallian_2016}, \cite[p.~260]{judson_beezer_2022}), which is that a field extension is a field $E$ that just has $F$ as a subfield.
\end{remark}

Before we answer why we use a different definition, we look at the following proposition.

\begin{proposition}
    Let $E$ and $F$ be fields. If $F \subseteq E$ then $E$ is an extension field of $F$.
\end{proposition}
\begin{proof}
    Construct the map $\phi: F \to E, x \mapsto x$. We note that for any $x,y\in F$, we have $\phi(x + y) = x + y = \phi(x) + \phi(y)$ and $\phi(xy) = xy = \phi(x)\phi(y)$, so $\phi$ is a homomorphism. Therefore, $F/\ker\phi \cong \im\phi$ by the FRIT (\myref{thrm-ring-isomorphism-1}). As fields do not have proper ideals (\myref{prop-ring-is-field-iff-no-proper-ideals}), therefore $\ker\phi = F$, which is not the case since $\phi(1) = 1 \neq 0$, or $\ker\phi = \{0\}$, which results in $F/\ker\phi \cong F$ by \myref{problem-integral-domain-iff-trivial-ideal-is-prime}. Thus $\im\phi$ is a subfield of $E$ that is isomorphic to $F$, proving that $E$ is an extension field of $F$.
\end{proof}

\begin{example}
    Consider the quadratic field
    \[
        \Q[\sqrt{2}] = \{a + b\sqrt{2} \vert a,b\in\Q\}.
    \]
    We claim that $\Q[\sqrt{2}]$ is an extension field of $\Q$ (i.e., $\Q[\sqrt{2}]/\Q$). One sees clearly that any $q \in \Q$ is expressible as $q + 0\sqrt{2}$, and so $q \in \Q[\sqrt2]$. Therefore $\Q \subseteq \Q[\sqrt2]$. As $\Q$ is a field, therefore $\Q$ is a subfield of $\Q[\sqrt2]$, which in turn means $\Q[\sqrt2]$ is an extension field of $\Q$.
\end{example}

We return to why we defined field extensions in the way that we do. The reason why is because, most of the time, the extension field that we use does \textit{not} contain the base field as a subset. Rather, the extension field contains a subfield that is \textit{isomorphic} to the base field. In fact, most authors actually use our definition of extension fields -- they find a subfield of the extension field that is isomorphic to the base field (see \cite[Theorem 13.1.3]{dummit_foote_2004}, \cite[p.~339, Proof]{gallian_2016}, \cite[Example 21.2]{judson_beezer_2022}).

We have this useful proposition to help us determine whether a field is an extension field of another field.

\begin{proposition}\label{prop-extension-field-if-homomorphism-between-fields}
    Let $E$ and $F$ be fields. Then $E$ is an extension field of $F$ if there is a non-trivial homomorphism $\phi: F \to E$.
\end{proposition}
\begin{proof}
    Any homomorphism from a field is either trivial or injective (\myref{thrm-homomorphism-from-field-is-injective-or-trivial}). Since $\phi$ is non-trivial, therefore $\phi$ is injective. Thus we see $F \cong \im\phi$. As $\im\phi$ is a subfield of $E$, therefore $E$ contains a subfield isomorphic to $F$.
\end{proof}

\begin{example}
    Consider the field $F = \R[x]/\princ{x^2+1}$ in \myref{example-R[x]-mod-x^2+1-is-isomorphic-to-C}. We note that $F$ is a field extension of $\R$ by considering the map $\phi: \R \to F, r \mapsto r + \princ{x^2+1}$. We note that $\phi$ is a homomorphism since, for all $a, b \in \R$, we have
    \begin{align*}
        \phi(a + b) &= (a + b) + \princ{x^2+1}\\
        &= \left(a + \princ{x^2+1}\right) + \left(b + \princ{x^2+1}\right)\\
        &= \phi(a) + \phi(b)
    \end{align*}
    and
    \begin{align*}
        \phi(ab) &= ab + \princ{x^2+1}\\
        &= \left(a + \princ{x^2+1}\right)\left(b + \princ{x^2+1}\right)\\
        &= \phi(a)\phi(b)
    \end{align*}
    which means $\phi$ is a homomorphism. Therefore $F$ is an extension field of $\R$ by \myref{prop-extension-field-if-homomorphism-between-fields}.

    If this still feels uncomfortable, recall that we proved that $F \cong \C$ in \myref{exercise-R[x]-mod-x^2+1-is-isomorphic-to-C}. We have already shown that $\R \subseteq \C$, and since $F \cong \C$, it is natural to think that $F$ is an extension field of the real numbers.
\end{example}

\begin{definition}
    A \textbf{field extension}\index{field!extension}, denoted $E/F$ and read ``$E$ over $F$'', indicates that $E$ is an extension field over the base field $F$.
\end{definition}
\begin{remark}
    We may say ``let $E/F$ be a field extension'', which is the same as saying ``let $E$ be an extension field of the field $F$''.
\end{remark}
\begin{remark}
    This notation should not be confused with the notation for quotient rings. However, in times of doubt of notation, we elect to explicitly state what the extension field is.
\end{remark}

Returning back to our exploration of extension fields, the critical observation that $\R[x]/\princ{x^2+1}$ contains a `zero' of $x^2+1$ lead Leopold Kronecker to come up with a generalization of this fact in 1887. This result is often called the \textbf{Fundamental Theorem of Field Theory} (e.g. \cite[Theorem 20.1]{gallian_2016}, \cite[Theorem 21.5]{judson_beezer_2022}).
\begin{theorem}[Fundamental Theorem of Field Theory]\label{thrm-fundamental-theorem-of-field-theory}\index{Fundamental Theorem of Field Theory}
    Let $F$ be a field and $f(x)$ be a non-constant polynomial in $F[x]$. Then there exists a field extension $E/F$ that has a zero of $f(x)$.
\end{theorem}
\begin{proof}
    Since $F$ is a field and thus a UFD, therefore $F[x]$ is a UFD (\myref{thrm-UFD-iff-polynomial-ring-is-UFD}). So $f(x)$ has an irreducible factor, say $p(x)$. Thus it suffices to show that an irreducible polynomial $p(x) \in F[x]$ has a zero in $E$.

    We note that $\princ{p(x)}$ is a maximal ideal since $p(x)$ is irreducible (\myref{thrm-irreducible-iff-principal-ideal-is-maximal}), and therefore $E = F[x]/\princ{p(x)}$ is a field (\myref{thrm-maximal-ideal-iff-quotient-ring-is-field}). Construct the map $\phi: F \to E$ where $a \mapsto a + \princ{p(x)}$. We note that $\phi$ is a homomorphism between fields since
    \begin{align*}
        \phi(a+b) &= (a+b) + \princ{p(x)}\\
        &= (a + \princ{p(x)}) + (b + \princ{p(x)})\\
        &= \phi(a) + \phi(b)
    \end{align*}
    and
    \begin{align*}
        \phi(ab) &= ab + \princ{p(x)}\\
        &= (a+\princ{p(x)})(b+\princ{p(x)})\\
        &= \phi(a)\phi(b).
    \end{align*}
    Therefore means that $E$ is a field extension of $F$ by \myref{prop-extension-field-if-homomorphism-between-fields}.
    
    It remains to show that $E$ contains a zero of $p(x)$. Write
    \[
        p(x) = a_0 + a_1x + a_2x^2 + \cdots + a_nx^n
    \]
    where $a_0, a_1, \dots, a_n \in F$. In $E$, the polynomial becomes
    \[
        \overline{p}(x) = (a_0 + \princ{p(x)}) + (a_1 + \princ{p(x)})x + (a_2 + \princ{p(x)})x^2 + \cdots + (a_n + \princ{p(x)})x^n
    \]
    For brevity let $\overline{a_i} = a_0 + \princ{p(x)}$. One sees clearly that $\alpha = x + \princ{p(x)} \in E$. Note
    \begin{align*}
        \overline{p}(\alpha) &= \overline{a_0} + \overline{a_1}\alpha + \overline{a_2}\alpha^2 + \cdots + \overline{a_n}\alpha^n\\
        &= (a_0 + \princ{p(x)}) + (a_1 + \princ{p(x)})(x + \princ{p(x)}) + (a_2 + \princ{p(x)})(x + \princ{p(x)})^2\\
        &\quad\quad+ \cdots + (a_n + \princ{p(x)})(x + \princ{p(x)})^n\\
        &= (a_0 + \princ{p(x)}) + (a_1x + \princ{p(x)}) + (a_2x^2 + \princ{p(x)}) + \cdots + (a_nx^n + \princ{p(x)})\\
        &= (a_0 + a_1x + a_2x^2 + \cdots + a_nx^n) + \princ{p(x)}\\
        &= p(x) + \princ{p(x)}\\
        &= 0 + \princ{p(x)}.
    \end{align*}
    Therefore, the irreducible polynomial $p(x) \in F[x]$ has a zero in $E$, after suitable conversion has been made.
\end{proof}

\begin{example}
    Consider the irreducible polynomial $f(x) = x^2 + x + 1 \in \Z_2[x]$. To find a field extension $E/\Z_2$ such that $f(x)$ has a zero, we may take $E = \Z_2[x]/\princ{f(x)}$, which is a field of 4 elements. Then by \myref{thrm-fundamental-theorem-of-field-theory} we know that $x + \princ{f(x)}$ is a zero of the `transformed' version of $f(x)$ in $E$. Indeed, we see
    \begin{align*}
        &(1 + \princ{f(x)})(x + \princ{f(x)})^2 + (1 + \princ{f(x)})(x + \princ{f(x)}) + (1 + \princ{f(x)})\\
        &= (x^2+\princ{f(x)}) + (x + \princ{f(x)}) + (1 + \princ{f(x)})\\
        &= (x^2 + x + 1) + \princ{f(x)}\\
        &= f(x) + \princ{f(x)}\\
        &= 0 + \princ{x^2+x+1},
    \end{align*}
    so $f(x)$ has a zero in $E$.
\end{example}

\begin{example}
    Consider the polynomial $f(x) = x^5 + 2x^2 + 2x + 2 \in \Z_3[x]$. We may factor $f(x)$ into irreducibles by noting $f(x) = (x^2+1)(x^3+2x+2)$. Then \myref{thrm-fundamental-theorem-of-field-theory} tells us that the fields
    \[
        \Z_3[x]/\princ{x^2+1} \quad\text{and}\quad \Z_3[x]/\princ{x^3+2x+2}
    \]
    are both extension fields of $\Z_3$ that contain a zero of $f(x)$.
\end{example}

The Fundamental Theorem of Field Theory (\myref{thrm-fundamental-theorem-of-field-theory}) may actually be extended to integral domains. Suppose $f(x) \in D[x]$ where $D$ is an integral domain. Since the field of fractions contains a subring isomorphic to the original integral domain (\myref{thrm-field-of-fractions-contains-integral-domain}), thus we may interpret $f(x)$ as a polynomial in $\Frac{D}[x]$ instead. Then the Fundamental Theorem of Field Theory gives us the required result.

However, this does not apply to commutative rings in general, as seen in the following exercise.

\begin{exercise}
    Let $f(x) = 2x + 1$ be a polynomial in $\Z_4[x]$. Prove that no ring that contains a subring that is isomorphic to $\Z_4$ has a zero of $f(x)$.    
\end{exercise}

\begin{exercise}
    Find two extension fields that contains a zero of the polynomial $x^4 + 3x^2 + 1 \in \Q[x]$.
\end{exercise}

Henceforth, to simplify the notation used for elements of the extension field, we will simply identify elements in the subset isomorphic to the base field with the actual element within the base field. Furthermore, whenever we say that an extension field $E$ \textit{contains} the base field $F$, it means that $E$ contains a subfield isomorphic to $F$.

\begin{example}
    Consider the extension field $E = \R[x]/\princ{x^2+1}$ of $\R$. The coset $1 + \princ{x^2+1}$ will henceforth be written as just 1, since the conversion from the element in $\R$ to the coset $1 + \princ{x^2+1}$ is simple -- just add the principal ideal $\princ{x^2+1}$ to the end.

    For polynomials within $\R[x]$, such as $x^2 + 2x + 3$, we may interpret $x^2 + 2x + 3$ as a polynomial in $E$, just remembering that the coefficients of that polynomial are actually the cosets $1 + \princ{x^2+1}$, $2 + \princ{x^2+1}$, and $3 + \princ{x^2+1}$ respectively.
\end{example}

\section{Splitting Fields}
To motivate our next definition, we return back to the extension field $E = \R[x]/\princ{x^2+1}$ of $\R$.
\begin{example}
    We note by the Fundamental Theorem of Field Theory (\myref{thrm-fundamental-theorem-of-field-theory}) that $\alpha = x + \princ{x^2+1}$ is a zero of the polynomial $x^2+1$ in $E$. One also sees that $-\alpha$ is also a zero of $x^2+1$, since
    \begin{align*}
        (-\alpha)^2 + 1 &= (-x + \princ{x^2+1})^2 + 1\\
        &= x^2 + 1 + \princ{x^2+1}\\
        &= 0 + \princ{x^2+1}.
    \end{align*}
    By the Factor Theorem (\myref{corollary-factor-theorem}), it should follow that both $x - \alpha$ and $x + \alpha$ are factors of $x^2 + 1$ in $E$. Let's try and verify this. One sees
    \begin{align*}
        (x-\alpha)(x+\alpha) &= x^2 - \alpha^2\\
        &= x^2 - (x^2 + \princ{x^2+1})\\
        &= x^2 - (-1 + \princ{x^2+1}) & (\text{since } x^2 + \princ{x^2+1} = -1 + \princ{x^2+1})\\
        &= x^2 + 1 + \princ{x^2+1}\\
        &= x^2 + 1 & (\text{we identity elements without the coset})
    \end{align*}
    so we indeed can factor $x^2 + 1$ into two linear factors within the extension field $E$.
\end{example}

One may wonder if polynomials from any field can be `split' nicely into linear factors within a certain extension field. In fact, they can -- these extension fields are called splitting fields.

Before we can define what a splitting field is, we need to look at the smallest subfield of an extension field that contains the base field and a specific subset of elements of the extension field.
\begin{definition}
    Let $F$ be a field and $E/F$ be a field extension. Let $S$ be a subset of $E$. Then $F(S)$, read as ``$F$ adjoin $S$'', is the smallest subfield of $E$ that contains $F$ and the set $S$.

    If $S$ is a finite set, say $S = \{a_1, a_2, \dots, a_n\}$, then we write $F(a_1, a_2, \dots, a_n)$, and say that $F(S)$ is finitely generated over $F$.
\end{definition}

\begin{proposition}
    $F(S)$ is the intersection of all subfields of $E$ that contain both $F$ and $S$.
\end{proposition}
\begin{proof}
    Let $K$ be the intersection of all subfields of $E$ that contain both $F$ and $S$.
    
    By \myref{prop-intersection-of-subfields-is-subfield} the intersection of subfields is a subfield, so $K$ is indeed a subfield of $E$. Note by definition of $K$ that it contains both $F$ and $S$. Since $F(S)$ contains both $F$ and the set $S$ by definition, thus we have $K \subseteq F(S)$.

    Also, since $F(S)$ contains both $F$ and $S$, it must be inside the intersection that is $K$, which means that $F(S) \subseteq K$.

    Therefore we see $K = F(S)$.
\end{proof}

With this definition, we can now define \textbf{splitting fields}.

\begin{definition}
    Let $F$ be a field, $E/F$ be a field extension, and $f(x) \in F[x]$ be a non-constant polynomial. We say that $f(x)$ \textbf{splits}\index{polynomial!splits} in $E$ if and only if there exist an element $a \in F$ and elements $a_1, a_2, \dots, a_n \in E$ such that
    \[
        f(x) = a(x-a_1)(x-a_2)\cdots(x-a_n).
    \]
    We call $E$ a \textbf{splitting field for $f(x)$ over $F$}\index{splitting field}\index{field!splitting} if
    \[
        E = F(a_1, a_2, \dots, a_n).
    \]
\end{definition}
\begin{remark}
    In other words, the splitting field of $f(x)$ over $F$ is a smallest extension field that allows $f(x)$ to split.
\end{remark}

We note that the splitting field for a polynomial not only depends on the polynomial itself, but also the base field.
\begin{example}
    Consider the polynomial $x^2 + 1$. If we interpret $x^2 + 1$ being in $\R[x]$, then naturally the splitting field of $x^2 + 1$ is $\C$, since $x^2 + 1 = (x-i)(x+i)$.

    However, if we think of $x^2+1$ as being a polynomial in $\Q$, then the splitting field is $\Q(i, -i)$. In fact, $\Q(i, -i) = \Q(i)$ since any field containing $i$ must also contain $-i$ (as it is the additive inverse of $i$). One sees that $\Q(i) \neq \C$ since $\C$ contains irrational numbers (such as $\sqrt2$) while $\Q(i)$ must at least contain $\Q$ and so does not contain irrational numbers.
\end{example}
\begin{example}
    Consider the polynomial $x^2 - 2 \in \Q[x]$. Clearly $x^2 - 2$ splits in $\R$ since $x^2 - 2 = (x - \sqrt2)(x + \sqrt2)$, but the splitting field of $x^2 - 2 \in \Q[x]$ is $\Q(\sqrt2, -\sqrt2)$ which, again, is $\Q(\sqrt 2)$.
\end{example}

One may wonder if a splitting field always exists for a given polynomial and a given field. The answer -- yes.

\begin{theorem}
    Let $F$ be a field and $f(x)$ be a non-constant polynomial in $F[x]$. Then there exists a splitting field $E$ for $f(x)$ over $F$.
\end{theorem}
\begin{proof}
    Let $f(x)$ have degree $n$. We use strong induction on $n$.
    
    When $n = 1$, we know $f(x)$ is of the form $ax + b$, and so we may write
    \[
        f(x) = a\left(x - \left(\frac ba\right)\right),
    \]
    which means $f(x)$ splits over the original field $F$.

    Now assume that, for some positive integer $k$, there exists a splitting field $E$ for any field $F$ and for any polynomial in $F[x]$ with degree that is at most $k$. We show that any polynomial of degree $k + 1$ in the field $F[x]$ has a splitting field.

    Let $f(x) \in F[x]$ be an arbitrary polynomial with degree $k + 1$. By the Fundamental Theorem of Field Theory (\myref{thrm-fundamental-theorem-of-field-theory}), there exists an extension field $E$ in which $f(x)$ has a zero. Suppose this zero is $a_1$, meaning that we may write
    \[
        f(x) = (x - a_1)g(x)
    \]
    for some $g(x) \in E[x]$, by the Factor Theorem (\myref{corollary-factor-theorem}). We note that $\deg g(x) < \deg f(x) = k + 1$ which means $\deg g(x) \leq k$. By the induction hypothesis, there is a splitting field of $g(x)$, say $L$, that contains $E$ and all the zeroes of $g(x)$, which are, say, $a_2, a_3, \dots, a_k, a_{k+1}$. Since $L$ contains $E$ we may also find $a_1 \in L$ (of course, this $a_1$ needs to be converted into the appropriate form within $L$). One then sees that $F(a_1, a_2, \dots, a_{k+1})$ is a splitting field for $f(x)$ over $F$, which shows that the theorem holds for $k + 1$ as well.

    The theorem is proven by induction.
\end{proof}

% TODO: Continue

\section{The Derivative}
% TODO: Add

\section{Zeroes of an Irreducible Polynomial}
% TODO: Add

\newpage

\section{Problems}
% TODO: Add
