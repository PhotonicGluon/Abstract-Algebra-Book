\chapter{Extension Fields and Splitting Fields}
For all of the previous chapters, we have looked at `substructures', subsets of the original structure that are, themselves, of the same type. For groups, we looked at subgroups; for rings, we looked at subrings; for vector spaces, we looked at subspaces. For fields, we can go the `reverse' direction, by finding a larger field that contains the original field, called an extension field.

\section{Extension Fields}
We motivate the definition for extension fields by looking at the following example.

\begin{example}\label{example-R[x]-mod-x^2+1-is-isomorphic-to-C}
    Consider the ring $F = \R[x]/\princ{x^2+1}$. We note that $x^2 + 1$ is irreducible over $\R$; this can be easily seen as $x^2+1$ has no zeroes in $\R$ and is therefore irreducible by \myref{thrm-degree-2-or-3-irreducible-iff-has-no-zeroes}. Therefore $\princ{x^2+1}$ is a maximal ideal in $\R[x]$ (\myref{thrm-irreducible-iff-principal-ideal-is-maximal}) which means that, in fact, $F$ is a field (\myref{thrm-maximal-ideal-iff-quotient-ring-is-field}).

    We find the general form of elements in $F$. Suppose $f(x) \in \R[x]$. Then by polynomial long division (\myref{thrm-polynomial-long-division}) we may write
    \[
        f(x) = q(x)(x^2+1) + r(x)
    \]
    where $r(x) = 0$ or $\deg r(x) < \deg(x^2+1) = 2$. It follows that $r(x) = ax + b$ for some $a, b \in \R$. Note that $q(x)(x^2+1) \in \princ{x^2+1}$, so $f(x) + \princ{x^2+1} = r(x) + \princ{x^2+1}$ in $F$, meaning that
    \[
        F = \left\{ax + b + \princ{x^2+1} \vert a,b\in \R\right\}.
    \]

    Now because
    \[
        (x^2 + 1) + \princ{x^2 + 1} = 0 + \princ{x^2+1}
    \]
    it follows quickly that
    \[
        x^2 + \princ{x^2 + 1} = -1 + \princ{x^2+1},
    \]
    so, in this field, the polynomial $x^2$ is `equivalent' to -1. This means that, in this field, there is a `solution' to the equation $x^2 + 1 = 0$. It should not be too surprising, in light of this observation, that $F \cong \C$, via the map $\phi: F \to C$ where
    \[
        ax + b + \princ{x^2 + 1} \mapsto b + ai.
    \]
    We leave it to \myref{exercise-R[x]-mod-x^2+1-is-isomorphic-to-C} (later) to verify that $F \cong C$.
\end{example}

\begin{exercise}\label{exercise-R[x]-mod-x^2+1-is-isomorphic-to-C}
    Verify that the field $F$ in \myref{example-R[x]-mod-x^2+1-is-isomorphic-to-C} is indeed isomorphic to $\C$ by using the map $\phi$ given. For completeness, also verify that $\phi$ is well-defined.
\end{exercise}

The idea that we can `extend' the real numbers to create a solution to $x^2 + 1 = 0$ is critical to our definition of \textbf{extension fields}.

\begin{definition}
    A field $E$ is an \textbf{extension field}\index{extension field} (or \textbf{field extension}\index{field!extension}) of a field $F$ if and only if $E$ contains a subfield isomorphic to $F$. In this case $F$ is called the \textbf{base field}\index{base field}\index{field!base}.
\end{definition}
\begin{remark}
    This definition differs from most authors' definitions of a field extension (cf. \cite[p.~511]{dummit_foote_2004}, \cite[p.~442]{artin_2011}, \cite[p.~338]{gallian_2016}, \cite[p.~260]{judson_beezer_2022}), which is that a field extension is a field $E$ that has $F$ as a subfield.
\end{remark}

Before we answer why we use a different definition, we look at the following proposition.

\begin{proposition}
    Let $E$ and $F$ be fields. If $F \subseteq E$ then $E$ is an extension field of $F$.
\end{proposition}
\begin{proof}
    Construct the map $\phi: F \to E, x \mapsto x$. We note that for any $x,y\in F$, we have $\phi(x + y) = x + y = \phi(x) + \phi(y)$ and $\phi(xy) = xy = \phi(x)\phi(y)$, so $\phi$ is a homomorphism. Therefore, $F/\ker\phi \cong \im\phi$ by the FRIT (\myref{thrm-ring-isomorphism-1}). As fields do not have proper ideals (\myref{prop-ring-is-field-iff-no-proper-ideals}), therefore $\ker\phi = F$, which is not the case since $\phi(1) = 1 \neq 0$, or $\ker\phi = \{0\}$, which results in $F/\ker\phi \cong F$ by \myref{problem-integral-domain-iff-trivial-ideal-is-prime}. Thus $\im\phi$ is a subfield of $E$ that is isomorphic to $F$, proving that $E$ is an extension field of $F$.
\end{proof}

\begin{example}
    Consider the quadratic field
    \[
        \Q[\sqrt{2}] = \{a + b\sqrt{2} \vert a,b\in\Q\}.
    \]
    We claim that $\Q[\sqrt{2}]$ is an extension field of $\Q$ (i.e., $\Q[\sqrt{2}]/\Q$). One sees clearly that any $q \in \Q$ is expressible as $q + 0\sqrt{2}$, and so $q \in \Q[\sqrt2]$. Therefore $\Q \subseteq \Q[\sqrt2]$. As $\Q$ is a field, therefore $\Q$ is a subfield of $\Q[\sqrt2]$, which in turn means $\Q[\sqrt2]$ is an extension field of $\Q$.
\end{example}

We return to why we defined field extensions in the way that we do. The reason why is because, most of the time, the extension field that we use does \textit{not} contain the base field as a subset. Rather, the extension field contains a subfield that is \textit{isomorphic} to the base field. In fact, most authors actually use our definition of extension fields -- they find a subfield of the extension field that is isomorphic to the base field (see \cite[Theorem 13.1.3]{dummit_foote_2004}, \cite[p.~339, Proof]{gallian_2016}, \cite[Example 21.2]{judson_beezer_2022}).

We have this useful proposition to help us determine whether a field is an extension field of another field.

\begin{proposition}\label{prop-extension-field-if-homomorphism-between-fields}
    Let $E$ and $F$ be fields. Then $E$ is an extension field of $F$ if there is a non-trivial homomorphism $\phi: F \to E$.
\end{proposition}
\begin{proof}
    Any homomorphism from a field is either trivial or injective (\myref{thrm-homomorphism-from-field-is-injective-or-trivial}). Since $\phi$ is non-trivial, therefore $\phi$ is injective. Thus we see $F \cong \im\phi$. As $\im\phi$ is a subfield of $E$, therefore $E$ contains a subfield isomorphic to $F$.
\end{proof}

\begin{example}
    Consider the field $F = \R[x]/\princ{x^2+1}$ in \myref{example-R[x]-mod-x^2+1-is-isomorphic-to-C}. We note that $F$ is a field extension of $\R$ by considering the map $\phi: \R \to F, r \mapsto r + \princ{x^2+1}$. We note that $\phi$ is a homomorphism since, for all $a, b \in \R$, we have
    \begin{align*}
        \phi(a + b) &= (a + b) + \princ{x^2+1}\\
        &= \left(a + \princ{x^2+1}\right) + \left(b + \princ{x^2+1}\right)\\
        &= \phi(a) + \phi(b)
    \end{align*}
    and
    \begin{align*}
        \phi(ab) &= ab + \princ{x^2+1}\\
        &= \left(a + \princ{x^2+1}\right)\left(b + \princ{x^2+1}\right)\\
        &= \phi(a)\phi(b)
    \end{align*}
    which means $\phi$ is a homomorphism. Therefore $F$ is an extension field of $\R$ by \myref{prop-extension-field-if-homomorphism-between-fields}.

    If this still feels uncomfortable, recall that we proved that $F \cong \C$ in \myref{exercise-R[x]-mod-x^2+1-is-isomorphic-to-C}. We have already shown that $\R \subseteq \C$, and since $F \cong \C$, it is natural to think that $F$ is an extension field of the real numbers.
\end{example}

Returning back to our exploration of extension fields, the critical observation that $\R[x]/\princ{x^2+1}$ contains a `zero' of $x^2+1$ lead Leopold Kronecker to come up with a generalization of this fact in 1887. This result is often called the \textbf{Fundamental Theorem of Field Theory} (e.g. \cite[Theorem 20.1]{gallian_2016}, \cite[Theorem 21.5]{judson_beezer_2022}).
\begin{theorem}[Fundamental Theorem of Field Theory]\label{thrm-fundamental-theorem-of-field-theory}\index{Fundamental Theorem of Field Theory}
    Let $F$ be a field and $f(x)$ be a non-constant polynomial in $F[x]$. Then there exists an extension field $E$ of $F$ that has a zero of $f(x)$.
\end{theorem}
\begin{proof}
    Since $F$ is a field and thus a UFD, therefore $F[x]$ is a UFD (\myref{thrm-UFD-iff-polynomial-ring-is-UFD}). So $f(x)$ has an irreducible factor, say $p(x)$. Thus it suffices to show that an irreducible polynomial $p(x) \in F[x]$ has a zero in $E$.

    We note that $\princ{p(x)}$ is a maximal ideal since $p(x)$ is irreducible (\myref{thrm-irreducible-iff-principal-ideal-is-maximal}), and therefore $E = F[x]/\princ{p(x)}$ is a field (\myref{thrm-maximal-ideal-iff-quotient-ring-is-field}). Construct the map $\phi: F \to E$ where $a \mapsto a + \princ{p(x)}$. We note that $\phi$ is a homomorphism between fields since
    \begin{align*}
        \phi(a+b) &= (a+b) + \princ{p(x)}\\
        &= (a + \princ{p(x)}) + (b + \princ{p(x)})\\
        &= \phi(a) + \phi(b)
    \end{align*}
    and
    \begin{align*}
        \phi(ab) &= ab + \princ{p(x)}\\
        &= (a+\princ{p(x)})(b+\princ{p(x)})\\
        &= \phi(a)\phi(b).
    \end{align*}
    Therefore means that $E$ is a field extension of $F$ by \myref{prop-extension-field-if-homomorphism-between-fields}.
    
    It remains to show that $E$ contains a zero of $p(x)$. Write
    \[
        p(x) = a_0 + a_1x + a_2x^2 + \cdots + a_nx^n
    \]
    where $a_0, a_1, \dots, a_n \in F$. In $E$, the polynomial becomes
    \[
        \overline{p}(x) = (a_0 + \princ{p(x)}) + (a_1 + \princ{p(x)})x + (a_2 + \princ{p(x)})x^2 + \cdots + (a_n + \princ{p(x)})x^n
    \]
    For brevity let $\overline{a_i} = a_0 + \princ{p(x)}$. One sees clearly that $\alpha = x + \princ{p(x)} \in E$. Note
    \begin{align*}
        \overline{p}(\alpha) &= \overline{a_0} + \overline{a_1}\alpha + \overline{a_2}\alpha^2 + \cdots + \overline{a_n}\alpha^n\\
        &= (a_0 + \princ{p(x)}) + (a_1 + \princ{p(x)})(x + \princ{p(x)}) + (a_2 + \princ{p(x)})(x + \princ{p(x)})^2\\
        &\quad\quad+ \cdots + (a_n + \princ{p(x)})(x + \princ{p(x)})^n\\
        &= (a_0 + \princ{p(x)}) + (a_1x + \princ{p(x)}) + (a_2x^2 + \princ{p(x)}) + \cdots + (a_nx^n + \princ{p(x)})\\
        &= (a_0 + a_1x + a_2x^2 + \cdots + a_nx^n) + \princ{p(x)}\\
        &= p(x) + \princ{p(x)}\\
        &= 0 + \princ{p(x)}.
    \end{align*}
    Therefore, the irreducible polynomial $p(x) \in F[x]$ has a zero in $E$, after suitable conversion has been made.
\end{proof}

% TODO: Add

\section{Splitting Fields}
% TODO: Add

\section{The Derivative}
% TODO: Add

\section{Zeroes of an Irreducible Polynomial}
% TODO: Add

\newpage

\section{Problems}
% TODO: Add
