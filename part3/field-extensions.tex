\chapter{Extension Fields and Splitting Fields}
For all of the previous chapters, we have looked at `substructures', subsets of the original structure that are, themselves, of the same type. For groups, we looked at subgroups; for rings, we looked at subrings; for vector spaces, we looked at subspaces. For fields, we can go the `reverse' direction, by finding a larger field that contains the original field, called an extension field.

\section{Extension Fields}
We motivate the definition for extension fields by looking at the following example.

\begin{example}\label{example-R[x]-mod-x^2+1-is-isomorphic-to-C}
    Consider the ring $F = \R[x]/\princ{x^2+1}$. We note that $x^2 + 1$ is irreducible over $\R$; this can be easily seen as $x^2+1$ has no zeroes in $\R$ and is therefore irreducible by \myref{thrm-degree-2-or-3-irreducible-iff-has-no-zeroes}. Therefore $\princ{x^2+1}$ is a maximal ideal in $\R[x]$ (\myref{thrm-irreducible-iff-principal-ideal-is-maximal}) which means that, in fact, $F$ is a field (\myref{thrm-maximal-ideal-iff-quotient-ring-is-field}).

    We find the general form of elements in $F$. Suppose $f(x) \in \R[x]$. Then by polynomial long division (\myref{thrm-polynomial-long-division}) we may write
    \[
        f(x) = q(x)(x^2+1) + r(x)
    \]
    where $r(x) = 0$ or $\deg r(x) < \deg(x^2+1) = 2$. It follows that $r(x) = ax + b$ for some $a, b \in \R$. Note that $q(x)(x^2+1) \in \princ{x^2+1}$, so $f(x) + \princ{x^2+1} = r(x) + \princ{x^2+1}$ in $F$, meaning that
    \[
        F = \left\{ax + b + \princ{x^2+1} \vert a,b\in \R\right\}.
    \]

    Now because
    \[
        (x^2 + 1) + \princ{x^2 + 1} = 0 + \princ{x^2+1}
    \]
    it follows quickly that
    \[
        x^2 + \princ{x^2 + 1} = -1 + \princ{x^2+1},
    \]
    so, in this field, the polynomial $x^2$ is `equivalent' to -1. This means that, in this field, there is a `solution' to the equation $x^2 + 1 = 0$. It should not be too surprising, in light of this observation, that $F \cong \C$, via the map $\phi: F \to \C$ where
    \[
        ax + b + \princ{x^2 + 1} \mapsto b + ai.
    \]
    In fact, this was how Cauchy first defined complex numbers in 1847.
\end{example}

\begin{exercise}
    Verify that the field $F$ in \myref{example-R[x]-mod-x^2+1-is-isomorphic-to-C} is indeed isomorphic to $\C$ by using the map $\phi$ given. For completeness, also verify that $\phi$ is well-defined.
\end{exercise}

The idea that we can `extend' the real numbers to create a solution to $x^2 + 1 = 0$ is critical to our definition of \textbf{extension fields}.

\begin{definition}
    A field $E$ is an \textbf{extension field}\index{extension field}\index{extension!field} of a field $F$ if and only if $E$ contains a subfield isomorphic to $F$. In this case $F$ is called the \textbf{base field}\index{base field}\index{field!base}.
\end{definition}
\begin{remark}
    This definition differs from most authors' definitions of a field extension (cf. \cite[p.~511]{dummit_foote_2004}, \cite[p.~442]{artin_2011}, \cite[p.~338]{gallian_2016}, \cite[p.~260]{judson_beezer_2022}), which is that a field extension is a field $E$ that just has $F$ as a subfield.
\end{remark}

Before we answer why we use a different definition, we look at the following proposition.

\begin{proposition}
    Let $E$ and $F$ be fields. If $F \subseteq E$ then $E$ is an extension field of $F$.
\end{proposition}
\begin{proof}
    Construct the map $\phi: F \to E, x \mapsto x$. We note that for any $x,y\in F$, we have $\phi(x + y) = x + y = \phi(x) + \phi(y)$ and $\phi(xy) = xy = \phi(x)\phi(y)$, so $\phi$ is a homomorphism. Therefore, $F/\ker\phi \cong \im\phi$ by the FRIT (\myref{thrm-ring-isomorphism-1}). As fields do not have proper ideals (\myref{prop-ring-is-field-iff-no-proper-ideals}), therefore $\ker\phi = F$, which is not the case since $\phi(1) = 1 \neq 0$, or $\ker\phi = \{0\}$, which results in $F/\ker\phi \cong F$ by \myref{problem-integral-domain-iff-trivial-ideal-is-prime}. Thus $\im\phi$ is a subfield of $E$ that is isomorphic to $F$, proving that $E$ is an extension field of $F$.
\end{proof}

\begin{example}
    Consider the quadratic field
    \[
        \Q[\sqrt{2}] = \{a + b\sqrt{2} \vert a,b\in\Q\}.
    \]
    We claim that $\Q[\sqrt{2}]$ is an extension field of $\Q$. One sees clearly that any $q \in \Q$ is expressible as $q + 0\sqrt{2}$, and so $q \in \Q[\sqrt2]$. Therefore $\Q \subseteq \Q[\sqrt2]$. As $\Q$ is a field, therefore $\Q$ is a subfield of $\Q[\sqrt2]$, which in turn means $\Q[\sqrt2]$ is an extension field of $\Q$.
\end{example}

We return to why we defined field extensions in the way that we do. The reason why is because, most of the time, the extension field that we use does \textit{not} contain the base field as a subset. Rather, the extension field contains a subfield that is \textit{isomorphic} to the base field. In fact, most authors actually use our definition of extension fields -- they find a subfield of the extension field that is isomorphic to the base field (see \cite[Theorem 13.1.3]{dummit_foote_2004}, \cite[p.~339, Proof]{gallian_2016}, \cite[Example 21.2]{judson_beezer_2022}).

We have this useful proposition to help us determine whether a field is an extension field of another field.

\begin{proposition}\label{prop-extension-field-if-homomorphism-between-fields}
    Let $E$ and $F$ be fields. Then $E$ is an extension field of $F$ if there is a non-trivial homomorphism $\phi: F \to E$.
\end{proposition}
\begin{proof}
    Any homomorphism from a field is either trivial or injective (\myref{thrm-homomorphism-from-field-is-injective-or-trivial}). Since $\phi$ is non-trivial, therefore $\phi$ is injective. Thus we see $F \cong \im\phi$. As $\im\phi$ is a subfield of $E$, therefore $E$ contains a subfield isomorphic to $F$.
\end{proof}

\begin{example}
    Consider the field $F = \R[x]/\princ{x^2+1}$ in \myref{example-R[x]-mod-x^2+1-is-isomorphic-to-C}. We note that $F$ is a field extension of $\R$ by considering the map $\phi: \R \to F, r \mapsto r + \princ{x^2+1}$. We note that $\phi$ is a homomorphism since, for all $a, b \in \R$, we have
    \begin{align*}
        \phi(a + b) &= (a + b) + \princ{x^2+1}\\
        &= \left(a + \princ{x^2+1}\right) + \left(b + \princ{x^2+1}\right)\\
        &= \phi(a) + \phi(b)
    \end{align*}
    and
    \begin{align*}
        \phi(ab) &= ab + \princ{x^2+1}\\
        &= \left(a + \princ{x^2+1}\right)\left(b + \princ{x^2+1}\right)\\
        &= \phi(a)\phi(b)
    \end{align*}
    which means $\phi$ is a homomorphism. Therefore $F$ is an extension field of $\R$ by \myref{prop-extension-field-if-homomorphism-between-fields}.

    If this still feels uncomfortable, recall that we proved that $F \cong \C$ in \myref{exercise-R[x]-mod-x^2+1-is-isomorphic-to-C}. We have already shown that $\R \subseteq \C$, and since $F \cong \C$, it is natural to think that $F$ is an extension field of the real numbers.
\end{example}

\begin{definition}
    A \textbf{field extension}\index{field!extension}, denoted $E/F$ and read ``$E$ over $F$'', indicates that $E$ is an extension field over the base field $F$.
\end{definition}
\begin{remark}
    We may say ``let $E/F$ be a field extension'', which is the same as saying ``let $E$ be an extension field of the field $F$''.
\end{remark}
\begin{remark}
    This notation should not be confused with the notation for quotient rings. However, in times of doubt of notation, we elect to explicitly state what the extension field is.
\end{remark}

Returning back to our exploration of extension fields, the critical observation that $\R[x]/\princ{x^2+1}$ contains a `zero' of $x^2+1$ lead Leopold Kronecker to come up with a generalization of this fact in 1887. This result is often called the \textbf{Fundamental Theorem of Field Theory} (e.g. \cite[Theorem 20.1]{gallian_2016}, \cite[Theorem 21.5]{judson_beezer_2022}).
\begin{theorem}[Fundamental Theorem of Field Theory]\label{thrm-fundamental-theorem-of-field-theory}\index{Fundamental Theorem of Field Theory}
    Let $F$ be a field and $f(x)$ be a non-constant polynomial in $F[x]$. Then there exists a field extension $E/F$ that has a zero of $f(x)$.
\end{theorem}
\begin{proof}
    Since $F$ is a field and thus a UFD, therefore $F[x]$ is a UFD (\myref{thrm-UFD-iff-polynomial-ring-is-UFD}). So $f(x)$ has an irreducible factor, say $p(x)$. Thus it suffices to show that an irreducible polynomial $p(x) \in F[x]$ has a zero in $E$.

    We note that $\princ{p(x)}$ is a maximal ideal since $p(x)$ is irreducible (\myref{thrm-irreducible-iff-principal-ideal-is-maximal}), and therefore $E = F[x]/\princ{p(x)}$ is a field (\myref{thrm-maximal-ideal-iff-quotient-ring-is-field}). Construct the map $\phi: F \to E$ where $a \mapsto a + \princ{p(x)}$. We note that $\phi$ is a homomorphism between fields since
    \begin{align*}
        \phi(a+b) &= (a+b) + \princ{p(x)}\\
        &= (a + \princ{p(x)}) + (b + \princ{p(x)})\\
        &= \phi(a) + \phi(b)
    \end{align*}
    and
    \begin{align*}
        \phi(ab) &= ab + \princ{p(x)}\\
        &= (a+\princ{p(x)})(b+\princ{p(x)})\\
        &= \phi(a)\phi(b).
    \end{align*}
    Therefore means that $E$ is a field extension of $F$ by \myref{prop-extension-field-if-homomorphism-between-fields}.

    It remains to show that $E$ contains a zero of $p(x)$. Write
    \[
        p(x) = a_0 + a_1x + a_2x^2 + \cdots + a_nx^n
    \]
    where $a_0, a_1, \dots, a_n \in F$. In $E$, the polynomial becomes
    \[
        \overline{p}(x) = (a_0 + \princ{p(x)}) + (a_1 + \princ{p(x)})x + (a_2 + \princ{p(x)})x^2 + \cdots + (a_n + \princ{p(x)})x^n
    \]
    For brevity let $\overline{a_i} = a_0 + \princ{p(x)}$. One sees clearly that $\alpha = x + \princ{p(x)} \in E$. Note
    \begin{align*}
        \overline{p}(\alpha) &= \overline{a_0} + \overline{a_1}\alpha + \overline{a_2}\alpha^2 + \cdots + \overline{a_n}\alpha^n\\
        &= (a_0 + \princ{p(x)}) + (a_1 + \princ{p(x)})(x + \princ{p(x)}) + (a_2 + \princ{p(x)})(x + \princ{p(x)})^2\\
        &\quad\quad+ \cdots + (a_n + \princ{p(x)})(x + \princ{p(x)})^n\\
        &= (a_0 + \princ{p(x)}) + (a_1x + \princ{p(x)}) + (a_2x^2 + \princ{p(x)}) + \cdots + (a_nx^n + \princ{p(x)})\\
        &= (a_0 + a_1x + a_2x^2 + \cdots + a_nx^n) + \princ{p(x)}\\
        &= p(x) + \princ{p(x)}\\
        &= 0 + \princ{p(x)}.
    \end{align*}
    Therefore, the irreducible polynomial $p(x) \in F[x]$ has a zero in $E$, after suitable conversion has been made.
\end{proof}

\begin{example}
    Consider the irreducible polynomial $f(x) = x^2 + x + 1 \in \Z_2[x]$. To find a field extension $E/\Z_2$ such that $f(x)$ has a zero, we may take $E = \Z_2[x]/\princ{f(x)}$, which is a field of 4 elements. Then by \myref{thrm-fundamental-theorem-of-field-theory} we know that $x + \princ{f(x)}$ is a zero of the `transformed' version of $f(x)$ in $E$. Indeed, we see
    \begin{align*}
        &(1 + \princ{f(x)})(x + \princ{f(x)})^2 + (1 + \princ{f(x)})(x + \princ{f(x)}) + (1 + \princ{f(x)})\\
        &= (x^2+\princ{f(x)}) + (x + \princ{f(x)}) + (1 + \princ{f(x)})\\
        &= (x^2 + x + 1) + \princ{f(x)}\\
        &= f(x) + \princ{f(x)}\\
        &= 0 + \princ{x^2+x+1},
    \end{align*}
    so $f(x)$ has a zero in $E$.
\end{example}

\begin{example}
    Consider the polynomial $f(x) = x^5 + 2x^2 + 2x + 2 \in \Z_3[x]$. We may factor $f(x)$ into irreducibles by noting $f(x) = (x^2+1)(x^3+2x+2)$. Then \myref{thrm-fundamental-theorem-of-field-theory} tells us that the fields
    \[
        \Z_3[x]/\princ{x^2+1} \quad\text{and}\quad \Z_3[x]/\princ{x^3+2x+2}
    \]
    are both extension fields of $\Z_3$ that contain a zero of $f(x)$.
\end{example}

The Fundamental Theorem of Field Theory (\myref{thrm-fundamental-theorem-of-field-theory}) may actually be extended to integral domains. Suppose $f(x) \in D[x]$ where $D$ is an integral domain. Since the field of fractions contains a subring isomorphic to the original integral domain (\myref{thrm-field-of-fractions-contains-integral-domain}), thus we may interpret $f(x)$ as a polynomial in $\Frac{D}[x]$ instead. Then the Fundamental Theorem of Field Theory gives us the required result.

However, this does not apply to commutative rings in general, as seen in the following exercise.

\begin{exercise}
    Let $f(x) = 2x + 1$ be a polynomial in $\Z_4[x]$. Prove that no ring that contains a subring that is isomorphic to $\Z_4$ has a zero of $f(x)$.
\end{exercise}

\begin{exercise}
    Find two extension fields that contains a zero of the polynomial $x^4 + 3x^2 + 1 \in \Q[x]$.
\end{exercise}

Henceforth, to simplify the notation used for elements of the extension field, we will simply identify elements in the subset isomorphic to the base field with the actual element within the base field. Furthermore, whenever we say that an extension field $E$ \textit{contains} the base field $F$, it means that $E$ contains a subfield isomorphic to $F$.

\begin{example}
    Consider the extension field $E = \R[x]/\princ{x^2+1}$ of $\R$. The coset $1 + \princ{x^2+1}$ will henceforth be written as just 1, since the conversion from the element in $\R$ to the coset $1 + \princ{x^2+1}$ is simple -- just add the principal ideal $\princ{x^2+1}$ to the end.

    For polynomials within $\R[x]$, such as $x^2 + 2x + 3$, we may interpret $x^2 + 2x + 3$ as a polynomial in $E$, just remembering that the coefficients of that polynomial are actually the cosets $1 + \princ{x^2+1}$, $2 + \princ{x^2+1}$, and $3 + \princ{x^2+1}$ respectively.
\end{example}

\section{Splitting Fields}
To motivate our next definition, we return back to the extension field $E = \R[x]/\princ{x^2+1}$ of $\R$.
\begin{example}
    We note by the Fundamental Theorem of Field Theory (\myref{thrm-fundamental-theorem-of-field-theory}) that $\alpha = x + \princ{x^2+1}$ is a zero of the polynomial $x^2+1$ in $E$. One also sees that $-\alpha$ is also a zero of $x^2+1$, since
    \begin{align*}
        (-\alpha)^2 + 1 &= (-x + \princ{x^2+1})^2 + 1\\
        &= x^2 + 1 + \princ{x^2+1}\\
        &= 0 + \princ{x^2+1}.
    \end{align*}
    By the Factor Theorem (\myref{corollary-factor-theorem}), it should follow that both $x - \alpha$ and $x + \alpha$ are factors of $x^2 + 1$ in $E$. Let's try and verify this. One sees
    \begin{align*}
        (x-\alpha)(x+\alpha) &= x^2 - \alpha^2\\
        &= x^2 - (x^2 + \princ{x^2+1})\\
        &= x^2 - (-1 + \princ{x^2+1}) & (\text{since } x^2 + \princ{x^2+1} = -1 + \princ{x^2+1})\\
        &= x^2 + 1 + \princ{x^2+1}\\
        &= x^2 + 1 & (\text{we identity elements without the coset})
    \end{align*}
    so we indeed can factor $x^2 + 1$ into two linear factors within the extension field $E$.
\end{example}

One may wonder if polynomials from any field can be `split' nicely into linear factors within a certain extension field. In fact, they can -- these extension fields are called splitting fields.

Before we can define what a splitting field is, we need to look at a special kind of extension field.

\begin{definition}
    Let $F$ be a field and $E/F$ be a field extension. Let $S$ be a subset of $E$. Then $F(S)$, read as ``$F$ adjoin $S$'', is the smallest subfield of $E$ that contains $F$ and the set $S$.

    If $S$ is a finite set, say $S = \{a_1, a_2, \dots, a_n\}$, then we write $F(a_1, a_2, \dots, a_n)$, and say that $F(S)$ is finitely generated over $F$.
\end{definition}

\begin{proposition}
    $F(S)$ is the intersection of all subfields of $E$ that contain both the field $F$ and the subset $S$.
\end{proposition}
\begin{proof}
    Let $K$ be the intersection of all subfields of $E$ that contain both $F$ and $S$.

    By \myref{prop-intersection-of-subfields-is-subfield} the intersection of subfields is a subfield, so $K$ is indeed a subfield of $E$. Note $K$ that it contains both $F$ and $S$ by definition of $K$. Since $F(S)$ contains both $F$ and the set $S$ by definition, thus we have $K \subseteq F(S)$.

    Also, since $F(S)$ contains both $F$ and $S$, it must be inside the intersection that is $K$, which means that $F(S) \subseteq K$.

    Therefore we see $K = F(S)$.
\end{proof}

There is a name for extension fields of the form $F(\alpha)$ where $\alpha$ is a single element.

\begin{definition}
    Let $F$ be a field, $E/F$ a field extension, and $\alpha \in E$. Then $F(\alpha)$ is called a \textbf{simple extension}\index{extension!simple} and $\alpha$ is called the \textbf{primitive element}\index{primitive element}\index{extension!simple!primitive element}.
\end{definition}

\begin{example}
    $\Q(i)$, $\R(1+2i)$, $\Q(\pi)$, and $\Q(\sqrt 2)$ are all simple extensions with the primitive elements of $i$, $1+2i$, $\pi$, and $\sqrt 2$ respectively.
\end{example}

A general form for simple extensions for specific types of primitive elements, such as $\Q(i)$ and $\Q(\sqrt 2)$, exists, and will simplify calculation drastically.

\begin{theorem}\label{thrm-simple-extension-isomorphism}
    Let $F$ be a field and let $p(x) \in F[x]$ be irreducible over $F$. If $\alpha \in E$ is a zero of $p(x)$ in some field extension $E/F$, then
    \[
        F(\alpha) \cong F[x] / \princ{p(x)}.
    \]
    Furthermore if $\deg p(x) = n$ then
    \[
        F(\alpha) \cong F[\alpha]
    \]
    and every element in $F[\alpha]$ can be expressed uniquely as $a_0 + a_1\alpha + a_2\alpha^2 + \cdots + a_{n-1}\alpha^{n-1}$ where $a_i \in F$.
\end{theorem}
\begin{proof}
    Consider the map $\phi: F[x] \to F(\alpha)$ where $f(x) \mapsto f(\alpha)$. We show that $\phi$ is a homomorphism, then find its image and kernel.
    \begin{itemize}
        \item \textbf{Homomorphism}: For any $f(x), g(x) \in F[x]$ we see
        \[
            \phi(f(x) + g(x)) = f(\alpha) + g(\alpha) = \phi(f(x)) + \phi(g(x))
        \]
        and
        \[
            \phi(f(x)g(x)) = f(\alpha)g(\alpha) = \phi(f(x))\phi(g(x))
        \]
        so $\phi$ is a homomorphism.

        \item \textbf{Image}: We note that $\im\phi$ is a subfield of $F(\alpha)$. Therefore $\im\phi$ contains both $F$ and $\alpha$. But $F(\alpha)$ is defined to be the smallest subfield of $E$ that contains both $F$ and $\alpha$, which means $\im\phi = F(\alpha)$.

        \item \textbf{Kernel}: We claim that $\ker\phi = \princ{p(x)}$. Since $p(\alpha) = 0$ by definition of $\alpha$, therefore $\princ{p(x)} \subseteq \ker\phi$. On the other hand, notice that $\princ{p(x)}$ is a maximal ideal (\myref{thrm-irreducible-iff-principal-ideal-is-maximal}), and since $\ker\phi \neq F[x]$ (as the constant polynomial 1 does not map to 0), therefore $\ker\phi = \princ{p(x)}$, by definition of a maximal ideal.
    \end{itemize}

    Hence, by the FRIT (\myref{thrm-ring-isomorphism-1}),
    \[
        F[x]/\princ{p(x)} \cong F(\alpha).
    \]

    Now, if $f(x) + \princ{p(x)} \in F[x]/\princ{p(x)}$ then we may use polynomial long division (\myref{thrm-polynomial-long-division}) to write $f(x) = q(x)p(x) + r(x)$ where $r(x)$ is a \textit{unique} polynomial where $r(x) = 0$ or $\deg r(x) < \deg p(x) = n$. Hence if $r(x) = a_0 + a_1x + \cdots + a_{n-1}x^{n-1}$ we see
    \[
        f(x) + \princ{p(x)} = (a_0 + a_1x + \cdots + a_{n-1}x^{n-1}) + \princ{p(x)}
    \]
    where $r(x)$ is unique. Now we show that the map $\psi: F[x]/\princ{p(x)} \to F[\alpha]$ given by $\psi(f(x) + \princ{p(x)}) = f(\alpha)$ is a well-defined isomorphism.
    \begin{itemize}
        \item \textbf{Well-defined}: First, by using the above observation, for any $f(x) + \princ{p(x)}, g(x) + \princ{p(x)} \in F[x]/\princ{p(x)}$, find $r_f(x), r_g(x) \in F[x]$ where
        \begin{align*}
            f(x) + \princ{p(x)} &= r_f(x) + \princ{p(x)}\\
            g(x) + \princ{p(x)} &= r_g(x) + \princ{p(x)}
        \end{align*}
        and so if $f(x) + \princ{p(x)} = g(x) + \princ{p(x)}$ we see $r_f(x) + \princ{p(x)} = r_g(x) + \princ{p(x)}$. Hence $r_f(x) - r_g(x) \in \princ{p(x)}$. Note elements in $\princ{p(x)}$ have degree of at least $p(x)$, or is the zero polynomial. Therefore $r_f(x) - r_g(x) = 0$, meaning $r_f(x) = r_g(x)$ and thus
        \[
            \psi(f(x) + \princ{p(x)}) = r_f(\alpha) = r_g(\alpha) = \psi(g(x) + \princ{p(x)}),
        \]
        showing that $\psi$ is well-defined.

        \item \textbf{Homomorphism}: For any $f(x) + \princ{p(x)}, g(x) + \princ{p(x)} \in F[x]/\princ{p(x)}$ we see that
        \begin{align*}
            \psi((f(x) + \princ{p(x)}) + (g(x) + \princ{p(x)})) &= \psi((f(x) + g(x)) + \princ{p(x)})\\
            &= f(\alpha) + g(\alpha)\\
            &= \psi(f(x) + \princ{p(x)}) + \phi(g(x) + \princ{p(x)})
        \end{align*}
        and
        \begin{align*}
            \psi((f(x) + \princ{p(x)})(g(x) + \princ{p(x)})) &= \psi((f(x)g(x)) + \princ{p(x)})\\
            &= f(\alpha)g(\alpha)\\
            &= \psi(f(x) + \princ{p(x)})\phi(g(x) + \princ{p(x)})
        \end{align*}
        which shows that $\psi$ is a homomorphism.

        \item \textbf{Injective}: Since $\psi$ is non-trivial we know that $\psi$ is injective by \myref{thrm-homomorphism-from-field-is-injective-or-trivial}.

        \item \textbf{Surjective}: Suppose $f(\alpha) = a_0 + a_1\alpha + \cdots + a_k\alpha^k \in F[\alpha]$. Let $p(x) = b_0 + b_1x + \cdots + b_{n-1}x^{n-1}$. Then by polynomial long division we see
        \[
            f(\alpha) = q(\alpha)p(\alpha) + r(\alpha).
        \]
        Where $r(x) = 0$ or $\deg r(x) < \deg p(x) = n$. Since $\alpha$ is a zero of $p(\alpha)$ thus $f(\alpha) = r(\alpha)$. Therefore $\psi(r(x) + \princ{p(x)}) = r(\alpha) = f(\alpha)$, which shows that $\psi$ is surjective.
    \end{itemize}
    Hence, $F[x]/\princ{p(x)} \cong F[\alpha]$, which proves that
    \[
        F(\alpha) \cong F[\alpha]
    \]
    and uniqueness is inherited from the fact that $r(\alpha)$ is unique.
\end{proof}
\begin{corollary}
    Let $F$ be a field and let $p(x) \in F[x]$ be irreducible over $F$. If $\alpha \in E$ is a zero of $p(x)$ in some field extension $E/F$, then $F(\alpha) = F[\alpha]$.
\end{corollary}
\begin{proof}
    Note that $F[\alpha] \subseteq E$ contains both $F$ and $\alpha$. Since $F(\alpha)$ is the intersection of subfields that contains both $F$ and $\alpha$, thus $F(\alpha) \subseteq F[\alpha]$. But \myref{thrm-simple-extension-isomorphism} proves that $F(\alpha) \cong F[\alpha]$. Thus $F(\alpha) = F[\alpha]$.
\end{proof}

\begin{example}
    $\Q(i) = \Q[i] = \{a + bi \vert a, b \in \Q\}$.
\end{example}
\begin{example}
    $\Q(\sqrt 2) = \Q[\sqrt{2}] = \{a + b\sqrt{2} \vert a, b \in \Q\}$.
\end{example}

Henceforth, for elements $\alpha$ that are roots of a polynomial with coefficients in $F$, we will write $F(\alpha)$ instead of $F[\alpha]$ to indicate that we are working with fields.

We note \myref{thrm-simple-extension-isomorphism} shows that an element of $F(\alpha)$, where $\alpha$ is the zero of some irreducible polynomial over $F$, takes the form $a_0+a_1\alpha + \cdots + a_{n-1}\alpha^{n-1}$. Since this is, in fact, a unique representation, therefore, using the language of vector spaces, the set
\[
    \{1, \alpha, \alpha^2, \dots, \alpha^{n-1}\}
\]
forms a basis for $F(\alpha)$, and thus $F(\alpha)$ has dimension $n$.

\begin{example}
    Consider $f(x) = x^6 - 2 \in \Q[x]$. Eisenstein's criteria (\myref{thrm-eisenstein-criterion}) with the prime 2 tells us that $f(x)$ is irreducible over $\Q$. Since $\sqrt[6]2 \in \R$ is a zero of $f(x)$, we know that $\Q(\sqrt[6]2)$ has 6 basis vectors, so it has the underlying set
    \[
        \left\{a_0 + a_12^{\frac16} + a_22^{\frac26} + a_32^{\frac36} + a_42^{\frac46} + a_52^{\frac56} \vert a_i \in \Q\right\}.
    \]
\end{example}

What about extensions of the form $F(a_1, a_2, \dots, a_n)$? We also have this useful result that simplifies calculation.
\begin{proposition}\label{prop-field-generated-by-S-inductive-definition}
    Let $F$ be a field, $E/F$ be a field extension, and $a_1, a_2, \dots, a_n \in E$ where $n \geq 2$. Then
    \[
        F(a_1, a_2, \dots, a_n) = F(a_1, a_2, \dots, a_{n-1})(a_n).
    \]
\end{proposition}
\begin{remark}
    We interpret $F(a_1, a_2, \dots, a_{n-1})(a_n)$ as the simple extension that contains the field $F(a_1, a_2, \dots, a_{n-1})$ and the element $a_n$.
\end{remark}
\begin{proof}
    See \myref{exercise-field-generated-by-S-inductive-definition} (later).
\end{proof}

\begin{example}
    Note that $i$ and $\sqrt 2$ are zeroes of the irreducible polynomials $x^2 + 1$ and $x^2 - 2$ respectively. Thus the extension field
    \begin{align*}
        \Q(\sqrt 2, i) &= \Q(\sqrt 2)(i)\\
        &= \{u + vi \vert u,v \in \Q(\sqrt2)\}\\
        &= \{(a+b\sqrt2) + (c+d\sqrt2)i \vert a, b, c, d \in \Q\}\\
        &= \{a + b\sqrt2 + ci + d\sqrt2i \vert a, b, c, d \in \Q\}.
    \end{align*}
\end{example}

\begin{exercise}\label{exercise-field-generated-by-S-inductive-definition}
    Prove \myref{prop-field-generated-by-S-inductive-definition}.
\end{exercise}

\newpage

With all these results, we can finally define \textbf{splitting fields}.

\begin{definition}
    Let $F$ be a field, $E/F$ be a field extension, and $f(x) \in F[x]$ be a non-constant polynomial. We say that $f(x)$ \textbf{splits}\index{polynomial!splits} in $E$ if and only if there exist an element $a \in F$ and elements $a_1, a_2, \dots, a_n \in E$ such that
    \[
        f(x) = a(x-a_1)(x-a_2)\cdots(x-a_n).
    \]
    We call $E$ a \textbf{splitting field for $f(x)$ over $F$}\index{splitting field}\index{field!splitting} if
    \[
        E = F(a_1, a_2, \dots, a_n).
    \]
\end{definition}
\begin{remark}
    In other words, the splitting field of $f(x)$ over $F$ is a smallest extension field that allows $f(x)$ to split.
\end{remark}

We note that the splitting field for a polynomial not only depends on the polynomial itself, but also the base field.

\begin{example}
    Consider the polynomial $x^2 + 1$. If we interpret $x^2 + 1$ being in $\R[x]$, then naturally the splitting field of $x^2 + 1$ is $\C$, since $x^2 + 1 = (x-i)(x+i)$.

    However, if we think of $x^2+1$ as being a polynomial in $\Q[x]$, then the splitting field is $\Q(i, -i)$. In fact, $\Q(i, -i) = \Q(i)$. One sees that $\Q(i) \neq \C$ since $\C$ contains irrational numbers (such as $\sqrt2$) while $\Q(i)$ must at least contain $\Q$ and so does not contain irrational numbers.
\end{example}

\begin{example}
    Consider the polynomial $x^2 - 2 \in \Q[x]$. Clearly $x^2 - 2$ splits in $\R$ since $x^2 - 2 = (x - \sqrt2)(x + \sqrt2)$, but the splitting field of $x^2 - 2 \in \Q[x]$ is $\Q(\sqrt2, -\sqrt2)$ which is $\Q(\sqrt 2)$.
\end{example}

Does a splitting field for a polynomial over a given field always exist? We answer in the affirmative by proving the following theorem.

\begin{theorem}
    Let $F$ be a field and $f(x)$ be a non-constant polynomial in $F[x]$. Then there exists a splitting field $E$ for $f(x)$ over $F$.
\end{theorem}
\begin{proof}
    Let $f(x)$ have degree $n$. We use strong induction on $n$.

    When $n = 1$, we know $f(x)$ is of the form $ax + b$, and so we may write
    \[
        f(x) = a\left(x - \left(\frac ba\right)\right),
    \]
    which means $f(x)$ splits over the original field $F$.

    Now assume that, for some positive integer $k$, there exists a splitting field $E$ for any field $F$ and for any polynomial in $F[x]$ with degree that is at most $k$. We show that any polynomial of degree $k + 1$ in the field $F[x]$ has a splitting field.

    Let $f(x) \in F[x]$ be an arbitrary polynomial with degree $k + 1$. By the Fundamental Theorem of Field Theory (\myref{thrm-fundamental-theorem-of-field-theory}), there exists an extension field $E$ in which $f(x)$ has a zero. Suppose this zero is $a_1$, meaning that we may write
    \[
        f(x) = (x - a_1)g(x)
    \]
    for some $g(x) \in E[x]$, by the Factor Theorem (\myref{corollary-factor-theorem}). We note that $\deg g(x) < \deg f(x) = k + 1$ which means $\deg g(x) \leq k$. By the induction hypothesis, there is a splitting field of $g(x)$, say $L$, that contains $E$ and all the zeroes of $g(x)$, which are, say, $a_2, a_3, \dots, a_k, a_{k+1}$. Since $L$ contains $E$ we may also find $a_1 \in L$ (of course, this $a_1$ needs to be converted into the appropriate form within $L$). One then sees that $F(a_1, a_2, \dots, a_{k+1})$ is a splitting field for $f(x)$ over $F$, which shows that the theorem holds for $k + 1$ as well.

    The theorem is proven by induction.
\end{proof}

\begin{example}
    Consider $f(x) = x^4 - x^2 - 2 = (x^2+1)(x^2 - 2)$ over $\Q$. One sees that the zeroes of $f(x)$ in $\C$ are $\pm i$ and $\pm\sqrt2$. Thus a splitting field of $f(x)$ over $\Q$ is
    \begin{align*}
        \Q(i, -i, \sqrt2, -\sqrt2) &= \Q(i, \sqrt2)\\
        &= \{a + b\sqrt2 + ci + d\sqrt2i \vert a, b, c, d \in \Q\}.
    \end{align*}
\end{example}

\begin{exercise}\label{exercise-splitting-field-sqrt2-sqrt3}
    Let $f(x) = x^4 - 5x^2 + 6 \in \Q[x]$.
    \begin{partquestions}{\roman*}
        \item Show that a splitting field of $f(x)$ over $\Q$ is $\Q(\sqrt2)(\sqrt3)$.
        \item Show that $p(x) = x^4 - 10x + 1$ is an irreducible polynomial over $\Q$ with a zero of $\sqrt2 + \sqrt3$.
        \item Hence show that $\Q(\sqrt2 + \sqrt3)$ is a splitting field of $f(x)$ over $\Q$.
    \end{partquestions}
\end{exercise}

Looking back at the polynomial $x^2+1$, we note that both $\R(i)$ and $\R[x]/\princ{x^2+1}$ split $x^2+1$. But actually these two fields are not that different. We end this section by proving that a splitting field of a polynomial over a given field is unique up to isomorphism.

Before we can prove this, we introduce some terminology and notation, and a preliminary result.

\begin{definition}
    Let $F$ and $F'$ be fields and $\phi: F \to F'$ be a homomorphism. Suppose $E/F$ and $E'/F'$ are field extensions, and suppose $\psi: E \to E'$ is a homomorphism. Then we say \textbf{$\psi$ agrees with $\phi$ on $F$}\index{agrees with} if and only if $\psi(x) = \phi(x)$ for all $x \in F$.
\end{definition}

\begin{definition}
    For given fields $F$ and $F'$, a polynomial $f(x) = a_0 + a_1x + \cdots + a_nx^n \in F[x]$, and an isomorphism $\phi: F \to F'$, we define
    \[
        \phi(f(x)) = \phi(a_0) + \phi(a_1)x + \cdots + \phi(a_n)x^n,
    \]
    which is a polynomial in $F'[x]$.
\end{definition}

\newpage

\begin{lemma}\label{lemma-isomorphism-extension}
    Let $F$ and $F'$ be fields, $p(x) \in F[x]$ be irreducible over $F$, and let $\alpha$ be a zero of $p(x)$ in some field extension $E/F$. If $\phi: F \to F'$ is an isomorphism and $\beta$ is a zero of $p(x)$ in some field extension $E'/F'$, then there exists an isomorphism $\overline{\phi}: F(\alpha) \to F'(\beta)$ that agrees with $\phi$ on $F$ and maps $\alpha$ to $\beta$.
\end{lemma}
\begin{proof}[Proof (cf. {\cite[Lemma 21.32]{judson_beezer_2022}})]
    We note that, by \myref{thrm-simple-extension-isomorphism}, if $\deg p(x) = n$, we have that both $F(\alpha)$ and $F'(\beta)$ have dimension $n$ with bases
    \[
        \{1, \alpha, \alpha^2, \dots, \alpha^{n-1}\} \quad\text{and}\quad\{1, \beta, \beta^2, \dots, \beta^{n-1}\}
    \]
    respectively. Define the map $\overline{\phi}: F(\alpha) \to F'(\beta)$ where
    \[
        \overline{\phi}\left(a_0 + a_1\alpha + \cdots + a_{n-1}\alpha^{n-1}\right) = \phi(a_0) + \phi(a_1)\beta + \cdots + \phi(a_{n-1})\beta^{n-1}.
    \]
    We show that this is an isomorphism.
    \begin{itemize}
        \item \textbf{Homomorphism}: Let $a_0 + a_1\alpha + \cdots + a_{n-1}\alpha^{n-1}$ and $b_0 + b_1\alpha + \cdots + b_{n-1}\alpha^{n-1}$ be elements in $F(\alpha)$. Note
        \begin{align*}
            \overline{\phi}\left(\sum_{i=0}^na_i\alpha^i + \sum_{i=0}^nb_i\alpha^i\right) &=\overline{\phi}\left(\sum_{i=0}^n(a_i+b_i)\alpha^i\right)\\
            &=\sum_{i=0}^n\phi(a_i+b_i)\beta^i\\
            &= \sum_{i=0}^n\phi(a_i)\beta^i + \sum_{i=0}^n\phi(b_i)\beta^i\\
            &= \overline{\phi}\left(\sum_{i=0}^na_i\alpha^i\right) + \overline{\phi}\left(\sum_{i=0}^nb_i\alpha^i\right).
        \end{align*}
        Also,
        \begin{align*}
            \overline{\phi}\left(\left(\sum_{i=0}^na_i\alpha^i\right)\left(\sum_{i=0}^nb_i\alpha^i\right)\right) &= \overline{\phi}\left(\sum_{k=0}^{2n}\left(\left(\sum_{i=0}^ka_ib_{k-i}\right)\alpha^i\right)\right)\\
            &= \sum_{k=0}^{2n}\left(\phi\left(\sum_{i=0}^ka_ib_{k-i}\right)\beta^i\right)\\
            &= \sum_{k=0}^{2n}\left(\left(\sum_{i=0}^k\phi(a_i)\phi(b_{k-i})\right)\beta^i\right)\\
            &=\left(\sum_{i=0}^n\phi(a_i)\beta^i\right)\left(\sum_{i=0}^n\phi(b_i)\beta^i\right)\\
            &= \overline{\phi}\left(\sum_{i=0}^na_i\alpha^i\right)\overline{\phi}\left(\sum_{i=0}^nb_i\alpha^i\right)
        \end{align*}
        so $\overline{\phi}$ is a homomorphism.

        \item \textbf{Injective}: Since $\overline{\phi}$ is non-trivial it is thus injective (\myref{thrm-homomorphism-from-field-is-injective-or-trivial}).

        \item \textbf{Surjective}: Suppose $b_0 + b_1\beta + \cdots + b_{n-1}\beta^{n-1} \in F'(\beta)$. Since $\phi$ is an isomorphism, for any $b_i \in F'$, there is a pre-image $a_i \in F$ such that $\phi(a_i) = b_i$. Consequently
        \[
            \overline{\phi}\left(a_0 + a_1\alpha + \cdots + a_{n-1}\alpha^{n-1}\right) = b_0 + b_1\beta + \cdots + b_{n-1}\beta^{n-1}
        \]
        so any element in $F'(\beta)$ has a pre-image in $F(\alpha)$.
    \end{itemize}
    Therefore $\overline{\phi}$ is an isomorphism.

    Finally, notice for any $r \in F$ that $\overline{\phi}(r) = \phi(r)$, so $\overline{\phi}$ agrees with $\phi$ on $F$. Also, clearly, $\overline{\phi}(\alpha) = \beta$. Therefore this isomorphism satisfies the required properties.
\end{proof}

We can now prove the following theorem.

\begin{theorem}\label{thrm-isomorphism-extension}
    Let $F$ and $F'$ be fields, $\phi: F \to F'$ be an isomorphism, and $f(x) \in F[x]$. If $E$ is a splitting field of $f(x)$ over $F$ and $E'$ is a splitting field of $\phi(f(x))$ over $F'$, then there exists an isomorphism from $E$ to $E'$ that agrees with $\phi$ on $F$.
\end{theorem}
\begin{proof}[Proof (see {\cite[Theorem 20.4]{gallian_2016}})]
    Let $\deg f(x) = n$; we induct on $n$.

    When $n = 1$, the polynomial $f(x)$ splits over the original field $F$. Similarly $\phi(f(x))$ splits over $F'$. Hence $E = F$ and $E' = F'$ and thus $\phi$ is the desired isomorphism.

    Now assume that the result holds for all non-constant polynomials of degree of at most $k$. We show the result holds for a polynomial of degree $k + 1$.

    Since $\deg f(x) = k+1$, we can find an irreducible factor $p(x) \in F[x]$ of $f(x)$. Let $\alpha$ be a zero of $p(x)$ in $E$ and let $\beta$ be a zero of $\phi(p(x))$ in $E'$. By \myref{lemma-isomorphism-extension} there exists an isomorphism $\psi: F(\alpha) \to F'(\beta)$ that agrees with $\phi$ on $F$ and maps $\alpha$ to $\beta$.

    Now since $\alpha$ is a zero of the irreducible factor $p(x)$ of $f(x)$, thus $\alpha$ is a zero of $f(x)$. We may hence write
    \[
        f(x) = (x-\alpha)g(x)
    \]
    by Factor Theorem (\myref{corollary-factor-theorem}), where $g(x) \in F(\alpha)[x]$. Since $E$ is a splitting field of $f(x)$ over $F$, thus $E$ must also be a splitting field of $g(x)$ over $F$; likewise $E'$ is a splitting field of $\psi(g(x))$. As $\deg g(x) < \deg f(x) = k+1$, so $\deg g(x) \leq k$ and thus there must exist an isomorphism $\sigma: E \to E'$ that agrees with $\psi$ on $F(\alpha)$ by the induction hypothesis. Consequently, since $\psi$ agrees with $\phi$ on $F$, therefore $\sigma$ also agrees with $\phi$ on $F$. Therefore $\sigma$ is the required isomorphism, proving the statement for the $k+1$ case.
\end{proof}

The uniqueness of splitting fields for a given polynomial over a given field is a natural corollary of the theorem.

\begin{corollary}\label{corollary-splitting-field-unique-up-to-isomorphism}
    Let $F$ be a field and $f(x) \in F[x]$. Then any two splitting fields of $f(x)$ over $F$ are isomorphic.
\end{corollary}
\begin{proof}
    Suppose $E$ and $E'$ are splitting fields of $f(x)$ over $F$. Then the result follows immediately from \myref{thrm-isomorphism-extension} by letting $\phi = \id$, i.e. the identity endomorphism from $F$ to $F$.
\end{proof}

In light of \myref{corollary-splitting-field-unique-up-to-isomorphism}, we may refer to ``the'' splitting field of $f(x)$ over $F$ without any ambiguity.

\begin{example}
    Both $\R(i)$ and $\R[x]/\princ{x^2+1}$ split $x^2+1$, but \myref{corollary-splitting-field-unique-up-to-isomorphism} tells us that $\R(i) \cong \R[x]/\princ{x^2+1}$. In fact, $R(i)$ is just the complex numbers $\C$, and we already proved that $\R[x]/\princ{x^2+1} \cong \C$ in \myref{example-R[x]-mod-x^2+1-is-isomorphic-to-C}. So \textit{the} splitting field of $x^2+1$ over $\R$ is $\R(i)$.
\end{example}

\section{The Formal Derivative}
Before we tackle the zeroes of irreducible polynomials, we borrow a tool whose origins are from calculus.

\begin{definition}
    Let $F$ be a field and $f(x) = a_0 + a_1x + a_2x^2 + a_3x^3 + \cdots + a_nx^n \in F[x]$. Then the \textbf{formal derivative}\index{formal derivative} (or just \textbf{derivative}\index{derivative}) of $f(x)$, denoted $f'(x)$, is
    \[
        a_1 + 2a_2x + 3a_3x^2 + \cdots + na_nx^{n-1}.
    \]
\end{definition}

Notice that in our definition we did not use the notion of a limit, unlike in a standard calculus course.

\begin{example}
    Let $f(x) = x^2 + 2x + 3 \in \Q[x]$. Then $f'(x) = 2x + 2$.
\end{example}

\begin{example}
    $\left((x-1)(x+2)(x-3)(x+4)\right)' = (x^4 + 2x^3 - 13x^2 - 14x + 24)' = 4x^3 + 6x^2 - 26x - 14$.
\end{example}

\begin{example}
    One sees that $(x^3 + 2x^2 + 3x + 4)' = 3x^2 + 4x + 3$. But we also have $(x^3 + 2x^2 + 3x + 12345)' = 3x^2 + 4x + 3$. So multiple polynomials can have the same derivative.
\end{example}

Those with a background in calculus would be familiar with the following properties of the derivative. In the following, let $F$ be a field, $f(x), g(x) \in F[x]$, and $k \in F$.

\begin{proposition}
    $(kf(x))' = kf'(x)$.
\end{proposition}
\begin{proof}
    Let $f(x) = a_0 + a_1x + \cdots + a_nx^n$. Then
    \begin{align*}
        (kf(x))' &= (k(a_0 + a_1x + \cdots + a_nx^n))'\\
        &= (ka_0 + (ka_1)x + \cdots + (ka_n)x^n)'\\
        &= ka_1 + \cdots + kna_nx^{n-1}\\
        &= k(a_1 + \cdots + na_nx^{n-1})\\
        &= kf'(x).\qedhere
    \end{align*}
\end{proof}

\begin{proposition}
    $(f(x) + g(x))' = f'(x) + g'(x)$.
\end{proposition}
\begin{proof}
    See \myref{exercise-derivative-sum-rule} (later).
\end{proof}

\begin{remark}
    In fact, what these first two properties show is that the derivative is a linear transformation from $F[x]$ to $F[x]$.
\end{remark}

\begin{proposition}
    $(f(x)g(x))' = f(x)g'(x) + f'(x)g(x)$.
\end{proposition}
\begin{proof}
    We first prove the fact that
    \[
        (x^nf(x))' = nx^{n-1}f(x) + x^nf'(x)
    \]
    for all polynomials $f(x) \in F[x]$ and positive integers $n$ using strong induction on $n$.

    When $n = 1$, one sees for $f(x) = a_0 + a_1x + \cdots + a_mx^m$ that
    \begin{align*}
        (xf(x))' &= \left(a_0x + a_1x^2 + a_2x^3 + \cdots + a_mx^{m+1}\right)'\\
        &= a_0 + 2a_1x + 3a_2x^2 + \cdots + (m+1)a_mx^m\\
        &= \left(a_0 + a_1x + a_2x^2 + \cdots + a_mx^m\right) + \left(a_1x + 2a_2x^2 + \cdots + ma_mx^m\right)\\
        &= \left(a_0 + a_1x + a_2x^2 + \cdots + a_mx^m\right) + x\left(a_1 + 2a_2x + \cdots + ma_mx^{m-1}\right)\\
        &= f(x) + xf'(x)
    \end{align*}
    which shows that the $n = 1$ case holds.

    Now assume that $(x^rf(x))' = rx^{r-1}f(x) + x^rf'(x)$ for all polynomials $f(x) \in F[x]$ and for all $r \leq k$ for some positive integer $k$. We note that
    \begin{align*}
        (x^{k+1}f(x))' &= \left(x\left(x^kf(x)\right)\right)'\\
        &= \left(x^kf(x)\right) + x\left(x^kf(x)\right)' & (\text{Case where }r=1)\\
        &= x^kf(x) + x\left(kx^{k-1}f(x) + x^kf'(x)\right) & (\text{Case where }r=k)\\
        &= x^kf(x) + kx^kf(x) + x^{k+1}f'(x)\\
        &= (k+1)x^kf(x) + x^{k+1}f'(x)
    \end{align*}
    so this holds for $k + 1$ as well. Hence by induction we have proven that $(x^nf(x))' = nx^{n-1}f(x) + x^nf'(x)$ for all positive integers $n$.

    Finally note that for any $f(x), g(x) \in F[x]$ where $f(x) = a_0 + a_1x + a_2x^2 + \cdots + a_mx^m$ that
    \begin{align*}
        &(f(x)g(x))'\\
        &= \left((a_0 + a_1x + a_2x^2 + \cdots + a_mx^m)g(x)\right)'\\
        &= \left(a_0g(x) + a_1xg(x) + a_2x^2g(x) + \cdots + a_mx^mg(x)\right)'\\
        &= (a_0g(x))' + (a_1xg(x))' + (a_2x^2g(x))' + \cdots + (a_mx^mg(x))'\\
        &= a_0(g(x))' + a_1(xg(x))' + a_2(x^2g(x))' + \cdots + a_m(x^mg(x))'\\
        &= a_0g'(x) + a_1(g(x) + xg'(x)) + a_2(2xg(x) + x^2g'(x)) + \cdots + a_m(mx^{m-1}g(x)\\
        &\quad\quad+ x^mg'(x))\\
        &= (a_0g'(x) + a_1xg'(x) + a_2x^2g'(x) + \cdots + a_mx^mg'(x)) + (a_1g(x) + 2a_2xg(x) + \cdots\\
        &\quad\quad+ ma_mx^{m-1}g(x))\\
        &= (a_0 + a_1x + a_2x^2 + \cdots + a_mx^m)g'(x) + (a_1 + 2a_2x + \cdots + ma_mx^{m-1})g(x)\\
        &= f(x)g'(x) + f'(x)g(x)
    \end{align*}
    which proves the proposition.
\end{proof}

\begin{exercise}\label{exercise-derivative-sum-rule}
    Prove for any field $F$ and any two polynomials $f(x), g(x) \in F[x]$ that $(f(x)+g(x))' = f'(x) + g'(x)$.
\end{exercise}

\begin{exercise}
    Prove for any field $F$ and any polynomial $f(x) \in F[x]$ that
    \[
        \left((f(x))^n\right)' = n(f(x))^{n-1}f'(x),
    \]
    where $n$ is a positive integer.
\end{exercise}

\section{Polynomial GCD}
The investigation into the zeroes of irreducible polynomials also requires the generalization of the GCD to polynomials.

\begin{definition}
    A \textbf{monic polynomial}\index{polynomial!monic} is a non-zero polynomial which has a leading coefficient of 1. In other words, a monic polynomial takes the form
    \[
        c_0 + c_1x + c_2x^2 + \cdots + c_{n-1}x^{n-1} + x^n
    \]
    where $n \geq 0$.
\end{definition}

\newpage

\begin{definition}
    Let $F$ be a field. Given $f(x), g(x) \in F[x]$, a monic polynomial $d(x) \in F[x]$ is said to be a \textbf{greatest common divisor (GCD)}\index{greatest common divisor!polynomial}\index{GCD!polynomial} of $f(x)$ and $g(x)$ if and only if $d(x)$ divides both $f(x)$ and $g(x)$ and if another polynomial $c(x) \in F[x]$ divides both $f(x)$ and $g(x)$ then $c(x)$ divides $d(x)$.

    We denote a GCD of $f(x)$ and $g(x)$ by $\gcd(f(x), g(x))$.
\end{definition}

We note an analogue of B\'ezout's lemma (\myref{lemma-bezout}) for the polynomial GCD.

\begin{theorem}\label{thrm-bezout-lemma-for-polynomials}
    Let $F$ be a polynomial. Let $f(x), g(x) \in F[x]$, and let $d(x)$ be a GCD of $f(x)$ and $g(x)$. Then there are polynomials $a(x), b(x) \in F[x]$ such that
    \[
        d(x) = a(x)f(x) + b(x)g(x).
    \]
    Furthermore the GCD of $f(x)$ and $g(x)$ is unique.
\end{theorem}
\begin{proof}
    Consider the set
    \[
        S = \left\{u(x)f(x) + v(x)g(x) \vert u(x), v(x) \in F[x]\right\}.
    \]
    Let $d(x) \in S$ be a monic polynomial of minimal degree. Take $a(x),b(x) \in F[x]$ such that $d(x) = a(x)f(x) + b(x)g(x)$.

    We show $d(x)$ divides $f(x)$. Performing polynomial long division (\myref{thrm-polynomial-long-division}) on $f(x)$ with divisor $d(x)$ yields
    \[
        f(x) = q(x)d(x) + r(x)
    \]
    where $r(x) = 0$ or $\deg r(x) < \deg d(x)$. Thus we see
    \begin{align*}
        r(x) &= f(x) - q(x)d(x)\\
        &= f(x)-q(x)\left(a(x)f(x) + b(x)g(x)\right)\\
        &= f(x) - a(x)f(x)q(x) - b(x)g(x)q(x)\\
        &= (\underbrace{1 - q(x)a(x)}_{u(x)})(f(x)) - (\underbrace{q(x)b(x)}_{v(x)})(g(x))\\
        &\in S
    \end{align*}
    and therefore if $\deg r(x) < \deg d(x)$ this contradicts the minimality of $d(x)$. Therefore $r(x) = 0$ and hence $f(x) = q(x)d(x)$, meaning $d(x)$ divides $f(x)$. Similarly $d(x)$ divides $g(x)$. Hence $d(x)$ is a common divisor of $f(x)$ and $g(x)$.

    We now show that $d(x)$ is a GCD of $f(x)$ and $g(x)$. Suppose $c(x) \in F[x]$ divides both $f(x)$ and $g(x)$, i.e. $f(x) = p(x)c(x)$ and $g(x) = q(x)c(x)$ for some $p(x), q(x) \in F[x]$. We then see that
    \begin{align*}
        d(x) &= a(x)f(x) + b(x)g(x)\\
        &= a(x)(p(x)c(x)) + b(x)(q(x)c(x))\\
        &= c(x)(a(x)p(x) + b(x)q(x))
    \end{align*}
    and so $c(x)$ divides $d(x)$. Therefore $d(x)$ is a GCD of $f(x)$ and $g(x)$.

    We now show that the GCD of $f(x)$ and $g(x)$ is unique. Let $d_0(x)$ be another GCD of $f(x)$ and $g(x)$. This means that there exists $p(x), q(x) \in F[x]$ such that $f(x) = p(x)d_0(x)$ and $g(x) = q(x)d_0(x)$. We then see
    \begin{align*}
        d(x) &= a(x)f(x) + b(x)g(x)\\
        &= a(x)(p(x)d_0(x)) + b(x)(q(x)d_0(x))\\
        &= d_0(x)(a(x)p(x) + b(x)q(x))
    \end{align*}
    and so we conclude $\deg d(x) = \deg d_0(x) + \deg\left(a(x)p(x) + b(x)q(x)\right)$ by properties of the degree. Since $d(x)$ is of minimal degree, thus we must have $\deg d(x) = \deg d_0(x)$ which means $\deg\left(a(x)p(x) + b(x)q(x)\right) = 0$. Hence $a(x)p(x) + b(x)q(x) = \alpha \in F$ where $\alpha \neq 0$, meaning $d(x) = \alpha d_0(x)$. But both $d(x)$ and $d_0(x)$ are monic polynomials, so the coefficients of the largest terms of both $d(x)$ and $d_0(x)$ are both 1. Therefore $\alpha = 1$, meaning $d(x) = d_0(x)$. This proves the uniqueness of the GCD.
\end{proof}

To end this section, we look at the analogue of coprime numbers in terms of polynomials.

\begin{definition}
    Let $F$ be a field. Two polynomials $f(x)$ and $g(x)$ in $F[x]$ are \textbf{coprime}\index{coprime!polynomial} if and only if $\gcd(f(x), g(x)) = 1$.
\end{definition}

\section{Zeroes of an Irreducible Polynomial}
Recall that polynomial $f(x)$ has a zero $\alpha$ if and only if $x-\alpha$ is a factor of $f(x)$, by the Factor Theorem (\myref{corollary-factor-theorem}). Then $\alpha$ is a zero of multiplicity $n$ if $(x-\alpha)^n$ is a factor of $f(x)$ but $(x-\alpha)^{n+1}$ is not.

\begin{definition}
    Let $\alpha$ be a zero of multiplicity $n$ of a polynomial $f(x)$. If $n = 1$ then $\alpha$ is called a \textbf{simple zero}\index{polynomial!zeroes!simple}. If $n \geq 2$ then $\alpha$ is called a \textbf{multiple zero}\index{polynomial!zeroes!multiple}.
\end{definition}

With the derivative, we can derive a test for multiple zeroes of a polynomial.

\begin{theorem}\label{thrm-criterion-for-multiple-zeroes}
    Let $F$ be a field and $f(x) \in F[x]$. Then $f(x)$ has a multiple zero in some field extension $E/F$ if and only if $f(x)$ and $f'(x)$ have a common factor of positive degree in $F[x]$.
\end{theorem}
\begin{proof}
    For the forward direction, let $\alpha \in E$ be a multiple zero of $f(x)$. Then there is a $g(x) \in E[x]$ such that $f(x) = (x-\alpha)^2g(x)$, by Factor Theorem (\myref{corollary-factor-theorem}). One sees that
    \begin{align*}
        f'(x) &= 2(x-\alpha)g(x) + (x-\alpha)^2g'(x)\\
        &= (x-\alpha)\left(2g(x) + 2(x-\alpha)g'(x)\right)
    \end{align*}
    and so $x-\alpha$ is a common factor of both $f(x)$ and $f'(x)$ in $E[x]$. Now, seeking a contradiction, suppose that $f(x)$ and $f'(x)$ do not have a common factor of positive degree in $F[x]$. Hence $\gcd(f(x), f'(x)) = k$ where $k \in F$. By \myref{thrm-bezout-lemma-for-polynomials}, we can find $a(x), b(x) \in F[x]$ such that $a(x)f(x) + b(x)f'(x) = k$. Viewing $a(x)f(x) + b(x)f'(x)$ as an element of $E[x]$, since $x-\alpha$ is a common factor of $f(x)$ and $f'(x)$ in $E[x]$, we see that $x-a$ divides $k$, by definition of the GCD. This is clearly absurd. Hence, $f(x)$ and $f'(x)$ has a common divisor of positive degree in $F[x]$.

    For the reverse direction, suppose $f(x)$ and $f'(x)$ have a common factor of positive degree. Then by the Fundamental Theorem of Field Theory (\myref{thrm-fundamental-theorem-of-field-theory}), this common factor must have a zero in some extension $E$, say $\alpha$. Thus $\alpha$ is a zero of $f(x)$ and $f'(x)$. Since $\alpha$ is a zero of $f(x)$, therefore $f(x) = (x-\alpha)q(x)$ for some $q(x) \in E[x]$ by Factor Theorem. Then
    \[
        f'(x) = q(x) + (x-\alpha)q'(x).
    \]
    Note $f'(\alpha) = 0$ since $\alpha$ is a zero of $f'(x)$. Thus $q(\alpha) = 0$, which means $(x-\alpha)$ is a factor of $q(x)$ by Factor Theorem again. Therefore $f(x)$ and $f'(x)$ share a common factor of positive degree, in particular $x-\alpha$.
\end{proof}

With this result, we can find the zeroes of an irreducible polynomial.
\begin{theorem}\label{thrm-zeroes-of-an-irreducible}
    Let $F$ be a field and $f(x) \in F[x]$ be irreducible over $F$.
    \begin{itemize}
        \item If $\Char{F} = 0$, then $f(x)$ has no multiple zeroes.
        \item If $\Char{F} = p$, a prime number, then $f(x)$ has a multiple zero if $f(x) = g(x^p)$ for some $g(x) \in F[x]$.
    \end{itemize}
\end{theorem}
\begin{proof}
    If $f(x)$ has a multiple zero, then \myref{thrm-criterion-for-multiple-zeroes} tells us that $f(x)$ and $f'(x)$ has a common divisor of positive degree in $F[x]$. As $f(x)$ is irreducible over $F$, the only divisor of positive degree of $f(x)$ in $F[x]$ is $f(x)$ itself (up to associates), which means $f(x)$ divides $f'(x)$. But $f'(x)$ has smaller degree than $f(x)$, so $f'(x) = 0$. Writing $f(x) = a_0 + a_1x + a_2x^2 + \cdots + a_nx^n$, since $f'(x) = 0$ we see $ra_r = 0$ for all $r \in \{1, 2, \dots, n\}$ by \myref{exercise-derivative-is-zero-implies-higher-terms-are-zero} (later).

    Now if $\Char{F} = 0$, there does \textit{not} exist a positive integer $k$ such that $ka = 0$ for all $a \in F$. Thus, if $ra_r = 0$, we must have $a_r = 0$. Thus $f(x) = a_0$, a constant polynomial. But an irreducible polynomial has to be a non-constant polynomial by definition. This contradiction means that $f(x)$ has no multiple zeroes in $F$ if $\Char{F} = 0$.

    If $\Char{F} = p$ instead, $pa = 0$ for all $a \in F$. So if $p$ does not divide $r$, we require $a_r = 0$ in order for $ra_r = 0$. Otherwise, if $p$ divides $r$ (say, $r = pk$), note $(pk)a_{pk} = p(ka_{pk}) = 0$, so $a_{pk}$ is free. Thus the only powers of $p$ that appear in $f(x)$ are those of the form $x^{kp} = \left(x^p\right)^k$. It follows that $f(x) = g(x^p)$ for some $g(x) \in F[x]$.
\end{proof}

\begin{exercise}\label{exercise-derivative-is-zero-implies-higher-terms-are-zero}
    Let $F$ be a field and let $f(x) = a_0 + a_1x + \cdots + a_nx^n$. Prove that $f'(x) = 0$ implies $ra_r = 0$ for all $r \in \{1, 2, \dots, n\}$.
\end{exercise}

We found in \myref{thrm-zeroes-of-an-irreducible} that an irreducible polynomial either has no multiple zeroes (which is the case for a field of characteristic 0) or an irreducible polynomial $f(x) = g(x^p)$ (which is the case for a field of prime characteristic). This result can be generalized to a much larger class of fields.

\begin{definition}
    Let $F$ be a field and $n$ be a positive integer. Define
    \[
        F^n = \{a^n \vert a \in F\}.
    \]
\end{definition}

\begin{definition}
    A field $F$ is called \textbf{perfect}\index{field!perfect}\index{perfect field} if $F$ has characteristic 0 or if $F$ has prime characteristic $p$ and $F^p = F$.
\end{definition}

The most important family of perfect fields of prime characteristic are the finite fields.

\begin{theorem}\label{thrm-finite-field-is-perfect}
    Every finite field is perfect.
\end{theorem}
\begin{proof}
    Let $F$ be a finite field of prime characteristic $p$. Recall the Frobenius endomorphism $\phi: F \to F$ where $\phi(x) = x^p$ for all $x \in F$. We prove that the Frobenius endomorphism is an automorphism within finite fields.
    \begin{itemize}
        \item \textbf{Injective}: Since $\phi$ is non-trivial it is thus injective (\myref{thrm-homomorphism-from-field-is-injective-or-trivial}).
        \item \textbf{Surjective}: Since $\phi$ is an injective function from $F$ to itself, it must be surjective by \myref{problem-injection-from-finite-set-to-itself-is-bijection}.
    \end{itemize}
    Hence $\phi$ is a bijection. Coupled with the fact that $\phi$ is an endomorphism means that $\phi$ is an automorphism. But clearly the image of $\phi$ is $F^p$, so $F = F^p$.
\end{proof}

The fact that the Frobenius endomorphism is an automorphism for finite fields leads to an interesting result.

\begin{proposition}[Freshman's Dream]\label{prop-freshman-dream}\index{Freshman's Dream}
    Let $F$ be a field of prime characteristic $p$. Then $(x + y)^{p^n} = x^{p^n} + y^{p^n}$ for all $x,y \in F$ and positive integers $n$.
\end{proposition}
\begin{remark}
    The name ``Freshman's Dream'' refers to the \textit{incorrect} expansion of $(x+y)^n$ as $x^n + y^n$ over $\R$. However, over finite fields, this incorrect expansion becomes correct.
\end{remark}
\begin{proof}
    We induct on $n$.

    For the base case of $n = 1$, since the Frobenius endomorphism is an automorphism, we see that $\phi(x+y) = \phi(x) + \phi(y)$ and so $(x+y)^p = x^p + y^p$ by definition of $\phi$.

    The induction step done in \myref{exercise-freshman-dream} (later) proves the rest of the proposition.
\end{proof}

\begin{exercise}\label{exercise-freshman-dream}
    Let $F$ be a field of prime characteristic $p$. Prove that $(x + y)^{p^n} = x^{p^n} + y^{p^n}$ for all $x,y \in F$ and for all positive integers $n$.
\end{exercise}

With this definition of perfect fields, we return back to exploring the zeroes of irreducible polynomials.

\begin{theorem}\label{thrm-irreducible-polynomial-over-perfect-field-has-no-multiple-zeroes}
    If $f(x)$ is an irreducible polynomial over a perfect field $F$, then $f(x)$ has no multiple zeroes.
\end{theorem}
\begin{proof}
    The case when $\Char{F} = 0$ is done by \myref{thrm-zeroes-of-an-irreducible}. So assume $F$ is a perfect field of characteristic $p$ and let $f(x)$ be an irreducible polynomial in $F[x]$. Seeking a contradiction, assume $f(x)$ has multiple zeroes. By \myref{thrm-zeroes-of-an-irreducible} we know $f(x) = g(x^p)$ for some $g(x) \in F[x]$, say $g(x) = a_0 + a_1x + \cdots + a_nx^n$. Since $F = F^p$, each $a_i$ in $F$ can be written as $b_i^p$ for some $b_i \in F$. Thus,
    \begin{align*}
        f(x) &= g(x^p)\\
        &= b_0^p + b_1^px^p + b_2^px^{2p} + \cdots + b_n^px^{np}\\
        &= (b_0 + b_1x + b_2x^2 + \cdots + b_nx^n)^p & (\myref{prop-freshman-dream})\\
        &= (h(x))^p
    \end{align*}
    where $h(x) = b_0 + b_1x + b_2x^2 + \cdots + b_nx^n \in F[x]$. But this clearly means $f(x)$ is not irreducible. Hence $f(x)$ does not have multiple zeroes.
\end{proof}

What if the irreducible polynomial \textit{does} have multiple zeroes in a splitting field? In this case, there is something remarkable about the multiplicities of the zeroes.

\begin{theorem}\label{thrm-zeroes-of-irreducible-over-splitting-field-have-same-multiplicity}
    Let $f(x)$ be an irreducible polynomial over a field $F$, and let $E$ be a splitting field of $f(x)$ over $F$. Then the zeroes of $f(x)$ in $E$ all have the same multiplicity.
\end{theorem}
\begin{proof}
    Let $\alpha, \beta \in E$ be distinct zeroes of $f(x)$. Suppose $\alpha$ has multiplicity $m$; we may thus write $f(x) = (x-\alpha)^mg(x)$ for some $g(x) \in E[x]$. Using \myref{lemma-isomorphism-extension} we choose $\phi$ to be the identity mapping $F$ to $F$ and so an isomorphism $\psi: E \to E$ exists such that $\psi(r) = r$ for all $r \in F$ and $\psi(\alpha) = \beta$. Thus
    \[
        f(x) = \psi(f(x)) = (x-\beta)^m\psi(g(x))
    \]
    and so the multiplicity of $\beta$ is greater than or equal to the multiplicity of $\alpha$. Interchanging $\alpha$ and $\beta$ means that the multiplicity of $\alpha$ is greater than or equal to the multiplicity of $\beta$. Hence $\alpha$ and $\beta$ have the same multiplicity.
\end{proof}

\begin{corollary}\label{corollary-factorization-of-irreducible-polynomial-over-splitting-field}
    Let $f(x)$ be an irreducible polynomial over a field $F$, and let $E$ be a splitting field of $f(x)$ over $F$. Then $f(x)$ takes the form
    \[
        a(x-a_1)^n(x-a_2)^n\cdots(x-a_m)^n
    \]
    where $a \in F$ and $a_1, a_2, \dots, a_m \in E$ are distinct.
\end{corollary}
\begin{proof}
    Follows from \myref{thrm-zeroes-of-irreducible-over-splitting-field-have-same-multiplicity} and the factorization of any polynomial in a splitting field.
\end{proof}

\newpage

\section{Problems}
\begin{problem}\label{problem-simple-extension-absorbs-field-elements}
    Let $F$ be a field, and let $p, q \in F$ where $p \neq 0$. Suppose $E/F$ is a field extension and $a \in E$. Prove that $F(pa + q) = F(a)$.
\end{problem}

\begin{problem}
    Find the splitting fields of the following polynomial. If the splitting field found is a field extension, express it as a simple extension if possible.
    \begin{partquestions}{\alph*}
        \item $x^3 - 3x + 2$ over $\Q$.
        \item $x^3 - 1$ over $\Q$.
        \item $x^4 + x^2 + 1$ over $\Q$.\newline
        (\textit{Hint: $x^4+x^2+1 = (x^2-x+1)(x^2+x+1)$.})
        \item $x^2 - 2\sqrt{2}x + 3$ over $\Q(\sqrt2)$.
    \end{partquestions}
\end{problem}

\begin{problem}
    Determine which of the following polynomials, if any, have multiple zeroes over the field $F$ specified. If the polynomial has a multiple zero, state it.
    \begin{partquestions}{\alph*}
        \item $x^4 + 2x + 7$ over $F = \Z_2$.
        \item $x^{19} + x^8 + 1$ over $F = \Z_3$.
        \item $2x^6 + x^4 + 2x^3 + 2$ over $F = \Z_3$.
        \item $x^8 + 3x^5 + x^3 + 5$ over $F = \Z_7$.
    \end{partquestions}
\end{problem}

\begin{problem}
    Let $p(x) \in \Z_2$ be a polynomial of degree 7 that is irreducible in $\Z_2$. Prove or disprove the following statements.
    \begin{partquestions}{\roman*}
        \item $\Z_2[x]/\princ{p(x)}$ is a field.
        \item $\left(\Z_2[x]/\princ{p(x)}\right)^\ast$ is a group.
        \item Every non-identity element in $\left(\Z_2[x]/\princ{p(x)}\right)^\ast$ is a generator.
    \end{partquestions}
\end{problem}

\begin{problem}
    Let $F$ be a field, $a \in F$, and $f(x) \in F[x]$. Show that $f(x)$ and $f(x+a)$ have the same splitting field over $F$.
\end{problem}

\begin{problem}
    Find all subfields of $\Q(\sqrt2)$.
\end{problem}

\begin{problem}\label{problem-(x^p^n-x)-only-has-simple-zeroes}
    Let $F$ be a field of prime characteristic $p$. Prove that $f(x) = x^{p^n} - x \in F[x]$, where $n$ is a positive integer, only has simple zeroes.
\end{problem}

\begin{problem}
    Let $F$ be a field of prime characteristic $p$, and let $a \in F$. Prove that $f(x) = x^p - a \in F[x]$ is either irreducible or splits over $F$.\newline
    (\textit{Hint: consider two cases -- either $f(x)$ has a zero in $F$ or it does not.})
\end{problem}
