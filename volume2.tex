\documentclass[
    b5paper,
    pagesize,
    10pt,
    bibtotoc,
    normalheadings,
    twoside,
    openany,
    chapterprefix,
    DIV=9
]{scrbook}

%=============== Packages ================
% Fonts
\usepackage[T1]{fontenc}
\usepackage[utf8]{inputenc}
\usepackage{tgpagella}
\usepackage{mathpazo}

% Math formatting
\usepackage{mathtools}
\usepackage{amsfonts}
\usepackage{amsmath}
\usepackage{amssymb}
\usepackage{amsthm}
\usepackage{thmtools}

% Page formatting, colour and images
\usepackage[inner=1.5cm, outer=2.5cm, vmargin=2.5cm]{geometry}
\usepackage{graphicx}
\usepackage{tocloft}
\usepackage[x11names]{xcolor}
\usepackage{wrapfig}
\usepackage{parskip}
\usepackage{fancyhdr}
\usepackage{emptypage}
\usepackage{imakeidx}
\usepackage{mdframed}
\usepackage{eso-pic}

% Hyperlinks and references
\usepackage[breaklinks]{hyperref}
\usepackage[capitalise, nameinlink]{cleveref}
\usepackage{crossreftools}
\usepackage[
    backend=bibtex,
    style=alphabetic,
    sorting=ynt
]{biblatex}

% Miscellaneous
\usepackage{multicol}
\usepackage{enumitem}

%============== Resources ================
\addbibresource{abstract-algebra.bib}

%============= Formatting ================
\linespread{1.05}

%============ Redefinitions ==============
\let\oldemptyset\emptyset
\let\emptyset\varnothing

\let\totient\varphi

\renewcommand{\vert}{ \ \vline \ }
\newcommand{\vertalt}{ \ | \ }

\newcommand{\myref}[1]{\textbf{\crthypercref{#1}}}
\newcommand{\myreffigures}[1]{\textbf{\cref{#1}}}

\renewcommand{\qedsymbol}{\ensuremath{\blacksquare}}
\newcommand{\qedsymbolalt}{\ensuremath{\square}}

%=========== Theorem Things ==============
% 'Results' declarations
\newcommand{\makenewresultstyle}[2]{
    \declaretheoremstyle[
        headfont=\normalfont\bfseries,
        bodyfont=\normalfont,
        notefont=\normalfont\bfseries\itshape,
        spaceabove=0pt,  % Space between previous paragraph and current block
        spacebelow=0pt,  % Space between current block and next paragraph
        mdframed={
            skipabove=3pt,  % Space between top of block and beginning of coloured frame
            skipbelow=3pt,  % Space between bottom of block and beginning of coloured frame
            hidealllines=true,
            backgroundcolor=#2,
            usetwoside=false,  % Needed for `leftmargin` and `rightmargin` to work
            leftmargin=-5pt,
            rightmargin=-5pt,
            innerleftmargin=5pt,
            innerrightmargin=5pt
        }
    ]{#1-style}
}

\makenewresultstyle{theorem}{DarkSeaGreen2}
\declaretheorem[name=Theorem,style=theorem-style,within=section]{theorem}
\renewcommand*{\thetheorem}{\ZeroRoman{part}.\arabic{chapter}.\arabic{section}.\arabic{theorem}}

\makenewresultstyle{lemma}{Honeydew2}
\declaretheorem[style=lemma-style,sibling=theorem]{lemma}

\makenewresultstyle{proposition}{Honeydew1}
\declaretheorem[style=proposition-style,sibling=theorem]{proposition}

\makenewresultstyle{corollary}{DarkSeaGreen1}
\declaretheorem[style=corollary-style,sibling=theorem]{corollary}

\makenewresultstyle{definition}{LightCyan1}
\declaretheorem[style=definition-style,sibling=theorem]{definition}

\makenewresultstyle{axiom}{Thistle2}
\declaretheorem[style=axiom-style,sibling=theorem]{axiom}

% 'Questions' declarations
\declaretheoremstyle[
    headfont=\normalfont\bfseries,
    bodyfont=\normalfont,
    spaceabove=5pt,  % Space between previous paragraph and current block
    prefoothook={\vspace{5pt}},
    mdframed={
        skipabove=10pt,  % Space between top of block and beginning of frame
        skipbelow=10pt,  % Space between bottom of block and beginning of frame
        usetwoside=false,
        innerleftmargin=10pt,
        innerrightmargin=10pt
    }
]{exercise-style}
\declaretheorem[style=exercise-style,within=chapter]{exercise}
\renewcommand*{\theexercise}{\ZeroRoman{part}.\arabic{chapter}.\arabic{exercise}}

\declaretheoremstyle[
    headfont=\normalfont\bfseries,
    bodyfont=\normalfont,
    notefont=\normalfont\bfseries\itshape,
    spaceabove=0pt,  % Space between previous paragraph and current block
    spacebelow=0pt,  % Space between current block and next paragraph
]{problem-style}
\declaretheorem[name=Problem,style=problem-style,within=chapter]{problem}
\renewcommand*{\theproblem}{\ZeroRoman{part}.\arabic{chapter}.\arabic{problem}}

% Miscellaneous declarations
\declaretheoremstyle[
    headfont=\normalfont\bfseries,
    bodyfont=\normalfont
]{general-style}
\declaretheorem[style=general-style,sibling=theorem]{example}
\declaretheorem[style=general-style,numbered=no]{remark}

%============ Environments ===============
\newenvironment{questions}
{\begin{enumerate}[label=\textbf{\arabic*.}]}
{\end{enumerate}}

\newenvironment{partquestions}[1]
{\begin{enumerate}[label=\textbf{(#1)}]}
{\end{enumerate}}

%=========== Custom Commands =============
\newcommand{\ZeroRoman}[1]{\ifcase\value{#1}\relax0\else\Roman{#1}\fi}  % Roman numeral

\newcommand{\code}[1]{\texttt{#1}}  % Code block

\newcommand{\lcm}{\mathrm{lcm}}  % Lowest common multiple function
\newcommand{\sgn}{\mathrm{sgn}}  % Signum function

\newcommand{\im}{\mathrm{im}\;}  % Image of a function
\newcommand{\id}{\mathrm{id}}    % Identity function

%======== Custom Chapter Styling =========
\makeatletter
\renewcommand{\chaptermark}[1]{
    \markboth{\if@mainmatter\chapapp~\thechapter.\ \fi#1}{}
}

\renewcommand*{\chapterformat}{
  \MakeUppercase{\chapapp\nobreakspace\thechapter}
}

\renewcommand*{\chapterlineswithprefixformat}[3]{
    \Ifstr{#1}{chapter}{
        \vspace{-60px}
        \Ifstr{#2}{\empty}{\vspace{40px}}{\raggedleft#2}
        \vspace{-15px}
        \rule{\linewidth}{1pt}\par\nobreak
        \centering{#3}
        \vspace{-10px}
        \rule{\linewidth}{1pt}\par\nobreak
        \vspace{-10px}
    }{#2#3}
}
\makeatother

%======== Figure Caption Format ==========
\usepackage[labelfont=bf]{caption}
\DeclareCaptionLabelFormat{custom}{#1 \ZeroRoman{part}.#2.}
\captionsetup{labelformat=custom,labelsep=space}

%============ Custom Header ==============
\fancypagestyle{plain}{\fancyhf{}\renewcommand{\headrulewidth}{0pt}}  % To clear page numbers from footer, and header line at the start of every chapter

\pagestyle{fancy}
\fancyhf{}  % Clear header/footer

\fancyhead[LE,RO]{\thepage}
\fancyhead[LO,RE]{\textit{\nouppercase\leftmark}}

%========= Customise TOC Heading =========
\makeatletter
\def\createtoc{
    \renewcommand\tableofcontents{
        \chapter*{\contentsname}
        \@starttoc{toc}
    }
    \tableofcontents
}
\makeatother

%======= Customise Draft Watermark =======
\newcommand{\setasdraft}{
    \usepackage{draftwatermark}
    \SetWatermarkLightness{0.95}
    \SetWatermarkScale{1}
}

%============= Index Pages ===============
\usepackage[
    totoc,
    columnsep=20pt,
    hangindent=8pt,
    subindent=20pt,
    subsubindent=30pt
]{idxlayout}

\makeindex[options= -s index-style.ist]

%======= Bibliography Formatting =========
% These two lines are here to ensure that URLs do not exceed the page by too much
\setcounter{biburllcpenalty}{7000}
\setcounter{biburlucpenalty}{8000}

\usepackage{xr}

\setasdraft  % TODO: Remove once no longer a draft

%=========== Global Variables ============
\newcommand{\version}{0.1}
\newcommand{\volumenumber}{2}
\newcommand{\volumename}{Rings}
\newcommand{\volumeimage}{integers-modulo-n.png}

%============== Resources ================
\externaldocument{part0/number-theory}
\externaldocument{part0/algebra}

\externaldocument{part1/basics-of-groups}
\externaldocument{part1/subgroups}
\externaldocument{part1/homomorphisms-and-isomorphisms}
\externaldocument{part1/more-groups}

%=========== Path to images ==============
\graphicspath{{part2/images}}

%=========== Custom Commands =============
\newcommand{\ideal}[1]{\mathfrak{#1}}                % Ideal of a ring
\newcommand{\princ}[1]{\left\langle#1\right\rangle}  % Principal ideal generated by the element

\newcommand{\Z}{\mathbb{Z}}                        % The ring of integers
\newcommand{\Q}{\mathbb{Q}}                        % The ring of rational numbers
\newcommand{\R}{\mathbb{R}}                        % The ring of real numbers
\newcommand{\C}{\mathbb{C}}                        % The ring of complex numbers

\newcommand{\Mn}[2]{\mathcal{M}_{#1\times#1}(#2)}  % The ring of n by n matrices with entries in the ring R
\newcommand{\ZeroM}[1]{\textbf{0}_{#1}}      % Zero matrix
\newcommand{\IdentityM}[1]{\textbf{I}_{#1}}  % Identity matrix

% \renewcommand{\H}{\mathbb{H}}   % Quaternion Ring
% \newcommand{\qi}{\textbf{i}}    % Quaternion i
% \newcommand{\qj}{\textbf{j}}    % Quaternion j
% \newcommand{\qk}{\textbf{k}}    % Quaternion k

\newcommand{\Ann}[2]{\mathrm{Ann}_{#1}(#2)}  % Annihilator of a subset A of a ring R
\newcommand{\Char}[1]{\mathrm{char}(#1)}     % Characteristic of a ring R
\newcommand{\Nilr}[1]{\mathfrak{N}_{#1}}     % Nilradical of a ring R

%========= Front Matter Pages ============
% Quote page
\newcommand{\quotepagetext}{
    [Some] of the major discoveries in ring theory have helped shape the course of development of modern abstract algebra... A course in ring theory is an indispensable part of the education of any fledgling algebraist.
}
\newcommand{\quotepageattribution}{Tsit-Yuen Lam, 2001}
\newcommand{\quotepagecitation}{\cite{lam_2001}}

% Preface
\newcommand{\prefacevolumetext}{
    This volume covers the basics of ring theory. %TODO: Add
}
\newcommand{\prefacevolumedate}{}  %TODO: Add

% Suggestions of use
\newcommand{\interdependencenotes}{
    % TODO: Update
}

%==== Include only relevant chapters =====
\IfFileExists{\jobname.run.xml}
{
    \includeonly{
        front-matter,
        % Main chapters
        part2/intro-to-rings,
        part2/basics-of-rings,
        part2/integral-domains,
        part2/ideals-and-quotient-rings,
        part2/ring-homomorphisms,
        part2/polynomial-rings,
        % Appendices
        part2/exercise-solutions,
        part2/problem-solutions,
        part2/appendices
    }
}
{
    % Do a full document generation initially to generate all the aux files
}

%=========================================
\begin{document}
\frontmatter  % Use lowercase roman numerals for page numbers
% Define roman numerals for front matter
\makeatletter\newcommand*{\rom}[1]{\Ifstr{#1}{0}{0}{\expandafter\@slowromancap\romannumeral #1@}}\makeatother

% Half title page
\thispagestyle{empty}
\null\vspace{4cm}
\begin{raggedleft}
    {\fontsize{24pt}{0pt}\selectfont \textbf{A Complete Introduction To Abstract Algebra}}\\
\end{raggedleft}

% Title page
\begin{titlepage}
    \null\vspace{4cm}
    \begin{raggedleft}
        {\fontsize{20pt}{0pt}\selectfont \textbf{A Complete}\\\textbf{Introduction To}}\\

        \vspace{0.4cm}
        {\fontsize{48pt}{0pt}\selectfont \textbf{ABSTRACT}}\\
        \vspace{0.15cm}
        {\fontsize{48pt}{0pt}\selectfont \textbf{ALGEBRA}}\\

        \vspace{0.5cm}
        {\fontsize{16pt}{0pt}\selectfont Version \version}\\
        
        \vspace{1.25cm}
        {\fontsize{20pt}{0pt}\selectfont Kan Onn Kit}\\
    \end{raggedleft}
    \vspace*{\fill}
\end{titlepage}

\newpage{}

% Edition notice
\clearpage\null\vfill
\thispagestyle{empty}
\begin{minipage}[b]{0.9\textwidth}
    \footnotesize\raggedright
    \setlength{\parskip}{0.5\baselineskip}

    Published by Kan Onn Kit\\
    Singapore
    \vspace{5cm}

    \textbf{A Complete Introduction To Abstract Algebra}\par
    Version \version
    \vspace{0.3cm}

    Copyright \copyright \ 2022 -- \the\year\ by Kan Onn Kit\par
    This work is licensed under a
    Creative Commons Attribution-NonCommercial-ShareAlike 4.0 International Licence.\par
    \pdfteximg{2.5cm}{images/CC_BY-NC-SA_4.0.pdf_tex}\\
    The full licence text is available at \url{http://creativecommons.org/licenses/by-nc-sa/4.0/}.\par    
    The source files for the project are available at \url{https://github.com/PhotonicGluon/Abstract-Algebra-Book}.
    \vspace{0.3cm}

    Typeset in 10pt \TeX~Gyre Pagella using PDF\LaTeX.
\end{minipage}

\vspace*{2\baselineskip}
\cleardoublepage

% "Quote" page
\thispagestyle{empty}
\vspace*{2cm}

\begin{center}
    \Large{\parbox{10cm}{
        \begin{raggedright}
            \Large
            Symmetry is a vast subject, significant in art and nature. Mathematics lies at its root, and it would be hard to find a better one on which to demonstrate the working of the mathematical intellect.
            \vspace{0.3cm}
            
            \hfill
            --- Hermann Weyl, 1952\\
            \vspace{-0.25cm}
            
            \hfill
            \normalsize
            ({\cite[p.~145]{weyl_1952}})
        \end{raggedright}
    }
}
\end{center}

\newpage

% Table of contents
\createtoc

% Acknowledgements
\chapter{Acknowledgements}
Undertaking such a monumental project is new to me, and I am indebted to the people who accompanied me on this journey.

I am eternally grateful to my parents, who have spent countless hours and an ungodly amount of effort to raise me into who I am today. Their omnipresent kindness, patience, and love for me are something I certainly do not deserve, and I thank them for taking care of me.

I would like to thank my tutor Leong Chong Ming, who got me interested in abstract algebra in the first place. His enthusiasm and eagerness to share his knowledge on the subject is the driving force behind my decision to write these books.

I am grateful for the help of my friend Low Ji Yuan, who has assisted me with countless revisions of the content in these books and given me another pair of eyes in the vetting of content.

I also sincerely appreciate the continued support from my mathematics tutors, Loke Weng Heng, Siow Yun Jie, and Teng Yen Ping, who have been there through my junior college years inspiring me with the wonders of mathematics. I am indebted to them for allowing me to excel in my final examinations.

My close friends, Aidan Tay, Gabriel Fong, and Low Ji Yuan, have accompanied me through two years of schooling (and math jokes). I offer infinite thanks to them for sticking with me and for encouraging this math nerd to pursue his wacky projects.

A thousand thanks go out to my teachers at the School of Science and Technology, Singapore, and specifically my form teacher Lee Tsi Yew Samuel, who instilled important character values into me so I can excel in my future endeavours.

% Preface
\chapter{Preface}
Although algebra has a long history, it has undergone quite striking changes in the past few decades. Abstract (or modern) algebra is widely recognised as an essential element of higher mathematical education. However, the results that it showcases are often hard to grasp and understand without prerequisite knowledge or a heavy background in mathematics. Most books on this subject are crafted for undergraduates at universities. They are not for a general mathematics enthusiast or one who seeks to understand more about the inner structure of algebra that mathematicians encounter frequently.

The exploration of such structures is fundamental to the current underpinning of scientific inquiries. For example, groups are important as they describe the symmetries which the laws of physics seem to obey. Finite fields are also used in coding theory and combinatorics. I hope this series of books will inspire more people to learn more about abstract algebra, beyond the simple introduction presented here.

In addition, I find that, in most textbooks, important details are left for the reader to figure out independently, without providing any additional guidance or help along the way. Exercises and problems are provided for topics taught in abstract algebra, but only a handful have written solutions provided, leaving readers unsure of the correctness of their answers. I believe that the completeness of a textbook is essential; no claim made should be without justification (unless absolutely necessary). These books offer a more complete picture of abstract algebra by providing full-worked solutions to all exercises and problems posed.

This series of books serves to achieve several goals.
\begin{itemize}
    \item Provide a step-by-step explanation of core results from abstract algebra, without ambiguity of the results discussed.
    \item Demystify the core steps that many textbooks gloss over when proving results or when writing the solutions to problems/exercises.
    \item Ensure that results from abstract algebra are as accessible, as approachable, and as understandable for as many people as possible.
\end{itemize}
I hope that these books can accomplish these goals and let readers enjoy the wonders of abstract algebra.

\hfill{\textit{24 September, 2023}}

% Suggestions on the use of this book
\chapter{Suggestions on the Use of This Book}
\section*{General Information}
\begin{itemize}
    \item We include both exercises and problems.
    \begin{itemize}
        \item An exercise can be thought of as a simple ``self-review'' question. Exercises ensure that the content of a particular section is understood and should not be too hard to answer.
        \item A problem is a more holistic version of an exercise. Generally, solutions to problems require a thorough understanding of the current chapter and may require results from other chapters.
    \end{itemize}
    \item A consistent labelling system for all the results within and between parts is necessary for a project as long as this one.
    \begin{itemize}
        \item All definitions, axioms, examples, lemmas, theorems, propositions, and corollaries are consecutively numbered, using the format
        \begin{quote}
            \code{[CHAPTER].[SECTION].[NUMBER]}
        \end{quote}
        For example, the fourth statement in chapter 2, section 3 is labelled \textbf{2.3.4}.
        \item Exercises and problems are also numbered consecutively, using the format
        \begin{quote}
            \code{[CHAPTER].[NUMBER]}
        \end{quote}
        For example, the third exercise in chapter 2 is labelled \textbf{2.3}. Likewise, the fourth exercise in chapter 3 is labelled \textbf{3.4}.
    \end{itemize}
    \item The symbol ``$\qedsymbol$'' marks the end of a proof.
\end{itemize}

\section*{Chapter Interdependence}
The diagram on the next page shows chapter interdependence. It should be used in conjunction with the table of contents and notes listed.

\newpage
\begin{center}
    \pdfteximg{\linewidth}{images/interdependence.pdf_tex}
\end{center}

\newpage

\textbf{Notes}:
\begin{itemize}
    \item Each part is largely independent from the other parts. However, part 0 is required reading for future parts of the book.
    \item Part 1 on group theory can be thought of as the fundamentals of abstract algebra.
    \begin{itemize}
        \item Chapter 8 is essentially independent from the rest of the other chapters. It provides motivation for the axioms of groups, but readers who want to skip this introduction can move straight to chapter 9.
        \item Chapters 9, 10, and 11 are considered to be the essentials of group theory.
        \item Chapter 11 is required reading for chapters 12, 13, 14, and 16.
        \item Chapter 14 only requires knowledge of the subgroup product from chapter 13 (specifically \myref{definition-subgroup-product} and \myref{prop-subgroup-product-is-subgroup}); otherwise these two chapters are relatively independent.
        \item Chapter 15 assumes knowledge of chapter 14, and knowledge of permutations and the symmetric group from chapter 12.
        \item Usually, group actions (chapter 16) would be read after the essentials of group theory; therefore chapter 16 could be read after chapter 11.
        \item Chapter 17 only requires chapters 14 and 16.
        \item Chapter 18 only require results from chapter 14, except for \myref{problem-S4-composition-series} which uses the alternating group introduced in chapter 15.
        \item Chapter 19 assumes full knowledge of chapter 14; minor results from chapter 15 (specifically, the alternating group), chapter 17 (the Third Sylow Theorem, \myref{thrm-sylow-3}), and chapter 18 (\myref{problem-S4-composition-series}) are required.
    \end{itemize}
    \item Part 2 on ring theory builds on ideas of group theory.
    \begin{itemize}
        \item Chapters 9 and 10 from group theory are required reading before continuing with ring theory.
        \item TODO: Add ring theory interdependence.
    \end{itemize}
\end{itemize}


\mainmatter  % Now use arabic numerals for page numbers
\chapter{Introduction to Rings}
In part I, we looked exclusively at groups and their operations. We discussed how groups are a generalisation of symmetry and looked at results related to groups. In this part, we look at rings.

\section{Extending Number Systems}
The entire field fo ring theory was kickstarted to generalize number systems.

We are intimately familiar with number systems in our daily lives. The simplest number system that was developed involved only the positive integers (which we denote by $\mathbb{N}$). What properties do we have in the positive integers?

Let's start with addition. Adding two positive numbers together still results in a positive integer. For example, we see $2 + 3 = 5$, $3 + 4 = 7$, $58 + 95 = 153$, and so on. It is impossible to add two positive integers and end up with a non-positive integer. This means that the set of positive integers is closed under addition. Furthermore, we want addition to be associative. Take for example the expression $1 + 2 + 3$. We want $1 + (2 + 3) = (1 + 2) + 3 = 6$, as we really don't care about the order of addition within brackets. Finally addition is commutative. As an example, the sum $4 + 6$ is equal to $6 + 4$, and we don't care what order we perform the addition in.

Let's now turn our attention to multiplication. One sees that multiplying two positive integers together still results in a positive integer, and that we don't care about the order of evaluating multiplication of three positive integers.

Finally, we see that multiplication distributes over addition. To see what this means, consider the expression $5(6+7)$. How could we evaluate this? Well, we could sum $6+7 = 13$ first, then multiply it by 5 to get $5\times13 = 65$. Alternatively, we could notice that $5(6+7) = 5 \times 6 + 5 \times 7$ and evaluate it that way. This is called left distribution; right distribution is defined similarly.

In summary, we may call the set of positive integers, along with addition and multiplication, a \textbf{semiring} (see \cite{mathworld_semiring-definition}).

\newpage

\begin{definition}
    A set $S$ together with two operations $+$ and $\cdot$ is called a \textbf{semiring}\index{semiring} if it satisfies the following properties.
    \begin{itemize}
        \item \textbf{Additive Closure}\index{axiom!semiring!additive closure}: For any two elements $a$ and $b$ in $S$, $a + b$ is also in $S$.
        \item \textbf{Additive Associativity}\index{axiom!semiring!additive associativity}: For any three elements $a$, $b$, and $c$ in $S$, $a+(b+c) = (a+b)+c$.
        \item \textbf{Additive Commutativity}\index{axiom!semiring!additive commutativity}: For any two elements $a$ and $b$ in $S$, $a + b = b + a$.
        \item \textbf{Multiplicative Closure}\index{axiom!semiring!multiplicative closure}: For any two elements $a$ and $b$ in $S$, $a \cdot b$ is also in $S$.
        \item \textbf{Multiplicative Associativity}\index{axiom!semiring!multiplicative associativity}: For any three elements $a$, $b$, and $c$ in $S$, $a\cdot(b\cdot c) = (a\cdot b)\cdot c$.
        \item \textbf{Left and Right Distributivity}\index{axiom!semiring!distributivity}: For any three elements $a$, $b$, and $c$ in $S$, $a\cdot(b + c) = (a \cdot b) + (a \cdot c)$ and $(a + b) \cdot c = (a \cdot c) + (b \cdot c)$.
    \end{itemize}
\end{definition}

However, one may notice that $\mathbb{N}$ isn't particularly fun to work in. We only have these trivial properties to work with, and these do not give us enough to generate results about the positive integers. So ancient civilizations decided to include the notion of a ``additive identity'', which is called zero (0). Now, adding zero to any positive integer results in the positive integer itself.

We also notice that \textit{multiplying} any number by 0 results in 0: this is obvious, especially for us who work with integers. This set of non-negative integers (which, for brevity, we denote by $N$ in this chapter only) with addition and multiplication is called a \textbf{rig} (see \cite{proofwiki_rig-definition}).
\begin{definition}
    A set $S$ together with two operations $+$ and $\cdot$ is called a \textbf{rig}\index{rig} if it is a semiring and satisfies these additional properties.
    \begin{itemize}
        \item \textbf{Additive Identity}\index{axiom!rig!additive identity}: There is an element $0_S$ in $S$ such that for any $x$ in $S$ we have $0_S + x = x + 0_S = x$.
        \item \textbf{Multiplication by Zero}\index{axiom!rig!multiplication by zero}: For any element $x$ in $S$ we have $0_S \cdot x = x \cdot 0_S = 0_S$.
    \end{itemize}
\end{definition}
\begin{remark}
    A rig is a ri\textit{\textbf{n}}g without \textit{\textbf{n}}egative elements.
\end{remark}

Now, for any element $n$ in $N$, it would be nice to have a `corresponding' element in that sums to the additive identity 0. Clearly $N$ doesn't have such an element, but the integers ($\Z$) does. For example, the integer 3 has a `corresponding' element -3 that results in a sum of 0, i.e. $3 + (-3) = 0$. In general, for any integer $n$, summing $n$ with $-n$ results in 0. This `corresponding' element is called the additive inverse of $n$. With additive inverses, we finally have a construction for a \textbf{ring}. Note that we follow \cite[p.~223]{dummit_foote_2004}, \cite[p.~115, Definition 1.1]{hungerford_1980}, and \cite{proofwiki_ring-definition} for the definition of a ring.
\begin{definition}
    A set $S$ together with two operations $+$ and $\cdot$ is called a \textbf{ring}\index{ring} if it is a rig and satisfies the \textbf{Additive Inverse} property, where for every element $x$ in $S$, there exists an element $-x$ in $S$ such that $x + (-x) = (-x) + x = 0_S$.
\end{definition}

\section{Rings as Algebraic Structures}
With an intuition of what rings are, we more concretely define what a ring is using algebraic structures.

In the previous part, we explored the concept of groups. We weaken the conditions required for that structure to form a \textbf{semigroup}.
\begin{definition}
    A \textbf{semigroup}\index{semigroup} is a set $S$ together with an operation $\ast$ satisfying the \textbf{semigroup axioms}\index{axiom!semigroup}.
    \begin{itemize}
        \item \textbf{Closure}\index{axiom!semigroup!closure}: For any two elements $a$ and $b$ in $S$, the element $a\ast b$ is also in $S$.
        \item \textbf{Associativity}\index{axiom!semigroup!associativity}: For any $a$, $b$, and $c$ in $S$, we have $a \ast (b \ast c) = (a \ast b) \ast c$.
    \end{itemize}
\end{definition}
\begin{example}
    Consider the set $S = \{1, 2, 3, 4\}$ with the operation $\ast$ such that $(S, \ast)$ has the Cayley table as shown below.
    \begin{table}[h]
        \centering
        \begin{tabular}{|l|l|l|l|l|}
            \hline
            $\ast$     & \textbf{1} & \textbf{2} & \textbf{3} & \textbf{4} \\ \hline
            \textbf{1} & 1          & 1          & 1          & 1          \\ \hline
            \textbf{2} & 2          & 2          & 2          & 2          \\ \hline
            \textbf{3} & 3          & 3          & 3          & 3          \\ \hline
            \textbf{4} & 4          & 4          & 4          & 4          \\ \hline
        \end{tabular}
    \end{table}
    
    One sees that $(S, \ast)$ is closed under $\ast$. In addition, $\ast$ is associative. Hence $(S, \ast)$ is a semigroup.
\end{example}

With all that set up, we are ready to rigorously define what a ring is.
\begin{definition}
    A \textbf{ring}\index{ring} is a set $R$ with two binary operations $+$ and $\cdot$ satisfying the following axioms.
    \begin{itemize}
        \item \textbf{Addition-Abelian}\index{axiom!ring!addition-abelian}: $(R, +)$ is an abelian group.
        \item \textbf{Multiplication-Semigroup}\index{axiom!ring!multiplication-semigroup}: $(R, \cdot)$ is a semigroup.
        \item \textbf{Distributive}\index{axiom!ring!distributive}: $\cdot$ is distributive over $+$. That is,
        \begin{itemize}
            \item $a \cdot (b + c) = (a \cdot b) + (b \cdot c)$; and
            \item $(a + b) \cdot c = (a \cdot c) + (b \cdot c)$.
        \end{itemize}
    \end{itemize}
    We denote such a ring by $(R, +, \cdot)$.
\end{definition}
\begin{remark}
    We do not need to define \textbf{Multiplication by Zero} as an axiom here because it is implied by the three ring axioms above. We prove this in the next chapter.
\end{remark}

We note two important types of rings here.
\begin{definition}
    A \textbf{ring with identity}\index{ring!with identity} is a ring $(R, +, \cdot)$ with an element $1_R$ such that for any $x \in R$ we have $1_R \cdot x = x \cdot 1_R = x$.
\end{definition}
\begin{remark}
    Other authors (e.g. \cite[p.~136]{cohn_1982}, \cite[pp.~145--146]{clark_1984}) define a ring as a ring with identity.
\end{remark}
\begin{example}
    We introduced the integers ($\mathbb{Z}$) in the previous section. One sees clearly that 1 is the multiplicative identity in the integers since $1n = n$ for any integer $n$, os $\mathbb{Z}$ is a ring with identity.
\end{example}

\begin{definition}
    A ring where $a \cdot b = b \cdot a$ for all $a$ and $b$ in $R$ is called a \textbf{commutative ring}\index{ring!commutative}.
\end{definition}
\begin{example}
    Considering the integers again, we see that $mn = nm$ for any two integers $m$ and $n$. Thus $\mathbb{Z}$ is a commutative ring.
\end{example}

We end this chapter by introducing the \textbf{trivial ring}.
\begin{definition}
    The \textbf{trivial ring}\index{trivial ring} (or \textbf{zero ring}\index{zero ring}), denoted $\textbf{0}$, is the ring $(\{0\}, +, \cdot)$ where
    \[
        0 + 0 = 0 \text{ and } 0 \cdot 0 = 0.    
    \]
\end{definition}
\begin{exercise}
    Prove that the trivial ring is a commutative ring with identity.
\end{exercise}

\chapter{Basics of Rings}
With an intuition and definition of rings out of the way, we are now ready to tackle the basics in this chapter.

\section{Obvious Rings}
Before we introduce some examples of rings, we make some remarks for the notation that is used in Ring Theory.
\begin{itemize}
    \item The multiplication symbol $\cdot$ is usually omitted, so $x \cdot y$ is written as $xy$.
    \item The additive identity of $R$ will always be denoted by 0 and the multiplicative identity of $R$ (if it exists) will always be denoted by 1.
    \item The additive inverse of the element $x$ will be denoted by $-x$ and the multiplicative inverse of $x$ (if it exists) will be denoted by $x^{-1}$.
    \item $n$ applications of $+$ on an element $x$ will be denoted $nx$ (and will be denoted $-nx$ if the element is $-x$), while $n$ applications of $\cdot$ on an element $x$ will be denoted $x^n$ (and will be denoted $x^{-n}$ if the element is $x^{-1}$ and if it exists).
\end{itemize}

Let's look at some examples of rings.
\begin{definition}
    The \textbf{ring of integers}\index{ring!of integers} is the set $\Z$ together with integer addition and multiplication.
\end{definition}
\begin{remark}
    We denote the ring of integers by $\Z$.
\end{remark}
\begin{proposition}
    $\Z$ is a commutative ring with identity.
\end{proposition}
\begin{proof}
    \myref{exercise-ring-of-integers-is-a-ring} (later) shows that $\Z$ is a ring. In addition, multiplication is commutative (\myref{axiom-multiplication-is-commutative}), and 1 is the multiplicative identity. Thus $\Z$ is a commutative ring with identity.
\end{proof}

\begin{definition}
    Let the integer $n > 2$. The \textbf{ring of integers modulo $n$}\index{ring!of integers!modulo $n$} is $(\Z_n, \oplus_n, \otimes_n)$, where $\oplus_n$ and $\otimes_n$ denote addition and multiplication modulo $n$ respectively.
\end{definition}
\begin{remark}
    We denote the ring of integers modulo $n$ by $\Z_n$.
\end{remark}
\begin{proposition}
    $\Z_n$ is a commutative ring with identity.
\end{proposition}
\begin{proof}
    We first prove the ring axioms before showing that it is commutative with a multiplicative identity.
    \begin{itemize}
        \item \textbf{Addition-Abelian}: We know $(\Z_n, \oplus_n)$ is an abelian group by \myref{prop-Zn-is-abelian-group}
        \item \textbf{Multiplication-Semigroup}: We can see that $(\Z_n, \otimes_n)$ is a semigroup as
        \begin{itemize}
            \item $\Z_n$ is closed under $\otimes_n$ because $a \otimes_n b \in \{0, 1, 2, \dots, n-1\} = \Z_n$; and
            \item multiplication is associative (\myref{axiom-multiplication-is-associative}), so multiplication modulo $n$ is associative.
        \end{itemize}
        \item \textbf{Distributive}: Since multiplication distributes over addition (\myref{axiom-distributivity}), thus multiplication modulo $n$ (i.e. $\otimes_n$) distributes over addition modulo $n$ (i.e. $\oplus_n$).
    \end{itemize}
    Hence $(\Z_n, \oplus_n, \otimes_n)$ is a ring.
    
    Furthermore, multiplication is commutative (\myref{axiom-multiplication-is-commutative}), so $\otimes_n$ is commutative. Also $\otimes_n$ has an identity of 1. Therefore $(\Z_n, \oplus_n, \otimes_n)$ is a commutative ring with identity.
\end{proof}

\begin{definition}
    The \textbf{ring of rational numbers}\index{ring!of rational numbers} is $(\Q, +, \times)$, where $+$ and $\times$ denote normal addition and multiplication.
\end{definition}
\begin{remark}
    We denote the ring of rational numbers by $(\Q, +, \times)$.
\end{remark}
\begin{proposition}
    $\Q$ is a commutative ring with identity.
\end{proposition}
\begin{proof}
    We first show that $\Q$ satisfies the ring axioms.
    \begin{itemize}
        \item \textbf{Addition-Abelian}: We know that $(\Q, +)$ is an abelian group from \myref{problem-Q-is-abelian-group-under-addition}.
        \item \textbf{Multiplication-Semigroup}: We note that $(\Q, \times)$ is a semigroup as
        \begin{itemize}
            \item $\Q$ is closed under $\times$ because multiplying two rational numbers together produce a rational number; and
            \item multiplication is associative (\myref{axiom-multiplication-is-associative}).
        \end{itemize}
        \item \textbf{Distributive}: Multiplication distributes over addition by \myref{axiom-distributivity}.
    \end{itemize}
    Hence $\Q$ is a ring. Furthermore, $\times$ has an identity of 1 and is commutative (\myref{axiom-multiplication-is-commutative}). So $\Q$ is a commutative ring with identity.
\end{proof}

\begin{definition}
    The \textbf{ring of real numbers}\index{ring!of real numbers} is the ring $(\R, +, \times)$ where $+$ and $\times$ denotes regular addition and multiplication respectively.
\end{definition}
\begin{remark}
    We denote the ring of real numbers by $(\R, +, \times)$.
\end{remark}
\begin{proposition}
    $\R$ is a commutative ring with identity.
\end{proposition}
\begin{proof}
    Replace $(\Q, +)$ with $(\R, +)$ and $(\Q, \times)$ with $(\R, \times)$ in the previous proof.
\end{proof}

We end this section by looking at the ring of complex numbers.
\begin{definition}
    Let the set of \textbf{complex numbers}\index{complex numbers}
    \[
        \C = \{a + bi \vert a, b \in \R\}
    \]
    where $i = \sqrt{-1}$ is known as the \textbf{imaginary unit}\index{imaginary unit}, where $i^2 = -1$. Define complex addition and multiplication by
    \begin{align*}
        (a+bi) + (c+di) &= (a+c) + (b+d)i,\\
        (a+bi) \cdot (c+di) &= (ac-bd) + (ad+bc)i.
    \end{align*}
    Then $\C$ under complex addition and multiplication is the \textbf{ring of complex numbers}\index{ring!of complex numbers}.
\end{definition}
\begin{remark}
    We denote the ring of complex numbers by $\C$.
\end{remark}
\begin{proposition}
    $\C$ is a commutative ring with identity.
\end{proposition}
\begin{proof}
    We first show that $\C$ satisfies the ring axioms.
    \begin{itemize}
        \item \textbf{Addition-Abelian}: We show that $(\C, +)$ satisfies the group axioms, and then show that $(\C, +)$ is commutative.
        \begin{itemize}
            \item \textbf{Closure}: Clearly for all real numbers $a$, $b$, $c$, and $d$ we have $a + c \in \R$ and $b+d \in \R$. Thus $(a+bi) + (c+di) = (a+c) + (b+d)i \in \C$, meaning $\C$ is closed under complex addition.
            
            \item \textbf{Associativity}: Let $a+bi, c+di, e+fi \in \C$. Then note that
            \begin{align*}
                &(a+bi) + ((c+di) + (e+fi))\\
                &= (a+bi) + ((c+e) + (d+f)i)\\
                &= (a+(c+e)) + (b+(d+f))i\\
                &= ((a+c)+e) + ((b+d)+f)i & (+ \text{ is associative, }\myref{axiom-addition-is-associative})\\
                &= ((a+c) + (b+d)i) + (e+fi)\\
                &= ((a+bi) + (c+di)) + (e+fi)
            \end{align*}
            so complex addition is associative.
            
            \item \textbf{Identity}: The identity in $\C$ is $0 + 0i = 0$ since
            \[
                (0+0i) + (a+bi) = (0+a) + (0+b)i = a+bi
            \]
            and complex addition is commutative (to be proved later), so $(\C,+)$ has an additive identity.
            
            \item \textbf{Inverse}: Let $a+bi \in \C$. Clearly $-a, -b \in \R$ and that
            \[
                (a+bi) + (-a+(-b)i)= (a+(-a)) + (b+(-b))i = 0
            \]
            and complex addition is commutative (to be proved later), so any $a+bi\in C$ has an additive inverse of $-a-bi \in \C$.

            \item \textbf{Commutative}: Let $a+bi, c+di \in \C$. Then
            \begin{align*}
                (a+bi) + (c+di) &= (a+c) + (b+d)i\\
                &= (c+a) + (d+b)i & (+\text{ is commutative, } \myref{axiom-addition-is-commutative})\\
                &= (c+di) + (a+bi)
            \end{align*}
            so complex addition is commutative.
        \end{itemize}
        
        \item \textbf{Multiplication-Semigroup}: We show that $(\C, \times)$ is a semigroup.
        \begin{itemize}
            \item \textbf{Closure}: Clearly for all real numbers $a$, $b$, $c$, and $d$ we have $ac, bd, ad, bc \in \R$, so $ac - bd, ad + bc \in \R$. Therefore
            \[
                (a+bi)(c+di) = (ac-bd) + (ad+bc)i \in \C
            \]
            which means $\C$ is closed under multiplication.

            \item \textbf{Associativity}: Let $a+bi, c+di, e+fi \in \C$. Note that
            \begin{align*}
                &(a+bi)((c+di)(e+fi))\\
                &= (a+bi)((ce-df)+(cf+de)i)\\
                &= (a(ce-df) - b(cf+de)) + (a(cf+de) + b(ce-df))i\\
                &= (ace - adf - bcf - bde) + (acf + ade + bce - bdf)i\\
                &= (ace - bde - adf - bcf) + (acf - bdf + ade + bce)i\\
                &= ((ac-bd)e - (ad+bc)f) + ((ac-bd)f + (ad+bc)e)i\\
                &= ((ac-bd)+(ad+bc)i)(e+fi)\\
                &= ((a+bi)(c+di))(e+fi)
            \end{align*}
            so complex multiplication is associative.
        \end{itemize}
        
        \item \textbf{Distributive}: We only prove left distributivity because we will show that complex multiplication is commutative later. Let $a+bi, c+di, e+fi \in \C$. Note that
        \begin{align*}
            &(a+bi)((c+di) + (e+fi))\\
            &= (a+bi)((c+e) + (d+f)i)\\
            &= (a(c+e)-b(d+f)) + (a(d+f) + b(c+e))i\\
            &= (ac+ae-bd-bf) + (ad+af+bc+be)i\\
            &= (ac-bd+ae-bf) + (ad+bc+af+be)i & (+ \text{ is associative})\\
            &= ((ac-bd) + (ad+bc)i) + ((ae - bf) + (af + be)i)\\
            &= (a+bi)(c+di) + (a+bi)(e+fi)
        \end{align*}
        so complex multiplication distributes over complex addition.
    \end{itemize}
    Hence $\C$ is a ring.
    
    \newpage
     
    We now show that complex multiplication is commutative. Let $a+bi, c+di \in C$. Then we see
    \begin{align*}
        (a+bi)(c+di) &= (ac-bd) + (ad+bc)i\\
        &= (ca-db) + (da+cb)i & (\times\text{ is commutative, } \myref{axiom-multiplication-is-commutative})\\
        &= (c+di)(a+bi)
    \end{align*}
    so complex multiplication is commutative.

    Finally we show that complex multiplication has an identity. Consider $1 + 0i \in \C$. Note that
    \[
        (1+0i)(a+bi) = (1a-0b) + (1b+0a)i = a+bi,
    \]
    and since complex multiplication is commutative, therefore $1+0i$ is the multiplicative identity in $\C$.
    
    Therefore $\C$ is a commutative ring with identity.
\end{proof}

These are just some examples of rings; we explore more later in this chapter.
\begin{exercise}\label{exercise-ring-of-integers-is-a-ring}
    Prove that $\Z$ is a ring under regular addition and multiplication.\newline
    (\textit{You do \textbf{not} need to prove the \textbf{Distributive} axiom.})
\end{exercise}

\section{General Properties of Rings}
We list some properties of rings here. For each of the propositions, assume $R$ is a ring.

\begin{proposition}\label{prop-multiplying-by-zero-is-zero}
    $0x = x0 = 0$ for all $x \in R$.
\end{proposition}
\begin{proof}
    We note that
    \begin{align*}
        0x &= (0 + 0)x & (0 \text{ is additive inverse})\\
        &= 0x + 0x & (\text{by \textbf{Distributive} axiom})
    \end{align*}
    so by `subtracting' $0x$ on both sides (i.e., adding $-0x$ on both sides) we see $0 = 0x$.
    
    Also
    \begin{align*}
        x0 &= x(0 + 0) & (0 \text{ is additive inverse})\\
        &= x0 + x0 & (\text{by \textbf{Distributive} axiom})
    \end{align*}
    so by `subtracting' $x0$ on both sides we see $0 = x0$.
    
    Therefore $0x = x0 = 0$ for all $x \in R$.
\end{proof}

\begin{proposition}\label{prop-product-of-element-and-additive-inverse-is-additive-inverse-of-product}
    $(-a)b = a(-b) = -(ab)$ for any $a$ and $b$ in $R$.
\end{proposition}
\begin{proof}
    We show that $(-a)b = -(ab)$ and $a(-b) = -(ab)$ to complete the proof.
    \begin{itemize}
        \item Note $(-a)b + ab = (-a + a)b = 0b = 0$ by \textbf{Distributive} axiom. Hence by subtracting $ab$ on both sides we see $(-a)b = -(ab)$.
        \item Note also $a(-b) + ab = a(-b + b) = a0 = 0$ by \textbf{Distributive} axiom. Hence by subtracting $ab$ on both sides we see $a(-b) = -(ab)$.
    \end{itemize}
    Result follows.
\end{proof}

\begin{proposition}
    $(-a)(-b) = ab$ for any $a$ and $b$ in $R$.
\end{proposition}
\begin{proof}
    See \myref{exercise-product-of-additive-inverses} (later).
\end{proof}

\begin{proposition}
    If $R$ has an identity, it is unique.
\end{proposition}
\begin{proof}
    Suppose 1 and $1'$ are identities, and consider the sum $1 + 1'$. Then
    \begin{align*}
        1 + 1' &= 1\times(1+1') & (\text{multiplying by identity }1)\\
        &= 1\times1 + 1\times1' & (\text{by \textbf{Distributive} axiom})\\
        &= 1 + 1. & (1 \text{ and } 1' \text{ are identities})
    \end{align*}
    Subtracting 1 on both sides yields $1 = 1'$, meaning that the identity is unique.
\end{proof}

\begin{exercise}\label{exercise-product-of-additive-inverses}
    Show that $(-a)(-b) = ab$ for any $a$ and $b$ in $R$.
\end{exercise}

% \section{Some Non-Obvious Rings}
\section{Matrix Rings}
The rings that we explored in previous sections can be thought of as the `obvious' rings, since they are number systems. As rings were made to generalize number systems, they should clearly be rings. However, there are less obvious rings.

% \subsection{Matrix Rings}
We looked at matrices in the context of the General/Special Linear Group of matrices. Here we see that matrices in fact form rings, known as matrix rings. Before that though, we need to define the operations within that ring.

\begin{definition}[Matrix Addition]\index{matrix addition}
    For any two matrices $\textbf{A}$ and $\textbf{B}$ with $n$ rows and columns and entries in the ring $(R, \oplus, \otimes)$, their sum is the matrix $\textbf{C} = \textbf{A} + \textbf{B}$ with $n$ rows and columns such that
    \[
        c_{i,j} = a_{i,j} \oplus b_{i,j}
    \]
    for all $i,j \in \{1, 2, \dots, n\}$.
\end{definition}
\begin{definition}[Matrix Multiplication]\index{matrix multiplication}
    For any two matrices $\textbf{A}$ and $\textbf{B}$ with $n$ rows and columns and entries in the ring $(R, \oplus, \otimes)$, their product is the matrix $\textbf{C} = \textbf{AB}$ with $n$ rows and columns such that, for all $i,j \in \{1, 2, \dots, n\}$, we have
    \begin{align*}
        c_{i,j} &= (a_{i,1}\otimes b_{1,j}) \oplus (a_{i,2}\otimes b_{2,j}) \oplus \cdots \oplus (a_{i,n}\otimes b_{n,j})\\
        &= \bigoplus_{k=1}^n (a_{i,k}\otimes b_{k,j}).
    \end{align*}
\end{definition}

We also define two matrices that will become useful when we work with matrix rings.
\begin{definition}
    The \textbf{zero matrix}\index{zero matrix} with $n$ rows and columns is
    \[
        \ZeroM{n} = 
        \begin{pmatrix}
            0 & 0 & 0 & \cdots & 0 \\
            0 & 0 & 0 & \cdots & 0 \\
            0 & 0 & 0 & \cdots & 0 \\
            \vdots & \vdots & \vdots & \ddots & \vdots \\
            0 & 0 & 0 & \cdots & 0 \\
        \end{pmatrix}
    \]
    where 0 is the additive identity (i.e. zero) in the ring $(R, \oplus, \otimes)$.
\end{definition}
\begin{definition}
    The \textbf{identity matrix}\index{identity matrix} with $n$ rows and columns is
    \[
        \IdentityM{n} = 
        \begin{pmatrix}
            1 & 0 & 0 & \cdots & 0 \\
            0 & 1 & 0 & \cdots & 0 \\
            0 & 0 & 1 & \cdots & 0 \\
            \vdots & \vdots & \vdots & \ddots & \vdots \\
            0 & 0 & 0 & \cdots & 1 \\
        \end{pmatrix}
    \]
    where 0 and 1 are the additive and multiplicative identities (i.e. zero and one) in the ring $(R, \oplus, \otimes)$ respectively. That is, the identity matrix is the matrix with 1s in the leading diagonal.
\end{definition}

We can now define what is a matrix ring.
\begin{definition}
    Let $(R, \oplus, \otimes)$ be a ring and $n$ be a positive integer. Then $\Mn{n}{R}$ under matrix addition and multiplication is a ring, known as the \textbf{matrix ring}\index{matrix ring} with elements in $(R, \oplus, \otimes)$.
\end{definition}
\begin{proposition}
    $\Mn{n}{R}$ is a ring with identity.
\end{proposition}
\begin{proof}
    We need to prove that the ring axioms hold.
    \begin{itemize}
        \item \textbf{Addition-Abelian}: We first prove that $(\Mn{n}{R}, +)$ is indeed an abelian group.
        \begin{itemize}
            \item \textbf{Closure}: Clearly the sum of any two matrices in $\Mn{n}{R}$ is also a square matrix with $n$ rows with elements inside $R$, meaning that $\Mn{n}{R}$ is closed under matrix addition.

            \item \textbf{Associativity}: Let the matrices $\textbf{A}$, $\textbf{B}$, and $\textbf{C}$ belong inside $\Mn{n}{R}$. Let $\textbf{P} = \textbf{A} + (\textbf{B} + \textbf{C})$ and $\textbf{Q} = (\textbf{A} + \textbf{B}) + \textbf{C}$. We note that $\textbf{P} = \textbf{Q}$ as
            \[
                p_{i,j} = a_{i,j} \oplus (b_{i,j} \oplus c_{i,j}) = (a_{i,j} \oplus b_{i,j}) \oplus c_{i,j} = q_{i,j}
            \]
            by associativity of $\oplus$, which proves that matrix addition is associative.
    
            \item \textbf{Identity}: We show that $\ZeroM{n}$ is the additive identity in $\Mn{n}{R}$. Let $\textbf{M} \in \Mn{n}{R}$; let $\textbf{N} = \textbf{M} + \ZeroM{n}$. Note that $n_{i,j} = m_{i,j} \oplus 0 = m_{i,j}$ so $\textbf{M} + \ZeroM{n} = \textbf{M}$. Therefore $\textbf{M} + \ZeroM{n} = \textbf{M}$ for any matrix in $\Mn{n}{R}$.
            
            \item \textbf{Inverse}: Let $\textbf{A} \in \Mn{n}{R}$. Define the matrix $\textbf{B} = -\textbf{A}$ such that $b_{i,j} = -a_{i,j}$. That is, $b_{i,j}$ contains the additive inverse of $a_{i,j}$ in the ring $R$. Then one sees that $\textbf{A} + \textbf{B} = \ZeroM{n}$. (We denote the additive inverse of a matrix $\textbf{M}$ by $-\textbf{M}$).

            \item \textbf{Commutative}: Let $\textbf{A}, \textbf{B} \in \Mn{n}{R}$. Set $\textbf{C} = \textbf{A} + \textbf{B}$ and $\textbf{D} = \textbf{B} + \textbf{C}$. Consider $c_{i,j} = a_{i,j} \oplus b_{i,j}$. Since $\oplus$ is commutative, thus $a_{i,j} \oplus b_{i,j} = b_{i,j} \oplus a_{i,j}$. But $d_{i,j} = b_{i,j} \oplus a_{i,j}$, so we have $c_{i,j} = d_{i,j}$. Therefore $\textbf{C} = \textbf{D}$.
        \end{itemize}

        \item \textbf{Multiplication-Semigroup}: We show that $(\Mn{n}{R}, \cdot)$ is a semigroup.
        \begin{itemize}
            \item \textbf{Closure}: In \myref{subsection-intro-to-matrices} we showed that matrix multiplication produces another $n \times n$ matrix. Furthermore the entries of the new matrix are elements of $R$. Hence $\Mn{n}{R}$ is closed under matrix multiplication.
        
            \item \textbf{Associativity}: We proved matrix multiplication is associative in \myref{subsection-GLR-matrix-group}.
        \end{itemize}
        
        \item \textbf{Distributive}: We prove only $\textbf{A}(\textbf{B} + \textbf{C}) = (\textbf{AB}) + (\textbf{AC})$ as the other case is proven similarly. Let $\textbf{R} = \textbf{A}(\textbf{B} + \textbf{C})$, $\textbf{G} = \textbf{AB}$, and $\textbf{H} = \textbf{AC}$. We note
        \begin{align*}
            r_{i,j} &= \bigoplus_{k=1}^n \left(a_{i,k} \otimes \left(b_{k,j} \oplus c_{k,j}\right)\right)\\
            &= \bigoplus_{k=1}^n \left((a_{i,k} \otimes b_{k,j}) \oplus (a_{i,k} \otimes c_{k,j})\right)\\
            &= \left(\bigoplus_{k=1}^n (a_{i,k} \otimes b_{k,j})\right) \oplus \left(\bigoplus_{k=1}^n (a_{i,k} \otimes c_{k,j})\right)\\
            &= g_{i,j}\oplus h_{i,j}
        \end{align*}
        which means $\textbf{R} = \textbf{G} + \textbf{H}$.
    \end{itemize}
    As all the ring axioms are satisfied, thus $\Mn{n}{R}$ is a ring.

    We now show that $\Mn{n}{R}$ has a multiplicative identity, namely the identity matrix $\IdentityM{n}$. Let $\textbf{A} \in \Mn{n}{R}$ and let $\textbf{B} = \IdentityM{n}$. Note that $b_{i,j} = 1$ if and only if $i = j$.
    \begin{itemize}
        \item Let $\textbf{C} = \textbf{AB}$ and we see
        \begin{align*}
            &c_{i,j}\\
            &= \bigoplus_{k=1}^n(a_{i,k}\otimes b_{k,j})\\
            &= (a_{i,1}\otimes b_{1,j}) \oplus \cdots \oplus (a_{i,j-1}\otimes b_{j-1,j}) \oplus (a_{i,j}\otimes b_{j,j})\\
            &\quad\quad\oplus (a_{i,{j+1}}\otimes b_{j+1,j}) \oplus \cdots \oplus (a_{i,n}\otimes b_{n,j})\\
            &= (a_{i,1}\otimes 0) \oplus \cdots \oplus (a_{i,{j-1}}\otimes 0)\oplus (a_{i,j}\otimes 1) \oplus (a_{i,{j+1}}\otimes 0) \oplus \cdots \oplus (a_{i,n}\otimes 0)\\
            &= 0 \oplus \cdots \oplus 0 \oplus a_{i,j} \oplus 0 \oplus \cdots \oplus 0\\
            &= a_{i,j}
        \end{align*}
        so $\textbf{A}\IdentityM{n} = \textbf{A}$.

        \item Now let $\textbf{D} = \textbf{BA}$ and we also see
        \begin{align*}
            &d_{i,j}\\
            &= \bigoplus_{k=1}^n(b_{i,k}\otimes a_{k,j})\\
            &= (b_{i,1}\otimes a_{1,j}) \oplus \cdots \oplus (b_{i,i-1}\otimes b_{i-1,j}) \oplus (b_{i,i}\otimes a_{i,j})\\
            &\quad\quad\oplus (b_{i,{i+1}}\otimes a_{i+1,j}) \oplus \cdots \oplus (b_{i,n}\otimes a_{n,j})\\
            &= (0 \otimes a_{1,j}) \oplus \cdots \oplus (0\otimes a_{i-1,j})\oplus (1\otimes a_{i,j}) \oplus (0\otimes a_{i+1,j}) \oplus \cdots \oplus (0\otimes a_{n,j})\\
            &= 0 \oplus \cdots \oplus 0 \oplus a_{i,j} \oplus 0 \oplus \cdots \oplus 0\\
            &= a_{i,j}
        \end{align*}
        so $\IdentityM{n}\textbf{A} = \textbf{A}$.
    \end{itemize}
    Therefore the identity matrix $\IdentityM{n}$ is the multiplicative identity.

    Hence $\Mn{n}{R}$ is a ring with identity.
\end{proof}

% \subsection{Hamilton's Quaternions}
% The quaternions is a way to extend the complex numbers into 4 dimensions. Irish mathematician William Rowan Hamilton was looking for a way represent points in 3-dimensional space using numbers. He knew how to add and subtract triples of numbers, but had difficulty in defining a way to multiply and divide these numbers, just like in the ring of complex numbers. The breakthrough in quaternions came in 1843 when he thought of using 4-dimensional numbers instead of 3-dimensional numbers.

% We define what quaternions are, and what the quaternion ring is.
% \begin{definition}
%     A \textbf{quaternion}\index{quaternion} is an expression of the form
%     \[
%         a + b\qi + c\qj + d\qk
%     \]
%     where $a,b,c,d \in \R$ and $\qi$, $\qj$, and $\qk$ are quantities such that
%     \begin{itemize}
%         \item $\qi\qj = -\qj\qi = \qk$;
%         \item $\qj\qk = -\qk\qj = \qi$;
%         \item $\qk\qi = -\qi\qk = \qj$; and
%         \item $\qi^2 = \qk^2 = \qk^2 = \qi\qj\qk = -1$.
%     \end{itemize}
% \end{definition}
% \begin{definition}
%     The \textbf{quaternion ring}\index{quaternion ring} is
%     \[
%         \H = \{a + b\qi + c\qj + d\qk \vert a,b,c,d \in \R\}
%     \]
%     where, for two quaternions $q_1 = a_1+b_1\qi+c_1\qj+d_1\qk$ and $q_2 = a_2+b_2\qi+c_2\qj+d_2\qk$, addition is
%     \[
%         q_1 + q_2 = (a_1+a_2) + (b_1+b_2)\qi + (c_1+c_2)\qj + (d_1+d_2)\qk
%     \]
%     and multiplication is
%     \begin{align*}
%         q_1q_2 &= (a_1a_2 - b_1b_2 - c_1c_2 - d_1d_2)\\
%         &+(a_1b_2 + b_1a_2 + c_1d_2 - d_1c_2)\qi\\
%         &+(a_1c_2 - b_1d_2 + c_1a_2 + d_1b_2)\qj\\
%         &+(a_1d_2 + b_1c_2 - c_1b_2 + d_1a_2)\qk.
%     \end{align*}
% \end{definition}

\section{More Definitions}
Suppose $R$ is a ring.
\begin{definition}
    We say that $a \neq 0$ is a \textbf{zero divisor}\index{zero divisor} in $R$ if there exists $b \neq 0$ such that $ab = 0$.
\end{definition}
\begin{example}
    Consider the ring $\Z_{12}$. Clearly 4 and 6 are in $\Z_{12}$, and their product is $24 = 2 \times 12 = 0$ in $\Z_{12}$. Hence 4 and 6 are zero divisors in $\Z_{12}$.
\end{example}
\begin{example}
    Let $R$ be the ring of functions with domain and codomain $[0, 1]$. We claim that $R$ has zero divisors. Consider the functions
    \begin{align*}
        f:[0,1]\to[0,1], x &\mapsto x\\
        g:[0,1]\to[0,1], x &\mapsto \begin{cases}
            0 & \text{ if } x \neq 0\\
            1 & \text{ if } x = 0
        \end{cases}
    \end{align*}
    Clearly neither of them are the zero function. However, consider $f(x)g(x)$.
    \begin{itemize}
        \item If $x \neq 0$, then $g(x) = 0$ which means $f(x)g(x) = 0$.
        \item If $x = 0$, then $f(x) = 0$ which means $f(x)g(x) = 0$.
    \end{itemize}
    Hence their product is the zero function, meaning that $R$ has zero divisors $f$ and $g$.
\end{example}
\begin{exercise}
    Does the ring $\Mn{2}{\mathbb{R}}$ have zero divisors?
\end{exercise}
We note one property about zero divisors, which will be used in future chapters.
\begin{proposition}\label{prop-zero-divisors-have-no-inverses}
    Zero divisors do not have inverses.
\end{proposition}
\begin{proof}
    Assume $a \neq 0$ and $b \neq 0$ are zero divisors in the ring $R$, so $ab = 0$. Seeking a contradiction, assume $a$ has an inverse, so
    \[
        b = (a^{-1}a)b = a^{-1}(ab) = a^{-1}0 = 0    
    \]
    which contradicts $b \neq 0$. Hence a zero divisor has no inverse.
\end{proof}

\begin{definition}
    Suppose $R$ is a ring with identity such that $0 \neq 1$. An element $u \in R$ is called a \textbf{unit}\index{unit} if there exists a $v \in R$ such that $uv=vu=1$. Equivalently, $u$ is a unit if it has a multiplicative inverse.
\end{definition}
\begin{example}
    3 and 7 are units in $\Z_{10}$ since $3 \times 7 = 7 \times 3 = 21 = 1$ in $\Z_{10}$.
\end{example}

\begin{definition}
    Suppose $R$ is a ring with identity such that $0 \neq 1$. If every non-zero element $x \in R$ is a unit, then $R$ is said to be a \textbf{division ring}\index{division ring}.
\end{definition}

\begin{definition}
    A commutative division ring is called a \textbf{field}\index{field}.
\end{definition}

\begin{example}
    Earlier, we shown that $\R$ is a commutative ring. We now show that $\R$ is actually a field by noting that every non-zero $x \in \R$ has a reciprocal $\frac1x$ that is a real number (\myref{axiom-reciprocal}) such that $x\left(\frac1x\right) = \left(\frac1x\right)x = 1$. Thus every non-zero $x$ in $\R$ is a unit, meaning that $\R$ is a division ring. Coupled with the fact that $\R$ is a commutative ring means that $\R$ is a field.
\end{example}
\begin{example}
    We also shown earlier that $\C$ is a commutative ring. We note that any non-zero complex number $z = a + bi$ has a multiplicative inverse given by
    \[
        w = \frac{a}{a^2+b^2} - \frac{b}{a^2+b^2}i
    \]
    since
    \begin{align*}
        zw &= (a+bi)\left(\frac{a}{a^2+b^2} - \frac{b}{a^2+b^2}i\right)\\
        &= \frac{(a+bi)(a-bi)}{a^2+b^2}\\
        &= \frac{a^2 - b^2i^2}{a^2+b^2}\\
        &= \frac{a^2+b^2}{a^2+b^2} & (i^2 = -1)\\
        &= 1.
    \end{align*}
    Thus any non-zero complex number is a unit, meaning that $\C$ is a division ring. As $\C$ is also a commutative ring, this means that $\C$ is a field.
\end{example}

We look at fields in more detail in part III.

\begin{exercise}
    Which of the following rings, if any, are fields?
    \begin{partquestions}{\alph*}
        \item $\Z$
        \item $\Q$
    \end{partquestions}
\end{exercise}

\section{Subrings}
We end this chapter off with an exploration about subrings.

\begin{definition}
    Let $R$ be a ring and $S$ be a subset of $R$. Then $S$ is a \textbf{subring}\index{subring} of $R$ if
    \begin{itemize}
        \item $(S, +) \leq (R, +)$, that is, the subset $S$ under addition is a subgroup of $R$ under addition; and
        \item for all $a$ and $b$ in $S$ we have $ab \in S$, i.e. $S$ is closed under multiplication.
    \end{itemize}
\end{definition}
\begin{remark}
    Alternatively, one may show that $S$ is a ring to prove that $S$ is a subring of $R$.
\end{remark}

\begin{example}
    We know that $\Z$ and $\Q$ are rings, and clearly $\Z \subseteq \Q$. Hence $\Z$ is a subring of $\Q$. Similarly, since $\Q \subseteq \R$ and $\R \subseteq \C$, thus $\Q$ is a subring of $\R$ and $\R$ is a subring of $\C$.
\end{example}

\begin{example}
    Consider the set of \textbf{gaussian integers}\index{gaussian integers}
    \[
        \Z[i] = \{a + bi \vert a,b\in\mathbb{Z}\},
    \]
    read as ``$\Z$ adjoin $i$''. We first show that $\Z[i]$ is a subring of $\C$.
    \begin{proof}
        Clearly $\Z[i] \subseteq \C$. We show that $(\Z[i], +) \leq (\C, +)$.
        \begin{itemize}
            \item Clearly the identity of $(\C, +)$, which is 0, is inside $(\Z[i], +)$ as $0 = 0 + 0i$.
            \item For any $a + bi, c+di \in \Z[i]$, we have
            \[
                (a+bi) + (-(c+di)) = (a-c) + (b-d)i \in \Z[i].
            \]
        \end{itemize}
        Thus the subgroup test (\myref{thrm-subgroup-test}) tells us that $(\Z[i], +) \leq (\C, +)$.

        Now we show that $\Z[i]$ is closed under multiplication. Let $a + bi, c+di \in \Z[i]$. Note that $ac, bd, ad, bc \in \Z$; the product of the two gaussian integers is
        \begin{align*}
            (a+bi)(c+di) &= (ac-bd) + (ad+bc)i\\
            &\in \Z[i]
        \end{align*}
        so $\Z[i]$ is closed under multiplication.
        
        Therefore $\Z[i]$ is a subring of $\C$.
    \end{proof}
\end{example}
\begin{exercise}
    Show that
    \[
        R = \left\{\begin{pmatrix}a&a\\a&a\end{pmatrix} \vert a \in \R\right\}
    \]
    is a subring of $\Mn{2}{\mathbb{R}}$.
\end{exercise}

\newpage

\section{Problems}
\begin{problem}
    Let $R$ be a ring. Prove that if $u \in R$ is a unit then so is $-u$.
\end{problem}

\begin{problem}
    Prove that the trivial ring is the unique ring with identity in which $0 = 1$.
\end{problem}

\begin{problem}
    Let $R$ be a set with an operation $\ast$ such that for all elements $x$ and $y$ in $R$ we have $x \ast y \in R$. If $(R, \ast, \ast)$ is a ring, describe the elements in $R$.
\end{problem}

\begin{problem}
    Prove that it is impossible for an element of a ring $R$ to be both a zero divisor and a unit.
\end{problem}

\begin{problem}
    Let $R$ be a ring with identity 1, and let $x$ be an element from that ring.
    \begin{partquestions}{\roman*}
        \item Find \textbf{four} closed forms for the geometric series $1 + x + x^2 + x^3 + \cdots + x^n$.
        \item What are the condition(s) such that the closed forms are valid?
        \item Evaluate 112 in the ring $\Z_{37}$.
        \item Hence, using the result(s) above, evaluate
        \[
            1 + 2^3 + 2^6 + 2^9 + \cdots + 2^{72}
        \]
        in the ring $\Z_{37}$.
    \end{partquestions}
\end{problem}

\begin{problem}
    Show that
    \[
        \Q[\sqrt2] = \{a + b\sqrt2 \vert a,b \in \Q\}
    \]
    is a ring. Hence show it is a field.
\end{problem}

\begin{problem}
    Let
    \[
        R = \left\{\begin{pmatrix}a&b\\0&0\end{pmatrix} \vert a,b \in \R\right\}    
    \]
    be a ring under matrix addition and multiplication.
    \begin{partquestions}{\roman*}
        \item Show that $R$ has no identity.
        \item Show that $R$ contains a non-trivial subring $S$ with identity.
    \end{partquestions}
\end{problem}

\begin{problem}
    A ring $R$ is called a \textbf{Boolean ring}\index{Boolean ring} if $r^2 = r$ for all $r \in R$.
    \begin{partquestions}{\roman*}
        \item Show that $r = -r$ for all $r \in R$.
        \item Prove that every Boolean ring is commutative.
    \end{partquestions}
\end{problem}

\begin{problem}
    Let $R$ be a commutative ring with identity. We say that an element $x$ in $R$ is \textbf{nilpotent}\index{nilpotent} if there exists a positive integer $n$ such that $x^n = 0$.
    \begin{partquestions}{\roman*}
        \item Show that the product of two units is a unit.
        \item Let $u \in R$ be a unit and $x \in R$ be nilpotent. Show that $ux$ is nilpotent.
        \item Show that $u - x$ is a unit.
    \end{partquestions}
\end{problem}

\chapter{Integral Domains}
We explored the absolute basics of rings in the previous chapter. When working in abstract algebra, we usually prefer rings with slightly more structure as they allow us to generate more useful and applicable results. Integral domains are an important category of rings, which we explore in this chapter.

\section{What is an Integral Domain?}
\begin{definition}
    An \textbf{integral domain}\index{integral domain}, generally denoted $D$, is a commutative ring with identity and contains no zero divisors.
\end{definition}
\begin{remark}
    Recall that an element $a \neq 0$ is a zero divisor if there exists a $b \neq 0$ such that $ab = 0$. So, equivalently, if $a \neq 0$ and $b \neq 0$ then $ab \neq 0$ for any $a$ and $b$ in the integral domain.
\end{remark}
\begin{remark}
    What this property grants us is a ``cancellation law'', similar to that for groups. We will prove this in \myref{prop-integral-domain-cancellation-law}. However, there may not be inverses when performing multiplication. This is similar to how the integers do not have multiplicative inverses in general, hence the name of ``integral domain''.
\end{remark}

\begin{example}\label{example-integers-is-integral-domain}
    We show that the integers, $\Z$, the namesake for integral domains, is indeed an integral domain. We note that $\Z$ is a commutative ring as multiplication is commutative. Furthermore, 1 is the multiplicative identity in $\Z$. All that remains is to show that there does not exist any zero divisors in $\Z$.

    Suppose $a$ and $b$ are non-zero elements in $\Z$. We show that $ab \neq 0$ to prove that there does not exist any zero divisors. Let's first consider the case when $b > 0$. We may write
    \[
        ab = \underbrace{a + a + \cdots + a}_{b \text{ times}}
    \]
    which clearly is not zero since $a \neq 0$. Now if $b < 0$ then note
    \[
        ab = (-a)(-b) = \underbrace{(-a) + (-a) + \cdots + (-a)}_{(-b) \text{times}}
    \]
    which is also not zero as $-a \neq 0$. Hence there are no zero divisors in $\Z$.

    Therefore $\Z$ is a commutative ring with identity without any zero divisors, meaning that $\Z$ is an integral domain.
\end{example}

\begin{example}
    We show that the ring $\Z[i]$, the gaussian integers, is an integral domain. Similar to $\Z$, we note multiplication is commutative with identity 1.

    Suppose $w$ and $z$ are non-zero elements in $\Z[i]$; write $w = a+bi$ and $z = c+di$. Note that $a^2+b^2 \neq 0$ and $c^2 + d^2 \neq 0$. By definition of complex multiplication we have
    \[
        wz = (a+bi)(c+di) = (ac-bd) + (ad+bc)i.
    \]
    We just need to show that $ac-bd \neq 0$ and $ad+bc \neq 0$. Note that for any real numbers $x$ and $y$, we have $x^2 + y^2 \neq 0$ if and only if both $x$ and $y$ are non-zero. So we see
    \begin{align*}
        (ac-bd)^2 + (ad+bc)^2 &= (a^2c^2 - 2abcd + b^2d^2) + (a^2d^2 + 2abcd + b^2c^2)\\
        &= a^2c^2 + a^2d^2 + b^2c^2 + b^2d^2\\
        &= (a^2 + b^2)(c^2 + d^2)\\
        &\neq 0
    \end{align*}
    which hence means that $ac-bd \neq 0$ and $ad+bc \neq 0$. Therefore $wz \neq 0$, meaning that $\Z[i]$ has no zero divisors. Thus $\Z[i]$ is indeed an integral domain.
\end{example}

We note an interesting result regarding fields here.
\begin{proposition}\label{prop-field-is-integral-domain}
    Any field is an integral domain.
\end{proposition}
\begin{proof}
    Any non-zero element in a field is a unit, i.e. has an inverse. Now by \myref{prop-zero-divisors-have-no-inverses}, any zero divisor must not have an inverse; the converse being that any element that has an inverse cannot be a zero divisor. Hence, there are no zero divisors in a field. Now a field is commutative, therefore meaning that the field is an integral domain.
\end{proof}

Let's look at some non-examples of integral domains.
\begin{example}
    The ring $\Mn{n}{R}$ where $R$ is any ring is not an integral domain since $\Mn{n}{R}$ is not commutative.
\end{example}
\begin{example}
    The ring $2\Z$ is not an integral domain as it does not contain a multiplicative identity (namely, 1).
\end{example}
\begin{example}
    The ring $\Z \times \Z$ under pairwise addition and multiplication is not an integral domain as $(0,1) \neq (0, 0)$ and $(1, 0) \neq (0, 0)$ but $(0,1)(1,0) = (0, 0)$, meaning that $\Z \times \Z$ contains at least one zero divisor.
\end{example}

\begin{exercise}
    Prove that the ring
    \[
        \Z[\sqrt2] = \left\{a + b\sqrt2 \vert a,b \in \Z\right\}
    \]
    is an integral domain.
\end{exercise}
\begin{exercise}
    Explain why all $\Z_n$ where $n$ is a positive composite number are not integral domains.
\end{exercise}

\section{General Results}
With an introduction of integral domains out of the way, we introduce some general results applicable to integral domains.
\begin{proposition}[Cancellation Law for Integral Domains]\index{integral domain!cancellation law}\label{prop-integral-domain-cancellation-law}
    Let $D$ be an integral domain, $r, x, y \in D$, and $r \neq 0$. Then the following statements are equivalent.
    \begin{enumerate}[label=(\arabic*)]
        \item $x = y$
        \item $rx = ry$
        \item $xr = yr$
    \end{enumerate}
\end{proposition}
\begin{proof}
    We prove the statements in order.
    \begin{itemize}
        \item $\boxed{(1) \implies (2)}$ Given $x = y$, this means $x - y = 0$. Multiplying $r$ on the left on both sides yields $r(x-y) = r0 = 0$. Distributing yields $rx - ry = 0$ which hence means $rx = ry$.
        
        \item $\boxed{(2) \implies (3)}$ Given $rx = ry$. Multiplying $r$ on the right on both sides yields $rxr = ryr$, thereby meaning $rxr - ryr = 0$. Factoring yields $r(xr - yr) = 0$. As $R$ is an integral domain, the only way for this to occur is if $r = 0$ (which is impossible) or $xr - yr = 0$. Therefore $xr = yr$.
        
        \item $\boxed{(3) \implies (1)}$ Given $xr = yr$ we may write $xr - yr = (x-y)r = 0$. Since $R$ is an integral domain we must have $x - y = 0$ or $r = 0$ (impossible as $r \neq 0$). Hence $x = y$.
    \end{itemize}
    This proves the proposition.
\end{proof}

\begin{theorem}\label{thrm-finite-integral-domain-is-field}
    Every finite integral domain is a field.
\end{theorem}
\begin{proof}
    A finite integral domain is a commutative ring with identity; all that remains is to prove that every non-zero element has an inverse.

    Suppose $D$ is a finite integral domain and take a non-zero element $r$ in $D$. Consider
    \begin{align*}
        A &= \{r^k \vert k \in \Z,\;k\geq 0\}\\
        &= \{1, r, r^2, r^3, \dots\}\\
        &\subseteq D.
    \end{align*}
    Note that since $D$ is finite, $A$ must be finite as well. Therefore there must exist positive integers $m$ and $n$ with $m > n$ such that $r^m = r^n$, meaning $r^m - r^n = 0$. As $m > n$, thus $r^n\left(r^{m-n}-1\right) = 0$. Hence, as there are no zero divisors in $R$, either $r^n = 0$ or $r^{m-n} - 1 = 0$.

    Now if $r^n = 0$ then $n > 1$ (since $r^1 = r \neq 0$ by assumption). Note $r^n = rr^{n-1} = 0$, which hence means $r^{n-1} = 0$ as $r \neq 0$ and $D$ has no zero divisors. But we may repeat this argument on $n - 1$ to eventually terminate at $r = 0$, which is a contradiction. Hence we must conclude that $r^{m-n} - 1 = 0$. We may write $r^{m-n}-1 = 0$ as $rr^{m-n-1} = 1$ which means that $r$ is a unit (as $r^{m-n-1}$ is its inverse). Therefore every non-zero element has an inverse.

    Hence $D$ is a field.
\end{proof}

\begin{remark}
    We note that finite fields are also known as \textbf{Galois fields}\index{Galois Field}, so-named in honour of \'Evariste Galois. We explore more properties of fields and Galois fields in part III.
\end{remark}

\begin{example}\label{example-Zp-is-field}
    Let $p$ be a prime and consider the ring $\Z_p$. Clearly multiplication is commutative with identity 1. Also, as the only way to factor $p$ is $1 \times p$ and since $p \notin \Z_p$, thus there are no zero divisors in $\Z_p$. Hence $\Z_p$ is an integral domain. Now $\Z_p$ only has $p$ elements (namely the integers 0 to $p - 1$), so it is finite. By \myref{thrm-finite-integral-domain-is-field} we know that $\Z_p$ is a field.
\end{example}

\begin{example}
    We show that $\Z_3[i] = \{a+bi \vert a,b \in \Z_3\}$ is a finite integral domain in order to show that it is a field. Clearly multiplication is commutative with identity 1, so we just need to show that there are no zero divisors in $\Z_3[i]$.
    \begin{table}[h]
        \centering
        \resizebox{\textwidth}{!}{
            \begin{tabular}{|l|l|l|l|l|l|l|l|l|}
            \hline
            $\boldsymbol{\times}$ & $\boldsymbol{1}$ & $\boldsymbol{2}$ & $\boldsymbol{i}$ & $\boldsymbol{1+i}$ & $\boldsymbol{2+i}$ & $\boldsymbol{2i}$ & $\boldsymbol{1+2i}$ & $\boldsymbol{2+2i}$ \\ \hline
            $\boldsymbol{1}$    & 1      & 2      & $i$    & $1+i$  & $2+i$  & $2i$   & $1+2i$ & $2+2i$ \\ \hline
            $\boldsymbol{2}$    & 2      & 1      & $2i$   & $2+2i$ & $1+2i$ & $i$    & $2+i$  & $1+i$  \\ \hline
            $\boldsymbol{i}$    & $i$    & $2i$   & 2      & $2+i$  & $2+2i$ & 1      & $1+i$  & $1+2i$ \\ \hline
            $\boldsymbol{1+i}$  & $1+i$  & $2+2i$ & $2+i$  & $2i$   & 1      & $1+2i$ & 2      & $i$    \\ \hline
            $\boldsymbol{2+i}$  & $2+i$  & $1+2i$ & $2+2i$ & 1      & $i$    & $1+i$  & $2i$   & 2      \\ \hline
            $\boldsymbol{2i}$   & $2i$   & $i$    & 1      & $1+2i$ & $1+i$  & 2      & $2+2i$ & $2+i$  \\ \hline
            $\boldsymbol{1+2i}$ & $1+2i$ & $2+i$  & $1+i$  & 2      & $2i$   & $2+2i$ & $i$    & 1      \\ \hline
            $\boldsymbol{2+2i}$ & $2+2i$ & $1+i$  & $1+2i$ & $i$    & 2      & $2+i$  & 1      & $2i$   \\ \hline
            \end{tabular}
        }
    \end{table}
    
    From the table, we have shown that no two non-zero elements in $\Z_3[i]$ multiply to zero. Therefore $\Z_3[i]$ has no zero divisors, which means $\Z_3[i]$ is an integral domain. Furthermore, as $\Z_3[i]$ is finite, therefore we know that it is a finite field by \myref{thrm-finite-integral-domain-is-field}.
\end{example}

\begin{exercise}\label{exercise-Zn2[alpha]}
    Is the ring
    \[
        \Z_2[\alpha] = \{a+b\alpha \vert a,b\in\Z_2\}
    \]
    where $\alpha^2 = 1 + \alpha$ under regular addition and multiplication a field?
\end{exercise}

\section{Characteristic of a Ring}
Before we introduce the characteristic of a ring, we clarify the meaning of the order of an element in the ring.
\begin{definition}
    Let $R$ be a ring and $x$ an element of the ring $R$.
    \begin{itemize}
        \item The \textbf{additive order}\index{order!additive} of $x$ is the order of $x$ in the group $(R, +)$. We denote the additive order of $x$ by $|x|_+$.
        \item If $R$ is a group with identity 1, the \textbf{multiplicative order}\index{order!multiplicative} of $x$ is the smallest positive integer $n$ such that $x^n = 1$ and is denoted $|x|_\times$, if it exists.
    \end{itemize}
\end{definition}

\begin{definition}
    Let $R$ be a ring. We say that \textbf{characteristic}\index{characteristic} of $R$ is $n$ if $n$ is the smallest positive integer such that
    \[
        nr = \underbrace{r+r+\cdots+r}_{n \text{ times}} = 0
    \]
    for all $r \in R$. We then write $\Char{R} = n$. If no such $n$ exists we say $\Char{R} = 0$.
\end{definition}

We look at two properties about the characteristic.
\begin{proposition}
    If $R$ is a ring with identity and $\Char{R} \neq 0$, then $\Char{R} = |1|_+$.
\end{proposition}
\begin{proof}
    Let $\Char{R} = n$ and $|1|_+ = m$. Our goal is to show that $m = n$.

    On one hand, note that
    \[
        n1 = \underbrace{1+1+1+\cdots+1}_{n \text{ times}} = 0
    \]
    by definition of the characteristic of a ring. Note also that $m1 = \overbrace{1+1+1+\cdots+1}^{m \text{ times}} = 0$ by definition of the order of an element in the group $(R, +)$. By \myref{problem-element-to-power-of-multiple-of-order-is-identity}, this means that $n$ is a multiple of $m$, which thus means $m \leq n$.

    On another hand, for any $r \in R$, we know that
    \begin{align*}
        \underbrace{r + r + r + \cdots + r}_{m \text{ times}} &= r(\underbrace{1+1+1+\cdots+1}_{m \text{ times}})\\
        &= r0 & (\text{since } |1|_+ = m)\\
        &= 0.
    \end{align*}
    The minimality of the characteristic $n$ means that $m$ has to be at least $n$, i.e. $m \geq n$.

    Since $m \leq n$ and $m \geq n$, thus $m = n$.
\end{proof}

\begin{proposition}\label{prop-zero-of-prime-characteristic-if-integral-domain}
    If $D$ is an integral domain, then either $\Char{D} = 0$ or $\Char{D} = p$ where $p$ is a prime.
\end{proposition}
\begin{proof}
    If $\Char{D} = 0$ we are done, so assume that $\Char{D} = n \neq 0$. Furthermore assume $n = ab$ with $1 \leq a,b \leq n$.

    Note that
    \[
        0 = \underbrace{1 + 1 + \cdots + 1}_{n \text{ times}} = (\underbrace{1+1+\cdots+1}_{a \text{ times}})(\underbrace{1+1+\cdots+1}_{b \text{ times}}).
    \]
    As $R$ is an integral domain, this means that either
    \[
        \underbrace{1+1+\cdots+1}_{a \text{ times}} = 0 \text{ or } \underbrace{1+1+\cdots+1}_{b \text{ times}} = 0.
    \]
    Without loss of generality, assume that $\underbrace{1+1+\cdots+1}_{a \text{ times}} = 0$. By minimality of $\Char{R} = n$, this means that $a \geq n$. But we assumed that $a \leq n$, so we must have $a = n$. Therefore $a = n$ and $b = 1$ which is exactly what we need to show that $n$ is a prime.
\end{proof}

\begin{exercise}\label{exercise-trivial-ring-is-not-an-integral-domain}
    Is the trivial ring an integral domain?
\end{exercise}
\begin{exercise}
    What is the characteristic of the ring $\Z_2[\alpha]$ (as defined in \myref{exercise-Zn2[alpha]})?
\end{exercise} 

\newpage

\section{Problems}
\begin{problem}
    Find two zero divisors in the ring
    \[
        \Z_5[i] = \{a + bi \vert a,b\in\Z_5\}.
    \]
\end{problem}

\begin{problem}
    Show that the ring of gaussian integers, $\Z[i]$, is an integral domain.
\end{problem}

\begin{problem}
    Let the integer $n$ be such that $\sqrt{|n|}$ is not an integer. Let the ring
    \[
        R = \Z[\sqrt{n}] = \{a + b\sqrt{n} \vert a,b\in \Z\}.
    \]
    \begin{partquestions}{\alph*}
        \item Show that $R$ is an integral domain.
        \item Is $R$ a field for all integers $n$?
    \end{partquestions}
\end{problem}

\begin{problem}
    Show that
    \[
        R = \left\{\begin{pmatrix}0&0\\0&0\end{pmatrix},\begin{pmatrix}1&0\\0&1\end{pmatrix},\begin{pmatrix}1&1\\1&0\end{pmatrix},\begin{pmatrix}0&1\\1&1\end{pmatrix}\right\}
    \]
    with entries in $\Z_2$ is
    \begin{partquestions}{\roman*}
        \item a subring of $\Mn{2}{\Z_2}$; and
        \item a field.
    \end{partquestions}
\end{problem}

\chapter{Ideals and Quotient Rings}
Ideals are a special type of subring. They help us define the idea of quotient rings in ring theory, similar to how quotient ring are defined in group theory. Ideals form a fundamental part of ring theory, and we will see them appearing repeatedly in the following chapters.

\section{History Behind Ideals}
The term ``ideal'' originated from Ernst Kummer, who was looking to study factorization of numbers. For instance, we may factor $45$ in two ``different'' ways, $45 = 5 \times 9$ and $45 = 3 \times 15$. However, these numbers have not been `factored enough', as we can reduce the factorization down to the primes, $45 = 3^3 \times 5$. This is a unique factorization. However, over algebraic rings, such as in $\Z[\sqrt{-3}]$, this idea of unique factorization down to the irreducible factors fails, since
\[
    4 = 2 \times 2 = (1+\sqrt{-3})(1-\sqrt{-3})
\]
and all the factors here are irreducible.

Kummer's idea was that these irreducible factors were not `reduced enough'; that there were better, ``ideal'' factors for which the ideal factorization of numbers hold. However, such a construction requires that one prove the existence of such ideals, and then show that they are indeed ideals -- too tedious for practical use. 

Richard Dedekind came up with an alternate definition for ``ideals'', defining not the numbers themselves, but the set of numbers that they divide. So instead of talking about the ``ideal number'' 2, we talk about the set of numbers that are divided by 2, namely $\{\dots, -4, -2, 0, 2, 4 \dots\}$, which is what we now call the principal ideal generated by 2. This way, sums and products of ideals become easier to handle, and more results can be created from their use.

The modern formulation of ideals are much weaker than the original ideals proposed by Dedekind; their motivation in modern times stems from the desire to create ``quotient rings'', similar to that of ``quotient groups'' in group theory. Ideals are the ``normal subgroups'' of ring theory. In the coming sections, we look at ideals, before working our way back to Dedekind's original principal ideal definition.

\newpage

\section{What is an Ideal?}
Rings have two operations defined on them (addition and multiplication). Therefore, we need to define which operation for which a coset will apply to.
\begin{definition}
    Given a ring $R$ and a subring $S$ of $R$, the \textbf{coset}\index{coset!for rings} of $S$ in $R$ with \textbf{representative}\index{coset!representative} $r \in R$ is
    \[
        r + S = \{r+s \vert s \in S\}.
    \]
\end{definition}
\begin{remark}
    As $(R,+)$ is an abelian group, thus, as groups, $(S,+)$ is a normal subgroup of $(R,+)$ and so $(R/S,+)$ is well-defined.
\end{remark}

At this point, we only know that $(R/S,+)$ is a subgroup of $(R,+)$. What conditions do we need on $S$ such that $R/S$ is a ring? Well, we need a ``well-defined'' multiplication operation on the cosets, specifically, for any two elements $a$ and $b$ in $R$, we want
\[
    (a+S)(b+s) = ab + S.
\]
We now try and find the condition(s) required for this operation to be well-defined.

Suppose $a+S = c+S$ and $b+S = d+S$ for some other elements $c$ and $d$ in $R$; our goal is to show $ab+S = cd+S$. Now by Coset Equality (\myref{lemma-coset-equality}), statements 1 and 5, we have $a-c \in S$ and $b-d \in S$. Set $a-c = s_1 \in S$ and $b-d = s_2 \in S$, so $a = c+s_1$ and $b = d+s_2$. Hence,
\[
    ab = (c+s_1)(d+s_2) = cd + cs_2 + s_1d + s_1s_2.
\]
Now for $ab + S = cd+S$ we need to have $ab-cd \in S$, again by Coset Equality statements 1 and 5. Therefore, we need to have $ab-cd = cs_2+s_1d+s_1s_2 \in S$. We note that $s_1s_2 \in S$ (as $S$ is a subring), so we just need both $cs_2$ and $s_1d$ to be in $S$ for $R/S$ to be a ring.

We are now ready to present the definition of an ideal.
\begin{definition}
    Let $R$ be a ring and a subset $I$ of $R$. Suppose $(I,+) \leq (R,+)$. Then $I$ is an \textbf{ideal}\index{ideal} (or \textbf{two-sided ideal}\index{ideal!two-sided}) if, for every $r \in R$ and $i \in R$, both $ri$ and $ir$ are in $I$.
\end{definition}
\begin{exercise}\label{exercise-ideal-is-a-subring}
    Let $I$ be an ideal of a ring $R$. Show that $I$ is a subring of $R$.
\end{exercise}
\begin{example}\label{example-nZ-ideal-of-Z}
    We show that the subring
    \[
        n\Z = \{\dots, -2n, -n, 0, n, 2n, \dots\}
    \]
    is an ideal of $\Z$.
    
    Suppose $m \in \Z$ and $an \in n\Z$.  Note that
    \[
        (m)(an) = (ma)n \in n\Z
    \]
    and
    \[
        (an)(m) = a(nm) = a(mn) = (am)n \in n\Z,
    \]
    with swapping of $m$ and $n$ possible since $\Z$ is a commutative ring. Therefore $n\Z$ is an ideal of $\Z$.
\end{example}

\begin{example}
    The subset $\{0\}$ is an ideal of any ring, called the \textbf{trivial ideal}\index{ideal!trivial} (or \textbf{zero ideal}\index{ideal!zero}). An ideal that is not $\{0\}$ is called a \textbf{non-trivial ideal}\index{ideal!non-trivial} (or \textbf{non-zero ideal}\index{ideal!non-zero}).
\end{example}
\begin{example}
    For any ring $R$, the ring itself is an ideal of $R$. An ideal $I$ that is a proper subset of the ring $R$ is a \textbf{proper ideal}\index{ideal!proper} of $R$.
\end{example}

We are almost ready to define the quotient ring; we just need to define the operations within it.
\begin{definition}[Coset Addition]
    The sum of the cosets $r+I$ and $s+I$ is $(r+s)+I$.
\end{definition}
\begin{definition}[Coset Multiplication]
    The product of $r+I$ and $s+I$ is $rs+I$.
\end{definition}

We can now define the quotient ring.
\begin{definition}
    Given a ring $R$ and an ideal $I$ of $R$, the \textbf{quotient ring}\index{quotient ring} of $R$ by $I$ is
    \[
        R/I = \{r + I \vert r \in R\},
    \]
    under coset addition and multiplication.
\end{definition}
\begin{remark}
    Some authors (e.g. \cite[p.~243]{dummit_foote_2004}) may choose to represent $r + I$ as $\overline{r}$. With this notation, addition and multiplication in the quotient ring becomes $\overline{r}+\overline{s} = \overline{r+s}$ and $\overline{r}\,\overline{s} = \overline{rs}$ respectively.
\end{remark}
\begin{remark}
    Like in group theory, $R/I$ is read as ``$R$ mod $I$''.
\end{remark}

\begin{proposition}
    If $R$ is a ring with ideal $I$, then $R/I$ is indeed a ring under coset addition and multiplication.
\end{proposition}
\begin{proof}
    We prove this using the ring axioms.
    \begin{itemize}
        \item \textbf{Addition-Abelian}: We show that $(R/I,+)$ is an abelian group.
        \begin{itemize}
            \item \textbf{Group}: As $(R,+)$ is an abelian group and $(I,+)$ is a normal subgroup of $(R,+)$, therefore $(R/I,+)$ is a well-defined quotient group.
            \item \textbf{Commutative}: All that remains is to show that coset addition is commutative. Let $r+I$ and $s+I$ both be in $R/I$. Then
            \begin{align*}
                (r+I) + (s+I) &= (r+s)+I\\
                &= (s+r) + I & (\text{since } + \text{ is commutative})\\
                &= (s+I) + (r+I)
            \end{align*}
            so coset addition is commutative.
        \end{itemize}
        \item \textbf{Multiplication-Semigroup}: We need to show that $R/I$ is a semigroup under coset multiplication.
        \begin{itemize}
            \item \textbf{Closure}: Let $r+I$ and $s+I$ be both cosets in $R/I$. Note that $rs \in R$ by closure of multiplication. Therefore
            \[
                (r+I)(s+I) = rs+I \in R/I
            \]
            which means that $R/I$ is closed under coset multiplication.

            \item \textbf{Associativity}: Let $r+I$, $s+I$ and $t+I$ be in $R/I$. Then
            \begin{align*}
                (r+I)((s+I)(t+I)) &= (r+I)(st+I)\\
                &= (r(st))+I\\
                &= ((rs)t) + I & (\text{since } \times \text{ is associative})\\
                &= ((rs)+I)(t+I)\\
                &= ((r+I)(s+I))(t+I)
            \end{align*}
            which means that coset multiplication is associative.

        \end{itemize}
        \item \textbf{Distributive}: Lastly, we need to show that coset multiplication distributes over coset addition. For the following, let the cosets $r+I$, $s+I$ and $t+I$ be in $R/I$.
        \begin{itemize}
            \item We first show left distribution.
            \begin{align*}
                (r+I)((s+I)+(t+I)) &= (r+I)((s+t)+I)\\
                &= (r(s+t))+I\\
                &= (rs+rt)+I\\
                &= (rs+I) + (rt+I)\\
                &=(r+I)(s+I) + (r+I)(t+I).
            \end{align*}
            Therefore the left distributive property has been shown.

            \item We now show right distribution.
            \begin{align*}
                ((r+I)+(s+I))(t+I) &= ((r+s)+I)(t+I)\\
                &= ((r+s)t)+I\\
                &= (rt+st)+I\\
                &= (rt+I) + (st+I)\\
                &= (r+I)(t+I) + (s+I)(t+I).
            \end{align*}
            Therefore the right distributive property has been shown.
        \end{itemize}
    \end{itemize}
    Therefore $R/I$ is a ring.
\end{proof}

\begin{example}
    Consider the subring $6\Z$. From \myref{example-nZ-ideal-of-Z} we know that $6\Z$ is an ideal of $\Z$. Thus $\Z/6\Z$ is a quotient ring; it is given by
    \[
        \Z/6\Z = \{0 + 6\Z, 1+6\Z, 2+6\Z, 3+6\Z, 4+6\Z, 5+6\Z\}.
    \]
    
    We explore two possible products in this quotient ring.
    \begin{itemize}
        \item $(3+6\Z)(4+6\Z) = 12+6\Z = 2\times 6 + 6\Z = 0 + 6\Z$. Thus we have found a pair of zero divisors in $\Z/6\Z$, namely $(3+6\Z)$ and $(4+6\Z)$.
        \item $(5+6\Z)^2 = 25+6\Z = (1 + 6\times4) + 6\Z = 1 + 6\Z$.
    \end{itemize}
\end{example}

\begin{exercise}
    Let the sets
    \begin{align*}
        R &= \left\{\begin{pmatrix}x&y\\0&z\end{pmatrix} \vert x,y,z \in \R\right\},\\
        I &= \left\{\begin{pmatrix}x&y\\0&0\end{pmatrix} \vert x,y \in \R\right\}.
    \end{align*}
    It is given that $R$ is a subring of $\Mn{2}{\R}$.
    \begin{partquestions}{\roman*}
        \item Show that $I$ is a subring of $R$.
        \item Show that $I$ is an ideal of $R$.
        \item Simplify
        \[
            \left(\begin{pmatrix}1&2\\0&1\end{pmatrix} + I\right)\left(\begin{pmatrix}1&-2\\0&1\end{pmatrix} + I\right)
        \]
        in $R/I$.
    \end{partquestions}
\end{exercise}

To save us hassle of testing whether a subset of a ring is an ideal, we introduce the test for ideal\index{test for ideal}.
\begin{theorem}[Test for Ideal]\label{thrm-test-for-ideal}
    Let $R$ be a ring and $I$ be a non-empty subset of $R$. Then $I$ is an ideal of $R$ if and only if
    \begin{enumerate}
        \item $x - y \in I$ for all $x$ and $y$ in $I$; and
        \item $ri$ and $ir$ are in $I$ for all $i$ in $I$ and $r$ in $R$.
    \end{enumerate}
\end{theorem}
\begin{proof}
    The forward direction is trivial to prove; if $I$ is an ideal then $I$ is a subring by \myref{exercise-ideal-is-a-subring}, meaning statement 1 holds. Statement 2 holds because that is the definition of an ideal.

    We now work in the reverse direction by assuming the two statements hold. First and foremost we know that $(I,+) \leq (R,+)$ by virtue of statement 1 and by applying the subgroup test (\myref{thrm-subgroup-test}). Now statement 2 tells us that $ri \in I$ for any $r \in R$, in particular we may choose an $r \in I$ so that $ri \in I$. Therefore $I$ is closed under multiplication, and so $I$ is a subring of $R$. Finally, statement 2 is exactly the condition for $I$ to be an ideal.
\end{proof}
\begin{remark}
    Like with the subgroup test (\myref{thrm-subgroup-test}), we can check whether $I$ is non-empty by seeing if the additive identity 0 is in $I$.
\end{remark}

We end this section by noting one result.
\begin{proposition}\label{prop-ideal-contains-unit-iff-ideal-is-whole-ring}
    Let $R$ be a ring with identity and $I$ an ideal of $R$. Then $I$ contains a unit if and only if $I = R$.
\end{proposition}
\begin{proof}
    See \myref{exercise-ideal-contains-unit-iff-ideal-is-whole-ring} (later).
\end{proof}

\begin{exercise}\label{exercise-ideal-contains-unit-iff-ideal-is-whole-ring}
    Let $R$ be a ring with identity 1, and let $I$ be an ideal of $R$.
    \begin{partquestions}{\roman*}
        \item If $1 \in I$, what does this imply about $I$?
        \item Prove $I$ contains a unit if and only if $I = R$.
    \end{partquestions}
\end{exercise}

\section{Ideal Operations}
We first look at the sum of ideals.
\begin{definition}\index{ideal!sum}
    Let $R$ be a ring and let $\ideal{a}$ and $\ideal{b}$ be ideals of $R$. Then the sum of the ideals $\ideal{a}$ and $\ideal{b}$ is
    \[
        \ideal{a} + \ideal{b} = \{a + b \vert a\in\ideal{a},\;b\in\ideal{b}\}.
    \]
\end{definition}
\begin{example}
    Consider the ring $\Z$ and the ideals $I = \{2n \vert n \in \Z\}$ and $J = \{3n \vert n \in \Z\}$. Then their sum is
    \[
        I + J = \{2a + 3b \vert a,b \in \Z\}.
    \]
\end{example}
\begin{proposition}\label{prop-sum-of-ideals-is-ideal}
    Let $R$ be a ring and let $\ideal{a}$ and $\ideal{b}$ be ideals of $R$. Then $\ideal{a} + \ideal{b}$ is an ideal of $R$.
\end{proposition}
\begin{proof}
    We note that since $\ideal{a}$ and $\ideal{b}$ are ideals, they are thus non-empty and so $\ideal{a}+\ideal{b}$ is non-empty.

    Let $a_1, a_2 \in \ideal{a}$ and $b_1, b_2 \in \ideal{b}$. Then we know $a_1 - a_2 \in \ideal{a}$ and $b_1 - b_2 \in \ideal{b}$ by the test for ideal. Therefore $(a_1 - a_2) + (b_1 - b_2) = (a_1 + b_1) - (a_2 + b_2) \in \ideal{a} + \ideal{b}$, satisfying the first statement.

    Now take any $r \in R$ and $a+b \in \ideal{a}+\ideal{b}$.
    \begin{itemize}
        \item $r(a+b) = ra + rb \in \ideal{a}+\ideal{b}$ since $ra \in \ideal{a}$ (because $\ideal{a}$ is an ideal) and $rb \in \ideal{b}$ (because $\ideal{b}$ is an ideal).
        \item $(a+b)r = ar + br \in \ideal{a}+\ideal{b}$ since $ar \in \ideal{a}$ (because $\ideal{a}$ is an ideal) and $br \in \ideal{b}$ (because $\ideal{b}$ is an ideal).
    \end{itemize}
    Therefore $\ideal{a} + \ideal{b}$ is an ideal by the test for ideal (\myref{thrm-test-for-ideal}).
\end{proof}

We now look at the product of ideals.
\begin{definition}\index{ideal!product}
    Let $R$ be a ring and let $\ideal{a}$ and $\ideal{b}$ be ideals of $R$. Then the product of the ideals $\ideal{a}$ and $\ideal{b}$ is
    \[
        \ideal{ab} = \{a_1b_1 + a_2b_2 + \cdots + a_nb_n \vert a_i\in\ideal{a},\;b_i\in\ideal{b}\}.
    \]
\end{definition}

\begin{example}
    Consider the ring $\Z$ and the ideals $I = \{2n \vert n \in \Z\}$ and $J = \{3n \vert n \in \Z\}$. Then their product is
    \begin{align*}
        IJ &= \{(2a_1)(3b_1) + (2a_2)(3b_2) + \cdots + (2a_n)(3b_n) \vert a_i,b_i \in \Z\}\\
        &= \{6(a_1b_1 + a_2b_2 + \cdots + a_nb_n) \vert a_i,b_i \in \Z\}\\
        &= \{6k \vert k \in \Z\}.
    \end{align*}
\end{example}

\begin{proposition}\label{prop-product-of-ideals-is-ideal}
    Let $R$ be a ring and let $\ideal{a}$ and $\ideal{b}$ be ideals of $R$. Then $\ideal{ab}$ is an ideal of $R$.
\end{proposition}
\begin{proof}
    Since $\ideal{a}$ and $\ideal{b}$ are non-empty as they are ideals, therefore $\ideal{ab}$ is non-empty.

    Let $a_1,a_2 \in \ideal{a}$ and $b_1,b_2 \in \ideal{b}$. Since $\ideal{a}$ is an ideal and is hence a subring, so $-a_2 \in \ideal{a}$. Thus $a_1b_1, (-a_2)(b_2) \in \ideal{ab}$. Clearly $a_1b_1 + (-a_2)(b_2) = a_1b_1 - a_2b_2 \in \ideal{ab}$ so this satisfies the first statement.

    Now take any $r \in R$ and let $a_1b_1 + \cdots + a_nb_n \in \ideal{ab}$.
    \begin{itemize}
        \item $r(a_1b_1 + \cdots + a_nb_n) = (ra_1)b_1 + \cdots + (ra_n)b_n$. Note $ra_i \in \ideal{a}$ since $\ideal{a}$ is an ideal, and $b_i \in \ideal{b}$. Hence $(ra_1)b_1 + \cdots + (ra_n)b_n \in \ideal{ab}$.
        \item $(a_1b_1 + \cdots + a_nb_n)r = a_1(b_1r) + \cdots + a_n(b_nr)$. Note $b_ir \in \ideal{b}$ since $\ideal{b}$ is an ideal, and $a_i \in \ideal{a}$. Hence $a_1(b_1r) + \cdots + a_n(b_nr) \in \ideal{ab}$.
    \end{itemize}

    Therefore $\ideal{ab}$ is an ideal by the test for ideal (\myref{thrm-test-for-ideal}).
\end{proof}

\begin{exercise}
    Let $R$ be a ring and let $\ideal{a}$ and $\ideal{b}$ be ideals of $R$. Prove that $\ideal{a} \cap \ideal{b}$ is an ideal of $R$.
\end{exercise}

\section{Principal Ideals}
We return to the original definition for ideals by Dedekind, which is what we now call principal ideals. They can be thought of as ideals that are `generated' by one element from the ring.
\begin{definition}
    Let $R$ be a commutative ring with identity and let $a$ be an element from $R$. Then the \textbf{principal ideal generated by $a$}\index{ideal!principal} is
    \[
        \princ{a} = aR = \{ar \vert r \in R\}.
    \]
\end{definition}
\begin{remark}
    Most authors (e.g. \cite[p.~123, Definition III.2.4]{hungerford_1980}, \cite[\S 158]{clark_1984}, \cite[p.~251]{dummit_foote_2004}) choose to denote the principal ideal generated by $a$ by $(a)$. However, to avoid ambiguity with normal parentheses, we follow \cite[p.~250, Example 3]{gallian_2016} and choose to denote it by $\princ{a}$ instead. Although one may be concerned with this being confused with the cyclic (sub)group generated by $a$, the meaning of this notation should be clear from context.
\end{remark}
\begin{remark}[see {\cite[p.~251]{dummit_foote_2004}}]
    If $R$ is a non-commutative ring, or if $R$ does not contain an identity, then the situation becomes more complicated. In particular,
    \begin{align*}
        \princ{a} &= \text{Smallest ideal of } R \text{ containing }a\\
        &= \bigcap_{\substack{\text{all ideals }I \\ \text{with } a \in I}}I
    \end{align*}
    
    As such, we restrict the discussion of principal ideals to commutative rings with identity.
\end{remark}

\begin{proposition}
    All principal ideals are ideals.
\end{proposition}
\begin{proof}
    See \myref{exercise-principal-ideal-is-ideal} (later).
\end{proof}

\begin{proposition}\label{prop-principal-ideals-equal-iff-associates}
    Let $D$ be an integral domain, and let $a,b\in D$. Then $\princ{a} = \princ{b}$ if and only if $a = bu$ for some unit $u$ in $D$.
\end{proposition}
\begin{proof}
    See \myref{problem-principal-ideals-equal-iff-associates} (later).
\end{proof}

\begin{example}
    The ideal $n\Z$ is a principal ideal of $\Z$ since
    \[
        n\Z = \{\dots, -2n, -n, 0, n, 2n, \dots\} = \princ{n}.
    \]
\end{example}
\begin{remark}
    In the context of $\Z$, we usually write the principal ideal $\princ{n}$ as $n\Z$.
\end{remark}

\begin{exercise}\label{exercise-principal-ideal-is-ideal}
    Show that all principal ideals are ideals.
\end{exercise}

\begin{exercise}\label{exercise-trivial-ideal-and-whole-ring-are-principal-ideals}
    Let $R$ be a commutative ring with identity. Show that the following ideals are principal in $R$.
    \begin{partquestions}{\alph*}
        \item The trivial ideal.
        \item The ring itself.
    \end{partquestions}
\end{exercise}

We look at two related definitions to the principal ideal before we look at an interesting proposition.
\begin{definition}
    A commutative ring where every ideal is principal is called a \textbf{principal ideal ring}\index{principal ideal ring}.
\end{definition}
\begin{definition}
    A principal ideal ring that is an integral domain is called a \textbf{principal ideal domain}\index{principal ideal domain}, or \textbf{PID}\index{PID}.
\end{definition}

We look at one example of a PID.
\begin{proposition}\label{prop-Z-is-PID}
    $\Z$ is a PID.
\end{proposition}
\begin{proof}
    We note that $\Z$ is an integral domain, so all that remains to be proven is that every ideal of $\Z$ is principal.

    Let $I$ be an ideal of $\Z$, and suppose $I$ is not the trivial ideal. This means that $I$ must contain both positive and negative numbers.
    
    Set $n = \min(I \cap \mathbb{N})$, i.e. $n$ is the smallest positive integer that is in $I$. Observe that since $n \in I$ we have $mn \in I$ for all $m \in \Z$. Therefore
    \[
        \princ{n} = \{mn \vert m \in \Z\} = n\Z \subseteq I.
    \]

    Now we want to show that $I \subseteq \princ{n}$. Suppose $a \in I$. By Euclid's division lemma (\myref{lemma-euclid-division}), we see that
    \[
        a = nq + r \text{ with } q,r\in\Z \text{ and } 0 \leq r < n,
    \]
    meaning that $r = a - nq \in I$. Note that $a$ and $n$ are in $I$, so $nq \in I$ and therefore $a - nq = r \in I$. If $r$ is positive, then we have found a smaller positive integer than $n$ in $I$, a contradiction since $n$ is the smallest positive integer in $I$. Hence $r = 0$, which means $a = nq \in \princ{n}$. Therefore $I \subseteq \princ{n} = n\Z$.

    As $n\Z \subseteq I$ and $I \subseteq n\Z$, therefore $I = n\Z$, meaning that any arbitrary ideal in $\Z$ is principal. Hence $\Z$ is a PID.
\end{proof}

\section{Prime and Maximal Ideals}
Let's first look at the definition of a prime ideal.
\begin{definition}
    Let $R$ be a commutative ring with identity. An proper ideal $P$ of $R$ is called a \textbf{prime ideal}\index{ideal!prime}\index{prime!ideal} if whenever $ab \in P$ we have $a \in P$ or $b \in P$.
\end{definition}
This definition may seem weird, but is completely natural when we consider the properties of primes in the positive integers. Recall that Euclid's Lemma (\myref{corollary-euclid}) tells us that if a prime $p$ divides $ab$, then either $p$ divides $a$ or $p$ divides $b$. Similarly, if the product $ab$ belongs within the prime ideal $P$, then either $a$ belongs in $P$ or $b$ belongs in $P$. However, prime ideals does not necessarily imply unique `factorization' like what occurs in the integers via the Fundamental Theorem of Arithmetic (\myref{thrm-fundamental-theorem-of-arithmetic}); we will explore the uniqueness criteria in a later chapter.

We look at the connection between this definition of a prime ideal and primes in the integers.
\begin{proposition}\label{prop-ideals-of-Z}
    The prime ideals of $\Z$ are the trivial ideal and $p\Z$, where $p$ is a prime number.
\end{proposition}
\begin{proof}
    First let's assume the ideal in question is the trivial ideal. Suppose $ab \in \{0\}$, meaning $ab = 0$. As $\Z$ is an integral domain, this means either $a = 0$ or $b = 0$, which therefore means either $a \in \{0\}$ or $b \in \{0\}$. Hence the trivial ideal is a prime ideal in $\Z$.

    Now suppose we have a non-trivial prime ideal $P$ of $\Z$. As $\Z$ is a PID, we may write $P = n\Z$ where $n \geq 2$ (note that if $n = 1$ we will get $P = \Z$). Furthermore write $n = ab$ where $1 \leq a,b \leq n$. Since $ab = n \in P$, therefore $a \in P$ or $b \in P$. Without loss of generality assume $a \in P$. We note $P = n\Z = \{\dots, -2n, -n, 0, n, 2n, \dots\}$ and $1 \leq a \leq n$, so we must have $a = n$. Therefore $a = n$ and $b = 1$, which shows that $n$ is prime. Hence, the prime ideal $P = n\Z$ where $n$ is prime.
\end{proof}
\begin{exercise}
    Is the principal ideal $\princ{2} = \{0, 2, 4, 6\}$ prime in $\Z_8$?
\end{exercise}

We now look at the definition of a maximal ideal.
\begin{definition}
    Let $R$ be a commutative ring with identity. An ideal $M \subset R$ is called a \textbf{maximal ideal}\index{ideal!maximal} if whenever an ideal $I$ is such that $M \subseteq I \subseteq R$ we have $I = M$ or $I = R$.
\end{definition}
\begin{remark}
    This means that the only ideal that properly contains a maximal ideal is the entire ring.
\end{remark}
\begin{remark}
    To show maximality of an ideal, we usually assume that $I \neq M$ and show that $I$ has to be equal to $R$.
\end{remark}
\begin{example}
    We show that $M = \princ{2} = \{0, 2, 4, \dots, 32, 34\}$ is a maximal ideal of $\Z_{36}$. Suppose we have an ideal $I$ of $\Z_{36}$ such that $M \subset I \subseteq \Z_{36}$. We show that $I = \Z_{36}$.

    Take an $n \in I \setminus M$. Then Euclid's division lemma (\myref{lemma-euclid-division}) tells us that we may write
    \[
        n = 2q + r \text{ with } 0 \leq r < 2
    \]
    Now since $n \notin M$, therefore $n$ is not a multiple of 2. Thus there must be a remainder, i.e. $1 \leq r < 2$, meaning that $r = 1$. Furthermore, one sees that $r = 1 = n - 2q$, and because $n \in I$ and $2q \in I$ (since $2q \in M$ and $M \subset I$), therefore $1 \in I$. Hence, by \myref{prop-ideal-contains-unit-iff-ideal-is-whole-ring}, $I = \Z_{36}$, which means that $\princ{2}$ is a maximal ideal of $\Z_{36}$.
\end{example}
\begin{exercise}
    What conditions must be placed on the positive integer $n$ so that $n\Z$ is a maximal ideal in $\Z$?
\end{exercise}

We note that there are much easier ways to determine whether an ideal is prime, maximal, or both. We state two important results here.

\begin{theorem}\label{thrm-prime-ideal-iff-quotient-ring-is-integral-domain}
    Let $R$ be a commutative ring with identity, and let $P$ be an ideal of $R$. Then $P$ is prime if and only if $R/P$ is an integral domain.
\end{theorem}
\begin{proof}
    We first prove the forward direction. Suppose $P$ is prime and $a+P, b+P \in R/P$ such that $(a+P)(b+P) = 0+P$. This means that $ab + P = 0 + P$. By Coset Equality (\myref{lemma-coset-equality}), this therefore means that $ab \in P$. Now as $P$ is prime, this thus means that either $a \in P$ or $b \in P$. Hence, $a + P = 0 + P$ or $b + P = 0 + P$. So we have shown that if two elements in $R/P$ multiply to zero, then one of the elements must be zero, meaning that $R/P$ has no zero divisors.
    
    Now clearly $(a+P)(b+P) = ab + P = ba + P = (b+P)(a+P)$ since $R$ is commutative, so $R/P$ is commutative. Furthermore one sees that $1 + P$ is the identity in $R/P$. Therefore, $R/P$ is an integral domain.

    Now we prove the reverse direction. Suppose $R/P$ is an integral domain. Take $a,b \in R$ such that $ab \in P$. This means that $ab + P = (a+P)(b+P) = 0 + P$. Therefore $a+P = 0 + P$ or $b + P = 0 + P$ since $R/P$ is an integral domain with no zero divisors. Hence $a \in P$ or $b \in P$ by Coset Equality.
    
    Now if $P = R$ then $R/P = \{0 + P\}$ which is the trivial quotient ring. But by \myref{exercise-trivial-ring-is-not-an-integral-domain} it is not an integral domain. Therefore $P \neq R$, meaning that $P$ is a prime ideal.
\end{proof}
\begin{exercise}
    Is $\princ{3 - i}$ a prime ideal in the Gaussian integers?
\end{exercise}

\begin{theorem}\label{thrm-maximal-ideal-iff-quotient-ring-is-field}
    Let $R$ be a commutative ring with identity, and let $M$ be an ideal of $R$. Then $M$ is maximal if and only if $R/M$ is a field.
\end{theorem}
\begin{proof}
    We first work in the forward direction. Suppose $M$ is a maximal ideal, and take $a + M \in R/M$ such that $a \neq 0$ and $a \notin M$ (which means $a + M$ is non-zero). Observe
    \[
        M \subset \princ{a} + M \subseteq R,
    \]
    with strict subset achieved because $a \notin M$, and the sum of ideals is an ideal. As $M$ is maximal, this means that $\princ{a} + M = R$. Therefore $1 \in \princ{a} + M$; write $1 = ar + m$ for some $r \in R$ and $m \in M$. Note $1-ar = m \in M$. Coset Equality (\myref{lemma-coset-equality}) therefore tells us that $ar + M = 1 + M$. Note $ar + M = (a+M)(r+M) = 1 + M$, so $(a+M)^{-1} = r+M$, meaning that any non-zero element in $R/M$ has an inverse. A similar argument to what is shown in \myref{thrm-prime-ideal-iff-quotient-ring-is-integral-domain} shows that $R/M$ is commutative with identity $1 + M$, so $R/M$ is a field.

    Now we work in the reverse direction; assume that $R/M$ is a field. Suppose $I$ is an ideal such that $M \subset I \subseteq R$. We want to show that $I = R$.

    Take $a \in I \setminus M$ with $a \neq 0$, so $a + M \in R/M$ is non-zero. As $R/M$ is a field, therefore there exists a $b + M \in R/M$ such that $(a+M)(b+M) = 1 + M$, meaning $ab+M = 1+M$. Then Coset Equality tells us that $ab - 1 \in M \subset I$. Since $ab \in I$ and $ab - 1 \in I$, the only way for this to happen is if $1 \in I$. By \myref{prop-ideal-contains-unit-iff-ideal-is-whole-ring} this means $I = R$.

    Finally, if $M = R$, then $R/M = \{0 + M\}$, the trivial ring. But in the trivial ring, the additive and multiplicative inverses is the same element, so $R/M$ is not a field by definition. Hence $M \neq R$, so $M \subset R$, meaning $M$ is maximal.
\end{proof}

\begin{corollary}\label{corollary-all-maximal-ideals-are-prime-ideals}
    All maximal ideals are prime ideals.
\end{corollary}
\begin{proof}
    If $M$ is a maximal ideal in a commutative ring with identity $R$, then $R/M$ is a field by \myref{thrm-maximal-ideal-iff-quotient-ring-is-field}. Note that any field is an integral domain by \myref{prop-field-is-integral-domain}. Therefore $R/M$ is an integral domain, meaning $M$ is prime by \myref{thrm-prime-ideal-iff-quotient-ring-is-integral-domain}.
\end{proof}

\begin{corollary}\label{corollary-prime-ideal-is-maximal-in-finite-commutative-ring-with-identity}
    All prime ideals in a finite commutative ring with identity are maximal.
\end{corollary}
\begin{proof}
    See \myref{exercise-prime-ideal-is-maximal-in-finite-commutative-ring-with-identity} (later).
\end{proof}
\begin{exercise}\label{exercise-prime-ideal-is-maximal-in-finite-commutative-ring-with-identity}
    Let $R$ be a finite commutative ring with identity, and let $P$ be a prime ideal in $R$. Show that $P$ is maximal in $R$.
\end{exercise}

\section{The Annihilator and Radical}
To end this chapter, we look at two constructs relating to a ring.
\begin{definition}
    Let $R$ be a commutative ring and $A$ a non-empty subset of $R$. The \textbf{annihilator}\index{annihilator} of $A$ is
    \[
        \Ann{R}{A} = \{r \in R \vert ra = 0 \text{ for all } a \in A\}.
    \]
\end{definition}
\begin{example}
    $\Ann{\Z}{\Z} = \{0\}$.
\end{example}
\begin{proposition}
    Let $R$ be a commutative ring and $A$ a non-empty subset of $R$. Then $\Ann{R}{A}$ is an ideal of $R$.
\end{proposition}
\begin{proof}
    See \myref{exercise-annihilator-is-an-ideal} (later).
\end{proof}

\begin{definition}
    Let $R$ be a commutative ring and $I$ an ideal of $R$. The \textbf{radical}\index{radical} of $I$ is
    \[
        \sqrt I = \{r \in R \vert r^n \in I \text{ for some } n \in \mathbb{N}\}.
    \]
\end{definition}
\begin{definition}
    The radical of the trivial ideal is called the \textbf{nilradical}\index{nilradical} and is given by
    \[
        \Nilr{R} = \sqrt{\{0\}} = \{r \in R \vert r^n = 0 \text{ for some } n \in \mathbb{N}\}.
    \]
    That is, the nilradical of a commutative ring $R$ is the set of all nilpotents in $R$.
\end{definition}
\begin{example}
    We find the nilradical of the ring $\Z_{12}$.
    \begin{align*}
        \Nilr{\Z_{12}} &= \{r \in \Z_{12} \vert r^n = 0 \text{ for some } n \in \mathbb{N}\}\\
        &= \{0, 6\}.
    \end{align*}
\end{example}
\begin{example}
    In the integers, we show that $\sqrt{4\Z} = 2\Z$.

    Suppose $r \in \sqrt{4\Z}$, so $r^n = 4m$ for some $m \in \Z$ and $n \in \mathbb{N}$. Clearly $r^n = 4m = 2(2m) \in 2\Z$, so $r \in 2\Z$. Thus this means that $\sqrt{4\Z} \subseteq 2\Z$.

    On another hand, suppose $r \in 2\Z$, meaning $r = 2m$ for some $m \in \Z$. Note that $r^2 = (2m)^2 = 4m^2 \in 4\Z$, so $r \in \sqrt{4\Z}$. Thus $2\Z \subseteq \sqrt{4\Z}$.

    Therefore $\sqrt{4\Z} = 2\Z$.
\end{example}

\newpage

\begin{proposition}
    Let $R$ be a commutative ring and $I$ be an ideal of $R$. Then $\sqrt{I}$ is an ideal of $R$.
\end{proposition}
\begin{proof}
    We again consider the test for ideal (\myref{thrm-test-for-ideal}) to prove this. Note that $0 \in \sqrt{I}$ since $0^1 = 0 \in I$, so $\sqrt{I}$ is non-empty.

    First let $r, s \in \sqrt{I}$, meaning $r^m \in I$ and $s^n \in I$ for some positive integers $m$ and $n$. Without loss of generality, assume $m \geq n$. Consider $(r-s)^{mn}$. The Binomial Theorem (\myref{thrm-binomial}) tells us that
    \[
        (r-s)^{mn} = \sum_{k=0}^{mn}(-1)^k{mn \choose k}r^{mn-k}s^k.
    \]
    Observe that at any one point, either $mn - k \geq m$ or $k \geq m$, so at any point either $r^{mn-k} \in I$ or $s^k \in I$, meaning that $(-1)^k{mn \choose k}r^{mn-k} \in I$. Therefore, $(r-s)^{mn} \in I$, which in turn means $r-s \in \sqrt{I}$.

    Next suppose $x \in R$ and $r \in \sqrt{I}$, meaning $r^n \in I$ for some positive integer $n$. Note that $(rx)^n = (xr)^n = x^nr^n \in I$ since $R$ is commutative and $r^n \in I$. Therefore $rx, xr \in \sqrt{I}$.

    By the test for ideal, we have $\sqrt{I}$ is an ideal of $R$.
\end{proof}

\begin{exercise}\label{exercise-annihilator-is-an-ideal}
    Let $R$ be a commutative ring and $A$ a non-empty subset of $R$. Show that $\Ann{R}{A}$ is an ideal of $R$.
\end{exercise}

\newpage

\section{Problems}
\begin{problem}
    Find $\Ann{\Z_{36}}{\{15\}}$.
\end{problem}

\begin{problem}
    Let $S = \{a + 2bi \vert a, b \in \Z\}$. Show that $S$ is a subring of $\Z[i]$ but not an ideal of $\Z[i]$.
\end{problem}

\begin{problem}
    Consider
    \[
        I = \left\{\begin{pmatrix}2a&2b\\2c&2d\end{pmatrix} \vert a,b,c,d \in \Z\right\}.
    \]
    Show that $I$ is an ideal of $\Mn{2}{\Z}$.
\end{problem}

\begin{problem}\label{problem-ring-is-field-iff-no-proper-ideals}
    Let $R$ be a commutative ring with identity and $I$ be an ideal of $R$.
    \begin{partquestions}{\alph*}
        \item Prove that if $R$ is a field then $I$ is either the trivial ring or $R$ (i.e., $R$ has no proper ideals). Hence prove that any field is a PID.
        \item Prove that if $R$ has no proper ideals, then $R$ is a field.
    \end{partquestions}
\end{problem}

\begin{problem}
    Let $R$ be a commutative ring and let $I$ and $J$ be ideals in $R$. Prove the following statements.
    \begin{partquestions}{\alph*}
        \item $\sqrt{\sqrt{I}} = \sqrt{I}$
        \item $\sqrt{I \cap J} = \sqrt{I} \cap \sqrt{J}$
    \end{partquestions}
\end{problem}

\begin{problem}
    Let $m$ and $n$ be positive integers, and let $d = \gcd(m,n)$ and $l = \lcm(m,n)$. Prove the following.
    \begin{partquestions}{\alph*}
        \item $m\Z \cap n\Z = l\Z$
        \item $m\Z + n\Z = d\Z$
    \end{partquestions}
\end{problem}

\begin{problem}
    Let $R$ be a commutative ring. Prove that $R / \Nilr{R}$ has no non-zero nilpotents.
\end{problem}

\begin{problem}
    Prove that every non-trivial prime ideal is a maximal ideal in a PID.
\end{problem}

\begin{problem}\label{problem-principal-ideals-equal-iff-associates}
    Prove \myref{prop-principal-ideals-equal-iff-associates}.
\end{problem}

\chapter{Ring Homomorphisms and Isomorphisms}
Just like with groups, rings too have homomorphisms and isomorphisms, although they are defined slightly differently than in groups. Like how group homomorphisms preserve some structure between the two groups, ring homomorphisms and isomorphisms also preserve structure between rings.

\section{Ring Homomorphisms, Endomorphisms, and Isomorphisms}
\begin{definition}
    Let $(R_1, +, \cdot)$ and $(R_2, \oplus, \otimes)$ be rings. A function $\phi: R_1 \to R_2$ is a \textbf{ring homomorphism}\index{ring homomorphism} if for all $a, b \in R_1$,
    \begin{align*}
        \phi(a+b) &= \phi(a) \oplus \phi(b), \text{ and}\\
        \phi(a\cdot b) &= \phi(a)\otimes\phi(b).
    \end{align*}
\end{definition}
\begin{remark}
    Like with group homomorphisms, we usually suppress the multiplication operation and use ``$+$'' for both addition operations. That is, the ring homomorphism conditions become
    \begin{align*}
        \phi(a+b) &= \phi(a) + \phi(b) \text{ and}\\
        \phi(ab) &= \phi(a)\phi(b).
    \end{align*}
\end{remark}

\begin{example}
    We show that the map $\phi: \Z \to \Z/n\Z, x \mapsto x + n\Z$ is a ring homomorphism.
     
    \begin{proof}
        Let $a, b \in \Z$. Note
        \begin{align*}
            \phi(a+b) &= (a+b) + n\Z\\
            &= (a + n\Z) + (b + n\Z) & (\text{Definition of coset addition})\\
            &=\phi(a)+\phi(b)
        \end{align*}
        and
        \begin{align*}
            \phi(ab) &= (ab) + n\Z\\
            &= (a + n\Z)(b + n\Z) & (\text{Definition of coset multiplication})\\
            &= \phi(a)\phi(b)
        \end{align*}
        so $\phi$ is a homomorphism.
    \end{proof}
\end{example}

\begin{example}
    Consider
    \[
        R = \left\{\begin{pmatrix}a&b\\0&c\end{pmatrix}\vert a,b,c\in\Z\right\}.
    \]
    Let $\phi: R \to \Z^2, \begin{pmatrix}a&b\\0&c\end{pmatrix} \mapsto (a,c)$. We show that $\phi$ is a ring homomorphism.

    \begin{proof}
        We see that
        \begin{align*}
            \phi\left(\begin{pmatrix}a&b\\0&c\end{pmatrix} + \begin{pmatrix}x&y\\0&z\end{pmatrix}\right) &= \phi\left(\begin{pmatrix}a+x&b+y\\0&c+z\end{pmatrix}\right)\\
            &= (a+x,c+z)\\
            &= (a,c) + (x,z)\\
            &= \phi\left(\begin{pmatrix}a&b\\0&c\end{pmatrix}\right) + \phi\left(\begin{pmatrix}x&y\\0&z\end{pmatrix}\right)
        \end{align*}
        and also
        \begin{align*}
            \phi\left(\begin{pmatrix}a&b\\0&c\end{pmatrix}\begin{pmatrix}x&y\\0&z\end{pmatrix}\right) &= \phi\left(\begin{pmatrix}ax&ay+bz\\0&cz\end{pmatrix}\right)\\
            &= (ax, cz)\\
            &= (a,c)(x,z)\\
            &= \phi\left(\begin{pmatrix}a&b\\0&c\end{pmatrix}\right)\phi\left(\begin{pmatrix}x&y\\0&z\end{pmatrix}\right)
        \end{align*}
        so $\phi$ is a homomorphism.
    \end{proof}
\end{example}
\begin{exercise}
    Let $\phi: \Mn{2}{\Z} \to \Z$ be such that
    \[
        \phi\left(\begin{pmatrix}a&b\\c&d\end{pmatrix}\right) = a+d.
    \]
    Is $\phi$ a ring homomorphism?
\end{exercise}
\begin{exercise}
    Let $R$ and $S$ be rings with additive identities $0_R$ and $0_S$ respectively. Show that the \textbf{trivial homomorphism}\index{trivial homomorphism} $\phi: R \to S, r \mapsto 0_S$ is, indeed, a ring homomorphism.
\end{exercise}

An endomorphism is a specific type of homomorphism.
\begin{definition}
    A \textbf{ring endomorphism}\index{ring endomorphism} of a ring $R$ is a homomorphism $\phi: R \to R$.
\end{definition}
\begin{example}
    Let $R$ be a commutative ring with prime characteristic $p$. The \textbf{Frobenius endomorphism}\index{Frobenius endomorphism} $\phi: R \to R$ is such that $\phi(r) = r^p$. We show that $\phi$ is a ring endomorphism.
    
    \begin{proof}
        Note that
        \begin{align*}
            \phi(a+b) &= (a+b)^p\\
            &= a^p + pa^{p-1}b + {p \choose 2}a^{p-2}b^2 + \cdots + pab^{p-1} + b^p.
        \end{align*}
        Note that the binomial coefficients ${p \choose k}$ where $1 \leq k \leq p-1$ are all multiples of $p$ (\myref{prop-binomial-coefficient-multiple-of-p}). As the characteristic of the ring $R$ is $p$, thus $px = 0$ for any $x \in R$. Therefore,
        \begin{align*}
            \phi(a+b) = &a^p + pa^{p-1}b + {p \choose 2}a^{p-2}b^2 + \cdots + pab^{p-1} + b^p\\
            &= a^p + 0 + 0 + \cdots + 0 + b^p\\
            &= a^p + b^p\\
            &=\phi(a) + \phi(b)
        \end{align*}

        Also,
        \[
            \phi(ab) = (ab)^p = a^pb^p = \phi(a)\phi(b).
        \]
        Therefore $\phi$ is a ring endomorphism.
    \end{proof}
\end{example}

\begin{exercise}
    Let $R$ be a ring. Show that the \textbf{identity homomorphism}\index{identity homomorphism} $\id: R \to R, r \mapsto r$ is a ring endomorphism.
\end{exercise}

Similar to group isomorphisms, rings too have an analogous `isomorphism' definition.
\begin{definition}
    A \textbf{ring isomorphism}\index{ring isomorphism} is a bijective ring homomorphism.
\end{definition}
Similar to groups, if two rings $R$ and $R'$ are isomorphic, then we write $R \cong R'$.

\begin{example}
    We show that $\Z_n \cong \Z/n\Z$.

    \begin{proof}
        Consider the map $\phi:\Z_n \to \Z/n\Z$ where $m \mapsto m + n\Z$. We show that $\phi$ is an isomorphism.
        \begin{itemize}
            \item \textbf{Homomorphism}:
            \[
                \phi(a+b) = (a+b) + n\Z = (a + n\Z) + (b + n\Z) = \phi(a) + \phi(b)
            \]
            and
            \[
                \phi(ab) = (ab) + n\Z = (a+n\Z)(b+n\Z) = \phi(a)\phi(b)
            \]

            \item \textbf{Injective}: Suppose $a, b \in \Z_n$ such that $\phi(a) = \phi(b)$. This means $a + n\Z = b + n\Z$. Now note that $0 \leq a,b < n$ so we have $a = b$.
            
            \item \textbf{Surjective}: Suppose $m + n\Z \in \Z/n\Z$. Applying Euclid's division lemma (\myref{lemma-euclid-division}) on $m$ we have
            \[
                m = nq + r
            \]
            with $0 \leq r < n$. One sees that
            \begin{align*}
                \phi(r) &= r + n\Z\\
                &= r + (nq + n\Z)\\
                &= (r + nq) + n\Z\\
                &= m + n\Z
            \end{align*}
            so $m + n\Z$ has a pre-image of $r$ in $\Z_n$.
        \end{itemize}
        Since $\phi$ is a bijective ring homomorphism, this $\phi$ is an isomorphism, meaning $\Z_n \cong \Z/n\Z$ as rings.
    \end{proof}
\end{example}
\begin{example}
    We show that $\Z \not\cong 2\Z$ as rings.

    \begin{proof}
        Suppose $\phi: \Z \to 2\Z$ is a ring isomorphism.

        Set $a = \phi(1) = 2\Z$. Note that
        \[
            a = \phi(1) = \phi(1\times1) = (\phi(1))^2 = a^2
        \]
        so $a^2 = a$, which means $a = 0$ or $a = 1$. But as $a \in 2\Z$, thus $a \neq 1$ which means $a = 0$.

        Now notice for any $n \in \Z$ we have
        \begin{align*}
            \phi(n) &= \phi(n1)\\
            &= \phi(n)\phi(1)\\
            &= \phi(n) \times 0\\
            &= 0.
        \end{align*}
        Thus one sees that $\phi(0) = \phi(1) = 0$ which means $\phi$ is not injective, a contradiction.
    \end{proof}
\end{example}
\begin{exercise}\label{exercise-identity-homomorphism-is-an-isomorphism}
    Show that the identity homomorphism is an isomorphism.
\end{exercise}

\section{Properties of Ring Homomorphisms}
For the following, let $R_1$ and $R_2$ be rings with additive identities $0_1$ and $0_2$ respectively. Also let $\phi: R_1 \to R_2$ be a ring homomorphism.

\begin{proposition}\label{prop-ring-image-of-additive-identity-is-additive-identity}
    $\phi(0_1) = 0_2$.
\end{proposition}
\begin{proof}
    See \myref{exercise-ring-image-of-identity-is-identity} (later).
\end{proof}

\begin{proposition}
    If $R_1$ and $R_2$ are division rings, then $\phi(1_1) = 1_2$ where $1_1$ and $1_2$ are the multiplicative identities of $R_1$ and $R_2$ respectively.
\end{proposition}
\begin{proof}
    See \myref{exercise-ring-image-of-identity-is-identity} (later).
\end{proof}

\begin{proposition}
    $\phi(-x) = -\phi(x)$ for all $x \in R_1$.
\end{proposition}
\begin{proof}
    See \myref{exercise-ring-image-of-inverse-is-inverse} (later).
\end{proof}

\begin{proposition}
    If $R_1$ and $R_2$ are division rings, then $\phi(x^{-1}) = (\phi(x))^{-1}$ for all $x \in R_1$. 
\end{proposition}
\begin{proof}
    See \myref{exercise-ring-image-of-inverse-is-inverse} (later).
\end{proof}

\begin{proposition}\label{prop-homomorphism-on-subring-is-subring}
    If $S$ is a subring of $R_1$, then
    \[
        \phi(S) = \{\phi(s) | s \in S\}
    \]
    is a subring of $R_2$.
\end{proposition}
\begin{proof}
    Let $S$ be a subring of $R_1$. Take $a, b \in \phi(S)$, which means that there exist $s_a$ and $s_b$ such that $\phi(s_a) = a$ and $\phi(s_b) = b$.
    \begin{itemize}
        \item We show that $(\phi(S), +) \leq (R_2, +)$.
        \begin{itemize}
            \item $\phi(S) \neq \emptyset$ since $\phi(0_1) = 0_2 \in \phi(S)$.
            \item $a - b = \phi(s_a) - \phi(s_b) = \phi(s_a-s_b) \in \phi(S)$.
        \end{itemize}

        \item Now we show $ab \in \phi(S)$.
        \[
            ab = \phi(s_a)\phi(s_b) = \phi(s_as_b) \in \phi(S).
        \]
    \end{itemize}
    Therefore $\phi(S)$ is a subring of $R_2$.
\end{proof}

\begin{proposition}
    If $\phi$ is surjective and $I$ is an ideal of $R_1$, then $\phi(I)$ is an ideal of $R_2$.
\end{proposition}
\begin{proof}
    From previous proposition $\phi(I)$ is a subring of $R_2$. We just need to show that $\phi(I)$ is an ideal of $R_2$.

    Take $a \in \phi(I)$ and $r_2 \in R_2$. As $\phi$ is surjective, we can find a $r_1 \in R_1$ such that $\phi(r_1) = r_2$. Also, let $a = \phi(i)$ for an $i \in I$.

    Note
    \begin{align*}
        ar_2 = \phi(i)\phi(r_1) = \phi(\underbrace{ir_1}_{\text{In }I}) \in \phi(I)\\
        r_2a = \phi(r_1)\phi(i) = \phi(\underbrace{r_1i}_{\text{In }I}) \in \phi(I)
    \end{align*}
    so $\phi(I)$ is an ideal of $R_2$.
\end{proof}

\begin{proposition}\label{prop-inverse-homomorphism-on-ideal-is-ideal}
    Let $J$ be an ideal of $R_2$. Then
    \[
        \phi^{-1}(J) = \{r \in R_1 \vert \phi(r) \in J\}
    \]
    is an ideal of $R_1$.
\end{proposition}
\begin{proof}
    Suppose $J$ is an ideal of $R_2$. We consider the test for ideal (\myref{thrm-test-for-ideal}) to show $\phi^{-1}(J)$ is an ideal of $R_1$.
    
    One sees that $\phi^{-1}(J) \neq \emptyset$ since $\phi(0_1) = 0_2 \in J$, so $0_1 \in \phi^{-1}(J)$.

    Let $a, b \in \phi^{-1}(J)$, so $\phi(a), \phi(b) \in J$. Note that
    \[
        \phi(a-b) = \phi(a) - \phi(b) \in J
    \]
    so $a-b \in \phi^{-1}(J)$ for all $a,b \in J$.

    Let $r \in R_1$ and $a \in \phi^{-1}(J)$. Note that $\phi(a) \in J$ and $\phi(r) \in R_2$, so
    \begin{align*}
        \underbrace{\phi(a)}_{\text{In }J}\underbrace{\phi(r)}_{\text{In }R_2} &\in J & (J\text{ is an ideal, so } rj \in J)\\
        \phi(r)\phi(a) &\in J & (\text{since } jr \in J).
    \end{align*}
    Note $\phi(a)\phi(r) = \phi(ar) \in J$, so $ar \in \phi^{-1}(J)$, and similarly we have $\phi(r)\phi(a) = \phi(ra) \in J$, so $ra \in \phi^{-1}(J)$.

    Therefore, by the test for ideal, is an ideal of $R_1$.
\end{proof}

\begin{exercise}\label{exercise-ring-image-of-identity-is-identity}
    Let $R_1$ and $R_2$ be rings, and $\phi: R_1 \to R_2$ be a ring homomorphism.
    \begin{partquestions}{\alph*}
        \item Show $\phi(0_1) = 0_2$, where $0_1$ and $0_2$ is the additive identity of $R_1$ and $R_2$ respectively.
        \item If $R_1$ and $R_2$ are division rings, then show $\phi(1_1) = 1_2$, where $1_1$ and $1_2$ is the multiplicative identity of $R_1$ and $R_2$ respectively.
    \end{partquestions}
\end{exercise}

\begin{exercise}\label{exercise-ring-image-of-inverse-is-inverse}
    Let $R_1$ and $R_2$ be rings, $x \in R_1$, and $\phi: R_1 \to R_2$ be a ring homomorphism.
    \begin{partquestions}{\alph*}
        \item Show that $\phi(-x) = -\phi(x)$.
        \item If $R_1$ and $R_2$ are division rings, show that $\phi(x^{-1}) = (\phi(x))^{-1}$.
    \end{partquestions}
\end{exercise}

\section{Image and Kernel}
Similar to groups, ring homomorphisms too have a image and kernel.
\begin{definition}
    The \textbf{image}\index{image} of a ring homomorphism $\phi: R_1 \to R_2$ is
    \[
        \im\phi = \{\phi(r) \vert r \in R_1\}.
    \]
\end{definition}
\begin{definition}
    The \textbf{kernel}\index{kernel} of a ring homomorphism $\phi:R_1 \to R_2$ is
    \[
        \ker\phi = \{r \in R_1 \vert \phi(r) = 0\}.
    \]
\end{definition}

\begin{example}\label{example-homomorphism-on-upper-triangle-matrices}
    Consider the ring
    \[
        R = \left\{\begin{pmatrix}a&b\\0&c\end{pmatrix}\vert a,b,c\in\Z\right\}.
    \]
    and the homomorphism $\phi: R \to \Z^2, \begin{pmatrix}a&b\\0&c\end{pmatrix} \mapsto (a,c)$.

    We note that $\phi$ is surjective; for any $(x,y)\in\Z^2$, we note that $\phi\left(\begin{pmatrix}x&0\\0&y\end{pmatrix}\right) = (x,y)$ so $(x,y)$ has a pre-image in $R$. Therefore $\im \phi = \Z^2$.

    We now find the kernel of $\phi$.
    \begin{align*}
        \ker\phi &= \left\{\begin{pmatrix}a&b\\0&c\end{pmatrix} \in R \vert \phi\left(\begin{pmatrix}a&b\\0&c\end{pmatrix}\right) = (0,0)\right\}\\
        &= \left\{\begin{pmatrix}a&b\\0&c\end{pmatrix} \in R \vert (a,c) = (0,0)\right\}\\
        &= \left\{\begin{pmatrix}0&n\\0&0\end{pmatrix} \vert n \in \Z\right\}.
    \end{align*}
\end{example}

We look at some results regarding the image and kernel of a ring homomorphism. These results may look familiar to those who read part I.
\begin{proposition}\label{prop-image-is-a-subring}
    Let $R_1$ and $R_2$ be rings, and let $\phi: R_1 \to R_2$ be a ring homomorphism. Then $\im\phi$ is a subring of $R_2$.
\end{proposition}
\begin{proof}
    Note that $\im\phi = \phi(R_1)$ is a subring of $R_2$ by \myref{prop-homomorphism-on-subring-is-subring}.
\end{proof}

\begin{proposition}\label{prop-kernel-is-an-ideal}
    Let $R_1$ and $R_2$ be rings, and let $\phi: R_1 \to R_2$ be a ring homomorphism. Then $\ker\phi$ is an ideal of $R_1$.
\end{proposition}
\begin{proof}
    See \myref{exercise-kernel-is-an-ideal} (later).
\end{proof}

\begin{proposition}
    Let $R_1$ and $R_2$ be rings, and let $\phi: R_1 \to R_2$ be a ring homomorphism. Then $\phi$ is injective if and only if $\ker\phi = \{0_1\}$.
\end{proposition}
\begin{proof}
    We first prove the forward direction; suppose $\phi$ is injective and let $a \in \ker\phi$. By definition of the kernel we have $\phi(a) = 0_2$. But by \myref{prop-ring-image-of-additive-identity-is-additive-identity}, we have $\phi(0_1) = 0_2$. Since $\phi$ is injective, therefore $a = 0_1$, meaning $\ker\phi = \{0_1\}$.

    We now prove the reverse direction; suppose $\ker\phi = \{0_1\}$. Now let $a,b \in R_1$ such that $\phi(a) = \phi(b)$. Therefore $\phi(a) - \phi(b) = \phi(a-b) = 0_2$. Therefore $a-b \in \ker\phi$ by definition of the kernel. However $\ker\phi = \{0_1\}$ which means that $a - b = 0_1$. Therefore $a = b$, meaning $\phi$ is injective.
\end{proof}

\begin{exercise}\label{exercise-kernel-is-an-ideal}
    Let $R_1$ and $R_2$ be rings, and let $\phi: R_1 \to R_2$ be a ring homomorphism. Prove that $\ker\phi$ is an ideal of $R_1$.
\end{exercise}

\newpage

\section{The Ring Isomorphism Theorems}
Similar to group theory, there are three main ring isomorphism theorems. However, we will only explicitly prove the first ring isomorphism theorem; the other two will be left as problems.

\begin{theorem}[First Ring Isomorphism Theorem (FRIT)]\label{thrm-ring-isomorphism-1}\index{ring isomorphism theorem!first}\index{FRIT}
    Let $R$ and $R'$ be rings. Let $\phi: R \to R'$ be a ring homomorphism, and let $\pi: R \to R/\ker\phi$ where $r\mapsto r + \ker\phi$ be the natural surjective homomorphism. Then there exists a unique ring isomorphism $\psi: R / \ker\phi \to \im\phi$ such that $\psi\pi = \phi$.
\end{theorem}
\begin{remark}
    Equivalently, the FRIT states that
    \[
        R / \ker\phi \cong \im\phi
    \]
    for any ring homomorphism $\phi$.
\end{remark}

We include the commutativity diagram of the maps stated to aid clarity:
\begin{figure}[h]
    \centering
    \pdfteximgframed[12pt]{0.25\textwidth}{part2/images/ring-homomorphisms/ring-iso-1-commutativity.pdf_tex}
    \caption{Commutativity Diagram for \myreffigures{thrm-ring-isomorphism-1}}
\end{figure}

In the diagram, $\phi$ sends elements from $R$ to $\im\phi$ and $\pi$ sends elements from $R$ to $R/\ker\phi$. Then the map $\psi$ is a unique map that sends elements from $R/\ker\phi$ to the image of $\phi$.

\begin{proof}[Proof (cf. {\cite[p.~302, Factor Theorem For Rings]{cohn_1982}})]
    Let the map $\phi$ be defined such that $\psi(r + \ker\phi) = \phi(r)$. We first show that $\psi$ is a well-defined ring isomorphism.
    \begin{itemize}
        \item \textbf{Well-Defined}: Suppose $a + \ker\phi$ and $b + \ker\phi$ are in $R/\ker\phi$ such that $a + \ker\phi = b+\ker\phi$. This means that $a - b \in \ker\phi$ by Coset Equality (\myref{lemma-coset-equality}). Thus $\phi(a-b) = 0$ by definition of the kernel. Hence $\phi(a) - \phi(b) = 0$ which means $\phi(a) = \phi(b)$. Therefore, we see that
        \[
            \psi(a + \ker\phi) = \phi(a) = \phi(b) = \psi(b + \ker\phi)
        \]
        which means $\psi$ is well-defined.

        \item \textbf{Homomorphism}: We first show that $\psi$ is a ring homomorphism.
        \begin{itemize}
            \item $\boxed{+}$: If $a + \ker\phi$ and $b + \ker\phi$ are in $R/\ker\phi$,
            \begin{align*}
                &\psi((a + \ker\phi)+(b+\ker\phi))\\
                &= \psi((a+b)+\ker\phi)\\
                &= \phi(a+b)\\
                &= \phi(a) + \phi(b)\\
                &= \psi(a + \ker\phi) + \psi(b + \ker\phi).
            \end{align*}
            \item $\boxed{\times}$: If $a + \ker\phi$ and $b + \ker\phi$ are in $R/\ker\phi$,
            \begin{align*}
                \psi((a + \ker\phi)(b+\ker\phi)) &= \psi((ab)+\ker\phi)\\
                &= \phi(ab)\\
                &= \phi(a)\phi(b)\\
                &= \psi(a + \ker\phi)\psi(b + \ker\phi).
            \end{align*}
        \end{itemize}
        Therefore $\psi$ is a ring homomorphism.

        \item \textbf{Injective}: Suppose $\psi(a+\ker\phi) = \psi(b+\ker\phi)$.
        \begin{align*}
            \phi(a) &= \phi(b) & (\text{definition of }\psi)\\
            \phi(a) - \phi(b) &= 0\\
            \phi(a-b) &= 0 & (\phi \text{ is a ring homomorphism})\\
            a - b &\in \ker\phi & (\text{definition of kernel})\\
            a + \ker\phi &= b + \ker\phi & (\text{Coset Equality})
        \end{align*}
        Therefore if $\psi(a+\ker\phi) = \psi(b+\ker\phi)$ then $a+\ker\phi = b+\ker\phi$, which means $\psi$ is injective.

        \item \textbf{Surjective}: Suppose $s \in \im\phi$, so there is an $r \in R$ such that $s = \phi(r)$. Clearly $\psi(r + \ker\phi) = \phi(r) = s$, so $s$ has a pre-image of $r + \ker\phi$, i.e. $\psi$ is surjective.
    \end{itemize}
    Therefore, $\psi$ is a well-defined bijective ring homomorphism, i.e. $\psi$ is a well-defined ring isomorphism.

    We now check that $\psi$ satisfies the requirement that $\psi\pi = \phi$. Let $x \in R$. Note that $\pi(x) = x + \ker\phi$, and
    \[
        \psi\pi(x) = \psi(x + \ker\phi) = \phi(x)
    \]
    for all $x \in R$, so $\psi\pi = \phi$.

    Finally we show that $\psi$ is unique. Suppose $f: R/\ker\phi \to \im\phi$ is an isomorphism satisfying $f\pi=\phi$. Take $x + \ker\phi \in R/\ker\phi$. Note that
    \begin{align*}
        f(x + \ker\phi) &= f(\pi(x))\\
        &= (f\pi)(x)\\
        &= \phi(x)\\
        &= (\psi\pi)(x)\\
        &= \psi(\pi(x))\\
        &= \psi(x + \ker\phi)
    \end{align*}
    for all $x \in R$, meaning that $f = \psi$. Therefore $\psi$ is unique.

    Hence, $\psi$ is a unique ring isomorphism satisfying $\psi\pi = \phi$.
\end{proof}

\begin{example}
    Consider the ring
    \[
        R = \left\{\begin{pmatrix}a&b\\0&c\end{pmatrix}\vert a,b,c\in\Z\right\}
    \]
    and the homomorphism $\phi: R \to \Z^2, \begin{pmatrix}a&b\\0&c\end{pmatrix} \mapsto (a,c)$. We found in \myref{example-homomorphism-on-upper-triangle-matrices} that $\phi$ is surjective (i.e., $\im\phi = \Z^2$) with kernel
    \[
        \left\{\begin{pmatrix}0&n\\0&0\end{pmatrix} \vert n \in \Z\right\}
    \]
    which, for brevity, we shall denote by $I$. Thus we see via the FRIT (\myref{thrm-ring-isomorphism-1}) that
    \[
        R/I \cong \Z^2.
    \]
\end{example}
\begin{exercise}
    Show that $\Z_n \cong \Z/n\Z$ via the ring homomorphism
    \[
        \phi: \Z \to \Z_n, m \mapsto m \mod n
    \]
    and by using the FRIT.
\end{exercise}

We briefly mention the other two main ring isomorphism theorems, although the proof of them will be left as problems. They are much less used than the FRIT, so we make only a passing mention of them.
\begin{theorem}[Second Ring Isomorphism Theorem]\label{thrm-ring-isomorphism-2}\index{ring isomorphism theorem!second}
    Let $R$ be a ring with subring $R$ and ideal $I$. Then
    \begin{enumerate}
        \item $S+I = \{s+i \vert s\in S,\;i\in I\}$ is a subring of $R$;
        \item $S \cap I$ is an ideal of $S$; and
        \item $(S+I)/I \cong S/(S\cap I)$.
    \end{enumerate}
\end{theorem}
\begin{proof}
    See \myref{problem-ring-isomorphism-2} (later).
\end{proof}

\begin{theorem}[Third Ring Isomorphism Theorem]\label{thrm-ring-isomorphism-3}\index{ring isomorphism theorem!third}
    Let $R$ be a ring with ideals $I$ and $J$ such that $I$ is a subset of $J$. Then
    \begin{enumerate}
        \item $J/I$ is an ideal of $R/I$; and
        \item $\frac{R/I}{J/I} \cong R/J$.
    \end{enumerate}
\end{theorem}
\begin{proof}
    See \myref{problem-ring-isomorphism-3} (later).
\end{proof}

\section{How Restrictive are Ring Homomorphisms?}
Although ring homomorphisms appear to be quite general, we explore how restricted they really are when dealing with certain rings.

\begin{example}\label{example-endomorphisms-of-Z}
    We find all ring endomorphisms of $\Z$.

    Let $\phi:\Z\to\Z$ be a ring endomorphism. Set $a = \phi(1)$. Note that
    \[
        a = \phi(1) = \phi(1\times1) = \phi(1)\phi(1) = a^2
    \]
    so $a^2 = a$. Thus $a = 0$ or $a = 1$ in $\Z$.

    If $a = 0$, then for any $n \in \Z$ we have
    \[
        \phi(n) = \phi(1n) = \phi(1)\phi(n) = 0\phi(n) = 0
    \]
    so $\phi(n) = 0$ for all $n \in \Z$, which is the trivial homomorphism.

    Now consider the case that $a = 1$. We claim that $\phi(n) = n$ for all $n \in \Z$.

    We leave the proof that $\phi(n) = n$ for all \textit{positive} integers $n$ for \myref{exercise-homomorphism-maps-n-to-n-if-n-is-positive} (later). Furthermore $\phi(0) = 0$ by the properties of ring homomorphism (specifically \myref{prop-ring-image-of-additive-identity-is-additive-identity}). Finally, note that for any non-negative integer $n$,
    \begin{align*}
        0 = \phi(0) &= \phi(n - n)\\
        &= \phi(n) + \phi(-n)\\
        &= n + \phi(-n)
    \end{align*}
    which means $\phi(-n) = -n$. Therefore $\phi(n) = n$ for all integers $n$.

    Therefore, the only ring endomorphisms $\phi:\Z\to\Z$ are $\phi(n) = 0$ or $\phi(n) = n$ for all integers $n$.
\end{example}
\begin{exercise}\label{exercise-homomorphism-maps-n-to-n-if-n-is-positive}
    In the above example, show that $\phi(n) = n$ for all positive integers $n$.
\end{exercise}
Note that in \myref{example-endomorphisms-of-Z} we started the entire computation with the observation that $\phi(1) = \phi(1)^2$. This means that $\phi(1)$ is an idempotent.
\begin{definition}
    Let $R$ be a ring. Then an element $x$ in $R$ is an \textbf{idempotent}\index{idempotent} if and only if $x^2 = x$.
\end{definition}
\begin{proposition}\label{prop-homomorphism-on-multiplicative-identity-is-idempotent}
    Let $R$ and $R'$ be rings, and let $\phi: R \to R'$ be a ring homomorphism. If $R$ is a ring with identity, then $\phi(1)$ is an idempotent.
\end{proposition}
\begin{proof}
    Note
    \[
        \phi(1) = \phi(1 \times 1) = \phi(1) \times \phi(1) = \left(\phi(1)\right)^2
    \]
    which means $\phi(1)$ is an idempotent.
\end{proof}

In \myref{example-endomorphisms-of-Z} we used the fact that the only idempotents of $\Z$ are 0 and 1. However, this is not true for all rings.

\begin{example}\label{example-homomorphisms-from-Z12-to-Z28}
    We find all ring homomorphisms $\phi: \Z_{12} \to \Z_{28}$. Note that $\Z_{12}$ is a ring with identity.

    By \myref{prop-homomorphism-on-multiplicative-identity-is-idempotent}, $a = \phi(1)$ is an idempotent. However, we cannot just assume that 0 and 1 are the \textit{only} idempotents in $\Z_{28}$; we need to check for them exhaustively.

    By exhaustion, we see that
    \begin{itemize}
        \item $0^2 = 0$;
        \item $1^2 = 1$;
        \item $8^2 = 64 = 2 \times 28 + 8 = 8$; and
        \item $21^2 = 441 = 15 \times 28 + 21 = 21$.
    \end{itemize}
    So the idempotents in $\Z_{28}$ are 0, 1, 8, and 21. This is not enough to narrow down the possible values of $\phi(1)$, so we need to invoke more facts.

    Recall from part I that $|\phi(1)|_+$ divides $|1|_+$ by \myref{exercise-order-of-homomorphism-divides-order}. Therefore $|\phi(1)|_+$ divides 12. Furthermore, \myref{thrm-order-of-element-in-cyclic-group} tells us that the additive order of an element $k$ in the group $(\Z_n, +)$ is $\frac{n}{\gcd(k,n)}$. So we must now exhaust all idempotents in $\Z_{28}$ to check whether it is a valid value of $\phi(1)$.
    \begin{itemize}
        \item $|0|_+ = 1$ which clearly divides 12, so it is a valid value of $\phi(1)$.
        \item $|1|_+ = 28$ which does not divide 12, so it is not a valid value of $\phi(1)$. This is one way that differs from the previous example, where 1 \textit{was} a possible value of $\phi(1)$.
        \item $|8|_+ = \frac{28}{\gcd(8,28)} = \frac{28}4 = 7$ which does not divide 12, so it is not a valid value of $\phi(1)$.
        \item $|21|_+ = \frac{28}{\gcd(21,28)} = \frac{28}7 = 4$ which divides 12, so it is a valid value of $\phi(1)$.
    \end{itemize}
    Hence $a \in \{0, 21\}$, i.e. $\phi(1) = 0$ or $\phi(1) = 21$.

    If $\phi(1) = 0$, then for any $n \in \Z_{12}$,
    \begin{align*}
        \phi(n) &= \phi(\underbrace{1 + 1 + \cdot + 1}_{n \text{ times}})\\
        &= \underbrace{\phi(1) + \phi(1) + \cdot + \phi(1)}_{n \text{ times}}\\
        &= \underbrace{0 + 0 + \cdots + 0}_{n \text{ times}}\\
        &= 0
    \end{align*}
    which means that $\phi(n) = 0$ for all $n \in \Z_{12}$, i.e. $\phi$ is trivial.

    If instead $\phi(1) = 21$, then
    \begin{align*}
        \phi(n) &= \underbrace{\phi(1) + \phi(1) + \cdot + \phi(1)}_{n \text{ times}}\\
        &= \underbrace{21 + 21 + \cdots + 21}_{n \text{ times}}\\
        &= 21n
    \end{align*}
    which means $\phi(n) = 21n$ for all $n \in \Z_{12}$.

    Thus the only homomorphisms $\phi: \Z_{12} \to \Z_{28}$ are $\phi(n) = 0$ and $\phi(n) = 21n$ for all $n \in \Z_{12}$.
\end{example}

\begin{exercise}\label{exercise-homomorphism-over-Q-fixes-elements-of-Q}
    Suppose $R$ and $R'$ are rings such that $\Q$ is a subring of both $R$ and $R'$. Let $\phi: R \to R'$ be a ring homomorphism such that $\phi(1) = 1$. Show that for any $q \in Q$ we have $\phi(q) = q$.
\end{exercise}

\newpage

\section{Problems}
\begin{problem}
    Let $R$ be a ring.
    \begin{partquestions}{\roman*}
        \item Show that $R/\{0\} \cong R$.
        \item Prove that $R$ is an integral domain if and only if $\{0\}$ is a prime ideal.
        \item Prove that $R$ is a field if and only if $\{0\}$ is a maximal ideal.
    \end{partquestions}
\end{problem}

\begin{problem}
    Find all ring endomorphisms of $\Q$.
\end{problem}

\begin{problem}
    Show $\Z^2 \not\cong \Q$.
\end{problem}

\begin{problem}
    Show $\Q[\sqrt2] \not\cong \Q[\sqrt3]$.
\end{problem}

\begin{problem}
    Let
    \[
        R = \left\{\begin{pmatrix}a&0\\0&b\end{pmatrix}\vert a,b \in \Z\right\},
    \]
    which is a subring of $\Mn{2}{\Z}$. Show that $R \cong \Z^2$.
\end{problem}

\begin{problem}
    Let $R$ and $R'$ be rings, and let $\phi: R \to R'$ be a ring isomorphism. Prove or disprove the following statements.
    \begin{partquestions}{\alph*}
        \item $\phi^{-1}: R' \to R$ is a ring isomorphism.
        \item If $R$ has a subring with $n$ elements, then so does $R'$.
        \item If $R$ has an ideal, then so does $R'$.
    \end{partquestions}
\end{problem}

\begin{problem}
    Find all ring endomorphisms of $\Z_{10}$.\newline
    Hence find all ring isomorphisms $\psi: \Z_{10} \to \Z_{10}$.
\end{problem}

\begin{problem}
    Find all endomorphisms over $\Q[\sqrt3]$.\newline
    Hence find all ring isomorphisms $\psi: \Q[\sqrt3] \to \Q[\sqrt3]$.
\end{problem}

\begin{problem}
    Let $R$ and $R'$ be commutative rings, $I$ be an ideal of $R$, and $\phi: R\to R'$ be a ring homomorphism.
    \begin{partquestions}{\roman*}
        \item Show that $\phi(\sqrt I) \subseteq \sqrt{\phi(I)}$.
        \item If $\phi$ is surjective with $\ker\phi \subseteq I$, prove that $\phi(\sqrt I) = \sqrt{\phi(I)}$.
    \end{partquestions}
\end{problem}

\begin{problem}\label{problem-ring-isomorphism-2}
    Let $R$ be a ring with subring $S$ and ideal $I$.
    \begin{partquestions}{\roman*}
        \item Prove $S+I$ is a subring of $R$.
        \item Prove $S \cap I$ is an ideal of $S$.
        \item Prove $S/(S\cap I)\cong (S+I)/I$.
    \end{partquestions}
\end{problem}

\newpage

\begin{problem}\label{problem-ring-isomorphism-3}
    Let $R$ be a ring with ideals $I$ and $J$ such that $I$ is a subset of $J$.
    \begin{partquestions}{\roman*}
        \item Prove that $J/I$ is an ideal of $R/I$.
        \item Prove that $\frac{R/I}{J/I} \cong R/J$.\newline
        (\textit{Note: remember to prove that the map is well-defined.})
    \end{partquestions}
\end{problem}

\chapter{Polynomial Rings and Division}
Polynomial rings are an important part of algebra and in ring theory, since polynomials are ubiquitous in modern algebra. We explore polynomials and polynomial rings in this chapter.

\section{What is a Polynomial Ring?}
We first define polynomials.
\begin{definition}
    A \textbf{polynomial}\index{polynomial} is an expression consisting of \textbf{variables}\index{polynomial!variable} (or \textbf{indeterminates}\index{polynomial!indeterminate}) and coefficients, that involves only the operations of addition, subtraction, multiplication, and positive-integer powers of variables.
\end{definition}
\begin{definition}
    Polynomials in a single variable are called \textbf{univariate polynomials}\index{polynomial!univariate} and takes the form
    \[
        a_0+a_1x+a_2x^2+\cdots+a_nx^n = \sum_{i=0}^n a_ix^i,
    \]
    where $a_0, a_1, a_2, \dots, a_n$ are called the \textbf{coefficients}\index{polynomial!coefficient} of the polynomial.
\end{definition}

Note that in the above definition, we did not explicitly state where the coefficients originate from; we do so now in the definition for a polynomial ring.
\begin{definition}
    Let $R$ be a commutative ring with identity, where $1 \neq 0$. Then the \textbf{polynomial ring}\index{polynomial ring} in \textbf{indeterminate}\index{indeterminate} (or \textbf{variable}\index{variable}) $x$ and coefficients in $R$ is
    \[
        R[x] = \{a_0 + a_1x + \cdots + a_nx^n \vert n \in \mathbb{N} \cup \{0\}, a_i \in R\},
    \]
    where for $f(x) = a_0 + \cdots + a_mx^m, g(x) = b_0 + \cdots + b_nx^n \in R[x]$, and assuming $m \leq n$, we define addition\index{polynomial!addition} and multiplication\index{polynomial!multiplication} by
    \begin{align*}
        f(x) + g(x) &= \sum_{i=0}^n\left((a_i+b_i)x^i\right)\\
        f(x)\times g(x) &= \sum_{k=0}^{m+n}\left(\left(\sum_{i=0}^k a_ib_{k-i}\right)x^k\right)
    \end{align*}
    respectively, where $a_l = 0$ for all $l > m$.
\end{definition}
\begin{proposition}
    The polynomial ring $R[x]$ is a commutative ring.
\end{proposition}
\begin{proof}
    We first prove that $R[x]$ is indeed a ring, before proving commutativity of multiplication of polynomials. For brevity, let $f(x), g(x), h(x) \in R[x]$ where
    \begin{align*}
        f(x) &= a_0 + a_1x + a_2x^2 + \cdots + a_mx^m,\\
        g(x) &= b_0 + b_1x + b_2x^2 + \cdots + b_nx^n,\\
        h(x) &= c_0 + c_1x + c_2x^2 + \cdots + c_lx^l,
    \end{align*}
    each $a_i$, $b_j$, and $c_k$ are elements from $R$, the integers $m$, $n$, and $l$ are all non-negative, and $a_m$, $b_n$, and $c_l$ are all non-zero. Without loss of generality, assume $m \geq n \geq l$, and `pad' the polynomials $g(x)$ and $h(x)$ with extra terms so that the highest power in $x$ is $m$.
    \begin{itemize}
        \item \textbf{Addition-Abelian}: We show that $(R[x], +)$ is an abelian group.
        \begin{itemize}
            \item \textbf{Closure}: We see
            \[
                f(x) + g(x) = \sum_{i=0}^n\left((a_i+b_i)x^i\right)
            \]
            and since $R$ is a ring, thus $a_i+b_i \in R$, meaning $f(x) + g(x)$ is another polynomial in $R[x]$. Therefore $R[x]$ is closed under addition.
            
            \item \textbf{Associativity}: Note
            \begin{align*}
                f(x) + (g(x) + h(x)) &= f(x) + \sum_{i=0}^m\left((b_i+c_i)x^i\right)\\
                &= \sum_{i=0}^m\left((a_i + (b_i + c_i))x^i\right)\\
                &= \sum_{i=0}^m\left(((a_i + b_i) + c_i)x^i\right) & (+ \text{ is associative in }R)\\
                &= \sum_{i=0}^m\left((a_i+b_i)x^i\right) + h(x)\\
                &= (f(x) + g(x)) + h(x)
            \end{align*}
            so addition of functions is associative.
            
            \item \textbf{Identity}: Note that $0 \in R$ is also the identity in $R[x]$, since
            \[
                0 + f(x) = \sum_{i=0}^m\left((0+a_i)x^i\right) = \sum_{i=0}^m\left(a_ix^i\right) = f(x).
            \]
            
            \item \textbf{Inverse}: For the polynomial $f(x)$, construct the polynomial $-f(x)$ where the coefficient of $x^i$ in $-f(x)$ is $-a_i$. Then
            \[
                f(x) + (-f(x)) = \sum_{i=0}^m\left((a_i+(-a_i))x^i\right) \sum_{i=0}^m\left(0x^i\right) = 0
           \]
           so $-f(x)$ is indeed the additive inverse of $f(x)$.
            
            \item \textbf{Commutativity}: One sees clearly that
            \[
                f(x) + g(x) = \sum_{i=0}^m\left((a_i+b_i)x^i\right) = \sum_{i=0}^m\left((b_i + a_i)x^i\right) = g(x) + f(x)
            \]
            since addition in $R$ is commutative. Therefore addition in $R[x]$ is also commutative.
        \end{itemize}

        \item \textbf{Multiplication-Subgroup}: We show that $(R[x], \times)$ is a subgroup.
        \begin{itemize}
            \item \textbf{Closure}: We note that
            \[
                f(x)\times g(x) = \sum_{k=0}^{m+n}\left(\underbrace{\left(\sum_{i=0}^k a_ib_{k-i}\right)}_{\text{In }R}x^k\right)
            \]
            so $f(x) \times g(x)$ is another polynomial in $R$.
            
            \item \textbf{Associativity}: \myref{exercise-polynomial-multiplication-is-associative} (later) proves that polynomial multiplication is associative.
        \end{itemize}

        \item \textbf{Distributive}: We finally show that $\times$ distributes over $+$. We only show that $f(x)(g(x) + h(x)) = f(x)g(x) + f(x)h(x)$ as we will later prove that $R[x]$ is commutative. Note
        \begin{align*}
            f(x)(g(x) + h(x)) &= \sum_{i=0}^m\left(\left(\sum_{j=0}^ia_j(b_{i-j}+c_{i-j})\right)x^i\right)\\
            &= \sum_{i=0}^m\left(\left(\sum_{j=0}^i(a_jb_{i-j}+a_jc_{i-j})\right)x^i\right) & (\text{Distribute in }R)\\
            &= \sum_{i=0}^m\left(\sum_{j=0}^i(a_jb_{i-j}x^i+a_jc_{i-j}x^i)\right)\\
            &= \sum_{i=0}^m\left(\sum_{j=0}^ia_jb_{i-j}x^i + \sum_{j=0}^ia_jc_{i-j}x^i\right)\\
            &= \sum_{i=0}^m\left(\sum_{j=0}^ia_jb_{i-j}x^i\right) + \sum_{i=0}^m\left(\sum_{j=0}^ia_jc_{i-j}x^i\right)\\
            &= f(x)g(x) + f(x)h(x)
        \end{align*}
        which is what was needed to be shown.
    \end{itemize}
    Therefore $R[x]$ is a ring.

    Now we prove commutativity of multiplication. Let $f(x) = a_0 + \cdots + a_mx^m, g(x) = b_0 + \cdots + b_nx^n \in R[x]$, and $m \leq n$. Then
    \begin{align*}
        f(x)g(x) &= \sum_{k=0}^{m+n}\left(\left(\sum_{i=0}^k a_ib_{k-i}\right)x^k\right)\\
        &= \sum_{k=0}^{m+n}\left(\left(\sum_{i=0}^k b_{k-i}a_i\right)x^k\right) & (R\text{ is commutative})\\
        &= \sum_{k=0}^{m+n}\left(\left(\sum_{i=0}^k b_i a_{k-i}\right)x^k\right)\\
        &= g(x)f(x)
    \end{align*}
    which therefore means that $R[x]$ is a commutative ring.
\end{proof}

Let's look at some examples of polynomial rings.
\begin{example}
    The polynomial ring $\R[x]$ is the most familiar for most of us, as this is the `standard' ring of polynomials. Examples of polynomials in this ring include $1+x$, $\sqrt2x^{10} - 5x^3 + \pi x$, and $1+x+x^2+\cdots+x^n$. However, infinite polynomials such as $1+x+x^2+\cdots$ do not belong in $\R[x]$.
\end{example}
\begin{example}
    Another commonly used polynomial ring is $\Q[x]$. Examples of polynomials in this ring are $1+x$, $\frac23x^5 - \frac7{11}x^{13}$, and $2x^2-5x-3$. However polynomials like $\sqrt2$, $\pi x + 1$, and $1+ex$ do not belong in $\Q[x]$.
\end{example}
\begin{example}
    We look at the polynomial ring $\Mn{2}{\R}[x]$. One example of a polynomial in this ring is
    \[
        \begin{pmatrix}2&1\\-2&0\end{pmatrix} + \begin{pmatrix}\sqrt2&1\\1&1\end{pmatrix}x + \begin{pmatrix}5&4\\e&2\end{pmatrix}x^2 + \begin{pmatrix}0&1\\0&0\end{pmatrix}x^5.
    \]
\end{example}

\begin{exercise}\label{exercise-polynomial-multiplication-is-associative}
    Prove that polynomial multiplication is associative.\newline
    (\textit{Hint: $\displaystyle \sum_{i=0}^k a_ib_{k-i} = \sum_{i+j=k} a_ib_j$. The second sum is the sum over all non-negative integers $i$ and $j$ with the property that $i+j=k$.})
\end{exercise}
\begin{exercise}
    Let $R$ be a ring. The \textbf{evaluation homomorphism}\index{evaluation homomorphism} is $\phi_a: R[x] \to R$ where $\phi_a(p(x)) = p(a)$ and $a \in R$. Prove that $\phi_a$ is indeed a ring homomorphism.
\end{exercise}
\begin{exercise}
    Let $I$ be a principal ideal of $\Z[x]$ generated by the polynomial $x^2 + 3x - 1$. Simplify $\left((x + 3) + I\right)\left((2x^2 + 3x - 1) + I\right)$ in the quotient ring $\Z[x]/I$.
\end{exercise}

We end this section by noting what form $R[x]$ takes when $x$ is \textit{not} a variable.
\begin{example}
    Recall that $\Q[\sqrt2] = \{a + b\sqrt2 \vert a,b \in \Q\}$. If we use the definition of $\Q[x]$, we see that
    \begin{align*}
        &\Q[\sqrt2] \\
        &= \{a_0 + a_1\sqrt2 + a_2(\sqrt2)^2 + a_3(\sqrt2)^3 + \cdots + a_n(\sqrt2)^n \vert a_i \in \Q \}\\
        &= \{a_0 + a_1\sqrt2 + a_2(2) + a_3(2\sqrt2) + \cdots + a_n(\sqrt2)^n \vert a_i \in \Q \}\\
        &= \{(a_0 + 2a_2 + 4a_4 + \cdots) + \sqrt2 (a_1 + 2a_3 + 4a_5 + \cdots) \vert a_i \in \Q\}\\
        &= \{a + b\sqrt2 \vert a,b \in Q\}
    \end{align*}
    which agrees with the previous definition of $\Q[\sqrt2]$.
\end{example}
\begin{example}
    Recall that $\Z[i]$, the gaussian integers, is the set $\{a+bi \vert a,b \in \Z\}$. If we use the definition of $\Z[x]$, we see that
    \begin{align*}
        \Z[i] &= \{a_0 + a_1i + a_2i^2 + a_3i^3 + \cdots + a_ni^n \vert a_i \in \Z\}\\
        &= \{a_0 + a_1i + a_2(-1) + a_3(-i) + \cdots + a_ni^n \vert a_i \in \Z\}\\
        &= \{(a_0 - a_2 + a_4 - \cdots) + i(a_1 - a_3 + a_5 - \cdots) \vert a_i \in \Z\}\\
        &= \{a + bi \vert a,b \in \Z\}
    \end{align*}
    which agrees with the previous definition of $\Z[i]$.
\end{example}

\section{Basic Terminology in Polynomial Rings}
We define the degree of a polynomial.
\begin{definition}
    Let $R[x]$ be a polynomial ring. The \textbf{degree}\index{degree} of a polynomial $f(x) \in R[x]$, denoted $\deg f(x)$, is the largest integer $k$ such that the coefficient of $x^k$ of $f(x)$ is non-zero.
\end{definition}
\begin{remark}
    For the zero polynomial (0), the degree is undefined.
\end{remark}
\begin{example}
    The degree of the polynomial $1+x+5x^2$ in $\Z[x]$ is 2.
\end{example}
\begin{example}
    The degree of the polynomial
    \[
        \begin{pmatrix}0&1\\3&0\end{pmatrix}x^5 + \begin{pmatrix}3&6\\7&2\end{pmatrix}x^4 + \begin{pmatrix}3&4\\9&4\end{pmatrix}
    \]
    in $\Mn{2}{\Z}[x]$ is 5.
\end{example}
\begin{exercise}
    Give an example of a degree 5 polynomial in the ring $\Z_2[x]$.
\end{exercise}

\begin{definition}
    Let $f(x) = a_0 + a_1x + a_2x^2 + \cdots + a_nx^n$ be a polynomial in the polynomial ring $R[x]$.
    \begin{itemize}
        \item The \textbf{constant term}\index{constant term} is $a_0$.
        \item The \textbf{leading term}\index{leading term} is the term $a_nx^n$.
        \item The \textbf{leading coefficient}\index{leading coefficient} is $a_n$.
    \end{itemize}
\end{definition}
\begin{remark}
    For the zero polynomial, the constant term is 0, the leading term is undefined, and the leading coefficient is undefined.
\end{remark}
\begin{example}
    Consider the polynomial
    \[
        \begin{pmatrix}0&1\\3&0\end{pmatrix}x^5 + \begin{pmatrix}3&6\\7&2\end{pmatrix}x^4 + \begin{pmatrix}3&4\\9&4\end{pmatrix}
    \]
    in $\Mn{2}{\Z}[x]$. Then
    \begin{itemize}
        \item the constant term is $\begin{pmatrix}3&4\\9&4\end{pmatrix}$;
        \item the leading term is $\begin{pmatrix}0&1\\3&0\end{pmatrix}x^5$; and
        \item the leading coefficient is $\begin{pmatrix}0&1\\3&0\end{pmatrix}$.
    \end{itemize}
\end{example}

\begin{definition}
    A \textbf{constant polynomial}\index{constant polynomial} is either the zero polynomial of a polynomial of degree 0.
\end{definition}
\begin{remark}
    This definition immediately implies that any constant polynomial in the polynomial ring $R[x]$ is an element of $R$.
\end{remark}

\section{Properties of Polynomials and Polynomial Rings}
We can now state the theorem that produces a condition for a ring to be an integral domain.
\begin{theorem}\label{thrm-integral-domain-iff-polynomial-ring-is-also}
    Let $R$ be a ring. Then $R$ is an integral domain if and only if $R[x]$ is an integral domain.
\end{theorem}
\begin{proof}
    We first need to show that $R$ is a commutative ring with identity if and only if $R[x]$ is a commutative ring with identity. We leave this for \myref{exercise-commutative-ring-with-identity-iff-polynomial-ring-is-also} (later). We only prove that $R$ has no zero divisors if and only if $R[x]$ has no zero divisors using a contrapositive proof.

    For the forward direction, take non-zero $a$ and $b$ in $R$ such that $ab = 0$. We may view both $a$ and $b$ as degree 0 polynomials in $R[x]$. Clearly these two multiply together to form the zero polynomial in $R[x]$, meaning that they are zero divisors in $R[x]$.

    For the reverse direction, take non-zero polynomials $f(x)$ and $g(x)$ in $R[x]$ such that $f(x)g(x) = 0$. Write
    \begin{align*}
        f(x) = a_0+a_1x+a_2x^2+\cdots+a_mx^m \text{ with } a_m \neq 0\\
        g(x) = b_0+b_1x+b_2x^2+\cdots+b_nx^n \text{ with } b_n \neq 0
    \end{align*}
    where all coefficients are in $R$. Multiplying them together yields something like
    \[
        a_mb_nx^{m+n} + (\text{A polynomial with degree less than }m+n) = 0
    \]
    which hence means that all coefficients must be zero. Therefore $a_mb_n = 0$. This means that we have found non-zero elements $a_m$ and $b_n$ in $R$ such that their product is zero, meaning that they are zero divisors.

    This completes the proof.
\end{proof}
\begin{exercise}\label{exercise-commutative-ring-with-identity-iff-polynomial-ring-is-also}
    Let $R$ be a ring.
    \begin{partquestions}{\alph*}
        \item Prove that $R$ is a ring with identity if and only if $R[x]$ is a ring with identity.
        \item Prove that $R$ is a commutative ring if and only if $R[x]$ is a commutative ring.
    \end{partquestions}
\end{exercise}

\newpage

\section{Problems}
\begin{problem}
    Show that $\princ{x}$ is a prime ideal in $\Z[x]$.
\end{problem}
\begin{problem}
    Let $I = \{f(x) \in \Z[x] \vert f(-2) = 0\}$ be a subset of $\Z[x]$, and let the map $\phi:\Z[x]\to\Z, f(x) \mapsto f(-2)$.
    \begin{partquestions}{\roman*}
        \item Show that $\phi$ is a ring homomorphism.
        \item Show that $I$ is an ideal of $\Z[x]$.
        \item Hence determine if the ideal $I$ is prime, maximal, or both.
    \end{partquestions}
\end{problem}
\begin{problem}
    Prove that $\Z[x] / \princ{x} \cong \Z$.
\end{problem}


\appendix
\section{Introduction to Rings}
\subsection*{Exercises}
\begin{questions}
    \item We prove the ring axioms.
    \begin{itemize}
        \item \textbf{Addition-Abelian}: $(\{0\}, +)$ is an abelian group since this is just the trivial group.
        \item \textbf{Multiplication-Semigroup}: $(\{0\}, \cdot)$ is an abelian group since this is, again, just the trivial group. So $(\{0\}, \cdot)$ is a semigroup.
        \item \textbf{Distributive}: We know $+$ and $\cdot$ distribute.
    \end{itemize}
    Hence $(\{0\}, +, \cdot)$ is a commutative ring with identity.
\end{questions}

\subsection*{Problems}
No problems.

\section{Basics of Rings}
\begin{questions}
    \item We note the following.
    \begin{itemize}
        \item \textbf{Addition-Abelian}: $(\Z, +)$ is an abelian group.
        \item \textbf{Multiplication-Semigroup}: $(\Z, \times)$ is a semigroup since
        \begin{itemize}
            \item multiplying two integers always results in an integer, so $\Z$ is closed under $\times$; and
            \item $\times$ is associative.
        \end{itemize}
        \item \textbf{Distributive}: We know $+$ and $\times$ distribute.
    \end{itemize}
    Hence $(\Z, +, \times)$ is a ring.

    \item Consider $(-a)(-b) + (-ab)$ and note
    \begin{align*}
        &(-a)(-b) + (-ab)\\
        &= (-a)(-b) + (-a)b & (\text{\myref{prop-product-of-element-and-additive-inverse-is-additive-inverse-of-product}})\\
        &= (-a)(-b + b) & (\text{by \textbf{Distributive} axiom})\\
        &= (-a)0\\
        &= 0 & (\myref{prop-multiplying-by-zero-is-zero})
    \end{align*}
    which means $(-a)(-b) = -(-ab) = ab$ as required.

    \item The ring $\Mn{2}{\mathbb{R}}$ indeed has zero divisors, as $\begin{pmatrix}0&1\\0&0\end{pmatrix} \neq \begin{pmatrix}0&0\\0&0\end{pmatrix}$ but $\begin{pmatrix}0&1\\0&0\end{pmatrix}^2 = \begin{pmatrix}0&0\\0&0\end{pmatrix}$ which means that $\begin{pmatrix}0&1\\0&0\end{pmatrix}$ is a zero divisor.

    \item \begin{partquestions}{\alph*}
        \item $\Z$ is not a field. Note that the multiplicative inverse of 2 is $\frac12$ which is not an integer. Hence not all non-zero elements in $\Z$ has a multiplicative inverse, meaning that not all non-zero elements are units.

        \item $\Q$ is a field. Note for any rational number $\frac ab$ (where $b \neq 0$) it has an inverse of $\frac ba$. Thus any non-zero rational number is a unit, which means $\Q$ is a division ring. Since $\Q$ is also a commutative ring, therefore $\Q$ is a field.
    \end{partquestions}

    \item Let $u$ and $v$ be units, meaning that $u^{-1}$ and $v^{-1}$ exist. Then one sees that $(uv)(v^{-1}u^{-1}) = (v^{-1}u^{-1})(uv) = 1$, which means that $uv$ is also a unit.

    \item We first show that $(R, +) \leq (\Mn{2}{\R}, +)$.
    \begin{itemize}
        \item Clearly the identity of $(\Mn{2}{\R}, +)$, the zero matrix $\begin{pmatrix}0&0\\0&0\end{pmatrix}$, is inside $R$.
        \item Consider $\begin{pmatrix}a&a\\a&a\end{pmatrix}, \begin{pmatrix}b&b\\b&b\end{pmatrix} \in R$. The additive inverse of the matrix $\begin{pmatrix}b&b\\b&b\end{pmatrix}$ is the matrix $\begin{pmatrix}-b&-b\\-b&-b\end{pmatrix}$, and so their sum is
        \[
            \begin{pmatrix}a&a\\a&a\end{pmatrix} + \begin{pmatrix}-b&-b\\-b&-b\end{pmatrix} = \begin{pmatrix}a-b&a-b\\a-b&a-b\end{pmatrix} \in R
        \]
        which means $R$ is closed under addition.
    \end{itemize}
    Hence $(R, +) \leq (\Mn{2}{\R}, +)$ by subgroup test.

    We now show that $R$ is closed under multiplication. Some calculation yields that
    \[
        \begin{pmatrix}a&a\\a&a\end{pmatrix}\begin{pmatrix}b&b\\b&b\end{pmatrix} = \begin{pmatrix}2ab&2ab\\2ab&2ab\end{pmatrix}
    \]
    which is clearly in $R$. Therefore $R$ is a subring of $\Mn{2}{\R}$.
\end{questions}

\section{Integral Domains}
\begin{questions}
    \item To find a $a+bi \in \Z_5[i]$ such that there exists a $c+di \in \Z_5[i]$ where $(a+bi)(c+di) = 0$ but both $a+bi$ and $c+di$ are non-zero. Expanding $(a+bi)(c+di)$ yields $(ac-bd)+(ad+bc)i = 0$. Therefore we must have $ac-bd = 0$ and $ad+bc = 0$. For simplicity let's choose $a=c=1$. Using second equation we have $d+b = 0$ which means $d = -b$. Hence $(1 - b(-b))+(-b + b)i = 1+b^2 = 0$. Therefore choosing $b = 2$ would make it work. Therefore one solution is $a = 1, b = 2, c = 1, d = -2 = 3$; i.e. two zero divisors are $1+2i$ and $1+3i$.

    \item \begin{partquestions}{\alph*}
        \item Note that multiplication is commutative with identity $1 = 1 + 0\sqrt{n} \in R$. We just need to show that there are no zero divisors in $R$.

        Take $a+b\sqrt n, c+d\sqrt n \in R$ such that $a+b\sqrt n \neq 0$ but $(a+b\sqrt n)(c+d\sqrt n) = 0$. We want to show $c = d = 0$. Consider
        \[
            \left((a+b\sqrt n)(\underbrace{a-b\sqrt n}_{\neq 0})\right)\left((c+d\sqrt n)(\underbrace{c-d\sqrt n}_{\neq 0})\right) = 0.
        \]
        This means that $(a^2-nb^2)(c^2-nd^2) = 0$, so either $a^2-nb^2 = 0$ or $c^2-nd^2 = 0$.

        Now if $n < 0$ then clearly we have to have $c = d = 0$. Otherwise we have $a = b\sqrt n$ or $c = d\sqrt n$. But $\sqrt n$ is not an integer, so the only way for equality is if $c = d = 0$. Thus $\Z[\sqrt n]$ has no zero divisors, meaning $\Z[\sqrt n]$ is an integral domain.

        \item Consider $2 + \sqrt 2 \in \Z[\sqrt 2]$. Its multiplicative inverse is
        \begin{align*}
            \frac{1}{2+\sqrt2} &= \frac{2-\sqrt2}{(2+\sqrt2)(2-\sqrt2)}\\
            &= \frac{2-\sqrt2}{4-2}\\
            &= 1 - \frac12\sqrt2 \notin \Z[\sqrt2].
        \end{align*}
        This means that $2+\sqrt2$, a non-zero element in $\Z[\sqrt2]$, does not have an inverse in $\Z[\sqrt2]$. Therefore $\Z[\sqrt2]$ is not a field, meaning $R$ is not a field in the general case.
    \end{partquestions}

    \item For brevity let O$ = \begin{pmatrix}0&0\\0&0\end{pmatrix}$, I$ = \begin{pmatrix}1&0\\0&1\end{pmatrix}$, A$ = \begin{pmatrix}1&1\\1&0\end{pmatrix}$, and B$ = \begin{pmatrix}0&1\\1&1\end{pmatrix}$.

    \begin{partquestions}{\roman*}
        \item Clearly one sees that $R$ is a subset of $\Mn{2}{\Z_2}$.
        \begin{itemize}
            \item We show $(R, +)\leq(\Mn{2}{\Z_2},+)$.
            \begin{table}[H]
                \centering
                \begin{tabular}{|l|l|l|l|l|}
                    \hline
                    \textbf{+} & \textbf{O} & \textbf{I} & \textbf{A} & \textbf{B} \\ \hline
                    \textbf{O} & O          & I          & A          & B          \\ \hline
                    \textbf{I} & I          & O          & B          & A          \\ \hline
                    \textbf{A} & A          & B          & O          & I          \\ \hline
                    \textbf{B} & B          & A          & I          & O          \\ \hline
                \end{tabular}
            \end{table}

            From the Cayley table, clearly the identity of the ring $\Mn{2}{\Z_2}$ is in $R$ and $R$ is closed under addition. Hence $(R, +)\leq(\Mn{2}{\Z_2},+)$

            \item We show $R$ is closed under multiplication.
            \begin{table}[H]
                \centering
                \begin{tabular}{|l|l|l|l|l|}
                    \hline
                    $\boldsymbol{\cdot}$ & \textbf{O} & \textbf{I} & \textbf{A} & \textbf{B} \\ \hline
                    \textbf{O}           & O          & O          & O          & O          \\ \hline
                    \textbf{I}           & O          & I          & A          & B          \\ \hline
                    \textbf{A}           & O          & A          & B          & I          \\ \hline
                    \textbf{B}           & O          & B          & I          & A          \\ \hline
                \end{tabular}
            \end{table}

            From the Cayley table, clearly $R$ is closed under multiplication.
        \end{itemize}
        Therefore $R$ is a subring of $\Mn{2}{\Z_2}$.

        \item Since $R$ is a subring of $\Mn{2}{\Z_2}$, it is a ring. Furthermore, by the Cayley table of $(R, \cdot)$, we see that $R$ is commutative with identity I. Finally, one sees that $\mathrm{A}^{-1} = \mathrm{B}$, $\mathrm{B}^{-1} = \mathrm{A}$, and $\mathrm{I}^{-1} = \mathrm{I}$. Therefore all non-zero elements of $R$ have inverses. Hence $R$ is a field.
    \end{partquestions}
\end{questions}

\section{Ideals and Quotient Rings}
\begin{questions}
    \item We are given that $(I, +) \leq (R,+)$, so all that remains to show is that $I$ is closed under multiplication. Take any two elements $x$ and $y$ in $I$. Since $I$ is an ideal, thus we have $ri \in I$ for any $r \in R$ and $i \in I$. Viewing $x$ as an element of $R$ and $y$ as an element of $I$, we see that $xy \in I$, meaning $I$ is closed under multiplication. Hence $I$ is a subring of $R$.
    
    \item \begin{partquestions}{\roman*}
        \item Clearly the zero matrix, the additive identity of $R$, is inside $I$. Also,
        \[
            \begin{pmatrix}a&b\\0&0\end{pmatrix} + (-\begin{pmatrix}c&d\\0&0\end{pmatrix}) = \begin{pmatrix}a-c&b-d\\0&0\end{pmatrix} \in I
        \]
        so $I$ is a subring of $R$.

        \item Let $\begin{pmatrix}a&b\\0&0\end{pmatrix} \in I$ and $\begin{pmatrix}x&y\\0&z\end{pmatrix} \in R$. We need to show that $I$ is both a left and right ideal.
        \begin{itemize}
            \item \textbf{Left Ideal}:
            \[
                \begin{pmatrix}x&y\\0&z\end{pmatrix}\begin{pmatrix}a&b\\0&0\end{pmatrix} = \begin{pmatrix}xa&xb\\0&0\end{pmatrix} \in I;
            \]
            and
            \item \textbf{Right Ideal}: \[
                \begin{pmatrix}a&b\\0&0\end{pmatrix}\begin{pmatrix}x&y\\0&z\end{pmatrix} = \begin{pmatrix}ax&ay+bz\\0&0\end{pmatrix} \in I.
            \]
        \end{itemize}
        Therefore $I$ is an ideal of $R$.

        \item $\begin{pmatrix}1&0\\0&1\end{pmatrix} + I$
    \end{partquestions}

    \item \begin{partquestions}{\roman*}
        \item If $1 \in I$, then for any element $r \in R$, we must have $r = 1r \in I$ since $I$ is an ideal of $R$. Therefore $R \subseteq I$. But by definition of an ideal, $I \subseteq R$. Therefore $I = R$.
        
        \item For the forward direction, if $I$ contains a unit $u$, then there exists a $v \in R$ such that $uv = 1$. Note $uv \in I$ since $u \in I$ and $I$ is an ideal, so $1 \in I$. By \textbf{(i)} we have $I = R$.
        
        For the reverse direction, note that $1 \in R$ and so $1 \in I$ since $I = R$. Clearly 1 is a unit since $1\times1 = 1$. Therefore $I$ contains a unit.
    \end{partquestions}

    \item We consider the test for ideal (\myref{thrm-test-for-ideal}) to prove that $\ideal{a}\cap\ideal{b}$ is an ideal. We note that as $\ideal{a}$ and $\ideal{b}$ are ideals, they are therefore subrings of $R$. Thus, 0 is in both $\ideal{a}$ and $\ideal{b}$, meaning $0 \in \ideal{a}\cap\ideal{b}$. Hence $\ideal{a}\cap\ideal{b}$ is non-empty.
    
    Suppose $i,j\in\ideal{a}\cap\ideal{b}$, so $i,j \in \ideal{a}$ and $i,j \in \ideal{b}$. Note $\ideal{a}$ and $\ideal{b}$ are ideals and so are subrings, which means that $i-j \in \ideal{a}$ and $i-j \in \ideal{b}$, which hence means $i-j \in \ideal{a}\cap\ideal{b}$, satisfying the first statement for the test for ideals.

    Now suppose $r \in R$ and $i \in \ideal{a}\cap\ideal{b}$. This means that $i \in \ideal{a}$ and $i \in \ideal{b}$. So we have $ri, ir \in \ideal{a}$ (since $\ideal{a}$ is an ideal) and $ri, ir \in \ideal{b}$ (since $\ideal{b}$ is an ideal). Therefore $ri,ir \in \ideal{a}\cap\ideal{b}$, so by the test for ideal we have $\ideal{a}\cap\ideal{b}$ is an ideal.

    \item Let $R$ be a commutative ring with identity and $\princ{a}$ be a principal ideal of $R$. We note any element in $\princ{a}$ takes the form $ar$ for some element $r \in R$. We consider the test for ideal (\myref{thrm-test-for-ideal}) to prove this. Clearly $\princ{a}$ is non-empty as $0 = a0 \in \princ{a}$.
    
    Let $ar_1, ar_2 \in \princ{a}$. Clearly $ar_1 - ar_2 = a(r_1-r_2) \in \princ{a}$, so the first condition for the test for ideal is satisfied. Now take any $r \in R$ and let $ax \in \princ{a}$. Then $r(ax) = (ax)r = a(xr) \in \princ{a}$ since $R$ is commutative. Therefore by the test for ideal we have $\princ{a}$ is an ideal of $R$.

    \item \begin{partquestions}{\alph*}
        \item Clearly $\{0\} = \princ{0}$ since $0r = 0$ for any $r \in R$.
        \item Let 1 be the identity of $R$. Then $R = \{r \vert r \in R\} = \{1r \vert r \in R\} = \princ{1}$.
    \end{partquestions}

    \item We show that $\princ2$ is indeed a prime ideal of $\Z_8$. Without loss of generality, assume that $a \leq b$.
    \begin{itemize}
        \item If $ab = 0 \in \princ{2}$, then clearly $a = b = 0$ which is in $\princ{2}$.
        \item If $ab = 2 \in \princ{2}$, then $a = 1$ and $b = 2$. Note $b = 2 \in \princ{2}$.
        \item If $ab = 4 \in \princ{2}$, then $a = 1$ and $b = 4$ or $a = b = 2$. Note $2 \in \princ{2}$ and $4 \in \princ{2}$.
        \item If $ab = 6 \in \princ{2}$, then $a = 1$ and $b = 6$ or $a = 2$ and $b = 3$. Note $2 \in \princ{2}$ and $6 \in \princ{2}$.
    \end{itemize}
    In all cases, we see that if $ab \in \princ{2}$, then at least one of $a$ or $b$ is also in $\princ{2}$. Thus $\princ{2}$ is a prime ideal of $\Z_8$.

    \item Note that $\Z$ is a PID (\myref{prop-Z-is-PID}). We claim that $n$ has to be prime. By way of contradiction suppose $n$ is composite, meaning $n = ab$ where $2 \leq a,b < n$. Note $\princ{n} = \princ{ab} = \{\dots, -ab, 0, ab, \dots\}$. Observe that
    \begin{align*}
        \princ{a} &= \{\dots, -a(b+1), -ab, -a(b-1), \dots, -a,\\
        &\quad\quad0, a, \dots, a(b-1), ab, a(b+1), \dots\}
    \end{align*}
    so $\princ{n} \subset \princ{a}$. Similarly, $\princ{n} \subset \princ{b}$. However, as $\princ{n}$ is a maximal ideal, there does not exist a positive integer $k$ such that $\princ{n} \subset \princ{k} \subset \Z$. This contradicts the fact that we have both $\princ{n} \subset \princ{a}$ and $\princ{n} \subset \princ{b}$. Therefore, $n$ has to be prime.

    \item No. Let $I = \princ{3-i}$. Observe that
    \begin{align*}
        ((1+i)+I)((1-2i)+I) &= (1+i)(1-2i) + I\\
        &= (1-2i+i-2i^2) + I\\
        &= \underbrace{(3 - i)}_{\text{In }I} + I\\
        &= 0 + I
    \end{align*}
    so $((1+i)+I)$ and $((1-2i)+I)$ are zero divisors in $\Z[i]/I$. Therefore $\Z[i]/I$ is not an integral domain, meaning that $I$ is not a prime ideal (\myref{thrm-prime-ideal-iff-quotient-ring-is-integral-domain}).

    \item If $P$ is a prime ideal of $R$, then $R/P$ is an integral domain by \myref{thrm-prime-ideal-iff-quotient-ring-is-integral-domain}. Now $R$ is finite, meaning that $R/P$ is finite. Therefore, by \myref{thrm-finite-integral-domain-is-field}, $R/P$ is a field which therefore means that $P$ is maximal by \myref{thrm-maximal-ideal-iff-quotient-ring-is-field}.
    
    \item We consider the test for ideal (\myref{thrm-test-for-ideal}) to prove that $\Ann{R}{A}$ is an ideal of $R$. We note that $\Ann{R}{A}$ is non-empty since 0 is in $\Ann{R}{A}$ (because $0a = 0$ for any $a \in A$).

    Take any $r, s \in \Ann{R}{A}$, and an $a \in A$. Then one sees clearly that $(r-s)a = rs - sa = 0 - 0 = 0$ so $r-s \in \Ann{R}{A}$.

    Now take an $r \in \Ann{R}{A}$, an $a \in A$, and a $x \in R$. Note $(rx)a = (xr)a = x(ra) = x0 = 0$ since $R$ is commutative, which means that $rx, xr \in \Ann{R}{A}$.

    By the test for ideal, $\Ann{R}{A}$ is an ideal of $R$.
\end{questions}

\section{Ring Homomorphisms and Isomorphisms}
\begin{questions}
    \item We show that $\phi$ is not a ring homomorphism. Consider the matrices
    \[
        \begin{pmatrix}1&0\\0&0\end{pmatrix} \text{ and } \begin{pmatrix}0&0\\0&1\end{pmatrix}.
    \]
    Note that $\phi\left(\begin{pmatrix}1&0\\0&0\end{pmatrix}\right) = 1 + 0 = 1$ and $\phi\left(\begin{pmatrix}0&0\\0&1\end{pmatrix}\right) = 0 + 1 = 1$, so
    \[
        \phi\left(\begin{pmatrix}1&0\\0&0\end{pmatrix}\right)\phi\left(\begin{pmatrix}0&0\\0&1\end{pmatrix}\right) = 1 \times 1 = 1.
    \]
    However, note
    \[
        \begin{pmatrix}1&0\\0&0\end{pmatrix}\begin{pmatrix}0&0\\0&1\end{pmatrix} = \begin{pmatrix}0&0\\0&0\end{pmatrix}
    \]
    so $\phi\left(\begin{pmatrix}1&0\\0&0\end{pmatrix}\begin{pmatrix}0&0\\0&1\end{pmatrix}\right) = 0$. Thus $\phi$ is not a homomorphism.

    \item Note
    \[
        \phi(a+b) = 0 = 0 + 0 = \phi(a) + \phi(b)
    \]
    and
    \[
        \phi(ab) = 0 = 0\times0 = \phi(a)\phi(b)
    \]
    so $\phi$ is indeed a ring homomorphism.

    \item Note
    \[
        \id(a+b) = a + b = \id(a) + \id(b)
    \]
    and
    \[
        \id(ab) = ab = \id(a)\id(b)
    \]
    so $\id$ is a ring endomorphism.

    \item We have shown that the identity homomorphism is a homomorphism, so we just need to prove that it is a bijection.
    \begin{itemize}
        \item \textbf{Injective}: Suppose $a, b \in R$ are such that $\phi(a) = \phi(b)$. But since $\phi(x) = x$ thus $a = b$.
        \item \textbf{Surjective}: As $\phi(x) = x$ thus any element is its own pre-image.
    \end{itemize}
    Therefore the identity homomorphism is an isomorphism. As it is also an endomorphism, therefore it is an automorphism.

    \item Note that
    \[
        \phi(0_1) = \phi(0_1 + 0_1) = \phi(0_1) + \phi(0_1)
    \]
    so by 'adding' $-\phi(0_1)$ on both sides we see that $\phi(0_1) = 0_2$.

    \item Note that
    \[
        \phi(1_1) = \phi(1_1 \times 1_1) = \phi(1_1)\phi(1_1).
    \]
    Since $R_1$ and $R_2$ are division rings, we may apply $\phi(1_1)^{-1}$ on both sides to yield $\phi(1_1) = 1_2$.

    \item \begin{partquestions}{\alph*}
        \item Notice that
        \[
            \phi(x + (-x)) = \phi(x) + \phi(-x)
        \]
        and
        \[
            \phi(x + (-x)) = \phi(0_1) = 0_2
        \]
        so subtracting $-\phi(x)$ on both sides yields $\phi(-x) = -\phi(x)$.

        \item Notice that
        \[
            \phi(xx^{-1}) = \phi(x)\phi(x^{-1})
        \]
        and
        \[
            \phi(xx^{-1}) = \phi(1_1) = 1_2
        \]
        so applying $\phi(x)^{-1}$ on the left on both sides $\phi(x^{-1}) = \phi(x)^{-1}$.
    \end{partquestions}

    \item Recall that $\{0_2\}$, the trivial ideal, is an ideal of $R_2$, where $0_2$ is the additive identity of $R_2$. Therefore $\ker\phi = \phi^{-1}(\{0\})$ is an ideal of $R_1$ by \myref{prop-inverse-homomorphism-on-ideal-is-ideal}.

    \item We show that $\phi$ is surjective. Note that for any $k \in \Z_n$, we have $k \leq n$. Thus, $\phi(k) = k$, so $\phi$ is surjective.

    We now find the kernel of $\phi$.
    \begin{align*}
        \ker\phi &= \{m \in \Z \vert \phi(m) = 0\}\\
        &= \{m \in \Z \vert m \cong 0 \pmod n\}\\
        &= \{kn \vert k \in \Z\}\\
        &= n\Z.
    \end{align*}

    The FRIT (\myref{thrm-ring-isomorphism-1}) on $\phi$ tells us that
    \[
        \Z/n\Z \cong \Z_n.
    \]

    \item Note that $\phi(1) = 1$ is given. Now suppose $\phi(k) = k$ for some positive integer $k$. Then
    \[
        \phi(k+1) = \phi(k) + \phi(1) = k + 1
    \]
    by induction hypothesis and by the base case. Thus by mathematical induction we prove the statement.

    \item We borrow some calculation in the case where $\phi(1) = 1$ from \myref{example-endomorphisms-of-Z} to yield $\phi(n) = n$ for all integers $n$. Now note that for any positive integer $n$ we have
    \begin{align*}
        1 = \phi(1) &= \phi\left(\underbrace{\frac1n + \frac1n + \cdots + \frac1n}_{n \text{ times}}\right)\\
        &= \underbrace{\phi\left(\frac1n\right) + \phi\left(\frac1n\right) + \cdots + \phi\left(\frac1n\right)}_{n \text{ times}}\\
        &= n\phi\left(\frac1n\right)
    \end{align*}
    which means $\phi\left(\frac1n\right) = \frac1n$.

    Note that for any positive $\frac mn \in \Q$ with $m$ and $n$ as positive integers, we have
    \[
        \phi\left(\frac mn\right) = \phi(m)\phi\left(\frac1n\right) = m \times \frac1n = \frac mn
    \]
    and
    \[
        \phi\left(-\frac mn\right) = \phi(-m)\phi\left(\frac1n\right) = (-m) \times \frac1n = -\frac mn
    \]
    so $\phi(q) = q$ for any $q \in \Q$.
\end{questions}

\section{Polynomial Rings}
\begin{questions}
    \item Let $p(x), q(x) \in R$. Note
    \[
        \phi_a(p(x)+q(x)) = p(a) + q(a) = \phi_a(p(x)) + \phi_a(q(x))
    \]
    and
    \[
        \phi_a(p(x)q(x)) = p(a)q(a) = \phi_a(p(x))\phi_a(q(x))
    \]
    so $\phi_a$ is indeed a homomorphism.

    \item For simplicity let
    \begin{align*}
        f(x)(g(x)h(x)) &= \sum_{k=0}^{m+n+l}u_kx^k\\
        (f(x)g(x))h(x) &= \sum_{k=0}^{m+n+l}v_kx^k
    \end{align*}
    for some $u_k, v_k \in \R$.
    \begin{itemize}
        \item On one hand,
        \begin{align*}
            u_k &= \sum_{r+s=k}\left(a_r\left(\sum_{p+q=s}b_pc_q\right)\right) & (\text{Definition of polynomial multiplication})\\
            &= \sum_{r+s=k}\left(\sum_{p+q=s}a_rb_pc_q\right)\\
            &= \sum_{p+q+r=k}a_rb_pc_q\\
            &= \sum_{p+q+r=k}a_pb_qc_r & (\text{Order of }p,\;q,\;r \text{ is arbitrary})
        \end{align*}

        \item On another hand,
        \begin{align*}
            v_k &= \sum_{r+s=k}\left(\left(\sum_{p+q=r}a_pb_q\right)c_s\right) & (\text{Definition of polynomial multiplication})\\
            &= \sum_{r+s=k}\left(\sum_{p+q=r}a_pb_qc_s\right)\\
            &= \sum_{p+q+s=k}a_pb_qc_s\\
            &= \sum_{p+q+r=k}a_pb_qc_r & (s\text{ is dummy variable})
        \end{align*}
    \end{itemize}
    Therefore $u_k = v_k$ for all $k$, meaning that $f(x)(g(x)h(x)) = (f(x)g(x))h(x)$.

    \item We note
    \begin{align*}
        \Q[\sqrt[3]{2}] &= \left\{a_0 + a_1\sqrt[3]{2} + a_2\left(\sqrt[3]{2}\right)^2 + a_3\left(\sqrt[3]{2}\right)^3 + \cdots + a_n\left(\sqrt[3]{2}\right)^n \vert a_i \in \Q\right\}\\
        &= \left\{a_0 + a_1\sqrt[3]{2} + a_2\sqrt[3]{4} + 2a_3 + \cdots + a_n\left(\sqrt[3]{2}\right)^n \vert a_i \in \Q\right\}\\
        &= \left\{(a_0 + 2a_3 + \cdots) + (a_1 + 2a_4 + \cdots)\sqrt[3]{2} + (a_2 + 2a_5 + \cdots)\sqrt[3]{4} \vert a_i \in \Q\right\}\\
        &= \left\{a + b\sqrt[3]{2} + c\sqrt[3]{4} \vert a,b,c \in \Q\right\}
    \end{align*}
    which establishes the required result.

    \item One example would be $x^5$. Essentially any polynomial where the highest term is $x^5$ would work.
    
    \item Let $f(x) = x^2 - 1 \in \Z_4[x]$. Note that
    \begin{align*}
        f(0) &= 0^2 - 1 = -1 = 3 \neq 0,\\
        f(1) &= 1^2 - 1 = 0,\\
        f(2) &= 2^2 - 1 = 3 \neq 0, \text{ and}\\
        f(3) &= 3^3 - 1 = 8 = 0
    \end{align*}
    so the zeroes of $f(x)$ are 1 and 3.

    \item \begin{partquestions}{\alph*}
        \item Suppose $r \in R$. If $r = 0$, then it is immediately a constant polynomial of $R[x]$ by definition. Otherwise, may interpret $r$ as a degree 0 polynomial in $R[x]$, which means $r$ is a constant polynomial of $R[x]$.

        \item Suppose $f(x)$ is a constant polynomial in $R[x]$. If $f(x) = 0$ then clearly $f(x) \in R$. Otherwise it takes the form $f(x) = a_0$ for some $a_0 \in R$. Thus clearly $f(x) \in R$.
    \end{partquestions}
    
    \item \begin{partquestions}{\alph*}
        \item We first suppose $D$ is a ring with identity 1. We may see 1 as a degree 0 polynomial in $D[x]$. Now for any polynomial $f(x) = a_0+a_1x+a_2x^2+\cdots+a_nx^n$ in $D[x]$ we have
        \begin{align*}
            (1)(f(x)) &= (1)(a_0+a_1x+\cdots+a_nx^n)\\
            &= (1a_0)+(1a_1)x+\cdots+(1a_n)x^n\\
            &= a_0+a_1x+\cdots+a_nx^n\\
            &= f(x)
        \end{align*}
        and
        \begin{align*}
            (f(x))(1) &= (a_0+a_1x+\cdots+a_nx^n)(1)\\
            &= (a_{0}1)+(a_{1}1)x+\cdots+(a_{n}1)x^n\\
            &= a_0+a_1x+\cdots+a_nx^n\\
            &= f(x)
        \end{align*}
        so 1 is the identity in $D[x]$.

        Now suppose $D[x]$ is a ring with identity $\id(x)$. We see that $\id(x)f(x) = f(x)\id(x) = f(x)$ for any $f(x) \in D[x]$, meaning $\deg(\id(x)f(x)) = \deg(f(x))$. We note that $\deg(\id(x)f(x)) = \deg(\id(x)) + \deg(f(x))$, this means that $\id$ has degree 0, meaning we may write $\id(x) = e$ for some $e \in D$. Now if $f(x) = a$ for some $a \in D$, we must have $\id(x)f(x) = ae = a$ and $f(x)\id(x) = ea = a$, meaning that $e$ is the identity of $D$.
        
        \item Suppose first that $D$ is a commutative ring. Let
        \[
            f(x) = \sum_{i=0}^ma_ix^i \text{ and } g(x) = \sum_{j=0}^nb_jx^j
        \]
        be polynomials in $D[x]$. Then
        \begin{align*}
            f(x)g(x) &= \left(\sum_{i=0}^ma_ix^i\right)\left(\sum_{j=0}^nb_jx^j\right)\\
            &= \sum_{k=0}^{m+n}\left(\sum_{i=0}^k a_{i}b_{k-i}\right)x^k\\
            &= \sum_{k=0}^{m+n}\left(\sum_{i=0}^k b_{k-i}a_{i}\right)x^k\\
            &= \sum_{k=0}^{m+n}\left(\sum_{i=0}^k b_{i}a_{k-i}\right)x^k\\
            &= \left(\sum_{j=0}^nb_jx^j\right)\left(\sum_{i=0}^ma_ix^i\right)\\
            &= g(x)f(x)
        \end{align*}
        which therefore means that $D[x]$ is commutative.

        Now suppose $D[x]$ is commutative. Consider the polynomials $f(x) = a$ and $g(x) = b$ where $a$ and $b$ are non-zero. We thus have $ab = f(x)g(x) = g(x)f(x) = ba$ for all $a,b \in D$ which means $D$ is commutative.
    \end{partquestions}

    \item For brevity, let
    \begin{align*}
        f(x) &= \sum_{i=0}^ma_ix^i,\\
        g(x) &= \sum_{i=0}^nb_ix^i
    \end{align*}
    be polynomials in $R[x]$. Without loss of generality assume $m \geq n$, and define $b_i = 0$ for $i > n$. Then
    \begin{align*}
        \phi(f(x) + g(x)) &= \phi\left(\sum_{i=0}^m (a_i+b_i)x^i\right)\\
        &= \sum_{i=0}^m (a_i+b_i + I)x^i\\
        &= \sum_{i=0}^m ((a_i + I) + (b_i + I))x^i\\
        &= \sum_{i=0}^m (a_i + I)x^i + \sum_{i=0}^m (b_i + I)x^i\\
        &= \sum_{i=0}^m (a_i + I)x^i + \sum_{i=0}^n (b_i + I)x^i & (\text{since } b_i = 0\text{ for } i > n)\\
        &= \phi(f(x)) + \phi(g(x))
    \end{align*}
    and
    \begin{align*}
        \phi(f(x)g(x)) &= \phi\left(\sum_{i=0}^{m+n}\left(\sum_{j=0}^i a_jb_{i-j}\right)x^i\right)\\
        &= \sum_{i=0}^{m+n}\left(\left(\sum_{j=0}^i a_jb_{i-j}\right) + I\right)x^i\\
        &= \sum_{i=0}^{m+n}\left(\sum_{j=0}^i (a_jb_{i-j} + I)\right)x^i\\
        &= \sum_{i=0}^{m+n}\left(\sum_{j=0}^i ((a_j+I)(b_{i-j}+I))\right)x^i\\
        &= \left(\sum_{i=0}^m(a_i+I)x^i\right)\left(\sum_{i=0}^n(b_i+I)x^i\right)\\
        &= \phi(f(x))\phi(g(x))
    \end{align*}
    so $\phi$ is indeed a ring homomorphism.

    \item We note
    \begin{align*}
        &P \text{ is a prime ideal of }R\\
        \iff&R/P \text{ is an integral domain} & (\myref{thrm-prime-ideal-iff-quotient-ring-is-integral-domain})\\
        \iff&(R/P)[x] \text{ is an integral domain} & (\myref{thrm-integral-domain-iff-polynomial-ring-is-integral-domain})\\
        \iff&R[x]/P[x] \text{ is an integral domain} & (\myref{prop-polynomial-ring-quotient-ideal-polynomial-ring-cong-quotient-polynomial-ring})\\
        \iff&P[x]\text{ is a prime ideal of }R[x] & (\myref{thrm-prime-ideal-iff-quotient-ring-is-integral-domain})
    \end{align*}
    proving the theorem.

    \item Note that
    \begin{align*}
        3x^4 + 3x^3 + 4x^2 + 3x + 3 &= 3x^2(x^2+2x+3) - 3x(x^2+2x+3) + 1(x^2+2x+3) + 10x\\
        &= (3x^2-3x+1)(x^2+2x+3) + 10x\\
        &= (3x^2+2x+1)(x^2+2x+3) + 0x & (\text{Coefficients in }\Z_5)\\
        &= (3x^2+2x+1)(x^2+2x+3)
    \end{align*}
    so dividing $3x^4 + 3x^3 + 4x^2 + 3x + 3$ by $x^2+2x+3$ yields $3x^2+2x+1$. Thus two factors of $3x^4 + 3x^3 + 4x^2 + 3x + 3$ are $x^2+2x+3$ and $3x^2+2x+1$.
\end{questions}


\chapter{Problem Solutions}
\section{Introduction to Rings}
\subsection*{Exercises}
\begin{questions}
    \item We prove the ring axioms.
    \begin{itemize}
        \item \textbf{Addition-Abelian}: $(\{0\}, +)$ is an abelian group since this is just the trivial group.
        \item \textbf{Multiplication-Semigroup}: $(\{0\}, \cdot)$ is an abelian group since this is, again, just the trivial group. So $(\{0\}, \cdot)$ is a semigroup.
        \item \textbf{Distributive}: We know $+$ and $\cdot$ distribute.
    \end{itemize}
    Hence $(\{0\}, +, \cdot)$ is a commutative ring with identity.
\end{questions}

\subsection*{Problems}
No problems.

\section{Basics of Rings}
\begin{questions}
    \item We note the following.
    \begin{itemize}
        \item \textbf{Addition-Abelian}: $(\Z, +)$ is an abelian group.
        \item \textbf{Multiplication-Semigroup}: $(\Z, \times)$ is a semigroup since
        \begin{itemize}
            \item multiplying two integers always results in an integer, so $\Z$ is closed under $\times$; and
            \item $\times$ is associative.
        \end{itemize}
        \item \textbf{Distributive}: We know $+$ and $\times$ distribute.
    \end{itemize}
    Hence $(\Z, +, \times)$ is a ring.

    \item Consider $(-a)(-b) + (-ab)$ and note
    \begin{align*}
        &(-a)(-b) + (-ab)\\
        &= (-a)(-b) + (-a)b & (\text{\myref{prop-product-of-element-and-additive-inverse-is-additive-inverse-of-product}})\\
        &= (-a)(-b + b) & (\text{by \textbf{Distributive} axiom})\\
        &= (-a)0\\
        &= 0 & (\myref{prop-multiplying-by-zero-is-zero})
    \end{align*}
    which means $(-a)(-b) = -(-ab) = ab$ as required.

    \item The ring $\Mn{2}{\mathbb{R}}$ indeed has zero divisors, as $\begin{pmatrix}0&1\\0&0\end{pmatrix} \neq \begin{pmatrix}0&0\\0&0\end{pmatrix}$ but $\begin{pmatrix}0&1\\0&0\end{pmatrix}^2 = \begin{pmatrix}0&0\\0&0\end{pmatrix}$ which means that $\begin{pmatrix}0&1\\0&0\end{pmatrix}$ is a zero divisor.

    \item \begin{partquestions}{\alph*}
        \item $\Z$ is not a field. Note that the multiplicative inverse of 2 is $\frac12$ which is not an integer. Hence not all non-zero elements in $\Z$ has a multiplicative inverse, meaning that not all non-zero elements are units.

        \item $\Q$ is a field. Note for any rational number $\frac ab$ (where $b \neq 0$) it has an inverse of $\frac ba$. Thus any non-zero rational number is a unit, which means $\Q$ is a division ring. Since $\Q$ is also a commutative ring, therefore $\Q$ is a field.
    \end{partquestions}

    \item Let $u$ and $v$ be units, meaning that $u^{-1}$ and $v^{-1}$ exist. Then one sees that $(uv)(v^{-1}u^{-1}) = (v^{-1}u^{-1})(uv) = 1$, which means that $uv$ is also a unit.

    \item We first show that $(R, +) \leq (\Mn{2}{\R}, +)$.
    \begin{itemize}
        \item Clearly the identity of $(\Mn{2}{\R}, +)$, the zero matrix $\begin{pmatrix}0&0\\0&0\end{pmatrix}$, is inside $R$.
        \item Consider $\begin{pmatrix}a&a\\a&a\end{pmatrix}, \begin{pmatrix}b&b\\b&b\end{pmatrix} \in R$. The additive inverse of the matrix $\begin{pmatrix}b&b\\b&b\end{pmatrix}$ is the matrix $\begin{pmatrix}-b&-b\\-b&-b\end{pmatrix}$, and so their sum is
        \[
            \begin{pmatrix}a&a\\a&a\end{pmatrix} + \begin{pmatrix}-b&-b\\-b&-b\end{pmatrix} = \begin{pmatrix}a-b&a-b\\a-b&a-b\end{pmatrix} \in R
        \]
        which means $R$ is closed under addition.
    \end{itemize}
    Hence $(R, +) \leq (\Mn{2}{\R}, +)$ by subgroup test.

    We now show that $R$ is closed under multiplication. Some calculation yields that
    \[
        \begin{pmatrix}a&a\\a&a\end{pmatrix}\begin{pmatrix}b&b\\b&b\end{pmatrix} = \begin{pmatrix}2ab&2ab\\2ab&2ab\end{pmatrix}
    \]
    which is clearly in $R$. Therefore $R$ is a subring of $\Mn{2}{\R}$.
\end{questions}

\section{Integral Domains}
\begin{questions}
    \item To find a $a+bi \in \Z_5[i]$ such that there exists a $c+di \in \Z_5[i]$ where $(a+bi)(c+di) = 0$ but both $a+bi$ and $c+di$ are non-zero. Expanding $(a+bi)(c+di)$ yields $(ac-bd)+(ad+bc)i = 0$. Therefore we must have $ac-bd = 0$ and $ad+bc = 0$. For simplicity let's choose $a=c=1$. Using second equation we have $d+b = 0$ which means $d = -b$. Hence $(1 - b(-b))+(-b + b)i = 1+b^2 = 0$. Therefore choosing $b = 2$ would make it work. Therefore one solution is $a = 1, b = 2, c = 1, d = -2 = 3$; i.e. two zero divisors are $1+2i$ and $1+3i$.
    
    \item Take $w, z \in \Z[i]$ such that $w \neq 0$ and $wz = 0$. We want to show that $z = 0$. Let $z = a+bi$ and $w = c+di$. Since $w \neq 0$ we must have $c^2+d^2 \neq 0$. Now
    \[
        (a+bi)(c+di) = (ac-bd)+(ad+bc)i = 0
    \]
    which means $ac - bd = 0$ and $ad+bc = 0$. Multiplying first equation by $d$ yields $acd - bd^2 = 0$; multiplying second equation by $c$ yields $acd + bc^2 = 0$. Now summing them up yields $bc^2+bd^2 = b(c^2+d^2) = 0$ which hence means $b = 0$ since $c^2+d^2 \neq 0$. Therefore $ac - 0d = 0$ implies $ac = 0$ and $ad+0c = 0$ implies $ad = 0$. Squaring both equations and adding them up yields $a^2c^2 + a^2d^2 = a^2(c^2+d^2) = 0$ which hence means $a^2$ (and thus $a$) is zero. Therefore we have shown $z = 0$, meaning that there are no zero divisors in $\Z[i]$, so $\Z[i]$ is an integral domain.

    \item \begin{partquestions}{\alph*}
        \item Note that multiplication is commutative with identity $1 = 1 + 0\sqrt{n} \in R$. We just need to show that there are no zero divisors in $R$.
        
        Take $a+b\sqrt n, c+d\sqrt n \in R$ such that $a+b\sqrt n \neq 0$ but $(a+b\sqrt n)(c+d\sqrt n) = 0$. We want to show $c = d = 0$. Consider
        \[
            \left((a+b\sqrt n)(\underbrace{a-b\sqrt n}_{\neq 0})\right)\left((c+d\sqrt n)(\underbrace{c-d\sqrt n}_{\neq 0})\right) = 0.
        \]
        This means that $(a^2-nb^2)(c^2-nd^2) = 0$, so either $a^2-nb^2 = 0$ or $c^2-nd^2 = 0$.

        Now if $n < 0$ then clearly we have to have $c = d = 0$. Otherwise we have $a = b\sqrt n$ or $c = d\sqrt n$. But $\sqrt n$ is not an integer, so the only way for equality is if $c = d = 0$. Thus $\Z[\sqrt n]$ has no zero divisors, meaning $\Z[\sqrt n]$ is an integral domain.

        \item Consider $2 + \sqrt 2 \in \Z[\sqrt 2]$. Its multiplicative inverse is
        \begin{align*}
            \frac{1}{2+\sqrt2} &= \frac{2-\sqrt2}{(2+\sqrt2)(2-\sqrt2)}\\
            &= \frac{2-\sqrt2}{4-2}\\
            &= 1 - \frac12\sqrt2 \notin \Z[\sqrt2].
        \end{align*}
        This means that $2+\sqrt2$, a non-zero element in $\Z[\sqrt2]$, does not have an inverse in $\Z[\sqrt2]$. Therefore $\Z[\sqrt2]$ is not a field, meaning $R$ is not a field in the general case.
    \end{partquestions}

    \newpage

    \item For brevity let O$ = \begin{pmatrix}0&0\\0&0\end{pmatrix}$, I$ = \begin{pmatrix}1&0\\0&1\end{pmatrix}$, A$ = \begin{pmatrix}1&1\\1&0\end{pmatrix}$, and B$ = \begin{pmatrix}0&1\\1&1\end{pmatrix}$.
    
    \begin{partquestions}{\roman*}
        \item Clearly one sees that $R$ is a subset of $\Mn{2}{\Z_2}$.
        \begin{itemize}
            \item We show $(R, +)\leq(\Mn{2}{\Z_2},+)$.
            \begin{table}[h]
                \centering
                \begin{tabular}{|l|l|l|l|l|}
                    \hline
                    \textbf{+} & \textbf{O} & \textbf{I} & \textbf{A} & \textbf{B} \\ \hline
                    \textbf{O} & O          & I          & A          & B          \\ \hline
                    \textbf{I} & I          & O          & B          & A          \\ \hline
                    \textbf{A} & A          & B          & O          & I          \\ \hline
                    \textbf{B} & B          & A          & I          & O          \\ \hline
                \end{tabular}
            \end{table}
            
            From the Cayley table, clearly the identity of the ring $\Mn{2}{\Z_2}$ is in $R$ and $R$ is closed under addition. Hence $(R, +)\leq(\Mn{2}{\Z_2},+)$

            \item We show $R$ is closed under multiplication.
            \begin{table}[h]
                \centering
                \begin{tabular}{|l|l|l|l|l|}
                    \hline
                    $\boldsymbol{\cdot}$ & \textbf{O} & \textbf{I} & \textbf{A} & \textbf{B} \\ \hline
                    \textbf{O}           & O          & O          & O          & O          \\ \hline
                    \textbf{I}           & O          & I          & A          & B          \\ \hline
                    \textbf{A}           & O          & A          & B          & I          \\ \hline
                    \textbf{B}           & O          & B          & I          & A          \\ \hline
                \end{tabular}
            \end{table}
            
            From the Cayley table, clearly $R$ is closed under multiplication.
        \end{itemize}
        Therefore $R$ is a subring of $\Mn{2}{\Z_2}$.

        \item Since $R$ is a subring of $\Mn{2}{\Z_2}$, it is a ring. Furthermore, by the Cayley table of $(R, \cdot)$, we see that $R$ is commutative with identity I. Finally, one sees that $\mathrm{A}^{-1} = \mathrm{B}$, $\mathrm{B}^{-1} = \mathrm{A}$, and $\mathrm{I}^{-1} = \mathrm{I}$. Therefore all non-zero elements of $R$ have inverses. Hence $R$ is a field.
    \end{partquestions}
\end{questions}

\section{Ideals and Quotient Rings}
\begin{questions}
    \item Note that $36 = 2^2 \times 3^2$. So
    \begin{align*}
        \Ann{\Z_{36}}{\{15\}} &= \{r \in \Z_{36} \vert 15r = 0\}\\
        &= \{r \in \Z_{15} \vert 3(5r) = 0\}\\
        &= \{r \in \Z_{15} \vert r \text{ is a multiple of }2^2\times3 = 12\}\\
        &= \{0,12,24\}.
    \end{align*}

    \item We first show that $S$ is a subring of $\Z[i]$.
    \begin{itemize}
        \item The identity of $\Z[i]$ is $0 = 0 + 2(0)i \in S$.
        \item For any $a+2bi, c+2di \in S$, clearly $a+2bi + (-(c + 2di)) = (a-c) + 2(b-d)i \in S$.
        \item For any $a+2bi, c+2di \in S$, one sees that
        \begin{align*}
            (a+2bi)(c+2di) &= ac + 2adi + 2bci + 4bdi^2\\
            &= (ac-4bd) + 2(ad+bc)i\\
            &\in S.
        \end{align*}
    \end{itemize}
    Therefore $S$ is a subring of $\Z[i]$.

    We now show that $S$ is not an ideal of $\Z[i]$. Consider $1+2i \in \S$ and $1+i \in \Z[i]$. Then
    \begin{align*}
        (1+2i)(1+i) &= 1+i+2i+2i^2\\
        &= -1 + 3i\\
        &\notin S
    \end{align*}
    so there exists a $s \in S$ and a $r \in \Z[i]$ such that $rs\notin S$, meaning that $S$ is not a left ideal (and hence is not an ideal).

    \item We consider the test for ideal (\myref{thrm-test-for-ideal}).
    \begin{itemize}
        \item Note that $\begin{pmatrix}0&0\\0&0\end{pmatrix}=\begin{pmatrix}2(0)&2(0)\\2(0)&(0)\end{pmatrix}$ is in $I$ so $I$ is non-empty.
        \item $\begin{pmatrix}2a&2b\\2c&2d\end{pmatrix}-\begin{pmatrix}2e&2f\\2g&2h\end{pmatrix} = \begin{pmatrix}2(a-e)&2(b-f)\\2(c-g)&2(d-h)\end{pmatrix} \in I$.
        \item To show left ideal, take $\begin{pmatrix}2a&2b\\2c&2d\end{pmatrix} \in I$ and $\begin{pmatrix}e&f\\g&h\end{pmatrix} \in \Mn{2}{\Z}$. Then
        \begin{align*}
            \begin{pmatrix}2a&2b\\2c&2d\end{pmatrix}\begin{pmatrix}e&f\\g&h\end{pmatrix} &= \begin{pmatrix}2ae+2bg&2af+2bh\\2ce+2dg&2cf+2dh\end{pmatrix}\\
            &= \begin{pmatrix}2(ae+bg)&2(af+bh)\\2(ce+dg)&2(cf+dh)\end{pmatrix}\\
            &\in I
        \end{align*}
        so $I$ is a left ideal of $\Mn{2}{\Z}$.
        \item To show right ideal, take $\begin{pmatrix}a&b\\c&d\end{pmatrix} \in \Mn{2}{\Z}$ and $\begin{pmatrix}2e&2f\\2g&2h\end{pmatrix} \in I$. Then
        \begin{align*}
            \begin{pmatrix}a&b\\c&d\end{pmatrix}\begin{pmatrix}2e&2f\\2g&2h\end{pmatrix} &= \begin{pmatrix}2ae+2bg&2af+2bh\\2ce+2dg&2cf+2dh\end{pmatrix}\\
            &= \begin{pmatrix}2(ae+bg)&2(af+bh)\\2(ce+dg)&2(cf+dh)\end{pmatrix}\\
            &\in I
        \end{align*}
        so $I$ is a right ideal of $\Mn{2}{\Z}$.
    \end{itemize}
    Therefore by the test for ideal we have $I$ is an ideal of $\Mn{2}{\Z}$.

    \item \begin{partquestions}{\alph*}
        \item Suppose $I$ is not the trivial ring; we want to show that $I = R$. Since $I$ is non-trivial there there exists a non-zero element $a$ in $I$. Note that $a^{-1}$ exists since $R$ is a field, so $a$ is a unit. By \myref{exercise-ideal-containing-1-is-whole-ring} this means $I = R$. Note that $\{0\} = \princ{0}$ and $R = \princ{1}$ by \myref{exercise-trivial-ideal-and-whole-ring-are-principal-ideals}, so $R$ is indeed a PID.

        \item Take a non-zero $x \in R$ and note that $\princ{x}$ is a non-trivial ideal. Since there are no proper ideals in $R$, thus $\princ{x} = R$. This means that $1 \in \princ{x}$ (since $\princ{x} = R$ is a ring with identity), meaning that there exists an element $r \in R$ such that $xr = 1$. Therefore $x$ is a unit.
        
        Since $x$ is an arbitrary non-zero element in $R$, this thus shows that all non-zero elements of the ring $R$ are units, meaning $R$ is a division ring.

        Finally, because $R$ is commutative, thus $R$ is a field.
    \end{partquestions}

    \item \begin{partquestions}{\alph*}
        \item Suppose $r \in \sqrt{\sqrt{I}}$, meaning that $r^m \in \sqrt{I}$ for some positive integer $m$, further meaning that $(r^m)^n \in I$ for some positive integer $n$. Note $(r^m)^n = r^{mn} \in I$, so $r \in \sqrt{I}$. Therefore $\sqrt{\sqrt{I}} \subseteq \sqrt{I}$.
        
        Now suppose $r \in \sqrt{I}$, meaning that $r^n \in I$ for some positive integer $n$. Note that $r = r^1 \in \sqrt{I}$, so $r \in \sqrt{\sqrt{I}}$. Hence $\sqrt{I} \subseteq \sqrt{\sqrt{I}}$.

        Therefore, since $\sqrt{\sqrt{I}} \subseteq \sqrt{I}$ and $\sqrt{I} \subseteq \sqrt{\sqrt{I}}$, thus $\sqrt{\sqrt{I}} = \sqrt{I}$.

        \item Suppose $r \in \sqrt{I\cap J}$, so $r^n \in I \cap J$ for some positive integer $n$. This means that $r^n \in I$ and $r^n \in J$. Hence $r \in \sqrt{I}$ and $r \in \sqrt{J}$ by definition of the radical, so $r \in \sqrt{I}\cap\sqrt{J}$. Thus $\sqrt{I\cap J} \subseteq \sqrt{I}\cap\sqrt{J}$.
        
        Now suppose $r \in \sqrt{I}\cap\sqrt{J}$, meaning that $r \in \sqrt{I}$ and $r \in \sqrt{J}$. Thus $r^m \in I$ and $r^n \in J$ for some positive integers $m$ and $n$. Note that
        \[
            (\underbrace{r^m}_{\text{In }I})^n \in I \text{ and } (\underbrace{r^n}_{\text{In }J})^m \in J
        \]
        so $r^{mn} \in I$ and $r^{mn} \in J$, meaning $r^{mn} \in I \cap J$. Thus $r \in \sqrt{I \cap J}$, showing that $\sqrt{I}\cap\sqrt{J} \subseteq \sqrt{I\cap J}$.

        Therefore $\sqrt{I}\cap\sqrt{J} = \sqrt{I\cap J}$.
    \end{partquestions}

    \item \begin{partquestions}{\alph*}
        \item Suppose $a \in m\Z\cap n\Z$. Thus $a \in m\Z$ and $a \in n\Z$, meaning that $a = mx = ny$ for some integers $x$ and $y$. Therefore $a = \lcm(m,n)z = lz$ for some integer $z$, meaning $a \in l\Z$. Hence $m\Z \cap n\Z \subseteq l\Z$.
        
        Now suppose $a \in l\Z$, so $a = lx$ for some integer $x$. Write $l = m\alpha = n\beta$ for some integers $\alpha$ and $\beta$. Note that
        \begin{align*}
            a &= (m\alpha)x = m(\alpha x) \in m\Z\\
            a &= (n\beta)x = n(\beta x) \in n\Z
        \end{align*}
        so $a \in m\Z \cap n\Z$. Thus $l\Z \subseteq m\Z \cap n\Z$.

        Therefore $m\Z\cap n\Z = l\Z$.

        \item Suppose $a \in m\Z + n\Z$, meaning that there exist integers $x$ and $y$ such that $a = mx + ny$. By definition of the GCD, write $m = d\alpha$ and $n = d\beta$ for some integers $\alpha$ and $\beta$. Hence
        \begin{align*}
            a &= (d\alpha)x + (d\beta)y\\
            &= d(\alpha x + \beta y)\\
            &\in d\Z
        \end{align*}
        so $m\Z + n\Z \subseteq d\Z$.

        On the other hand, suppose $a \in d\Z$, meaning $a = dt$ for some integer $t$. By B\'{e}zout's Lemma (\myref{lemma-bezout}), we may write $d = mx + ny$ for some integers $x$ and $y$. Hence
        \begin{align*}
            a &= (mx + ny)t\\
            &= m(xt) + n(yt)\\
            &\in m\Z + n\Z
        \end{align*}
        which means $d\Z \subseteq m\Z + n\Z$.

        Therefore $m\Z + n\Z = d\Z$.
    \end{partquestions}

    \item Let $r \in R$, and suppose $x = r + \Nilr{R} \in R/\Nilr{R}$ is nilpotent, i.e. there is a positive integer $n$ such that
    \[
        x^n = (r + \Nilr{R})^n = r^n + \Nilr{R} = 0 + \Nilr{R}.
    \]
    Coset Equality (\myref{lemma-coset-equality}) thus tells us that $r^n \in \Nilr{R}$. Note that $\Nilr{R}$ contains all the nilpotents of $R$. Thus $r^n$ is a nilpotent of $R$, i.e. there exists a positive integer $m$ such that $(r^n)^m = 0$. But clearly $(r^n)^m = r^{mn} = 0$, so $r$ is nilpotent, meaning $r \in \Nilr{R}$. Hence $x = r + \Nilr{R} = 0 + \Nilr{R}$, meaning that the only nilpotent of $R/\Nilr{R}$ is the zero element. Therefore $R/\Nilr{R}$ has no non-zero nilpotents.

    \item Suppose $R$ is a PID and $I$ is a non-zero prime ideal. Let $J$ be an ideal such that $I \subseteq J \subseteq R$. Since $R$ is a PID, write $I = \princ{a}$ and $J = \princ{b}$ for some elements $a$ and $b$ in $R$. Note $a \in \princ{a} = I \subseteq J = \princ{b}$, so there exists an $r \in R$ such that $a = rb$. Now since $a = rb \in \princ{a} = I$ and $I$ is prime, therefore $r \in I$ or $b \in I$.
    \begin{itemize}
        \item If $r \in I$, write $r = sa$ for some $s \in R$. Then
        \[
            a = rb = (sa)b = (as)b = a(sb)
        \]
        since an integral domain is commutative. Thus $a - a(sb) = a(1-sb) = 0$. Now as $R$ is an integral domain thus either $a = 0$ (impossible since $a \neq 0$) or $1-sb = 0$. So $1-sb = 0$, meaning $sb = 1 \in J$ since $b \in J$. By \myref{exercise-ideal-containing-1-is-whole-ring} we have $J = R$.
        \item If instead $b \in I$, take any $x \in J = \princ{b}$, so $x = rb$ for some $r \in R$. Thus $x = rb \in I$ since $b \in I$, so $J \subseteq I$. But $I \subseteq J$, so $J = I$.
    \end{itemize}
    Therefore we have shown that $I$ is maximal.

    \item First we work in the forward direction. Suppose $\princ{a} = \princ{b}$. As $a \in \princ{a} = \princ{b}$, thus $a = bx$ for some $x \in R$. Also, as $b \in \princ{b} = \princ{a}$, thus $b = ay$ for some $y \in R$. Therefore
    \[
        b = ay = (bx)y = b(xy)
    \]
    which means $xy = 1$. Thus $x$ and $y$ are units, meaning $a = bx$ with $x$ being a unit.

    Now we work in the reverse direction; suppose $a = bu$ for some unit $u$ in $D$.
    \begin{itemize}
        \item Take $r \in \princ{a}$, so $r = ax$ for some $x$ in $D$. Thus $r = (bu)x = b(ux) \in \princ{b}$, so $\princ{a} \subseteq \princ{b}$.
        \item Note $b = au^{-1}$ since $u$ is a unit. Take $s \in \princ{b}$, so $s = by$ for some $y$ in $D$. But as $b = au^{-1}$, hence $s = (au^{-1})y = a(u^{-1}y) \in \princ{a}$, so $\princ{b} \subseteq \princ{a}$.
    \end{itemize}
    Therefore we see that $\princ{a} = \princ{b}$.
\end{questions}

\chapter{Ring Homomorphisms and Isomorphisms}
Like with groups, rings too have homomorphisms and isomorphisms, although they are defined slightly differently than in groups. Similar to how group homomorphisms preserve some structure between the two groups, ring homomorphisms and isomorphisms also preserve structure between rings.

\section{Ring Homomorphisms and Isomorphisms}
\begin{definition}
    Let $(R_1, +, \cdot)$ and $(R_2, \oplus, \otimes)$ be rings. A map $\phi: R_1 \to R_2$ is a \term{ring homomorphism}\index{homomorphism!ring} if and only if for all $a, b \in R_1$, we have
    \begin{align*}
        \phi(a+b) &= \phi(a) \oplus \phi(b) \text{ and}\\
        \phi(a\cdot b) &= \phi(a)\otimes\phi(b).
    \end{align*}
\end{definition}
\begin{remark}
    Like with group homomorphisms, we usually use ``$+$'' for both addition operations and suppress the multiplication operation. That is, the ring homomorphism conditions become
    \begin{align*}
        \phi(a+b) &= \phi(a) + \phi(b) \text{ and}\\
        \phi(ab) &= \phi(a)\phi(b).
    \end{align*}
\end{remark}

\begin{example}
    We show that the map $\phi: \Z \to \Z/n\Z, x \mapsto x + n\Z$ is a ring homomorphism. Let $a, b \in \Z$. Note
    \begin{align*}
        \phi(a+b) &= (a+b) + n\Z\\
        &= (a + n\Z) + (b + n\Z) & (\text{Definition of coset addition})\\
        &=\phi(a)+\phi(b)
    \end{align*}
    and
    \begin{align*}
        \phi(ab) &= ab + n\Z\\
        &= (a + n\Z)(b + n\Z) & (\text{Definition of coset multiplication})\\
        &= \phi(a)\phi(b)
    \end{align*}
    so $\phi$ is a homomorphism.
\end{example}

\begin{example}
    Consider the ring
    \[
        R = \left\{\begin{pmatrix}a&b\\0&c\end{pmatrix}\vert a,b,c\in\Z\right\}.
    \]
    The map $\phi: R \to \Z^2, \begin{pmatrix}a&b\\0&c\end{pmatrix} \mapsto (a,c)$ is a ring homomorphism since, for any $\begin{pmatrix}a&b\\0&c\end{pmatrix}, \begin{pmatrix}x&y\\0&z\end{pmatrix} \in R$, we have
    \begin{align*}
        \phi\left(\begin{pmatrix}a&b\\0&c\end{pmatrix} + \begin{pmatrix}x&y\\0&z\end{pmatrix}\right) &= \phi\left(\begin{pmatrix}a+x&b+y\\0&c+z\end{pmatrix}\right)\\
        &= (a+x,c+z)\\
        &= (a,c) + (x,z)\\
        &= \phi\left(\begin{pmatrix}a&b\\0&c\end{pmatrix}\right) + \phi\left(\begin{pmatrix}x&y\\0&z\end{pmatrix}\right)
    \end{align*}
    and
    \begin{align*}
        \phi\left(\begin{pmatrix}a&b\\0&c\end{pmatrix}\begin{pmatrix}x&y\\0&z\end{pmatrix}\right) &= \phi\left(\begin{pmatrix}ax&ay+bz\\0&cz\end{pmatrix}\right)\\
        &= (ax, cz)\\
        &= (a,c)(x,z)\\
        &= \phi\left(\begin{pmatrix}a&b\\0&c\end{pmatrix}\right)\phi\left(\begin{pmatrix}x&y\\0&z\end{pmatrix}\right).
    \end{align*}
\end{example}
\begin{exercise}
    Let the function $\phi: \Mn{2}{\Z} \to \Z$ be defined such that
    \[
        \phi\left(\begin{pmatrix}a&b\\c&d\end{pmatrix}\right) = a+d.
    \]
    Is $\phi$ a ring homomorphism?
\end{exercise}
\begin{exercise}
    Let $R$ and $S$ be rings with additive identities $0_R$ and $0_S$ respectively. Show that the \term{trivial homomorphism}\index{homomorphism!trivial} $\phi: R \to S, r \mapsto 0_S$ is, indeed, a ring homomorphism.
\end{exercise}

\pagebreak

An endomorphism is a specific type of homomorphism.
\begin{definition}
    A \term{ring endomorphism}\index{endomorphism!ring} of a ring $R$ is a homomorphism $\phi: R \to R$.
\end{definition}
\begin{example}
    Let $R$ be a commutative ring with prime characteristic $p$. The \term{Frobenius endomorphism}\index{Frobenius endomorphism} $\phi: R \to R$ is such that $\phi(r) = r^p$. We show that $\phi$ is a ring endomorphism.

    Note that for any $a, b \in R$ that
    \begin{align*}
        \phi(a+b) &= (a+b)^p\\
        &= a^p + pa^{p-1}b + {p \choose 2}a^{p-2}b^2 + \cdots + pab^{p-1} + b^p.
    \end{align*}
    Note that the binomial coefficients ${p \choose k}$ where $1 \leq k \leq p-1$ are all multiples of $p$ (\myref{prop-binomial-coefficient-multiple-of-p}). As the characteristic of the ring $R$ is $p$, thus $px = 0$ for any $x \in R$. Therefore,
    \begin{align*}
        \phi(a+b) = &a^p + pa^{p-1}b + {p \choose 2}a^{p-2}b^2 + \cdots + pab^{p-1} + b^p\\
        &= a^p + 0 + 0 + \cdots + 0 + b^p\\
        &= a^p + b^p\\
        &=\phi(a) + \phi(b).
    \end{align*}

    Also,
    \[
        \phi(ab) = (ab)^p = a^pb^p = \phi(a)\phi(b).
    \]
    Therefore $\phi$ is a ring endomorphism.
\end{example}

\begin{exercise}
    Let $R$ be a ring. Show that the \term{identity homomorphism}\index{homomorphism!identity} $\id: R \to R, r \mapsto r$ is a ring endomorphism.
\end{exercise}

The definition of ring isomorphisms is analogous to that of group isomorphisms.

\begin{definition}
    A \term{ring isomorphism}\index{isomorphism!ring} is a bijective ring homomorphism.
\end{definition}

Similar to groups, we write $R_1 \cong R_2$ if and only if $R_1$ and $R_2$ are (ring) isomorphic to each other.

\begin{example}\label{example-Zn-ring-isomorphic-to-Z/nZ}
    We show that $\Z_n \cong \Z/n\Z$. Consider the map $\phi:\Z_n \to \Z/n\Z$ where $m \mapsto m + n\Z$. We show that $\phi$ is an isomorphism.
    \begin{itemize}
        \item \textbf{Homomorphism}: For any $a, b \in \Z_n$ we see
        \[
            \phi(a+b) = (a+b) + n\Z = (a + n\Z) + (b + n\Z) = \phi(a) + \phi(b)
        \]
        and
        \[
            \phi(ab) = (ab) + n\Z = (a+n\Z)(b+n\Z) = \phi(a)\phi(b).
        \]

        \item \textbf{Injective}: Suppose $a, b \in \Z_n$ such that $\phi(a) = \phi(b)$. This means $a + n\Z = b + n\Z$, i.e. $a \equiv b \pmod n$. Now note that $0 \leq a,b < n$ so we have $a = b$.

        \item \textbf{Surjective}: Suppose $m + n\Z \in \Z/n\Z$. Applying Euclid's division lemma (\myref{lemma-euclid-division}) on $m$ we have $m = nq + r$ with $0 \leq r < n$. One sees that
        \begin{align*}
            \phi(r) &= r + n\Z\\
            &= r + (nq + n\Z)\\
            &= (r + nq) + n\Z\\
            &= m + n\Z
        \end{align*}
        so $m + n\Z$ has a pre-image of $r$ in $\Z_n$.
    \end{itemize}
    Since $\phi$ is a bijective ring homomorphism, thus $\phi$ is an isomorphism, meaning $\Z_n \cong \Z/n\Z$ as rings.
\end{example}
\begin{example}
    We show that $\Z \not\cong 2\Z$ as rings. Suppose $\phi: \Z \to 2\Z$ is a ring isomorphism. Set $a = \phi(1) = 2\Z$. Note that
    \[
        a = \phi(1) = \phi(1\times1) = (\phi(1))^2 = a^2
    \]
    so $a^2 = a$, which means $a = 0$ or $a = 1$. But as $a \in 2\Z$, thus $a \neq 1$ which means $a = 0$.

    But notice for any $n \in \Z$ we have
    \begin{align*}
        \phi(n) &= \phi(n1)\\
        &= \phi(n)\phi(1)\\
        &= \phi(n) \times 0\\
        &= 0.
    \end{align*}
    Thus one sees that $\phi(0) = \phi(1) = 0$ which means $\phi$ is not injective, a contradiction.
\end{example}

\begin{definition}
    A bijective ring endomorphism is called a \term{ring automorphism}\index{automorphism!ring}.
\end{definition}

\begin{exercise}\label{exercise-identity-homomorphism-is-an-isomorphism}
    Show that the identity homomorphism is actually an automorphism.
\end{exercise}

\section{Properties of Ring Homomorphisms}
For the following, let $R_1$ and $R_2$ be rings with additive identities $0_1$ and $0_2$ respectively. Also let $\phi: R_1 \to R_2$ be a ring homomorphism.

\begin{proposition}\label{prop-ring-image-of-additive-identity-is-additive-identity}
    $\phi(0_1) = 0_2$.
\end{proposition}
\begin{proof}
    See \myref{exercise-ring-image-of-identity-is-identity} (later).
\end{proof}

\begin{proposition}
    If $R_1$ and $R_2$ are division rings, then $\phi(1_1) = 1_2$ where $1_1$ and $1_2$ are the multiplicative identities of $R_1$ and $R_2$ respectively.
\end{proposition}
\begin{proof}
    See \myref{exercise-ring-image-of-identity-is-identity} (later).
\end{proof}

\begin{proposition}
    $\phi(-x) = -\phi(x)$ for all $x \in R_1$.
\end{proposition}
\begin{proof}
    See \myref{exercise-ring-image-of-inverse-is-inverse} (later).
\end{proof}

\begin{proposition}\label{prop-inverse-under-ring-homomorphism}
    If $R_1$ and $R_2$ are both division rings, then $\phi(x^{-1}) = (\phi(x))^{-1}$ for all $x \in R_1$.
\end{proposition}
\begin{proof}
    See \myref{exercise-ring-image-of-inverse-is-inverse} (later).
\end{proof}

\begin{proposition}\label{prop-homomorphism-on-subring-is-subring}
    If $S$ is a subring of $R_1$, then
    \[
        \phi(S) = \{\phi(s) | s \in S\}
    \]
    is a subring of $R_2$.
\end{proposition}
\begin{proof}
    Let $S$ be a subring of $R_1$. Take $a, b \in \phi(S)$, which means that there exist $s_a, s_b\in S$ such that $\phi(s_a) = a$ and $\phi(s_b) = b$.
    \begin{itemize}
        \item We show that $(\phi(S), +) \leq (R_2, +)$.
        \begin{itemize}
            \item Note that $\phi(S) \neq \emptyset$ since $\phi(0_1) = 0_2 \in \phi(S)$.
            \item Also note that $a - b = \phi(s_a) - \phi(s_b) = \phi(s_a-s_b) \in \phi(S)$.
        \end{itemize}

        \item One also sees that
        \[
            ab = \phi(s_a)\phi(s_b) = \phi(s_as_b) \in \phi(S).
        \]
    \end{itemize}
    Therefore $\phi(S)$ is a subring of $R_2$.
\end{proof}

\begin{proposition}
    If $\phi$ is surjective and $I$ is an ideal of $R_1$, then $\phi(I)$ is an ideal of $R_2$.
\end{proposition}
\begin{proof}
    From previous proposition $\phi(I)$ is a subring of $R_2$. We just need to show that $\phi(I)$ is an ideal of $R_2$.

    Take $a \in \phi(I)$ and $r_2 \in R_2$. As $\phi$ is surjective, we can find a $r_1 \in R_1$ such that $\phi(r_1) = r_2$. Also, let $a = \phi(i)$ for an $i \in I$.

    Note
    \begin{align*}
        ar_2 = \phi(i)\phi(r_1) = \phi(ir_1) \in \phi(I)\\
        r_2a = \phi(r_1)\phi(i) = \phi(r_1i) \in \phi(I)
    \end{align*}
    so $\phi(I)$ is an ideal of $R_2$.
\end{proof}

\begin{proposition}\label{prop-inverse-homomorphism-on-ideal-is-ideal}
    Let $J$ be an ideal of $R_2$. Then
    \[
        \phi^{-1}(J) = \{r \in R_1 \vert \phi(r) \in J\}
    \]
    is an ideal of $R_1$.
\end{proposition}
\begin{proof}
    Suppose $J$ is an ideal of $R_2$. We consider the test for ideal (\myref{thrm-test-for-ideal}) to show $\phi^{-1}(J)$ is an ideal of $R_1$.

    One sees that $\phi^{-1}(J) \neq \emptyset$ since $\phi(0_1) = 0_2 \in J$, so $0_1 \in \phi^{-1}(J)$.

    Let $a, b \in \phi^{-1}(J)$, so $\phi(a), \phi(b) \in J$. Note that
    \[
        \phi(a-b) = \phi(a) - \phi(b) \in J
    \]
    so $a-b \in \phi^{-1}(J)$ for all $a,b \in J$.

    Let $r \in R_1$ and $a \in \phi^{-1}(J)$. Note that $\phi(a) \in J$ and $\phi(r) \in R_2$, so $\underbrace{\phi(a)}_{\text{In }J}\underbrace{\phi(r)}_{\text{In }R_2} \in J$ and $\phi(r)\phi(a) \in J$. Note $\phi(a)\phi(r) = \phi(ar) \in J$, so $ar \in \phi^{-1}(J)$, and similarly we have $\phi(r)\phi(a) = \phi(ra) \in J$, so $ra \in \phi^{-1}(J)$.

    Therefore, by the test for ideal, $\phi^{-1}(J)$ is an ideal of $R_1$.
\end{proof}

\begin{exercise}\label{exercise-ring-image-of-identity-is-identity}
    Let $R_1$ and $R_2$ be rings, and $\phi: R_1 \to R_2$ be a ring homomorphism.
    \begin{partquestions}{\alph*}
        \item Show that $\phi(0_1) = 0_2$, where $0_1$ and $0_2$ are the additive identities of $R_1$ and $R_2$ respectively.
        \item If $R_1$ and $R_2$ are division rings, then show that $\phi(1_1) = 1_2$, where $1_1$ and $1_2$ are the multiplicative identities of $R_1$ and $R_2$ respectively.
    \end{partquestions}
\end{exercise}

\begin{exercise}\label{exercise-ring-image-of-inverse-is-inverse}
    Let $R_1$ and $R_2$ be rings, $x \in R_1$, and $\phi: R_1 \to R_2$ be a ring homomorphism.
    \begin{partquestions}{\alph*}
        \item Show that $\phi(-x) = -\phi(x)$.
        \item If $R_1$ and $R_2$ are division rings, show that $\phi(x^{-1}) = (\phi(x))^{-1}$.
    \end{partquestions}
\end{exercise}

\section{Image and Kernel}
Similar to group homomorphisms, ring homomorphisms too have a image and kernel.
\begin{definition}
    The \term{image}\index{image} of a ring homomorphism $\phi: R_1 \to R_2$ is
    \[
        \im\phi = \{\phi(r) \vert r \in R_1\}.
    \]
\end{definition}
\begin{definition}
    The \term{kernel}\index{kernel} of a ring homomorphism $\phi:R_1 \to R_2$ is
    \[
        \ker\phi = \{r \in R_1 \vert \phi(r) = 0\}.
    \]
\end{definition}

\begin{example}\label{example-homomorphism-on-upper-triangle-matrices}
    Consider the ring
    \[
        R = \left\{\begin{pmatrix}a&b\\0&c\end{pmatrix}\vert a,b,c\in\Z\right\}
    \]
    and the homomorphism $\phi: R \to \Z^2, \begin{pmatrix}a&b\\0&c\end{pmatrix} \mapsto (a,c)$.

    We note that $\phi$ is surjective; for any $(x,y)\in\Z^2$, we see that 
    \[
        \phi\left(\begin{pmatrix}x&0\\0&y\end{pmatrix}\right) = (x,y)
    \]
    so any $(x,y)$ has a pre-image in $R$. Therefore $\im \phi = \Z^2$.

    We now find the kernel of $\phi$.
    \begin{align*}
        \ker\phi &= \left\{\begin{pmatrix}a&b\\0&c\end{pmatrix} \in R \vert \phi\left(\begin{pmatrix}a&b\\0&c\end{pmatrix}\right) = (0,0)\right\}\\
        &= \left\{\begin{pmatrix}a&b\\0&c\end{pmatrix} \in R \vert (a,c) = (0,0)\right\}\\
        &= \left\{\begin{pmatrix}0&n\\0&0\end{pmatrix} \vert n \in \Z\right\}.
    \end{align*}
\end{example}

We look at some results regarding the image and kernel of a ring homomorphism. These results may look familiar to those in part I.
\begin{proposition}\label{prop-image-is-a-subring}
    Let $R_1$ and $R_2$ be rings, and let $\phi: R_1 \to R_2$ be a ring homomorphism. Then $\im\phi$ is a subring of $R_2$.
\end{proposition}
\begin{proof}
    \myref{prop-homomorphism-on-subring-is-subring} tells us $\im\phi = \phi(R_1)$ is a subring of $R_2$.
\end{proof}

\begin{proposition}\label{prop-kernel-is-an-ideal}
    Let $R_1$ and $R_2$ be rings, and let $\phi: R_1 \to R_2$ be a ring homomorphism. Then $\ker\phi$ is an ideal of $R_1$.
\end{proposition}
\begin{proof}
    See \myref{exercise-kernel-is-an-ideal} (later).
\end{proof}

\begin{proposition}
    Let $R_1$ and $R_2$ be rings, and let $\phi: R_1 \to R_2$ be a ring homomorphism. Then $\phi$ is injective if and only if $\ker\phi = \{0_1\}$.
\end{proposition}
\begin{proof}
    We first prove the forward direction; suppose $\phi$ is injective and let $a \in \ker\phi$. By definition of the kernel we have $\phi(a) = 0_2$. But by \myref{prop-ring-image-of-additive-identity-is-additive-identity}, we have $\phi(0_1) = 0_2$. Since $\phi$ is injective, therefore $a = 0_1$, meaning $\ker\phi = \{0_1\}$.

    We now prove the reverse direction; suppose $\ker\phi = \{0_1\}$. Now let $a,b \in R_1$ such that $\phi(a) = \phi(b)$. Therefore $\phi(a) - \phi(b) = \phi(a-b) = 0_2$. Therefore $a-b \in \ker\phi$ by definition of the kernel. However $\ker\phi = \{0_1\}$ which means that $a - b = 0_1$. Therefore $a = b$, meaning $\phi$ is injective.
\end{proof}

\begin{exercise}\label{exercise-kernel-is-an-ideal}
    Let $R_1$ and $R_2$ be rings, and let $\phi: R_1 \to R_2$ be a ring homomorphism. Prove that $\ker\phi$ is an ideal of $R_1$.
\end{exercise}

\section{The Ring Isomorphism Theorems}
Similar to group theory, there are three main ring isomorphism theorems. However, we will only explicitly prove the first ring isomorphism theorem; the other two will be left as problems.

\begin{theorem}[First Ring Isomorphism Theorem (FRIT)]\label{thrm-ring-isomorphism-1}\index{isomorphism theorem!ring!first}\index{FRIT}
    Let $R$ and $R'$ be rings. Let $\phi: R \to R'$ be a ring homomorphism, and let $\pi: R \to R/\ker\phi$, $r\mapsto r + \ker\phi$ be the natural surjective homomorphism. Then there exists a unique ring isomorphism $\psi: R / \ker\phi \to \im\phi$ such that $\psi\pi = \phi$.
\end{theorem}
\begin{remark}
    Equivalently, the FRIT states that
    \[
        R / \ker\phi \cong \im\phi
    \]
    for any ring homomorphism $\phi$. This means that the set of pre-images that have an image of $\phi(x)$ is $x\ker\phi$.
\end{remark}

We include the commutativity diagram of the stated maps for clarity.
\begin{figure}[h]
    \centering
    \pdfteximgframed[12pt]{0.3\textwidth}{part2/images/ring-homomorphisms/ring-iso-1-commutativity.pdf_tex}
    \caption{Commutativity Diagram for \myreffigures{thrm-ring-isomorphism-1}}
\end{figure}

In the diagram, $\phi$ sends elements from $R$ to $\im\phi$ and $\pi$ sends elements from $R$ to $R/\ker\phi$. Then the map $\psi$ is a unique map that sends elements from $R/\ker\phi$ to the image of $\phi$.

\begin{proof}[Proof (cf. {\cite[p.~302, Factor Theorem For Rings]{cohn_1982}})]
    Let the map $\psi$ be defined such that $\psi(r + \ker\phi) = \phi(r)$. We first show that $\psi$ is a well-defined ring isomorphism.
    \begin{itemize}
        \item \textbf{Well-Defined}: Suppose $a + \ker\phi$ and $b + \ker\phi$ are in $R/\ker\phi$ such that $a + \ker\phi = b+\ker\phi$. This means that $a - b \in \ker\phi$, i.e. $\phi(a-b) = 0$ by definition of the kernel. Hence $\phi(a) - \phi(b) = 0$ which means $\phi(a) = \phi(b)$. Therefore, we see that
        \[
            \psi(a + \ker\phi) = \phi(a) = \phi(b) = \psi(b + \ker\phi)
        \]
        which means $\psi$ is well-defined.

        \item \textbf{Homomorphism}: Let $a + \ker\phi, b + \ker\phi \in R/\ker\phi$. Then note
        \begin{align*}
            &\psi((a + \ker\phi)+(b+\ker\phi))\\
            &= \psi((a+b)+\ker\phi)\\
            &= \phi(a+b)\\
            &= \phi(a) + \phi(b)\\
            &= \psi(a + \ker\phi) + \psi(b + \ker\phi)
        \end{align*}
        and
        \begin{align*}
            \psi((a + \ker\phi)(b+\ker\phi)) &= \psi((ab)+\ker\phi)\\
            &= \phi(ab)\\
            &= \phi(a)\phi(b)\\
            &= \psi(a + \ker\phi)\psi(b + \ker\phi),
        \end{align*}
        so $\psi$ is a ring homomorphism.

        \item \textbf{Injective}: Suppose $a + \ker\phi, b + \ker\phi \in R/\ker\phi$ such that $\psi(a+\ker\phi) = \psi(b+\ker\phi)$. So,
        \begin{align*}
            \phi(a) &= \phi(b) & (\text{definition of }\psi)\\
            \phi(a) - \phi(b) &= 0\\
            \phi(a-b) &= 0 & (\phi \text{ is a ring homomorphism})\\
            a - b &\in \ker\phi & (\text{definition of kernel})\\
            a + \ker\phi &= b + \ker\phi.
        \end{align*}
        Therefore if $\psi(a+\ker\phi) = \psi(b+\ker\phi)$ then $a+\ker\phi = b+\ker\phi$, which means $\psi$ is injective.

        \item \textbf{Surjective}: Suppose $s \in \im\phi$, so there is an $r \in R$ such that $s = \phi(r)$. Clearly $\psi(r + \ker\phi) = \phi(r) = s$, so $s$ has a pre-image of $r + \ker\phi$, i.e. $\psi$ is surjective.
    \end{itemize}
    Therefore, $\psi$ is a well-defined bijective ring homomorphism, i.e. $\psi$ is a well-defined ring isomorphism.

    We now check that $\psi$ satisfies the requirement that $\psi\pi = \phi$. Note that $\pi(x) = x + \ker\phi$ and
    \[
        \psi\pi(x) = \psi(x + \ker\phi) = \phi(x)
    \]
    for all $x \in R$, so $\psi\pi = \phi$.

    Finally, we show that $\psi$ is unique. Suppose $f: R/\ker\phi \to \im\phi$ is an isomorphism satisfying $f\pi=\phi$. Note that
    \begin{align*}
        f(x + \ker\phi) &= f(\pi(x))\\
        &= (f\pi)(x)\\
        &= \phi(x)\\
        &= (\psi\pi)(x)\\
        &= \psi(\pi(x))\\
        &= \psi(x + \ker\phi)
    \end{align*}
    for all $x \in R$, meaning that $f = \psi$. Therefore $\psi$ is unique.

    Hence, $\psi$ is a unique ring isomorphism satisfying $\psi\pi = \phi$.
\end{proof}

\begin{example}
    Consider the ring
    \[
        R = \left\{\begin{pmatrix}a&b\\0&c\end{pmatrix}\vert a,b,c\in\Z\right\}
    \]
    and the homomorphism $\phi: R \to \Z^2, \begin{pmatrix}a&b\\0&c\end{pmatrix} \mapsto (a,c)$. We found in \myref{example-homomorphism-on-upper-triangle-matrices} that $\phi$ is surjective (i.e., $\im\phi = \Z^2$) with kernel
    \[
        \left\{\begin{pmatrix}0&n\\0&0\end{pmatrix} \vert n \in \Z\right\}
    \]
    which, for brevity, we shall denote by $I$. Thus the FRIT (\myref{thrm-ring-isomorphism-1}) tells us that
    \[
        R/I \cong \Z^2.
    \]
\end{example}
\begin{exercise}
    Show that $\Z_n \cong \Z/n\Z$ by considering the ring homomorphism $\phi: \Z \to \Z_n$ where $m \mapsto m \mod n$ and by using the FRIT (\myref{thrm-ring-isomorphism-1}).
\end{exercise}

We briefly mention the other two main ring isomorphism theorems, although the proof of them will be left as problems. They are much less used than the FRIT, so we make only a passing mention of them.

\newpage

\begin{theorem}[Second Ring Isomorphism Theorem]\label{thrm-ring-isomorphism-2}\index{isomorphism theorem!ring!second}
    Let $R$ be a ring with subring $S$ and ideal $I$. Then
    \begin{enumerate}
        \item $S+I = \{s+i \vert s\in S,\;i\in I\}$ is a subring of $R$;
        \item $S \cap I$ is an ideal of $S$; and
        \item $(S+I)/I \cong S/(S\cap I)$.
    \end{enumerate}
\end{theorem}
\begin{proof}
    See \myref{problem-ring-isomorphism-2} (later).
\end{proof}

\begin{theorem}[Third Ring Isomorphism Theorem]\label{thrm-ring-isomorphism-3}\index{isomorphism theorem!ring!third}
    Let $R$ be a ring with ideals $I$ and $J$ such that $I$ is a subset of $J$. Then
    \begin{enumerate}
        \item $J/I$ is an ideal of $R/I$; and
        \item $\frac{R/I}{J/I} \cong R/J$.
    \end{enumerate}
\end{theorem}
\begin{proof}
    See \myref{problem-ring-isomorphism-3} (later).
\end{proof}

\section{Restrictiveness of Ring Homomorphisms}
Although ring homomorphisms appear to be quite general, we explore how restricted they really are when dealing with certain rings.

\begin{example}\label{example-endomorphisms-of-Z}
    We find all ring endomorphisms of $\Z$.

    Let $\phi:\Z\to\Z$ be a ring endomorphism. Set $a = \phi(1)$. Note that
    \[
        a = \phi(1) = \phi(1\times1) = \phi(1)\phi(1) = a^2
    \]
    so $a^2 = a$. Thus $a = 0$ or $a = 1$ in $\Z$.

    If $a = 0$, then for any $n \in \Z$ we have
    \[
        \phi(n) = \phi(1n) = \phi(1)\phi(n) = 0\phi(n) = 0
    \]
    so $\phi(n) = 0$ for all $n \in \Z$, which is the trivial homomorphism.

    Now consider the case that $a = 1$. We claim that $\phi(n) = n$ for all $n \in \Z$. We leave the proof that $\phi(n) = n$ for all \textit{positive} integers $n$ for \myref{exercise-homomorphism-maps-n-to-n-if-n-is-positive} (later). Furthermore $\phi(0) = 0$ by the properties of ring homomorphism (specifically \myref{prop-ring-image-of-additive-identity-is-additive-identity}). Finally, note that for any non-negative integer $n$,
    \begin{align*}
        0 &= \phi(0)\\
        &= \phi(n - n)\\
        &= \phi(n) + \phi(-n)\\
        &= n + \phi(-n)
    \end{align*}
    which means $\phi(-n) = -n$. Thus $\phi(n) = n$ for all integers $n$, which is the identity endomorphism.

    Therefore, the only ring endomorphisms $\phi:\Z\to\Z$ are the trivial homomorphism and the identity endomorphism.
\end{example}
\begin{exercise}\label{exercise-homomorphism-maps-n-to-n-if-n-is-positive}
    For the map $\phi$ in \myref{example-endomorphisms-of-Z}, show that $\phi(n) = n$ for all positive integers $n$.
\end{exercise}
Note that in \myref{example-endomorphisms-of-Z} we started the entire computation with the observation that $\phi(1) = \phi(1)^2$. This means that $\phi(1)$ is an idempotent.
\begin{definition}
    Let $R$ be a ring. Then an element $x \in R$ is an \term{idempotent}\index{idempotent} if and only if $x^2 = x$.
\end{definition}
\begin{proposition}\label{prop-homomorphism-on-multiplicative-identity-is-idempotent}
    Let $R$ and $R'$ be rings, and let $\phi: R \to R'$ be a ring homomorphism. If $R$ is a ring with identity, then $\phi(1)$ is an idempotent.
\end{proposition}
\begin{proof}
    Note
    \[
        \phi(1) = \phi(1 \times 1) = \phi(1) \times \phi(1) = \left(\phi(1)\right)^2
    \]
    which means $\phi(1)$ is an idempotent.
\end{proof}

In \myref{example-endomorphisms-of-Z} we used the fact that the only idempotents of $\Z$ are 0 and 1. However, this is not true for a general ring.

\begin{example}\label{example-homomorphisms-from-Z12-to-Z28}
    We find all ring homomorphisms $\phi: \Z_{12} \to \Z_{28}$. Note that $\Z_{12}$ is a ring with an identity of 1.

    \myref{prop-homomorphism-on-multiplicative-identity-is-idempotent} tells us that $\phi(1)$ is an idempotent, so we need to find all idempotents of $\Z_{28}$. However, we cannot just assume that 0 and 1 are the \textbf{only} idempotents in $\Z_{28}$; we need to check for them exhaustively.

    By exhaustion, we see that $0^2 = 0$, $1^2 = 1$, $8^2 = 64 = 2 \times 28 + 8 = 8$, $21^2 = 441 = 15 \times 28 + 21 = 21$. So the idempotents in $\Z_{28}$ are 0, 1, 8, and 21. This is not enough to narrow down the possible values of $\phi(1)$, so we need to use more facts.

    Recall from part I that $|\phi(1)|_+$ divides $|1|_+$ by \myref{exercise-order-of-homomorphism-divides-order}. Therefore $|\phi(1)|_+$ divides 12. Furthermore, \myref{thrm-order-of-element-in-cyclic-group} tells us that the additive order of an element $k$ in the group $(\Z_n, +)$ is $\frac{n}{\gcd(k,n)}$. So we must exhaust all idempotents in $\Z_{28}$ to check whether they are valid values for $\phi(1)$.

    % \begin{itemize}
    %     \item $|0|_+ = 1$ which divides 12, so 0 is a valid value of $\phi(1)$.
    %     \item $|1|_+ = 28$ which does not divide 12, so 1 is not a valid value of $\phi(1)$. Note that this is different from the previous example, where 1 was a possible value of $\phi(1)$.
    %     \item $|8|_+ = \frac{28}{\gcd(8,28)} = \frac{28}4 = 7$ which does not divide 12, so 8 is not a valid value of $\phi(1)$.
    %     \item $|21|_+ = \frac{28}{\gcd(21,28)} = \frac{28}7 = 4$ which divides 12, so 21 is a valid value of $\phi(1)$.
    % \end{itemize}
    % Hence $\phi(1) = 0$ or $\phi(1) = 21$.

    \begin{multicols}{2}
        \begin{itemize}
            \item $|0|_+ = \frac{28}{\gcd(0,28)} = \frac{28}{28} = 1$
            \item $|1|_+ = \frac{28}{\gcd(1,28)} = \frac{28}{1} = 28$
            \item $|8|_+ = \frac{28}{\gcd(8,28)} = \frac{28}{4} = 7$
            \item $|21|_+ = \frac{28}{\gcd(21,28)} = \frac{28}{7}= 4$
        \end{itemize}
    \end{multicols}

    Of the four idempotents, only $|0|_+ = 1$ and $|21|_+ = 4$ divides 12, which means $\phi(1) = 0$ or $\phi(1) = 21$.

    If $\phi(1) = 0$, then for any $n \in \Z_{12}$,
    \begin{align*}
        \phi(n) &= \phi(\underbrace{1 + 1 + \cdot + 1}_{n \text{ times}})\\
        &= \underbrace{\phi(1) + \phi(1) + \cdot + \phi(1)}_{n \text{ times}}\\
        &= \underbrace{0 + 0 + \cdots + 0}_{n \text{ times}}\\
        &= 0
    \end{align*}
    which means that $\phi(n) = 0$ for all $n \in \Z_{12}$, i.e. $\phi$ is trivial.

    If instead $\phi(1) = 21$, then
    \begin{align*}
        \phi(n) &= \underbrace{\phi(1) + \phi(1) + \cdot + \phi(1)}_{n \text{ times}}\\
        &= \underbrace{21 + 21 + \cdots + 21}_{n \text{ times}}\\
        &= 21n
    \end{align*}
    which means $\phi(n) = 21n$ for all $n \in \Z_{12}$.

    Thus the only homomorphisms $\phi: \Z_{12} \to \Z_{28}$ are $\phi(n) = 0$ and $\phi(n) = 21n$ for all $n \in \Z_{12}$.
\end{example}

\begin{exercise}\label{exercise-homomorphism-over-Q-fixes-elements-of-Q}
    Suppose $R$ and $R'$ are rings such that $\Q$ is a subring of both $R$ and $R'$. Let $\phi: R \to R'$ be a ring homomorphism such that $\phi(1) = 1$. Show that for any $q \in \Q$ we have $\phi(q) = q$.
\end{exercise}

\newpage

\section{Problems}
\begin{problem}\label{problem-integral-domain-iff-trivial-ideal-is-prime}
    Let $R$ be a ring.
    \begin{partquestions}{\roman*}
        \item Show that $R/\{0\} \cong R$.
        \item Prove that $R$ is an integral domain if and only if $\{0\}$ is a prime ideal.
        \item Prove that $R$ is a field if and only if $\{0\}$ is a maximal ideal.
    \end{partquestions}
\end{problem}

\begin{problem}
    Find all ring endomorphisms of $\Q$.
\end{problem}

\begin{problem}
    Show $\Z^2 \not\cong \Q$.
\end{problem}

\begin{problem}
    Show $\Q[\sqrt2] \not\cong \Q[\sqrt3]$.
\end{problem}

\begin{problem}
    Consider the subring
    \[
        R = \left\{\begin{pmatrix}a&0\\0&b\end{pmatrix}\vert a,b \in \Z\right\}
    \]
    of $\Mn{2}{\Z}$. Show that $R \cong \Z^2$.
\end{problem}

\begin{problem}\label{problem-properties-of-ring-isomorphism}
    Let $R$ and $R'$ be rings, and let $\phi: R \to R'$ be a ring isomorphism. Prove or disprove the following statements.
    \begin{partquestions}{\alph*}
        \item $\phi^{-1}: R' \to R$ is a ring isomorphism.
        \item If $R$ has a subring with $n$ elements, then so does $R'$.
        \item If $R$ has an ideal, then so does $R'$.
    \end{partquestions}
\end{problem}

\begin{problem}
    Find all ring endomorphisms of $\Z_{10}$. Hence find all ring automorphisms $\psi$ of $\Z_{10}$.
\end{problem}

\begin{problem}
    Find all ring endomorphisms of $\Q[\sqrt3]$. Hence find all ring automorphisms $\psi$ of $\Q[\sqrt3]$.
\end{problem}

\begin{problem}
    Let $R$ and $R'$ be commutative rings, $I$ be an ideal of $R$, and $\phi: R\to R'$ be a ring homomorphism.
    \begin{partquestions}{\roman*}
        \item Show that $\phi(\sqrt I) \subseteq \sqrt{\phi(I)}$.
        \item If $\phi$ is surjective with $\ker\phi \subseteq I$, prove that $\phi(\sqrt I) = \sqrt{\phi(I)}$.
    \end{partquestions}
\end{problem}

\pagebreak

\begin{problem}\label{problem-ring-isomorphism-2}
    Let $R$ be a ring with a subring $S$ and ideal $I$. Prove that
    \begin{partquestions}{\roman*}
        \item $S+I$ is a subring of $R$;
        \item $S \cap I$ is an ideal of $S$; and
        \item $S/(S\cap I)\cong (S+I)/I$.
    \end{partquestions}
\end{problem}

\begin{problem}\label{problem-ring-isomorphism-3}
    Let $R$ be a ring with ideals $I$ and $J$ such that $I$ is a subset of $J$.
    \begin{partquestions}{\roman*}
        \item Prove that $J/I$ is an ideal of $R/I$.
        \item Prove that $\frac{R/I}{J/I} \cong R/J$.\newline
        (\textit{Note: remember to prove that the map is well-defined.})
    \end{partquestions}
\end{problem}

\section{Polynomial Rings}
\begin{questions}
    \item For brevity let $I = \princ{x} = \{xP(x) \vert P(x) \in \Z[x]\}$. This means that $I$ is the set of polynomials with integer coefficients and with constant term 0. Now suppose $f(x), g(x) \in \Z[x]$; write
    \begin{align*}
        f(x) &= a_0 + a_1x + \cdots + a_mx^m\\
        g(x) &= b_0 + b_1x + \cdots + b_nx^n
    \end{align*}
    where $a_i, b_i \in \Z$ and $m$ and $n$ are positive integers. Note that
    \[
        f(x)g(x) = a_0b_0 + (a_1b_0+a_0b_1)x + \cdots.
    \]
    Now if $f(x)g(x) \in I$, this means that $a_0b_0 = 0$. Hence either $a_0 = 0$ or $b_0 = 0$, meaning that either $f(x)$ has zero constant term (so $f(x) \in I$) or $g(x)$ has zero constant term (so $g(x) \in I$). Thus $I$ is prime.

    \item \begin{partquestions}{\roman*}
        \item Let $f(x), g(x) \in \Z[x]$. Then
        \[
            \phi(f(x) + g(x)) = f(-2) + g(-2) = \phi(f(x)) + \phi(g(x))
        \]
        and
        \[
            \phi(f(x)g(x)) = f(-2)g(-2) = \phi(f(x))\phi(g(x))
        \]
        so $\phi$ is a ring homomorphism.

        \item Note that
        \begin{align*}
            \ker\phi &= \{f(x) \in \Z[x] \vert \phi(f(x)) = 0\}\\
            &= \{f(x) \in \Z[x] \vert f(-2) = 0\}\\
            &= I.
        \end{align*}
        \myref{prop-kernel-is-an-ideal} tells us that $\ker\phi$ is an ideal of $\Z[x]$, so $I$ is an ideal of $\Z[x]$.

        \item We first show that $\phi$ is surjective. Let $n \in \Z$, note that $n$ is a degree zero polynomial, so $n \in \Z[x]$. Clearly $\phi(n) = n$ so $n$ is its own pre-image. Therefore $\im\phi = \Z$.
        
        By FRIT (\myref{thrm-ring-isomorphism-1}),
        \[
            \Z[x]/I \cong \Z.
        \]
        Note that $\Z$ is an integral domain but not a field. Thus $I$ is prime but not maximal.
    \end{partquestions}
\end{questions}


\chapter{Image Acknowledgements}
Unless otherwise stated, all images are the author's own work, and are released under the same licence as this book.

\section{Cover Page}
Image of the heptagon was created by L\'{a}szl\'{o} N\'{e}meth on Wikimedia. It is licensed under the CC0 1.0 Universal (CC0 1.0) Public Domain Dedication. The original file is available \href{https://commons.wikimedia.org/wiki/File:Regular_polygon_7_annotated.svg}{here}.

\printbibliography[heading=bibintoc, title={References and Bibliography}]
\printindex


\end{document}
