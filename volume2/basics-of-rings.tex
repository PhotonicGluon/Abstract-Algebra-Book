\chapter{Basics of Rings}
With basic algebraic structures and the definition of a ring out of the way, we are now ready to tackle the basics of rings in this chapter.

\section{Basic Examples of Rings}
Before we introduce some examples of rings, we make some remarks for the notation that is used in Ring Theory.
\begin{itemize}
    \item The multiplication symbol $\cdot$ is usually omitted, so $x \cdot y$ is written as $xy$.
    \item The additive identity of $R$ will always be denoted by 0 and the multiplicative identity of $R$ (if it exists) will always be denoted by 1.
    \item The additive inverse of the element $x$ will be denoted by $-x$ and the multiplicative inverse of $x$ (if it exists) will be denoted by $x^{-1}$.
    \item $n$ applications of $+$ on an element $x$ will be denoted $nx$ (and will be denoted $-nx$ if the element is $-x$), while $n$ applications of $\cdot$ on an element $x$ will be denoted $x^n$ (and will be denoted $x^{-n}$ if the element is $x^{-1}$ and if it exists).
\end{itemize}

Let's look at some examples of rings.
\begin{example}
    We show that $(\Z_n, \oplus_n, \otimes_n)$, where $\oplus_n$ and $\otimes_n$ denote addition and multiplication modulo $n$ respectively, form a ring.
    \begin{itemize}
        \item \textbf{Addition-Abelian}: We know $(\Z_n, \oplus_n)$ is an abelian group from Volume I.
        \item \textbf{Multiplication-Semigroup}: We can see that $(\Z_n, \otimes_n)$ is a semigroup as
        \begin{itemize}
            \item $\Z_n$ is closed under $\otimes_n$ because $a \otimes_n b \in \{0, 1, 2, \dots, n-1\} = \Z_n$; and
            \item multiplication is associative, so multiplication modulo $n$ is associative.
        \end{itemize}
        \item \textbf{Distributive}: It is clear that $\oplus_n$ and $\otimes_n$ distribute.
    \end{itemize}
    Hence $(\Z_n, \oplus_n, \otimes_n)$ is a ring. Furthermore, $\otimes_n$ has an identity of 1 and is commutative, so in fact $(\Z_n, \otimes_n)$ is a commutative monoid. Therefore $(\Z_n, \oplus_n, \otimes_n)$ is a commutative ring with identity.

    Note that, in this volume, $\Z_n$ refers to the ring $(\Z_n, \oplus_n, \otimes_n)$.
\end{example}

\newpage

\begin{example}
    We show that $(\Q, +, \times)$, where $+$ and $\times$ denote normal addition and multiplication, form a ring.
    \begin{itemize}
        \item \textbf{Addition-Abelian}: From Volume I, we know that $(\Q, +)$ is an abelian group.
        \item \textbf{Multiplication-Semigroup}: We note that $(\Q, \times)$ is a semigroup as
        \begin{itemize}
            \item $Q$ is closed under $\times$ because multiplying two rational numbers together produce a rational number; and
            \item multiplication is associative.
        \end{itemize}
        \item \textbf{Distributive}: It is clear that $+$ and $\times$ distribute.
    \end{itemize}
    Hence $(\Q, +, \times)$ is a ring. Furthermore, $\times$ has an identity of 1 and is commutative. So, just like $\Z_n$, we see that $(\Q, +, \times)$ is a commutative ring with identity.

    Note that, in this volume, $\Q$ refers to the ring $(\Q, +, \times)$.
\end{example}

\begin{exercise}
    Prove that $\Z$ is a ring under regular addition and multiplication.\newline
    (\textit{You do \textbf{not} need to prove the \textbf{Distributive} axiom.})
\end{exercise}
Note that, in this volume, $\Z$ refers to the ring $(\Z, +, \times)$.

We remark that $\R$ is a ring under regular addition and multiplication, although we will not prove it here.

We look at one more ring: the ring of $n \times n$ matrices.

Let $\Mn{n}{R}$ denote the set of $n\times n$ matrices with entries in the ring $(R, \oplus, \otimes)$.
\begin{itemize}
    \item We define \textbf{matrix addition}\index{matrix addition} within this set. Suppose we have $\textbf{A}, \textbf{B} \in \Mn{n}{R}$, and let their sum be $\textbf{C} = \textbf{A} + \textbf{B}$. Then
    \[
        c_{i,j} = a_{i,j} \oplus b_{i,j}    
    \]
    where $\oplus$ is the addition operation in the ring $R$.
    
    \item We define \textbf{matrix multiplication}\index{matrix multiplication} within this set. Suppose $\textbf{A}, \textbf{B} \in \Mn{n}{R}$, and let their product be $\textbf{C} = \textbf{AB}$. Then
    \begin{align*}
        c_{i,j} &= (a_{i,1}\otimes b_{1,j}) \oplus (a_{i,2}\otimes b_{2,j}) \oplus \cdots \oplus (a_{i,n}\otimes b_{n,j})\\
        &= \sum_{k=1}^n (a_{i,k}\otimes b_{i,k})
    \end{align*}
    where $\oplus$ and $\otimes$ are the addition and multiplication operations in the ring $R$ respectively.
\end{itemize}
For brevity, we let $\Mn{n}{R}$ denote the ring under matrix addition and multiplication.

\newpage

\begin{proposition}
    $\Mn{n}{R}$ is a ring under matrix addition and matrix multiplication.
\end{proposition}
\begin{proof}
    We need to prove that the ring axioms hold.
    \begin{itemize}
        \item \textbf{Addition-Abelian}: We first prove that $(\Mn{n}{R}, +)$ is indeed an abelian group.
        \begin{itemize}
            \item \textbf{Closure}: Clearly the sum of any two matrices in $\Mn{n}{R}$ is also a square matrix with $n$ rows with elements inside $R$, meaning that $\Mn{n}{R}$ is closed under matrix addition.

            \item \textbf{Associative}: Let the matrices $\textbf{A}$, $\textbf{B}$, and $\textbf{C}$ belong inside $\Mn{n}{R}$. Let $\textbf{P} = \textbf{A} + (\textbf{B} + \textbf{C})$ and $\textbf{Q} = (\textbf{A} + \textbf{B}) + \textbf{C}$. We note that $\textbf{P} = \textbf{Q}$ as
            \[
                p_{i,j} = a_{i,j} \oplus (b_{i,j} \oplus c_{i,j}) = (a_{i,j} \oplus b_{i,j}) \oplus c_{i,j} = q_{i,j}
            \]
            by associativity of $\oplus$, which proves that matrix addition is associative.
    
            \item \textbf{Identity}: Denote the $n \times n$ matrix of all zeros by $\textbf{0}_n$. One sees clearly that this is the identity in $\Mn{n}{R}$ as $\textbf{M} + \textbf{0}_n = \textbf{M}$ for any matrix in $\Mn{n}{R}$.
            
            \item \textbf{Inverse}: Let $\textbf{A} \in \Mn{n}{R}$. Define the matrix $\textbf{B} = -\textbf{A}$ such that $b_{i,j} = -a_{i,j}$. That is, $b_{i,j}$ contains the additive inverse of $a_{i,j}$ in the ring $R$. Then one sees that $\textbf{A} + \textbf{B} = \textbf{0}_n$. (We denote the additive inverse of a matrix $\textbf{M}$ by $-\textbf{M}$).

            \item \textbf{Commutative}: Let $\textbf{A}, \textbf{B} \in \Mn{n}{R}$. Set $\textbf{C} = \textbf{A} + \textbf{B}$ and $\textbf{D} = \textbf{B} + \textbf{C}$. Consider $c_{i,j} = a_{i,j} \oplus b_{i,j}$. Since $\oplus$ is commutative, thus $a_{i,j} \oplus b_{i,j} = b_{i,j} \oplus a_{i,j}$. But $d_{i,j} = b_{i,j} \oplus a_{i,j}$, so we have $c_{i,j} = d_{i,j}$. Therefore $\textbf{C} = \textbf{D}$.
        \end{itemize}

        \item \textbf{Multiplication-Semigroup}: We show that $(\Mn{n}{R}, \cdot)$ is a semigroup.
        \begin{itemize}
            \item \textbf{Closure}: In Volume I we showed that matrix multiplication produces another $n \times n$ matrix. Furthermore the entries of the new matrix are elements of $R$. Hence $\Mn{n}{R}$ is closed under matrix multiplication.
        
            \item \textbf{Associative}: We proved matrix multiplication is associative in Volume I.
        \end{itemize}
        
        \item \textbf{Distributive}: We prove only $\textbf{A}(\textbf{B} + \textbf{C}) = (\textbf{AB}) + (\textbf{AC})$ as the other case is proven similarly. Let $\textbf{R} = \textbf{A}(\textbf{B} + \textbf{C})$, $\textbf{G} = \textbf{AB}$, and $\textbf{H} = \textbf{AC}$. We note
        \begin{align*}
            r_{i,j} &= \sum_{k=1}^n \left(a_{i,k} \otimes \left(b_{k,j} \oplus c_{k,j}\right)\right)\\
            &= \sum_{k=1}^n \left((a_{i,k} \otimes b_{k,j}) \oplus (a_{i,k} \otimes c_{k,j})\right)\\
            &= \left(\sum_{k=1}^n (a_{i,k} \otimes b_{k,j})\right) \oplus \left(\sum_{k=1}^n (a_{i,k} \otimes c_{k,j})\right)\\
            &= g_{i,j}\oplus h_{i,j}
        \end{align*}
        which means $\textbf{R} = \textbf{G} + \textbf{H}$.
    \end{itemize}
    As all the ring axioms are satisfied, thus $\Mn{n}{R}$ is a ring.
\end{proof}

\section{General Properties of Rings}
We list some properties of rings here. For each of the propositions, assume $R$ is a ring.

\begin{proposition}\label{prop-multiplying-by-zero-is-zero}
    $0x = x0 = 0$ for all $x \in R$.
\end{proposition}
\begin{proof}
    We note that
    \begin{align*}
        0x &= (0 + 0)x & (0 \text{ is additive inverse})\\
        &= 0x + 0x & (\text{by \textbf{Distributive} axiom})
    \end{align*}
    so by `subtracting' $0x$ on both sides (i.e., adding $-0x$ on both sides) we see $0 = 0x$. A similar argument shows that $x0 = 0$.
\end{proof}

\begin{proposition}\label{prop-product-of-element-and-additive-inverse-is-additive-inverse-of-product}
    $(-a)b = a(-b) = -(ab)$ for any $a$ and $b$ in $R$.
\end{proposition}
\begin{proof}
    We show that $(-a)b = -(ab)$ and $a(-b) = -(ab)$ to complete the proof.
    \begin{itemize}
        \item Note $(-a)b + ab = (-a + a)b = 0b = 0$ by \textbf{Distributive} axiom. Hence by subtracting $ab$ on both sides we see $(-a)b = -(ab)$.
        \item Note also $a(-b) + ab = a(-b + b) = a0 = 0$ by \textbf{Distributive} axiom. Hence by subtracting $ab$ on both sides we see $a(-b) = -(ab)$.
    \end{itemize}
    Result follows.
\end{proof}

\begin{proposition}
    $(-a)(-b) = ab$ for any $a$ and $b$ in $R$.
\end{proposition}
\begin{proof}
    See \myref{exercise-product-of-additive-inverses} (later).
\end{proof}

\begin{proposition}
    If $R$ has an identity, it is unique.
\end{proposition}
\begin{proof}
    Suppose 1 and $1'$ are identities, and consider the sum $1 + 1'$. Then
    \begin{align*}
        1 + 1' &= 1'(1+1') & (\text{multiplying by identity }1')\\
        &= 1'1 + 1'1' & (\text{by \textbf{Distributive} axiom})\\
        &= 1' + 1'. & (1 \text{ and } 1' \text{ are identities})
    \end{align*}
    Subtracting $1'$ on both sides yields $1 = 1'$, meaning that the identity is unique.
\end{proof}

\begin{exercise}\label{exercise-product-of-additive-inverses}
    Show that $(-a)(-b) = ab$ for any $a$ and $b$ in $R$.
\end{exercise}

\section{More Definitions}
Suppose $R$ is a ring.
\begin{definition}
    We say that $a \neq 0$ is a \textbf{zero divisor}\index{zero divisor} in $R$ if there exists $b \neq 0$ such that $ab = 0$.
\end{definition}
\begin{example}
    Consider the ring $\Z_{12}$. Clearly 4 and 6 are in $\Z_{12}$, and their product is $24 = 2 \times 12 = 0$ in $\Z_{12}$. Hence 4 and 6 are zero divisors in $\Z_{12}$.
\end{example}
\begin{example}
    We claim that the ring
    \[
        R = \{f: \R \to \R \vert f: [0, 1] \to [0, 1]\}
    \]
    has zero divisors. Consider the functions
    \begin{align*}
        f(x) &= x\\
        g(x) &= \begin{cases}
            0 & \text{ if } x \neq 0\\
            1 & \text{ if } x = 0
        \end{cases}
    \end{align*}
    Clearly neither of them are the zero function. However, consider $f(x)g(x)$.
    \begin{itemize}
        \item If $x \neq 0$, then $g(x) = 0$ which means $f(x)g(x) = 0$.
        \item If $x = 0$, then $f(x) = 0$ which means $f(x)g(x) = 0$.
    \end{itemize}
    Hence their product is the zero function, meaning that $R$ has zero divisors $f$ and $g$.
\end{example}
\begin{exercise}
    Does the ring $\Mn{2}{\mathbb{R}}$ have zero divisors?
\end{exercise}
We note one property about zero divisors, which will be used in future chapters.
\begin{proposition}\label{prop-zero-divisors-have-no-inverses}
    Zero divisors do not have inverses.
\end{proposition}
\begin{proof}
    Assume $a \neq 0$ and $b \neq 0$ are zero divisors in the ring $R$, so $ab = 0$. Seeking a contradiction, assume $a$ has an inverse, so
    \[
        b = (a^{-1}a)b = a^{-1}(ab) = a^{-1}0 = 0    
    \]
    which contradicts $b \neq 0$. Hence a zero divisor has no inverse.
\end{proof}

\begin{definition}
    Suppose $R$ is a ring with identity such that $0 \neq 1$. An element $u \in R$ is called a \textbf{unit}\index{unit} if there exists a $v \in R$ such that $uv=vu=1$. Equivalently, $u$ is a unit if it has a multiplicative inverse.
\end{definition}
\begin{example}
    3 and 7 are units in $\Z_{10}$ since $3 \times 7 = 7 \times 3 = 21 = 1$ in $\Z_{10}$.
\end{example}

\begin{definition}
    Suppose $R$ is a ring with identity such that $0 \neq 1$. If every non-zero element $x \in R$ is a unit, then $R$ is said to be a \textbf{division ring}\index{division ring}.
\end{definition}

\begin{definition}
    A commutative division ring where $0 \neq 1$ is called a \textbf{field}\index{field}.
\end{definition}
We look at fields in more detail in Volume III.

\begin{exercise}\label{exercise-Z-is-not-a-field}
    Is $\Z$ a field?
\end{exercise}

\section{Subrings}
We end this chapter off with an exploration about subrings.

\begin{definition}
    Let $R$ be a ring and $S$ be a subset of $R$. Then $S$ is a \textbf{subring}\index{subring} of $R$ if
    \begin{itemize}
        \item $(S, +) \leq (R, +)$, that is, the subset $S$ under addition is a subgroup of $R$ under addition; and
        \item for all $a$ and $b$ in $S$ we have $ab \in S$, i.e. $S$ is closed under multiplication.
    \end{itemize}
\end{definition}
\begin{remark}
    Alternatively, one may show that $S$ is a ring to prove that $S$ is a subring of $R$.
\end{remark}

\begin{example}
    We know that $\Z$ and $\Q$ are rings, and clearly $\Z \subseteq \Q$. Hence $\Z$ is a subring of $\Q$. Similarly, since $\Q \subseteq \R$, thus $\Q$ is a subring of $\R$.
\end{example}

\begin{example}
    Consider the ring of \textbf{complex numbers}\index{complex numbers},
    \[
        \C = \{x + yi \vert x, y \in \R\},
    \]
    and the set of \textbf{gaussian integers}\index{gaussian integers},
    \[
        \Z[i] = \{a + bi \vert a,b\in\mathbb{Z}\},
    \]
    where $i$ is the imaginary unit (i.e., $i^2 = -1$). We will show that $\Z[i]$ is a subring of $\C$.
    \begin{proof}
        Clearly $\Z[i] \subseteq \C$. We show that $(\Z[i], +) \leq (\C, +)$.
        \begin{itemize}
            \item Clearly the identity of $(\C, +)$, which is 0, is inside $(\Z[i], +)$ as $0 = 0 + 0i$.
            \item For any $a + bi, c+di \in \Z[i]$, we have $(a+bi) + (-(c+di)) = (a-c) + (b-d)i \in \Z[i]$.
        \end{itemize}
        Thus the subgroup test (\myref{thrm-subgroup-test}) tells us $(\Z[i], +) \leq (\C, +)$.

        Now we show that $\Z[i]$ is closed under multiplication. Let $a + bi, c+di \in \Z[i]$. Then their product is
        \begin{align*}
            (a+bi)(c+di) &= ac+adi+bci+bdi^2\\
            &= (ac-bd) + (ad+bc)i\\
            &\in \Z[i]
        \end{align*}
        so $\Z[i]$ is closed under multiplication. Therefore $\Z[i]$ is a subring of $\C$.
    \end{proof}
\end{example}
\begin{remark}
    One sees that $\R \subseteq \C$ so $\R$ is actually a subring of $\C$.
\end{remark}
\begin{exercise}
    Show that
    \[
        R = \left\{\begin{pmatrix}a&a\\a&a\end{pmatrix} \vert a \in \R\right\}
    \]
    is a subring of $\Mn{2}{\mathbb{R}}$.
\end{exercise}

\newpage

\section{Problems}
\begin{problem}
    Let $R$ be a ring. Prove that if $u \in R$ is a unit then so is $-u$.
\end{problem}

\begin{problem}
    Prove that the trivial ring is the unique ring with identity in which $0 = 1$.
\end{problem}

\begin{problem}
    Let $(R, \ast)$ be a magma. If $(R, \ast, \ast)$ is a ring, describe the elements in $R$.
\end{problem}

\begin{problem}
    Prove that it is impossible for an element of a ring $R$ to be both a zero divisor and a unit.
\end{problem}

\begin{problem}
    Let $R$ be a ring with identity 1, and let $x$ be an element from that ring.
    \begin{partquestions}{\roman*}
        \item Find \textbf{four} closed forms for the geometric series $1 + x + x^2 + x^3 + \cdots + x^n$.
        \item What are the condition(s) such that the closed forms are valid?
        \item Evaluate 112 in the ring $\Z_{37}$.
        \item Hence, using the result(s) above, evaluate
        \[
            1 + 2^3 + 2^6 + 2^9 + \cdots + 2^{72}
        \]
        in the ring $\Z_{37}$.
    \end{partquestions}
\end{problem}

\begin{problem}
    Show that
    \[
        \Q[\sqrt2] = \{a + b\sqrt2 \vert a,b \in \Q\}
    \]
    is a ring. Hence show it is a field.
\end{problem}

\begin{problem}
    Let
    \[
        R = \left\{\begin{pmatrix}a&b\\0&0\end{pmatrix} \vert a,b \in \R\right\}    
    \]
    be a ring under matrix addition and multiplication.
    \begin{partquestions}{\roman*}
        \item Show that $R$ has no identity.
        \item Show that $R$ contains a non-trivial subring $S$ with identity.
    \end{partquestions}
\end{problem}

\begin{problem}
    A ring $R$ is called a \textbf{Boolean ring}\index{Boolean ring} if $r^2 = r$ for all $r \in R$.
    \begin{partquestions}{\roman*}
        \item Show that $r = -r$ for all $r \in R$.
        \item Prove that every Boolean ring is commutative.
    \end{partquestions}
\end{problem}

\begin{problem}
    Let $R$ be a commutative ring with identity. We say that an element $x$ in $R$ is \textbf{nilpotent}\index{nilpotent} if there exists a positive integer $n$ such that $x^n = 0$.
    \begin{partquestions}{\roman*}
        \item Show that the product of two units is a unit.
        \item Let $u \in R$ be a unit and $x \in R$ be nilpotent. Show that $ux$ is nilpotent.
        \item Show that $u - x$ is a unit.
    \end{partquestions}
\end{problem}
