\chapter{Polynomial Rings and Division}
Polynomial rings are an important part of algebra and in ring theory, since polynomials are ubiquitous in modern algebra. We explore polynomials and polynomial rings in this chapter.

\section{What is a Polynomial Ring?}
We first define polynomials.
\begin{definition}
    A \textbf{polynomial}\index{polynomial} is an expression consisting of \textbf{variables}\index{polynomial!variable} (or \textbf{indeterminates}\index{polynomial!indeterminate}) and coefficients, that involves only the operations of addition, subtraction, multiplication, and positive-integer powers of variables.
\end{definition}
\begin{definition}
    Polynomials in a single variable are called \textbf{univariate polynomials}\index{polynomial!univariate} and takes the form
    \[
        a_0+a_1x+a_2x^2+\cdots+a_nx^n = \sum_{i=0}^n a_ix^i,
    \]
    where $a_0, a_1, a_2, \dots, a_n$ are called the \textbf{coefficients}\index{polynomial!coefficient} of the polynomial.
\end{definition}

Note that in the above definition, we did not explicitly state where the coefficients originate from; we do so now in the definition for a polynomial ring.
\begin{definition}
    Let $R$ be a commutative ring with identity, where $1 \neq 0$. Then the \textbf{polynomial ring}\index{polynomial ring} in \textbf{indeterminate}\index{indeterminate} (or \textbf{variable}\index{variable}) $x$ and coefficients in $R$ is
    \[
        R[x] = \{a_0 + a_1x + \cdots + a_nx^n \vert n \in \mathbb{N} \cup \{0\}, a_i \in R\},
    \]
    where for $f(x) = a_0 + \cdots + a_mx^m, g(x) = b_0 + \cdots + b_nx^n \in R[x]$, and assuming $m \leq n$, we define
    \begin{align*}
        f(x) + g(x) &= \sum_{i=0}^n\left((a_i+b_i)x^i\right),\\
        f(x)g(x) &= \sum_{k=0}^{m+n}\left(\left(\sum_{i=0}^k a_ib_{k-i}\right)x^k\right),
    \end{align*}
    where $a_l = 0$ for all $l > m$.
\end{definition}
\begin{proposition}
    A polynomial ring is a commutative ring.
\end{proposition}
\begin{proof}[Proof (cf. {\cite{proofwiki_polynomial-ring-is-commutative}})]
    Let $R[x]$ be a polynomial ring, and let $f(x) = a_0 + \cdots + a_mx^m, g(x) = b_0 + \cdots + b_nx^n \in R[x]$, and $m \leq n$. We show $f(x)g(x) = g(x)f(x)$.
    \begin{align*}
        f(x)g(x) &= \sum_{k=0}^{m+n}\left(\left(\sum_{i=0}^k a_ib_{k-i}\right)x^k\right)\\
        &= \sum_{k=0}^{m+n}\left(\left(\sum_{i=0}^k b_{k-i}a_i\right)x^k\right) & (R\text{ is commutative})\\
        &= \sum_{k=0}^{m+n}\left(\left(\sum_{i=0}^k b_i a_{k-i}\right)x^k\right)\\
        &= g(x)f(x)
    \end{align*}
    which therefore means that $R[x]$ is a commutative ring.
\end{proof}
\begin{exercise}
    Let $R$ be a ring. The \textbf{evaluation homomorphism}\index{evaluation homomorphism} is $\phi_a: R[x] \to R$ where $\phi_a(p(x)) = p(a)$ and $a \in R$. Prove that $\phi_a$ is indeed a ring homomorphism.
\end{exercise}

Let's look at some examples of polynomial rings.
\begin{example}
    The polynomial ring $\R[x]$ is the most familiar for most of us, as this is the `standard' ring of polynomials. Examples of polynomials in this ring include $1+x$, $\sqrt2x^{10} - 5x^3 + \pi x$, and $1+x+x^2+\cdots+x^n$. However, infinite polynomials such as $1+x+x^2+\cdots$ do not belong in $\R[x]$.
\end{example}
\begin{example}
    Another commonly used polynomial ring is $\Q[x]$. Examples of polynomials in this ring are $1+x$, $\frac23x^5 - \frac7{11}x^{13}$, and $2x^2-5x-3$. However polynomials like $\sqrt2$, $\pi x + 1$, and $1+ex$ do not belong in $\Q[x]$.
\end{example}
\begin{example}
    We look at the polynomial ring $\Mn{2}{\R}[x]$. One example of a polynomial in this ring is
    \[
        \begin{pmatrix}2&1\\-2&0\end{pmatrix} + \begin{pmatrix}\sqrt2&1\\1&1\end{pmatrix}x + \begin{pmatrix}5&4\\e&2\end{pmatrix}x^2 + \begin{pmatrix}0&1\\0&0\end{pmatrix}x^5.
    \]
\end{example}

\begin{exercise}
    Let $I$ be a principal ideal of $\Z[x]$ generated by the polynomial $x^2 + 3x - 1$. Simplify $\left((x + 3) + I\right)\left((2x^2 + 3x - 1) + I\right)$ in the quotient ring $\Z[x]/I$.
\end{exercise}

We end this section by noting what form $R[x]$ takes when $x$ is \textit{not} a variable.
\begin{example}
    Recall that $\Q[\sqrt2] = \{a + b\sqrt2 \vert a,b \in \Q\}$. If we use the definition of $\Q[x]$, we see that
    \begin{align*}
        &\Q[\sqrt2] \\
        &= \{a_0 + a_1\sqrt2 + a_2(\sqrt2)^2 + a_3(\sqrt2)^3 + \cdots + a_n(\sqrt2)^n \vert a_i \in \Q \}\\
        &= \{a_0 + a_1\sqrt2 + a_2(2) + a_3(2\sqrt2) + \cdots + a_n(\sqrt2)^n \vert a_i \in \Q \}\\
        &= \{(a_0 + 2a_2 + 4a_4 + \cdots) + \sqrt2 (a_1 + 2a_3 + 4a_5 + \cdots) \vert a_i \in \Q\}\\
        &= \{a + b\sqrt2 \vert a,b \in Q\}
    \end{align*}
    which agrees with the previous definition of $\Q[\sqrt2]$.
\end{example}
\begin{example}
    Recall that $\Z[i]$, the gaussian integers, is the set $\{a+bi \vert a,b \in \Z\}$. If we use the definition of $\Z[x]$, we see that
    \begin{align*}
        \Z[i] &= \{a_0 + a_1i + a_2i^2 + a_3i^3 + \cdots + a_ni^n \vert a_i \in \Z\}\\
        &= \{a_0 + a_1i + a_2(-1) + a_3(-i) + \cdots + a_ni^n \vert a_i \in \Z\}\\
        &= \{(a_0 - a_2 + a_4 - \cdots) + i(a_1 - a_3 + a_5 - \cdots) \vert a_i \in \Z\}\\
        &= \{a + bi \vert a,b \in \Z\}
    \end{align*}
    which agrees with the previous definition of $\Z[i]$.
\end{example}

\section{Basic Terminology in Polynomial Rings}
We define the degree of a polynomial.
\begin{definition}
    Let $R[x]$ be a polynomial ring. The \textbf{degree}\index{degree} of a polynomial $f(x) \in R[x]$, denoted $\deg f(x)$, is the largest integer $k$ such that the coefficient of $x^k$ of $f(x)$ is non-zero.
\end{definition}
\begin{remark}
    For the zero polynomial (0), the degree is undefined.
\end{remark}
\begin{example}
    The degree of the polynomial $1+x+5x^2$ in $\Z[x]$ is 2.
\end{example}
\begin{example}
    The degree of the polynomial
    \[
        \begin{pmatrix}0&1\\3&0\end{pmatrix}x^5 + \begin{pmatrix}3&6\\7&2\end{pmatrix}x^4 + \begin{pmatrix}3&4\\9&4\end{pmatrix}
    \]
    in $\Mn{2}{\Z}[x]$ is 5.
\end{example}
\begin{exercise}
    Give an example of a degree 5 polynomial in the ring $\Z_2[x]$.
\end{exercise}

\begin{definition}
    Let $f(x) = a_0 + a_1x + a_2x^2 + \cdots + a_nx^n$ be a polynomial in the polynomial ring $R[x]$.
    \begin{itemize}
        \item The \textbf{constant term}\index{constant term} is $a_0$.
        \item The \textbf{leading term}\index{leading term} is the term $a_nx^n$.
        \item The \textbf{leading coefficient}\index{leading coefficient} is $a_n$.
    \end{itemize}
\end{definition}
\begin{remark}
    For the zero polynomial, the constant term is 0, the leading term is undefined, and the leading coefficient is undefined.
\end{remark}
\begin{example}
    Consider the polynomial
    \[
        \begin{pmatrix}0&1\\3&0\end{pmatrix}x^5 + \begin{pmatrix}3&6\\7&2\end{pmatrix}x^4 + \begin{pmatrix}3&4\\9&4\end{pmatrix}
    \]
    in $\Mn{2}{\Z}[x]$. Then
    \begin{itemize}
        \item the constant term is $\begin{pmatrix}3&4\\9&4\end{pmatrix}$;
        \item the leading term is $\begin{pmatrix}0&1\\3&0\end{pmatrix}x^5$; and
        \item the leading coefficient is $\begin{pmatrix}0&1\\3&0\end{pmatrix}$.
    \end{itemize}
\end{example}

\begin{definition}
    A \textbf{constant polynomial}\index{constant polynomial} is either the zero polynomial of a polynomial of degree 0.
\end{definition}
\begin{remark}
    This definition immediately implies that any constant polynomial in the polynomial ring $R[x]$ is an element of $R$.
\end{remark}

\section{Properties of Polynomials and Polynomial Rings}
We can now state the theorem that produces a condition for a ring to be an integral domain.
\begin{theorem}\label{thrm-integral-domain-iff-polynomial-ring-is-also}
    Let $R$ be a ring. Then $R$ is an integral domain if and only if $R[x]$ is an integral domain.
\end{theorem}
\begin{proof}
    We first need to show that $R$ is a commutative ring with identity if and only if $R[x]$ is a commutative ring with identity. We leave this for \myref{exercise-commutative-ring-with-identity-iff-polynomial-ring-is-also} (later). We only prove that $R$ has no zero divisors if and only if $R[x]$ has no zero divisors using a contrapositive proof.

    For the forward direction, take non-zero $a$ and $b$ in $R$ such that $ab = 0$. We may view both $a$ and $b$ as degree 0 polynomials in $R[x]$. Clearly these two multiply together to form the zero polynomial in $R[x]$, meaning that they are zero divisors in $R[x]$.

    For the reverse direction, take non-zero polynomials $f(x)$ and $g(x)$ in $R[x]$ such that $f(x)g(x) = 0$. Write
    \begin{align*}
        f(x) = a_0+a_1x+a_2x^2+\cdots+a_mx^m \text{ with } a_m \neq 0\\
        g(x) = b_0+b_1x+b_2x^2+\cdots+b_nx^n \text{ with } b_n \neq 0
    \end{align*}
    where all coefficients are in $R$. Multiplying them together yields something like
    \[
        a_mb_nx^{m+n} + (\text{A polynomial with degree less than }m+n) = 0
    \]
    which hence means that all coefficients must be zero. Therefore $a_mb_n = 0$. This means that we have found non-zero elements $a_m$ and $b_n$ in $R$ such that their product is zero, meaning that they are zero divisors.

    This completes the proof.
\end{proof}
\begin{exercise}\label{exercise-commutative-ring-with-identity-iff-polynomial-ring-is-also}
    Let $R$ be a ring.
    \begin{partquestions}{\alph*}
        \item Prove that $R$ is a ring with identity if and only if $R[x]$ is a ring with identity.
        \item Prove that $R$ is a commutative ring if and only if $R[x]$ is a commutative ring.
    \end{partquestions}
\end{exercise}

\newpage

\section{Problems}
\begin{problem}
    Show that $\princ{x}$ is a prime ideal in $\Z[x]$.
\end{problem}
\begin{problem}
    Let $I = \{f(x) \in \Z[x] \vert f(-2) = 0\}$ be a subset of $\Z[x]$, and let the map $\phi:\Z[x]\to\Z, f(x) \mapsto f(-2)$.
    \begin{partquestions}{\roman*}
        \item Show that $\phi$ is a ring homomorphism.
        \item Show that $I$ is an ideal of $\Z[x]$.
        \item Hence determine if the ideal $I$ is prime, maximal, or both.
    \end{partquestions}
\end{problem}
\begin{problem}
    Prove that $\Z[x] / \princ{x} \cong \Z$.
\end{problem}
