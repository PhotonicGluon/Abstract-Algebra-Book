\chapter{Functions / Maps}
Functions (or maps) play a fundamental role in mathematics. Functions compare and relate different kinds of mathematical structures to each other, and provide a way to relate elements from one structure to another.

\section{Introduction to Functions / Maps}
\begin{definition}
    A \textbf{function}\index{function} (or a \textbf{map}\index{map}) $f$ from a set $X$ to a set $Y$ assigns each value in $X$ to exactly one element in $Y$, and is denoted by $f: X \to Y$.
\end{definition}
\begin{definition}
    For a function $f: X \to Y$, the set $X$ is called the \textbf{domain}\index{domain} of the function and the set $Y$ is called the \textbf{codomain}\index{codomain} of the function.
\end{definition}
\begin{example}
    Consider the simple function $f: \mathbb{Z} \to \mathbb{Q}$ where $f(n) = \frac1n$. In this case, $f$ has a domain of $\mathbb{Z}$, i.e. the integers, and a codomain of $\mathbb{Q}$, i.e. the rational numbers.

    We may evaluate the function $f$ at $2 \in \mathbb{Z}$ to get the resulting value of $f(2) = \frac12$.
\end{example}

\textbf{Arrow notation}\index{function!arrow notation} can also be used to define the rule of a function. There is no good way of defining how to use arrow notation, but some examples should help illustrate the basic idea.
\begin{example}
    Consider $f: \mathbb{N} \to \mathbb{Q}$ where $f(n) = \frac1n$. We may write this more succinctly as $f: \mathbb{N} \to \mathbb{Q}, n \mapsto \frac1n$. Specifically, $n \mapsto \frac1n$ is read as ``$n$ maps to $\frac1n$''.
    
    It is important to note that $\to$ is used to indicate the domain and codomain, and that $\mapsto$ indicates how an element in the domain is `transformed' into an element in the codomain.
\end{example}
\begin{example}
    The function $g: \mathbb{R} \to \mathbb{R}$ where $g(x) = x^2 - 2x + 1$ can be more succinctly written as $g: \mathbb{R} \to \mathbb{R}, x \mapsto x^2 - 2x + 1$.
\end{example}
\begin{example}
    Let $h: \mathbb{Z} \to \mathbb{R}$ where $h(x^2) = x$. This can be written succinctly as either $h: \mathbb{Z} \to \mathbb{R}, x^2 \mapsto x$ or $h: \mathbb{Z} \to \mathbb{R}, n \mapsto \sqrt n$.
\end{example}

We now look at the definition of the image and range.
\begin{definition}
    Let $f: X \to Y$ be a function, and $x \in X$.
    \begin{itemize}
        \item The \textbf{image}\index{function!image} of an element $x \in X$ under the function $f$ is denoted $f(x)$ and is defined to be the value after applying $f$ to $x$.
        \item The \textbf{image} or \textbf{range}\index{function!range} of $f$ is denoted by either $\im f$ or $f(X)$ and is the set of the images of all elements in the domain.
    \end{itemize}
\end{definition}
\begin{example}
    Consider the function $f: \mathbb{Z} \to \mathbb{Z}, n \mapsto 1$.
    \begin{itemize}
        \item The image of 0 under $f$ is 1.
        \item The range/image of $f$ is the \textit{set} $\{1\}$, i.e. $\im f = f(\mathbb{Z}) = \{1\}$.
    \end{itemize}
    It is important to note that the image of an element is a single element, while the image of the function is a set.
\end{example}
\begin{example}
    Consider the function $g: \mathbb{Z} \to \mathbb{Z}, n \mapsto |n|$, where $|n|$ denotes the absolute value of $n$.
    \begin{itemize}
        \item The image of 2 under $g$ is $|2| = 2$.
        \item The image of -3 under $g$ is $|-3| = 3$.
        \item The image of 0 under $g$ is $|0| = 0$.
    \end{itemize}
    The range of the function $g$ is the set of non-negative integers, i.e. $\im g = g(\mathbb{Z}) = \mathbb{N} \cup \{0\}$.
\end{example}

\begin{exercise}
    Let a function $f: \{1, 2, 3\} \to \{1, 4, 9, 16, 25\}$ be given by the relation $f(x) = x^2$.
    \begin{partquestions}{\roman*}
        \item Use arrow notation to write a definition for $f$.
        \item State the domain, codomain, and range of $f$.
        \item What is the image of 2 under $f$?
        \item Is the function $g: \{1, 2, 3\} \to \{1, 8\}, x \mapsto x^3$ \textit{valid}?
    \end{partquestions}
\end{exercise}

We end this section with defining equality of two functions.
\begin{definition}
    Let $f: A \to B$ and $g: C \to D$ be functions. Then $f$ and $g$ are \textbf{equal}\index{function!equality} if and only if
    \begin{itemize}
        \item $A = C$ and $B = D$; and
        \item for all $x \in A = C$, we have $f(x) = g(x)$.
    \end{itemize}
    We denote $f = g$ if the two functions are equal.
\end{definition}
In other words, two functions $f$ and $g$ are equal if their domain and codomain sets are the same and their output values agree on the whole domain.
\begin{example}
    Consider the functions $f: \mathbb{Z} \to \mathbb{Z}, x \mapsto (x-1)^2$ and $g: \mathbb{Z} \to \mathbb{Z}, x \mapsto x^2 - 2x + 1$. Since the two functions' domains and codomains are the same, and because $(x-1)^2 = x^2 - 2x + 1$, thus $f = g$.
\end{example}
\begin{example}
    The functions $f: \mathbb{Z} \to \mathbb{R}, x \mapsto (x-1)^2$ and $g: \mathbb{Z} \to \mathbb{Q}, x \mapsto x^2 - 2x + 1$ are not equal because their codomains differ.
\end{example}

\section{Well-Defined Functions}
\begin{definition}
    A function $f: X \to Y$ is \textbf{well-defined} if and only if for each $x \in X$ there is a unique $y \in Y$ such that $f(x) = y$.
\end{definition}
Informally, \textit{similar inputs} produce \textit{identical outputs} for a well-defined function.
\begin{remark}
    Functions that are not well-defined are called `ambiguous' or `ill-defined' functions.
\end{remark}
\begin{remark}
    If a function is not well-defined, then it is not a valid function.
\end{remark}

\begin{example}
    Let $S_1$ and $S_2$ be sets, and let $S = S_1 \cup S_2$. Let $f: S \to \{1, 2\}$, such that
    \[
        f(x) = \begin{cases}
            1 & \textrm{ if } x \in S_1\\
            2 & \textrm{ if } x \in S_2
        \end{cases}
    \]
    Then $f$ is well-defined if $S_1 \cap S_2 = \emptyset$. For example, if $S_1 = \{1, 2\}$ and $S_2 = \{3, 4\}$, then $f$ is well-defined.
    
    On the other hand, if $S_1 \cap S_2 \neq \emptyset$, then $f$ is not well-defined. For example, if $S_1 = \{1, 2\}$ and $S_2 = \{2, 3\}$, then $f(2) = 1$ and $f(2) = 2$ simultaneously.
\end{example}

\begin{exercise}
    Is the function
    \[
        f: \mathbb{Q} \to \mathbb{Z},\;\frac pq \mapsto p + q    
    \]
    well-defined?
\end{exercise}

\section{Function Composition}
\begin{definition}
    Let $f: X \to Y$ and $g: Y \to Z$ be functions. Then \textbf{composing $f$ with $g$}\index{function!composition} produces a function $h: X \to Z$ where $h(x) = f(g(x))$. We denote $h = f \circ g$ where $\circ$ is the function composition operator.
\end{definition}
\begin{remark}
    We may also alternatively write $fg$ in place of $f \circ g$.
\end{remark}

It is important to note the following about function composition.
\begin{itemize}
    \item Function composition is associative\index{function!composition!associative}. That is, if $f$, $g$, and $h$ are composable, then $f \circ (g \circ h) = (f \circ g) \circ h$. As parentheses do not change the result, they are usually omitted.
    \item The composition $f \circ g$ is only meaningful if the codomain of $g$ is a subset of the domain of $f$. That is, if $f: A \to B$ and $g: C \to D$, then $f \circ g$ is only meaningful if $\im g \subseteq A$.
\end{itemize}

\begin{exercise}
    Let $f: \mathbb{R} \to \mathbb{R}$ and $g: \mathbb{R} \to \mathbb{R}$. Write down the rule of the function $fg$ if $f(x) = x^2 - x + 1$ and $g(y) = \frac1{y^2+1}$.
\end{exercise}

\section{Injective, Surjective, and Bijective Functions}
\begin{definition}
    A function $f: X \to Y$ is \textbf{injective}\index{function!injective} (or \textbf{one-to-one}\index{function!one-to-one}) if $f(x_1) = f(x_2)$ implies $x_1 = x_2$.
\end{definition}
\begin{remark}
    Equivalently, if $x_1 \neq x_2$ then $f(x_1) \neq f(x_2)$ for all $x_1$ and $x_2$ in $X$.
\end{remark}
\begin{example}
    Consider $f: \mathbb{N} \to \mathbb{N}, n \mapsto n^2$. We show that $f$ is injective.
    
    Note that if $n_1, n_2 \in \mathbb{N}$ are such that $f(n_1) = n_1^2 = f(n_2) = n_2^2$ then $n_1 = n_2$ (since $n_1, n_2 > 0$ so taking the square root is okay). Thus $f$ is injective.
\end{example}
\begin{example}
    Consider instead $g: \mathbb{Z} \to \mathbb{Z}, n \mapsto n^2$. Then $g$ is not injective since $g(-2) = g(2) = 4$.
\end{example}

\begin{definition}
    A function $f: X \to Y$ is \textbf{surjective}\index{function!surjective} (or \textbf{onto}\index{function!onto}) if for every $y \in Y$, there exists an $x \in X$ (called the \textbf{pre-image}\index{function!pre-image} of $y$) such that $f(x) = y$.
\end{definition}
\begin{remark}
    Equivalently, the image of $f$ is equal to its codomain, i.e. $\im f = Y$.
\end{remark}
\begin{example}
    Let $S$ denote the set of non-negative real numbers, i.e. $S = \{x\in\mathbb{R} | x \geq 0\}$. Consider the function $f: \mathbb{R} \to S, x \mapsto x^2$. We show that $f$ is surjective.

    Let $y \in S$. Note that $\sqrt{y} \in \mathbb{R}$ since $y$ is a non-negative real number. Observe that $f(\sqrt{y}) = (\sqrt y)^2 = y$. Thus any $y \in Y$ has a pre-image $\sqrt y \in X$. Thus $f$ is surjective.
\end{example}
\begin{example}
    Consider instead the function $g: \mathbb{R} \to \mathbb{R}, x \mapsto x^2$. Then $g$ is not surjective because there is no real number $x \in \mathbb{R}$ such that $g(x) = -1$.
\end{example}

\begin{definition}
    A function is \textbf{bijective}\index{function!bijective} (or a \textbf{bijection}\index{bijection} or a \textbf{one-to-one correspondence}\index{function!one-to-one!correspondence}) if the function is both injective and surjective.
\end{definition}
\begin{example}
    Consider the function $f: \mathbb{R} \to \mathbb{R}, x \mapsto x^3$. We show that $f$ is bijective.
    \begin{itemize}
        \item \textbf{Injective}: Let $a, b \in \mathbb{R}$ such that $f(a) = f(b)$, i.e. $a^3 = b^3$. Clearly we may take the cube root on both sides to yield $a = b$, so $f$ is injective.
        \item \textbf{Surjective}: Let $y \in \mathbb{R}$. Set $x=y^{\frac13}$. Note $x \in \mathbb{R}$ and observe that $f(x) = \left(y^{\frac13}\right)^3 = y$. Thus $y$ has a pre-image of $y^{\frac13}$ in $\mathbb{R}$ and so $f$ is surjective.
    \end{itemize}
    Since $f$ is both injective and surjective it is thus bijective.
\end{example}

\begin{definition}
    Let $A$ and $B$ be sets. Then $A$ and $B$ are \textbf{equinumerous}\index{set!equinumerous} if there exists a bijective function $f: A \to B$. In this case, $A$ and $B$ have the same cardinality, i.e., $|A| = |B|$.
\end{definition}
\begin{remark}
    It should be noted that bijectivity is also implied if the function $f: X \to Y$ is injective and the sets $X$ and $Y$ are equinumerous.
\end{remark}

\begin{exercise}
    Define the function $f: \mathbb{N} \to \mathbb{Z}$ such that
    \[
        f(x) = \begin{cases}
            \frac{x}{2} & \text{ if } x \text{ is even}\\
            \frac{1-x}{2} & \text{ if } x \text{ is odd} 
        \end{cases}
    \]
    By considering $f$, prove that $|\mathbb{N}| = |\mathbb{Z}|$.
\end{exercise}
