\chapter{Mathematical Logic and Proof Writing}
The heart of mathematics is in its logical statements. These statements are used to produce a series of logical steps to prove a claim. We explore the basics of logic and proof writing here.

\section{Mathematical Statements}
\begin{definition}
    A \textbf{(mathematical) statement}\index{statement} (or \textbf{proposition}\index{proposition}) is a sentence that is definitely true or definitely false, but not both.
\end{definition}
\begin{remark}
    A statement can be written in English, or using mathematical notation.
\end{remark}
\begin{example}
    The sentence ``every square with length $x$ has area $x^2$'' is a true mathematical statement.
\end{example}
\begin{example}
    The sentence ``every circle with radius $x$ has area $x^2$'' is a false mathematical statement.
\end{example}
\begin{example}
    The sentence ``$12 \in \mathbb{Z}$'' is a true mathematical statement.
\end{example}
\begin{example}
    The sentence ``$\sqrt2 \in \mathbb{Z}$'' is a false mathematical statement.
\end{example}

We may name statements using variables like $P$, $Q$, $R$, etc.
\begin{example}
    If
    \begin{align*}
        P: &\ \text{Every odd number is one more than an even number}\\
        Q: &\ \text{Every triangle has sides of equal length}\\
        R: &\ \frac12 \in \mathbb{Q}
    \end{align*}
    then $P$ is true, $Q$ is false, and $R$ is true.
\end{example}

There are a few operations\index{logical operation} that may be carried out on mathematical statements. For the following, assume $P$ and $Q$ are mathematical statements.
\begin{itemize}
    \item \textbf{Logical NOT}\index{logical operation!NOT}: Uses the symbol $\lnot$. The statement $\lnot P$ is read as ``not $P$''.
    \item \textbf{Logical AND}\index{logical operation!AND}: Uses the symbol $\land$. The statement $P\land Q$ is read as ``$P$ and $Q$''.
    \item \textbf{Logical OR}\index{logical operation!OR}: Uses the symbol $\lor$. The statement $P\lor Q$ is read as ``$P$ or $Q$''.
    \item \textbf{Conditional}\index{logical operation!conditional}: Uses the symbol $\implies$. The statement $P \implies Q$ is read as ``$P$ implies $Q$''.  
\end{itemize}
\begin{example}
    If $P$ is the statement ``3 is an odd number'' and $Q$ is the statement ``4 is an odd number'', then
    \begin{itemize}
        \item $P\land Q$ is ``3 is an odd number \textbf{and} 4 is an odd number'', which is false;
        \item $P\lor Q$ is ``3 is an odd number \textbf{or} 4 is an odd number'', which is true; and
        \item $\lnot Q$ is ``4 is \textbf{not} an odd number'', which is true.
    \end{itemize}
\end{example}
\begin{exercise}
    Let $P$ be ``1 is a positive number'', $Q$ be ``$-1 > 0$'', and $R$ be ``1 is an odd number''. Is the statement ``$\lnot((P\lor Q)\land R)$ is false'' true?
\end{exercise}

We use \textbf{truth tables}\index{truth table} to explore the relationships between statements and operators. They list all possibilities of the truth or falsity of the statements $P$ and $Q$, and then write the truth for each of the combinations with operations. In a truth table, we denote true statements by ``T'' and false statements by ``F''. For example, the truth table for the logical AND operator is:
\begin{table}[h]
    \centering
    \begin{tabular}{|l|l||l|}
        \hline
        $P$ & $Q$ & $P\land Q$ \\ \hline
        F   & F   & F          \\ \hline
        F   & T   & F          \\ \hline
        T   & F   & F          \\ \hline
        T   & T   & T          \\ \hline
    \end{tabular}
\end{table}

The truth table for the logical OR operator is:
\begin{table}[h]
    \centering
    \begin{tabular}{|l|l||l|}
        \hline
        $P$ & $Q$ & $P\lor Q$ \\ \hline
        F   & F   & F         \\ \hline
        F   & T   & T         \\ \hline
        T   & F   & T         \\ \hline
        T   & T   & T         \\ \hline
    \end{tabular}
\end{table}

The truth table for the logical NOT operator is:
\begin{table}[h]
    \centering
    \begin{tabular}{|l||l|}
        \hline
        $P$ & $\lnot P$ \\ \hline
        F   & T         \\ \hline
        T   & F         \\ \hline
    \end{tabular}
\end{table}

We motivate the truth table for the conditional by considering the statements ``you pass the exam'' and ``you pass the course'', which we will denote by $P$ and $Q$ respectively. So $P \implies Q$ would be ``if you pass the exam then you pass the course''.
\begin{itemize}
    \item If $P$ and $Q$ are true, then that means that you passed the exam and passed the course. Hence, ``if you pass the exam then you pass the course'' is \textbf{true}, meaning $P \implies Q$ is true.
    \item If $P$ is true and $Q$ is false, then that means that you passed the exam but failed the course. Hence, the promise that ``if you pass the exam then you pass the course'' is broken, meaning $P \implies Q$ is false.
    \item Now consider the third case when $P$ is false but $Q$ is true. This means that you failed the exam but passed the course. This does not mean that the promise was broken; you could have passed the course through other means. The only promise was that if you pass the exam then you pass the course; the promise was not that passing the exam was the \textit{only} way of passing the course. Since the promise was not broken, thus the promise was kept, so $P \implies Q$ is true.
    \item Finally we consider the case when $P$ and $Q$ are both false: you failed the exam and failed the course. The promise certainly was not broken in this case, so $P \implies Q$ is true.
\end{itemize}
\begin{remark}
    A conditional statement that is true by the virtue of the fact that $P$ is false is called \textbf{vacuously true}\index{vacuously true}.
\end{remark}

\newpage

In summary, the truth table for the conditional is:
\begin{table}[h]
    \centering
    \begin{tabular}{|l|l||l|}
        \hline
        $P$ & $Q$ & $P\implies Q$ \\ \hline
        F   & F   & T             \\ \hline
        F   & T   & T             \\ \hline
        T   & F   & F             \\ \hline
        T   & T   & T             \\ \hline
    \end{tabular}
\end{table}

\begin{exercise}
    Let $P$ and $Q$ be statements. Draw the truth table for $P \land (\lnot Q)$.
\end{exercise}

We end this section by introducing the idea of the \textbf{biconditional}\index{logical operation!biconditional}.
\begin{definition}
    Let $P$ and $Q$ be mathematical statements. If both $(P \implies Q)$ and $(Q \implies P)$ are true, then we write $(P \iff Q)$. In other words, $(P \iff Q) \equiv ((P \implies Q) \land (Q \implies P))$.
\end{definition}
\begin{remark}
    The statement $(P \iff Q)$ can be written in several ways in English:
    \begin{itemize}
        \item $P$ if and only if $Q$;
        \item $P$ is a necessary and sufficient condition for $Q$; or
        \item $P$ is equivalent to $Q$.
    \end{itemize}
\end{remark}

\newpage

The truth table for the biconditional is:
\begin{table}[h]
    \centering
    \begin{tabular}{|l|l||l|}
        \hline
        $P$ & $Q$ & $P\iff Q$ \\ \hline
        F   & F   & T         \\ \hline
        F   & T   & F         \\ \hline
        T   & F   & F         \\ \hline
        T   & T   & T         \\ \hline
    \end{tabular}
\end{table}

\begin{exercise}
    Suppose $n$ is an integer. Let $P$ be the statement ``$n$ is a multiple of 5'' and $Q$ be the statement ``the last digit of $n$ is 0 or 5''. Let the statement $R = (P \iff Q)$.
    \begin{partquestions}{\roman*}
        \item Write the statement $R$ in English.
        \item Is the statement $R$ true? Justify your answer.
    \end{partquestions}
\end{exercise}

\section{Properties of Logical Operators}
A central idea of this section is that every mathematical statement can be built from just $\land$, $\lor$, and $\lnot$.

\begin{example}
    We show that $(P \implies Q) \equiv (\lnot P) \lor Q$ by considering the truth table $(\lnot P) \lor Q$.
    \begin{table}[h]
        \centering
        \begin{tabular}{|l|l||l||l|}
            \hline
            $P$ & $Q$ & $\lnot P$ & $(\lnot P) \lor Q$ \\ \hline
            F   & F   & T         & T                  \\ \hline
            F   & T   & T         & T                  \\ \hline
            T   & F   & F         & F                  \\ \hline
            T   & T   & F         & T                  \\ \hline
        \end{tabular}
    \end{table}

    \newpage
    
    By inspection, we see that $(\lnot P) \lor Q$ has the same truth table as $P \implies Q$. Thus $(P \implies Q) \equiv (\lnot P) \lor Q$.
\end{example}
\begin{remark}
    We separate the intermediate value(s) (e.g. $\lnot P$ in the above example) from the rest by drawing a double line to the sides of the intermediate values.
\end{remark}

\begin{example}
    We show that $(P \iff Q) \equiv (P \land Q) \lor ((\lnot P) \land (\lnot Q))$. For brevity, let $R = (\lnot P) \land (\lnot Q)$.
    \begin{table}[h]
        \centering
        \begin{tabular}{|l|l||l|l|l|l||l|}
            \hline
            $P$ & $Q$ & $\lnot P$ & $\lnot Q$ & $P \land Q$ & $R$ & $(P \land Q) \lor R$ \\ \hline
            F   & F   & T         & T         & F           & T   & T                    \\ \hline
            F   & T   & T         & F         & F           & F   & F                    \\ \hline
            T   & F   & F         & T         & F           & F   & F                    \\ \hline
            T   & T   & F         & F         & T           & F   & T                    \\ \hline
        \end{tabular}
    \end{table}

    By inspection of the truth table we establish the required result.
\end{example}

\begin{exercise}
    Show that
    \[
        ((\lnot P) \iff Q) \equiv (P \implies (\lnot Q)) \land ((\lnot Q) \implies P)
    \]
    by drawing a truth table.
\end{exercise}

We note some important properties of logical operators.
\begin{itemize}
    \item \textbf{Contrapositive}\index{contrapositive}: $(P \implies Q) \equiv ((\lnot Q) \implies (\lnot P))$
    \item \textbf{De Morgan's Laws}: \begin{itemize}
        \item $(\lnot (P \land Q)) \equiv ((\lnot P) \lor (\lnot Q))$
        \item $(\lnot (P \lor Q)) \equiv ((\lnot P) \land (\lnot Q))$
    \end{itemize}

    \newpage

    \item \textbf{Commutativity of AND and OR}: \begin{itemize}
        \item $P \land Q \equiv Q \land P$
        \item $P \lor Q \equiv Q \lor P$
    \end{itemize}
    \item \textbf{Associativity of AND and OR}: \begin{itemize}
        \item $P \land (Q \land R) \equiv (P \land Q) \land R$
        \item $P \lor (Q \lor R) \equiv (P \lor Q) \lor R$
    \end{itemize}
    \item \textbf{Distributive Rules}: \begin{itemize}
        \item $P \land (Q \lor R) \equiv (P \land Q) \lor (P \land R)$
        \item $P \lor (Q \land R) \equiv (P \lor Q) \land (P \lor R)$
    \end{itemize}
\end{itemize}
\begin{remark}
    The most important one of these properties would arguably be the contrapositive. We will use this result several times later and in later volumes.
\end{remark}
\begin{exercise}
    Simplify the statement
    \[
        ((P \lor \lnot Q) \land \lnot R) \lor ((P \lor \lnot Q) \land (P \lor R) \land (P \lor \lnot R))
    \]
    into a statement that uses only \textbf{three} operators in total.
\end{exercise}

\section{Quantifiers}
\begin{definition}
    The \textbf{universal quantifier}\index{quantifier!universal} is $\forall$ and is read as ``for all''.
\end{definition}
\begin{example}
    The statement ``for every integer $n$, the integer $2n$ is an even number'' can be written as ``$\forall n \in \mathbb{Z}, 2n$ is even''.
\end{example}

\begin{definition}
    The \textbf{existential quantifier}\index{quantifier!existential} is $\exists$ and is  read as ``there exists''.
\end{definition}
\begin{example}
    The statement ``there exists a subset of $\mathbb{Z}$ which has 10 elements'' is equivalent to ``$\exists S \subseteq \mathbb{Z} \text{ such that } |S| = 10$''.
\end{example}
\begin{remark}
    We usually shorten the ``such that'' as ``s.t.'', so the above statement is written symbolically as ``$\exists S \subseteq \mathbb{Z} \text{ s.t. } |S| = 10$''.
\end{remark}

We can also combine the quantifiers together with logical operators.
\begin{example}
    The statement ``every odd integer is one less than an even integer'' can be written as ``$\forall n \in \mathbb{Z}$, $\exists k \in \mathbb{Z} \text { s.t. } n = 2k - 1$''.
\end{example}
\begin{example}
    Fermat's Last Theorem, which states that for all integers $n\geq 3$ the equation $x^n + y^n = z^n$ has no solution for $x, y, z \in \mathbb{N}$, can be written as
    \[
        \left[(n \in \mathbb{Z}) \land (n \geq 3)\right] \implies \left[\forall x, y, z \in \mathbb{N}, x^n + y^n \neq z^n\right].
    \]
    It should be noted that when translating from English to \textit{symbolic writing}, we may need to change some things around.
\end{example}

\begin{exercise}
    Convert the statement
    \begin{quote}
        for all positive integers $n > 2$ there exist integers $a$ and $b$ such that $a^3 + b^4 = n^5$
    \end{quote}
    into symbolic notation using quantifiers and logical operators.
\end{exercise}

\newpage

We now look at negating quantifiers\index{quantifier!negation}. Note that
\begin{itemize}
    \item $\lnot(\forall x, P(x)) \equiv \exists x \text{ s.t. } \lnot P(x)$; and
    \item $\lnot(\exists x \text{ s.t. } P(x)) \equiv \forall x, \lnot P(x)$.
\end{itemize}
\begin{example}
    Consider the statement ``for every integer $n$, the integer $2n$ is even''. One may write that using quantifiers as
    \[
        \forall n \in\mathbb{Z}, 2n \text{ is even}.
    \]
    Negating the above statement would make it
    \[
        \exists n \in \mathbb{Z}, 2n \text{ is not even},
    \]
    i.e., ``there exists an integer $n$ such that $2n$ is odd''.
\end{example}

We also note that $\lnot(P \implies Q) \equiv P \land \lnot Q$. We leave verifying this identity as an exercise to the reader.

\begin{example}
    Consider the statement ``if $n$ is odd then $n^3$ is odd''. We may write this using logical operators as ``($n$ is odd) $\implies$ ($n^3$ is odd)''. The negation of such a statement would hence be ``$n$ is odd \textbf{and} $n^3$ is \textbf{not} odd'', i.e. ``$n$ is odd and $n^3$ is even''.
\end{example}

\begin{exercise}
    Consider the statement ``if $x$ is a non-zero real number, then there exists a real number $y$ such that $xy = 1$''.
    \begin{partquestions}{\roman*}
        \item Write down two statements $P$ and $Q$ in symbols such that the above statement is $P \implies Q$.
        \item Negate the above statement, writing your answer in symbolic notation.
    \end{partquestions}
\end{exercise}

\newpage

\section{Hierarchy of Mathematical Results}
Mathematical statements often have a certain `tier' attached to them. We look at the hierarchy of some of these `tiers'.
\begin{itemize}
    \item \textbf{Proposition}\index{proposition}: A proposition can be thought of as a general proven result. Equivalent names for a proposition are ``\textbf{Claim}'' or ``\textbf{Observation}''.
    \item \textbf{Lemma}\index{lemma}: A lemma can be thought of as a `small' proven result that can help build other mathematical results. For example, Euclid's lemma that ``if a prime $p$ divides the product $ab$ of two integers $a$ and $b$, then $p$ must divide at least one of those integers $a$ or $b$'' is used in the proof of the Fundamental Theorem of Arithmetic.
    \item \textbf{Theorem}\index{theorem}: A theorem can be thought of as a `big' proven result. For example, the Fundamental Theorem of Arithmetic is core to arithmetic as it describes how integers can be uniquely decomposed into its prime factors.
    \item \textbf{Corollary}\index{corollary}: A corollary is said to be a `follow-up result' from a theorem. These results usually follow `very quickly' from a theorem or a proposition. For example, the AM-GM inequality is a corollary of the Pythagorean theorem.
    \item \textbf{Conjecture}\index{conjecture}: A conjecture is a mathematical statement whose truth or falsity is not known.
\end{itemize}

\section{Mathematical Proof Techniques}
\subsection{Direct Proof}
In essence, a direct proof\index{proof!direct} for the statement ``if $P$ then $Q$'' would involve
\begin{enumerate}
    \item supposing $P$ is true, (usually written as ``suppose $P$'')
    \item unpacking $P$ via definitions; making appropriate calculations and logical arguments; repacking $Q$ via definitions,
    \item concluding that $Q$ is true. (usually written as ``thus $Q$'')
\end{enumerate}
\begin{example}
    We look at a proof of the statement ``if the integer $n$ is odd then $n^2$ is odd''.
    \begin{proof}
        Suppose $n\in\mathbb{Z}$ is an odd number.
        
        Then $n$ can be written in the form $n = 2k + 1$ where $k$ is an integer. Hence $n^2 = (2k+1)^2 = 4k^2 + 4k + 1$. Note $4k^2 + 4k + 1 = 2(2k^2 + 2k) + 1$, so $n^2$ is one more than a multiple of 2.
        
        Thus $n^2 = 2(2k^2 + 2k) + 1$ is odd.
    \end{proof}
\end{example}

\begin{example}
    We look at a proof of the statement ``if $n$ is an integer then $1 + (-1)^n(2n-1)$ is a multiple of 4''.
    \begin{proof}
        Suppose $n$ is an integer. Then, $n$ is either odd or even. We look at two cases.
        \begin{itemize}
            \item When $n$ is odd, $(-1)^n = -1$ and $n = 2k+1$ for some integer $k$. Thus
            \[
                1 + (-1)^n(2n-1) = 1 - (2(2k+1)-1) = -4k
            \]
            which is a multiple of 4.
            \item When $n$ is even, $(-1)^n = 1$ and $n = 2k$ for some integer $k$. Thus
            \[
                1 + (-1)^n(2n-1) = 1 + (2(2k)-1) = 4k
            \]
            which is a multiple of 4.
        \end{itemize}
        Hence, in both cases, $1 + (-1)^n(2n-1)$ is a multiple of 4.
    \end{proof}
\end{example}

\begin{exercise}
    Prove that $m + n$ is even if the integers $m$ and $n$ have the same parity (i.e., both odd or both even). 
\end{exercise}

\subsection{Contrapositive Proof}
We now look at a \textbf{contrapositive proof}\index{proof!contrapositive}. Recall that $(P \implies Q) = (\lnot Q \implies \lnot P)$. Hence, a contrapositive proof for the statement ``if $P$ then $Q$'' would involve
\begin{enumerate}
    \item supposing $\lnot Q$ is true,
    \item working towards showing that $\lnot P$ is true,
    \item concluding that $\lnot P$ is true.
\end{enumerate}

Generally, we would want to prove in the direction from simplicity to complexity. So if $P$ is more complex than $Q$, we may consider using a contrapositive proof.

\begin{example}\label{example-if-(n-1)(n-5)-is-even-then-n-is-odd}
    Suppose $n$ is an integer. We prove the statement ``if $n^2 - 6n + 5$ is even, then $n$ is odd''. We note that a direct proof would be tedious and problematic. Using a contrapositive proof would be easier.
    
    We first note that the contrapositive statement that we want to prove is ``if $n$ is \textbf{not} odd, then $n^2 - 6n + 5$ is \textbf{not} even'', that is, ``if $n$ is even, then $n^2 - 6n + 5$ is odd''.
    \begin{proof}
        We consider a proof by contrapositive.
        
        Suppose $n$ is even. Then $n = 2k$ where $k$ is an integer. Note
        \begin{align*}
            n^2 - 6n + 5 &= (2k)^2 - 6(2k) + 5\\
            &= 4k^2 - 12k + 5\\
            &= (4k^2 - 12k + 4) + 1\\
            &= 2(2k^2 - 6k + 2) + 1
        \end{align*}
        which means that $n^2 - 6n + 5$ is one more than a multiple of 2, which hence means $n^2 - 6n + 5$ is odd.
    \end{proof}    
\end{example}

\begin{example}
    Suppose $x$ and $y$ are real numbers. We prove the statement ``$x \leq y$ if $x^3 + xy^2 \leq x^2y + y^3$'' using a contrapositive proof.
    
    We first note that the contrapositive statement that we want to prove is ``if $x > y$ then $x^3 + xy^2 > x^2y + y^3$''.

    \begin{proof}
        We consider a proof by contrapositive.
        
        Assume $x > y$. Then $x - y > 0$. Also, since $x > y$, thus $x$ and $y$ are not both zero. Hence $x^2 + y^2 > 0$.
        Observe
        \[
            (x-y)(x^2+y^2) > 0 \times (x^2+y^2) = 0        
        \]
        so $(x-y)(x^2+y^2) = x^3 + xy^2 - x^2y - y^3 > 0$. Therefore $x^3 + xy^2 > x^2y + y^3$.
    \end{proof}
\end{example}

\begin{exercise}
    Suppose that $a$ and $b$ are integers. Prove that $a$ is even or $b$ is odd if $a(b^2 + 5)$ is even.
\end{exercise}

\subsection{Proof by Contradiction}
The third proof technique is called a \textbf{proof by contradiction}\index{proof!contradiction}. This method can be used to prove any kind of statement. The basic idea is to assume that the statement we want to prove is false, and then show that this assumption leads to a contradiction. A proof by contradiction for the statement ``$P$'' (yes, just $P$) would involve
\begin{enumerate}
    \item supposing $\lnot P$ is true,
    \item working towards forming a statement $C$ such that another statement $C \land \lnot C$ is formed,
    \item observing that $C \land \lnot C$ is impossible (since $C \land \lnot C$ is always false),
    \item concluding that $P$ is true.
\end{enumerate}
Usually, when writing a proof by contradiction, we would like to inform the reader that a proof by contradiction is being employed. Language such as ``by way of contradiction'', ``towards a contradiction'', ``suppose for the sake of contradiction'' etc. may be used to signpost the use of a proof by contradiction.
\begin{remark}
    Some authors would also signal the use of contradiction by using the initialism ``BWOC'' (by way of contradiction). 
\end{remark}

\begin{example}\label{example-sqrt2-is-irrational}
    We prove the classic result that ``$\sqrt 2$ is irrational'' via a proof by contradiction.
    \begin{proof}
        By way of contradiction, assume that $\sqrt2 = \frac ab$ for some integers $a$ and $b$. Furthermore let this fraction be fully reduced; in particular, this means that $a$ and $b$ are not both even. Squaring both sides yields $2 = \frac{a^2}{b^2}$, meaning  $a^2 = 2b^2$. Hence $a^2$ is even, so write $a = 2c$ where $c$ is an integer. This leads to $2b^2 = (2c)^2 = 4c^2$ which implies $b^2 = 2c^2$. Hence $b$ is even, which contradicts the fact that $a$ and $b$ are not both even.
        
        Hence, $\sqrt 2$ is irrational.
    \end{proof}
\end{example}
\begin{remark}
    It is not necessary to have the final statement that ``$\sqrt 2$ is irrational'' (or, more generally, ``$P$ is true'') as it is implied from the proof by contradiction.
\end{remark}

\begin{example}
    We prove the statement that ``for every positive rational number $x$, there exists a positive rational number $y$ such that $y < x$'' by way of contradiction.
    
    We note that the negation of the above statement is ``there exists a rational number $x$ such that for every positive rational number $y$ we have $y \geq x$''.
    \begin{proof}
        Suppose for the sake of contradiction that there exists a rational number $x$ such that for every positive rational number $y$ we have $y \geq x$. Write $x = \frac pq$ where $p$ and $q$ are positive integers.
        
        Now consider the rational number $\frac{p-1}{q}$. Clearly $\frac{p-1}{q} < \frac pq = x$. By assumption, every positive rational number $y$ satisfies $y \geq x$. Hence, $\frac{p-1}{q}$ is non-positive, meaning $\frac{p-1}{q} \leq 0$. Since $q$ is positive, hence $p - 1 \leq 0$ which means $p \leq 1$. But as $p$ is a positive integer, we conclude $p = 1$. Hence $x = \frac 1q$.
        
        We now consider the rational number $\frac{1}{q+1}$. Clearly $\frac{1}{q+1} < \frac{1}{q} = x$. By assumption we must conclude that $\frac{1}{q+1}$ is non-positive. However, $1 > 0$ and $q + 1 > 0$, so $\frac{1}{q+1}$ is positive. Hence we have the fact that $\frac{1}{q+1}$ is positive and non-positive simultaneously, leading to a contradiction.
    \end{proof}
\end{example}
\begin{remark}
    The statement above is one where a direct proof would be easier. We provide a direct proof of it below.
    \begin{proof}
        Since $x$ is a positive rational number write $x = \frac pq$ where $p$ and $q$ are positive integers. Then set $y = \frac{p}{q+1}$. Clearly $\frac{p}{q+1} < \frac{p}{q} = x$ and $\frac{p}{q+1}$ is positive, hence we have found a $y$ such that $y < x$.
    \end{proof}
    However, we use this statement as an example for how a proof by contradiction can be constructed.
\end{remark}

\begin{exercise}
    Prove that there exist no integers $a$ and $b$ such that $2a + 4b = 1$.
\end{exercise}

We now look at a proof by contradiction for conditional statements. Recall that $(\lnot(P \implies Q)) \equiv (P \land \lnot Q)$. Hence, to prove the statement ``if $P$ then $Q$'', we would
\begin{enumerate}
    \item suppose that $P \land \lnot Q$ is true,
    \item work towards a statement $C$ such that another statement $C \land \lnot C$ is formed,
    \item observe that $C \land \lnot C$ is impossible,
    \item conclude that $P \implies Q$.
\end{enumerate}

\begin{example}
    Suppose $a$ and $b$ are real numbers. We prove the statement ``if $a$ is rational and $ab$ is irrational then $b$ is irrational'' using a proof by contradiction.
    
    We note that the statement we want to contradict is ``$a$ is rational and $ab$ is irrational \textbf{and} $b$ is \textbf{not} irrational'', i.e. ``$a$ is rational and $b$ is rational and $ab$ is irrational''.
    \begin{proof}
        By way of contradiction assume $a$ is rational, $b$ is rational, and $ab$ is irrational. We may then write $a = \frac mn$ and $b = \frac pq$ where $m, n, p, q \in \mathbb{Z}$. Hence $ab = \left(\frac mn\right)\left(\frac pq\right) = \frac{mp}{nq}$ which is clearly rational. Therefore we have that $ab$ is irrational (by assumption) and $ab$ is rational, a contradiction.
    \end{proof}
\end{example}

\begin{example}
    Suppose $a$, $b$, and $c$ are integers. We prove the statement that ``if $a^2 + b^2 = c^2$ then at least one of $a$ or $b$ is even'' using a proof by contradiction.
    
    We note that the statement we want to contradict is ``$a^2 + b^2 = c^2$ \textbf{and not} (at least one of $a$ or $b$ is even)'', i.e. ``$a^2 + b^2 = c^2$ \textbf{and} both $a$ and $b$ are odd''.
    \begin{proof}
        Seeking a contradiction, assume that $a^2 + b^2 = c^2$ and both $a$ and $b$ are odd. Thus we may write $a = 2m + 1$ and $b = 2n + 1$ where $m$ and $n$ are integers. Hence
        \begin{align*}
            a^2 + b^2 &= (2m+1)^2 + (2n+1)^2\\
            &= (4m^2+4m+1) + (4n^2+4n+1)\\
            &= 4m^2 + 4n^2 + 4m + 4n + 2\\
            &= 2(2m^2 + 2n^2 + 2m + 2n +1)
        \end{align*}
        which means that $c^2 = a^2 + b^2$ is even. Hence $c$ is even, which means we may write $c = 2k$ where $k$ is an integer. This leads to
        \[
            c^2 = 4k^2 = 2(2m^2 + 2n^2 + 2m + 2n + 1) = a^2 + b^2.
        \]
        Clearly $4k^2$ is a multiple of 4, while $2(2m^2 + 2n^2 + 2m + 2n + 1)$ is not. Yet, they are equal to each other, a contradiction.
    \end{proof}
\end{example}

\begin{exercise}
    Prove that $\frac{a+b}{2} \geq \sqrt{ab}$ if $a$ and $b$ are positive real numbers by way of contradiction.
\end{exercise}

Despite the power of proof by contradiction, it's best to use it only when the direct and contrapositive approaches do not seem to work.
\begin{example}
    Suppose $n$ is an integer. We prove the statement ``if $n^2 - 6n + 5$ is even then $n$ is odd'' using a proof by contradiction.
    \begin{proof}
        Working towards a contradiction, assume $n^2 - 6n + 5$ is even and $n$ is \textbf{not} odd, i.e. $n$ is even. Then $n = 2k$ for some integer $k$. Note that
        \begin{align*}
            n^2 - 6n + 5 &= (2k)^2 - 6(2k) + 5\\
            &= 4k^2 - 12k + 5\\
            &= (4k^2 - 12k + 4) + 1\\
            &= 2(2k^2 - 5k + 2) + 1
        \end{align*}
        which means that $n^2 - 6n + 5$ is odd. Hence, $n^2 - 6n + 5$ is even (by assumption) and $n^2 - 6n + 5$ is odd (as above), a contradiction.
    \end{proof}
    While there is nothing wrong with this proof, notice that part of it assumes that $n$ is even and concludes that  $n^2 - 6n + 5$ is odd, which is the contrapositive approach done in \myref{example-if-(n-1)(n-5)-is-even-then-n-is-odd}.
\end{example}

\subsection{Proof by Mathematical Induction}
Mathematical induction\index{proof!induction} is a method for proving that a proposition $P_n$ is true for every positive integer $n$, that is, that the infinitely many cases $P_1, P_2, P_3, \dots,$ all hold. Informal metaphors help to explain this technique, such as falling dominoes or climbing a ladder:
\begin{quote}
    Mathematical induction proves that we can climb as high as we like on a ladder, by proving that we can climb onto the bottom rung (the base case) and that from each rung we can climb up to the next one (the induction step).
\end{quote}

A proof by induction consists of two steps. The first, the \textbf{base case}\index{proof!induction!base case}, proves the statement for $n = 1$ without assuming any knowledge of other cases. The second, the \textbf{induction step}\index{proof!induction!induction step}, proves that if the statement holds for any given case $n = k$, then it must also hold for the next case $n = k + 1$. These two steps establish that the statement holds for all positive integers $n$.

The base case does not necessarily need to begin with $n = 1$. Sometimes we may begin with $n = 0$, and possibly with any fixed natural number $n = N$, establishing the truth of the statement for all natural numbers $n \geq N$.

In summary, mathematical induction involves two steps:
\begin{itemize}
    \item \textbf{Base Case}: Prove the statement for the initial value.
    \item \textbf{Induction Step}: Prove that for every $n$, if the statement holds for $n$, then it holds for $n + 1$.
\end{itemize}

\begin{example}
    We prove the famous identity
    \[
        1 + 2 + 3 + \cdots + n = \frac{n(n+1)}2
    \]
    using mathematical induction.
    \begin{proof}
        When $n = 1$, the left hand side is 1; the right hand side is $\frac{1(1+1)}{2} = 1$. Thus the initial case is true.

        Now assume that the statement holds for some positive integer $k$, meaning
        \[
            1 + \cdots + k = \frac{k(k+1)}2.
        \]
        We are to prove the statement true for $k+1$, meaning
        \[
            1 + \cdots + k + (k+1) = \frac{(k+1)(k+2)}2.
        \]
        We work slowly:
        \begin{align*}
            1 + \cdots + k + (k+1) &= \frac{k(k+1)}{2} + (k+1) & (\text{by hypothesis})\\
            &= \frac{k(k+1)}2 + \frac{2(k+1)}{2}\\
            &= \frac{k(k+1) + 2(k+1)}2\\
            &= \frac{(k+1)(k+2)}2
        \end{align*}
        which proves the case for $k + 1$. Hence $1 + 2 + 3 + \cdots + n = \frac{n(n+1)}2$.
    \end{proof}
\end{example}

\begin{example}
    Suppose $x > -1$. We will prove that $(1+x)^n \geq 1+nx$ if $n$ is a positive integer.
    \begin{proof}
        When $n = 1$, the left hand side is $(1+x)^1 = 1+x$ which is exactly the right hand side. Thus the base case is true.
        
        Assume that the statement holds for some positive integer $k$, i.e. $(1+x)^k \geq 1+kx$. We show that the statement holds for $k+1$, i.e. $(1+x)^{k+1} \geq 1+(k+1)x$.
        
        We first note that since $x>-1$, thus $1+x > 0$. We start with our induction hypothesis.
        \begin{align*}
            (1+x)^k &\geq 1+kx\\
            (1+x)^k(1+x) &\geq (1+kx)(1+x) & (\text{since }1+x > 0)\\
            (1+x)^{k+1} &\geq 1 + x + kx + kx^2\\
            &= 1+(k+1)x + kx^2\\
            &> 1+(k+1)x
        \end{align*}
        Hence we see $(1+x)^{k+1} \geq 1+(k+1)x$, meaning that the statement is true for $k+1$.
        
        Therefore by mathematical induction we have $(1+x)^n \geq 1+nx$ if $n$ is a positive integer.
    \end{proof}
\end{example}

\begin{exercise}
    Prove by induction that $a^2 - 1$ is a multiple of 8 for all positive odd integers $a$.
\end{exercise}

We now look at another form of mathematical induction, called \textbf{strong induction}\index{proof!induction!strong}. Unlike regular induction, strong induction assumes that all preceding cases are true, and proves the truth of the next case.

Strong mathematical induction involves two steps:
\begin{itemize}
    \item \textbf{Base Cases}: Prove the statement for the initial values.
    \item \textbf{Induction Step}: Prove that for every $n$, if the statement holds for all (positive) integers $m$ that are at most $n$, then it holds for $n + 1$.
\end{itemize}

\begin{example}
    We prove that every integer $n \geq 8$ can be expressed in the form $3a + 5b$ where $a$ and $b$ are non-negative integers.
    \begin{proof}
        We use strong induction on $n$.
        
        We show the base cases of 8, 9, and 10 hold:
        \begin{itemize}
            \item When $n = 8$, note $8 = 3 + 5$.
            \item When $n = 9$, note $9 = 3 \times 3 + 5 \times 0$.
            \item When $n = 10$, note $10 = 3 \times 0 + 5 \times 2$.
        \end{itemize}
        
        Now assume that for some positive integer $k \geq 8$, for every integer $m$ satisfying $8 \leq m \leq k$ the statement holds true, i.e. $m$ can be written in the form $3a + 5b$. We are to show that the statement for $k+1$ is true, i.e. $k+1$ can be expressed in the form $3a + 5b$.
        
        By hypothesis, $k - 2$ can be expressed in the form $3a+5b$. Hence $k+1 = (k-2) + 3 = 3(a+1) + 5b$, proving the statement for $k+1$.
        
        Therefore by mathematical induction, every integer $n \geq 8$ can be expressed in the form $3a + 5b$ where $a$ and $b$ are non-negative integers.
    \end{proof}
\end{example}

\begin{example}\label{example-strong-induction-on-function}
    Consider the function $f: \left(\mathbb{N}\right)^2\to\mathbb{N}$ where
    \[
        f(m, n) =
        \begin{cases}
            n & \text{if } m = 1, \\
            m & \text{if } n = 1, \\
            f\left(n-1,f(n-1,m-1)\right) & \text{otherwise.}
        \end{cases}
    \]
    We will prove the non-obvious fact that $f(n+1, n) = 2$ for all positive integers $n$.
    \begin{proof}
        We show the base cases of 1 and 2 hold:
        \begin{itemize}
            \item When $n = 1$, we have $f(2, 1) = 2$, so the first case is true.
            \item When $n = 2$, we have $f(3, 2)$. Note that
            \[
                f(3,2) = f(1, f(1, 2)) = f(1, 2) = 2
            \]
            so the second case is true.
        \end{itemize}
        
        Now suppose for some positive integer $k$, for every integer $1 \leq m \leq k$ the statement holds true, i.e. $f(m+1,m) = 2$. We want to show that the case for $k+1$ is true, i.e. $f(k+2, k+1) = 2$.
        \begin{align*}
            f(k+2, k+1) &= f(k, f(k, k+1))\\
            &= f(k, f(k, f(k, k-1)))\\
            &= f(k, f(k, 2)) & (\text{hypothesis on } k-1)\\
            &= f(k, f(1, f(1, k-1)))\\
            &= f(k, f(1, k-1))\\
            &= f(k, k-1) \\
            &= 2 & (\text{hypothesis on } k-1)
        \end{align*}
        which proves that the statement for $k+1$ holds. Hence by mathematical induction, $f(n+1, n) = 2$.
    \end{proof}
\end{example}

\begin{exercise}
    Let the function $f: \mathbb{N} \to \mathbb{Z}$ be defined such that $f(1) = 0$, $f(2) = 1$, and $f(n+2) = 3f(n+1) - 2f(n) + 1$ for all positive integers $n$. Prove that $f(n) = 2^n - n - 1$ for all positive integers $n$.
\end{exercise}

\section{Proving Non-Conditional Statements}
\subsection{Biconditional Statements}\index{proof!biconditional}
Recall that a biconditional statement is a statement like ``$P \iff Q$'', i.e., ``$P$ if and only if $Q$''. We prove such a statement by proving that $P \implies Q$ and $Q \implies P$. Each of these statements may be proved using any of the proof techniques that we covered.

\begin{example}
    We will prove the biconditional statement ``the integer $n$ is even if and only if $n^2$ is even''.
    \begin{proof}
        We prove the forward direction ($n$ is even implies $n^2$ is even) first by using direct proof. Assume that $n$ is even. Then we may write $n = 2k$ where $k$ is an integer. Hence $n^2 = (2k)^2 = 4k^2 = 2(2k^2)$ which is even.

        We now prove the reverse direction ($n^2$ is even implies $n$ is even) via a proof by contrapositive. Suppose $n$ is \textbf{not} even, meaning $n$ is odd. Hence $n = 2k + 1$ where $k$ is an integer. Observe $n^2 = (2k+1)^2 = 4k^2 + 4k + 1 = 2(2k^2 + 2k) + 1$ which is odd.
    \end{proof}
\end{example}
\begin{example}
    Suppose $n$ is an integer. We will prove ``$n$ is a multiple of 6 if and only if $n$ is a multiple of 2 and 3''.
    \begin{proof}
        We prove the forward direction first by using direct proof. Assume $n$ is a multiple of 6, meaning $n = 6k$ for some integer $k$. Clearly $6k = 2(3k)$ and $6k = 3(2k)$, so $n$ is both a multiple of 2 and 3.
        
        We now prove the reverse direction, again using direct proof. Assume $n$ is a multiple of 2 and 3, so we may write $n = 2a$ and $n = 3b$ for some integers $a$ and $b$. Then $2a = 3b$. Hence $a = \frac 32 b$ and $b = \frac 23 a$. Since $a$ and $b$ are integers, hence we conclude $b$ is a multiple of 2 and $a$ is a multiple of 3. Write $a = 3p$ and $b = 2q$ where $p$ and $q$ are integers. Hence $n = 2(3p) = 6p$ and $n = 3(2q) = 6q$. In both cases we see $n$ is a multiple of 6.
    \end{proof}
\end{example}

\begin{exercise}
    Let $n$ be an integer. Prove that $n$ is one more than a multiple of 5 if and only if $n$ is of the form $5k - 4$ where $k$ is an integer.
\end{exercise}

\subsection{Existence Statements}
Some statements only assert the existence of something. These statements are called \textbf{existence statements} and one only has to provide a particular example that shows it is true.\index{proof!existence proof} 
\begin{example}
    The statement ``there exists an even prime number'' is readily proven by noticing that 2 is an even prime number.
\end{example}
\begin{example}
    The statement ``an integer that can be expressed as the sum of two perfect cubes in two different ways exists'' is proven by giving the example 1729:
    \begin{itemize}
        \item $1729 = 1^3 + 12^3$; and
        \item $1729 = 9^3 + 10^3$.
    \end{itemize}
\end{example}
Note that while an example suffices to prove an existence statement, a single example does not prove a conditional statement.
\begin{exercise}
    Prove that a positive integer that is one less than a perfect cube and two less than a perfect square exists.
\end{exercise}

\newpage

Existence proofs fall into two categories: \textbf{constructive}\index{proof!constructive} proofs and \textbf{non-constructive}\index{proof!non-constructive} proofs.
\begin{itemize}
    \item Constructive proofs provide an explicit example that proves the statement. We have only seen constructive proofs so far.
    \item Non-constructive proofs prove that an example exists without providing it.
\end{itemize}

\begin{example}
    We prove the classic statement that ``there exist irrational $x$ and $y$ such that $x^y$ is rational'' using a non-constructive proof.
    \begin{proof}
        Let $x = \sqrt2^{\sqrt2}$ and $y = \sqrt2$. We know that $\sqrt2$ is irrational from \myref{example-sqrt2-is-irrational}. Now consider two cases.
        \begin{itemize}
            \item If $x$ is rational, then we have found two irrational numbers (in particular, $\sqrt 2$ and $\sqrt 2$) such that their exponentiation (i.e.,  $x = \sqrt2^{\sqrt2}$) is rational, proving the claim.
            \item If $x$ is irrational, then \[x^y = \left(\sqrt2^{\sqrt2}\right)^{\sqrt2} = (\sqrt2)^{\sqrt2 \times \sqrt2} = (\sqrt2)^2 = 2\]
            is rational.
        \end{itemize}
        Hence, either way, we have an irrational number to an irrational power that is rational.
    \end{proof}
    
    \newpage

    Notice that we did not explicitly prove whether $\sqrt2^{\sqrt2}$ is rational or irrational; we just showed that either case leads to a case where two irrational numbers, when exponentiated, results in a rational number.
\end{example}

\begin{exercise}
    Let $x = \sqrt2$ and $y = 2\log_2{3}$. It may be assumed that $\sqrt2$ is irrational.
    \begin{partquestions}{\roman*}
        \item Prove that $y$ is irrational.
        \item Produce a constructive proof that there exist irrational $x$ and $y$ such that $x^y$ is rational.
    \end{partquestions}
\end{exercise}
