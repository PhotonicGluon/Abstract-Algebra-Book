\chapter{Algebra}
Competent algebraic manipulation in the real numbers is assumed. We collect a few important and useful results when we manipulate algebraic objects.

\section{The Binomial Theorem}
Sometimes we are forced to expand expressions such as $(x-1)^5$, $(3x^2 + 5)^4$, and $(7x - 3)^9$. These expressions can be readily expanded using the \textbf{binomial theorem}\index{Bionomial Theorem}.
\begin{theorem}[Binomial Theorem]\label{thrm-binomial}
    Let $n$ be a non-negative integer. Then
    \[
        (x+y)^n = \sum_{k=0}^n {n \choose k}x^ky^{n-k} = \sum_{k=0}^n {n \choose k}x^{n-k}y^k
    \]
    where
    \[
        {n \choose k} = \frac{n!}{k!(n-k)!}.
    \]
\end{theorem}
We note that ${n \choose k}$ is read as ``$n$ choose $k$'' and is known as the \textbf{binomial coefficient}\index{binomial coefficient}.

\begin{example}
    $(x+1)^4 = x^4 + 4x^3 + 6x^2 + 4x + 1$
\end{example}
\begin{example}
    $(x-1)^5 = x^5-5x^4+10x^3-10x^2+5x-1$
\end{example}
\begin{example}
    $(3x^2 + 5)^4 = 81x^8+540x^6+1350x^4+1500x^2+625$
\end{example}
\begin{exercise}
    Find the coefficient of $x^6$ in $(7x-3)^9$.
\end{exercise}

We note two facts about the binomial coefficient here.
\begin{proposition}
    ${n\choose k} = {n \choose{k-n}}$ for all integers $0 \leq k \leq n$.
\end{proposition}
\begin{proposition}\label{prop-binomial-coefficient-multiple-of-n}
    $n\choose k$ is a multiple of $n$ for all integers $1 \leq k \leq n - 1$.
\end{proposition}
