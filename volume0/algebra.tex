\chapter{Algebra}
Competent algebraic manipulation in the real numbers is assumed. We collect a few important and useful concepts and results when we manipulate algebraic objects.

\section{Summation}
When we are summing many similar terms, we use the capital Greek letter sigma ($\sum$) to represent the sum of similar terms. 
\begin{definition}\index{summation}
    Define
    \[
        \sum_{i=m}^{n}a_i = a_m + a_{m+1} + a_{m+2} + \cdots + a_{n-1} + a_n,
    \]
    where $i$ is the \textbf{index of summation}\index{summation!index}, $a_i$ represents each \textbf{summand}\index{summation!summand} of the sum, $m$ is the \textbf{lower bound of summation}\index{summation!lower bound}, and $n$ is the \textbf{upper bound of summation}\index{summation!upper bound}.
\end{definition}
\begin{remark}
    $\displaystyle \sum_{i=m}^{n}a_i$ is read as ``the sum of $a_i$ from $i=m$ to $n$''.
\end{remark}
\begin{remark}
    The ``$i = m$'' at the bottom of the summation means that the index of summation starts at $m$, increments by 1 for each successive term, and continues until $i = n$.
\end{remark}

\begin{example}
    The sum of the first 10 positive integers can be written as
    \[
        \sum_{i=1}^{10}i = 1 + 2 + \cdots + 9 + 10
    \]
    and is evaluated to 55.
\end{example}

\begin{example}
    The sum of the squares from 3 to 9 can be written as
    \[
        \sum_{i=3}^{9}i^2 = 3^2 + 4^2 + \cdots + 8^2 + 9^2.
    \]
    There is no reason to just use ``$i$'' for the index; we may also use other letters, such as $j$:
    \[
        \sum_{j=3}^{9}j^2 = 3^2 + 4^2 + \cdots + 8^2 + 9^2.
    \]
\end{example}

\begin{example}
    The case where the summation has one summand is just equal to the summand itself. That is,
    \[
        \sum_{i=k}^{k}a_i = a_k.
    \]
    For instance,
    \[
        \sum_{i=4}^4(i+2)^2 = (4+2)^2 = 36.
    \]
\end{example}

There is one degenerate case for summation, which we note as a definition.
\begin{definition}
    Let $k$ and $n$ be integers such that $k > n$. Define
    \[
        \sum_{i=k}^{n}a_i = 0.
    \]
\end{definition}
\begin{example}
    We see
    \[
        \sum_{i=5}^{4}i = \sum_{i=3}^{2}1 = \sum_{i=12}^{3}(i^4 + 2i^3 + 3i^2 + 4i + 5) = 0
    \]
    by virtue of the above definition.
\end{example}
\begin{exercise}
    Evaluate the following sums.
    \begin{multicols}{3}
        \begin{partquestions}{\alph*}
            \item $\displaystyle \sum_{i=3}^{5}(7ix+11)$
            \item $\displaystyle \sum_{x=3}^{5}(7ix+11)$
            \item $\displaystyle \sum_{i=-3}^{-5}(-7ix-11)$
            \item $\displaystyle \sum_{i=4}^{8}ijk$
            \item $\displaystyle \sum_{j=4}^{8}ijk$
            \item $\displaystyle \sum_{n=4}^{8}ijk$
            \item $\displaystyle \sum_{i=3}^{7}13$
            \item $\displaystyle \sum_{i=1}^{3}i^2 + \sum_{j=4}^{6}j^2$
            \item $\displaystyle \sum_{i=1}^{3}\left(\sum_{j=5}^{7}(i+j)\right)$
        \end{partquestions}
    \end{multicols}
\end{exercise}

We note some properties of summation.
\begin{itemize}
    \item $\displaystyle \sum_{i=m}^na_i = \sum_{j=m}^na_j$\hfill(Index renaming)
    \item $\displaystyle \sum_{i=m}^n(Ca_i) = C\sum_{i=m}^na_i$\hfill(Factor constants out of a sum)
    \item $\displaystyle \left(\sum_{i=m}^na_i\right) \pm \left(\sum_{i=m}^nb_i\right) = \sum_{i=m}^n(a_i \pm b_i)$\hfill(Break sum across sum or difference)
    \item $\displaystyle \sum_{i=m}^ka_i = \left(\sum_{i=m}^na_i\right) + \left(\sum_{j={n+1}}^ka_j\right)$\hfill(Splitting the sum)
    \item $\displaystyle \sum_{i=m}^na_i = \sum_{i=m+k}^{n+k}a_{i-k}$\hfill(Index shift)
\end{itemize}

\begin{exercise}
    Given $\displaystyle \sum_{i=1}^5a_i^2 = 100$, $\displaystyle \sum_{i=2}^5a_{i-1} = 10$, and $\displaystyle \sum_{i=1}^5(a_i+2)^2 = 200$, what is $a_5$?
\end{exercise}

\section{The Binomial Theorem}
Sometimes we are forced to expand expressions such as $(x-1)^5$, $(3x^2 + 5)^4$, and $(7x - 3)^9$. These expressions can be readily expanded using the \textbf{binomial theorem}\index{Bionomial Theorem}.
\begin{theorem}[Binomial Theorem]\label{thrm-binomial}
    Let $n$ be a non-negative integer. Then
    \[
        (x+y)^n = \sum_{k=0}^n {n \choose k}x^ky^{n-k} = \sum_{k=0}^n {n \choose k}x^{n-k}y^k
    \]
    where
    \[
        {n \choose k} = \frac{n!}{k!(n-k)!}.
    \]
\end{theorem}
We note that ${n \choose k}$ is read as ``$n$ choose $k$'' and is known as the \textbf{binomial coefficient}\index{binomial coefficient}.

\begin{example}
    $(x+1)^4 = x^4 + 4x^3 + 6x^2 + 4x + 1$
\end{example}
\begin{example}
    $(x-1)^5 = x^5-5x^4+10x^3-10x^2+5x-1$
\end{example}
\begin{example}
    $(3x^2 + 5)^4 = 81x^8+540x^6+1350x^4+1500x^2+625$
\end{example}
\begin{exercise}
    Find the coefficient of $x^6$ in $(7x-3)^9$.
\end{exercise}

We note two facts about the binomial coefficient here.
\begin{proposition}
    ${n\choose k} = {n \choose{k-n}}$ for all integers $0 \leq k \leq n$.
\end{proposition}
\begin{proposition}\label{prop-binomial-coefficient-multiple-of-n}
    $n\choose k$ is a multiple of $n$ for all integers $1 \leq k \leq n - 1$.
\end{proposition}
