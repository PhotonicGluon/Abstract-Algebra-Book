\section{Mathematical Logic and Proof Writing}
\begin{questions}
    \item We work from the inner-most bracket outwards. We note $P$ is true, $Q$ is false, and $R$ is true.
    \begin{itemize}
        \item $P \lor Q$ is ``1 is a positive number \textbf{or} $-1 > 0$'', which is true since $P$ is true.
        \item $(P \lor Q) \land R$ is ``(1 is a positive number or $-1 > 0$) \textbf{and} 1 is an odd number'', which is true since $P \lor Q$ is true and 1 is, indeed, an odd number.
        \item $\lnot((P \lor Q) \land R)$ is false, since $(P \lor Q) \land R$ is true.
    \end{itemize}
    Hence the statement ``$\lnot((P \lor Q) \land R)$ is false'' is a true statement.
    
    \item The truth table for $P \land (\lnot Q)$ is given below:
    \begin{table}[h]
        \centering
        \begin{tabular}{|l|l||l|}
            \hline
            $P$ & $Q$ & $P\land (\lnot Q)$ \\ \hline
            F   & F   & F                  \\ \hline
            F   & T   & F                  \\ \hline
            T   & F   & T                  \\ \hline
            T   & T   & F                  \\ \hline
        \end{tabular}
    \end{table}
    
    \item \begin{partquestions}{\roman*}
        \item $R$: $n$ is a multiple of 5 if and only if the last digit of $n$ is 0 or 5.
        \item If $n$ is a multiple of 5, then its last digit necessarily has to be 5 or 0, hence $P \implies Q$. If the last digit is 5 or 0, then the number $n$ is a multiple of 5, hence $Q \implies P$. Therefore $P \iff Q$.
    \end{partquestions}
    
    \item For brevity, let $R = (P \implies (\lnot Q))$ and $S = ((\lnot Q) \implies P)$. So we want to show that $((\lnot P) \iff Q) \equiv R \land S$.
    \begin{table}[h]
        \centering
        \begin{tabular}{|l|l||l|l|l|l||l|l|}
            \hline
            $P$ & $Q$ & $\lnot P$ & $\lnot Q$ & $R$ & $S$ & $R \land S$ & $(\lnot P) \iff Q$ \\ \hline
            F   & F   & T         & T         & T   & F   & F           & F                  \\ \hline
            F   & T   & T         & F         & T   & T   & T           & T                  \\ \hline
            T   & F   & F         & T         & T   & T   & T           & T                  \\ \hline
            T   & T   & F         & F         & F   & T   & F           & F                  \\ \hline
        \end{tabular}
    \end{table}

    From inspection, the truth tables of $(\lnot P) \iff Q$ and $R \land S$ are the same, proving our required result.
    
    \item For brevity, denote $X = P \lor \lnot Q$, $Y = \lnot R$, and $Z = (P \lor R) \land (P \lor \lnot R)$. Thus the original statement is something like $(X \land Y) \lor (X \land Z)$, which by distributive rules is equal to $X \land (Y \lor Z)$. Note $Z \equiv P \lor (R \land \lnot R)$ by distributive rules, which is equal to $P$ since $R \land \lnot R$ is always false. Hence $X \land (Y \lor Z) \equiv (P \lor \lnot Q) \land (\lnot R \lor P)$. Commutativity of $\lor$ means that $\lnot R \lor P \equiv P \lor \lnot R$, so $(P \lor \lnot Q) \land (\lnot R \lor P) \equiv (P \lor \lnot Q) \land (P \lor \lnot R)$. Now by distributive rules $(P \lor \lnot Q) \land (P \lor \lnot R) \equiv P \lor (\lnot Q \land \lnot R)$. Finally, by De Morgan's Law, $\lnot Q \land \lnot R \equiv \lnot(Q \lor R)$, so the original statement is equal to $P \lor \lnot(Q \lor R)$.
    
    \item $\left[(n \in \mathbb{Z}) \land (n > 2)\right] \implies \left[\exists a, b \in \mathbb{Z} \text{ s.t. } a^3 + b^4 = n^5\right]$
    
    \item \begin{partquestions}{\roman*}
        \item Let
        \begin{align*}
            P:&\ (x \in \mathbb{R}) \land (x \neq 0)\\
            Q:&\ \exists y \in \mathbb{R}, xy = 1
        \end{align*}
        then the given statement is $P \implies Q$ as required.
        \item $(x \in \mathbb{R}) \land (x \neq 0) \land (\forall y \in \mathbb{R}, xy \neq 1)$
    \end{partquestions}
    
    \item Suppose that $m$ and $n$ have the same parity. We split into two cases.
    \begin{itemize}
        \item If both $m$ and $n$ are even, then we may write $m = 2a$ and $n = 2b$ where $a$ and $b$ are integers. Hence
        \[
            m + n = 2a + 2b = 2(a+b)        
        \]
        which clearly means that $m + n$ is even.
        \item If both $m$ and $n$ are odd, then we may write $m = 2a + 1$ and $n = 2b + 1$ where $a$ and $b$ are integers. Hence
        \[
            m + n = (2a + 1) + (2b + 1) = 2(a + b + 1)        
        \]
        which clearly means that $m+n$ is even.
    \end{itemize}
    Hence in both cases $m + n$ is even.
    
    \item We consider a proof by contrapositive; the statement that we want to prove is ''if \textbf{not} ($a$ is even or $b$ is odd) then $a(b^2+5)$ is \textbf{not} even''. That is, ''if $a$ is \textbf{not} even \textbf{and} $b$ is \textbf{not} odd then $a(b^2+5)$ is \textbf{not} even'', meaning ''if $a$ is odd and $b$ is even then $a(b^2+5)$ is odd''.
    
    Suppose that $a$ is odd and $b$ is even. Then we may write $a = 2m + 1$ and $b = 2n$ where $m$ and $n$ are integers. Hence
    \begin{align*}
        a(b^2+5) &= (2m+1)\left((2n)^2 + 5\right)\\
        &= (2m+1)(4n^2 + 5)\\
        &= 8mn^2 + 10m + 4n^2 + 5\\
        &= 8mn^2 + 10m + 4n^2 + 4 + 1\\
        &= 2(4mn^2 + 5m + 2n^2 + 2) + 1
    \end{align*}
    which clearly means that $a(b^2+5)$ is odd.
    
    \item By way of contradiction assume there exist integers $a$ and $b$ such that $2a + 4b = 1$. Then dividing both sides by 2 leads to $a + 2b = \frac12$. Note the left hand side is clearly an integer, while the right hand side is not an integer, a contradiction.
    
    \item By way of contradiction assume that $a$ and $b$ are positive real numbers, and $\frac{a+b}{2} < \sqrt{ab}$. This means $a+b<2\sqrt ab$. Squaring both sides yields $(a+b)^2 < 4ab$. Note
    \[
        (a+b)^2 = a^2 + 2ab + b^2 < 4ab    
    \]
    which implies $a^2 + b^2 < 2ab$, leading to $a^2 - 2ab + b^2 < 0$. However $a^2 - 2ab + b^2 = (a-b)^2 \geq 0$ for all positive real numbers $a$ and $b$. Hence we have $(a-b)^2 < 0$ and $(a-b)^2 \geq 0$, a contradiction.
    
    \item We note that a positive odd number is of the form $2n - 1$ where $n$ is a positive integer. Setting $a = 2n - 1$; we induct on $n$.
    
    When $n = 1$, we have $a^2 - 1 = (2(1) - 1)^2 - 1 = 1 - 1 = 0$ which is clearly a multiple of 8.
    
    Assume that the statement holds for some positive integer $k$, i.e. $(2k-1)^2 - 1 = 8m$ for some integer $m$. We show that the statement holds for $k + 1$.
    
    We note that $(2k-1)^2 - 1 = 4k^2 - 4k$. Observe
    \begin{align*}
        (2(k+1)-1)^2 - 1 &= (2k+1)^2 - 1\\
        &= 4k^2 + 4k + 1 - 1\\
        &= (4k^2 - 4k) + 8k\\
        &= 8m + 8k & (\text{by hypothesis})\\
        &= 8(m+k)
    \end{align*}
    which means that $(2(k+1)-1)^2 - 1$ is a multiple of 8, proving that the statement holds for $k+1$. By mathematical induction, $a^2 - 1$ is a multiple of 8 for all positive odd integers $a$.
    
    \item We induct on $n$ using strong induction.
    
    We prove the base cases of 1 and 2 first:
    \begin{itemize}
        \item When $n = 1$, we have $2^1 - 1 - 1 = 0 = f(1)$, so this case is true.
        \item When $n = 2$, we have $2^2 - 2 - 1 = 1 = f(2)$, so this case is also true.
    \end{itemize}
    
    Assume that for some positive integer $k \geq 2$, for all positive integers $m \leq k$ we have $f(m) = 2^m - m - 1$. We want to show that $f(k+1) = 2^{k+1} - (k+1) - 1$.
    \begin{align*}
        f(k+1) &= 3f(k) - 2f(k-1) + 1\\
        &= 3(2^k - k - 1) - & (\text{hypothesis})\\
        &\quad\quad2(2^{k-1} - (k-1) - 1) + 1 & (\text{hypothesis}) \\
        &= 3\times 2^k - 3k - 3 -2^k + 2k + 1\\
        &= 2\times2^k - k - 2\\
        &= 2^{k+1} - (k+1) - 1
    \end{align*}
    so the statement holds for $k+1$.
    
    By mathematical induction, we conclude $f(n) = 2^n - n - 1$ for all positive integers $n$.
    
    \item We prove the forward direction using direct proof. Assume $n$ is one more than a multiple of 5. Then we may write $n = 5a + 1$ where $a$ is an integer. Note $5a + 1 = 5a + (5 - 4) = (5a + 5) - 4 = 5(a+1) - 4$. Setting $k = a+1$ yields required result.
    
    We now prove the reverse direction, using direct proof as well. Assume $n = 5k - 4$. Observe $5k - 4 = 5k - 5 + 1 = 5(k-1) + 1$, meaning $n$ is one more than a multiple of 5.
    
    \item The number 7 satisfies this as $7 = 2^3 - 1$ and $7 = 3^2 - 2$.
    
    \item \begin{partquestions}{\roman*}
        \item We use a proof by contradiction to prove this claim. Seeking a contradiction, assume $y$ is rational. Write $y = \frac pq$ where $p$ and $q$ are integers. Note $2^y = 2^{\frac pq}$ and $2^y = 2^{2\log_2{3}} = 9$. Hence $2^{\frac pq} = 9$ meaning $2^p = 9^q$. However $2^p$ is always even and $9^q$ is always odd, a contradiction.
        \item Note
        \[
            x^y = (\sqrt2)^{2\log_2{3}} = \left(\sqrt{2}^2\right)^{\log_2{3}} = 2^{\log_2{3}} = 3        
        \]
        which is rational.
    \end{partquestions}
\end{questions}
