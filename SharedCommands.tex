\documentclass[a5paper,pagesize,10pt,bibtotoc,normalheadings,DIV=9,twoside=false]{scrbook}
\usepackage[utf8]{inputenc}
\usepackage{tocloft}
\usepackage{mathtools}
\usepackage{amsfonts}
\usepackage{enumitem}
\usepackage{hyperref}
\usepackage{amsmath}
\usepackage{amsthm}
\usepackage{amssymb}
\usepackage[hmargin=2cm, vmargin=2.5cm]{geometry}
\usepackage{graphicx}
\usepackage{wrapfig}
\usepackage{parskip}
\usepackage{framed}

\usepackage[
    backend=bibtex,
    style=alphabetic,
    sorting=ynt
]{biblatex}

%=========== Path to images ==============
\graphicspath{{./images/}}

%============== Resources ================
\addbibresource{../AbstractAlgebra.bib}

%============ Redefinitions ==============
\let\oldemptyset\emptyset
\let\emptyset\varnothing

\let\totient\varphi

\renewcommand{\vert}{ \ | \ }

%======== Theorem-Like Things ============
\newtheoremstyle{exercise-style}
  {-5pt}  % Measure of space to leave above the theorem
  {\topsep}  % Measure of space to leave below the theorem
  {}  % Name of the font to use in the body of the theorem
  {0pt}  % Measure of space to indent
  {\bfseries}  % Name of the head font
  {.}  % Punctuation between head and body
  { }  % Space after theorem head; " " = normal inter-word space
  {\thmname{#1}\thmnumber{ #2}\textnormal{\thmnote{ (#3)}}}

\newtheorem{theorem}{Theorem}[section]
\renewcommand{\thetheorem}{\Roman{part}.\arabic{chapter}.\arabic{section}.\arabic{theorem}}

\newtheorem{conjecture}[theorem]{Conjecture}
\newtheorem{proposition}[theorem]{Proposition}
\newtheorem{definition}[theorem]{Definition}
\newtheorem{lemma}[theorem]{Lemma}
\newtheorem{lemma-thrm}{Lemma}[theorem]
\newtheorem{corollary}[theorem]{Corollary}
\newtheorem{corollary-thrm}{Corollary}[theorem]
\theoremstyle{definition}\newtheorem*{remark}{Remark}
\theoremstyle{definition}\newtheorem{example}[theorem]{Example}

\theoremstyle{exercise-style}\newtheorem{exercisehidden}{Exercise}[chapter]
\renewcommand{\theexercisehidden}{\Roman{part}.\arabic{chapter}.\arabic{exercisehidden}}

\theoremstyle{definition}\newtheorem{problem}{Problem}[chapter]
\renewcommand{\theproblem}{\Roman{part}.\arabic{chapter}.\arabic{problem}}

%============ Environments ===============
\newenvironment{exercise}
{\begin{framed}\noindent\begin{exercisehidden}}
{\end{exercisehidden}\end{framed}}

%=========== Custom Commands =============
\newcommand{\code}[1]{\texttt{#1}}  % Code block
\makeatletter\newcommand*{\rom}[1]{\expandafter\@slowromancap\romannumeral #1@}\makeatother

\newcommand{\lcm}{\mathrm{lcm}}  % Lowest common multiple function
\newcommand{\sgn}{\mathrm{sgn}}  % Signum function

\newcommand{\im}{\mathrm{im}\;}  % Image of a function
\newcommand{\id}{\mathrm{id}}    % Identity function

\newcommand{\An}[1]{\mathrm{A}_{#1}}                    % Alternating group of degree n
\newcommand{\Aut}[1]{\mathrm{Aut}(#1)}                  % Group of automorphisms of G
\newcommand{\C}[2]{\mathrm{C}_{#1}(#2)}                 % Centralizer of an element in G
\newcommand{\Cl}[1]{\mathrm{Cl}(#1)}                    % Conjugacy class of the element x
\newcommand{\Cn}[1]{\mathrm{C}_{#1}}                    % Cyclic group of order n
\newcommand{\GL}[2]{\mathrm{GL}_{#1}\left(#2\right)}    % General Linear Group of degree n
\newcommand{\Inn}[1]{\mathrm{Inn}(#1)}                  % Group of inner automorphisms of G
\newcommand{\N}[2]{\mathrm{N}_{#1}(#2)}                 % Normalizer of S in G
\newcommand{\Out}[1]{\mathrm{Out}(#1)}                  % Group of outer automorphisms of G
\newcommand{\SL}[2]{\mathrm{SL}_{#1}\left(#2\right)}    % Special Linear Group of degree n
\newcommand{\Sn}[1]{\mathrm{S}_{#1}}                    % Symmetric group of degree n
\newcommand{\Syl}[2]{\mathrm{Syl}_{#1}(#2)}             % Set of Sylow p-groups of G
\newcommand{\Sym}[1]{\mathrm{Sym}(#1)}                  % Symmetric group of a set
\newcommand{\Un}[1]{\mathcal{U}_{#1}}                   % Group of units modulo n
\newcommand{\Z}[1]{\mathrm{Z}(#1)}                      % Center of a group G

\newcommand{\Stab}[2]{\mathrm{Stab}_{#1}(#2)}  % Stabilzer of x by G
\newcommand{\Fix}[2]{\mathrm{Fix}_{#1}(#2)}    % Set of all elements in X which is fixed by g
\newcommand{\Orb}[2]{\mathrm{Orb}_{#1}(#2)}    % Orbit of x under G