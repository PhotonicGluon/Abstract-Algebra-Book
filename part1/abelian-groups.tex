\chapter{Abelian Groups}
Abelian groups are nice to work with, in the sense that the group operation is commutative. Thus, simplification of the group product is possible via commutativity. In this chapter, we focus on a theorem that describes the structure of all finite abelian groups.

\section{The Fundamental Theorem of Finite Abelian Groups}
We start off the chapter by introducing the theorem at the core of finite abelian groups, which was introduced by Leopold Kronecker in 1858.

\begin{theorem}[Fundamental Theorem of Finite Abelian Groups]\index{Fundamental Theorem of Finite Abelian Groups}\label{thrm-fundamental-theorem-finite-abelian-groups}
    Suppose $G$ is a finite abelian group. Then
    \[
        G \cong \Cn{p_1^{n_1}} \times \Cn{p_2^{n_2}} \times \cdots \times \Cn{p_k^{n_k}}
    \]
    where the $p_i$'s are \textit{not} necessarily distinct primes. Moreover, the isomorphism is unique up to the order of which the factors are written.
\end{theorem}

Obviously, this theorem is very technical. Essentially, what the theorem asserts is that any finite abelian group can be reduced to an (external) direct product of cyclic $p$-groups. We leave the proof of the theorem to the last section of this chapter.

This is a very powerful theorem; questions about finite abelian groups have been reduced to questions regarding cyclic groups, which we have explored in depth in \myref{section-more-about-cyclic-groups}. Thus, the combination of \myref{thrm-fundamental-theorem-finite-abelian-groups} with results of cyclic groups usually answers any question about finite abelian groups.

\section{Isomorphism Classes of Abelian Groups}
The Fundamental Theorem of Finite Abelian Groups is very powerful. One application of it allows us to construct all abelian groups of any given order.

\begin{example}
    Suppose we wish to classify all abelian groups of order $60 = 2^2 \times 3 \times 5$. Then the Fundamental Theorem of Finite Abelian Groups (\myref{thrm-fundamental-theorem-finite-abelian-groups}) tells us that there are only two possibilities,
    \begin{itemize}
        \item $\Cn{2} \times \Cn{2} \times \Cn{3} \times \Cn{5}$; and
        \item $\Cn{4} \times \Cn{3} \times \Cn{5}$.
    \end{itemize}
\end{example}

\begin{example}
    Suppose we wish to classify all abelian groups of order $540 = 2^2 \times 3^3 \times 5$. Then the Fundamental Theorem of Finite Abelian Groups (\myref{thrm-fundamental-theorem-finite-abelian-groups}) tells us that there are only six possibilities,
    \begin{itemize}
        \item $\Cn{2} \times \Cn{2} \times \Cn{3} \times \Cn{3} \times \Cn{3} \times \Cn{5}$;
        \item $\Cn{2} \times \Cn{2} \times \Cn{3} \times \Cn{9} \times \Cn{5}$;
        \item $\Cn{2} \times \Cn{2} \times \Cn{27} \times \Cn{5}$;
        \item $\Cn{4} \times \Cn{3} \times \Cn{3} \times \Cn{3} \times \Cn{5}$;
        \item $\Cn{4} \times \Cn{3} \times \Cn{9} \times \Cn{5}$; and
        \item $\Cn{4} \times \Cn{27} \times \Cn{5}$.
    \end{itemize}
\end{example}

\begin{exercise}
    Classify all abelian groups of order 100.
\end{exercise}

% With \myref{thrm-fundamental-theorem-finite-abelian-groups}, we obtain a much stronger converse to Lagrange's Theorem (\myref{thrm-lagrange}) than we could obtain by using Sylow's Theorems, but only for abelian groups.

We also note a much stronger converse to Lagrange's Theorem (\myref{thrm-lagrange}) than we could obtain by using Sylow's Theorems, but only for abelian groups. This does not use the Fundamental Theorem of Finite Abelian Groups, but this is an important result nonetheless.

\begin{theorem}
    If a positive integer $m$ divides the order of a finite abelian group $G$, then $G$ has a subgroup of order $m$.
\end{theorem}
\begin{proof}
    Suppose $|G| = n$ and that $m$ divides $n$. We use strong induction on $n$.

    When $n = 1$, then clearly $m = 1$. The trivial subgroup is a subgroup of any group, and it has order 1. Thus the theorem holds for the case when $n = 1$.

    Now assume that, for all abelian groups $G'$, that for any $m$ that divides the order of $G'$ there exists a subgroup of $G'$ of order $m$, for any $m \leq k$, for some positive integer $k$. We show that for an abelian group of order $k+1$ this statement holds as well.
    
    Let $G$ be an abelian group of order $k+1$, and let $p$ be a prime that divides $m$. Then from a corollary of Cauchy's Theorem (\myref{corollary-cauchy-theorem-with-subgroup}) it follows that $G$ has a subgroup of order $p$, say $N$. As all subgroups of abelian groups are normal, thus $G/N$ is a quotient group. In fact, $G/N$ is an abelian group of order $\frac np$. Since $m$ divides $n$ thus $\frac mp$ divides $\frac np$. Therefore by the Induction Hypothesis we know that $G/N$ has a subgroup of the form $H/N$ where $H \leq G$ (see \myref{problem-subgroup-of-quotient-group-is-quotient-group}) with order $\frac mp$. Then one sees clearly that
    \begin{align*}
        |H| &= \frac{|H|}{|N|} \times |N|\\
        &= |H/N| \times |N|\\
        &= \frac mp \times p\\
        &= m
    \end{align*}
    and so $H$ is a subgroup of $G$ with order $m$, proving the case for $k+1$.

    Therefore by mathematical induction this theorem is proven.
\end{proof}
