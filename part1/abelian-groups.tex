\chapter{Abelian Groups}
Abelian groups are nice to work with, in the sense that the group operation is commutative. Thus, simplification of the group product is possible via commutativity. In this chapter, we focus on a theorem that describes the structure of all finite abelian groups.

\section{The Fundamental Theorem of Finite Abelian Groups}
We start off the chapter by introducing the theorem at the core of finite abelian groups, which was introduced by Leopold Kronecker in 1858.

\begin{theorem}[Fundamental Theorem of Finite Abelian Groups]\index{Fundamental Theorem of Finite Abelian Groups}\label{thrm-fundamental-theorem-finite-abelian-groups}
    Suppose $G$ is a finite abelian group. Then
    \[
        G \cong \Cn{p_1^{n_1}} \times \Cn{p_2^{n_2}} \times \cdots \times \Cn{p_k^{n_k}}
    \]
    where the $p_i$'s are \textit{not} necessarily distinct primes. Moreover, the isomorphism is unique up to the order of which the factors are written.
\end{theorem}

Obviously, this theorem is very technical. Essentially, what the theorem asserts is that any finite abelian group can be reduced to an (external) direct product of cyclic $p$-groups. We leave the proof of the theorem to the last section of this chapter.

This is a very powerful theorem; questions about finite abelian groups have been reduced to questions regarding cyclic groups, which we have explored in depth in \myref{section-more-about-cyclic-groups}. Thus, the combination of \myref{thrm-fundamental-theorem-finite-abelian-groups} with results of cyclic groups usually answers any question about finite abelian groups.

\section{Isomorphism Classes of Abelian Groups}
