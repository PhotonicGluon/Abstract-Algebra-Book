\section{Group Actions}
\begin{questions}
    \item We prove the two group action axioms.
    \begin{itemize}
        \item \textbf{Identity}: $\alpha(e, x) = exe^{-1} = x$.
        \item \textbf{Compatibility}: Note
        \begin{align*}
            \alpha(g, \alpha(h, x)) &= \alpha(g, hxh^{-1})\\
            &= gh x h^{-1}g^{-1}\\
            &= (gh)x(gh)^{-1}\\
            &= \alpha(gh, x).
        \end{align*}
    \end{itemize}
    Therefore $\alpha$ is a group action of $G$ on $G$.

    \item Recall there are 6 elements in $\Sn{3}$: $\id$, $\begin{pmatrix}1 & 2 & 3\end{pmatrix}$, $\begin{pmatrix}1 & 3 & 2\end{pmatrix}$, $\begin{pmatrix}1 & 2\end{pmatrix}$, $\begin{pmatrix}1 & 3\end{pmatrix}$, and $\begin{pmatrix}2 & 3\end{pmatrix}$. Clearly the identity has all elements of $X$ as fixed points. It is also clear that $\begin{pmatrix}1 & 2 & 3\end{pmatrix}$ and $\begin{pmatrix}1 & 3 & 2\end{pmatrix}$ have no fixed points since they permute all elements. For the rest, the fixed points are the missing element from the cycle notation, i.e. $\begin{pmatrix}1 & 2\end{pmatrix}$ has fixed point 3, $\begin{pmatrix}1 & 3\end{pmatrix}$ has fixed point 2, and $\begin{pmatrix}2 & 3\end{pmatrix}$ has fixed point 1.

    \item For 1, it is $\{\id, \begin{pmatrix}2 & 3\end{pmatrix}\}$. For 2, it is $\{\id, \begin{pmatrix}1 & 3\end{pmatrix}\}$. For 3, it is $\{\id, \begin{pmatrix}1 & 2\end{pmatrix}\}$.

    \item We work from the statement forwards. Note that each of these statements are ``if and only if'' statements.
    \begin{align*}
	    g \cdot x = h \cdot x &\iff g^{-1} \cdot (g \cdot x) = g^{-1} \cdot (h \cdot x)\\
	    &\iff (g^{-1}g) \cdot x = (g^{-1}h) \cdot x\\
	    &\iff e \cdot x = (g^{-1}h) \cdot x\\
	    &\iff x = (g^{-1}h) \cdot x\\
	    &\iff (g^{-1}h) \cdot x = x\\
	    &\iff g^{-1}h \in \Stab{G}{x}
	\end{align*}

	\item \begin{partquestions}{\alph*}
		\item An orbit takes the form $\Orb{G}{x}$. Clearly $e \cdot x = x$ so $x \in \Orb{G}{x}$ and thus $\Orb{G}{x}$ is non-empty.
	    \item Let $x \in X$. Since $e \cdot x = x$, so $x \in \Orb{G}{x}$.
	    \item Suppose $x \in \Orb{G}{x_1} \cap \Orb{G}{x_2}$ (as their intersection is non-empty). Then there exists $g_1, g_2 \in G$ such that $g_1\cdot x_1 = x = g_2\cdot x_2$. Thus,
	    \begin{align*}
	        x_1 &= e \cdot x_1\\
	        &= (g_1^{-1}g_1)\cdot x_1\\
	        &= g_1^{-1} \cdot (g_1 \cdot x_1)\\
	        &= g_1^{-1} \cdot (g_2 \cdot x_2)\\
	        &= (g_1^{-1}g_2) \cdot x_2.
	    \end{align*}
	    Now suppose $y \in \Orb{G}{x_1}$. Then $y = g\cdot x_1$ for some $g \in G$. Hence,
	    \begin{align*}
	        y &= g\cdot x_1 \\
	        &= g \cdot \left((g_1^{-1}g_2) \cdot x_2\right)\\
	        &= (\underbrace{gg_1^{-1}g_2}_{\text{In } G})\cdot x_2\\
	        &\in \Orb{G}{x_2}
	    \end{align*}
	    which means any element in $\Orb{G}{x_1}$ is also in $\Orb{G}{x_2}$. Hence, $\Orb{G}{x_1}$ is a subset of $\Orb{G}{x_2}$. A similar argument can be used to show that $\Orb{G}{x_2}$ is a subset of $\Orb{G}{x_1}$. Hence $\Orb{G}{x_1} = \Orb{G}{x_2}$.
	\end{partquestions}

	\item We prove the forward direction first: suppose the action is transitive. So there exists $x \in X$ such that $\Orb{G}{x} = X$. Now recall that distinct orbits partition $X$ (\myref{prop-distinct-orbits-partition-set}); since $X$ is an orbit, therefore the only partition using orbits of $X$ is $\{X\}$. In particular, any other $y \in X$ must also have an orbit of $X$.

	The reverse direction is trivial: suppose $\Orb{G}{x} = X$ for all $x \in X$. Then certainly there exists an element $x \in X$ such that $\Orb{G}{x} = X$, meaning that the group action is transitive.

	\item \begin{partquestions}{\alph*}
	    \item Consider $x = n$. The orbit of $n$ is all of $X$. Consider the permutation $\sigma = \begin{pmatrix}k & n\end{pmatrix}$ where $1 \leq k \leq n$. Clearly $\sigma \in \Sn{n}$. Note that $\sigma \cdot n = \sigma(n) = k$. Thus, $\Orb{G}{n} = X$, meaning that the group action ``$\cdot$'' given by $g \cdot x \mapsto g(x)$ is transitive.
	    \item Note that $|X| = n$ and $|\Sn{n}| = n!$. By Orbit-Stabilizer theorem (\myref{thrm-orbit-stabilizer}), the stabilizer of $x$ by $G$ must have order $\frac{n!}{n} = (n-1)!$.
	\end{partquestions}

	\item By the Orbit-Stabilizer theorem (\myref{thrm-orbit-stabilizer}),
	\[
        |\Orb{G}{x}| = \frac{|G|}{|\Stab{G}{x}|} = [G : \Stab{G}{x}].
	\]
	Under the group action of conjugation, $\Orb{G}{x} = \Cl{x}$ and $\Stab{G}{x} = \Centralizer{G}{x}$. Hence, $|\Cl{x}| = [G : \Centralizer{G}{x}]$ as required.

	\item \begin{partquestions}{\alph*}
	    \item One sees that $\CenterGrp{D_3} = \{e\}$ based on the group table of $D_3$.
	    \item Recall that every element in $D_3$ can be expressed in the form $r^as^b$ where $a \in \{0, 1, 2\}$ and $b \in \{0, 1\}$. One finds that $\Cl{r} = \{r, r^2\}$ and $\Cl{s} = \{s, rs, rs^2\}$.
	    \item The class equation is $6 = 1 + 2 + 3$.
	\end{partquestions}

	\item By the first part of Cauchy's Theorem (\myref{thrm-cauchy}) there exists an element (say $x$) with order $p$. Consider $H = \langle x \rangle$. Note that $|H| = p$ and $H \leq G$. Hence we found a subgroup of $G$ of order $p$.
\end{questions}
