\section{Further Properties of Homomorphisms}
\begin{questions}
    \item $\phi$ is a homomorphism since
    \begin{align*}
        \phi(a \oplus_3 b) &= 2(a\oplus_3 b)\\
        &= (2a) \oplus_6 (2b)\\
        &= \phi(a) \oplus_6 \phi(b).
    \end{align*}
    The image is $\{0, 2, 4\}$.

    \item $\phi$ is a homomorphism since
    \begin{align*}
        \phi(a+b) &= i^{a+b}\\
        &=i^ai^b\\
        &=\phi(a)\phi(b).
    \end{align*}
    The kernel is the set of values which map to the identity of $H$, i.e. $\{n \in \Z \vert \phi(n) = 1\}$. Now note $H$ is a cyclic group and $|i| = 4$. Thus $i^4 = 1$. Furthermore $i^8 = (i^4)^2 = 1, i^{12} = 1, \dots, i^{4k} = 1$. Thus $\ker\phi = \{4n \vert n \in \Z\} = 4\Z$ (using coset notation).

    \item We prove the forward direction first. Suppose that $\phi$ is injective. Clearly $\phi(e_G) = e_H$. Let $x$ be an element which is in the kernel of $\phi$, meaning $\phi(x) = e_H$. Then, $\phi(x) = \phi(e_G) = e_H$ which means $x = e_G$ by injectivity of $\phi$. Hence the kernel is trivial.

    Now we prove the reverse direction. Suppose the kernel of $\phi$ is trivial, i.e. $\ker \phi = \{e_G\}$. Suppose now there exists elements $x$ and $y$ in $G$ such that $\phi(x) = \phi(y)$. This means that $(\phi(x))^{-1} = \phi(x^{-1}) = \phi(y^{-1}) = (\phi(y))^{-1}$. Hence,
    \[
        \phi(xy^{-1})
        = \phi(x)\phi(y^{-1})
        = \phi(x)\left(\phi(y)\right)^{-1}
        = e_H.
    \]
    Now since the kernel is trivial, this must mean that $xy^{-1} = e_G$ which immediately leads $x=y$. Hence $\phi$ is injective.

    \item Note $G / \ker \phi \cong \im \phi$ by the Fundamental Homomorphism Theorem (\myref{thrm-isomorphism-1}). Furthermore, we note that $|G / \ker \phi| = \frac{|G|}{|\ker\phi|}$ by Lagrange's Theorem (\myref{thrm-lagrange}). Hence, $\frac{|G|}{|\ker\phi|} = |\im\phi|$ which leads to the result quickly.

    \item The Diamond Isomorphism Theorem (\myref{thrm-isomorphism-2}), statement 6, states that $H / (H\cap N) \cong HN / N$. Taking orders on both sides yields $\frac{|H|}{|H \cap N|} = \frac{|HN|}{|N|}$. Rearranging yields required result.

    \item \begin{partquestions}{\roman*}
        \item Note $H = x\Z = \{ax \vert a \in \Z\}$ and $N = mx\Z = \{a(mx) \vert a \in \Z\}$, which necessarily means $N \subseteq H$.
        \item Let $G = \Z$. Then both $H$ and $N$ are clearly subgroups of $G$. Now since $G$ is abelian (since addition is commutative), therefore $H$ and $N$ are normal by \myref{prop-subgroup-of-abelian-group-is-normal}.
        \item The Third Isomorphism Theorem (\myref{thrm-isomorphism-3}) tells us that
        \[
            (G/N)/(H/N) \cong G/H.
        \]
        Now we know from \myref{problem-Zn-isomorphic-to-Z-by-nZ} that $|G/H| = |\Z/(x\Z)| = x$ and $|G/N| = |\Z/(y\Z)| = y$. Hence
        \[
            \frac{x}{|H/N|} = y
        \]
        which quickly implies $|H/N| = \frac yx$.
    \end{partquestions}
\end{questions}
