\section{Abelian Groups}
\begin{questions}
    \item There is only one way, namely $\Cn{p_1}\times\Cn{p_2}\times\cdots\times\Cn{p_n}$.

    \item The smallest $n$ is 4, since anything smaller is just a cyclic group of prime order (or the trivial group). The two groups required are $\Cn{4}$ and $\Cn{2} \times \Cn{2}$ (which is actually isomorphic to $D_2$, the dihedral group of degree 2).

    \item First note that we can write 4 as a sum in 5 distinct ways:
    \begin{itemize}
        \item 4;
        \item 3 + 1;
        \item 2 + 2;
        \item 2 + 1 + 1; and
        \item 1 + 1 + 1 + 1.
    \end{itemize}
    Consequently the distinct isomorphism classes for an abelian group of order $p^4$ are
    \begin{itemize}
        \item $\Cn{p^4}$;
        \item $\Cn{p^3}\times\Cn{p}$;
        \item $\Cn{p^2}\times\Cn{p^2}$;
        \item $\Cn{p^2}\times\Cn{p}\times\Cn{p}$; and
        \item $\Cn{p}\times\Cn{p}\times\Cn{p}\times\Cn{p}$.
    \end{itemize}

    \item Immediately we may conclude that $a \neq b$, since otherwise $a^2 = b^2$. Because $|a| = 4$ and $|b| = 4$, we know that $G$ contains two distinct subgroups of order 4, say $A$ and $B$. So ``$\Cn{4} \times \Cn{4}$'' must be a part of the isomorphism class of $G$. But $\Cn{4} \times \Cn{4}$ has an order of 16. Therefore the isomorphism class of $G$ must be $\Cn{4} \times \Cn{4}$.

    \item Since $x^4 = e$, therefore there cannot be an element of order 8 or order 16 inside the group. So the maximum order of a `component' of the isomorphism class is 4. Consequently, the only 3 choices are
    \begin{itemize}
        \item $\Cn{4}\times\Cn{4}$;
        \item $\Cn{4}\times\Cn{2}\times\Cn{2}$; and
        \item $\Cn{2}\times\Cn{2}\times\Cn{2}\times\Cn{2}$.
    \end{itemize}
\end{questions}
