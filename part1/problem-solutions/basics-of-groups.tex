\section{Basics of Groups}
\begin{questions}
    \item The group table of $D_4$ is given as follows.
    \begin{table}[H]
        \centering
        \begin{tabular}{|l|l|l|l|l|l|l|l|l|}
        \hline
        $\ast$ & $e$    & $r$    & $r^2$  & $r^3$  & $s$    & $rs$   & $r^2s$ & $r^3s$ \\ \hline
        $e$    & $e$    & $r$    & $r^2$  & $r^3$  & $s$    & $rs$   & $r^2s$ & $r^3s$ \\ \hline
        $r$    & $r$    & $r^2$  & $r^3$  & $e$    & $rs$   & $r^2s$ & $r^3s$ & $s$    \\ \hline
        $r^2$  & $r^2$  & $r^3$  & $e$    & $r$    & $r^2s$ & $r^3s$ & $s$    & $rs$   \\ \hline
        $r^3$  & $r^3$  & $e$    & $r$    & $r^2$  & $r^3s$ & $s$    & $rs$   & $r^2s$ \\ \hline
        $s$    & $s$    & $r^3s$ & $r^2s$ & $rs$   & $e$    & $r^3$  & $r^2$  & $r$    \\ \hline
        $rs$   & $rs$   & $s$    & $r^3s$ & $r^2s$ & $r$    & $e$    & $r^3$  & $r^2$  \\ \hline
        $r^2s$ & $r^2s$ & $rs$   & $s$    & $r^3s$ & $r^2$  & $r$    & $e$    & $r^3$  \\ \hline
        $r^3s$ & $r^3s$ & $r^2s$ & $rs$   & $s$    & $r^3$  & $r^2$  & $r$    & $e$    \\ \hline
        \end{tabular}
    \end{table}
    \begin{partquestions}{\alph*}
        \item $D_4$ is not abelian because $rs \neq sr = r^3s$.
        \item We simplify $r^3srsr^3sr^3sr^2$.
        \begin{align*}
            r^3 sr sr^3 sr^3 sr^2 &= r^3srs(r^3s)(r^3s)r^2\\
            &= r^3 srs(e)r^2\\
            &= r^3 sr sr^2\\
            &= r^2(rs rs)r^2\\
            &= r^2(e)r^2\\
            &= r^4\\
            &= e
        \end{align*}
    \end{partquestions}

    \item We need to prove each of the group axioms in order to prove that $(\Q, +)$ is indeed a group.
    \begin{itemize}
        \item \textbf{Closure}: Let $\frac ab$ and $\frac cd$ be rational numbers where $b, d \neq 0$. Their sum is $\frac{ad+bc}{bd}$, which is also rational. Therefore $\Q$ is closed under addition.

        \item \textbf{Associativity}: Addition is associative by \myref{axiom-addition-is-associative}.

        \item \textbf{Identity}: 0 is the identity since
        \[
            0 + \frac ab = \frac ab + 0 = \frac ab
        \]
        for any rational number $\frac ab$ (with $b \neq 0$).

        \item \textbf{Inverse}: For any rational number $\frac ab$, its inverse is $-\frac ab$ since
        \[
            \frac ab + \left(-\frac ab\right) = \left(-\frac ab\right) + \frac ab = 0
        \]
        for any rational number $\frac ab$ (with $b \neq 0$).
    \end{itemize}
    Furthermore addition is assumed to be commutative by \myref{axiom-addition-is-commutative}. Therefore $(\Q, +)$ is an abelian group.

    \item If every element in $G$ is its own inverse, then for every element $g$ in $G$, $g^{-1} = g$. Consider $(gh)^{-1}$ where $g$ and $h$ are elements in $g$. On one hand, by Shoes and Socks, $(gh)^{-1} = h^{-1}g^{-1} = hg$ since each element is its own inverse. On the other hand, since $gh$ is an element in $G$, thus $(gh)^{-1} = gh$. Thus $gh = hg$ which means $G$ is abelian.

    \item Recall that $n = |x|$ is the smallest positive integer that satisfies $x^n = e$.

    We prove the forward direction first. Suppose $m$ is a multiple of $n$, say $m = qn$ for some integer $q$. Then
    \[
        x^m = x^{qn} = \left(x^n\right)^q = e^q = e
    \]
    which means $x^m = e$.

    We now prove the reverse direction. Suppose $x^m = e$. Using Euclid's division lemma (\myref{lemma-euclid-division}), we write $m = qn + r$ where $q$ and $r$ are integers with $0 \leq r < n$. Hence
    \[
        x^m = x^{qn + r} = x^{qn}x^r = \left(x^n\right)^qx^r = e^qx^r = x^r.
    \]
    Note that for all integers $k$ where $1 \leq k < n$, we have $x^k \neq e$ since $n$ is the smallest positive integer such that $x^n = e$. Hence, if $x^r = e$, we conclude $r = 0$. Therefore $m = qn$, meaning $m$ is a multiple of $n$.

    \item \begin{partquestions}{\alph*}
        \item Note that $(gh)^2 = ghgh$. Given that $(gh)^2 = g^2h^2 = gghh$. By cancellation law, $hg = gh$ which means $G$ is abelian.
        \item Suppose $G$ is abelian. Clearly $(gh)^1 = gh$. Suppose $(gh)^{k} = g^kh^k$ for some positive integer $k$. Then
        \begin{align*}
            (gh)^{k+1} &= (gh)(gh)^k\\
            &= (gh)(g^kh^k) & (\text{by assumption})\\
            &= ghg^kh^k\\
            &= g(hg^k)h^k\\
            &= g(g^kh)h^k & (\text{since } G \text{ is abelian})\\
            &= gg^khh^k\\
            &= g^{k+1}h^{k+1}
        \end{align*}
        so $(gh)^{k+1} = g^{k+1}h^{k+1}$ assuming $(gh)^k = g^kh^k$. Thus the claim is proven by mathematical induction.
    \end{partquestions}

    \item Note that $|1| = n$ since $1^2 = 1 \oplus_n 1 = 2$, $1^3 = 1 \oplus_n 1 \oplus_n 1 = 3$, $1^4 = 4$, ..., $1^{n-1} = n-1$ and $1^n = 0$ which is the identity. Since the group $(\Z_n, \oplus_n)$ has an element with the same order as the group, it is thus cyclic with order $n$ and generator 1.

    \item We show that $(A, \circ)$ is a group.
    \begin{itemize}
            \item \textbf{Closure}: Function composition is closed by definition.
            \item \textbf{Associativity}: Function composition is associative.
            \item \textbf{Identity}: By performing brute-force computation, we find that $T^6(x, y) = (x, y)$. Hence $T^6$ is the identity of $A$.
            \item \textbf{Inverse}: If $r = 6$ then $T^r$ is its own inverse. Otherwise, $T^{6-r}$ is the inverse of $T^r$.
    \end{itemize}
    Thus, $(A, \circ)$ is a group, with order 6.
\end{questions}
