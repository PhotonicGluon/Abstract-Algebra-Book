\section{Group Actions}
\begin{questions}
    \item \begin{partquestions}{\roman*}
        \item Since $G = D_5$ has order $10 = 2 \times 5$, by \myref{thrm-cauchy}, $G$ must have non-trivial proper subgroups of orders 2 and 5.
        \item For the subgroup with order 2, $\{e, s\} \leq G$. For the subgroup with order 5, $\{e, r, r^2, r^3, r^4\} \leq G$.
    \end{partquestions}

    \item Suppose that $n = mp$ where $m$ is a positive integer and $p$ is an odd prime. Then there must exist an element $x$ such that $x^p = e$ by Cauchy's Theorem (\myref{thrm-cauchy}). But all elements in $G$ satisfy $x^2 = e$. Since $p$ is odd, thus $x^p \neq e$ which is a contradiction. Hence, $n$ cannot be a multiple of an odd prime, meaning that $n = 2^k$ where $k$ is a positive integer.

    \item We define the map $\phi: G \to S, g \mapsto g \cdot x$ where $x \in S$ is a fixed element. We show that $\phi$ is a bijection.
    \begin{itemize}
        \item \textbf{Injective}: Suppose $g, h \in G$ are such that $\phi(g) = \phi(h)$, meaning that $g\cdot x = h\cdot x$. Hence $(g^{-1}h) \cdot x = x$ which quickly implies that $g^{-1}h = e$ since the group action is free. Therefore $g = h$ which proves that $\phi$ is injective.
        \item \textbf{Surjective}: Suppose $y \in S$. Then since the group action is transitive, there must exist an element $g \in G$ such that $g \cdot x = y$. Hence, $\phi(g) = g\cdot x = y$, meaning that the pre-image of $y$ is $g$. Therefore $\phi$ is surjective.
    \end{itemize}
    Thus $\phi$ is bijective, which means that $|G| = |S|$.

    \item Recall that $\Orb{G}{x} = \{y \in X \vert g\cdot x = y \text{ for some } g \in G\}$.

    Let $x \in X$. Then by the Orbit-Stabilizer theorem (\myref{thrm-orbit-stabilizer}), $|\Orb{G}{x}| = \frac{|G|}{|\Stab{G}{x}|}$. Since $\Stab{G}{x} \leq G$ thus it has order of either 1, 5, or 25 by Lagrange's Theorem (\myref{thrm-lagrange}). Hence, the number of elements in $\Orb{G}{x}$ is either 1, 5, or 25.

    Now $X$ has 24 elements. Since $\Orb{G}{x}$ can, at most, be the entire set $X$ which has 24 elements, thus $|\Orb{G}{x}| \neq 25$. Hence $\Orb{G}{x}$ has either 1 or 5 elements. Now by \myref{exercise-distinct-orbits-partition-set}, distinct orbits must partition the set $X$. Let the number of orbits of size 1 be $a$ and the number of orbits of size 5 be $b$. Hence, $1a + 5b = 24$. Since $b$ is an integer, thus $5b$ must be a multiple of 5, which means that $a \geq 1$. Hence, there exists an orbit of size 1, which means that there is a $g \in G$ with a fixed point.

    \item We note that the group in question that acts upon the bracelet is the group $D_3$. We consider Burnside's Lemma (\myref{lemma-burnside}) to answer this question. There are 6 actions to consider:
    \begin{itemize}
        \item $\boxed{e}$: The number of fixed points is the total number of colourings, $n^3$.
        \item $\boxed{r}$: Rotating a bracelet $120^\circ$ results in all points affecting one another, so the only fixed points would be colourings of the same colour. There are $n$ such arrangements.
        \item $\boxed{r^2}$: Similar argument as $r$ yields $n$ arrangements.
        \item $\boxed{s}$: This `fixes' one bead and flips the other two about a line. A fixed point thus requires the two beads that flipped about the line to be of the same colour, while the third bead is free. Hence, there are $n^2$ possible colourings.
        \item $\boxed{rs}$: We note that $rs$ is yet another reflection. Thus a similar argument as $s$ yields $n^2$ arrangements.
        \item $\boxed{r^2s}$: Similar argument as $s$ yields $n^2$ arrangements.
    \end{itemize}
    Note that $|D_3| = 6$, so by Burnside's Lemma,
    \[
        |X/G| = \frac16\left(n^3 + n + n + n^2 + n^2 + n^2\right) = \frac16 n(n+1)(n+2)
    \]
    meaning that the total number of distinct braces of 3 beads with $n$ colours is $\frac16 n(n+1)(n+2)$.

    \item Let $G$ be a group of order $p^2$. We note that $\CenterGrp{G} \leq G$, so by Lagrange's Theorem (\myref{thrm-lagrange}) the order of $\CenterGrp{G}$ must divide the order of $G$, meaning $|\CenterGrp{G}|$ divides $p^2$. Hence $|\CenterGrp{G}|$ is 1, $p$, or $p^2$.

    We note that $|\CenterGrp{G}| \neq 1$ by \myref{example-group-with-prime-power-order-has-non-trivial-center}. If instead $|\CenterGrp{G}| = p$, note
    \[
        |G/\CenterGrp{G}| = \frac{|G|}{|\CenterGrp{G}|} = \frac{p^2}{p} = p
    \]
    so $G/\CenterGrp{G}$ is a group of prime order. Hence by a corollary of Lagrange's Theorem (\myref{corollary-group-with-prime-order-is-cyclic}), $G/\CenterGrp{G}$ is cyclic. But by \myref{problem-quotient-of-group-mod-center-is-cyclic-implies-abelian}, $G = \CenterGrp{G}$, meaning $p^2 = |G| = |\CenterGrp{G}| = p$, a contradiction.

    Hence $|\CenterGrp{G}| = p^2$. Since $\CenterGrp{G} \leq G$ and $|G| = |\CenterGrp{G}| = p^2$, therefore $G = \CenterGrp{G}$, meaning $G$ is abelian by \myref{problem-center-of-G}.

    \item We first look at elements inside $\Omega$. Suppose $x \in \Omega$. Then $g \cdot x = x$ for any $g \in G$. Recall that $\Orb{G}{x} = \{y \in X \vert g \cdot x = y \text{ for some } x \in X\}$. Hence $x \in \Omega$ if and only if $\Orb{G}{x} = \{x\}$, which means $|\Orb{G}{x}| = 1$.

    Now consider $x \notin \Omega$, meaning $|\Orb{G}{x}| \neq 1$. Recall $|G| = p^n$ for some $n \geq 1$ and prime $p$. By Orbit-Stabilizer theorem (\myref{thrm-orbit-stabilizer}), one obtains
    \[
        |\Stab{G}{x}| = \frac{|G|}{|\Orb{G}{x}|} = \frac{p^n}{|\Orb{G}{x}|}.
    \]
    Since $|\Stab{G}{x}|$ is an integer, thus $\frac{p^n}{|\Orb{G}{x}|}$ must be an integer, meaning $|\Orb{G}{x}|$ divides $p^n$. Therefore if $x \notin \Omega$ then $|\Orb{G}{x}| \equiv 0 \pmod p$.

    Finally, recall that by \myref{exercise-distinct-orbits-partition-set} distinct orbits partition $X$. Hence the number of elements in $X$ is the sum of the number of elements in each of the distinct orbits of $X$. Now for each orbit $\Orb{G}{x}$ where $x \notin \Omega$, the number of elements in it is a multiple of $p$, while for $x \in \Omega$ there is only one element in its orbit. Hence, $|X| \equiv |\Omega| \pmod p$ since there are $|\Omega|$ orbits with only one element.
\end{questions}
