\section{Subgroups}
\begin{questions}
    \item We note that $G$ contains $\{e, r, r^2, r^3, s, rs, r^2s, r^3s\}$.
    \begin{partquestions}{\alph*}
        \item Yes, this is the trivial subgroup.
        \item No, it is not closed. ($rs$ can be generated by $r \ast s$ but is not in the set)
        \item No, the identity $e$ is missing.
        \item Yes, $\{r, r^3, r^4, r^6\} = \{r, r^3, e, r^2\} = \langle r \rangle$ which is a cyclic subgroup of $G$.
    \end{partquestions}

    \item \begin{partquestions}{\alph*}
        \item Clearly $K$ is a subset of $G$. Note $e \in K$ since $e^2 = e \in H$, so $K$ is non-empty.
        
        Let $x, y \in K$, so $x^2 \in H$ and $y^2 \in H$. We note $y^{-1} \in K$ since $(y^{-1})^2 = (y^2)^{-1} \in H$. Therefore $(xy^{-1})^2 = xy^{-1}xy^{-1} \in H$, so $xy^{-1} \in K$. Hence $K \leq G$ by subgroup test (\myref{thrm-subgroup-test}).

        \item We show that $H \subseteq K$. Note that for any $h \in H$ we have $h^2 \in H$ (since $H$ is a (sub)group and so it is closed). Thus $h \in K$. This shows that $H \subseteq K$. Now because $H \subseteq K$ and $H$ is a (sub)group, thus $H \leq K$ by definition of subgroup.
    \end{partquestions}

    \item \begin{partquestions}{\alph*}
        \item We have proved that $\CenterGrp{G} \leq G$ so we only prove normality. Let $g$ and $z$ be arbitrary elements from $G$ and $\CenterGrp{G}$ respectively. Then
        \begin{align*}
            gzg^{-1} &= g(zg^{-1})\\
            &= g(g^{-1}z) & (\text{since }z \in \CenterGrp{G})\\
            &= (gg^{-1})z \\
            &= z\\
            &\in \CenterGrp{G}
        \end{align*}
        which proves that $\CenterGrp{G} \unlhd G$.

        \item We first work in the forward direction by assuming $G = \CenterGrp{G}$. Then for all $z \in \CenterGrp{G} = G$ we have $gz = zg$ for any $g \in G$ by definition, which means that $G$ is abelian.

        We now work in the reverse direction by assuming that $G$ is abelian. Note $\CenterGrp{G} = \{z \in G \vert gz = zg \text{ for all } g \in G\}$. But since $G$ is abelian, $gh = hg$ for all $g$ and $h$ in $G$. Thus every element in $G$ satisfies the condition to be in the center of $G$, meaning $\CenterGrp{G} = G$.

        \item We note that $D_4 = \{e, r, r^2, r^3, s, rs, r^2s, r^3s\}$. Since $\CenterGrp{D_4}$ is a subgroup of $D_4$ it has a maximum order of 2, by Lagrange's theorem (\myref{thrm-lagrange}) and the note given. Since 2 is prime the subgroups must be cyclic. Thus the non-trivial proper subgroups of $D_4$ are $\{e, r^2\}$ and $\{e, s\}$ (since $|r^2| = |s| = 2$). Now like how we proved that $\langle s \rangle = \{e, s\}$ is not a normal subgroup in $D_3$ in \myref{example-normal-subgroups-of-d3}, $\{e, s\}$ is not a normal subgroup of $D_4$. One verifies easily that $\{e, r^2\} = \langle r^2 \rangle$ is a normal subgroup of $D_4$. Thus $\CenterGrp{D_4} = \langle r^2 \rangle$ since $\CenterGrp{D_4}$ must be a normal subgroup of $D_4$ with order not exceeding 2.
    \end{partquestions}

    \item \begin{partquestions}{\alph*}
        \item We will prove this statement.

        Clearly since $H \subseteq G$ and $K \subseteq G$ (as both are subgroups), thus $H \cap K \subseteq G$. Also $e \in H \cap K$ since $e \in H$ and $e \in K$ as both are subgroups of $G$. Therefore $H \cap K$ is non-empty.

        Let $x$ and $y$ be in $H \cap K$, meaning that $x, y \in H$ and $x, y \in K$. Thus $xy^{-1} \in H$ and $xy^{-1} \in K$ as both are subgroups of $G$. Hence $xy^{-1} \in H \cap K$.
        
        Therefore $H \cap K \leq G$ by the subgroup test (\myref{thrm-subgroup-test}).

        \item We will prove this statement. One sees that $H \cap K \subseteq H$. Since $H \cap K \leq G$, it is thus a group. Hence $H \cap K \leq H$ by definition of a subgroup.

        \item We will disprove this statement. Consider
        \begin{align*}
            &G = \Z_6 \text{ under }\oplus_6,\\
            &H = \{0, 2, 4\},\text{ and}\\
            &K = \{0, 3\}.
        \end{align*}
        Clearly $H \leq G$ and $K \leq G$. Note $H \cup K = \{0, 2, 3, 4\}$. But $H \cup K$ is not closed since $2 \oplus_6 3 = 5 \not \in H \cup K$. Hence $H \cup K \not\leq G$.

        \item We will disprove this statement. Since $H \cup K$ is not closed it is not a group, meaning it cannot be a subgroup.
    \end{partquestions}

    \item By Lagrange's Theorem (\myref{thrm-lagrange}), the order of a subgroup must divide the order of the group. Since $H \leq G$ is proper, and since $1024 = 2^{10}$, the largest order that $H$ can be is $512$ with $[G:H] = 2$. An example is $G = \Z_{1024}$ and $H = \langle 2 \rangle$, since $|H| = |2| = 512$ as $2 \times 512 = 1024 \equiv 0 \pmod{1024}$.

    \item Let $|G| = 2n$. The identity is its own inverse, leaving an odd number of non-identity elements.
    
    Suppose $x$ is an element of $G$ with $|x| > 2$; we cannot have $x^{-1} = x$ (otherwise $x^2 = e$). Thus $x^{-1}$ and $x$ are distinct. Pair every one of these $x$'s with its inverse $x^{-1}$.

    Remember that there is an odd number of non-identity elements. Hence, there must be at least one element which has not been paired off with any of the others, which is therefore its own self inverse.

    Since this element is not the identity, thus it has to have order 2 (as $g^{-1} = g$ implies $g^2 = e$).

    \item Suppose $G = \langle g \rangle$ and $H \leq G$. Then any element in $H$ is of the form $g^a$ where $a$ is an integer. Suppose $m$ is the smallest positive integer $m$ such that $g^m \in H$. Suppose now $g^n \in H$ for some $n$. By Euclid's division lemma (\myref{lemma-euclid-division}), $n = mq + r$ where $q$ and $r$ are non-negative integers such that $0 \leq r < m$. Hence,
    \[
        g^n = g^{mq}g^r = (g^m)^q g^r.
    \]
    Now, $m$ is the smallest positive integer such that $g^m \in H$. This means that if $r \neq 0$, $g^r \not\in H$ as $0 \leq r < m$. Hence, $r = 0$, which means
    \[
        g^n = (g^m)^q.
    \]
    Thus, every element in the subgroup $H$ can be formed by applying $g^m$ a certain number of times, meaning $H$ is cyclic with generator $g^m$.

    \item \begin{partquestions}{\roman*}
        \item Let $xH$ be a coset in $G$. Since cosets partition $G$, either $xH = H$ or $xH = G \setminus H$ (since there are only two distinct cosets).
        \begin{itemize}
                \item If $xH = H$, then $xH = Hx$ by Element in Coset (\myref{corollary-equivalence-of-element-in-coset}).
                \item If $xH = G \setminus H$, then $xH \neq H$, meaning $Hx \neq H$ by contrapositive of Element in Coset. But if $Hx \neq H$, then $Hx = G \setminus H$ (since there are only two distinct cosets). So $xH = G \setminus H = Hx$.
        \end{itemize}
        Therefore $H$ is a normal subgroup of $G$, i.e. $H \lhd G$.

        \item Note that $G/H$ is a group since $H \lhd G$. Also since $[G:H] = 2$ thus $|G/H| = 2$.
        
        If $x \in G$ then $x^2H = (xH)^2 = H$ since the order of $G/H$ is 2, meaning that any non-identity element inside it (like $xH$) has an order of at most 2. Since $x^2H = H$, therefore $x^2 \in H$ by Element in Coset (\myref{corollary-equivalence-of-element-in-coset}).
        
        \item Suppose $x$ has odd order. Write $|x| = 2k - 1$ where $k$ is a positive integer. Hence $x^{2k-1} = e \in H$ since $H < G$. Therefore
        \[
            x = x^{2k} = \left(x^k\right)^2 \in H        
        \]
        which means $x \in H$.
    \end{partquestions}

    \item \begin{partquestions}{\alph*}
        \item Suppose we have an element $x \in H \cap K$, meaning that $x \in H$ and $x \in K$. By a corollary of Lagrange's Theorem (\myref{corollary-order-of-group-multiple-of-order-of-element}), the order of $x$ must divide the order of its group. Hence, $|x|$ divides $|H|$ and $|x|$ divides $|K|$ simultaneously, meaning that $|x| = \gcd(|H|, |K|)$. But the GCD of the orders of both subgroups is 1. Hence, $|x| = 1$, meaning the only element in the intersection $H \cap K$ is the identity $e$.
        
        \item Consider $hkh^{-1}k^{-1}$.
        \begin{itemize}
            \item On one hand, note that $hkh^{-1}k^{-1} = h(kh^{-1}k^{-1})$. Clearly $h \in H$ and $kh^{-1}k^{-1} \in H$ by normality of $H$. Therefore $hkh^{-1}k^{-1} \in H$.
            \item On another hand, $hkh^{-1}k^{-1} = (hkh^{-1})k^{-1}$. Note $hkh^{-1} \in K$ by normality of $K$ and $k^{-1} \in K$, so $hkh^{-1}k^{-1} \in K$.
        \end{itemize}
        Therefore $hkh^{-1}k^{-1} \in H \cap K$. But by \textbf{(a)}, the only element in $H \cap K$ is the identity. Thus, $hkh^{-1}k^{-1} = e$ which the result follows quickly.
    \end{partquestions}
    
    \item \begin{partquestions}{\alph*}
        \item $m = 6$.
        \item We first prove that all groups of order less than 6 are abelian, and then find a non-abelian group of order 6.

        We note that a group of order 1 is the trivial group which is abelian. The groups of order 2, 3, and 5 are groups of prime order, meaning that they are cyclic and hence abelian. We are left with a group of order 4.

        We note that the order of an element of a group of order 4 must divide 4 (\myref{corollary-order-of-group-multiple-of-order-of-element}). Hence the possible orders of an element in such a group is 1, 2, or 4. An element of order 1 is the identity. If an element with order 4 exists, then the group is cyclic and hence abelian. So we assume that all elements are either order 1 or order 2 (in fact, the orders must be 1, 2, 2, 2). This is precisely the group
        \[
            D_2 = \langle r, s \vert r^2 = s^2 = e, rs = sr\rangle
        \]
        which is abelian. Hence all groups of order 4 are abelian.

        We now show that a group of order 6 can be non-abelian. We note that the group
        \[
            D_3 =  \langle r, s \vert r^3 = s^2 = e, rs = sr^2\rangle
        \]
        has order 6 and because $rs = sr^2 \neq sr$, thus $D_3$ is non-abelian. Hence $m = 6$.

        \item For all even $n \geq 6$, the group $D_{\frac n2}$ has $n$ elements and $rs = sr^{\frac n2 - 1} \neq sr$, so $D_{\frac n2}$ is non-abelian.
    \end{partquestions}

    \item Suppose $G / \CenterGrp{G}$ is cyclic. Then by definition, $G / \CenterGrp{G} = \langle g\CenterGrp{G}\rangle$ for some $g \in G$, and any element in $G/\CenterGrp{G}$ is of the form $g^n\CenterGrp{G}$.

    Now take $x, y \in G$. By \myref{lemma-left-coset-partition}, left cosets partition the group, so we may assume $x \in g^m\CenterGrp{G}$ and $y \in g^n\CenterGrp{G}$, meaning $x = g^mz_1$ and $y = g^nz_2$ for some $z_1, z_2 \in \CenterGrp{G}$. We note
    \begin{align*}
        xy &= (g^mz_1)(g^nz_2)\\
        &= g^m(z_1g^n)z_2\\
        &= g^m(g^nz_1)z_2 & (\text{since }z_1 \in \CenterGrp{G})\\
        &= (g^mg^n)(z_1z_2)\\
        &= g^{m+n}z_1z_2\\
        &= g^{n+m}z_2z_1\\
        &= g^ng^mz_2z_1\\
        &= g^n(g^mz_2)z_1\\
        &= g^n(z_2g^m)z_1\\
        &= (g^nz_2)(g^mz_1)\\
        &= yx
    \end{align*}
    which means that $xy = yx$ for any $x, y \in G$. Hence $G$ is abelian.
\end{questions}
