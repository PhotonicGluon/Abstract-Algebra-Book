\section{Further Properties of Homomorphisms}
\begin{questions}
    \item Construct the map $\phi: G \to \{e\}$ where $\phi(g) = e$. Clearly $\phi$ is a homomorphism as
    \[
        \phi(gh) = e = ee = \phi(g)\phi(h).
    \]
    Also, one sees that $\im\phi = \{e\}$ and $\ker\phi = G$. The Fundamental Homomorphism Theorem (\myref{thrm-isomorphism-1}) tells us that
    \[
        G / \ker\phi \cong \im\phi
    \]
    which immediately implies $G/G \cong \{e\}$.

    \item Consider $\phi: G \to R$ where $(x, y) \mapsto x\sqrt3 - y\sqrt2$. We show that $\phi$ is a homomorphism, find its image, and find its kernel.
    \begin{itemize}
        \item \textbf{Homomorphism}: Let $(x_1, y_1), (x_2, y_2) \in G$. Then
        \begin{align*}
            &\phi((x_1,y_1)(x_2,y_2))\\
            &= \phi((x_1+x_2,y_1+y_2))\\
            &= (x_1+x_2)\sqrt3 - (y_1+y_2)\sqrt2\\
            &= (x_1\sqrt3 - y_1\sqrt2) + (x_2\sqrt3 - y_2\sqrt2)\\
            &= \phi((x_1, y_1)) + \phi((x_2, y_2)).
        \end{align*}

        \item \textbf{Image}: We show that $\phi$ is surjective to show that $\im\phi = R$. For any $r \in R$, we have $\phi((\frac{r}{\sqrt3}, 0)) = \frac{r}{\sqrt3} \times \sqrt3 + 0 = r$ and $(\frac{r}{\sqrt3}, 0) \in G$, so $\phi$ is surjective.
        
        \item \textbf{Kernel}: 
        \begin{align*}
            \ker\phi &= \{(x, y) \in G \vert \phi((x, y)) = 0\}\\
            &= \{(x, y) \in G \vert x\sqrt3-y\sqrt2 = 0\}\\
            &= \left\{(x, y) \in G \vert y = \frac{\sqrt{3}}{\sqrt{2}}x\right\}\\
            &= \left\{(x, \frac{\sqrt{3}}{\sqrt{2}}x) \vert x \in \mathbb{R}\right\}\\
            &= \left\{(r\sqrt2, \frac{\sqrt{3}}{\sqrt{2}}(r\sqrt2)) \vert r \in \mathbb{R}\right\}\\
            &= \{(r\sqrt2, r\sqrt3) \vert r \in \mathbb{R}\}\\
            &= H.
        \end{align*}
    \end{itemize}
    Thus $G / H \cong R$ by the Fundamental Homomorphism Theorem (\myref{thrm-isomorphism-1}).

    \item We are given that $K \subseteq H$. Hence
    \begin{align*}
        HK &= \{hk \vert h \in H, k \in K \subseteq H\}\\
        &\subseteq \{hk \vert h \in H, k \in H\}\\
        &= \{h_1h_2 \vert h_1, h_2 \in H\}\\
        &= H. & (H \leq G \text{ so } H \text{ is closed})
    \end{align*}
    Therefore $HK \subseteq H$. Also, we know that $H \leq HK$ by the Diamond Isomorphism Theorem (\myref{thrm-isomorphism-2}), statement 3, so $H \subseteq HK$. Hence we obtain the fact that $H \subseteq HK \subseteq H$ which means $HK = H$ as required.
    
    \item \begin{partquestions}{\roman*}
        \item Consider the map $\phi: I \to G, (g, g^{-1}) \mapsto g$. We show that $\phi$ is an isomorphism:
        \begin{itemize}
            \item \textbf{Homomorphism}: Recall that $G$ is abelian, so $gh = hg$ for any $g, h \in G$. Let $(g, g^{-1}), (h, h^{-1}) \in I$. Then
            \begin{align*}
                \phi((g, g^{-1})(h, h^{-1})) &= \phi((gh, g^{-1}h^{-1}))\\
                &= \phi((gh, h^{-1}g^{-1}))\\
                &= \phi((gh, (gh)^{-1}))\\
                &= gh\\
                &= \phi((g, g^{-1}))\phi((h, h^{-1})).
            \end{align*}

            \item \textbf{Injective}: Suppose $(g, g^{-1}), (h, h^{-1}) \in I$ such that we have $\phi((g, g^{-1})) = \phi((h, h^{-1}))$. Then $g = h$ by definition of $\phi$ which clearly means $(g, g^{-1}) = (h, h^{-1})$.

            \item \textbf{Surjective}: Suppose $g \in G$. Then $(g, g^{-1}) \in I$ and $\phi((g, g^{-1})) = g$. Thus $g \in G$ has a pre-image $(g, g^{-1}) \in I$, so $\phi$ is surjective.
        \end{itemize}
        Hence $\phi$ is an isomorphism, meaning $I \cong G$.
        
        \item Consider the map $\psi: G^2 \to G, (g_1, g_2) \mapsto g_1g_2$. We show that $\psi$ is a homomorphism, then find its image and kernel.
        \begin{itemize}
            \item \textbf{Homomorphism}: Let $(g_1, g_2), (h_1, h_2) \in G^2$, so
            \begin{align*}
                \psi((g_1, g_2)(h_1, h_2)) &= \psi((g_1h_1, g_2h_2))\\
                &= g_1h_1g_2h_2\\
                &= g_1g_2h_1h_2 & (G \text{ is abelian})\\
                &= (g_1g_2)(h_1h_2)\\
                &= \psi((g_1, g_2))\psi((h_1, h_2))
            \end{align*}
            which means $\psi$ is a homomorphism.

            \item \textbf{Image}: We show that $\psi$ is surjective to show $\im \psi = G$. Consider any $g \in G$. Clearly we have $\psi((g, e)) = ge = g$, so $\psi$ is surjective.
            
            \item \textbf{Kernel}:
            \begin{align*}
                \ker\psi &= \{(g, h) \in G^2 \vert \psi((g, h)) = e\}\\
                &= \{(g, h) \in G^2 \vert gh = e\}\\
                &= \{(g, h) \in G^2 \vert h = g^{-1}\}\\
                &= \{(g, g^{-1}) \ | g \in G\}\\
                &= I.
            \end{align*}
        \end{itemize}

        Thus we have $G^2 / I \cong G$ by the Fundamental Homomorphism Theorem (\myref{thrm-isomorphism-1}). But since $I \cong G$, we further have $G^2/G \cong G$ as needed.
    \end{partquestions}

    \item We show that $\phi$ is an isomorphism.
    \begin{itemize}
        \item \textbf{Homomorphism} Let $am, bm \in m\mathbb{Z}$. Then
        \begin{align*}
            \phi(am + bm) &= \phi((a+b)m)\\
            &= (a+b) + \frac nm \mathbb{Z}\\
            &= \left(a + \frac nm \mathbb{Z}\right) + \left(b + \frac nm \mathbb{Z}\right)\\
            &= \phi(am) + \phi(bm)
        \end{align*}
        which means $\phi$ is a homomorphism.

        \item \textbf{Image}: Suppose $k + \frac nm \mathbb{Z} \in \mathbb{Z}/(\frac nm \mathbb{Z})$. Clearly $\phi(k) = k + \frac nm \mathbb{Z}$ which means $\phi$ is surjective. Hence $\im\phi = \mathbb{Z}/(\frac nm \mathbb{Z})$.
        
        \item \textbf{Kernel}:
        \begin{align*}
            \ker\phi &= \left\{am \vert \phi(am) = \frac nm \mathbb{Z}\right\}\\
            &= \left\{am \vert a + \frac nm \mathbb{Z} = \frac nm \mathbb{Z}\right\}\\
            &= \left\{am \vert a  = k\left(\frac nm\right),\; k \in \mathbb{Z}\right\}\\
            &= \left\{k\left(\frac nm\right)m \vert k \in \mathbb{Z}\right\}\\
            &= \{kn \vert k \in \mathbb{Z}\}\\
            &= n\mathbb{Z}
        \end{align*}
        so the kernel of $\phi$ is $n\mathbb{Z}$.
    \end{itemize}
    Hence $G/H \cong \mathbb{Z}/(\frac nm \mathbb{Z})$ by the Fundamental Homomorphism Theorem (\myref{thrm-isomorphism-1}). But \myref{problem-Zn-isomorphic-to-Z-by-nZ} tells us that $\mathbb{Z}/(\frac nm \mathbb{Z}) \cong \mathbb{Z}_{\frac nm}$. Hence $G/H \cong \mathbb{Z}_{\frac nm}$.
\end{questions}
