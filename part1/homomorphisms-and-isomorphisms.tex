\chapter{Homomorphisms and Isomorphisms}
Now that we have introduced the idea of a group, one wonders about how elements of one group can be mapped to elements of another group. Such a mapping can be defined between any two groups, but we look at a specific subset of such maps, called homomorphisms. We will also look at bijective homomorphisms, known as isomorphisms, and discover what they can tell us about the groups that they are mapping to and from.

\section{Homomorphisms}
\begin{definition}
    Suppose $(G, \ast)$ and $(H, \star)$ are groups. A map $\phi: G \to H$ is a \textbf{homomorphism}\index{homomorphism} if and only if
    \[
        \phi(x \ast y) = \phi(x) \star \phi(y)
    \]
    for all $x, y \in G$.
\end{definition}
\begin{remark}
    The term ``homomorphism'' can be roughly translated as ``same form'' from Ancient Greek.
\end{remark}
\begin{remark}
    We usually suppress the binary operations of $\ast$ and $\star$ when working with homomorphisms. Thus, the above condition is usually written as $\phi(xy) = \phi(x)\phi(y)$. It is important to note that $xy$ uses the group operation on $G$ (i.e., $\ast$) while $\phi(x)\phi(y)$ uses the group operation on $H$ (i.e., $\star$).
\end{remark}

\begin{example}
    Let $G$ be any group, and let $g \in G$. Let $\phi: \Z \to G$ (where $\Z$ is the additive group of integers) be such that $\phi(n) = g^n$ for all integers $n$. Then $\phi$ is a homomorphism, since
    \[
        \phi(m + n) = g^{m+n} = g^m g^n = \phi(m)\phi(n).
    \]
\end{example}

\begin{example}
    Let $G = \{r \in \R \vert r \neq 0\}$ and $H = (0, \infty)$ be groups under regular multiplication. Define $\phi: G \to H, x \mapsto |x|$ where $|x|$ represents the absolute value of $x$. Then $\phi$ is a homomorphism as
    \[
        \phi(xy) = |xy| = |x||y| = \phi(x)\phi(y).
    \]
\end{example}

\begin{exercise}
    Let $G = (\mathbb{N}, +)$ and $H = (\mathbb{N}, \times)$. Let $\phi: G \to H$. Determine if the following maps are homomorphisms.
    \begin{partquestions}{\alph*}
        \item $\phi(n) = n$
        \item $\phi(n) = 2^n$
    \end{partquestions}
\end{exercise}
\begin{exercise}
    Let $G$ and $H$ be groups with identities $e_G$ and $e_H$ respectively. Show that the \textbf{trivial homomorphism}\index{homomorphism!trivial} $\phi: G \to H, g \mapsto e_H$ is, indeed, a homomorphism.
\end{exercise}

\section{Properties of Homomorphisms}
Let us now look at some properties of homomorphisms. For the following properties, let
\begin{itemize}
    \item $G_1$ and $G_2$ be groups;
    \item $H_1 \leq G_1$ and $H_2 \leq G_2$;
    \item $e_1$ and $e_2$ be the identities of $G_1$ and $G_2$ respectively; and
    \item $\phi: G_1 \to G_2$ be a homomorphism.
\end{itemize}

\begin{proposition}\label{prop-homomorphism-maps-identities-to-each-other}
    $\phi(e_1) = e_2$.
\end{proposition}
\begin{proof}
    Let $x$ in $G_1$. Then $e_1x = x$. Thus $\phi(e_1x) = \phi(x)$ by applying $\phi$ on both sides. Hence $\phi(e_1)\phi(x) = \phi(x)$ by applying the definition of a homomorphism. Therefore, by cancellation law, $\phi(e_1) = e_2$.
\end{proof}

\begin{proposition}\label{prop-homomorphism-maps-inverses}
    For all $x$ in $G_1$, $\phi(x^{-1}) = \left(\phi(x)\right)^{-1}$.
\end{proposition}
\begin{proof}
    Note that $xx^{-1} = e_1$. Thus, $\phi(xx^{-1}) = \phi(e_1) = e_2$ by applying $\phi$ on both sides. Note also that $\phi(xx^{-1}) = \phi(x)\phi(x^{-1})$ by definition of homomorphism. Hence, $\phi(x)\phi(x^{-1}) = e_2$ which quickly implies $\phi(x^{-1}) = \left(\phi(x)\right)^{-1}$ after left-multiplying both sides by $\left(\phi(x)\right)^{-1}$.
\end{proof}

\begin{proposition}
    $\phi(x^n) = (\phi(x))^n$ for all $x \in G_1$ and $n \in \Z$.
\end{proposition}
\begin{proof}
    We first prove the proposition for non-negative integers $n$ via induction.

    When $n = 0$, then $\phi(x^0) = \phi(e_G) = e_H = (\phi(x))^0$ by \myref{prop-homomorphism-maps-identities-to-each-other}, so the statement holds for $n = 0$.

    Now suppose the statement holds for some non-negative integer $k$, i.e. $\phi(x^k) = (\phi(x))^k$. We show that the statement holds for $k + 1$, i.e. $\phi(x^{k+1}) = (\phi(x))^{k+1}$.

    We see
    \begin{align*}
        \phi(x^{k+1}) &= \phi(x^kx) \\
        &= \phi(x^k)\phi(x) & (\text{definition of homomorphism})\\
        &= (\phi(x))^k\phi(x) & (\text{by hypothesis})\\
        &= (\phi(x))^{k+1}
    \end{align*}
    so the statement holds for $k+1$.

    Thus $\phi(x^n) = (\phi(x))^n$ for all non-negative integers $n$.

    For the case when $n < 0$, write $n = -m$ where $m > 0$. Then
    \begin{align*}
        \phi(x^n) &= \phi(x^{-m})\\
        &= \phi((x^m)^{-1})\\
        &= (\phi(x^m))^{-1} & (\myref{prop-homomorphism-maps-inverses})\\
        &= ((\phi(x))^m)^{-1} & (\text{by the case for } m>0)\\
        &= (\phi(x))^{-m}\\
        &= (\phi(x))^n.
    \end{align*}

    This proves the proposition for all integers $n$.
\end{proof}

For the next few properties, define
\begin{gather*}
    \phi(H_1) = \{\phi(h) \vert h \in H_1\}, \text{ and}\\
    \phi^{-1}(H_2) = \{g \in G_1 \vert \phi(g) \in H_2\}.
\end{gather*}

\begin{proposition}
    $\phi(H_1) \leq G_2$.
\end{proposition}
\begin{proof}
    See \myref{exercise-homomorphism-image-is-subgroup} (later).
\end{proof}

\begin{proposition}\label{prop-homomorphism-inverse-is-subgroup}
    $\phi^{-1}(H_2) \leq G_1$.
\end{proposition}
\begin{proof}
    The codomain of $\phi^{-1}$ is $G_1$, so $\phi^{-1}(H_2) \subseteq G_1$. Clearly $e_1 \in \phi^{-1}(H_2)$ since $\phi(e_1) = e_2 \in H_2$. Now suppose that $x$ and $y$ are in $\phi^{-1}(H_2)$, meaning that $\phi(x)$ and $\phi(y)$ are in $H_2$. Since $H_2 \leq G_2$, therefore
    \[
        \phi(x)\left(\phi(y)\right)^{-1} \in H_2
    \]
    as $H_2$ is closed. Note that $\left(\phi(y)\right)^{-1} = \phi(y^{-1})$, so
    \[
        \phi(x)\phi(y^{-1}) = \phi(xy^{-1}) \in H_2
    \]
    which means $xy^{-1} \in \phi^{-1}(H_2)$. Hence we see that $\phi^{-1}(H_2) \leq G_1$ by the subgroup test (\myref{thrm-subgroup-test}).
\end{proof}

\begin{proposition}
    Suppose $H_2 \unlhd G_2$. Then $\phi^{-1}(H_2) \unlhd G_1$.
\end{proposition}
\begin{proof}
    Since $H_2 \leq G_2$, therefore $\phi^{-1}(H_2) \leq G_1$ by \myref{prop-homomorphism-inverse-is-subgroup}. We just need to prove normality.

    Take $n \in \phi^{-1}(H_2)$ and $g \in G_1$. We will show that $gng^{-1} \in \phi^{-1}(H_2)$ which is sufficient to prove normality.

    Consider $\phi(gng^{-1})$. We note
    \begin{align*}
        \phi(gng^{-1}) &= \phi(g)\phi(n)\phi(g^{-1}) \\
        &= \underbrace{\phi(g)}_{\text{In }G_2} \underbrace{\phi(n)}_{\text{In }H_2} \underbrace{\left(\phi(g)\right)^{-1}}_{\text{In }G_2}\\
        &= g'n'(g')^{-1}
    \end{align*}
    where $g' = \phi(g)$ and $n' = \phi(n)$. Since $H_2$ is normal, so for all $g$ in $G_2$ and $n$ in $H_2$ we know $gng^{-1}$ is in $H_2$. Therefore $\phi(gng^{-1}) = g'n'(g')^{-1}$ is in $H_2$, meaning that $gng^{-1}$ is in $\phi^{-1}(H_2)$.

    This proves that $\phi^{-1}(H_2) \unlhd G_1$.
\end{proof}

\begin{exercise}\label{exercise-homomorphism-image-is-subgroup}
    Prove that $\phi(H_1) \leq G_2$.
\end{exercise}
\begin{exercise}
    Prove or disprove: if $H_1 \unlhd G_1$, then $\phi(H_1) \unlhd G_2$.
\end{exercise}
\begin{exercise}\label{exercise-order-of-homomorphism-divides-order}
    Let $G$ and $H$ be groups, and let $\phi: G \to H$ be a homomorphism. Prove that $|\phi(a)|$ divides $|a|$ for any $a \in G$.
\end{exercise}

\section{Isomorphisms}
We now look a special (and important) type of homomorphism: \textbf{isomorphisms}, which roughly translates as ``equal form'' from Ancient Greek.

\begin{definition}
    Let $(G, \ast)$ and $(H, \star)$ be groups. Let $\phi: G \to H$ be a homomorphism. Then $\phi$ is an \textbf{isomorphism}\index{isomorphism} if $\phi$ is a bijection.
\end{definition}
\begin{definition}
    Let $G$ and $H$ be groups. We say that $G$ and $H$ are \textbf{isomorphic}\index{isomorphic} if there exists an isomorphism $\phi: G \to H$.

    If two groups $G$ and $H$ are isomorphic, we write $G \cong H$.
\end{definition}

\begin{example}
    Let $G = (\Z_2, \oplus_2)$ and $H = \{1, -1\}$ be a group under regular multiplication. Define the map $\phi: G \to H$ such that $\phi(0) = 1$ and $\phi(1) = -1$.

    \begin{itemize}
        \item \textbf{Homomorphism}: We note that
        \begin{itemize}
            \item $\phi(0\oplus_20) = \phi(0) = 1 = 1 \times 1 = \phi(0)\phi(0)$;
            \item $\phi(0 \oplus_2 1) = \phi(1) = -1 = 1 \times (-1) = \phi(0)\phi(1)$;
            \item $\phi(1 \oplus_2 0) = \phi(1) = -1 = (-1) \times 1 = \phi(1)\phi(0)$; and
            \item $\phi(1 \oplus_2 1) = \phi(0) = 1 = (-1) \times (-1) = \phi(1)\phi(1)$.
        \end{itemize}
        Thus $\phi(x\oplus_2y) = \phi(x)\times\phi(y)$ for all $x, y \in G$.

        \item \textbf{Injective}: Clearly if $\phi(x) = \phi(y)$ then $x = y$ based on the definition of $\phi$, meaning $\phi$ is injective.

        \item \textbf{Surjective}: Let $y \in H$.
        \begin{itemize}
            \item If $y = 1$ then note that $0 \in G$ is its pre-image since $\phi(0) = 1$.
            \item If $y = -1$ then note that $1 \in G$ is its pre-image since $\phi(1) = -1$.
        \end{itemize}
        So every $y \in H$ has a pre-image under $\phi$, i.e. $\phi$ is surjective.
    \end{itemize}
    Thus $\phi$ is an isomorphism, which means $G \cong H$.
\end{example}

\begin{example}
    Let $S = (0,\infty)$. We show that $(\R, +) \cong (S, \times)$ by considering the map $\phi: \R \to S, x \mapsto e^x$.
    \begin{itemize}
        \item \textbf{Homomorphism}: Let $x, y \in \R$. Then we see
        \[
            \phi(x+y) = e^{x+y} = e^xe^y = \phi(x)\phi(y)
        \]
        so $\phi$ is a homomorphism.

        \item \textbf{Injective}: Suppose $x$ and $y$ are elements in $\R$ such that $\phi(x) = \phi(y)$, meaning $e^x = e^y$. Applying the natural logarithm ($\ln$) on both sides yields $x = y$.

        \item \textbf{Surjective}: Suppose $y \in S$. Then $\ln y \in \R$, so $\phi(\ln y) = e^{\ln y} = y$, meaning every element in the codomain $S$ has a preimage.
    \end{itemize}

    Thus $\phi$ is an isomorphism, meaning that $(\R, +) \cong (S, \times)$.
\end{example}

\begin{exercise}\label{exercise-identity-map-is-isomorphism}
    The \textbf{identity function}\index{identity function} (or \textbf{identity map}\index{identity map}) $\id: S \to S$, where $S$ is a set, is the function such that $\id(x) = x$ for all $x$ in $S$.
    \begin{partquestions}{\roman*}
        \item Show that $\id$ is a bijection.
        \item If $S$ is a group, show that $\id$ is an isomorphism.
    \end{partquestions}
\end{exercise}
\begin{exercise}
    Let the groups $G = (\{1, 2, 3, 4\}, \otimes_5)$ and $H = (\{1, 3, 7, 9\}, \otimes_{10})$.
    \begin{partquestions}{\roman*}
        \item Show that $G = \langle 3 \rangle$ and $H = \langle 7 \rangle$.
        \item Prove that $G \cong H$ by considering $\phi: G \to H, 3^k \mapsto 7^k$.
    \end{partquestions}
\end{exercise}
\begin{exercise}\label{exercise-composition-of-isomorphisms-is-isomorphisms}
    Let $A$, $B$, and $C$ be sets. Let $f: A \to B$ and $g: B \to C$ be bijections. Let $h = g\circ f$ where $\circ$ represents function composition, that is $h: A \to C, x \mapsto g(f(x))$.
    \begin{partquestions}{\roman*}
        \item Show that $h$ is a bijection.
        \item If $A$, $B$, and $C$ are groups, and $f$ and $g$ are isomorphisms, show that $h$ is an isomorphism.
    \end{partquestions}
\end{exercise}

\section{Consequences of Isomorphisms}
Isomorphisms between groups means that the two groups `share the same structure', in a manner of speaking. We look at a theorem that showcases the sharing of some of these properties.
\begin{theorem}\label{thrm-isomorphism-consequences}
    Let $\phi: G \to H$ be an isomorphism between the groups $G$ and $H$. Then
    \begin{enumerate}
        \item $|G| = |H|$;
        \item $\phi^{-1}: H \to G$ is an isomorphism;
        \item if $G$ is abelian then so is $H$;
        \item if $G$ is cyclic then so is $H$; and
        \item if $G$ has a subgroup of order $n$, then so does $H$.
    \end{enumerate}
\end{theorem}

\begin{proof}
    We prove each of these statements individually.
    \begin{enumerate}
        \item Follows immediately from properties of a bijective function.

        \item Since $\phi$ is an isomorphism, it is bijective, which means that $\phi^{-1}$ exists and is also bijective. All that remains is to show that $\phi^{-1}$ is a homomorphism.

        Let $u$ and $v$ be in $H$. Then, since $\phi$ is surjective, there exist elements $x$ and $y$ in $G$ such that $\phi(x) = u$ and $\phi(y) = v$. Hence,
        \begin{align*}
            \phi^{-1}(uv) &= \phi^{-1}\left(\phi(x)\phi(y)\right)\\
            &= \phi^{-1}\left(\phi(xy)\right) & (\phi \text{ is a homomorphism})\\
            &= xy\\
            &= \phi^{-1}(u) \phi^{-1}(v).
        \end{align*}
        Thus $\phi^{-1}$ is an isomorphism.

        \item Suppose $u$ and $v$ are in $H$. Let $x$ and $y$ be elements in $G$ such that $\phi(x) = u$ and $\phi(y) = v$. Thus
        \begin{align*}
            uv = &= \phi(x)\phi(y) \\
            &= \phi(xy)\\
            &= \phi(yx) & (G \text{ is abelian})\\
            &= \phi(y)\phi(x)\\
            &= vu
        \end{align*}
        which means that $uv = vu$. Hence $H$ is abelian.

        \item Suppose $u$ is in $H$. As $\phi$ is an isomorphism, there is an $x \in G$ where $\phi(x) = u$. Since $G$ is cyclic, suppose $g$ is the generator of $G$, so $x = g^n$ for some integer $n$. This means that
        \begin{align*}
            u &= \phi(x)\\
            &= \phi(g^n)\\
            &= \left(\phi(g)\right)^n\\
            &\in \left\langle \phi(g) \right\rangle.
        \end{align*}
        Thus any element $u$ in $H$ is in $\left\langle \phi(g) \right\rangle$, meaning $H \subseteq \left\langle \phi(g) \right\rangle$.

        However, as $\phi(g) \in H$, thus $\left\langle \phi(g) \right\rangle \leq H$ which means that $\left\langle \phi(g) \right\rangle \subseteq H$. Therefore, we have $H \subseteq \left\langle \phi(g) \right\rangle$ and $\left\langle \phi(g) \right\rangle \subseteq H$ simultaneously, meaning $H = \left\langle \phi(g) \right\rangle$, i.e. $H$ is a cyclic group.

        \item Suppose $K \leq G$ with $|K| = n$. Consider the subgroup $\phi(K)$. By properties of homomorphism, $\phi(K) \leq H = \phi(G)$. Now by statement 1, $|K| = |\phi(K)| = n$, meaning that there is a subgroup of $H$ with order $n$, namely the subgroup $\phi(K)$.
    \end{enumerate}

    This proves the theorem.
\end{proof}

\begin{exercise}
    Let $\phi: G \to H$ be an isomorphism between the groups $G$ and $H$. Show that if $G$ has a normal subgroup with order $k$, then $H$ also has a normal subgroup of order $k$.
\end{exercise}

\begin{exercise}
    Prove that the ``isomorphism'' relation $\cong$ is an equivalence relation. In particular, for the groups $G$, $H$, and $K$, prove that
    \begin{partquestions}{\alph*}
        \item $G \cong G$;
        \item if $G \cong H$ then $H \cong G$; and
        \item if $G \cong H$ and $H \cong K$ then $G \cong K$.
    \end{partquestions}
\end{exercise}

\section{Links to Cyclic Groups}
With the tool of isomorphisms under our belt, we can prove two important theorems regarding cyclic groups. Before that, however, we formally introduce the idea of infinite cyclic groups.
\begin{definition}
    An infinite cyclic group\index{cyclic group!infinite} $G$ generated by $g$ is denoted by $\langle g \rangle$ and has order $|G| = \infty$. So,
    \[
        G = \{\dots, g^{-2}, g^{-1}, e, g, g^2, \dots\}.
    \]
\end{definition}

For brevity, we also have notation regarding the integers under addition.
\begin{itemize}
    \item When we write $\Z_n$, we mean the group $(\Z_n, \oplus_n)$\index{integers under addition!modulo $n$}.
    \item When we write $\Z$, we mean the group $(\Z, +)$\index{integers under addition}.
\end{itemize}

\begin{theorem}
    If $G$ is an infinite cyclic group with generator $g$, then $G \cong \Z$.
\end{theorem}
\begin{proof}[Proof (see \cite{proofwiki_infinite-cyclic-group})]
    Consider $\phi: \Z \to G$ where $\phi(n) = g^n$. We prove that $\phi$ is an isomorphism.
    \begin{itemize}
        \item \textbf{Homomorphism}: Let $m, n \in \Z$. Note
        \[
            \phi(m+n) = g^{m+n} = g^mg^n = \phi(m)\phi(n)
        \]
        so $\phi$ is a homomorphism.

        \item \textbf{Injective}: Let $m$ and $n$ be integers such that $\phi(m) = \phi(n)$. Without loss of generality, assume that $m \leq n$. Since $\phi(m) = \phi(n)$ we have $g^m = g^n = g^mg^{n-m}$ which implies that $g^{n-m} = e$ by cancellation law.

        Seeking a contradiction, suppose $m < n$. Since $g^{n-m} = e$, so $|g|$ divides $n-m$ (\myref{lemma-order-of-an-element-that-is-equivalent-to-identity}). However, $g$ is a generator of $G$, which means that $|g| = |G| = \infty$ divides $n-m$, which is absurd and thus a contradiction. Therefore, $m = n$, so $\phi(m) = \phi(n)$ implies that $m = n$. Hence $\phi$ is injective.

        \item \textbf{Surjective}: Suppose $x \in G = \langle g\rangle$, so $x = g^n$ for some integer $n$. Then $\phi(n) = g^n = x$ which means that $x$ has a preimage of $n$. Hence $\phi$ is surjective.
    \end{itemize}

    Therefore, $\phi$ is an isomorphism which means $G \cong \Z$.
\end{proof}

\begin{theorem}\label{thrm-finite-cyclic-group-isomorphic-to-Zn}
    If $G$ is a finite cyclic group of order $n$ with generator $g$, then $G \cong \Z_n$.
\end{theorem}
\begin{proof}[Proof (cf. {\cite[\S 63]{clark_1984}})]
    Consider $\phi: \Z_n \to G$ such that $\phi(m) = g^m$. We prove that $\phi$ is an isomorphism.
    \begin{itemize}
        \item \textbf{Homomorphism}: Let $l,m \in \Z$. We see that
        \[
            \phi(l+m) = g^{l+m} = g^lg^m = \phi(l)\phi(m)
        \]
        and so $\phi$ is a homomorphism.

        \item \textbf{Injective}: Let $l$ and $m$ be integers such that $\phi(l) = \phi(m)$. Without loss of generality, assume that $l \leq m$. Since $\phi(l) = \phi(m)$ we have $g^l = g^m = g^lg^{m-l}$ which implies that $g^{m-l} = e$ by cancellation law. Therefore $|g|$ divides $m-l$ by \myref{lemma-order-of-an-element-that-is-equivalent-to-identity}.

        By way of contradiction, assume $l < m$, which means $m - l \in \{1, 2,\dots, n-1\}$. Therefore $m-l \leq n - 1 < n$. But since $g$ is a generator of $G$, thus $|g| = |G| = n$. Therefore, we have $n$ dividing $m-l$, which is smaller than $n$, a contradiction. Thus $l = m$, meaning $\phi$ is injective.

        \item \textbf{Surjective}: Suppose $x \in G = \langle g\rangle$, so $x = g^m$ for some integer $m$ in $\Z_n$. Then $\phi(m) = g^m = x$ which means that $x$ has a preimage of $m$. Hence $\phi$ is surjective.
    \end{itemize}

    Therefore, $\phi$ is an isomorphism which means $G \cong \Z_n$.
\end{proof}

\begin{exercise}
    Let $G = (\{1, 3, 7, 9\}, \otimes_{10})$ be a group. Find the positive integer $n$ such that $G \cong \Z_n$.
\end{exercise}

\newpage

\section{Problems}
\begin{problem}
    Let $G$ be a group and $g \in G$. Define the map $f: G \to G, x \mapsto gxg^{-1}$. Prove that $f$ is an isomorphism.
\end{problem}

\begin{problem}
    Let $\Q_{>0}$ denote the set of positive rational numbers. Let the groups $G = (\Q, +)$ and $H = (\Q_{>0}, \times)$. Prove that $G \not\cong H$.
\end{problem}

\begin{problem}
    Define a map $\phi: \Z \to \Z$ such that $\phi(n) = 2n$.
    \begin{partquestions}{\alph*}
        \item Prove that $\phi$ is a homomorphism.
        \item Prove that $\phi$ is injective.
        \item Prove that there does \textit{not} exist a homomorphism $\psi: \Z \to \Z$ where $\psi(\phi(n)) = n$.
    \end{partquestions}
\end{problem}

\begin{problem}
    Let $G$ be a group. Define a map $f: G \to G$ such that $f(g) = g^{-1}$ for all $g$ in $G$. Prove that $G$ is abelian if and only if $f$ is a homomorphism.
\end{problem}

\begin{problem}
    Let $G$ and $H$ be groups. Suppose that we have a surjective homomorphism $\phi: G \to H$. Prove that if $G$ is abelian, then so is $H$.
\end{problem}

\begin{problem}
    Let $G$ and $H$ be groups. Suppose that we have a surjective homomorphism $\phi: G \to H$. Let $N \unlhd G$. Show that $\phi(N) \unlhd H$.\newline
    (That is, the image of $N$ under $\phi$ is a normal subgroup of $H$.)
\end{problem}

\begin{problem}\label{problem-Zn-isomorphic-to-Z-by-nZ}
    Let $G = \Z_n$, and let $H = \Z/(n\Z)$ be under addition. Prove that $G \cong H$.
\end{problem}
\begin{remark}
    Whenever we are dealing with homomorphisms involving $\Z_n$, it is usually easier to replace it with $\Z/(n\Z)$ and do the homomorphism that way.
\end{remark}

\begin{problem}\label{problem-subgroup-of-quotient-group-is-quotient-group}
    Let $G$ be a group and $N \unlhd G$. Let $B$ be a subgroup of the quotient group $G/N$. Prove that $B = A/N$, where $A$ is a subgroup of $G$ such that $N \subseteq A$.
\end{problem}
