\setpartpreamble[u][\textwidth]{
    \quoteattr{
        [The] axioms for a group are short and natural... [yet] somehow hidden behind these axioms is the monster simple group, a huge and extraordinary mathematical object, which appears to rely on numerous bizarre coincidences to exist. The axioms for groups give no obvious hint that anything like this exists.
    }
    {
        Richard Borcherds, 2009
    }
    {
        \cite{cook_borcherds_2009}
    }

    Groups are one of the most fundamental structures in abstract algebra. They underpin the ideas of symmetry and allow us to explore the relationships between symmetrical objects. They are at the core of the analysis of all the ways that a thing can be symmetric. It would be an understatement to say that groups are important in abstract algebra; without them, there can be no further and more in-depth exploration of the other structures.

    We start with an introduction and motivation for the study of groups. We then move on to the fundamentals of group theory, namely the basic properties of groups, subgroups, and homomorphisms and isomorphisms. With these fundamentals in place, one would be ready to move on to handling more complex analyses with groups.

    We then look at symmetry groups and link them to how all groups are, in fact, representations of symmetry. Direct products of groups come afterwards, as a short aside before getting into the meatier topic of the homomorphism/isomorphism theorems. These theorems are core to the analysis of the \textit{equivalence} of groups, and thus we dedicate more time to the discussion of their proofs.

    We then look at a couple of interesting groups that come up in group theory. The most critical of the bunch would be cyclic groups and matrix groups. Cyclic groups would be revealed to be a fundamental type of group; matrices would reappear again in ring theory and field theory.

    We conclude this part on group theory with more advanced topics. Group actions, usually introduced right at the start of group theory courses, are left to the end due to the numerous theorems that require the results of the preceding chapters. The Sylow theorems give possible \textit{components} of groups of a certain order, while composition series gives the actual decomposition of a given group. We end this part by looking at simple groups, which can be thought of as the atoms making up finite groups.
}
\part{Group Theory}
