\setpartpreamble[u][\textwidth]{
    \quoteattr{
        [The] axioms for a group are short and natural... [yet] somehow hidden behind these axioms is the monster simple group, a huge and extraordinary mathematical object, which appears to rely on numerous bizarre coincidences to exist. The axioms for groups give no obvious hint that anything like this exists.
    }
    {
        Richard Borcherds, 2009
    }
    {
        \cite{cook_borcherds_2009}
    }

    Groups are one of the most fundamental structures in abstract algebra. They underpin the ideas of symmetry and allow us to explore the relationships between symmetrical objects. An analysis of all the different ways an object can be symmetric is also possible with groups. It would be an understatement to say that groups are important in abstract algebra; without them, there can be no further and more in-depth exploration of the other structures.

    Part I is a simple introduction to the world of groups. As with most books on this topic, we concentrate on abstract groups, and, in particular, on finite groups. We also discuss and explore some crucial results about the structure of groups. The content covered in this part should be ample for one to understand the fundamentals of group theory, and appreciate the wonders of groups and symmetry.
}
\part{Group Theory}
