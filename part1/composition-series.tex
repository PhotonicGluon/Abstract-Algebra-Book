\chapter{Composition Series}
In the previous chapter, we introduced simple groups, which are groups with no non-trivial proper normal subgroups. We may think of such groups as \textit{atoms} making up any group, as we will soon find out that every group may be uniquely decomposed into a series of groups in what is known as the group's composition series up to isomorphism.

\section{Subgroup Series and Subnormal Series}
We introduce the idea of a \textbf{subgroup series}\index{subgroup series} of a group.
\begin{definition}
    Let $G$ be a group. A \textbf{subgroup series without repetition}\index{subgroup series!without repetition} of $G$ is a chain of subgroups $H_0, \dots, H_n$ such that
    \[
        1 = H_0 < H_1 < H_2 < \cdots < H_n = G,
    \]
    where 1 is the trivial subgroup, and where $H_i < H_{i+1}$ (for $0 \leq i < n$). The \textbf{length} of such a series is $n$, or is infinite if there is no end.
\end{definition}
\begin{remark}
    The above definition gives an \textbf{ascending}\index{subgroup series!ascending} subgroup series without repetition. A \textbf{descending}\index{subgroup series!descending} subgroup series without repetition is
    \[
        G = K_0 > K_1 > K_2 > \cdots > K_n = 1,
    \]
    where the indexing is reversed.
\end{remark}

Subgroup series can simplify the study of a group to the study of simpler subgroups and their relations.

\begin{example}
    Consider the dihedral group of order 8, $D_4$. One sees clearly that $H = \{e, r, r^2, r^3\}$ is a subgroup of $D_4$, and that $K = \{e, r^2\}$ is a subgroup of $H$. Thus a possible (ascending) subgroup series without repetition is
    \[
        1 < \{e, r^2\} < \{e, r, r^2, r^3\} < D_4.
    \]
\end{example}

We now define a \textbf{subnormal series} of a group $G$.
\begin{definition}
    Let $G$ be a group. A \textbf{subnormal series}\index{subnormal series} of $G$ is a sequence of subgroups $H_0, \dots, H_n$ such that
    \[
        1 = H_0 \lhd H_1 \lhd H_2 \lhd \cdots \lhd H_n = G.    
    \]
    The quotient groups $H_{i+1}/H_i$ are called the \textbf{factor groups}\index{subnormal series!factor group} of the subnormal series.
\end{definition}
\begin{remark}
    Note that there is no requirement that $H_i \lhd G$, only that $H_i \lhd H_{i+1}$. If all subgroups in a subnormal series are such that $H_i \lhd G$, then such a series is called a \textbf{normal series}\index{normal series} of $G$.
\end{remark}
\begin{remark}
    Similar to subgroup series, there are ascending subnormal series and descending subnormal series.
\end{remark}

\begin{example}
    We consider the dihedral group of order 8 again. One may verify that $H = \{e, r, r^2, r^3\} \lhd D_4$ and $K = \{e, r^2\} \lhd H$. Hence a possible subnormal series of $D_4$ is
    \[
        1 \lhd \{e, r^2\} \lhd \{e, r, r^2, r^3\} \lhd D_4.
    \]

    In fact,
    \begin{itemize}
        \item $1 \lhd D_4$;
        \item $1 \lhd \{e, r^2\} \lhd D_4$; and
        \item $1 \lhd \{e, r, r^2, r^3\} \lhd D_4$
    \end{itemize}
    are all valid subnormal series of $D_4$.
\end{example}

\begin{exercise}
    Let the group $G = \mathbb{Z}_4$.
    \begin{partquestions}{\roman*}
        \item Find a subnormal series of length 2 for $G$.
        \item Hence find the factor groups of that subnormal series.
        \item Explain whether the subnormal series found in \textbf{(i)} is a normal series.
    \end{partquestions}
\end{exercise}

\section{Maximal Normal Subgroup}
Before looking at a more interesting type of subnormal series, we have to introduce the idea of a \textbf{maximal normal subgroup}.
\begin{definition}
    Let $G$ be a group and $N \lhd G$. Then $N$ is said to be a \textbf{maximal normal subgroup}\index{maximal normal subgroup} of $G$ if there does \textbf{not} exist a $H \lhd G$ such that $N < H < G$.
\end{definition}
\begin{remark}
    This means that if $N$ is a maximal normal subgroup of $G$ and there exists a subgroup $H$ such that $N \leq H \leq G$, then either $H = N$ or $H = G$.
\end{remark}

There is an important consequence that we examine here.
\begin{theorem}\label{thrm-maximal-normal-subgroup-iff-quotient-is-simple}
    Let $G$ be a group and $N \lhd G$. Then $N$ is a maximal normal subgroup of $G$ if and only if $G/N$ is simple.
\end{theorem}

\begin{proof}
    We prove the forward direction first; assume that $N$ is a maximal normal subgroup of $G$. Suppose there exists a normal subgroup of $G/N$. Then by \myref{problem-subgroup-of-quotient-group-is-quotient-group} we know that that subgroup takes the form of $H/N$, where $N \leq H \leq G$, such that $H/N \unlhd G/N$. Since $N$ is a maximal normal subgroup this means that $H = N$ or $H = G$. In the first case $H/N \cong 1$ and in the second $H / N = G/N$. Hence $G/N$ only has two normal subgroups, the trivial subgroup and itself, meaning that it has no non-trivial proper normal subgroups and is hence simple.
    
    We now prove the reverse direction; assume that $G/N$ is simple. Suppose there exists a $H \leq G$ such that $N \leq H \leq G$. By \myref{problem-subgroup-of-quotient-group-is-quotient-group} we know $H/N \leq G/N$. In fact, since $N \leq H \leq G$ we know $N \unlhd H \unlhd G$ which means $H/N \unlhd G/N$. As $G/N$ is simple, thus $H/N \cong 1$ or $H/N = G/N$. The case where $H/N \cong 1$ occurs if and only if $H = N$ and the case where $H/N = G/N$ occurs when $H = G$. Hence, this means that $N$ is a maximal normal subgroup of $G$.
\end{proof}

\begin{exercise}
    Find the order of the maximal normal subgroup of $\mathbb{Z}_{120}$.
\end{exercise}

\section{Composition Series}
We can now look at a more interesting type of subnormal series: the \textbf{composition series} of a group.
\begin{definition}
    Let $G$ be a group. A \textbf{composition series}\index{composition series} of $G$ is a subnormal series of finite length
    \[
        1 = H_0 \lhd H_1 \lhd H_2 \lhd \cdots \lhd H_n = G  
    \]
    such that $H_i$ is a maximal normal subgroup of $H_{i+1}$ (for $0 \leq i < n$). The factor groups $H_{i+1}/H_i$ are called the \textbf{composition factors}\index{composition factor}, and the length $n$ is called the \textbf{composition length}\index{composition length}.
\end{definition}
\begin{remark}
    Some authors (e.g. {\cite[p.~257]{cohn_1982}}, \cite{proofwiki_composition-series-definition}, and {\cite[\S 73]{clark_1984}}) require composition series to be a normal series. However we follow {\cite[Definition II.8.3]{hungerford_1980}}, {\cite[Definition 15.12]{humphreys_1996}}, and {\cite[p.~85]{milne_2021}} and others and only require it to be a subnormal series.
\end{remark}
\begin{remark}
    An equivalent definition of a composition series is that it is a normal series where the composition factors $H_{i+1}/H_i$ are simple, as guaranteed by \myref{thrm-maximal-normal-subgroup-iff-quotient-is-simple}.
\end{remark}

For composition series, we often write a group that is isomorphic to the subgroup rather than the actual subgroup. For example, instead of the subgroup $\{0, 3\}$ in $\mathbb{Z}_6$, we may write $\mathbb{Z}_2$ or $\Cn{2}$ instead. Composition series written in this form instead are called composition series \textbf{up to isomorphism}\index{composition series!up to isomorphism}.

\begin{example}
    Consider the group $\mathbb{Z}_{12}$. Three composition series of it are
    \begin{align*}
        &1 \lhd \mathbb{Z}_2 \lhd \mathbb{Z}_6 \lhd \mathbb{Z}_{12};\\
        &1 \lhd \mathbb{Z}_2 \lhd \mathbb{Z}_4 \lhd \mathbb{Z}_{12}; \text{ and}\\
        &1 \lhd \mathbb{Z}_3 \lhd \mathbb{Z}_6 \lhd \mathbb{Z}_{12}.
    \end{align*}
    
    \newpage

    The composition factors in each case (up to isomorphism) are:
    \begin{itemize}
        \item $\mathbb{Z}_2$, $\mathbb{Z}_3$, $\mathbb{Z}_2$;
        \item $\mathbb{Z}_2$, $\mathbb{Z}_2$, $\mathbb{Z}_3$; and
        \item $\mathbb{Z}_3$, $\mathbb{Z}_2$, $\mathbb{Z}_2$.
    \end{itemize}
    Each composition series has composition length 3.
\end{example}

\begin{example}
    Consider $D_6$, the symmetry group of the regular hexagon. The three composition series of $D_6$, up to isomorphism, are
    \begin{align*}
        &1 \lhd \Cn{2} \lhd \Cn{6} \lhd D_6;\\
        &1 \lhd \Cn{3} \lhd \Cn{6} \lhd D_6; \text{ and}\\
        &1 \lhd \Cn{3} \lhd D_3 \lhd D_6.
    \end{align*}
    We note that the subgroup that is isomorphic to
    \begin{itemize}
        \item $\Cn{2}$ is $\{e, s\} = \langle s \rangle$;
        \item $\Cn{3}$ is $\{e, r^2, r^4\} = \langle r^2\rangle$;
        \item $\Cn{6}$ is $\{e, r, r^2, r^3, r^4, r^5\} = \langle r\rangle$; and
        \item $D_3$ is $\{e, r^2, r^4, s, r^2s, r^4s\}$.
    \end{itemize}
    All composition series' lengths are all 3.
\end{example}
\textbf{Important note}: It is crucial to remember that all of the groups written in the composition series (except for the trivial group and the group itself) represent \textbf{subgroups} of the group itself (up to isomorphism), not the actual groups themselves.

\begin{exercise}
    Find the two composition series for $\Cn{6}$ (up to isomorphism), their composition lengths, and their composition factors (up to isomorphism).
\end{exercise}

One might wonder which groups have a composition series. As mentioned, not all groups have a composition series (e.g., $\mathbb{Z}$). However, finite groups will definitely have a composition series.
\begin{theorem}\label{thrm-finite-group-has-composition-series}
    Any finite group has a composition series.
\end{theorem}
\begin{proof}[Proof (see {\cite[Theorem I.8.4]{hungerford_1980}}, \cite{proofwiki_finite-group-has-composition-series})]
    Let $G$ be a finite group. If $G$ is simple then it does not have a non-trivial proper normal subgroup, so its composition series is $\{e\} \lhd G$.
    
    Otherwise, $G$ has a non-trivial proper normal subgroup, at least one of which has maximal order. Let that subgroup be $G_1$. By \myref{thrm-maximal-normal-subgroup-iff-quotient-is-simple} this means that $G/G_1$ is simple. If $G_1$ does not have a non-trivial proper normal subgroup then $G$ has composition series $\{e\} \lhd G_1 \lhd G$. Otherwise let $G_2$ be the maximal normal subgroup of $G_1$, and so on. 
    
    Now Lagrange's Theorem (\myref{thrm-lagrange}) says that the order of the subgroup must divide the order of the group. Since $G_1$, $G_2$, etc. are non-trivial proper subgroups of $G$, the orders of the found subgroups can be at most half the order of the preceding group. Hence the orders of the subgroups are strictly decreasing, meaning
    \[
        |G| > |G_1| > |G_2| > \cdots > |G_n| = 1,
    \]
    with the process terminating at some integer $n$ where $G_n = \{e\}$.

    Now set $H_i = G_{n-i}$. Note $H_{i+1}/H_i = G_i/G_{i+1}$ is simple as discussed above. Hence
    \[
        \{e\} = H_0 \lhd H_1 \lhd H_2 \lhd \cdots \lhd H_n = G    
    \]
    is a composition series of $G$.
\end{proof}

\section{Zassenhaus' Lemma}
Before we continue, we look at \textbf{Zassenhaus' Lemma}, which is sometimes called the \textbf{Fourth Isomorphism Theorem}\index{Isomorphism Theorem!Fourth}. The proof of this lemma is fairly technical, but we try to include as much detail as necessary to clarify difficult parts of the proof.

\begin{lemma}[Zassenhaus]\label{lemma-zassenhaus}\index{Zassenhaus' Lemma}
    Let $G$ be a group. Let $H$, $H'$, $K$, and $K'$ be subgroups of $G$ such that $H' \unlhd H$ and $K' \unlhd K$. Then
    \begin{enumerate}
        \item $H'(H \cap K') \unlhd H' (H \cap K)$;
        \item $K'(H' \cap K) \unlhd K' (H \cap K)$; and
        \item $\frac{H'(H\cap K)}{H'(H\cap K')} \cong \frac{K'(H\cap K)}{K'(H'\cap K)}$,
    \end{enumerate}
    where an expression like $H'(H \cap K')$ denotes the subset product of $H'$ and $H \cap K'$.
\end{lemma}
\begin{proof}[Proof (see {\cite[Lemma I.8.9]{hungerford_1980}})]
    We first note that $H \cap K$ is a non-empty subset of $H$ since $e \in H \cap K$. As $H'$ is a normal subgroup of any non-empty subset of $H$, including $H \cap K$, thus $H' \unlhd H \cap K$. So $H' \cap (H \cap K) \unlhd H \cap K$ by the Diamond Isomorphism Theorem (\myref{thrm-isomorphism-2}), statement 4. Since $H'\unlhd H$ we know $H' \subseteq H$, and so $H' \cap (H \cap K) = H' \cap K$, meaning $H' \cap K \unlhd H \cap K$. Similarly, $H \cap K' \unlhd H \cap K$ by interchanging $H'$ with $K'$. By a corollary of the Diamond Isomorphism Theorem (\myref{corollary-subgroup-product-is-normal-subgroup-if-subgroups-are-normal}), we know that $(H' \cap K)(H \cap K') \unlhd H \cap K$. For brevity, let $\mathcal{A} = (H' \cap K)(H \cap K')$.
    
    Define the map $\phi: H'(H\cap K) \to (H\cap K)/\mathcal{A}$, $hc \mapsto h\mathcal{A}$ where $h \in H'$ and $c \in H \cap K$. We show that $\phi$ is a well-defined homomorphic surjection with a goal to use the Fundamental Homomorphism Theorem to finish the proof.
    
    \newpage

    \begin{itemize}
        \item \textbf{Well-defined}: Suppose $h_1, h_2 \in H'$ and $c_1, c_2 \in H \cap K$ such that $h_1h_2 = c_1c_2$. Then $c_1c_2^{-1} = h_2^{-1}h_1$. We note that
        \[
            (H \cap K) \cap H' = H' \cap K \leq \mathcal{A}    
        \]
        by Diamond Isomorphism Theorem statement 2. Also we see $c_1c_2^{-1} \in (H \cap K) \cap H'$ since $c_1c_2^{-1} \in H \cap K$ and $c_1c_2^{-1} = h_2^{-1}h_1 \in H'$. Thus $c_1c_2^{-1} \in \mathcal{A}$, meaning $c_1\mathcal{A} = c_2\mathcal{A}$ by Coset Equality, statements 1 and 5. Hence
        \[
            \phi(h_1c_1) = c_1\mathcal{A} = c_2\mathcal{A} = \phi(h_2c_2)        
        \]
        which means $\phi$ is well-defined.
        
        \item \textbf{Homomorphism}: We first note that $H \cap K \leq H$ by Diamond Isomorphism Theorem statement 1. Since $H' \unlhd H$ (given), thus $(H\cap K)H' \leq H$ by Diamond Isomorphism Theorem statement 3, which further means $H'(H\cap K) = (H\cap K)H'$ by \myref{prop-subgroup-product-is-subgroup}. So for any $c_1h_2 \in (H\cap K)H' = H'(H\cap K)$ we have $c_1h_2 = h_3c_1$ for some $h_3 \in H'$. Therefore for $h_1, h_2 \in H'$ and $c_1, c_2 \in H \cap K$ we have
        \begin{align*}
            \phi((h_1c_1)(h_2c_2)) &= \phi(h_1(c_1h_2)c_2)\\
            &= \phi(h_1(h_3c_1)c_2)\\
            &= \phi((h_1h_3)(c_1c_2))\\
            &= (c_1c_2)\mathcal{A}\\
            &= (c_1\mathcal{A})(c_2\mathcal{A})\\
            &= \phi(h_1c_1)\phi(h_2c_2)
        \end{align*}
        which shows that $\phi$ is a homomorphism.
        
        \item \textbf{Image}: We prove that $\phi$ is surjective to show $\im \phi = (H\cap K)/\mathcal{A}$. Take any $c\mathcal{A} \in (H\cap K) / \mathcal{A}$. Clearly $e \in H$, so $\phi(c) = \phi(ec) = c\mathcal{A}$, meaning $\phi$ is surjective. Hence $\im \phi = (H\cap K)/\mathcal{A}$.
        
        \item \textbf{Kernel}: Let $hc \in \ker\phi$ where $h \in H'$ and $c \in H \cap K$. Then $\phi(hc) = \mathcal{A}$, meaning $c\mathcal{A} = \mathcal{A}$ and so $c \in \mathcal{A}$. Let $c = xy$ where $x \in H'\cap K$ and $y \in H \cap K'$. So $hc \in \ker\phi$ if $hc = h(xy) = (hx)y$. Note that $h \in H'$, and $x \in H' \cap K \subseteq H'$ so $x \in H'$, thus $hx \in H'$. Also $y \in H \cap K'$, so $hc = (hx)y \in H'(H \cap K')$. Thus if $hc \in \ker\phi$ then $hc \in H'(H \cap K')$, meaning $\ker\phi \subseteq H'(H \cap K')$.
        
        Now take $hy \in H'(H \cap K')$ where $h \in H'$ and $y \in H \cap K'$. Since $H \cap K' \unlhd H \cap K$, thus $H \cap K' \subseteq H \cap K$. Thus $\phi(hy) = y\mathcal{A}$. Note $H \cap K' \leq (H' \cap K)(H \cap K') = \mathcal{A}$ by Diamond Isomorphism Theorem statement 2, so $y \in H \cap K' \subseteq \mathcal{A}$ which means $y \in \mathcal{A}$, meaning $y\mathcal{A} = \mathcal{A}$. Hence $\phi(hy) = \mathcal{A}$, so $hy \in \ker\phi$. Therefore any $hy \in H'(H \cap K')$ also belongs in $\ker\phi$, meaning $H'(H \cap K') \subseteq \ker\phi$.
        
        As $\ker\phi \subseteq H'(H\cap K')$ and $H'(H\cap K') \subseteq \ker\phi$, therefore $\ker\phi = H'(H \cap K')$.
    \end{itemize}
    
    Now finding that $\ker\phi = H'(H \cap K')$, and knowing that $\ker\phi \unlhd H'(H\cap K)$ (\myref{prop-kernel-is-normal-subgroup-of-domain}), means that $H'(H \cap K') \unlhd H'(H\cap K)$, proving statement 1 of this lemma.
    
    A similar argument follows for a similar map $\psi: K'(H\cap K) \to (H\cap K)/\mathcal{A}$, which yields $\im\psi = (H\cap K)/\mathcal{A}$ and $\ker\psi = K'(H\cap K)$ which proves statement 2.
    
    The final step is to apply the Fundamental Homomorphism Theorem (\myref{thrm-isomorphism-1}) to both $\phi$ and $\psi$. This yields
    \[
        \frac{H'(H\cap K)}{H'(H\cap K')} \cong \frac{H\cap K}{\mathcal{A}} 
        \text{ and }
        \frac{K'(H\cap K)}{K'(H'\cap K)} \cong \frac{H\cap K}{\mathcal{A}}
    \]
    whence statement 3 immediately follows.
\end{proof}

\section{Schreier Refinement Theorem and Jordan-H\"older Theorem}
This next section proves two fundamental theorems about composition series, but will be fairly technical.

Before we look at them, we introduce \textbf{refinements}.
\begin{definition}
    Let $G$ be a group. Let $\mathcal{H} = \langle H_i \rangle_{i\in\{0,\dots,m\}}$ and $\mathcal{K} = \langle K_j \rangle_{j\in\{0,\dots,n\}}$ be subnormal series for $G$, where
    \begin{align*}
        1 = H_0 \lhd H_1 \lhd H_2 \lhd \cdots \lhd H_m &= G \text{ and}\\
        1 = K_0 \lhd K_1 \lhd K_2 \lhd \cdots \lhd K_n &= G.
    \end{align*}
    If every $H_i$ appears in $\mathcal{K}$ (i.e., $\mathcal{H}$ is a subsequence of $\mathcal{K}$), then $\mathcal{K}$ is a \textbf{refinement}\index{refinement} of $\mathcal{H}$. Furthermore, $\mathcal{K}$ is a \textbf{proper refinement}\index{refinement!proper} of $\mathcal{H}$ if at least one $K_j$ is not in $\mathcal{H}$.
\end{definition}
\begin{example}
    We previously found that
    \[
        1 \lhd \{e, r^2\} \lhd D_4 \text{ and } 1 \lhd \{e, r^2\} \lhd \{e, r, r^2, r^3\} \lhd D_4.    
    \]
    Thus the subnormal series $1 \lhd \{e, r^2\} \lhd \{e, r, r^2, r^3\} \lhd D_4$ is a proper refinement of the subnormal series $1 \lhd \{e, r^2\} \lhd D_4$.
\end{example}

We can now introduce the \textbf{Schreier Refinement Theorem}.
\begin{theorem}[Schreier Refinement Theorem]\label{thrm-schreier}\index{Schreier Refinement Theorem}
    Two (sub)normal series for any finite group have refinements of equal length whose factor groups are isomorphic.
\end{theorem}
\begin{remark}
    This theorem is also called the \textbf{Schreier-Zassenhaus Theorem} (e.g. in {\cite[\S 72]{clark_1984}}, \cite{proofwiki_schreier-zassenhaus}) due to the use of Zassenhaus' Lemma in the proof.
\end{remark}

\begin{proof}[Proof (see {\cite[\S 72]{clark_1984}} and \cite{proofwiki_schreier-zassenhaus})]
    Let $G$ be a finite group, and suppose that
    \begin{equation}\label{eqn:thrm-schreier-eqn-a}
        1 = H_0 \lhd H_1 \lhd H_2 \lhd \cdots \lhd H_m = G    
        \tag{A}
    \end{equation}
    and
    \begin{equation}\label{eqn:thrm-schreier-eqn-b}
        1 = K_0 \lhd K_1 \lhd K_2 \lhd \cdots \lhd K_n = G    
        \tag{B}
    \end{equation}
    are two subnormal series of $G$. We shall refine the subnormal series \myref{eqn:thrm-schreier-eqn-a} to a series \myref{eqn:thrm-schreier-eqn-a-prime} and show that it is isomorphic to a refinement \myref{eqn:thrm-schreier-eqn-b-prime} of \myref{eqn:thrm-schreier-eqn-b}.
    
    Let a new series be formed
    \begin{equation}\label{eqn:thrm-schreier-eqn-a-prime}
        1 = \hat{H}_0 \subseteq \hat{H}_1 \subseteq \hat{H}_2 \subseteq \cdots \subseteq \hat{H}_{mn - 1} \subseteq \hat{H}_{mn} = G
        \tag{A'}
    \end{equation}
    such that $\hat{H}_k = H_q(H_{q+1}\cap K_r)$ for $k = qn + r$, where $0 \leq q < m$ and $0 \leq r \leq n$. It needs to be demonstrated that $\hat{H}_k$ is well-defined. Consider the ambiguity of the $k = qn + r$ in the case where $r = n$, so $k = qn + n = (q+1)n + 0$. For \myref{eqn:thrm-schreier-eqn-a-prime} to be well defined, we require
    \[
        H_q(H_{q+1} \cap K_n) = H_{q+1}(H_{q+2} \cap K_0).
    \]
    We note $H_q \subseteq H_{q+1}$. Thus the left-hand side is
    \begin{align*}
        H_q(H_{q+1} \cap K_n) &= H_q(H_{q+1} \cap G) & (\text{note that } K_n = G)\\
        &= H_qH_{q+1}\\
        &= H_{q+1} & (\text{by \myref{problem-subgroup-product-equal-to-subgroup-if-one-is-subgroup-of-another}})
    \end{align*}
    while the right-hand side is
    \begin{align*}
        H_{q+1}(H_{q+2} \cap K_0) &= H_{q+1}(H_{q+2} \cap \{e\})\\
        &= H_{q+1}(\{e\})\\
        &= H_{q+1},
    \end{align*}
    so this is well-defined.
    
    We next need to demonstrate that $\hat{H}_k \unlhd \hat{H}_{k+1}$. Write $k$ as $qn+r$ in which $0 \leq q < m$ and $0 \leq r < n$. Hence
    \[
        \hat{H}_{k+1} = \hat{H}_{qn+(r+1)} = H_q(H_{q+1}\cap K_{r+1}). 
    \]
    Since $K_r \lhd K_{r+1}$ we have
    \[
        \hat{H}_k = H_q(H_{q+1}\cap K_r) \unlhd H_q(H_{q+1}\cap K_{r+1}) = \hat{H}_{k+1}
    \]
    by Zassenhaus' Lemma (\myref{lemma-zassenhaus}), statement 1. Thus $\hat{H}_k \unlhd \hat{H}_{k+1}$, which is what we wanted to prove. However we note that \myref{eqn:thrm-schreier-eqn-a-prime} may not be a subnormal series, as it is possible for $\hat{H}_k = \hat{H}_{k+1}$ for some $k$. We address this concern near the end.
    
    Having created series \myref{eqn:thrm-schreier-eqn-a-prime}, use the same procedure to form
    \begin{equation}\label{eqn:thrm-schreier-eqn-b-prime}
        1 = \hat{K}_0 \subseteq \hat{K}_1 \subseteq \hat{K}_2 \subseteq \cdots \subseteq \hat{K}_{mn - 1} \subseteq \hat{K}_{mn} = G
        \tag{B'}
    \end{equation}
    such that
    \[
        \hat{K}_l = H_q(H_{q+1}\cap K_r)
    \]
    for $l = qm + r$, where $0 \leq q < n$ and $0 \leq r \leq m$. The conclusions made for \myref{eqn:thrm-schreier-eqn-a-prime} apply to \myref{eqn:thrm-schreier-eqn-b-prime}.
    
    Now let $k = un + v$ and let $l = vm + u$. It follows that
    \begin{align*}
        \hat{H}_{k+1}/\hat{H}_k &= \frac{H_u(H_{u+1}\cap K_{v+1})}{H_u(H_{u+1}\cap K_v)}\\
        &\cong \frac{K_v(K_{v+1}\cap H_{u+1})}{K_v(K_{v+1}\cap H_u)}\\
        &= \hat{K}_{l+1}/\hat{K}_l
    \end{align*}
    with the isomorphism achieved by Zassenhaus' Lemma, statement 3. Thus \myref{eqn:thrm-schreier-eqn-a-prime} and \myref{eqn:thrm-schreier-eqn-b-prime} have isomorphic factors.
    
    We return to the concern about the possibility that $\hat{H}_k = \hat{H}_{k+1}$ (or $\hat{K}_l = \hat{K}_{l+1}$) for some $k$ (or $l$). Eliminating redundancies in both \myref{eqn:thrm-schreier-eqn-a-prime} and \myref{eqn:thrm-schreier-eqn-b-prime} would yield two refinements of equal length, whose factor groups are isomorphic.
    
    Finally, observe that if both original series \myref{eqn:thrm-schreier-eqn-a} and \myref{eqn:thrm-schreier-eqn-b} were normal series of $G$, then each $\hat{H}_k$ and $\hat{K}_l$ would be a normal subgroup of $G$, so their respective refined series would also be a normal series.
\end{proof}

We can now introduce and prove the \textbf{Jordan-H\"older Theorem}.
\begin{theorem}[Jordan-H\"older]\label{thrm-jordan-holder}\index{Jordan-H\"older Theorem}
    Two composition series for a finite group have the same length and composition factors.
\end{theorem}
\begin{proof}
    The Schreier Refinement Theorem (\myref{thrm-schreier}) tells us that two subnormal series have refinements of equal length whose factors are isomorphic.
    
    Now a composition series has no proper refinements since all subgroups inside a composition series are maximal normal subgroups. Hence any refinements must be identical to the composition series themselves.
    
    Thus any two composition series (which are their own refinements) for a finite group have the same length and composition factors.
\end{proof}

\begin{example}
    We found earlier that $D_6$ has 3 distinct composition series:
    \begin{align*}
        &1 \lhd \Cn{2} \lhd \Cn{6} \lhd D_6;\\
        &1 \lhd \Cn{3} \lhd \Cn{6} \lhd D_6; \text{ and}\\
        &1 \lhd \Cn{3} \lhd D_3 \lhd D_6.
    \end{align*}
    The factors in each case (in order) are:
    \begin{itemize}
        \item $\Cn{2}/1 \cong \Cn{2}$, $\Cn{6}/\Cn{2} \cong \Cn{3}$ (\myref{exercise-Zmn-mod-Zn-cong-Zn}), $D_6 / \Cn{6}$;
        \item $\Cn{3}/1 \cong \Cn{3}$, $\Cn{6}/\Cn{3} \cong \Cn{2}$, $D_6 / \Cn{6}$; and
        \item $\Cn{3}/1 \cong \Cn{3}$, $D_3 / \Cn{3} \cong \Cn{2}$, $D_6 / D_3$.
    \end{itemize}
    By the Jordan-H\"older theorem (\myref{thrm-jordan-holder}), the factors must be isomorphic to each other. Looking at the last 2 factor lists, only one factor from each is unaccounted for, namely $D_6 / \Cn{6}$ and $D_6 / D_3$. Since Jordan-H\"older guarantees that they are isomorphic to each other, hence, technically, $D_6 / \Cn{6} \cong D_6 / D_3$.

    \textbf{Important note}: The $\Cn{6}$ and $D_3$ above represent \textbf{subgroups} of $D_6$ (up to isomorphism), not the actual groups themselves. The subgroup of $D_6$ that is isomorphic to $\Cn{6}$ is $\langle r \rangle$, while the subgroup isomorphic to $D_3$ is $\langle r^2, s\rangle$. Therefore the correct statement is
    \[
        D_6 / \langle r \rangle \cong D_6 / \langle r^2, s\rangle.
    \]
\end{example}

\begin{exercise}
    Let $p$ be a prime. Show that any group with order $p^2$ has only one composition series.
\end{exercise}

\newpage

\section{Problems}
\begin{problem}
    Let the \textit{Klein four-group}\index{Klein four-group} with presentation
    \[
        \langle a, b \vert a^2 = b^2 = (ab)^2 = e \rangle
    \]
    be denoted by $\mathrm{V}$. Recall that $\mathrm{V} \cong (\mathbb{Z}_2)^2$.
    \begin{partquestions}{\roman*}
        \item Find the unique composition series for $\mathrm{V}$ up to isomorphism.
        \item Let $\mathrm{Q}$ denote the quaternion group. Find the two unique composition series for $\mathrm{Q}$ up to isomorphism.\newline
        (\textit{Hint: consider the alternate definition of the quaternion group.})
        \item Let $H$ be the subgroup of $\mathrm{Q}$ that is isomorphic to $\Cn{4}$, and $K$ be a subgroup of $\mathrm{Q}$ that is isomorphic to $\mathrm{V}$. Prove that
        \[
            \mathrm{Q}/H \cong \mathrm{Q}/K.
        \]
        (\textit{Hint: consider \myref{problem-cartesian-product-of-group-by-group-isomorphic-to-group}.})
    \end{partquestions}
\end{problem}

\begin{problem}\label{problem-S4-composition-series}
    Find the unique composition series of $\Sn{4}$ (up to isomorphism).\newline
    (\textit{Hint: consider the orders of elements in $\An{4}$ and \myref{problem-subgroup-of-index-2} to prove why a particular subgroup with a particular order cannot exist.})
\end{problem}
