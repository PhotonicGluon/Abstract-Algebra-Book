\documentclass[
    a5paper,
    pagesize,
    11pt,
    bibtotoc,
    normalheadings,
    twoside,
    openany,
    chapterprefix,
    DIV=9
]{scrbook}

\usepackage[utf8]{inputenc}
\usepackage{tocloft}
\usepackage{mathtools}
\usepackage{amsfonts}
\usepackage{enumitem}
\usepackage{amsmath}
\usepackage{amsthm}
\usepackage{amssymb}
\usepackage[hmargin=2cm, vmargin=2.5cm]{geometry}
\usepackage{graphicx}
\usepackage{wrapfig}
\usepackage{parskip}
\usepackage{framed}
\usepackage{fancyhdr}
\usepackage{emptypage}
\usepackage{multicol}
\usepackage{imakeidx}
\usepackage[breaklinks]{hyperref}
\usepackage[capitalise, nameinlink]{cleveref}
\usepackage[x11names]{xcolor}
\usepackage{crossreftools}

\usepackage[
    backend=bibtex,
    style=alphabetic,
    sorting=ynt
]{biblatex}

%=========== Path to images ==============
\graphicspath{{./images/}}

%============== Resources ================
\addbibresource{../AbstractAlgebra.bib}

%============ Redefinitions ==============
\let\oldemptyset\emptyset
\let\emptyset\varnothing

\let\totient\varphi

\renewcommand{\vert}{ \ \vline \ }
\newcommand{\vertalt}{ \ | \ }

\newcommand{\myref}[1]{\textbf{\crthypercref{#1}}}
\newcommand{\myreffigures}[1]{\textbf{\cref{#1}}}

\renewcommand{\qedsymbol}{\ensuremath{\blacksquare}}

%=========== Theorem Styles ==============
\newtheoremstyle{theorem-style}
    {-12pt}      % Space above
    {-5pt}       % Space below
    {}           % Font to use in the theorem
    {0pt}        % Measure of space to indent
    {\bfseries}  % Name of the head font
    {.}          % Punctuation between head and body
    { }          % Space after theorem head; " " = normal inter-word space
    {\thmname{#1}\thmnumber{ #2}\textit{\thmnote{ (#3)}}}

\newtheoremstyle{definition-style}
    {-12pt}      % Space above
    {-5pt}       % Space below
    {}           % Font to use in the definition
    {0pt}        % Measure of space to indent
    {\bfseries}  % Name of the head font
    {.}          % Punctuation between head and body
    { }          % Space after theorem head; " " = normal inter-word space
    {\thmname{#1}\thmnumber{ #2}\textnormal{\thmnote{ (#3)}}}

\newtheoremstyle{exercise-style}
    {-5pt}       % Space above
    {\topsep}    % Space below
    {}           % Font to use in the exercise
    {0pt}        % Measure of space to indent
    {\bfseries}  % Name of the head font
    {.}          % Punctuation between head and body
    { }          % Space after theorem head; " " = normal inter-word space
    {\thmname{#1}\thmnumber{ #2}\textnormal{\thmnote{ (#3)}}}

%======== Theorem-Like Things ============
\theoremstyle{theorem-style}\newtheorem{theoremhidden}{Theorem}[section]
\renewcommand{\thetheoremhidden}{\Roman{part}.\arabic{chapter}.\arabic{section}.\arabic{theoremhidden}}

\theoremstyle{theorem-style}\newtheorem{lemmahidden}[theoremhidden]{Lemma}

\theoremstyle{theorem-style}\newtheorem{propositionhidden}[theoremhidden]{Proposition}

\theoremstyle{theorem-style}\newtheorem{corollaryhidden}[theoremhidden]{Corollary}

\theoremstyle{definition-style}\newtheorem{definitionhidden}[theoremhidden]{Definition}

\theoremstyle{exercise-style}\newtheorem{exercisehidden}{Exercise}[chapter]
\renewcommand{\theexercisehidden}{\Roman{part}.\arabic{chapter}.\arabic{exercisehidden}}

\theoremstyle{definition}\newtheorem{problem}{Problem}[chapter]
\renewcommand{\theproblem}{\Roman{part}.\arabic{chapter}.\arabic{problem}}

\theoremstyle{definition}\newtheorem*{remark}{Remark}
\theoremstyle{definition}\newtheorem{example}[theoremhidden]{Example}

%============ Environments ===============
\newenvironment{theorem}
{\definecolor{shadecolor}{named}{DarkSeaGreen2}\begin{shaded}\noindent\begin{theoremhidden}}
{\end{theoremhidden}\end{shaded}}

\newenvironment{lemma}
{\definecolor{shadecolor}{named}{Honeydew2}\begin{shaded}\noindent\begin{lemmahidden}}
{\end{lemmahidden}\end{shaded}}

\newenvironment{proposition}
{\definecolor{shadecolor}{named}{Honeydew1}\begin{shaded}\noindent\begin{propositionhidden}}
{\end{propositionhidden}\end{shaded}}

\newenvironment{corollary}
{\definecolor{shadecolor}{named}{DarkSeaGreen1}\begin{shaded}\noindent\begin{corollaryhidden}}
{\end{corollaryhidden}\end{shaded}}

\newenvironment{definition}
{\definecolor{shadecolor}{named}{LightCyan1}\begin{shaded}\noindent\begin{definitionhidden}}
{\end{definitionhidden}\end{shaded}}

\newenvironment{exercise}
{\begin{framed}\noindent\begin{exercisehidden}}
{\end{exercisehidden}\end{framed}}

%=========== Custom Commands =============
\newcommand{\code}[1]{\texttt{#1}}  % Code block
\makeatletter\newcommand*{\rom}[1]{\Ifstr{#1}{0}{0}{\expandafter\@slowromancap\romannumeral #1@}}\makeatother  % Roman numeral

\newcommand{\lcm}{\mathrm{lcm}}  % Lowest common multiple function
\newcommand{\sgn}{\mathrm{sgn}}  % Signum function

\newcommand{\im}{\mathrm{im}\;}  % Image of a function
\newcommand{\id}{\mathrm{id}}    % Identity function

%======== Custom Chapter Styling =========
\makeatletter
\renewcommand{\chaptermark}[1]{
    \markboth{\if@mainmatter\chapapp~\thechapter.\ \fi#1}{}
}

\renewcommand*{\chapterformat}{
  \MakeUppercase{\chapapp\nobreakspace\thechapter}
}

\renewcommand*{\chapterlineswithprefixformat}[3]{
    \Ifstr{#1}{chapter}{
        \vspace{-60px}
        \Ifstr{#2}{\empty}{\vspace{40px}}{\raggedleft#2}
        \vspace{-15px}
        \rule{\linewidth}{1pt}\par\nobreak
        \centering{#3}
        \vspace{-10px}
        \rule{\linewidth}{1pt}\par\nobreak
        \vspace{-10px}
    }{#2#3}
}
\makeatother

%======== Figure Caption Format ==========
\usepackage[labelfont=bf]{caption}
\DeclareCaptionLabelFormat{custom}{#1 \Roman{part}.#2.}
\captionsetup{labelformat=custom,labelsep=space}

%============ Custom Header ==============
\fancypagestyle{plain}{\fancyhf{}\renewcommand{\headrulewidth}{0pt}}  % To clear page numbers from footer, and header line at the start of every chapter

\pagestyle{fancy}
\fancyhf{}  % Clear header/footer

\fancyhead[LE,RO]{\thepage}
\fancyhead[LO,RE]{\textit{\nouppercase\leftmark}}

%========= Customise TOC Heading =========
\makeatletter
\def\createtoc{
    \renewcommand\tableofcontents{
        \chapter*{\contentsname}
        \@starttoc{toc}
    }
    \tableofcontents
}
\makeatother

%======= Customise Draft Watermark =======
\newcommand{\setasdraft}{
    \usepackage{draftwatermark}
    \SetWatermarkLightness{0.95}
    \SetWatermarkScale{5}
}

%========= Front Matter Pages ============
\def\volumetitle{Volume \rom{\volumenumber}: \volumename}

\def\frontmatterpages{
    \frontmatter  % Use lowercase roman numerals for page numbers

    % Title page
    \begin{titlepage}
        \centering{
            \selectfont
            \Huge
            \textbf{Abstract Algebra}\\
            \vspace{-0.2cm}
            
            \Large
            \textbf{A Simple Introduction}\\
            \vspace{0.5cm}
            
            \LARGE
            \volumetitle
            \vspace{2cm}
        }\\
        \centering{\Large{Overwrite}}
        \vspace{\fill}

        \includegraphics[width=5cm]{\volumeimage}
        \vspace{\fill}

        \centering \small{\textit{Version \version}}
    \end{titlepage}

    \newpage{}

    % Edition notice
    \clearpage\null\vfill
    \thispagestyle{empty}
    \begin{minipage}[b]{0.9\textwidth}
        \footnotesize\raggedright
        \setlength{\parskip}{0.5\baselineskip}

        Published by Kan Onn Kit\\
        Singapore
        \vspace{5cm}

        \textbf{Abstract Algebra: A Simple Introduction -- \volumetitle}\par
        Version \version
        \vspace{0.3cm}

        Copyright \copyright \ 2022 -- \the\year\ by Kan Onn Kit\par
        This work is licensed under a
        Creative Commons Attribution-NonCommercial-ShareAlike 4.0 International Licence.\par
        \includegraphics[width=2.5cm]{../Images/CC BY-NC-SA 4.0.png}\\  % With reference to the volumes' folders
        The full licence text is available at \url{http://creativecommons.org/licenses/by-nc-sa/4.0/}.\par    
        The source files for the project are available \href{https://github.com/PhotonicGluon/Abstract-Algebra-Book}{here}.
        \vspace{0.3cm}

        Typeset in 11pt Computer Modern Roman using PDF\LaTeX.
    \end{minipage}

    \vspace*{2\baselineskip}
    \cleardoublepage

    % "Quote" page
    \thispagestyle{empty}
    \vspace*{2cm}

    \begin{center}
        \Large{\parbox{10cm}{
            \begin{raggedright}
                \Large
                \quotepagetext
                \vspace{0.3cm}
                
                \hfill
                --- \quotepageattribution\\
                \vspace{-0.25cm}
                
                \hfill
                \normalsize
                (\quotepagecitation)
            \end{raggedright}
        }
    }
    \end{center}

    \newpage

    % Table of contents
    \createtoc
    \setcounter{part}{\volumenumber}

    % Acknowledgements
    \chapter{Acknowledgements}
    Undertaking such a monumental project is new to me, and I am indebted to the people who accompanied me on this journey.

    I am eternally grateful to my parents, who have spent countless hours and an ungodly amount of effort to raise me into who I am today. Their omnipresent kindness, patience, and love for me are something I certainly do not deserve, and I thank them for taking care of me.
    
    I would like to thank my tutor Leong Chong Ming, who got me interested in abstract algebra in the first place. His enthusiasm and eagerness in sharing his knowledge on the subject is the driving force behind my decision to write these books.

    I am grateful for the help of my friend Low Ji Yuan, who has assisted me with countless revisions of the content in these books and given me another pair of eyes in the vetting of content.

    I also sincerely appreciate the support from my mathematics tutors, Loke Weng Heng, Siow Yun Jie, and Teng Yen Ping, who has been there through my junior college years inspiring me with the wonders of mathematics. I am indebted to them for allowing me to excel in my final examinations.

    My close friends, Aidan Tay, Gabriel Fong, and Low Ji Yuan, accompanied me through two years of schooling (and math jokes). I offer infinite thanks to them for sticking with me and for encouraging this math nerd to pursue his wacky projects.

    A thousand thanks go out to my teachers at the School of Science and Technology, Singapore, and specifically my form teacher Lee Tsi Yew Samuel, who instilled important character values into me so I can excel in my future endeavours.

    % Preface
    \chapter{Preface}
    Although algebra has a long history, it has undergone quite striking changes in the past few decades. Abstract (or modern) algebra is widely recognised as an essential element of higher mathematical education. The results that it showcases, however, are often hard to grasp and understand without prerequisite knowledge or with a heavy background in mathematics. Most books on this subject are crafted for undergraduates at universities. They are not for a general mathematics enthusiast or one who seeks to understand more about the inner structure of algebra that mathematicians encounter frequently.

    The exploration of such structures is fundamental to the current underpinning of scientific inquiries. For example, groups are important as they describe the symmetries which the laws of physics seem to obey. Finite fields are also used in coding theory and combinatorics. I hope this series of books will inspire more people to learn more about abstract algebra, beyond the simple introduction presented here.

    This series of books serves to achieve several goals.
    \begin{itemize}
        \item Provide a step-by-step explanation of core results from abstract algebra, without ambiguity of the results discussed.
        \item Demystify the core steps that many textbooks skip over when writing proofs.
        \item Ensure that results from abstract algebra are as accessible, as approachable, and as understandable for as many people as possible.
    \end{itemize}
    I hope that these books can accomplish these goals and let readers enjoy the wonders of abstract algebra.

    \hfill{\textit{22 March, 2023}}

    \section*{Preface for Volume \rom{\volumenumber}}
    \prefacevolumetext
    
    \hfill{\textit{\prefacevolumedate}}

    % Suggestions on the use of this book
    \chapter{Suggestions on the Use of This Book}
    \section*{General Information}
    \begin{itemize}
        \item For most volumes, we include both exercises and problems.
        \begin{itemize}
            \item An exercise can be thought of as a simple ``self-review'' question. Exercises ensure that the content of a particular section is understood and should not be too hard to answer.
            \item A problem is a more holistic version of an exercise. Generally, solutions to problems require a thorough understanding of the current chapter and may require results from other chapters.
        \end{itemize}
        \item A consistent labelling system for all the results within and between volumes is necessary for a project as long as this one.
        \begin{itemize}
            \item All definitions, examples, lemmas, theorems, propositions, and corollaries are consecutively numbered, using the format
            \begin{quote}
                \code{[VOLUME].[CHAPTER].[SECTION].[NUMBER]}
            \end{quote}
            For example, the fourth statement in Volume I, chapter 2, section 3 is labelled \textbf{I.2.3.4}.
            \item Exercises and problems are also numbered consecutively, using the format
            \begin{quote}
                \code{[VOLUME].[CHAPTER].[NUMBER]}
            \end{quote}
            For example, the third exercise in Volume I, chapter 2 is labelled \textbf{I.2.3}. Likewise, the fourth exercise in Volume II, chapter 3 is labelled \textbf{II.3.4}.
        \end{itemize}
        \item Volume numbers are always written in Roman numerals, except for Volume 0 which will be written as a zero.
        \item The symbol ``$\qedsymbol$'' marks the end of a proof.
    \end{itemize}

    \section*{Chapter Interdependence}
    The diagram on the next page shows chapter interdependence. It should be used in conjunction with the table of contents and notes listed.

    \newpage
    \includegraphics[width=\linewidth]{Interdependence.png}
    
    \newpage

    \textbf{Notes}:
    \interdependencenotes

    \mainmatter  % Now use arabic numerals for page numbers
}

%============= Index Pages ===============
\usepackage[
    totoc,
    columnsep=20pt,
    hangindent=8pt,
    subindent=20pt,
    subsubindent=30pt
]{idxlayout}

\makeindex[options= -s ../index-style.ist]

%======= Bibliography Formatting =========
% These two lines are here to ensure that URLs do not exceed the page by too much
\setcounter{biburllcpenalty}{7000}
\setcounter{biburlucpenalty}{8000}


%=========== Global Variables ============
\newcommand{\version}{0.10}
\newcommand{\volumenumber}{0}
\newcommand{\volumename}{Prerequisites}
\newcommand{\volumeimage}{cover/Venn Diagram.png}

%============= Formatting ================
\linespread{1.05}

%======== Theorem-Like Things ============
\renewcommand{\thetheorem}{\arabic{part}.\arabic{chapter}.\arabic{section}.\arabic{theorem}}
\renewcommand{\theexercisehidden}{\arabic{part}.\arabic{chapter}.\arabic{exercisehidden}}
\renewcommand{\theproblem}{\arabic{part}.\arabic{chapter}.\arabic{problem}}

%========= Front Matter Pages ============
% Quote page
\newcommand{\quotepagetext}{
    Aus dem Paradies, das Cantor uns geschaffen, soll uns niemand vertreiben k\"{o}nnen.\\
    (\textit{No one shall expel us from the Paradise that Cantor has created.})
}
\newcommand{\quotepageattribution}{David Hilbert, 1926}
\newcommand{\quotepagecitation}{\cite[p.~170]{hilbert_1926}}

% Preface
\newcommand{\prefacevolumetext}{
    To understand the subject material covered in the later volumes, it is essential that the prerequisites and fundamentals are understood. Volume 0 serves to give one sufficient knowledge to dive into the meat of abstract algebra. We cover basic set theory, functions/mappings, mathematical logic and proof writing, elementary number theory, and simple modular arithmetic, which should be plenty for one to understand the content covered in future volumes.
}
\newcommand{\prefacevolumedate}{22 March, 2023}

% Suggestions of use
\newcommand{\interdependencenotes}{
    \begin{itemize}
        \item All chapters of this volume are essential to understand for future volumes. Readers who are familiar with the content may skip this volume.
        \item Knowledge of chapter 1 is needed for both chapters 2 and 3.
        \item The only thing that is needed from chapter 2 for chapter 3 is knowledge of function notation. In particular, \myref{example-strong-induction-on-function} is the only example that requires knowledge of functions.
        \item Chapters 4 and 5 are independent from the other chapters of this book.
        \item B\'{e}zout's Lemma (\myref{lemma-bezout}) is the sole result that is used from chapter 4 in chapter 5.
    \end{itemize}
}

%=========================================
\begin{document}
\frontmatterpages

%=========================================
\chapter{Set Theory}
Set theory is the branch of mathematics that studies sets, which can be informally described as collections of objects.

Set theory begins with a binary relation between an object $x$ and a set\index{set} $S$. If $x$ is an element\index{element} of $S$, we write $x \in S$. This is read as ``$x$ is an element of the set $S$''. Otherwise, we write $x \notin S$. For convenience, if $x \in S$ and $y \in S$, we may write $x, y \in S$. If $z \in S$ also, we may write $x, y, z \in S$. The same applies for ``$\notin$''.

A set is described by listing elements separated by commas, or by a characterizing property of its elements, within braces $\{ \ \}$. Since sets are objects, the membership relation can relate sets as well.

A set is completely determined by its elements. Two sets are equal if and only if they have precisely the same elements.

We denote the \textbf{empty set}\index{set!empty} by $\emptyset$, which is given by $\{\}$. That is, $\emptyset = \{\}$.

Now if we have two sets $A$ and $B$, and if all the members of set A are also members of set B, then we say that $A$ is a \textbf{subset}\index{subset} of $B$, and we write $A \subseteq B$. For example, $\{1, 2\} \subseteq \{1, 2, 3\}$ but $\{1, 4\} \not\subseteq \{1, 2, 3\}$. It should be noted that for any set $S$, the empty $\emptyset \subseteq S$. It should also be noted that $S \subseteq S$ for any set $S$. Thus $A$ is called a \textbf{proper subset}\index{subset!proper} of $B$ if $A \neq B$ and $A \subseteq B$ (in this case, we write $A \subset B$).

\begin{exercise}
    Let $S$ be a non-empty set. Determine whether the following statements are true or false.

    \begin{multicols}{2}
        \begin{enumerate}[label=(\alph*)]
            \item $\{1, 2\} \subset \{1, 2, 3, 4\}$
            \item $\{1, 2, 3\} \subseteq \{1, 2, 4\}$
            \item $\emptyset \subseteq \emptyset$
            \item $S \subset S$
            \item $S \in \{S, \emptyset\}$
            \item $\{S\} \notin \{S, \emptyset\}$
            \item $S \subseteq \{S, \emptyset\}$
            \item $\{S\} \subseteq \{S, \emptyset\}$
        \end{enumerate}
    \end{multicols}
\end{exercise}

Set theory also features operations on sets. We list some of them here.
\begin{itemize}
    \item The \textbf{union}\index{set!union} of two sets is the set of all objects that are a member of $A$, or $B$, or both. It is denoted $A \cup B$. For example, $\{1, 2, 3\} \cup \{2, 3, 4\} = \{1, 2, 3, 4\}$.
    \item The \textbf{intersection}\index{set!intersection} of two sets is the set of all objects that are a member of \textit{both} the sets $A$ and $B$. It is denoted $A \cap B$. For example, $\{1, 2, 3\} \cap \{2, 3, 4\} = \{2, 3\}$.
    \item The \textbf{set difference}\index{set!difference} of $S$ and $A$, denoted $S \setminus A$, is the set of all members of $S$ that are not members of $A$. For example, $\{1, 2, 3\} \setminus \{2, 3, 4\} = \{1\}$ and $\{2, 3, 4\} \setminus \{1, 2, 3\} = \{4\}$.
    \item The \textbf{Cartesian Product}\index{Cartesian Product} of $A$ and $B$, denoted $A \times B$, is the set whose members are all possible ordered pairs $(a, b)$, where $a$ is an element of $A$ and $b$ is an element of $B$. For example,
    \[
        \{1, 2, 3\} \times \{4, 5\} = \{(1, 4), (1, 5), (2, 4), (2, 5), (3, 4), (3, 5)\}.
    \]
    In particular, the Cartesian product $A \times A = A^2$, $A\times A \times A = A^3$, and so on.
\end{itemize}

\begin{exercise}
    Let $S = \{1, 2, 3, 4\}$, $T = \{2, 3, 5\}$, $U = \{(2, 2), (3, 3), (5, 5)\}$. Determine whether the following statements are true or false.
    \begin{multicols}{2}
        \begin{enumerate}[label=(\alph*)]
            \item $S \cup T = \{1, 2, 3, 4, 5\}$
            \item $S \cup U = \{1, 2, 3, (5, 5)\}$
            \item $S \cap T = \{2, 3\}$
            \item $T \cap U = \emptyset$
            \item $S \setminus T = \{1, 4\}$
            \item $S \setminus \{1, 4\} = T$
            \item $T^2 = U$
            \item $U \subset (S \cup T)^2$
        \end{enumerate}
    \end{multicols}
\end{exercise}
It is quite useful to have a notation that describes a set that is defined by a logical formula that evaluates to true for an element of the set, and false otherwise. In this form, set-builder\index{set!builder notation} notation has three parts: a variable, a vertical bar separator, and the logical formula. Thus there is a variable on the left of the separator, and a rule on the right of it. These three parts are contained in curly brackets:
\[
    \{x \ | \ \Phi(x)\}
\]
The vertical bar is a separator that can be read as ``such that'', ``for which'', or ``with the property that''. The formula $\Phi(x)$ is said to be the rule.

A domain $X$ can appear on the left of the vertical bar:
\[
    \{x \in X \ | \ \Phi(x)\},
\]
or by adjoining it to the logical formula:
\[
    \{x \ | \ x \in X,\; \Phi(x)\}.
\]
\begin{example}
    The set $S = \{x \in \mathbb{R} \ | \ x \geq 0 \}$ is the set of all non-negative real numbers.
\end{example}

The cardinality\index{cardinality} of a set $S$, denoted $|S|$, is the number of elements in $S$. The cardinality of the empty set is zero. The list of elements of some sets is endless, or infinite. In this book, we denote the cardinality of such infinite sets as $|S| = \infty$ (even though it is poorly defined in other contexts).

Finally, some notation that will be used:
\begin{itemize}
    \item $\mathbb{Z}$ denotes the set of integers\index{set!of integers} (from the German ``z\"ahlen", which means ``numbers");
    \item $\mathbb{Q}$ denotes the set of rational numbers\index{set!of rational numbers}; and
    \item $\mathbb{R}$ denotes the set of real numbers\index{set!of real numbers}.
\end{itemize}

\begin{exercise}
    Let the sets
    \begin{align*}
        S &= \{x \in \mathbb{Q} \vert x \leq 0\}\\
        T &= \{y \in \mathbb{Z} \vert -2 \leq y \leq 10 \text{ and } y \text{ is an even number} \}
    \end{align*}
    List the elements in the set $S \cap T$.
\end{exercise}

%=========================================
\chapter{Functions / Mappings}
A function\index{function} (or a map\index{map}) from a set $X$ to a set $Y$ is an assignment of an element of $Y$ to each element of $X$. The set $X$ is called the \textbf{domain}\index{domain} of the function and the set $Y$ is called the \textbf{codomain}\index{codomain} of the function.

A function, its domain, and its codomain, are declared by the notation $f: X\to Y$, and the value of a function $f$ at an element $x$ of $X$, denoted by $f(x)$, is called the \textbf{image}\index{function!image} of $x$ under $f$. The image or \textbf{range}\index{function!range} of a function is the set of the images of all elements in the domain. The image and the codomain are not necessarily the same. The image of the function $f: X \to Y$ is denoted either as $\im f$ or $f(X)$.

Two functions $f$ and $g$ are equal\index{function!equality} if their domain and codomain sets are the same and their output values agree on the whole domain. More formally, given $f: X \to Y$ and $g: X \to Y$, $f = g$ if $f(x) = g(x)$ for all $x$ in $X$.

\textbf{Arrow notation}\index{function!arrow notation} can also be used to define the rule of a function. For example, $f: \mathbb{R} \to \mathbb{R}, x \mapsto x+1$ is the function which takes a real number as input and outputs that number plus 1. In the above example, ``$x + 1$'' is the \textbf{rule}\index{function!rule} of the function.

\newpage

\begin{exercise}
    Let a function $f: \{1, 2, 3\} \to \{1, 4, 9, 16, 25\}$ be given by the relation $f(x) = x^2$.
    \begin{enumerate}[label=(\roman*)]
        \item Use arrow notation to write a definition for $f$.
        \item State the domain, codomain, and range of $f$.
        \item Is the function $g: \{1, 2, 3\} \to \{1, 8\}, x \mapsto x^3$ \textit{valid}?
    \end{enumerate}
\end{exercise}

A function is said to be \textbf{well-defined}\index{function!well-defined} if \textit{similar inputs} produce \textit{identical outputs}. More formally, a function $f: A \to B$ is well defined if for each $a \in A$ there is a unique $b \in B$ such that $f(a) = b$. An example may help to illustrate this point.
\begin{example}
    Let $S_1$ and $S_2$ be sets, and let $S = S_1 \cup S_2$. Let $f: S \to \{1, 2\}$, such that $f(x) = 1$ if $x \in S_1$ and $f(x) = 2$ if $x \in S_2$. Then $f$ is well-defined if $S_1 \cap S_2 = \emptyset$. For example, if $S_1 = \{1, 2\}$ and $S_2 = \{3, 4\}$, then $f$ is well-defined. On the other hand, if $S_1 \cap S_2 \neq \emptyset$, then $f$ is not well-defined. For example, if $S_1 = \{1, 2\}$ and $S_2 = \{2, 3\}$, then $f(2) = 1$ and $f(2) = 2$ simultaneously.
\end{example}
\begin{exercise}
    Is the function
    \[
        f: \mathbb{Q} \to \mathbb{Z}, \frac pq \mapsto p + q    
    \]
    well-defined?
\end{exercise}

Functions may also be \textbf{composed}\index{function!composition} with each other. This is done by using the function composition operator $\circ$. The composition of two functions, say $f: X \to Y$ and $g: Y \to Z$, produces a function $f \circ g: X \to Z$ such that $(f \circ g)(x) = f(g(x))$.

It is important to note the following about function composition.
\begin{itemize}
    \item Function composition is associative\index{function!composition!associative}. That is, if $f$, $g$, and $h$ are composable, then $f \circ (g \circ h) = (f \circ g) \circ h$. Since the parentheses do not change the result, they are generally omitted.
    \item The composition $f \circ g$ is only meaningful if the codomain of $g$ is a subset of the domain of $f$. That is, if $f: A \to B$ and $g: C \to D$, then $f \circ g$ is only meaningful if $\im g \subseteq A$.
\end{itemize}
We may also alternatively write $fg$ in place of $f \circ g$.

\begin{exercise}
    Let $f: \mathbb{R} \to \mathbb{R}$ and $g: \mathbb{R} \to \mathbb{R}$. Write down the rule of the function $fg$ if $f(x) = x^2 - x + 1$ and $g(y) = \frac1{y^2+1}$.
\end{exercise}

A function $f: X \to Y$ is said to be \textbf{injective}\index{function!injective} (or \textbf{one-to-one}\index{function!one-to-one}) if $f(x_1) = f(x_2)$ implies $x_1 = x_2$. Equivalently, if $x_1 \neq x_2$ then $f(x_1) \neq f(x_2)$ for all $x_1$ and $x_2$ in $X$.

A function $f: X \to Y$ is said to be \textbf{surjective}\index{function!surjective} (or \textbf{onto}\index{function!onto}) is a function whose image is equal to its codomain. Equivalently, $f$ is surjective if and only if for all $y \in Y$, there exists $x \in X$ such that $f(x) = y$. In this case, $x$ is said to be the \textbf{pre-image}\index{function!pre-image} of $y$.

A function is \textbf{bijective}\index{function!bijective} if it is both injective and surjective. A bijective function is also called a \textbf{bijection}\index{bijection} or a \textbf{one-to-one correspondence}\index{function!one-to-one!correspondence}. A function is bijective if and only if every possible image is mapped to by exactly one argument.

If a function $f: A \to B$ is bijective, then the sets $|A|$ and $|B|$ have the same cardinality, i.e., $|A| = |B|$. The sets $A$ and $B$ are said to be \textbf{equinumerous}\index{set!equinumerous} (i.e., `have the same number of elements') in this case. It should be noted that bijectivity is also implied if the function $f: X \to Y$ is injective and the sets $X$ and $Y$ are equinumerous.

\begin{exercise}
    Let the set $S = \{x \in \mathbb{Z} \vert x > 0\}$. Define the function $f: S \to \mathbb{Z}$ such that
    \[
        f(x) = \begin{cases}
            \frac{x}{2} & \text{ if } x \text{ is even}\\
            \frac{1-x}{2} & \text{ if } x \text{ is odd} 
        \end{cases}
    \]
    By considering $f$, prove that $|S| = |\mathbb{Z}|$.
\end{exercise}

%=========================================
\chapter{Mathematical Logic and Proof Writing}
\section{Mathematical Statements}
\begin{definition}
    A \textbf{(mathematical) statement}\index{statement} is a sentence that is definitely true or definitely false.
\end{definition}
\begin{remark}
    A statement can be written in English, or using mathematical notation.
\end{remark}
\begin{example}
    The sentence ``every square with length $x$ has area $x^2$'' is a true mathematical statement.
\end{example}
\begin{example}
    The sentence ``every circle with radius $x$ has area $x^2$'' is a false mathematical statement.
\end{example}
\begin{example}
    The sentence ``$12 \in \mathbb{Z}$'' is a true mathematical statement.
\end{example}
\begin{example}
    The sentence ``$\sqrt2 \in \mathbb{Z}$'' is a false mathematical statement.
\end{example}

We may name statements using variables like $P$, $Q$, $R$, etc.
\begin{example}
    If
    \begin{align*}
        P: &\ \text{Every odd number is one more than an even number}\\
        Q: &\ \text{Every triangle has sides of equal length}\\
        R: &\ \frac12 \in \mathbb{Q}
    \end{align*}
    then $P$ is true, $Q$ is false, and $R$ is true.
\end{example}

There are a few operations\index{logical operation} that may be carried out on mathematical statements. For the following, assume $P$ and $Q$ are mathematical statements.
\begin{itemize}
    \item \textbf{Logical AND}\index{logical operation!AND}: Uses the symbol $\land$. The statement $P\land Q$ is read as ``$P$ and $Q$''.
    \item \textbf{Logical OR}\index{logical operation!OR}: Uses the symbol $\lor$. The statement $P\lor Q$ is read as ``$P$ or $Q$''.
    \item \textbf{Logical NOT}\index{logical operation!NOT}: Uses the symbol $\lnot$. The statement $\lnot P$ is read as ``not $P$''.
    \item \textbf{Conditional}\index{logical operation!conditional}: Uses the symbol $\implies$. The statement $P \implies Q$ is read as ``$P$ implies $Q$''.  
\end{itemize}
\begin{example}
    Consider the statements
    \begin{align*}
        P: &\ \text{3 is an odd number}\\
        Q: &\ \text{4 is an odd number}
    \end{align*}
    Then
    \begin{itemize}
        \item $P\land Q$ is ``3 is an odd number \textbf{and} 4 is an odd number'', which is false.
        \item $P\lor Q$ is ``3 is an odd number \textbf{or} 4 is an odd number'', which is true.
        \item $\lnot Q$ is ``4 is \textbf{not} an odd number'', which is true.
    \end{itemize}
\end{example}
\begin{exercise}
    Let $P$ be the statement that ``1 is a positive number'', $Q$ be the statement ``$-1 > 0$'', and $R$ be the statement ``1 is an odd number''. Is the statement ``$\lnot((P\lor Q)\land R)$ is false'' true?
\end{exercise}

To explore the relationship between statements and operators, we use \textbf{truth tables}\index{truth table}. A truth table is like an operation table for statements. The idea is to list all possibilities of the truth or falsity of the statements $P$ and $Q$, and then write the truth for each of the combinations with operations.

For example, here's the truth table for the logical AND operator:
\begin{table}[h]
    \centering
    \begin{tabular}{|l|l||l|}
        \hline
        $P$ & $Q$ & $P\land Q$ \\ \hline
        F   & F   & F          \\ \hline
        F   & T   & F          \\ \hline
        T   & F   & F          \\ \hline
        T   & T   & T          \\ \hline
    \end{tabular}
\end{table}

(Note that we denote true statements by ``T'' and false statements by ``F'').

Here's the truth table for the logical OR operator:
\begin{table}[h]
    \centering
    \begin{tabular}{|l|l||l|}
        \hline
        $P$ & $Q$ & $P\lor Q$ \\ \hline
        F   & F   & F         \\ \hline
        F   & T   & T         \\ \hline
        T   & F   & T         \\ \hline
        T   & T   & T         \\ \hline
    \end{tabular}
\end{table}

And the truth table for the logical NOT operator:
\begin{table}[h]
    \centering
    \begin{tabular}{|l||l|}
        \hline
        $P$ & $\lnot P$ \\ \hline
        F   & F         \\ \hline
        T   & F         \\ \hline
    \end{tabular}
\end{table}

We motivate the truth table for the conditional by considering these two statements:
\begin{align*}
    P: &\ \text{You have a red card}\\
    Q: &\ \text{You have a green card}
\end{align*}
Then, $P \implies Q$ is the statement ``if you have a red card, then you have a green card''.
\begin{itemize}
    \item If $P$ and $Q$ are true, then that means that you have both a red and green card. Hence, ``if you have a red card, then you have a green card'' is \textbf{true}, meaning $P \implies Q$ is true.
    \item If $P$ is true and $Q$ is false, then that means that you have a red card but not a green card. Hence, ``if you have a red card, then you have a green card'' is \textbf{false}, meaning $P \implies Q$ is false.
    \item Now consider the case when $P$ is false. Then regardless of what $Q$ is, the initial premise of the statement ``if you have a red card'' is not satisfied. Hence the ``promise'' that ``if you have a red card, then you have a green card'' is \textbf{not untrue}, which means that it is \textbf{vacuously true}. Hence, if $P$ is false, then $P \implies Q$ is true.
\end{itemize}

In summary, the truth table for the conditional is:
\begin{table}[h]
    \centering
    \begin{tabular}{|l|l||l|}
        \hline
        $P$ & $Q$ & $P\implies Q$ \\ \hline
        F   & F   & T             \\ \hline
        F   & T   & T             \\ \hline
        T   & F   & F             \\ \hline
        T   & T   & T             \\ \hline
    \end{tabular}
\end{table}

\begin{exercise}
    Let $P$ and $Q$ be statements. Draw the truth table for $P \land (\lnot Q)$.
\end{exercise}

We end this section by introducing the idea of the \textbf{biconditional}\index{logical operation!biconditional}.
\begin{definition}
    Let $P$ and $Q$ be mathematical statements. If both $(P \implies Q)$ and $(Q \implies P)$ are true, then we write $(P \iff Q)$. In other words, $(P \iff Q) = ((P \implies Q) \land (Q \implies P))$.
\end{definition}
\begin{remark}
    The statement $(P \iff Q)$ can be written in several ways in English:
    \begin{itemize}
        \item $P$ if and only if $Q$;
        \item $P$ is a necessary and sufficient condition for $Q$; or
        \item $P$ is equivalent to $Q$.
    \end{itemize}
\end{remark}

The truth table for the biconditional is:
\begin{table}[h]
    \centering
    \begin{tabular}{|l|l||l|}
        \hline
        $P$ & $Q$ & $P\iff Q$ \\ \hline
        F   & F   & T         \\ \hline
        F   & T   & F         \\ \hline
        T   & F   & F         \\ \hline
        T   & T   & T         \\ \hline
    \end{tabular}
\end{table}

\begin{exercise}
    Suppose $n$ is an integer. Let $P$ be the statement ``$n$ is a multiple of 5'' and $Q$ be the statement ``the last digit of $n$ is 0 or 5''. Let the statement $R = (P \iff Q)$.
    \begin{enumerate}[label=(\roman*)]
        \item Write the statement $R$ in English.
        \item Is the statement $R$ true? Justify your answer.
    \end{enumerate}
\end{exercise}

\section{Properties of Logical Operators}
A central idea of this section is that every mathematical statement can be built from just $\land$, $\lor$, and $\lnot$.

\begin{example}
    We show that $(P \implies Q) = (\lnot P) \lor Q$ by considering the truth table $(\lnot P) \lor Q$.
    \begin{table}[h]
        \centering
        \begin{tabular}{|l|l||l||l|}
            \hline
            $P$ & $Q$ & $\lnot P$ & $(\lnot P) \lor Q$ \\ \hline
            F   & F   & T         & T                  \\ \hline
            F   & T   & T         & T                  \\ \hline
            T   & F   & F         & F                  \\ \hline
            T   & T   & F         & T                  \\ \hline
        \end{tabular}
    \end{table}
    
    By inspection, we see that $(\lnot P) \lor Q$ has the same truth table as $P \implies Q$. Thus $(P \implies Q) = (\lnot P) \lor Q$.
\end{example}
\begin{remark}
    We separate the intermediate value(s) (e.g. $\lnot P$ in the above example) from the rest by drawing a double line to the sides of the intermediate values.
\end{remark}

\begin{example}
    We show that $(P \iff Q) = (P \land Q) \lor ((\lnot P) \land (\lnot Q))$. For brevity, let $R = (\lnot P) \land (\lnot Q)$.
    \begin{table}[h]
        \centering
        \begin{tabular}{|l|l||l|l|l|l||l|}
            \hline
            $P$ & $Q$ & $\lnot P$ & $\lnot Q$ & $P \land Q$ & $R$ & $(P \land Q) \lor R$ \\ \hline
            F   & F   & T         & T         & F           & T   & T                    \\ \hline
            F   & T   & T         & F         & F           & F   & F                    \\ \hline
            T   & F   & F         & T         & F           & F   & F                    \\ \hline
            T   & T   & F         & F         & T           & F   & T                    \\ \hline
        \end{tabular}
    \end{table}

    By inspection of the truth table we establish the required result.
\end{example}

\begin{exercise}
    Show that
    \[
        ((\lnot P) \iff Q) = (P \implies (\lnot Q)) \land ((\lnot Q) \implies P)
    \]
    by drawing a truth table.
\end{exercise}

We note some important properties of logical operators.
\begin{itemize}
    \item \textbf{Contrapositive}\index{contrapositive}: $(P \implies Q) = ((\lnot Q) \implies (\lnot P))$
    \item \textbf{De Morgan's Laws}: \begin{itemize}
        \item $(\lnot (P \land Q)) = ((\lnot P) \lor (\lnot Q))$
        \item $(\lnot (P \lor Q)) = ((\lnot P) \land (\lnot Q))$
    \end{itemize}
    \item \textbf{Commutativity of AND and OR}: \begin{itemize}
        \item $P \land Q = Q \land P$
        \item $P \lor Q = Q \lor P$
    \end{itemize}
    \item \textbf{Associativity of AND and OR}: \begin{itemize}
        \item $P \land (Q \land R) = (P \land Q) \land R$
        \item $P \lor (Q \lor R) = (P \lor Q) \lor R$
    \end{itemize}
    \item \textbf{Distributive Rules}: \begin{itemize}
        \item $P \land (Q \lor R) = (P \land Q) \lor (P \land R)$
        \item $P \lor (Q \land R) = (P \lor Q) \land (P \lor R)$
    \end{itemize}
\end{itemize}
\begin{remark}
    The most important one of these properties would arguably be the contrapositive. We will use this result several times later and in later volumes.
\end{remark}
\begin{exercise}
    Simplify the statement
    \[
        ((P \lor \lnot Q) \land \lnot R) \lor ((P \lor \lnot Q) \land (P \lor R) \land (P \lor \lnot R))
    \]
    into a statement that uses only \textbf{three} operators in total.
\end{exercise}

\section{Quantifiers}
\begin{definition}
    The \textbf{universal quantifier}\index{quantifier!universal} is $\forall$ and is read as ``for all''.
\end{definition}
\begin{example}
    The statement ``for every integer $n$, the integer $2n$ is an even number'' can be written as ``$\forall n \in \mathbb{Z}, 2n$ is even''.
\end{example}

\begin{definition}
    The \textbf{existential quantifier}\index{quantifier!existential} is $\exists$ and is  read as ``there exists''.
\end{definition}
\begin{example}
    The statement ``there exists a subset of $\mathbb{Z}$ which has 10 elements'' can be written as ``$\exists S \subseteq \mathbb{Z}, |S| = 10$''.
\end{example}

We can also combine the quantifiers together with logical operators.
\begin{example}
    The statement ``every odd integer is one less than an even integer'' can be written as ``$\forall n \in \mathbb{Z}$, $\exists k \in \mathbb{Z}$, $n = 2k - 1$''.
\end{example}
\begin{example}
    Let $S$ denote the set of positive integers. Fermat's Last Theorem, which states that for all integers $n\geq 3$ the equation $x^n + y^n = z^n$ has no solution for $x, y, z \in S$, can be written as
    \[
        \left[(n \in \mathbb{Z}) \land (n \geq 3)\right] \implies \left[\forall x, y, z \in S, x^n + y^n \neq z^n\right].
    \]
    It should be noted that when translating from English to \textit{symbolic writing}, we may need to change some things around.
\end{example}

\begin{exercise}
    Let $S$ denote the set of positive integers. Convert the statement
    \begin{quote}
        for all positive integers $n > 2$ there exists integers $a$ and $b$ such that $a^3 + b^4 = n^5$
    \end{quote}
    into symbolic notation using quantifiers and logical operators.
\end{exercise}

We now look at negating quantifiers\index{quantifier!negation}. Note that
\begin{itemize}
    \item $\lnot(\forall x, P) = (\exists x, \lnot P)$; and
    \item $\lnot(\exists x, P) = (\forall x, \lnot P)$.
\end{itemize}
\begin{example}
    Consider the statement ``for every integer $n$, the integer $2n$ is even''. One may write that using quantifiers as
    \[
        \forall n \in\mathbb{Z}, 2n \text{ is even}.
    \]
    Negating the above statement would make it
    \[
        \exists n \in \mathbb{Z}, 2n \text{ is not even},
    \]
    i.e., ``there exists an integer $n$ such that $2n$ is odd''.
\end{example}

We also note that $\lnot(P \implies Q) = P \land \lnot Q$. We leave verifying this identity as an exercise to the reader.

\begin{example}
    Consider the statement ``if $n$ is odd then $n^3$ is odd''. We may write this using logical operators as ``($n$ is odd) $\implies$ ($n^3$ is odd)''. The negation of such a statement would hence be ``$n$ is odd \textbf{and} $n^3$ is \textbf{not} odd'', i.e. ``$n$ is odd and $n^3$ is even''.
\end{example}

\newpage

\begin{exercise}
    Consider the statement ``if $x$ is a non-zero real number, then there exists a real number $y$ such that $xy = 1$''.
    \begin{enumerate}[label=(\roman*)]
        \item Write down two statements $P$ and $Q$ in symbols such that the above statement is $P \implies Q$.
        \item Negate the above statement, writing your answer in symbolic notation.
    \end{enumerate}
\end{exercise}

\section{Hierarchy of Mathematical Results}
Mathematical statements often have a certain `tier' attached to them. We look at the hierarchy of some of these `tiers'.
\begin{itemize}
    \item \textbf{Proposition}\index{proposition}: A proposition can be thought of as a general proven result. Equivalent names for a proposition are ``\textbf{Claim}'' or ``\textbf{Observation}''.
    \item \textbf{Lemma}\index{lemma}: A lemma can be thought of as a `small' proven result that can help build other mathematical results. For example, Euclid's lemma that ``if a prime $p$ divides the product $ab$ of two integers $a$ and $b$, then $p$ must divide at least one of those integers $a$ or $b$'' is used in the proof of the Fundamental Theorem of Arithmetic.
    \item \textbf{Theorem}\index{theorem}: A theorem can be thought of as a `big' proven result. For example, the Fundamental Theorem of Arithmetic is core to arithmetic as it describes how integers can be uniquely decomposed into its prime factors.
    \item \textbf{Corollary}\index{corollary}: A corollary is said to be a `follow-up result' from a theorem. These results usually follow `very quickly' from a theorem or a proposition. For example, the AM-GM inequality is a corollary of the Pythagorean theorem.
    \item \textbf{Conjecture}\index{conjecture}: A conjecture is a mathematical statement whose truth or falsity is not known.
\end{itemize}

\section{Mathematical Proof Techniques}
\subsection{Direct Proof}
In essence, a direct proof\index{proof!direct} for the statement ``if $P$ then $Q$'' would involve
\begin{enumerate}
    \item supposing $P$ is true, (usually written as ``suppose $P$'')
    \item unpacking $P$ via definitions; make appropriate calculations and logical arguments; repack $Q$ via definition,
    \item concluding that $Q$ is true. (usually written as ``thus $Q$'')
\end{enumerate}
\begin{example}
    We look at a proof of the statement ``if the integer $n$ is odd then $n^2$ is odd''.
    \begin{proof}
        Suppose $n\in\mathbb{Z}$ is an odd number.
        
        Then $n$ can be written in the form $n = 2k + 1$ where $k$ is an integer. Hence $n^2 = (2k+1)^2 = 4k^2 + 4k + 1$. Note $4k^2 + 4k + 1 = 2(2k^2 + 2k) + 1$, so $n^2$ is one more than a multiple of 2.
        
        Thus $n^2 = 2(2k^2 + 2k) + 1$ is odd.
    \end{proof}
\end{example}

\begin{example}
    We look at a proof of the statement ``if $n$ is an integer then $1 + (-1)^n(2n-1)$ is a multiple of 4''.
    \begin{proof}
        Suppose $n$ is an integer. Then, $n$ is either odd or even. We look at two cases.
        \begin{itemize}
            \item When $n$ is odd, $(-1)^n = -1$ and $n = 2k+1$ for some integer $k$. Thus
            \[
                1 + (-1)^n(2n-1) = 1 - (2(2k+1)-1) = -4k
            \]
            which is a multiple of 4.
            \item When $n$ is even, $(-1)^n = 1$ and $n = 2k$ for some integer $k$. Thus
            \[
                1 + (-1)^n(2n-1) = 1 + (2(2k)-1) = 4k
            \]
            which is a multiple of 4.
        \end{itemize}
        Hence, in both cases, $1 + (-1)^n(2n-1)$ is a multiple of 4.
    \end{proof}
\end{example}

\begin{exercise}
    Prove that $m + n$ is even if the integers $m$ and $n$ have the same parity (i.e., both odd or both even). 
\end{exercise}

\subsection{Contrapositive Proof}
We now look at a \textbf{contrapositive proof}\index{proof!contrapositive}. Recall that $(P \implies Q) = (\lnot Q \implies \lnot P)$. Hence, a contrapositive proof for the statement ``if $P$ then $Q$'' would involve
\begin{enumerate}
    \item supposing $\lnot Q$ is true,
    \item working towards showing that $\lnot P$ is true,
    \item concluding that $\lnot P$ is true.
\end{enumerate}

Generally, we would want to prove in the direction from simplicity to complexity. So if $P$ is more complex than $Q$, we may consider using a contrapositive proof.

\begin{example}\label{example-if-(n-1)(n-5)-is-even-then-n-is-odd}
    Suppose $n$ is an integer. We prove the statement ``if $n^2 - 6n + 5$ is even, then $n$ is odd''. We note that a direct proof would be tedious and problematic. Using a contrapositive proof would be easier.
    
    We first note that the contrapositive statement that we want to prove is ``if $n$ is \textbf{not} odd, then $n^2 - 6n + 5$ is \textbf{not} even'', that is, ``if $n$ is even, then $n^2 - 6n + 5$ is odd''.
    \begin{proof}
        We consider a proof by contrapositive.
        
        Suppose $n$ is even. Then $n = 2k$ where $k$ is an integer. Note
        \begin{align*}
            n^2 - 6n + 5 &= (2k)^2 - 6(2k) + 5\\
            &= 4k^2 - 12k + 5\\
            &= (4k^2 - 12k + 4) + 1\\
            &= 2(2k^2 - 6k + 2) + 1
        \end{align*}
        which means that $n^2 - 6n + 5$ is one more than a multiple of 2, which hence means $n^2 - 6n + 5$ is odd.
    \end{proof}    
\end{example}

\begin{example}
    Suppose $x$ and $y$ are real numbers. We prove the statement ``$x \leq y$ if $x^3 + xy^2 \leq x^2y + y^3$'' using a contrapositive proof.
    
    We first note that the contrapositive statement that we want to prove is ``if $x > y$ then $x^3 + xy^2 > x^2y + y^3$''.

    \newpage

    \begin{proof}
        We consider a proof by contrapositive.
        
        Assume $x > y$. Then $x - y > 0$. Also, since $x > y$, thus $x$ and $y$ are not both zero. Hence $x^2 + y^2 > 0$.
        Observe
        \[
            (x-y)(x^2+y^2) > 0 \times (x^2+y^2) = 0        
        \]
        so $(x-y)(x^2+y^2) = x^3 + xy^2 - x^2y - y^3 > 0$. Therefore $x^3 + xy^2 > x^2y + y^3$.
    \end{proof}
\end{example}

\begin{exercise}
    Suppose that $a$ and $b$ are integers. Prove that $a$ is even or $b$ is odd if $a(b^2 + 5)$ is even.
\end{exercise}

\subsection{Proof by Contradiction}
The third proof technique is called a \textbf{proof by contradiction}\index{proof!contradiction}. This method can be used to prove any kind of statement. The basic idea is to assume that the statement we want to prove is false, and then show that this assumption leads to a contradiction. A proof by contradiction for the statement ``$P$'' (yes, just $P$) would involve
\begin{enumerate}
    \item supposing $\lnot P$ is true,
    \item working towards forming a statement $C$ such that another statement $C \land \lnot C$ is formed,
    \item observing that $C \land \lnot C$ is impossible (since $C \land \lnot C$ is always false),
    \item concluding that $P$ is true.
\end{enumerate}
Usually, when writing a proof by contradiction, we would like to inform the reader that a proof by contradiction is being employed. Language such as ``by way of contradiction'', ``towards a contradiction'', ``suppose for the sake of contradiction'' etc. may be used to signpost the use of a proof by contradiction.
\begin{remark}
    Some authors would also signal the use of contradiction by using the initialism ``BWOC'' (by way of contradiction). 
\end{remark}

\begin{example}\label{example-sqrt2-is-irrational}
    We prove the classic result that ``$\sqrt 2$ is irrational'' via a proof by contradiction.
    \begin{proof}
        By way of contradiction, assume that $\sqrt2 = \frac ab$ for some integers $a$ and $b$. Furthermore let this fraction be fully reduced; in particular, this means that $a$ and $b$ are not both even. Squaring both sides yields $2 = \frac{a^2}{b^2}$, meaning  $a^2 = 2b^2$. Hence $a^2$ is even, so write $a = 2c$ where $c$ is an integer. This leads to $2b^2 = (2c)^2 = 4c^2$ which implies $b^2 = 2c^2$. Hence $b$ is even, which contradicts the fact that $a$ and $b$ are not both even.
        
        Hence, $\sqrt 2$ is irrational.
    \end{proof}
\end{example}
\begin{remark}
    It is not necessary to have the final statement that ``$\sqrt 2$ is irrational'' (or, more generally, ``$P$ is true'') as it is implied from the proof by contradiction.
\end{remark}

\begin{example}
    We prove the statement that ``for every positive rational number $x$, there exists a positive rational number $y$ such that $y < x$'' by way of contradiction.
    
    We note that the negation of the above statement is ``there exists a rational number $x$ such that for every positive rational number $y$ we have $y \geq x$''.
    \begin{proof}
        Suppose for the sake of contradiction that there exists a rational number $x$ such that for every positive rational number $y$ we have $y \geq x$. Write $x = \frac pq$ where $p$ and $q$ are positive integers.
        
        Now consider the rational number $\frac{p-1}{q}$. Clearly $\frac{p-1}{q} < \frac pq = x$. By assumption, every positive rational number $y$ satisfies $y \geq x$. Hence, $\frac{p-1}{q}$ is non-positive, meaning $\frac{p-1}{q} \leq 0$. Since $q$ is positive, hence $p - 1 \leq 0$ which means $p \leq 1$. But as $p$ is a positive integer, we conclude $p = 1$. Hence $x = \frac 1q$.
        
        We now consider the rational number $\frac{1}{q+1}$. Clearly $\frac{1}{q+1} < \frac{1}{q} = x$. By assumption we must conclude that $\frac{1}{q+1}$ is non-positive. However, $1 > 0$ and $q + 1 > 0$, so $\frac{1}{q+1}$ is positive. Hence we have the fact that $\frac{1}{q+1}$ is positive and non-positive simultaneously, leading to a contradiction.
    \end{proof}
\end{example}
\begin{remark}
    The statement above is one where a direct proof would be easier. We provide a direct proof of it below.
    \begin{proof}
        Since $x$ is a positive rational number write $x = \frac pq$ where $p$ and $q$ are positive integers. Then set $y = \frac{p}{q+1}$. Clearly $\frac{p}{q+1} < \frac{p}{q} = x$ and $\frac{p}{q+1}$ is positive, hence we have found a $y$ such that $y < x$.
    \end{proof}
    However, we use this statement as an example for how a proof by contradiction can be constructed.
\end{remark}

\begin{exercise}
    Prove that there exist no integers $a$ and $b$ such that $2a + 4b = 1$.
\end{exercise}

\newpage

We now look at a proof by contradiction for conditional statements. Recall that $(\lnot(P \implies Q)) = (P \land \lnot Q)$. Hence, to prove the statement ``if $P$ then $Q$'', we would
\begin{enumerate}
    \item suppose that $P \land \lnot Q$ is true,
    \item work towards a statement $C$ such that another statement $C \land \lnot C$ is formed,
    \item observe that $C \land \lnot C$ is impossible,
    \item conclude that $P \implies Q$.
\end{enumerate}

\begin{example}
    Suppose $a$ and $b$ are real numbers. We prove the statement ``if $a$ is rational and $ab$ is irrational then $b$ is irrational'' using a proof by contradiction.
    
    We note that the statement we want to contradict is ``$a$ is rational and $ab$ is irrational \textbf{and} $b$ is \textbf{not} irrational'', i.e. ``$a$ is rational and $ab$ is irrational and $b$ is rational''.
    \begin{proof}
        By way of contradiction assume $a$ is rational and $ab$ is irrational and $b$ is rational. We may then write $a = \frac mn$ and $b = \frac pq$ where $m, n, p, q \in \mathbb{Z}$. Hence $ab = \left(\frac mn\right)\left(\frac pq\right) = \frac{mp}{nq}$ which is clearly rational. Therefore we have that $ab$ is irrational (by assumption) and $ab$ is rational, a contradiction.
    \end{proof}
\end{example}

\begin{example}
    Suppose $a$, $b$, and $c$ are integers. We prove the statement that ``if $a^2 + b^2 = c^2$ then at least one of $a$ or $b$ is even'' using a proof by contradiction.
    
    We note that the statement we want to contradict is ``$a^2 + b^2 = c^2$ \textbf{and not} (at least one of $a$ or $b$ is even)'', i.e. ``$a^2 + b^2 = c^2$ \textbf{and} both $a$ and $b$ are odd''.
    \begin{proof}
        Seeking a contradiction, assume that $a^2 + b^2 = c^2$ and both $a$ and $b$ are odd. Thus we may write $a = 2m + 1$ and $b = 2n + 1$ where $m$ and $n$ are integers. Hence
        \begin{align*}
            a^2 + b^2 &= (2m+1)^2 + (2n+1)^2\\
            &= (4m^2+4m+1) + (4n^2+4n+1)\\
            &= 4m^2 + 4n^2 + 4m + 4n + 2\\
            &= 2(2m^2 + 2n^2 + 2m + 2n +1)
        \end{align*}
        which means that $c^2 = a^2 + b^2$ is even. Hence $c$ is even, which means we may write $c = 2k$ where $k$ is an integer. This leads to
        \[
            c^2 = 4k^2 = 2(2m^2 + 2n^2 + 2m + 2n + 1) = a^2 + b^2.
        \]
        Clearly $4k^2$ is a multiple of 4, while $2(2m^2 + 2n^2 + 2m + 2n + 1)$ is not. Yet, they are equal to each other, a contradiction.
    \end{proof}
\end{example}

\begin{exercise}
    Prove that $\frac{a+b}{2} \geq \sqrt{ab}$ if $a$ and $b$ are positive real numbers by way of contradiction.
\end{exercise}

Despite the power of proof by contradiction, it's best to use it only when the direct and contrapositive approaches do not seem to work.
\begin{example}
    Suppose $n$ is an integer. We prove the statement ``if $n^2 - 6n + 5$ is even then $n$ is odd'' using a proof by contradiction.
    \begin{proof}
        Working towards a contradiction, assume $n^2 - 6n + 5$ is even and $n$ is \textbf{not} odd, i.e. $n$ is even. Then $n = 2k$ for some integer $k$. Note that
        \begin{align*}
            n^2 - 6n + 5 &= (2k)^2 - 6(2k) + 5\\
            &= 4k^2 - 12k + 5\\
            &= (4k^2 - 12k + 4) + 1\\
            &= 2(2k^2 - 5k + 2) + 1
        \end{align*}
        which means that $n^2 - 6n + 5$ is odd. Hence, $n^2 - 6n + 5$ is even (by assumption) and $n^2 - 6n + 5$ is odd (as above), a contradiction.
    \end{proof}
    While there is nothing wrong with this proof, notice that part of it assumes that $n$ is even and concludes that  $n^2 - 6n + 5$ is odd, which is the contrapositive approach done in \myref{example-if-(n-1)(n-5)-is-even-then-n-is-odd}.
\end{example}

\subsection{Proof by Mathematical Induction}
Mathematical induction\index{proof!induction} is a method for proving that a proposition $P_n$ is true for every positive integer $n$, that is, that the infinitely many cases $P_1, P_2, P_3, \dots,$ all hold. Informal metaphors help to explain this technique, such as falling dominoes or climbing a ladder:
\begin{quote}
    Mathematical induction proves that we can climb as high as we like on a ladder, by proving that we can climb onto the bottom rung (the base case) and that from each rung we can climb up to the next one (the induction step).
\end{quote}

A proof by induction consists of two steps. The first, the \textbf{base case}\index{proof!induction!base case}, proves the statement for $n = 1$ without assuming any knowledge of other cases. The second, the \textbf{induction step}\index{proof!induction!induction step}, proves that if the statement holds for any given case $n = k$, then it must also hold for the next case $n = k + 1$. These two steps establish that the statement holds for every positive integer $n$.

The base case does not necessarily need to begin with $n = 1$. Sometimes we may begin with $n = 0$, and possibly with any fixed natural number $n = N$, establishing the truth of the statement for all natural numbers $n \geq N$.

In summary, mathematical induction involves two steps:
\begin{itemize}
    \item \textbf{Base Case}: Prove the statement for the initial value.
    \item \textbf{Induction Step}: Prove that for every $n$, if the statement holds for $n$, then it holds for $n + 1$.
\end{itemize}

\begin{example}
    We prove the famous identity
    \[
        1 + 2 + 3 + \cdots + n = \frac{n(n+1)}2
    \]
    using mathematical induction.
    \begin{proof}
        When $n = 1$, the left hand side is 1; the right hand side is $\frac{1(1+1)}{2} = 1$. Thus the initial case is true.

        Now assume that the statement holds for some positive integer $k$, meaning
        \[
            1 + \cdots + k = \frac{k(k+1)}2.
        \]
        We are to prove the statement true for $k+1$, meaning
        \[
            1 + \cdots + k + (k+1) = \frac{(k+1)(k+2)}2.
        \]
        We work slowly:
        \begin{align*}
            1 + \cdots + k + (k+1) &= \frac{k(k+1)}{2} + (k+1) & (\text{by hypothesis})\\
            &= \frac{k(k+1)}2 + \frac{2(k+1)}{2}\\
            &= \frac{k(k+1) + 2(k+1)}2\\
            &= \frac{(k+1)(k+2)}2
        \end{align*}
        which proves the case for $k + 1$. Hence $1 + 2 + 3 + \cdots + n = \frac{n(n+1)}2$.
    \end{proof}
\end{example}

\begin{example}
    Suppose $x > -1$. We will prove that $(1+x)^n \geq 1+nx$ if $n$ is a positive integer.
    \begin{proof}
        When $n = 1$, the left hand side is $(1+x)^1 = 1+x$ which is exactly the right hand side. Thus the base case is true.
        
        Assume that the statement holds for some positive integer $k$, i.e. $(1+x)^k \geq 1+kx$. We show that the statement holds for $k+1$, i.e. $(1+x)^{k+1} \geq 1+(k+1)x$.
        
        We first note that since $x>-1$, thus $1+x > 0$. We start with our induction hypothesis.
        \begin{align*}
            (1+x)^k &\geq 1+kx\\
            (1+x)^k(1+x) &\geq (1+kx)(1+x) & (\text{since }1+x > 0)\\
            (1+x)^{k+1} &\geq 1 + x + kx + kx^2\\
            &= 1+(k+1)x + kx^2\\
            &> 1+(k+1)x
        \end{align*}
        Hence we see $(1+x)^{k+1} \geq 1+(k+1)x$, meaning that the statement is true for $k+1$.
        
        Therefore by mathematical induction we have $(1+x)^n \geq 1+nx$ if $n$ is a positive integer.
    \end{proof}
\end{example}

\begin{exercise}
    Prove by induction that $a^2 - 1$ is a multiple of 8 for all positive odd integers $a$.
\end{exercise}

We now look at another form of mathematical induction, called \textbf{strong induction}\index{proof!induction!strong}. Unlike regular induction, strong induction assumes that all preceding cases are true, and proves the next case.

Strong mathematical induction involves two steps:
\begin{itemize}
    \item \textbf{Base Cases}: Prove the statement for the initial values.
    \item \textbf{Induction Step}: Prove that for every $n$, if the statement holds for all (positive) integers $m$ that are at most $n$, then it holds for $n + 1$.
\end{itemize}

\begin{example}
    We prove that every integer $n \geq 8$ can be expressed in the form $3a + 5b$ where $a$ and $b$ are non-negative integers.
    \begin{proof}
        We use strong induction on $n$.
        
        We show the base cases of 8, 9, and 10 hold:
        \begin{itemize}
            \item When $n = 8$, note $8 = 3 + 5$.
            \item When $n = 9$, note $9 = 3 \times 3 + 5 \times 0$.
            \item When $n = 10$, note $10 = 3 \times 0 + 5 \times 2$.
        \end{itemize}
        
        Now assume that for some positive integer $k \geq 8$, for every integer $m$ satisfying $8 \leq m \leq k$ the statement holds true, i.e. $m$ can be written in the form $3a + 5b$. We are to show that the statement for $k+1$ is true, i.e. $k+1$ can be expressed in the form $3a + 5b$.
        
        By hypothesis, $k - 2$ can be expressed in the form $3a+5b$. Hence $k+1 = (k-2) + 3 = 3(a+1) + 5b$, proving the statement for $k+1$.
        
        Therefore by mathematical induction, every integer $n \geq 8$ can be expressed in the form $3a + 5b$ where $a$ and $b$ are non-negative integers.
    \end{proof}
\end{example}

\begin{example}\label{example-strong-induction-on-function}
    Let $\mathbb{Z}^+$ denote the set of positive integers, and consider the function $f: \left(\mathbb{Z}^+\right)^2\to\mathbb{Z}^+$ where
    \[
        f(m, n) =
        \begin{cases}
            n & \text{if } m = 1, \\
            m & \text{if } n = 1, \\
            f\left(n-1,f(n-1,m-1)\right) & \text{otherwise.}
        \end{cases}
    \]
    We will prove the non-obvious fact that $f(n+1, n) = 2$ for all positive integers $n$.
    \begin{proof}
        We show the base cases of 1 and 2 hold:
        \begin{itemize}
            \item When $n = 1$, we have $f(2, 1) = 2$, so the first case is true.
            \item When $n = 2$, we have $f(3, 2)$. Note that
            \[
                f(3,2) = f(1, f(1, 2)) = f(1, 2) = 2
            \]
            so the second case is true.
        \end{itemize}
        
        Now suppose for some positive integer $k$, for every integer $1 \leq m \leq k$ the statement holds true, i.e. $f(m+1,m) = 2$. We want to show that the case for $k+1$ is true, i.e. $f(k+2, k+1) = 2$.
        \begin{align*}
            f(k+2, k+1) &= f(k, f(k, k+1))\\
            &= f(k, f(k, f(k, k-1)))\\
            &= f(k, f(k, 2)) & (\text{hypothesis on } k-1)\\
            &= f(k, f(1, f(1, k-1)))\\
            &= f(k, f(1, k-1))\\
            &= f(k, k-1) \\
            &= 2 & (\text{hypothesis on } k-1)
        \end{align*}
        which proves that the statement for $k+1$ holds. Hence by mathematical induction, $f(n+1, n) = 2$.
    \end{proof}
\end{example}

\begin{exercise}
    Let $\mathbb{Z}^+$ denote the set of positive integers. Let the function $f: \mathbb{Z}^+ \to \mathbb{Z}$ be defined such that $f(1) = 0$, $f(2) = 1$, and $f(n+2) = 3f(n+1) - 2f(n) + 1$ for all positive integers $n$. Prove that $f(n) = 2^n - n - 1$ for all positive integers $n$.
\end{exercise}

\section{Proving Non-Conditional Statements}
\subsection{Biconditional Statements}
Recall that a biconditional statement is a statement like ``$P \iff Q$'', i.e., ``$P$ if and only if $Q$''. We prove such a statement by proving\index{proof!biconditional} two things:
\begin{itemize}
    \item $P \implies Q$; and
    \item $Q \implies P$.
\end{itemize}
Each of these statements may be proved using any of the proof techniques that we covered.

\begin{example}
    We will prove the biconditional statement ``the integer $n$ is even if and only if $n^2$ is even''.
    \begin{proof}
        We prove the forward direction ($n$ is even implies $n^2$ is even) first by using direct proof. Assume that $n$ is even. Then we may write $n = 2k$ where $k$ is an integer. Hence $n^2 = (2k)^2 = 4k^2 = 2(2k^2)$ which is even.

        We now prove the reverse direction ($n^2$ is even implies $n$ is even) via a proof by contrapositive. Suppose $n$ is \textbf{not} even, meaning $n$ is odd. Hence $n = 2k + 1$ where $k$ is an integer. Observe $n^2 = (2k+1)^2 = 4k^2 + 4k + 1 = 2(2k^2 + 2k) + 1$ which is odd.
    \end{proof}
\end{example}
\begin{example}
    Suppose $n$ is an integer. We will prove ``$n$ is a multiple of 6 if and only if $n$ is a multiple of 2 and 3''.
    \begin{proof}
        We prove the forward direction first by using direct proof. Assume $n$ is a multiple of 6, meaning $n = 6k$ for some integer $k$. Clearly $6k = 2(3k)$ and $6k = 3(2k)$, so $n$ is both a multiple of 2 and 3.
        
        We now prove the reverse direction, again using direct proof. Assume $n$ is a multiple of 2 and 3, so we may write $n = 2a$ and $n = 3b$ for some integers $a$ and $b$. Then $2a = 3b$. Hence $a = \frac 32 b$ and $b = \frac 23 a$. Since $a$ and $b$ are integers, hence we conclude $b$ is a multiple of 2 and $a$ is a multiple of 3. Write $a = 3p$ and $b = 2q$ where $p$ and $q$ are integers. Hence $n = 2(3p) = 6p$ and $n = 3(2q) = 6q$. In both cases we see $n$ is a multiple of 6.
    \end{proof}
\end{example}

\begin{exercise}
    Let $n$ be an integer. Prove that $n$ is one more than a multiple of 5 if and only if $n$ is of the form $5k - 4$ where $k$ is an integer.
\end{exercise}

\subsection{Existence Statements}
Some statements only assert the existence of something. These statements are called \textbf{existence statements} and one only has to provide a particular example that shows it is true.\index{proof!existence proof} 
\begin{example}
    The statement ``there exists an even prime number'' is readily proven by noticing that 2 is an even prime number.
\end{example}
\begin{example}
    The statement ``an integer that can be expressed as the sum of two perfect cubes in two different ways exists'' is proven by giving the example 1729:
    \begin{itemize}
        \item $1729 = 1^3 + 12^3$; and
        \item $1729 = 9^3 + 10^3$.
    \end{itemize}
\end{example}
Note that while an example suffices to prove an existence statement, a single example does not prove a conditional statement.
\begin{exercise}
    Prove that a positive integer that is one less than a perfect cube and two less than a perfect square exists.
\end{exercise}

\newpage

Existence proofs fall into two categories: \textbf{constructive}\index{proof!constructive} proofs and \textbf{non-constructive}\index{proof!non-constructive} proofs.
\begin{itemize}
    \item Constructive proofs provide an explicit example that proves the statement. We have only seen constructive proofs so far.
    \item Non-constructive proofs prove that an example exists without providing it.
\end{itemize}

\begin{example}
    We prove the classic statement that ``there exist irrational $x$ and $y$ such that $x^y$ is rational'' using a non-constructive proof.
    \begin{proof}
        Let $x = \sqrt2^{\sqrt2}$ and $y = \sqrt2$. We know that $\sqrt2$ is irrational from \myref{example-sqrt2-is-irrational}. Now consider two cases.
        \begin{itemize}
            \item If $x$ is rational, then we have found two irrational numbers (in particular, $\sqrt 2$ and $\sqrt 2$) such that their exponentiation (i.e.,  $x = \sqrt2^{\sqrt2}$) is rational, proving the claim.
            \item If $x$ is irrational, then \[x^y = \left(\sqrt2^{\sqrt2}\right)^{\sqrt2} = (\sqrt2)^{\sqrt2 \times \sqrt2} = (\sqrt2)^2 = 2\]
            is rational.
        \end{itemize}
        Hence, either way, we have an irrational number to an irrational power that is rational.
    \end{proof}
    Notice that we did not explicitly prove whether $\sqrt2^{\sqrt2}$ is rational or irrational; we just showed that either case leads to a case where two irrational numbers, when exponentiated, results in a rational number.
\end{example}

\begin{exercise}
    Let $x = \sqrt2$ and $y = 2\log_2{3}$. It may be assumed that $\sqrt2$ is irrational.
    \begin{enumerate}[label=(\roman*)]
        \item Prove that $y$ is irrational.
        \item Produce a constructive proof that there exist irrational $x$ and $y$ such that $x^y$ is rational.
    \end{enumerate}
\end{exercise}

%=========================================
\chapter{Number Theory}
\section{Divisibility}
We say that \textbf{$a$ divides $b$}\index{divides} if there is an integer $k$ such that $ak = b$. This is denoted $a\vert b$. For example, $7\vert 63$ since $7 \times 9 = 63$. However, 8 does not divide 63, so we write $8 \nmid 63$. A consequence of this definition is that every number divides zero since $a \times 0 = 0$ for every integer $a$.

If $a$ divides $b$, then $b$ is a \textbf{multiple of $a$}\index{multiple}. For example, 63 is a multiple of 7, but 63 is not a multiple of 8.

Here are some basic facts about divisibility that are not difficult to prove.
\begin{itemize}
    \item If $a\vert b$ then $a\vert bc$ for all integers $c$.
    \item If $a\vert b$ and $b\vert c$ then $a\vert c$.
    \item If $a\vert b$ and $a\vert c$ then $a\vert sb+tc$ for all integers $s$ and $t$.
    \item If $c \neq 0$, then $a\vert b$ if and only if $ac\vert bc$.
\end{itemize}

A number $p > 1$ with no positive divisors other than 1 and itself is called a \textbf{prime}\index{prime}. Every other number greater than 1 is called \textbf{composite}\index{composite}. For example, 2, 3, 5, 7, 11, and 13 are all prime, but 4, 6, 8, and 9 are composite. The number 1 is considered neither prime nor composite.

\begin{theorem}[Fundamental Theorem of Arithmetic]\label{thrm-fundamental-theorem-of-arithmetic}
    For every integer $n > 1$, $n$ can be expressed as the product of one or more prime numbers, uniquely up to the order in which they appear.
\end{theorem}

\begin{exercise}
    Express 44100 as a product of primes.
\end{exercise}

\section{Euclid's Division Lemma}
When dividing an integer by another, we may sometimes have a \textbf{remainder}\index{remainder} left over. For example, when 63 is divided by 8, a remainder of 7 is left over. More precisely, if you divide $n$ by $d$, then you get a quotient $q$ and a remainder $r$. This is given by Euclid's Division Lemma\index{Euclid's Division Lemma}.
\begin{lemma}[Euclid's Division Lemma]\label{lemma-euclid-division}
    Given two integers $n$ and $d$, with $n \neq 0$, there exist unique integers $q$ and $r$ such that
    \[
        n = qd + r
    \]
    and $0 \leq r < |d|$ where $|d|$ denotes the absolute value of $d$.
\end{lemma}
\begin{remark}
    Some authors may call this lemma \textbf{Euclid's division algorithm}, or \textbf{the division algorithm}\index{division algorithm}.
\end{remark}

\begin{example}
    Using $n = 63$ and $d = 8$, we will have $63 = 7\times8 + 7$.
\end{example}
\begin{example}
    Using $n = 14$ and $d = -3$, we will have $13 = -5\times3 + 1$.
\end{example}

\begin{exercise}
    Express $-210$ in the form $a-13b$, where $a$ and $b$ are positive integers with $0 \leq a \leq 12$.
\end{exercise}

\section{Greatest Common Divisor (GCD) and Lowest Common Multiple (LCM)}
In number theory, the idea of a greatest common divisor and the least common multiple are omnipresent.

\begin{definition}
    Let $m$ and $n$ be two non-zero integers. Then an integer $d$ is said to be the \textbf{greatest common divisor}\index{greatest common divisor} (GCD)!\index{GCD} of $m$ and $n$ if $m = pd$ and $n = qd$ for some integers $p$ and $q$, and that $d$ is the largest possible integer that achieves this.
\end{definition}
The GCD of $m$ and $n$ is denoted by $\gcd(m, n)$.

\begin{example}
    $\gcd(2, 8) = 2$ since $2 = 1 \times 2$ and $8 = 4 \times 2$.
\end{example}
\begin{example}
    $\gcd(42, 231) = 21$ since $42 = 2 \times 21$ and $231 = 11 \times 21$.
\end{example}
\begin{example}
    $\gcd(-10, 25) = 5$ since $-10 = -2 \times 5$ and $25 = 5 \times 5$.
\end{example}
\begin{exercise}
    Find $\gcd(-112, -35)$.
\end{exercise}

\begin{remark}
    If $\gcd(m, n) = 1$ for non-zero integers $m$ and $n$, then $m$ and $n$ are said to be \textbf{coprime}\index{coprime} to each other.
\end{remark}
\begin{example}
    17 and 18 are coprime as $\gcd(17, 18) = 1$.
\end{example}

We now look at the lowest common multiple of two integers.
\begin{definition}
    Let $m$ and $n$ be two non-zero integers. Then an integer $l$ is said to be the \textbf{lowest common multiple}\index{lowest common multiple} (LCM)\index{LCM} of $m$ and $n$ if $m = pl$ and $n = ql$ for some integers $p$ and $q$, and that $l$ is the smallest possible \textbf{positive} integer that achieves this.
\end{definition}
The LCM of $m$ and $n$ is denoted by $\lcm(m,n)$.

\begin{example}
    $\lcm(2, 8) = 8$ since $8 = 4 \times 2$ and $8 = 1 \times 8$.
\end{example}
\begin{example}
    $\lcm(42, 231) = 462$ since $462 = 11 \times 42$ and $462 = 2 \times 231$.
\end{example}
\begin{example}
    $\lcm(-10, 25) = 50$ since $50 = 5 \times -10$ and $50 = 2 \times 25$.
\end{example}
\begin{exercise}
    Find $\lcm(-112, -35)$.
\end{exercise}

We note several properties of the GCD and LCM.
\begin{lemma}[B\'{e}zout]\label{lemma-bezout}\index{B\'{e}zout's Lemma}
    Let $m$ and $n$ be non-zero integers such that $\gcd(m, n) = d$. Then there exist integers $x$ and $y$ such that $mx + ny = d$. Moreover, the integers of the form $am + bn$ (where $a$ and $b$ are integers) are multiples of $d$.
\end{lemma}
\begin{proposition}\label{prop-product-of-gcd-and-lcm}
    Let $m$ and $n$ be non-zero integers. Then
    \[
        |mn| = \gcd(m,n) \times \lcm(m,n)    
    \]
    where $|mn|$ denotes the absolute value of $mn$.
\end{proposition}

\begin{exercise}
    Suppose $m = 42$ and $n = 70$.
    \begin{enumerate}[label=(\roman*)]
        \item Let $d = \gcd(m,n)$. Find $d$.
        \item Hence find $\lcm(m,n)$.
        \item Find a pair of integers $x$ and $y$ such that $mx + ny = d$.
    \end{enumerate}
\end{exercise}

%=========================================
\chapter{Modular Arithmetic}
\section{Modulo and Modular Congruence}

Given an integer $n>1$, called a \textbf{modulus}\index{modulus}, two integers $a$ and $b$ are said to be \textbf{congruent modulo $n$}\index{congruence} if $n$ is a divisor of their difference (that is, if there is an integer $k$ such that $a - b = kn$). It is denoted $a \equiv b \pmod{n}$. The parentheses mean that ``$\pmod{n}$'' applies to the entire equation, not just to the right-hand side (here, $b$). This notation is not to be confused with the notation ``$b \mod n$'' (without parentheses), which refers to the modulo operation.
\begin{remark}
    Equivalently, $a \equiv b \pmod n$ means that $a = kn + b$ for some integer $k$.
\end{remark}
\begin{example}
    In modulo 12, note that $38 \equiv 14 \pmod{12}$ since $38 - 14 = 24 = 2 \times 12$. Another way to express this is to say that both 38 and 14 have the same remainder 2, when divided by 12.
\end{example}

\begin{exercise}
    Let $m = 5$ and $n = 3$.
    \begin{enumerate}[label=(\alph*)]
        \item State the value of $17 \mod m$.
        \item Find an $x$ where $0 \leq x < m$ and $19 \equiv x \pmod m$.
        \item If $A = 1234n + 5$, what is $A \mod n$?
    \end{enumerate}
\end{exercise}

The definition of congruence also applies to negative values. For example:
\begin{itemize}
    \item $-3 \equiv 2 \pmod5$
    \item $-8 \equiv 7 \equiv 2 \pmod5$
    \item $-1 \equiv n-1 \pmod{n}$
\end{itemize}

\begin{exercise}
    Explain why $-n \equiv n \pmod{2n}$.
\end{exercise}

The operation of congruence modulo $n$ has a few properties\index{congruence!properties} which we will state without proof. Let $k$ be an integer, $a_1 \equiv b_1 \pmod n$ and $a_2 \equiv b_2 \pmod n$, or if $a \equiv b \pmod n$. Then
\begin{itemize}
    \item $a + k \equiv b + k \pmod n$;
    \item $ka \equiv kb \pmod n$;
    \item $ka \equiv kb \pmod {kn}$;
    \item $a_1 + a_2 \equiv b_1 + b_2 \pmod n$;
    \item $a_1 - a_2 \equiv b_1 - b_2 \pmod n$;
    \item $a_1a_2 \equiv b_1b_2 \pmod n$;
    \item $a^k \equiv b^k \pmod n$ if $k \geq 0$;
    \item if $a+k \equiv b+k \pmod n$ then $a \equiv b \pmod n$;
    \item if $ka \equiv kb \pmod n$ and $\gcd(k, n) = 1$, then $a \equiv b \pmod n$; and
    \item if $ka \equiv kb \pmod{kn}$ where $k \neq 0$ then $a \equiv b \pmod n$.
\end{itemize}

\begin{exercise}
    Find the last two digits of $778899^{112233}$.
\end{exercise}

\section{Modular Multiplicative Inverse}
\begin{definition}
    Let $m$ be a positive integer, and let $a$ be an integer. Then an integer $x$ that makes $ax \equiv 1 \pmod m$ is said to be the \textbf{modular multiplicative inverse of $a$}\index{multiplicative inverse!modular}.
\end{definition}
\begin{remark}
    In these books, however, we refer to the modular multiplicative inverse as just \textbf{multiplicative inverse}\index{multiplicative inverse}.
\end{remark}
\begin{example}
    4 is the multiplicative inverse of 7 modulo 9 since $4 \times 7 = 28 = 3 \times 9 + 1 \equiv 1 \pmod 9$.
\end{example}

\begin{proposition}\label{prop-multiplicative-inverse-exists-iff-coprime}
    A multiplicative inverse of $a$ modulo $m$ exists if and only if $\gcd(a,m) = 1$.
\end{proposition}
\begin{proof}
    We first work forwards and suppose $k$ is the multiplicative inverse of $a$ modulo $m$. Then $ka \equiv 1 \pmod m$. Hence $ka - 1 \equiv 0 \pmod m$, so $m$ divides $ka - 1$. This means that $ka - 1$ is a multiple of $m$, so $ka - 1 = pm$ for some integer $p$. Therefore $ka + pm = 1$ By B\'{e}zout's Lemma (\myref{lemma-bezout}) this means that $\gcd(a, m) = 1$.
    
    Now, working in the reverse direction, suppose $\gcd(a, m) = 1$. By B\'{e}zout's Lemma this means there exist integers $k$ and $p$ such that $ka + pm = 1$. Then $ka - 1 = pm$ which means $m$ divides $ka - 1$, which hence means $ka - 1 \equiv 0 \pmod m$ which the result quickly yields.
\end{proof}

\begin{example}
    The number 20 has a multiplicative inverse modulo 31 since $\gcd(20, 31) = 1$. One can verify that 14 is the multiplicative inverse of 20 modulo 31.
\end{example}

\begin{exercise}
    Find the modular multiplicative inverse of 123 modulo 5.
\end{exercise}

%=========================================
\appendix
\chapter{Exercise Solutions}

\section{Set Theory}
\begin{enumerate}
    \item \begin{enumerate}[label=(\alph*)]
        \item True, as both 1 and 2 appear in the set $\{1, 2, 3, 4\}$.
        \item False, 3 does not appear in $\{1, 2, 4\}$.
        \item True. Any set is a subset of itself, including the empty set.
        \item False, the set $S$ does not contain any element that is not in $S$. That is, $S \subseteq S$ but not $S \subset S$.
        \item True. $S$ is indeed an element of $\{S, \emptyset\}$.
        \item True. The set containing S is not an element of $\{S, \emptyset\}$.
        \item False, the set $S$ is not a subset of the set $\{S, \emptyset\}$.
        \item True. The set containing $S$ is a subset of the set containing $S$ and the empty set.
    \end{enumerate}
    
    \item \begin{enumerate}[label=(\alph*)]
        \item True.
        \item False, $S \cup U = \{1, 2, 3, 4, (2, 2), (3, 3), (5, 5)\}$.
        \item True.
        \item True.
        \item True.
        \item False, $S \setminus \{1, 4\} = \{2, 3\}$, not $T = \{2, 3, 5\}$.
        \item True.
        \item True. $(S \cup T)^2 = \{(1,1), (2,2), (3,3), (4,4), (5,5)\}$, so $U = \{(2,2), (3,3), (5,5)\} \subset (S \cup T)^2$.
    \end{enumerate}

    \item We note $S$ are all the non-positive rational numbers, and the $T$ has elements $\{-2, 0, 2, \dots, 8, 10\}$. Hence $S \cap T$ has only two elements, namely $-2$ and $0$.
\end{enumerate}

\section{Functions / Mappings}
\begin{enumerate}
    \item \begin{enumerate}[label=(\roman*)]
        \item $f: \{1, 2, 3\} \to \{1, 4, 9, 16, 25\}, x \mapsto x^2$. (Or just $x \mapsto x^2$)
        \item Domain is $\{1, 2, 3\}$, codomain is $\{1, 4, 9, 16, 25\}$, range is $\{1, 4, 9\}$.
        \item No. The element 3 would map to 27, which is not in the codomain.
    \end{enumerate}
    
    \item It is not well-defined. Note $\frac 12 = \frac 24$, but $f(\frac12) = 1 + 2 = 3$ and $f(\frac24) = 2 + 4 = 6$.
    
    \item $fg(x) = \left(\frac1{x^2+1}\right)^2 - \frac1{x^2+1} + 1$.
    
    \item We prove the requirements of a bijection one by one.
    \begin{itemize}
        \item \textbf{Injective}: Suppose $x_1, x_2 \in S$ such that $f(x_1) = f(x_2)$. We split into three cases.
        \begin{itemize}
            \item The first case is if $f(x_1) = f(x_2) = 0$. In this case, one sees clearly that $x_1 = x_2 = 1$.
            \item The second case is if $f(x_1) = f(x_2) > 0$. Now since $x \neq 1$ (as this case leads to $f(x_1) = 0$), the `valid' odd numbers are at least 3. Therefore, $\frac{1-x}{2} \leq \frac{1-3}{2} = -1 < 0$, so $x_1$ and $x_2$ cannot be odd. Hence, $x_1$ and $x_2$ are even, meaning $\frac{x_1}{2} = \frac{x_2}{2}$ which quickly implies $x_1 = x_2$.
            \item The third case is if $f(x_1) = f(x_2) < 0$. As argued above, this means that $x_1$ and $x_2$ must be odd numbers of at least 3. Hence, $\frac{1-x_1}{2} = \frac{1-x_2}{2}$ which quickly implies $x_1 = x_2$.
        \end{itemize}
        Thus, in all three cases, $f(x_1) = f(x_2)$ implies $x_1 = x_2$, meaning $f$ is injective.
        \item \textbf{Surjective}: Suppose $y \in \mathbb{Z}$. We split into three cases again.
        \begin{itemize}
            \item If $y = 0$, then setting $x = 1$ satisfies $f(x) = y$.
            \item Now suppose $y > 0$. We note $2y \in S$, and clearly $2y$ is an even integer. So setting $x = 2y$ satisfies $f(x) = \frac{2y}{y} = y$.
            \item Suppose $y < 0$. Note $-2y > 0$, and $1 - 2y > 0 \in S$. Furthermore $1 - 2y$ is clearly an odd integer. Hence setting $x = 1 - 2y$ satisfies $f(x) = \frac{1-(1-2y)}{2} = y$.
        \end{itemize}
        Therefore for every $y \in \mathbb{Z}$, there exists a pre-image $x \in S$ such that $f(x) = y$. Hence $f$ is surjective.
    \end{itemize}
    Therefore, as $f$ is both injective and surjective, $f$ is bijective. Hence, $|S| = |\mathbb{Z}|$.
\end{enumerate}

\section{Mathematical Logic and Proof Writing}
\begin{enumerate}
    \item We work from the inner-most bracket outwards. We note $P$ is true, $Q$ is false, and $R$ is true.
    \begin{itemize}
        \item $P \lor Q$ is ``1 is a positive number \textbf{or} $-1 > 0$'', which is true since $P$ is true.
        \item $(P \lor Q) \land R$ is ``(1 is a positive number or $-1 > 0$) \textbf{and} 1 is an odd number'', which is true since $P \lor Q$ is true and 1 is, indeed, an odd number.
        \item $\lnot((P \lor Q) \land R)$ is false, since $(P \lor Q) \land R$ is true.
    \end{itemize}
    Hence the statement ``$\lnot((P \lor Q) \land R)$ is false'' is a true statement.
    
    \item The truth table for $P \land (\lnot Q)$ is given below:
    \begin{table}[h]
        \centering
        \begin{tabular}{|l|l||l|}
            \hline
            $P$ & $Q$ & $P\land (\lnot Q)$ \\ \hline
            F   & F   & F                  \\ \hline
            F   & T   & F                  \\ \hline
            T   & F   & T                  \\ \hline
            T   & T   & F                  \\ \hline
        \end{tabular}
    \end{table}
    
    \item \begin{enumerate}[label=(\roman*)]
        \item $R$: $n$ is a multiple of 5 if and only if the last digit of $n$ is 0 or 5.
        \item If $n$ is a multiple of 5, then its last digit necessarily has to be 5 or 0, hence $P \implies Q$. If the last digit is 5 or 0, then the number $n$ is a multiple of 5, hence $Q \implies P$. Therefore $P \iff Q$.
    \end{enumerate}
    
    \item For brevity, let $R = (P \implies (\lnot Q))$ and $S = ((\lnot Q) \implies P)$. So we want to show that $((\lnot P) \iff Q) = R \land S$.
    \begin{table}[h]
        \centering
        \begin{tabular}{|l|l||l|l|l|l||l|l|}
            \hline
            $P$ & $Q$ & $\lnot P$ & $\lnot Q$ & $R$ & $S$ & $R \land S$ & $(\lnot P) \iff Q$ \\ \hline
            F   & F   & T         & T         & T   & F   & F           & F                  \\ \hline
            F   & T   & T         & F         & T   & T   & T           & T                  \\ \hline
            T   & F   & F         & T         & T   & T   & T           & T                  \\ \hline
            T   & T   & F         & F         & F   & T   & F           & F                  \\ \hline
        \end{tabular}
    \end{table}

    From inspection, the truth tables of $(\lnot P) \iff Q$ and $R \land S$ are the same, proving our required result.
    
    \item For brevity, denote $X = P \lor \lnot Q$, $Y = \lnot R$, and $Z = (P \lor R) \land (P \lor \lnot R)$. Thus the original statement is something like $(X \land Y) \lor (X \land Z)$, which by distributive rules is equal to $X \land (Y \lor Z)$. Note $Z = P \lor (R \land \lnot R)$ by distributive rules, which is equal to $P$ since $R \land \lnot R$ is always false. Hence $X \land (Y \lor Z) = (P \lor \lnot Q) \land (\lnot R \lor P)$. Commutativity of $\lor$ means that $\lnot R \lor P = P \lor \lnot R$, so $(P \lor \lnot Q) \land (\lnot R \lor P) = (P \lor \lnot Q) \land (P \lor \lnot R)$. Now by distributive rules $(P \lor \lnot Q) \land (P \lor \lnot R) = P \lor (\lnot Q \land \lnot R)$. Finally, by De Morgan's Law, $\lnot Q \land \lnot R = \lnot(Q \lor R)$, so the original statement is equal to $P \lor \lnot(Q \lor R)$.
    
    \item $\left[(n \in \mathbb{Z}) \land (n > 2)\right] \implies \left[\exists a, b \in \mathbb{Z}, a^3 + b^4 = n^5\right]$
    
    \item \begin{enumerate}[label=(\roman*)]
        \item Let
        \begin{align*}
            P:&\ (x \in \mathbb{R}) \land (x \neq 0)\\
            Q:&\ \exists y \in \mathbb{R}, xy = 1
        \end{align*}
        then the given statement is $P \implies Q$ as required.
        \item $(x \in \mathbb{R}) \land (x \neq 0) \land (\forall y \in \mathbb{R}, xy \neq 1)$
    \end{enumerate}
    
    \item Suppose that $m$ and $n$ have the same parity. We split into two cases.
    \begin{itemize}
        \item If both $m$ and $n$ are even, then we may write $m = 2a$ and $n = 2b$ where $a$ and $b$ are integers. Hence
        \[
            m + n = 2a + 2b = 2(a+b)        
        \]
        which clearly means that $m + n$ is even.
        \item If both $m$ and $n$ are odd, then we may write $m = 2a + 1$ and $n = 2b + 1$ where $a$ and $b$ are integers. Hence
        \[
            m + n = (2a + 1) + (2b + 1) = 2(a + b + 1)        
        \]
        which clearly means that $m+n$ is even.
    \end{itemize}
    Hence in both cases $m + n$ is even.
    
    \item We consider a proof by contrapositive; the statement that we want to prove is ''if \textbf{not} ($a$ is even or $b$ is odd) then $a(b^2+5)$ is \textbf{not} even''. That is, ''if $a$ is \textbf{not} even \textbf{and} $b$ is \textbf{not} odd then $a(b^2+5)$ is \textbf{not} even'', meaning ''if $a$ is odd and $b$ is even then $a(b^2+5)$ is odd''.
    
    Suppose that $a$ is odd and $b$ is even. Then we may write $a = 2m + 1$ and $b = 2n$ where $m$ and $n$ are integers. Hence
    \begin{align*}
        a(b^2+5) &= (2m+1)\left((2n)^2 + 5\right)\\
        &= (2m+1)(4n^2 + 5)\\
        &= 8mn^2 + 10m + 4n^2 + 5\\
        &= 8mn^2 + 10m + 4n^2 + 4 + 1\\
        &= 2(4mn^2 + 5m + 2n^2 + 2) + 1
    \end{align*}
    which clearly means that $a(b^2+5)$ is odd.
    
    \item By way of contradiction assume there exist integers $a$ and $b$ such that $2a + 4b = 1$. Then dividing both sides by 2 leads to $a + 2b = \frac12$. Note the left hand side is clearly an integer, while the right hand side is not an integer, a contradiction.
    
    \item By way of contradiction assume that $a$ and $b$ are positive real numbers, and $\frac{a+b}{2} < \sqrt{ab}$. This means $a+b<2\sqrt ab$. Squaring both sides yields $(a+b)^2 < 4ab$. Note
    \[
        (a+b)^2 = a^2 + 2ab + b^2 < 4ab    
    \]
    which implies $a^2 + b^2 < 2ab$, leading to $a^2 - 2ab + b^2 < 0$. However $a^2 - 2ab + b^2 = (a-b)^2 \geq 0$ for all positive real numbers $a$ and $b$. Hence we have $(a-b)^2 < 0$ and $(a-b)^2 \geq 0$, a contradiction.
    
    \item We note that a positive odd number is of the form $2n - 1$ where $n$ is a positive integer. Setting $a = 2n - 1$; we induct on $n$.
    
    When $n = 1$, we have $a^2 - 1 = (2(1) - 1)^2 - 1 = 1 - 1 = 0$ which is clearly a multiple of 8.
    
    Assume that the statement holds for some positive integer $k$, i.e. $(2k-1)^2 - 1 = 8m$ for some integer $m$. We show that the statement holds for $k + 1$.
    
    We note that $(2k-1)^2 - 1 = 4k^2 - 4k$. Observe
    \begin{align*}
        (2(k+1)-1)^2 - 1 &= (2k+1)^2 - 1\\
        &= 4k^2 + 4k + 1 - 1\\
        &= (4k^2 - 4k) + 8k\\
        &= 8m + 8k & (\text{by hypothesis})\\
        &= 8(m+k)
    \end{align*}
    which means that $(2(k+1)-1)^2 - 1$ is a multiple of 8, proving that the statement holds for $k+1$. By mathematical induction, $a^2 - 1$ is a multiple of 8 for all positive odd integers $a$.
    
    \item We induct on $n$ using strong induction.
    
    We prove the base cases of 1 and 2 first:
    \begin{itemize}
        \item When $n = 1$, we have $2^1 - 1 - 1 = 0 = f(1)$, so this case is true.
        \item When $n = 2$, we have $2^2 - 2 - 1 = 1 = f(2)$, so this case is also true.
    \end{itemize}
    
    Assume that for some positive integer $k \geq 2$, for all positive integers $m \leq k$ we have $f(m) = 2^m - m - 1$. We want to show that $f(k+1) = 2^{k+1} - (k+1) - 1$.
    \begin{align*}
        f(k+1) &= 3f(k) - 2f(k-1) + 1\\
        &= 3(2^k - k - 1) - & (\text{hypothesis})\\
        &\quad\quad2(2^{k-1} - (k-1) - 1) + 1 & (\text{hypothesis}) \\
        &= 3\times 2^k - 3k - 3 -2^k + 2k + 1\\
        &= 2\times2^k - k - 2\\
        &= 2^{k+1} - (k+1) - 1
    \end{align*}
    so the statement holds for $k+1$.
    
    By mathematical induction, we conclude $f(n) = 2^n - n - 1$ for all positive integers $n$.
    
    \item We prove the forward direction using direct proof. Assume $n$ is one more than a multiple of 5. Then we may write $n = 5a + 1$ where $a$ is an integer. Note $5a + 1 = 5a + (5 - 4) = (5a + 5) - 4 = 5(a+1) - 4$. Setting $k = a+1$ yields required result.
    
    We now prove the reverse direction, using direct proof as well. Assume $n = 5k - 4$. Observe $5k - 4 = 5k - 5 + 1 = 5(k-1) + 1$, meaning $n$ is one more than a multiple of 5.
    
    \item The number 7 satisfies this as $7 = 2^3 - 1$ and $7 = 3^2 - 2$.
    
    \item \begin{enumerate}[label=(\roman*)]
        \item We use a proof by contradiction to prove this claim. Seeking a contradiction, assume $y$ is rational. Write $y = \frac pq$ where $p$ and $q$ are integers. Note $2^y = 2^{\frac pq}$ and $2^y = 2^{2\log_2{3}} = 9$. Hence $2^{\frac pq} = 9$ meaning $2^p = 9^q$. However $2^p$ is always even and $9^q$ is always odd, a contradiction.
        \item Note
        \[
            x^y = (\sqrt2)^{2\log_2{3}} = \left(\sqrt{2}^2\right)^{\log_2{3}} = 2^{\log_2{3}} = 3        
        \]
        which is rational.
    \end{enumerate}
\end{enumerate}

\section{Number Theory}
\begin{enumerate}
    \item $44100 = 2^2 \times 3^2 \times 5^2 \times 7^2$.
    \item $-210 = 11 - 13 \times 17$, so $a = 11$ and $b = 17$.
    \item $\gcd(-112, -35) = 7$ since $-112 = -16 \times 7$ and $-35 = -5 \times 7$, with 7 being the largest integer that achieves this.
    \item $\lcm(-112, -35) = 560$ since $560 = -5 \times -112$ and $-35 = -16 \times -35$, with 560 being the smallest \textit{positive} integer that achieves this.
    \item \begin{enumerate}[label=(\roman*)]
        \item $\gcd(42, 70) = 14$ since $42 = 3 \times 14$ and $70 = 5 \times 14$, and 14 is the largest integer achieving this.
        \item $\lcm(42, 70) = \frac{42 \times 70}{\gcd(m, n)} = \frac{2940}{14} = 210$ by \myref{prop-product-of-gcd-and-lcm}.
        \item Note that $x = 2$ and $y = -1$ works as $42 \times 2 + 70 \times (-1) = 84 - 70 = 14$.
    \end{enumerate}
\end{enumerate}

\section{Modular Arithmetic}
\begin{enumerate}
    \item \begin{enumerate}[label=(\alph*)]
        \item $17 \mod 5 = 2$ since $17 = 3 \times 5 + 2$ by the division algorithm.
        \item As $19 = 3 \times 5 + 4$, thus $19 \equiv 4 \pmod 5$. Hence $x = 4$.
        \item $A \mod n = 5 \mod 3 = 2$.
    \end{enumerate}
    \item $-n = (-1) \times 2n + n$, which means that $-n \equiv n \pmod{2n}$.
    \item Finding the last two digits of a number is the same as finding the remainder of that number when divided by 100. We note $778899 \equiv 99 \pmod{100}$, so $778899^{112233} \equiv 99^{112233} \pmod{100}$. Furthermore, $99 \equiv -1 \pmod{100}$, so $99^{112233}\equiv (-1)^{112233} \equiv -1 \equiv 99 \pmod{100}$. Hence the last two digits of $778899^{112233}$ are both 9.
    \item Note $123 \equiv 3 \pmod 5$. One can easily find by trial and error that 2 is the multiplicative inverse of 3, since $3 \times 2 = 6 \equiv 1 \pmod 5$. Hence the multiplicative inverse of 123 is 2 modulo 5.
\end{enumerate}

%=========================================
\chapter{Image Acknowledgements}
Unless otherwise stated, all images are the author's own work, and are released under the same licence as this book.

%=========================================
\printbibliography[heading=bibintoc, title={References and Bibliography}]
\printindex

\end{document}
