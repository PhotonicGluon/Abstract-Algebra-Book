\documentclass[
  a5paper,
  pagesize,
  10pt,
  bibtotoc,
  normalheadings,
  twoside,
  openany,
  chapterprefix,
  DIV=9
]{scrbook}

\usepackage[utf8]{inputenc}
\usepackage{tocloft}
\usepackage{mathtools}
\usepackage{amsfonts}
\usepackage{enumitem}
\usepackage{hyperref}
\usepackage{amsmath}
\usepackage{amsthm}
\usepackage{amssymb}
\usepackage[hmargin=2cm, vmargin=2.5cm]{geometry}
\usepackage{graphicx}
\usepackage{wrapfig}
\usepackage{parskip}
\usepackage{framed}
\usepackage{fancyhdr}
\usepackage{emptypage}

\usepackage[
    backend=bibtex,
    style=alphabetic,
    sorting=ynt
]{biblatex}

%=========== Path to images ==============
\graphicspath{{./images/}}

%============== Resources ================
\addbibresource{../AbstractAlgebra.bib}

%============ Redefinitions ==============
\let\oldemptyset\emptyset
\let\emptyset\varnothing

\let\totient\varphi

\renewcommand{\vert}{ \ | \ }

%======== Theorem-Like Things ============
\newtheoremstyle{exercise-style}
    {-5pt}       % Space above
    {\topsep}    % Space below
    {}           % Font to use in exercise
    {0pt}        % Measure of space to indent
    {\bfseries}  % Name of the head font
    {.}          % Punctuation between head and body
    { }          % Space after theorem head; " " = normal inter-word space
    {\thmname{#1}\thmnumber{ #2}\textnormal{\thmnote{ (#3)}}}

\newtheorem{theorem}{Theorem}[section]
\renewcommand{\thetheorem}{\Roman{part}.\arabic{chapter}.\arabic{section}.\arabic{theorem}}

\newtheorem{conjecture}[theorem]{Conjecture}
\newtheorem{proposition}[theorem]{Proposition}
\newtheorem{definition}[theorem]{Definition}
\newtheorem{lemma}[theorem]{Lemma}
\newtheorem{lemma-thrm}{Lemma}[theorem]
\newtheorem{corollary}[theorem]{Corollary}
\newtheorem{corollary-thrm}{Corollary}[theorem]
\theoremstyle{definition}\newtheorem*{remark}{Remark}
\theoremstyle{definition}\newtheorem{example}[theorem]{Example}

\theoremstyle{exercise-style}\newtheorem{exercisehidden}{Exercise}[chapter]
\renewcommand{\theexercisehidden}{\Roman{part}.\arabic{chapter}.\arabic{exercisehidden}}

\theoremstyle{definition}\newtheorem{problem}{Problem}[chapter]
\renewcommand{\theproblem}{\Roman{part}.\arabic{chapter}.\arabic{problem}}

%============ Environments ===============
\newenvironment{exercise}
{\begin{framed}\noindent\begin{exercisehidden}}
{\end{exercisehidden}\end{framed}}

%=========== Custom Commands =============
\newcommand{\code}[1]{\texttt{#1}}  % Code block
\makeatletter\newcommand*{\rom}[1]{\Ifstr{#1}{0}{0}{\expandafter\@slowromancap\romannumeral #1@}}\makeatother  % Roman numeral

\newcommand{\lcm}{\mathrm{lcm}}  % Lowest common multiple function
\newcommand{\sgn}{\mathrm{sgn}}  % Signum function

\newcommand{\im}{\mathrm{im}\;}  % Image of a function
\newcommand{\id}{\mathrm{id}}    % Identity function

\newcommand{\An}[1]{\mathrm{A}_{#1}}                  % Alternating group of degree n
\newcommand{\Aut}[1]{\mathrm{Aut}(#1)}                % Group of automorphisms of G
\newcommand{\C}[2]{\mathrm{C}_{#1}(#2)}               % Centralizer of an element in G
\newcommand{\Cl}[1]{\mathrm{Cl}(#1)}                  % Conjugacy class of the element x
\newcommand{\Cn}[1]{\mathrm{C}_{#1}}                  % Cyclic group of order n
\newcommand{\GL}[2]{\mathrm{GL}_{#1}\left(#2\right)}  % General Linear Group of degree n
\newcommand{\Inn}[1]{\mathrm{Inn}(#1)}                % Group of inner automorphisms of G
\newcommand{\N}[2]{\mathrm{N}_{#1}(#2)}               % Normalizer of S in G
\newcommand{\Out}[1]{\mathrm{Out}(#1)}                % Group of outer automorphisms of G
\newcommand{\SL}[2]{\mathrm{SL}_{#1}\left(#2\right)}  % Special Linear Group of degree n
\newcommand{\Sn}[1]{\mathrm{S}_{#1}}                  % Symmetric group of degree n
\newcommand{\Syl}[2]{\mathrm{Syl}_{#1}(#2)}           % Set of Sylow p-groups of G
\newcommand{\Sym}[1]{\mathrm{Sym}(#1)}                % Symmetric group of a set
\newcommand{\Un}[1]{\mathcal{U}_{#1}}                 % Group of units modulo n
\newcommand{\Z}[1]{\mathrm{Z}(#1)}                    % Center of a group G

\newcommand{\Stab}[2]{\mathrm{Stab}_{#1}(#2)}  % Stabilzer of x by G
\newcommand{\Fix}[2]{\mathrm{Fix}_{#1}(#2)}    % Set of all elements in X which is fixed by g
\newcommand{\Orb}[2]{\mathrm{Orb}_{#1}(#2)}    % Orbit of x under G

%============ Custom Header ==============
\fancypagestyle{plain}{\fancyhf{}\renewcommand{\headrulewidth}{0pt}} % To clear page numbers from footer, and header line at the start of every chapter

\pagestyle{fancy}
\fancyhf{}% Clear header/footer

\fancyhead[LE,RO]{\thepage}
\fancyhead[LO,RE]{\textit{\nouppercase\leftmark}}

%======== Custom Chapter Styling =========
\makeatletter
\renewcommand*{\chapterformat}{
  \MakeUppercase{\chapapp\nobreakspace\thechapter}
}

\renewcommand*{\chapterlineswithprefixformat}[3]{
    \Ifstr{#1}{chapter}{
        \vspace{-60px}
        \Ifstr{#2}{\empty}{\vspace{40px}}{\raggedleft#2}
        \vspace{-15px}
        \rule{\linewidth}{1pt}\par\nobreak
        \centering{#3}
        \vspace{-10px}
        \rule{\linewidth}{1pt}\par\nobreak
        \vspace{-10px}
    }{#2#3}
}
\makeatother

%== Customise Table of Contents Heading ==
\makeatletter
\def\createtoc{
    \renewcommand\tableofcontents{
        \chapter*{\contentsname}
        \@starttoc{toc}
    }
    \tableofcontents
}
\makeatother

%========= Front Matter Pages ============
\def\volumetitle{Volume \rom{\volumenumber}: \volumename}

\def\frontmatterpages{
    \frontmatter  % Use lowercase roman numerals for page numbers

    % Title page
    \begin{titlepage}
        \centering{
            \selectfont
            \Huge
            \textbf{Abstract Algebra}\\
            \vspace{-0.2cm}
            
            \Large
            \textbf{A Simple Introduction}\\
            \vspace{0.5cm}
            
            \LARGE
            \volumetitle
        }\\
        \vspace{2cm}
        \centering{\Large{Overwrite}}\\
        \vspace{\fill}
        \centering \small{\textit{Version \version}}
    \end{titlepage}

    \newpage{}

    % Edition notice
    \clearpage\null\vfill
    \thispagestyle{empty}
    \begin{minipage}[b]{0.9\textwidth}
        \footnotesize\raggedright
        \setlength{\parskip}{0.5\baselineskip}

        Published by Kan Onn Kit\\
        Singapore
        \vspace{7cm}

        \textbf{Abstract Algebra: A Simple Introduction -- \volumetitle}\par
        Version \version
        \vspace{0.35cm}

        Copyright \copyright \ 2022 -- \the\year\ by Kan Onn Kit\par
        This work is licensed under a
        Creative Commons Attribution-NonCommercial-ShareAlike 4.0 International Licence.\par
        \includegraphics[width=2.5cm]{../Images/CC BY-NC-SA 4.0.png}\\  % With reference to the volumes' folders
        The full licence text is available at \url{http://creativecommons.org/licenses/by-nc-sa/4.0/}.\par    
        The source files for the project are available \href{https://github.com/PhotonicGluon/Abstract-Algebra-Book}{here}.
        \vspace{0.35cm}

        Typeset in 10pt Computer Modern Roman using PDF\LaTeX.
    \end{minipage}

    \vspace*{2\baselineskip}
    \cleardoublepage

    % "Quote" page
    \thispagestyle{empty}
    \vspace*{2cm}

    \begin{center}
        \Large{\parbox{10cm}{
            \begin{raggedright}
                \Large
                \quotepagetext
                \vspace{0.3cm}
                
                \hfill
                --- \quotepageattribution\\
                \vspace{-0.25cm}
                
                \hfill
                \normalsize
                (\quotepagecitation)
            \end{raggedright}
        }
    }
    \end{center}

    \newpage

    % Table of contents
    \createtoc
    \setcounter{part}{\volumenumber}

    % Preface
    \chapter{Preface}
    Although algebra has a long history, it has undergone some quite striking changes in the past few decades. Abstract algebra is widely recognised as an essential element of higher mathematical education. The results and theorems that it showcases, however, are oft hard to grasp and understand without prerequisite knowledge or with a heavy background in mathematics. Most books on this subject are crafted for undergraduates at universities. They are not for a general mathematics enthusiast or one who seeks to understand more about the inner structure of algebra that many mathematicians encounter frequently.

    It is thus the goal of this series of books to provide a step-by-step explanation of core results from abstract algebra; to demystify the core steps that many textbooks skip over when writing proofs. I aim to ensure that the results from such an essential field of study are as accessible, as approachable, and as understandable for as many people as possible.

    \prefacevolumetext

    \hfill{\textit{27 January, 2023}}

    \mainmatter  % Now use arabic numerals for page numbers
}


%============= Formatting ================
\linespread{1.05}

%======== Theorem-Like Things ============
\renewcommand{\thetheorem}{\arabic{part}.\arabic{chapter}.\arabic{section}.\arabic{theorem}}
\renewcommand{\theexercisehidden}{\arabic{part}.\arabic{chapter}.\arabic{exercisehidden}}
\renewcommand{\theproblem}{\arabic{part}.\arabic{chapter}.\arabic{problem}}

%============== Title Page ===============
\begin{document}
\begin{titlepage}
    \centering{
        {
            \selectfont
            \Huge
            \textbf{Abstract Algebra}\\
            \vspace{-2mm}
            \Large
            \textbf{A Gentle Introduction}\\
            \vspace{5mm}
            \LARGE
            Volume 0: Prerequisites
        }
    }\\
    \vspace{20mm}
    \centering{\Large{Overwrite}}\\
    \vspace{\fill}
    \centering \large{\the\year}\\
    \centering \small{\textit{Version 0.1}}
\end{titlepage}

%=========================================
\newpage{}
\thispagestyle{empty}

\vspace*{2cm}

\begin{center}
    \Large{\parbox{10cm}{
        \begin{raggedright}
        {
            \large
            Aus dem Paradies, das Cantor uns geschaffen, soll uns niemand vertreiben können.\\
            \textit{(No one shall expel us from the Paradise that Cantor has created.)}
        }

            \vspace{.5cm}
            \hfill{--- David Hilbert, 1926}\\
            \vspace{-.25cm}
            \normalsize
            \hfill{(\cite{hilbert_1926} p. 170)}
        \end{raggedright}
    }
}
\end{center}

\newpage

%=========================================
\chapter*{Preface}
Although algebra has a long history, it has undergone some quite striking changes in the past few decades. Abstract algebra is widely recognised as an essential element of higher mathematical education. The results and theorems that it showcases, however, are oft hard to grasp and understand without prerequisite knowledge or with a heavy background in mathematics. Most books on this subject are crafted for undergraduates at universities. They are not for a general mathematics enthusiast or one who seeks to understand more about the inner structure of algebra that many mathematicians encounter frequently.

It is thus the goal of this series of books to provide a step-by-step explanation of core results from abstract algebra; to demystify the core steps that many textbooks skip over when writing proofs. I aim to ensure that the results from such an essential field of study are as accessible, as approachable, and as understandable for as many people as possible.

This volume, in particular, serves to provide the necessary prerequisites needed to engage with the subject material. In particular, this volume covers basic set theory, functions/mappings, elementary number theory, simple modular arithmetic, mathematical logic and proof writing, and mathematical induction. This background should be sufficient to understand the subject material described in the other volumes.

\hfill{\textit{27 January, 2023}}
\newpage

%=========================================
\tableofcontents
\setcounter{part}{0}

%=========================================
\chapter{Set Theory}
Set theory is the branch of mathematics that studies sets, which can be informally described as collections of objects.

Set theory begins with a binary relation between an object $x$ and a set $S$. If $x$ is an element of $S$, we write $x \in S$. This is read as ``$x$ is an element of the set $S$''. Otherwise, we write $x \notin S$. For convenience, if $x \in S$ and $y \in S$, we may write $x, y \in S$. If $z \in S$ also, we may write $x, y, z \in S$. The same applies for ``$\notin$''.

A set is described by listing elements separated by commas, or by a characterizing property of its elements, within braces $\{ \ \}$. Since sets are objects, the membership relation can relate sets as well.

\textit{A set is completely determined by its elements. Two sets are equal if and only if they have precisely the same elements}.

We denote the \textbf{empty set} by $\emptyset$, which is given by $\{\}$. That is, $\emptyset = \{\}$.

Now if we have two sets $A$ and $B$, and if all the members of set A are also members of set B, then we say that $A$ is a \textbf{subset} of $B$, and we write $A \subseteq B$. For example, $\{1, 2\} \subseteq \{1, 2, 3\}$ but $\{1, 4\} \not\subseteq \{1, 2, 3\}$. It should be noted that for any set $S$, the empty $\emptyset \subseteq S$. It should also be noted that $S \subseteq S$ for any set $S$. Thus $A$ is called a \textbf{proper subset} of $B$ if $A \neq B$ and $A \subseteq B$ (in this case, we write $A \subset B$).

\newpage

\begin{exercise}
    Let $S$ be a non-empty set. Determine whether the following statements are true or false.
    \begin{enumerate}[label=(\alph*)]
        \item $\{1, 2, 3\} \subseteq \{1, 2, 4\}$
        \item $\emptyset \subseteq \emptyset$
        \item $S \subset S$
        \item $S \in \{S, \emptyset\}$
        \item $\{S\} \notin \{S, \emptyset\}$
        \item $S \subseteq \{S, \emptyset\}$
        \item $\{S\} \subseteq \{S, \emptyset\}$
    \end{enumerate}
\end{exercise}

Set theory also features operations on sets. We list some of them here.
\begin{itemize}
    \item The \textbf{union} of two sets is the set of all objects that are a member of $A$, or $B$, or both. It is denoted $A \cup B$. For example, $\{1, 2, 3\} \cup \{2, 3, 4\} = \{1, 2, 3, 4\}$.
    \item The \textbf{intersection} of two sets is the set of all objects that are a member of \textit{both} the sets $A$ and $B$. It is denoted $A \cap B$. For example, $\{1, 2, 3\} \cap \{2, 3, 4\} = \{2, 3\}$.
    \item The \textbf{set difference} of $S$ and $A$, denoted $S \setminus A$, is the set of all members of $S$ that are not members of $A$. For example, $\{1, 2, 3\} \setminus \{2, 3, 4\} = \{1\}$ and $\{2, 3, 4\} \setminus \{1, 2, 3\} = \{4\}$.
    \item The \textbf{Cartesian Product} of $A$ and $B$, denoted $A \times B$, is the set whose members are all possible ordered pairs $(a, b)$, where $a$ is an element of $A$ and $b$ is an element of $B$. For example, $\{1, 2, 3\} \times \{4, 5\} = \{(1, 4), (1, 5), (2, 4), (2, 5), (3, 4), (3, 5)\}$. In particular, the Cartesian product $A \times A = A^2$, $A\times A \times A = A^3$, and so on.
\end{itemize}

\newpage

\begin{exercise}
    Let $S = \{1, 2, 3, 4\}$, $T = \{2, 3, 5\}$, $U = \{(2, 2), (3, 3), (5, 5)\}$. Determine whether the following statements are true or false.
    \begin{enumerate}[label=(\alph*)]
        \item $S \cup T = \{1, 2, 3, 4, 5\}$
        \item $S \cup U = \{1, 2, 3, 4, (5, 5)\}$
        \item $S \cap T = \{2, 3\}$
        \item $T \cap U = \emptyset$
        \item $S \setminus T = \{1, 4\}$
        \item $S \setminus \{1, 4\} = T$
        \item $T^2 = U$
        \item $U \subset (S \cup T)^2$
    \end{enumerate}
\end{exercise}
It is quite useful to have a notation that describes a set that is defined by a logical formula that evaluates to true for an element of the set, and false otherwise. In this form, set-builder notation has three parts: a variable, a vertical bar separator, and the logical formula. Thus there is a variable on the left of the separator, and a rule on the right of it. These three parts are contained in curly brackets:
\[
    \{x \ | \ \Phi(x)\}
\]
The vertical bar is a separator that can be read as ``such that'', ``for which'', or ``with the property that''. The formula $\Phi(x)$ is said to be the rule.

A domain $X$ can appear on the left of the vertical bar:
\[
    \{x \in X \ | \ \Phi(x)\},
\]
or by adjoining it to the logical formula:
\[
    \{x \ | \ x \in X, \Phi(x)\}.
\]
\begin{example}
    The set $S = \{x \in \mathbb{R} \ | \ x \geq 0 \}$ is the set of all non-negative real numbers.
\end{example}

The cardinality of a set $S$, denoted $|S|$, is the number of elements in $S$. The cardinality of the empty set is zero. The list of elements of some sets is endless, or infinite. In this book, we denote the cardinality of such infinite sets as $|S| = \infty$ (even though it is poorly defined in other contexts).

Finally, some notation that will be used:
\begin{itemize}
    \item $\mathbb{Z}$ denotes the set of integers (from the German ``z\"ahlen", which means ``numbers");
    \item $\mathbb{Q}$ denotes the set of rational numbers; and
    \item $\mathbb{R}$ denotes the set of real numbers.
\end{itemize}

\begin{exercise}
    List the elements in the set
    \[
        \{x \in \mathbb{Q} \vert x \leq 0\} \cap \{y \in \mathbb{Z} \vert -2 \leq y \leq 10 \text{ and } y \text{ is an even number} \}.
    \]
\end{exercise}

%=========================================
\chapter{Mathematical Logic and Proof Writing}
\section{Mathematical Statements}
\begin{definition}
    A \textbf{(mathematical) statement} is a sentence that is definitely true or definitely false.
\end{definition}
\begin{remark}
    A statement can be written in english, or using mathematical notation.
\end{remark}
\begin{example}
    The sentence ``every square with length $x$ has area $x^2$'' is a true mathematical statement.
\end{example}
\begin{example}
    The sentence ``every circle with length $x$ has area $x^2$'' is a false mathematical statement.
\end{example}
\begin{example}
    The sentence ``$12 \in \mathbb{Z}$'' is a true mathematical statement.
\end{example}
\begin{example}
    The sentence ``$\sqrt2 \in \mathbb{Z}$'' is a false mathematical statement.
\end{example}

We may name statements using variables like $P$, $Q$, $R$, etc.
\begin{example}
    If
    \begin{align*}
        P: &\ \text{Every odd number is one more than an even number}\\
        Q: &\ \text{Every triangle has sides of equal length}\\
        R: &\ \frac12 \in \mathbb{Q}
    \end{align*}
    then $P$ is true, $Q$ is false, and $R$ is true.
\end{example}

There are a few operations that may be carried out on mathematical statements. For the following, assume $P$ and $Q$ are mathematical statements.
\begin{itemize}
    \item \textbf{Logical AND}: Uses the symbol $\land$. The statement $P\land Q$ is read as ``$P$ and $Q$''.
    \item \textbf{Logical OR}: Uses the symbol $\lor$. The statement $P\lor Q$ is read as ``$P$ or $Q$''.
    \item \textbf{Logical NOT}: Uses the symbol $\lnot$. The statement $\lnot P$ is read as ``not $P$''.
    \item \textbf{Conditional}: Uses the symbol $\implies$. The statement $P \implies Q$ is read as ``$P$ implies $Q$''.  
\end{itemize}
\begin{example}
    Consider the statements
    \begin{align*}
        P: &\ \text{3 is an odd number,}\\
        Q: &\ \text{4 is an odd number,}\\
        R: &\ \text{6 is an even number.}
    \end{align*}
    Then
    \begin{itemize}
        \item $P\land Q$ is ``3 is an odd number \textbf{and} 4 is an odd number'', which is false.
        \item $P\lor Q$ is ``3 is an odd number \textbf{or} 4 is an odd number'', which is true.
        \item $\lnot Q$ is ``4 is \textbf{not} an odd number'', which is true.
    \end{itemize}
\end{example}
\begin{exercise}
    Let $P$ be the statement that ``1 is a positive number'', $Q$ be the statement ``$-1 > 0$'', and $R$ be the statement ``1 is an odd number''. Is the statement ``$\lnot((P\lor Q)\land R)$ is false'' true?
\end{exercise}

To explore the relationship between statements and operators, we use \textbf{truth tables}. A truth table is like an operation table for statements. The idea is to list all possibilities of the truth or falsity of the statements $P$ and $Q$, and then write the truth for each of the combinations with operations.

For example, here's the truth table for the logical and operator:
\begin{table}[h]
    \centering
    \begin{tabular}{|l|l||l|}
        \hline
        $P$ & $Q$ & $P\land Q$ \\ \hline
        F   & F   & F          \\ \hline
        F   & T   & F          \\ \hline
        T   & F   & F          \\ \hline
        T   & T   & T          \\ \hline
    \end{tabular}
\end{table}

Note that we denote true statements by ``T'' and false statements by ``F''.

Here's the truth table for the logical or operator:
\begin{table}[h]
    \centering
    \begin{tabular}{|l|l||l|}
        \hline
        $P$ & $Q$ & $P\lor Q$ \\ \hline
        F   & F   & F         \\ \hline
        F   & T   & T         \\ \hline
        T   & F   & T         \\ \hline
        T   & T   & T         \\ \hline
    \end{tabular}
\end{table}

And the truth table for the logical not operator:
\begin{table}[h]
    \centering
    \begin{tabular}{|l||l|}
        \hline
        $P$ & $\lnot P$ \\ \hline
        F   & F         \\ \hline
        T   & F         \\ \hline
    \end{tabular}
\end{table}

We motivate the truth table for the implication operator by considering these two statements:
\begin{align*}
    P: &\ \text{You have a red card}\\
    Q: &\ \text{You have a green card}
\end{align*}
Then, $P \implies Q$ is the statement ``if you have a red card, then you have a green card''.
\begin{itemize}
    \item If $P$ and $Q$ are true, then that means that you have both a red and green card. Hence, ``if you have a red card, then you have a green card'' is \textbf{true}, meaning $P \implies Q$ is true.
    \item If $P$ is true and $Q$ is false, then that means that you have a red card but not a green card. Hence, ``if you have a red card, then you have a green card'' is \textbf{false}, meaning $P \implies Q$ is false.
    \item Now consider the case when $P$ is false. Then regardless of what $Q$ is, the initial premise of the statement ``if you have a red card'' is not satisfied. Hence the ``promise'' that ``if you have a red card, then you have a green card'' is \textbf{not untrue}, which means that it is \textbf{true}. Hence, if $P$ is false, then $P \implies Q$ is true.
\end{itemize}
In summary, the truth table for the implication operator is:
\begin{table}[h]
    \centering
    \begin{tabular}{|l|l||l|}
        \hline
        $P$ & $Q$ & $P\implies Q$ \\ \hline
        F   & F   & T             \\ \hline
        F   & T   & T             \\ \hline
        T   & F   & F             \\ \hline
        T   & T   & T             \\ \hline
    \end{tabular}
\end{table}

\begin{exercise}
    Let $P$ and $Q$ be statements. Write the truth table for $P \land (\lnot Q)$.
\end{exercise}

We end this section by introducing the idea of the \textbf{biconditional}.
\begin{definition}
    Let $P$ and $Q$ be mathematical statements. If both $(P \implies Q)$ and $(Q \implies P)$ are true, then we write $(P \iff Q)$. In other words, $(P \iff Q) = ((P \implies Q) \land (Q \implies P))$.
\end{definition}
\begin{remark}
    The statement $(P \iff Q)$ can be written in several other ways:
    \begin{itemize}
        \item $P$ if and only if $Q$
        \item $P$ is a necessary and sufficient condition for $Q$
        \item $P$ is equivalent to $Q$
    \end{itemize}
\end{remark}

The truth table for the biconditional operator is:
\begin{table}[h]
    \centering
    \begin{tabular}{|l|l||l|}
        \hline
        $P$ & $Q$ & $P\iff Q$ \\ \hline
        F   & F   & T         \\ \hline
        F   & T   & F         \\ \hline
        T   & F   & F         \\ \hline
        T   & T   & T         \\ \hline
    \end{tabular}
\end{table}

\begin{exercise}
    Suppose $n$ is an integer. Let $P$ be the statement ``$n$ is a multiple of 5'' and $Q$ be the statement ``the last digit of $n$ is 0 or 5''. Let $R = (P \iff Q)$.
    \begin{enumerate}[label=(\roman*)]
        \item Write $R$ in English.
        \item Is $R$ true?
    \end{enumerate}
\end{exercise}

\section{Properties of Logical Operators}
A central idea of this section is that every mathematical statement can be built from just $\land$, $\lor$, and $\lnot$.

\begin{example}
    We show that $(P \implies Q) = (\lnot P) \lor Q$ by considering the truth table $(\lnot P) \lor Q$.
    \begin{table}[h]
\centering
\begin{tabular}{|l|l||l||l|}
\hline
$P$ & $Q$ & $\lnot P$ & $(\lnot P) \lor Q$ \\ \hline
F   & F   & T         & T                  \\ \hline
F   & T   & T         & T                  \\ \hline
T   & F   & F         & F                  \\ \hline
T   & T   & F         & T                  \\ \hline
\end{tabular}
\end{table}
    
    By inspection, we see that $(\lnot P) \lor Q$ has the same truth table as $P \implies Q$. Thus $(P \implies Q) = (\lnot P) \lor Q$.
\end{example}
\begin{remark}
    We separate the intermediate value(s) (e.g., $\lnot P$ in the above example) from the rest by writing a double line to the sides of the intermediate values.
\end{remark}

\begin{example}
    We show that $(P \iff Q) = (P \land Q) \lor ((\lnot P) \land (\lnot Q))$.
    \begin{table}[h]
\centering
\begin{tabular}{|l|l||l|l|l|l||l|}
\hline
$P$ & $Q$ & $\lnot P$ & $\lnot Q$ & $P \land Q$ & $(\lnot P) \land (\lnot Q)$ & $(P \land Q) \lor ((\lnot P) \land (\lnot Q))$ \\ \hline
F   & F   & T         & T         & F           & T                           & T                                              \\ \hline
F   & T   & T         & F         & F           & F                           & F                                              \\ \hline
T   & F   & F         & T         & F           & F                           & F                                              \\ \hline
T   & T   & F         & F         & T           & F                           & T                                              \\ \hline
\end{tabular}
\end{table}

    By inspection of the truth table we establish the required result.
\end{example}

\begin{exercise}
    Show that
    \[
        ((\lnot P) \iff Q) = (P \implies (\lnot Q)) \land ((\lnot Q) \implies P).
    \]
\end{exercise}

We note some important properties of logical operators.
\begin{itemize}
    \item \textbf{Contrapositive}: $(P \implies Q) = ((\lnot Q) \implies (\lnot P))$
    \item \textbf{De Morgan's Laws}: \begin{itemize}
        \item $(\lnot (P \land Q)) = ((\lnot P) \lor (\lnot Q))$
        \item $(\lnot (P \lor Q)) = ((\lnot P) \land (\lnot Q))$
    \end{itemize}
    \item \textbf{Commutativity of AND and OR}: $P \land Q = Q \land P$, as well as $P \lor Q = Q \lor P$
    \item \textbf{Associativity of AND and OR}: $P \land (Q \land R) = (P \land Q) \land R$, as well as $P \lor (Q \lor R) = (P \lor Q) \lor R$
    \item \textbf{Distributive Rules}: \begin{itemize}
        \item $P \land (Q \lor R) = (P \land Q) \lor (P \land R)$
        \item $P \lor (Q \land R) = (P \lor Q) \land (P \lor R)$
    \end{itemize}
\end{itemize}
\begin{remark}
    The most important one of these properties would arguably be the contrapositive. We will use this result several times in later volumes.
\end{remark}
\begin{exercise}
    Simplify the statement
    \[
        ((P \lor \lnot Q) \land \lnot R) \lor ((P \lor \lnot Q) \land (P \lor R) \land (P \lor \lnot R))
    \]
    into a statement that uses only \textbf{three} operators.
\end{exercise}

\section{Quantifiers}
\section{Conditional Statements}
\section{Proof by Contradiction}

%=========================================
\chapter{Mathematical Induction}

Mathematical induction is a method for proving that a proposition $P_n$ is true for every positive integer $n$, that is, that the infinitely many cases $P_1, P_2, P_3, \dots,$ all hold. Informal metaphors help to explain this technique, such as falling dominoes or climbing a ladder:
\begin{quote}
    Mathematical induction proves that we can climb as high as we like on a ladder, by proving that we can climb onto the bottom rung (the base case) and that from each rung we can climb up to the next one (the induction step).
\end{quote}

A proof by induction consists of two steps. The first, the \textbf{base case}, proves the statement for $n = 1$ without assuming any knowledge of other cases. The second, the \textbf{induction step}, proves that if the statement holds for any given case $n = k$, then it must also hold for the next case $n = k + 1$. These two steps establish that the statement holds for every natural number $n$. The base case does not necessarily begin with $n = 1$, but sometimes with $n = 0$, and possibly with any fixed natural number $n = N$, establishing the truth of the statement for all natural numbers $n \geq N$.

In summary, mathematical induction involves two steps:
\begin{itemize}
    \item \textbf{Base Case(s)}: Prove the statement for initial values.
    \item \textbf{Induction Step}: Prove that for every $n$, if the statement holds for $n$, then it holds for $n + 1$. In fact, one could prove that the statement holds for $n + k$, given that there are $k$ base case(s).
\end{itemize}

\begin{example}
    We prove the famous identity
    \[
        1 + 2 + 3 + \cdots + n = \frac{n(n+1)}2.
    \]

    When $n = 1$, the left hand side is 1; the right hand side is $\frac{1(1+1)}{2} = 1$. Thus the initial case is true.

    Now assume that the statement holds for some positive integer $k$, meaning
    \[
        1 + \cdots + k = \frac{k(k+1)}2.
    \]
    We are to prove the statement true for $k+1$, meaning
    \[
        1 + \cdots + k + (k+1) = \frac{(k+1)(k+2)}2.
    \]
    We work slowly:
    \begin{align*}
        1 + \cdots + k + (k+1) &= \frac{k(k+1)}{2} + (k+1) & \text{(by hypothesis)}\\
        &= \frac{k(k+1)}2 + \frac{2(k+1)}{2}\\
        &= \frac{k(k+1) + 2(k+1)}2\\
        &= \frac{(k+1)(k+2)}2
    \end{align*}
    which proves the case for $k + 1$. Hence $1 + 2 + 3 + \cdots + n = \frac{n(n+1)}2$.
\end{example}

\begin{example}
    We look at an example of 2-step induction. Let $\mathbb{Z}^+$ denote the set of positive integers, and consider the function $f: \left(\mathbb{Z}^+\right)^2\to\mathbb{Z}^+$ where
    \[
        f(m, n) =
        \begin{cases}
            n & \text{if } m = 1, \\
            m & \text{if } n = 1, \\
            f\left(n-1,f(n-1,m-1)\right) & \text{otherwise.}
        \end{cases}
    \]
    We will prove the non-obvious fact that $f(n+1, n) = 2$ for all positive integers $n$.

    When $n = 1$, we have $f(2, 1)$ which obviously equals 2, so the first case is true.

    When $n = 2$ we have $f(3, 2)$. Note that $f(3,2) = f(1, f(1, 2)) = f(1, 2) = 2$, so the second case is true.

    Now suppose the statement is true for $k$, i.e. $f(k+1,k) = 2$. We want to show that the case for $k+2$ is true, that is, $f(k+3, k+2) = 2$.
    \begin{align*}
        f(k+3, k+2) &= f(k+1, f(k+1, k+2))\\
        &= f(k+1, f(k+1, f(k+1, k)))\\
        &= f(k+1, f(k+1, 2)) & \text{(by hypothesis)}\\
        &= f(k+1, f(1, f(1, k)))\\
        &= f(k+1, f(1, k))\\
        &= f(k+1, k)\\
        &= 2 & \text{(by hypothesis)}
    \end{align*}
    which proves that the statement for $k+2$ holds. Hence by mathematical induction, $f(n+1, n) = 2$.
\end{example}

\begin{exercise}
    Let $S$ be a set. The \textbf{power set} of $S$, denoted $\mathcal{P}(S)$, is the set that contains all subsets of $S$.
    \begin{enumerate}[label=(\roman*)]
        \item Determine whether the following statements are true.
        \begin{enumerate}[label=(\alph*)]
            \item $S \in \mathcal{P}(S)$
            \item $\emptyset \in \mathcal{P}(S)$
        \end{enumerate}
        \item Suppose $|S| = n$. Prove by induction that $|\mathcal{P}(S)| = 2^n$ for all sets $S$.
    \end{enumerate}
\end{exercise}

%=========================================
\chapter{Functions / Mappings}
A function (or a map) from a set $X$ to a set $Y$ is an assignment of an element of $Y$ to each element of $X$. The set $X$ is called the \textbf{domain} of the function and the set $Y$ is called the \textbf{codomain} of the function.

A function, its domain, and its codomain, are declared by the notation $f: X\to Y$, and the value of a function $f$ at an element $x$ of $X$, denoted by $f(x)$, is called the \textbf{image} of $x$ under $f$.

The \textbf{image} or \textbf{range} of a function is the set of the images of all elements in the domain. The image and the codomain are not necessarily the same. The image of the function $f: X \to Y$ is denoted either as $\im f$ or $f(X)$.

Two functions $f$ and $g$ are equal if their domain and codomain sets are the same and their output values agree on the whole domain. More formally, given $f: X \to Y$ and $g: X \to Y$, $f = g$ if $f(x) = g(x)$ for all $x$ in $X$.

\textbf{Arrow notation} can also be used to define the rule of a function. For example, $f: \mathbb{R} \to \mathbb{R}, x \mapsto x+1$ is the function which takes a real number as input and outputs that number plus 1. In the above example, ``$x + 1$'' is the \textbf{rule} of the function.

\begin{exercise}
    Let a function $f: \{1, 2, 3\} \to \{1, 4, 9, 16, 25\}$ be given by the relation $f(x) = x^2$.
    \begin{enumerate}[label=(\roman*)]
        \item Use arrow notation to write a definition for $f$.
        \item State the domain, codomain, and range of $f$.
        \item Is the function $g: \{1, 2, 3\} \to \{1, 8\}, x \mapsto x^3$ \textit{valid}?
    \end{enumerate}
\end{exercise}

\newpage

A function is said to be \textbf{well-defined} if \textit{similar inputs} produce \textit{identical outputs}. More formally, a function $f: A \to B$ is well defined if for each $a \in A$ there is a unique $b \in B$ such that $f(a) = b$. An example may help to illustrate this point.
\begin{example}
    Let $S_1$ and $S_2$ be sets, and let $S = S_1 \cup S_2$. Let $f: S \to \{1, 2\}$, such that $f(x) = 1$ if $x \in S_1$ and $f(x) = 2$ if $x \in S_2$. Then $f$ is well-defined if $S_1 \cap S_2 = \emptyset$. For example, if $S_1 = \{1, 2\}$ and $S_2 = \{3, 4\}$, then $f$ is well-defined. On the other hand, if $S_1 \cap S_2 \neq \emptyset$, then $f$ is not well-defined. For example, if $S_1 = \{1, 2\}$ and $S_2 = \{2, 3\}$, then $f(2) = 1$ and $f(2) = 2$ simultaneously.
\end{example}
\begin{exercise}
    Is the function
    \[
        f: \mathbb{Q} \to \mathbb{Z}, \frac pq \mapsto p + q    
    \]
    well-defined?
\end{exercise}

Functions may also be \textbf{composed} with each other. This is done by using the function composition operator $\circ$. The composition of two functions, say $f: X \to Y$ and $g: Y \to Z$, produces a function $f \circ g: X \to Z$ such that $(f \circ g)(x) = f(g(x))$. It is important to note the following:
\begin{itemize}
    \item Function composition is associative. That is, if $f$, $g$, and $h$ are composable, then $f \circ (g \circ h) = (f \circ g) \circ h$. Since the parentheses do not change the result, they are generally omitted.
    \item The composition $f \circ g$ is only meaningful if the codomain of $g$ is a subset of the domain of $f$. That is, if $f: A \to B$ and $g: C \to D$, then $f \circ g$ is only meaningful if $\im g \subseteq A$.
\end{itemize}
We may also alternatively write $fg$ in place of $f \circ g$.

\begin{exercise}
    Let $f: \mathbb{R} \to \mathbb{R}$ and $g: \mathbb{R} \to \mathbb{R}$. Write down the rule of the function $fg$ if $f(x) = x^2 - x + 1$ and $g(y) = \frac1{y^2+1}$.
\end{exercise}

A function $f: X \to Y$ is said to be \textbf{injective} (or \textbf{one-to-one}) if $f(x_1) = f(x_2)$ implies $x_1 = x_2$. Equivalently, if $x_1 \neq x_2$ then $f(x_1) \neq f(x_2)$ for all $x_1$ and $x_2$ in $X$.

A function $f: X \to Y$ is said to be \textbf{surjective} (or \textbf{onto}) is a function whose image is equal to its codomain. Equivalently, $f$ is surjective if and only if for all $y \in Y$, there exists $x \in X$ such that $f(x) = y$. In this case, $x$ is said to be the \textbf{pre-image} of $y$.

A function is \textbf{bijective} if it is both injective and surjective. A bijective function is also called a \textbf{bijection} or a \textbf{one-to-one correspondence}. A function is bijective if and only if every possible image is mapped to by exactly one argument. It should be noted that bijectivity is implied if the function $f: X \to Y$ is injective and the sets $X$ and $Y$ `have the same number of elements' (or more formally, \textbf{equinumerous}).

\begin{exercise}
    Let the set $S = \{x \in \mathbb{Z} \vert x > 0\}$. Define the function $f: S \to \mathbb{Z}$ such that
    \[
        f(x) = \begin{cases}
            \frac{x}{2} & \text{ if } x \text{ is even}\\
            \frac{1-x}{2} & \text{ if } x \text{ is odd} 
        \end{cases}
    \]
    Prove that $f$ is a bijection.
\end{exercise}

%=========================================
\chapter{Number Theory}
\section{Divisibility}
We say that \textbf{$a$ divides $b$} if there is an integer $k$ such that $ak = b$. This is denoted $a\;|\;b$. For example, $7\;|\;63$ since $7 \times 9 = 63$. However, 8 does not divide 63, so we write $8 \nmid 63$. A consequence of this definition is that every number divides zero since $a \times 0 = 0$ for every integer $a$.

If $a$ divides $b$, then $b$ is a \textbf{multiple of $a$}. For example, 63 is a multiple of 7, but 63 is not a multiple of 8.

Here are some basic facts about divisibility that are not difficult to prove.
\begin{itemize}
    \item If $a\;|\;b$ then $a\;|\;bc$ for all integers $c$.
    \item If $a\;|\;b$ and $b\;|\;c$ then $a\;|\;c$.
    \item If $a\;|\;b$ and $a\;|\;c$ then $a\;|\;sb+tc$ for all integers $s$ and $t$.
    \item If $c \neq 0$, then $a\;|\;b$ if and only if $ac\;|\;bc$.
\end{itemize}

A number $p > 1$ with no positive divisors other than 1 and itself is called a \textbf{prime}. Every other number greater than 1 is called \textbf{composite}. For example, 2, 3, 5, 7, 11, and 13 are all prime, but 4, 6, 8, and 9 are composite. The number 1 is considered neither prime nor composite.

\begin{theorem}[Fundamental Theorem of Arithmetic]\label{thrm-fundamental-theorem-of-arithmetic}
    For every integer $n > 1$, $n$ can be expressed as the product of one or more prime numbers, uniquely up to the order in which they appear.
\end{theorem}

\begin{exercise}
    Express 44100 as a product of primes.
\end{exercise}

\section{Euclid's Division Lemma}
When dividing an integer by another, we may sometimes have a \textbf{remainder} left over. For example, when 63 is divided by 8, a remainder of 7 is left over. More precisely, if you divide $n$ by $d$, then you get a quotient $q$ and a remainder $r$. This is given by Euclid's Division Lemma.
\begin{lemma}[Euclid's Division Lemma]
    Given two integers $n$ and $d$, with $n \neq 0$, there exist unique integers $q$ and $r$ such that
    \[
        n = qd + r
    \]
    and $0 \leq r < |d|$ where $|d|$ denotes the absolute value of $d$.
\end{lemma}
\begin{remark}
    We will often call this lemma \textbf{Euclid's division algorithm}, or \textbf{the division algorithm}.
\end{remark}

\begin{example}
    Using $n = 63$ and $d = 8$, we will have $63 = 7\times8 + 7$.
\end{example}
\begin{example}
    Using $n = 14$ and $d = -3$, we will have $13 = -5\times3 + 1$.
\end{example}

\begin{exercise}
    Express $-210$ in the form $a-13b$, where $a$ and $b$ are positive integers with $0 \leq a \leq 12$.
\end{exercise}

\section{Greatest Common Divisor (GCD) and Lowest Common Multiple (LCM)}
In number theory, the idea of a greatest common divisor and the least common multiple are omnipresent.

\begin{definition}
    Let $m$ and $n$ be two non-zero integers. Then an integer $d$ is said to be the \textbf{greatest common divisor} of $m$ and $n$ if $m = pd$ and $n = qd$ for some integers $p$ and $q$, and that $d$ is the largest possible integer that achieves this.
\end{definition}
The greatest common divisor (GCD) of $m$ and $n$ is denoted by $\gcd(m, n)$.

\begin{example}
    $\gcd(2, 8) = 2$ since $2 = 1 \times 2$ and $8 = 4 \times 2$.
\end{example}
\begin{example}
    $\gcd(42, 231) = 21$ since $42 = 2 \times 21$ and $231 = 11 \times 21$.
\end{example}
\begin{example}
    $\gcd(-10, 25) = 5$ since $-10 = -2 \times 5$ and $25 = 5 \times 5$.
\end{example}
\begin{exercise}
    Find $\gcd(-112, -35)$.
\end{exercise}

\begin{remark}
    If $\gcd(m, n) = 1$ for non-zero integers $m$ and $n$, then $m$ and $n$ are said to be \textbf{coprime} to each other.
\end{remark}
\begin{example}
    17 and 18 are coprime to each other as $\gcd(17, 18) = 1$.
\end{example}

We now look at the lowest common multiple of two integers.
\begin{definition}
    Let $m$ and $n$ be two non-zero integers. Then an integer $l$ is said to be the \textbf{lowest common multiple} of $m$ and $n$ if $m = pl$ and $n = ql$ for some integers $p$ and $q$, and that $l$ is the smallest possible \textbf{positive} integer that achieves this.
\end{definition}
The lowest common multiple (LCM) of $m$ and $n$ is denoted by $\lcm(m,n)$.

\begin{example}
    $\lcm(2, 8) = 8$ since $8 = 4 \times 2$ and $8 = 1 \times 8$.
\end{example}
\begin{example}
    $\gcd(42, 231) = 462$ since $462 = 11 \times 42$ and $462 = 2 \times 231$.
\end{example}
\begin{example}
    $\lcm(-10, 25) = 50$ since $50 = 5 \times -10$ and $50 = 2 \times 25$.
\end{example}
\begin{exercise}
    Find $\lcm(-112, -35)$.
\end{exercise}

\newpage

We note several properties of the GCD and LCM.
\begin{lemma}[B\'{e}zout]\label{lemma-bezout}
    Let $m$ and $n$ be non-zero integers such that $\gcd(m, n) = d$. Then there exist integers $x$ and $y$ such that $mx + ny = d$. Moreover, the integers of the form $am + bn$ (where $a$ and $b$ are integers) are multiples of $d$.
\end{lemma}
\begin{proposition}
    Let $m$ and $n$ be non-zero integers. Then
    \[
        |mn| = \gcd(m,n) \times \lcm(m,n)    
    \]
    where $|mn|$ denotes the absolute value of $mn$.
\end{proposition}

\begin{exercise}
    Suppose $m = 42$ and $n = 70$.
    \begin{enumerate}[label=(\roman*)]
        \item Let $d = \gcd(m,n)$. Find $d$.
        \item Hence find $\lcm(m,n)$.
        \item Find a pair of integers $x$ and $y$ such that $mx + ny = d$.
    \end{enumerate}
\end{exercise}

%=========================================
\chapter{Modular Arithmetic}
\section{Modulo and Modular Congruence}

Given an integer $n>1$, called a \textbf{modulus}, two integers $a$ and $b$ are said to be \textbf{congruent modulo $n$} if $n$ is a divisor of their difference (that is, if there is an integer $k$ such that $a - b = kn$). It is denoted $a \equiv b \pmod{n}$. The parentheses mean that $\pmod{n}$ applies to the entire equation, not just to the right-hand side (here, $b$). This notation is not to be confused with the notation $b \mod n$ (without parentheses), which refers to the modulo operation. \begin{remark}
    Equivalently, $a \equiv b \pmod n$ means that $a = kn + b$ for some integer $k$.
\end{remark}
\begin{example}
    In modulus 12, one sees that $38 \equiv 14 \pmod{12}$ since $38 - 14 = 24 = 2 \times 12$. Another way to express this is to say that both 38 and 14 have the same remainder 2, when divided by 12.
\end{example}
\begin{exercise}
    Let $m = 5$ and $n = 3$.
    \begin{enumerate}[label=(\alph*)]
        \item State the value of $17 \mod m$.
        \item Find the value of $x$ such that $19 \equiv x \pmod m$ and $0 \leq x < m$.
        \item If $A = 1234n + 5$, what is $A \mod n$?
    \end{enumerate}
\end{exercise}

\newpage

The definition of congruence also applies to negative values. For example:
\begin{itemize}
    \item $-3 \equiv 2 \pmod5$
    \item $-8 \equiv 7 \equiv 2 \pmod5$
    \item $-1 \equiv n-1 \pmod{n}$
\end{itemize}

\begin{exercise}
    Explain why $-n \equiv n \pmod{2n}$.
\end{exercise}

The operation of congruence modulo $n$ has a few properties which we will state without proof. Let $a_1 \equiv b_1 \pmod n$ and $a_2 \equiv b_2 \pmod n$, or if $a \equiv b \pmod n$, then
\begin{itemize}
    \item $a + k \equiv b + k \pmod n$ for any integer $k$;
    \item $ka \equiv kb \pmod n$ for any integer $k$;
    \item $ka \equiv kb \pmod {kn}$ for any integer $k$;
    \item $a_1 + a_2 \equiv b_1 + b_2 \pmod n$;
    \item $a_1 - a_2 \equiv b_1 - b_2 \pmod n$;
    \item $a_1a_2 \equiv b_1b_2 \pmod n$;
    \item $a^k \equiv b^k \pmod n$ for any non-negative integer $k$;
    \item if $a+k \equiv b+k \pmod n$ then $a \equiv b \pmod n$ for any integer $k$;
    \item if $ka \equiv kb \pmod n$ and $k$ is an integer such that $\gcd(k, n) = 1$, then $a \equiv b \pmod n$; and
    \item if $ka \equiv kb \pmod{kn}$ where $k$ is a non-zero integer, then $a \equiv b \pmod n$.
\end{itemize}

\begin{exercise}
    Find the last two digits of $778899^{112233}$.
\end{exercise}

\section{Modular Multiplicative Inverse}
\begin{definition}
    Let $m$ be a positive integer, and let $a$ be an integer. Then an integer $x$ that makes $ax \equiv 1 \pmod m$ is said to be the \textbf{modular multiplicative inverse of $a$}.
\end{definition}
\begin{remark}
    In these books, however, we refer to the modular multiplicative inverse as just \textbf{multiplicative inverse}.
\end{remark}
\begin{example}
    4 is the multiplicative inverse of 7 modulo 9 since $4 \times 7 = 28 = 3 \times 9 + 1 \equiv 1 \pmod 9$.
\end{example}

\begin{proposition}\label{prop-multiplicative-inverse-exists-iff-coprime}
    A multiplicative inverse of $a$ modulo $m$ exists if and only if $\gcd(a,m) = 1$.
\end{proposition}
\begin{proof}
    We first work forwards and suppose $k$ is the multiplicative inverse of $a$ modulo $m$. Then $ka \equiv 1 \pmod m$. Hence $ka - 1 \equiv 0 \pmod m$, so $m$ divides $ka - 1$. This means that $ka - 1$ is a multiple of $m$, so $ka - 1 = pm$ for some integer $p$. Therefore $ka + pm = 1$ By B\'{e}zout's Lemma (\textbf{Lemma \ref{lemma-bezout}}) this means that $\gcd(a, m) = 1$.
    
    Now, working in the reverse direction, suppose $\gcd(a, m) = 1$. By B\'{e}zout's Lemma this means there exist integers $k$ and $p$ such that $ka + pm = 1$. Then $ka - 1 = pm$ which means $m$ divides $ka - 1$, which hence means $ka - 1 \equiv 0 \pmod m$ which the result quickly yields.
\end{proof}

\begin{example}
    The number 20 has a multiplicative inverse modulo 31 since $\gcd(20, 31) = 1$. One can verify that 14 is the multiplicative inverse of 20 modulo 31.
\end{example}

\begin{exercise}
    Find the modular multiplicative inverse of 123 modulo 5.
\end{exercise}

%=========================================
\addcontentsline{toc}{chapter}{Exercise Solutions}
\chapter*{Exercise Solutions}

\addcontentsline{toc}{section}{Chapter 1}
\section*{Chapter 1}
\begin{enumerate}
    \item \begin{enumerate}[label=(\alph*)]
        \item False, 3 does not appear in $\{1, 2, 4\}$.
        \item True. Any set is a subset of itself, including the empty set.
        \item False, the set $S$ does not contain any element that is not in $S$. That is, $S \subseteq S$ but not $S \subset S$.
        \item True. $S$ is indeed an element of $\{S, \emptyset\}$.
        \item True. The set containing S is not an element of $\{S, \emptyset\}$.
        \item False, the set $S$ is not a subset of the set $\{S, \emptyset\}$.
        \item True. The set containing $S$ is a subset of the set containing $S$ and the empty set.
    \end{enumerate}
    
    \item \begin{enumerate}[label=(\alph*)]
        \item True.
        \item False, $S \cup U = \{1, 2, 3, 4, (2, 2), (3, 3), (5, 5)\}$.
        \item True.
        \item True.
        \item True.
        \item False, $S \setminus \{1, 4\} = \{2, 3\}$, not $T = \{2, 3, 5\}$.
        \item True.
        \item True. $(S \cup T)^2 = \{(1,1), (2,2), (3,3), (4,4), (5,5)\}$, so $U = \{(2,2), (3,3), (5,5)\} \subset (S \cup T)^2$.
    \end{enumerate}
    \item We note $\{x \in \mathbb{Q} \vert x \leq 0\}$ are all the non-positive rational numbers, and the second set has elements $\{-2, 0, 2, \dots, 8, 10\}$. Hence their intersection has only two elements, namely $-2$ and $0$.
\end{enumerate}

\addcontentsline{toc}{section}{Chapter 2}
\section*{Chapter 2}
\begin{enumerate}
    \item We work from the inner-most bracket outwards. We note $P$ is true, $Q$ is false, and $R$ is true.
    \begin{itemize}
        \item $P \lor Q$ is ``1 is a positive number \textbf{or} $-1 > 0$'', which is true since $P$ is true.
        \item $(P \lor Q) \land R$ is ``(1 is a positive number or $-1 > 0$) \textbf{and} 1 is an odd number'', which is true since $P \lor Q$ is true and 1 is, indeed, an odd number.
        \item $\lnot((P \lor Q) \land R)$ is false, since $(P \lor Q) \land R$ is true.
    \end{itemize}
    Hence the statement ``$\lnot((P \lor Q) \land R)$ is false'' is a true statement.
    
    \item The truth table for $P \land (\lnot Q)$ is given below:
    \begin{table}[h]
        \centering
        \begin{tabular}{|l|l||l|}
            \hline
            $P$ & $Q$ & $P\land (\lnot Q)$ \\ \hline
            F   & F   & F                  \\ \hline
            F   & T   & F                  \\ \hline
            T   & F   & T                  \\ \hline
            T   & T   & F                  \\ \hline
        \end{tabular}
    \end{table}
    
    \item \begin{enumerate}[label=(\roman*)]
        \item $R$: $n$ is a multiple of 5 if and only if the last digit of $n$ is 0 or 5.
        \item If $n$ is a multiple of 5, then its last digit necessarily has to be 5 or 0, hence $P \implies Q$. If the last digit is 5 or 0, then the number $n$ is a multiple of 5, hence $Q \implies P$. Therefore $P \iff Q$.
    \end{enumerate}
    
    \item For brevity, let $R = (P \implies (\lnot Q))$ and $S = ((\lnot Q) \implies P)$. So we want to show that $((\lnot P) \iff Q) = R \land S$.
    \begin{table}[h]
\centering
\begin{tabular}{|l|l||l|l|l|l||l|l|}
\hline
$P$ & $Q$ & $\lnot P$ & $\lnot Q$ & $R$ & $S$ & $R \land S$ & $(\lnot P) \iff Q$ \\ \hline
F   & F   & T         & T         & T   & F   & F           & F                  \\ \hline
F   & T   & T         & F         & T   & T   & T           & T                  \\ \hline
T   & F   & F         & T         & T   & T   & T           & T                  \\ \hline
T   & T   & F         & F         & F   & T   & F           & F                  \\ \hline
\end{tabular}
\end{table}

    From inspection, the truth tables of $(\lnot P) \iff Q$ and $R \land S$ are the same, proving our required result.
    
    \item For brevity, denote $X = P \lor \lnot Q$, $Y = \lnot R$, and $Z = (P \lor R) \land (P \lor \lnot R)$. Thus the original statement is something like $(X \land Y) \lor (X \land Z)$, which by distributive rules is equal to $X \land (Y \lor Z)$. Note $Z = P \lor (R \land \lnot R)$ by distributive rules, which is equal to $P$ since $R \land \lnot R$ is always false. Hence $X \land (Y \lor Z) = (P \lor \lnot Q) \land (\lnot R \lor P)$. Commutativity of $\lor$ means that $\lnot R \lor P = P \lor \lnot R$, so $(P \lor \lnot Q) \land (\lnot R \lor P) = (P \lor \lnot Q) \land (P \lor \lnot R)$. Now by distributive rules $(P \lor \lnot Q) \land (P \lor \lnot R) = P \lor (\lnot Q \land \lnot R)$. Finally, by De Morgan's Law, $\lnot Q \land \lnot R = \lnot(Q \lor R)$, so the original statement is equal to $P \lor \lnot(Q \lor R)$.
\end{enumerate}

\addcontentsline{toc}{section}{Chapter 3}
\section*{Chapter 3}
\begin{enumerate}[label=(\roman*)]
    \item \begin{enumerate}[label=(\alph*)]
        \item True. $S$ is a subset of $S$, so $S \in \mathcal{P}(S)$.
        \item True. $\emptyset$ is a subset of any set, so $\emptyset \in \mathcal{P}(S)$.
    \end{enumerate}
    \item
    \begin{proof}
        We induct on the size of $S$.
        
        Note that the smallest size a set can be is 0, which means that $S$ is the empty set. When $n = 0$, then the only subset of $S$ is the empty set, so $\mathcal{P}(S) = \mathcal{P}(\emptyset) = \{\emptyset\}$, meaning that $\mathcal{P}(\emptyset)$ has 1 element. Hence $|\mathcal{P}(S)| = 1 = 2^0$ in the case where $n = 0$.
        
        Now assume that the statement holds for some non-negative integer $k$, meaning that if $|S| = k$ then $|\mathcal{P}(S)| = 2^k$. We want to prove that the statement holds if $|S| = k+1$.
        
        Consider a set $S$ with $k + 1$ elements. Since all elements in $S$ are distinct (by definition of a set), pick any element in $S$ and call it the `first element', and call the remaining elements `the rest'. Now any subset of $S$ can either contain the `first element' or not. `The rest' has a power set of order $2^k$ by our inductive hypothesis. Hence, the power set of $S$ has order $2 \times 2^k = 2^{k+1}$, any subset of $S$ can either contain the `first element' or not. This proves the case for $k+1$, and therefore concludes the proof that $|\mathcal{S}(P)| = 2^{n}$ if $|S| = n$.
    \end{proof}
\end{enumerate}

\addcontentsline{toc}{section}{Chapter 4}
\section*{Chapter 4}
\begin{enumerate}
    \item \begin{enumerate}[label=(\roman*)]
        \item $f: \{1, 2, 3\} \to \{1, 4, 9, 16, 25\}, x \mapsto x^2$. (Or just $x \mapsto x^2$)
        \item Domain is $\{1, 2, 3\}$, codomain is $\{1, 4, 9, 16, 25\}$, range is $\{1, 4, 9\}$.
        \item No. The element 3 would map to 27, which is not in the codomain.
    \end{enumerate}
    \item It is not well-defined. Note $\frac 12 = \frac 24$, but $f(\frac12) = 1 + 2 = 3$ and $f(\frac24) = 2 + 4 = 6$.
    \item $fg(x) = \left(\frac1{x^2+1}\right)^2 - \frac1{x^2+1} + 1$.
    \item We prove the requirements of a bijection one by one.
    \begin{itemize}
        \item \textbf{Injective}: Suppose $x_1, x_2 \in S$ such that $f(x_1) = f(x_2)$. We split into three cases.
        \begin{itemize}
            \item The first case is if $f(x_1) = f(x_2) = 0$. In this case, one sees clearly that $x_1 = x_2 = 1$.
            \item The second case is if $f(x_1) = f(x_2) > 0$. Now since $x \neq 1$ (as this case leads to $f(x_1) = 0$), the `valid' odd numbers are at least 3. Therefore, $\frac{1-x}{2} \leq \frac{1-3}{2} = -1 < 0$, so $x_1$ and $x_2$ cannot be odd. Hence, $x_1$ and $x_2$ are even, meaning $\frac{x_1}{2} = \frac{x_2}{2}$ which quickly implies $x_1 = x_2$.
            \item The third case is if $f(x_1) = f(x_2) < 0$. As argued above, this means that $x_1$ and $x_2$ must be odd numbers of at least 3. Hence, $\frac{1-x_1}{2} = \frac{1-x_2}{2}$ which quickly implies $x_1 = x_2$.
        \end{itemize}
        Thus, in all three cases, $f(x_1) = f(x_2)$ implies $x_1 = x_2$, meaning $f$ is injective.
        \item \textbf{Surjective}: Suppose $y \in \mathbb{Z}$. We split into three cases again.
        \begin{itemize}
            \item If $y = 0$, then setting $x = 1$ satisfies $f(x) = y$.
            \item Now suppose $y > 0$. We note $2y \in S$, and clearly $2y$ is an even integer. So setting $x = 2y$ satisfies $f(x) = \frac{2y}{y} = y$.
            \item Suppose $y < 0$. Note $-2y > 0$, and $1 - 2y > 0 \in S$. Furthermore $1 - 2y$ is clearly an odd integer. Hence setting $x = 1 - 2y$ satisfies $f(x) = \frac{1-(1-2y)}{2} = y$.
        \end{itemize}
        Therefore for every $y \in \mathbb{Z}$, there exists a pre-image $x \in S$ such that $f(x) = y$. Hence $f$ is surjective.
    \end{itemize}
    Therefore, as $f$ is both injective and surjective, $f$ is bijective.
\end{enumerate}

\addcontentsline{toc}{section}{Chapter 5}
\section*{Chapter 5}
\begin{enumerate}
    \item $44100 = 2^2 \times 3^2 \times 5^2 \times 7^2$.
    \item $-210 = 11 - 13 \times 17$, so $a = 11$ and $b = 17$.
    \item $\gcd(-112, -35) = 7$ since $-112 = -16 \times 7$ and $-35 = -5 \times 7$, with 7 being the largest integer that achieves this.
    \item $\lcm(-112, -35) = 560$ since $560 = -5 \times -112$ and $-35 = -16 \times -35$, with 560 being the smallest \textit{positive} integer that achieves this.
    \item \begin{enumerate}[label=(\roman*)]
        \item $\gcd(42, 70) = 14$ since $42 = 3 \times 14$ and $70 = 5 \times 14$, and 14 is the largest integer achieving this.
        \item $\lcm(42, 70) = \frac{42 \times 70}{\gcd(m, n)} = \frac{2940}{14} = 210$.
        \item Note that $x = 2$ and $y = -1$ works as $42 \times 2 + 70 \times (-1) = 84 - 70 = 14$.
    \end{enumerate}
\end{enumerate}

\addcontentsline{toc}{section}{Chapter 6}
\section*{Chapter 6}
\begin{enumerate}
    \item \begin{enumerate}[label=(\alph*)]
        \item $17 \mod 5 = 2$ since $17 = 3 \times 5 + 2$ by the division algorithm.
        \item As $19 = 3 \times 5 + 4$, thus $19 \equiv 4 \pmod 5$. Hence $x = 4$.
        \item $A \mod n = 5 \mod 3 = 2$.
    \end{enumerate}
    \item $-n = (-1) \times 2n + n$, which means that $-n \equiv n \pmod{2n}$.
    \item Finding the last two digits of a number is the same as finding the remainder of that number when divided by 100. We note $778899 \equiv 99 \pmod{100}$, so $778899^{112233} \equiv 99^{112233} \pmod{100}$. Furthermore, $99 \equiv -1 \pmod{100}$, so $99^{112233}\equiv (-1)^{112233} \equiv -1 \equiv 99 \pmod{100}$. Hence the last two digits of $778899^{112233}$ is 9 and 9.
    \item Note $123 \equiv 3 \pmod 5$. One can easily find by trial and error that 2 is the multiplicative inverse of 3, since $3 \times 2 = 6 \equiv 1 \pmod 5$. Hence the multiplicative inverse of 123 is 2 modulo 5.
\end{enumerate}

%=========================================
\addcontentsline{toc}{chapter}{Image Acknowledgements}
\chapter*{Image Acknowledgements}
Images are cited as they appear.

%=========================================
\printbibliography[heading=bibintoc, title={References and Bibliography}]

\end{document}
