\documentclass[
    a5paper,
    pagesize,
    11pt,
    bibtotoc,
    normalheadings,
    twoside,
    openany,
    chapterprefix,
    DIV=9
]{scrbook}

\usepackage[utf8]{inputenc}
\usepackage{tocloft}
\usepackage{mathtools}
\usepackage{amsfonts}
\usepackage{enumitem}
\usepackage{amsmath}
\usepackage{amsthm}
\usepackage{amssymb}
\usepackage[hmargin=2cm, vmargin=2.5cm]{geometry}
\usepackage{graphicx}
\usepackage{wrapfig}
\usepackage{parskip}
\usepackage{framed}
\usepackage{fancyhdr}
\usepackage{emptypage}
\usepackage{multicol}
\usepackage{imakeidx}
\usepackage[breaklinks]{hyperref}
\usepackage[capitalise, nameinlink]{cleveref}
\usepackage{crossreftools}

\usepackage[
    backend=bibtex,
    style=alphabetic,
    sorting=ynt
]{biblatex}

%=========== Path to images ==============
\graphicspath{{./images/}}

%============== Resources ================
\addbibresource{../AbstractAlgebra.bib}

%============ Redefinitions ==============
\let\oldemptyset\emptyset
\let\emptyset\varnothing

\let\totient\varphi

\renewcommand{\vert}{ \ \vline \ }
\newcommand{\vertalt}{ \ | \ }

\newcommand{\myref}[1]{\textbf{\crthypercref{#1}}}
\newcommand{\myreffigures}[1]{\textbf{\cref{#1}}}

%======== Theorem-Like Things ============
\newtheoremstyle{exercise-style}
    {-5pt}       % Space above
    {\topsep}    % Space below
    {}           % Font to use in exercise
    {0pt}        % Measure of space to indent
    {\bfseries}  % Name of the head font
    {.}          % Punctuation between head and body
    { }          % Space after theorem head; " " = normal inter-word space
    {\thmname{#1}\thmnumber{ #2}\textnormal{\thmnote{ (#3)}}}

\newtheorem{theorem}{Theorem}[section]
\renewcommand{\thetheorem}{\Roman{part}.\arabic{chapter}.\arabic{section}.\arabic{theorem}}

\newtheorem{conjecture}[theorem]{Conjecture}
\newtheorem{proposition}[theorem]{Proposition}
\newtheorem{definition}[theorem]{Definition}
\newtheorem{lemma}[theorem]{Lemma}
\newtheorem{corollary}[theorem]{Corollary}
\theoremstyle{definition}\newtheorem*{remark}{Remark}
\theoremstyle{definition}\newtheorem{example}[theorem]{Example}

\theoremstyle{exercise-style}\newtheorem{exercisehidden}{Exercise}[chapter]
\renewcommand{\theexercisehidden}{\Roman{part}.\arabic{chapter}.\arabic{exercisehidden}}

\theoremstyle{definition}\newtheorem{problem}{Problem}[chapter]
\renewcommand{\theproblem}{\Roman{part}.\arabic{chapter}.\arabic{problem}}

%============ Environments ===============
\newenvironment{exercise}
{\begin{framed}\noindent\begin{exercisehidden}}
{\end{exercisehidden}\end{framed}}

%=========== Custom Commands =============
\newcommand{\code}[1]{\texttt{#1}}  % Code block
\makeatletter\newcommand*{\rom}[1]{\Ifstr{#1}{0}{0}{\expandafter\@slowromancap\romannumeral #1@}}\makeatother  % Roman numeral

\newcommand{\lcm}{\mathrm{lcm}}  % Lowest common multiple function
\newcommand{\sgn}{\mathrm{sgn}}  % Signum function

\newcommand{\im}{\mathrm{im}\;}  % Image of a function
\newcommand{\id}{\mathrm{id}}    % Identity function

%======== Custom Chapter Styling =========
\makeatletter
\renewcommand{\chaptermark}[1]{
    \markboth{\if@mainmatter\chaptername~\thechapter.\ \fi#1}{}
}

\renewcommand*{\chapterformat}{
  \MakeUppercase{\chapapp\nobreakspace\thechapter}
}

\renewcommand*{\chapterlineswithprefixformat}[3]{
    \Ifstr{#1}{chapter}{
        \vspace{-60px}
        \Ifstr{#2}{\empty}{\vspace{40px}}{\raggedleft#2}
        \vspace{-15px}
        \rule{\linewidth}{1pt}\par\nobreak
        \centering{#3}
        \vspace{-10px}
        \rule{\linewidth}{1pt}\par\nobreak
        \vspace{-10px}
    }{#2#3}
}
\makeatother

%============ Custom Header ==============
\fancypagestyle{plain}{\fancyhf{}\renewcommand{\headrulewidth}{0pt}}  % To clear page numbers from footer, and header line at the start of every chapter

\pagestyle{fancy}
\fancyhf{}  % Clear header/footer

\fancyhead[LE,RO]{\thepage}
\fancyhead[LO,RE]{\textit{\nouppercase\leftmark}}

%========= Customise TOC Heading =========
\makeatletter
\def\createtoc{
    \renewcommand\tableofcontents{
        \chapter*{\contentsname}
        \@starttoc{toc}
    }
    \tableofcontents
}
\makeatother

%========= Front Matter Pages ============
\def\volumetitle{Volume \rom{\volumenumber}: \volumename}

\def\frontmatterpages{
    \frontmatter  % Use lowercase roman numerals for page numbers

    % Title page
    \begin{titlepage}
        \centering{
            \selectfont
            \Huge
            \textbf{Abstract Algebra}\\
            \vspace{-0.2cm}
            
            \Large
            \textbf{A Simple Introduction}\\
            \vspace{0.5cm}
            
            \LARGE
            \volumetitle
            \vspace{2cm}
        }\\
        \centering{\Large{Overwrite}}
        \vspace{\fill}

        \includegraphics[width=5cm]{\volumeimage}
        \vspace{\fill}

        \centering \small{\textit{Version \version}}
    \end{titlepage}

    \newpage{}

    % Edition notice
    \clearpage\null\vfill
    \thispagestyle{empty}
    \begin{minipage}[b]{0.9\textwidth}
        \footnotesize\raggedright
        \setlength{\parskip}{0.5\baselineskip}

        Published by Kan Onn Kit\\
        Singapore
        \vspace{5cm}

        \textbf{Abstract Algebra: A Simple Introduction -- \volumetitle}\par
        Version \version
        \vspace{0.3cm}

        Copyright \copyright \ 2022 -- \the\year\ by Kan Onn Kit\par
        This work is licensed under a
        Creative Commons Attribution-NonCommercial-ShareAlike 4.0 International Licence.\par
        \includegraphics[width=2.5cm]{../Images/CC BY-NC-SA 4.0.png}\\  % With reference to the volumes' folders
        The full licence text is available at \url{http://creativecommons.org/licenses/by-nc-sa/4.0/}.\par    
        The source files for the project are available \href{https://github.com/PhotonicGluon/Abstract-Algebra-Book}{here}.
        \vspace{0.3cm}

        Typeset in 11pt Computer Modern Roman using PDF\LaTeX.
    \end{minipage}

    \vspace*{2\baselineskip}
    \cleardoublepage

    % "Quote" page
    \thispagestyle{empty}
    \vspace*{2cm}

    \begin{center}
        \Large{\parbox{10cm}{
            \begin{raggedright}
                \Large
                \quotepagetext
                \vspace{0.3cm}
                
                \hfill
                --- \quotepageattribution\\
                \vspace{-0.25cm}
                
                \hfill
                \normalsize
                (\quotepagecitation)
            \end{raggedright}
        }
    }
    \end{center}

    \newpage

    % Table of contents
    \createtoc
    \setcounter{part}{\volumenumber}

    % Preface
    \chapter{Preface}
    Although algebra has a long history, it has undergone some quite striking changes in the past few decades. Abstract algebra is widely recognised as an essential element of higher mathematical education. The results and theorems that it showcases, however, are oft hard to grasp and understand without prerequisite knowledge or with a heavy background in mathematics. Most books on this subject are crafted for undergraduates at universities. They are not for a general mathematics enthusiast or one who seeks to understand more about the inner structure of algebra that many mathematicians encounter frequently.

    It is thus the goal of this series of books to provide a step-by-step explanation of core results from abstract algebra; to demystify the core steps that many textbooks skip over when writing proofs. I aim to ensure that the results from such an essential field of study are as accessible, as approachable, and as understandable for as many people as possible.

    \section*{Preface for Volume \rom{\volumenumber}}
    \prefacevolumetext

    \hfill{\textit{27 January, 2023}}

    % Suggestions on the use of this book
    \chapter{Suggestions on the Use of This Book}
    \section*{General Information}
    \begin{itemize}
        \item For most volumes, we include both exercises and problems.
        \begin{itemize}
            \item An exercise can be thought of as a simple ``self-review'' question. Exercises ensure that the content of a particular section is understood and should not be too hard to answer.
            \item A problem is a more holistic version of an exercise. Generally, solutions to problems require a thorough understanding of the current chapter and may require results from other chapters.
        \end{itemize}
        \item A consistent labelling system for all the results within and between volumes is necessary for a project as long as this one.
        \begin{itemize}
            \item All definitions, examples, lemmas, theorems, propositions, and corollaries are consecutively numbered, using the format
            \begin{quote}
                \code{[VOLUME].[CHAPTER].[SECTION].[NUMBER]}
            \end{quote}
            For example, the fourth statement in Volume I, chapter 2, section 3 is labelled \textbf{I.2.3.4}.
            \item Exercises and problems are also numbered consecutively, using the format
            \begin{quote}
                \code{[VOLUME].[CHAPTER].[NUMBER]}
            \end{quote}
            For example, the third exercise in Volume I, chapter 2 is labelled \textbf{I.2.3}. Likewise, the fourth exercise in Volume II, chapter 3 is labelled \textbf{II.3.4}.
        \end{itemize}
        \item Volume numbers are always written in roman numerals, except for Volume 0 which will be written as a zero.
        \item The symbol ``$\qedsymbol$'' marks the end of a proof.
    \end{itemize}

    \section*{Interdependence of Volumes and Chapters}
    The diagram on the next page shows volume and chapter interdependence. It should be used in conjunction with the table of contents and notes listed.

    \newpage
    \includegraphics[width=\linewidth]{Interdependence.png}
    
    \newpage

    \textbf{Notes}:
    \interdependencenotes

    \mainmatter  % Now use arabic numerals for page numbers
}

%============= Index Pages ===============
\usepackage[
    totoc,
    columnsep=20pt,
    hangindent=8pt,
    subindent=20pt,
    subsubindent=30pt
]{idxlayout}

\makeindex[options= -s ../index-style.ist]

%======== Figure Caption Format ==========
\usepackage[labelfont=bf]{caption}
\DeclareCaptionLabelSeparator{custom}{. }
\captionsetup{labelsep=custom}
