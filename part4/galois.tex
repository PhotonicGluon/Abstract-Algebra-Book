\chapter{Galois Theory}
Classical algebra was largely dedicated to finding the solutions to polynomials. The quadratic formula solved any equation of degree 2; the cubic formula any equation of degree 3; the quartic formula any equation of degree 4. However, a quintic formula eluded mathematicians for countless years, until Galois finally proved that such a task was impossible. In this penultimate chapter, we prove the Fundamental Theorem of Galois Theory and establish the insolvability of the quintic.

\section{Field Automorphisms}
Recall from \myref{section-automorphism-groups} that the set of automorphisms of a group forms a group, called the group of automorphisms. We note a similar result for that of field automorphisms.

\begin{proposition}
    The set of all automorphisms of a field $F$ is a group under function composition.
\end{proposition}
\begin{remark}
    We may reuse the notation used for the group of automorphisms of a group and denote the group of automorphisms of a field $F$ by $\Aut{F}$.
\end{remark}
\begin{proof}
    We prove the four group axioms.
    \begin{itemize}
        \item \textbf{Closure}: If $f, g \in \Aut{F}$, and $h = fg$, then $h: F \to F$ is a bijection. Furthermore $h$ is a homomorphism because for $x, y \in F$ we see
        \begin{align*}
            h(x+y) &= f(g(x + y))\\
            &= f(g(x) + g(y)) & (g \text{ is an isomorphism})\\
            &= f(g(x)) + f(g(y)) & (f \text{ is an isomorphism})\\
            &= h(x) + h(y)
        \end{align*}
        and
        \begin{align*}
            h(xy) &= f(g(xy))\\
            &= f(g(x)g(y)) & (g \text{ is an isomorphism})\\
            &= f(g(x))f(g(y)) & (f \text{ is an isomorphism})\\
            &= h(x)h(y).
        \end{align*}
        Therefore $h$ is a bijective homomorphism, meaning $h$ is an isomorphism. Because the domain and codomain of $h$ are both $F$, thus $h$ is an automorphism, meaning $h = fg \in \Aut{F}$.

        \item \textbf{Associativity}: Function composition is associative by \myref{axiom-function-composition-associative}.
        
        \item \textbf{Identity}: We have proved that the identity endomorphism is an automorphism (\myref{exercise-identity-homomorphism-is-an-isomorphism}) which means that $\id \in \Aut{F}$. By definition of $\id$ we know that $f\circ\id = f$ and $\id\circ f = f$ for any $f \in \Aut{F}$, meaning that $\id$ is indeed the identity of $\Aut{F}$.
        
        \item \textbf{Inverse}: Suppose $f \in \Aut{F}$, meaning $f$ is an automorphism. Then we know that $f^{-1}$ is an automorphism with
        \[
            f\circ f^{-1} = \id \text{ and } f^{-1} \circ f = \id
        \]
        which means that $f^{-1}$ is indeed the inverse of $f$.
    \end{itemize}
    Since the four group axioms are satisfied, therefore $\Aut{F}$ is a group under function composition.
\end{proof}

We are particularly interested in automorphisms of an extension field that fixes elements of the base field.
\begin{proposition}\label{prop-galois-group-is-indeed-group}
    Let $F$ be a field and $E/F$ a field extension. Then the set of all automorphisms of $E$ that fix $F$ elementwise is a group. In other words, the set of automorphisms $\phi: E \to E$ such that $\phi(a) = a$ for all $a \in F$ is a group.
\end{proposition}
\begin{proof}
    We show that this set is a subfield of $\Aut{E}$. Note that $\id$ is an element of this subfield since $\id(a) = a$ for all $a \in E$, including $a \in F$.

    Now suppose $\phi$ and $\psi$ be two automorphisms of $E$ such that $\phi(a) = a$ and $\psi(a) = a$ for all $a \in F$. Then we note that $\phi\psi^{-1}$ is an automorphism of $E$ with
    \begin{align*}
        \phi\psi^{-1}(a) &= \phi(\psi^{-1}(a))\\
        &= \phi(a) & (\text{since }\psi(a) = a \text{ thus } a = \psi^{-1}(a))\\
        &= a & (\text{since }\phi(a) = a)
    \end{align*}
    and so $\phi\psi^{-1}$ is an automorphism of $E$ that fixes $F$ elementwise.

    Therefore the result follows by the subgroup test.
\end{proof}

We have a special name for this group.
\begin{definition}
    Let $F$ be a field and $E/F$ a field extension. The group of all automorphisms of $E$ that fix $F$ elementwise is called the \textbf{Galois group of $E/F$}\index{Galois group} and is denoted $\Gal{E/F}$. That is,
    \[
        \Gal{E/F} = \left\{\phi \in \Aut{E} \vert \phi(a) = a \text{ for all } a \in F\right\}.
    \]
\end{definition}
\begin{remark}
    Let $F$ be a field. If $E$ is the splitting field of $f(x) \in F[x]$ over $F$, then the Galois group of $f(x)$ is $\Gal{E/F}$.
\end{remark}

\begin{example}
    One sees that $\phi: \C \to \C$ where $a+bi \mapsto a - bi$ is an automorphism. Since
    \[
        \phi(a) = \phi(a + 0i) = a - 0i = a
    \]
    for all $a \in \R$, therefore we see $\phi \in \Gal{\C/\R}$.
\end{example}

\begin{example}\label{example-galois-group-of-Q-sqrt3-sqrt5-over-Q}
    Consider the field $\Q \subset \Q(\sqrt5) \subset \Q(\sqrt3, \sqrt5)$. Then $\sigma: \Q(\sqrt3,\sqrt5)\to\Q(\sqrt3,\sqrt5)$ given by
    \[
        \sigma(a + b\sqrt3 + c\sqrt5 + d\sqrt{15}) = a - b\sqrt3 + c\sqrt5 + d\sqrt{15}
    \]
    is an automorphism. Also
    \[
        \sigma(a + b\sqrt{5}) = a+b\sqrt5
    \]
    so $\sigma$ fixes $\Q(\sqrt5)$. Similarly $\tau: \Q(\sqrt3,\sqrt5)\to\Q(\sqrt3,\sqrt5)$ given by
    \[
        \tau(a + b\sqrt3 + c\sqrt5 + d\sqrt{15}) = a + b\sqrt3 - c\sqrt5 + d\sqrt{15}
    \]
    is an automorphism that fixes $\Q(\sqrt3)$. The automorphism $\mu = \sigma\tau$ fixes $\Q$. It will be clear soon that $S = \{\id, \sigma, \tau, \mu\}$ is $\Gal{\Q(\sqrt3,\sqrt5)/\Q}$.

    \begin{minipage}[c]{0.475\textwidth}
        \begin{table}[H]
            \centering
            \begin{tabular}{|l|l|l|l|l|}
                \hline
                & $\boldsymbol{\id}$ & $\boldsymbol{\sigma}$ & $\boldsymbol{\tau}$ & $\boldsymbol{\mu}$ \\ \hline
                $\boldsymbol{\id}$ & $\id$ & $\sigma$ & $\tau$ & $\mu$ \\ \hline
                $\boldsymbol{\sigma}$ & $\sigma$ & $\id$ & $\mu$ & $\tau$ \\ \hline
                $\boldsymbol{\tau}$ & $\tau$ & $\mu$ & $\id$ & $\sigma$ \\ \hline
                $\boldsymbol{\mu}$ & $\mu$ & $\tau$ & $\sigma$ & $\id$ \\ \hline
            \end{tabular}
        \end{table}
    \end{minipage}
    \begin{minipage}[c]{0.475\textwidth}
        \begin{table}[H]
            \centering
            \begin{tabular}{|l|l|l|l|l|}
                \hline
                & $\boldsymbol{(0, 0)}$ & $\boldsymbol{(0, 1)}$ & $\boldsymbol{(1, 0)}$ & $\boldsymbol{(1, 1)}$ \\ \hline
                $\boldsymbol{(0, 0)}$ & $(0, 0)$ & $(0, 1)$ & $(1, 0)$ & $(1, 1)$ \\ \hline
                $\boldsymbol{(0, 1)}$ & $(0, 1)$ & $(0, 0)$ & $(1, 1)$ & $(1, 0)$ \\ \hline
                $\boldsymbol{(1, 0)}$ & $(1, 0)$ & $(1, 1)$ & $(0, 0)$ & $(0, 1)$ \\ \hline
                $\boldsymbol{(1, 1)}$ & $(1, 1)$ & $(1, 0)$ & $(0, 1)$ & $(0, 0)$ \\ \hline
            \end{tabular}
        \end{table}
    \end{minipage}

    From the above tables, we see that $\Gal{\Q(\sqrt3,\sqrt5)/\Q} \cong \Z_2^2$. We also note that $[\Q(\sqrt3,\sqrt5):\Q] = 4$ by \myref{example-Q-sqrt3-sqrt5}. It is no coincidence that
    \[
        |\Gal{\Q(\sqrt3,\sqrt5)/\Q}| = [\Q(\sqrt3,\sqrt5):\Q] = 4
    \]
    as we will see later.
\end{example}

Why do we care about the Galois group? It turns out that an element of such a group (which is an automorphism) permutes the zeroes of a polynomial with coefficients in the base field.

\begin{proposition}\label{prop-galois-field-automorphism-permutes-zeroes-of-polynomial}
    Let $F$ be a field and $E/F$ a field extension. Let $f(x) \in F[x]$. Then any $\sigma \in \Gal{E/F}$ defines a permutation of the zeroes of $f(x)$ in $E$.
\end{proposition}
\begin{proof}[Proof (see {\cite[Proposition 23.5]{judson_beezer_2022}})]
    Let $f(x) = a_0 + a_1x + a_2x^2 + \cdots + a_nx^n$ and suppose $\alpha \in E$ is a zero of $f(x)$. Then for any $\sigma \in \Gal{E/F}$ we see
    \begin{align*}
        0 &= \sigma(0) & (\sigma \text{ fixes elements in }F)\\
        &= \sigma(f(\alpha)) & (\alpha \text{ is a zero of }f(x))\\
        &= \sigma(a_0 + a_1\alpha + a_2\alpha^2 + \cdots + a_n\alpha^n)\\
        &= a_0 + a_1\sigma(\alpha) + a_2\sigma(\alpha)^2 + \cdots + a_n(\sigma(\alpha))^n & (\sigma \text{ fixes elements in }F)
    \end{align*}
    which means that $\sigma(\alpha)$ is also a zero of $f(x)$.
\end{proof}

A partial converse of the above proposition exists for special kinds of elements, called conjugates.

\begin{definition}
    Let $F$ be a field and $E/F$ be a field extension. Two elements $\alpha,\beta \in E$ are \textbf{conjugate over $F$}\index{conjugate} if and only if they have the same minimal polynomial.
\end{definition}

\begin{example}
    In the field extension $\Q(\sqrt2)/\Q$, the elements $\sqrt2$ and $-\sqrt2$ are conjugates over $\Q$ since they are both zeroes of the minimal polynomial $x^2 - 2$.
\end{example}

\begin{example}
    In the field extension $\C/\R$, we see $i$ and $-i$ are conjugates over $\R$ since they are both zeroes of the minimal polynomial $x^2 + 1$.
\end{example}

\begin{proposition}
    Let $F$ be a field, $E/F$ be a field extension, and $\alpha,\beta \in E$ be conjugates over $F$. Then there exists an isomorphism $\sigma: F(\alpha) \to F(\beta)$ such that $\sigma(a) = a$ for all $a \in F$.
\end{proposition}
\begin{proof}
    Use \myref{lemma-isomorphism-extension} where, using the notation in that lemma, we set $F' = F$ and $\phi = \id$ and the result follows.
\end{proof}

We earlier found that $|\Gal{\Q(\sqrt3,\sqrt5)/\Q}| = [\Q(\sqrt3,\sqrt5):\Q] = 4$. We show that this is, in fact, not a coincidence.

\begin{theorem}\label{thrm-order-of-galois-group-is-degree-of-field-extension}
    Let $F$ be a field. Let $f(x) \in F[x]$ and suppose $E$ is the splitting field of $f(x)$ over $F$. If $f(x)$ only has simple zeroes, then
    \[
        |\Gal{E/F}| = [E:F],
    \]
    that is, the order of the Galois group of $E/F$ is equal to the degree of the field extension $E/F$.
\end{theorem}
\begin{proof}[Proof (cf. {\cite[Theorem 23.7]{judson_beezer_2022}})]
    We use strong induction on $[E:F]$.

    When $[E:F] = 1$, we have $E = F$ (\myref{prop-finite-extension-of-degree-1-means-extension-equals-base-field}). Therefore $\Gal{E/F}$ consists of only the identity automorphism and so $|\Gal{E/F}| = 1 = [E:F]$.

    Now assume that for any field $F'$, any $\mathfrak{f}(x) \in F'[x]$, and the splitting field $E'$ of $\mathfrak{f}(x)$ over $F'$ such that $[E':F'] \leq k$ for some positive integer $k$, that $|\Gal{E'/F'}| = [E':F']$. We show that the theorem holds for a field $F$, a polynomial $f(x) \in F[x]$, and the splitting field $E$ of $f(x)$ over $F$ such that $[E:F] = k+1$, i.e. we are to show $|\Gal{E/F}| = k + 1$.

    Let $f(x) = p(x)q(x)$, where $p(x), q(x) \in E[x]$ and $p(x)$ is irreducible of degree $d$. If $d = 1$, then $f(x)$ has a zero in $F$ and so $p(x) \in F[x]$. This thus means that $q(x) \in F[x]$ also, and thus $E = F$ which leads to $[E:F] = 1$. If instead $d > 1$, let $\alpha \in E$ be a zero of $p(x)$. Using \myref{lemma-isomorphism-extension} where we pick $F' = F$, $\phi = \id$, and $\beta$ to be a zero of $p(x)$ in $E$, we obtain an isomorphism $\phi: F(\alpha) \to F(\beta)$ where elements of $F$ are fixed under $\phi$ and where $\alpha$ is mapped to $\beta$, i.e. $\beta = \phi(\alpha)$. Since $f(x)$ only has simple zeroes, thus $p(x)$ has $d$ distinct zeroes in $E$, say $\beta_1, \beta_2, \dots, \beta_d$. By \myref{prop-galois-field-automorphism-permutes-zeroes-of-polynomial} we therefore see that there are exactly $d$ isomorphisms $\sigma: F(\alpha) \to F(\beta_i)$ that fix $F$, one for each zero $\beta_1, \beta_2, \dots, \beta_d$ of $p(x)$.

    Since $E$ is a splitting field of $f(x)$ over $F$, thus $E$ is also a splitting field of $f(x)$ over $F(\alpha)$. Similarly $E$ is a splitting field of $f(x)$ over $F(\beta)$. By Tower Law (\myref{thrm-tower-law}) one notes that
    \[
        k+1 = [E:F] = [E:F(\alpha)]\underbrace{[F(\alpha):F]}_{d}
    \]
    and so $[E:F(\alpha)] = \frac{k + 1}{d}$. By Induction Hypothesis, this means that $|\Gal{E:F(\alpha)} = \frac{k+1}{d}$. In other words, there are $\frac{k+1}{d}$ automorphisms $\psi: E \to E$ that fix $F(\alpha)$. But there are $d$ isomorphisms $\sigma: F(\alpha) \to F(\beta_i)$ that fix $F$. Therefore there must be $\frac{k+1}{d} \times d = k+1$ choices for an automorphism in $\Aut{E}$ that fixes $F$, i.e. $|\Gal{E/F}| = k+1$ as required.

    By mathematical induction the claim is proven.
\end{proof}

\begin{corollary}
    Let $F$ be a finite field with a finite extension $E$ such that $[E:F] = k$. Then $\Gal{E/F}$ is cyclic with order $k$.
\end{corollary}
\begin{proof}
    Let $p$ be the characteristic of $E$ and $F$, and assume that the orders of $E$ and $F$ are $p^m$ and $p^n$ respectively. Then by \myref{thrm-subfields-of-finite-field} we know that $m = nk$. Also by \myref{thrm-finite-field-is-unique} we may assume that $E$ is the splitting field of $x^{p^m} - x$ over a subfield of $E$ of order $p$. In particular this means that $E$ must be the splitting field of $x^{p^m} - x$ over $F$. Since $[E:F] = k$ thus $|\Gal{E/F}| = k$ by \myref{thrm-order-of-galois-group-is-degree-of-field-extension}.

    To prove that $\Gal{E/F}$ is cyclic we need to find a generator of $\Gal{E/F}$. In particular we need to find an isomorphism in $\Gal{E/F}$ of order $k$. Consider the map $\sigma: E \to E$ where $\phi(a) = a^{p^n}$ for any $a \in E$. We show that $\sigma \in \Aut{E}$.
    \begin{itemize}
        \item \textbf{Homomorphism}: Note that if $a, b \in E$ then
        \begin{align*}
            \sigma(a+b) &= (a+b)^{p^n}\\
            &= a^{p^n} + b^{p^n} & (\myref{prop-freshman-dream})\\
            &= \sigma(a) + \sigma(b)
        \end{align*}
        and
        \begin{align*}
            \sigma(ab) &= (ab)^{p^n}\\
            &= a^{p^n}b^{p^n}\\
            &= \sigma(a)\sigma(b)
        \end{align*}
        which means that $\sigma$ is a homomorphism.
        
        \item \textbf{Injective}: Since $\sigma$ is non-trivial it must be injective by \myref{thrm-homomorphism-from-field-is-injective-or-trivial}.
        
        \item \textbf{Surjective}: By \myref{problem-injection-from-finite-set-to-itself-is-bijection} we know that $\sigma$ is surjective.
    \end{itemize}
    Therefore $\sigma$ is an isomorphism from $E$ to $E$, i.e. $\sigma$ is an automorphism. So $\sigma \in \Aut{E}$.
    
    We also note that because $F$ is the splitting field of $x^{p^n}-x$ over the base field of order $p$, therefore every element of $F$ is a zero of $x^{p^n}-x$, in particular $a^{p^n} - a = 0$ which means $a = a^{p^n}$ for any $a \in F$. Hence $\sigma(a) = a$ for all $a \in F$, which shows that $\sigma \in \Gal{E/F}$.

    Finally we show that the order of $\sigma$ in $\Gal{E/F}$ is $k$. Recall that since $E$ is the splitting field of $x^{p^m} - x$ over the base field of order $p$, therefore every element of $E$ is a zero of $x^{p^m} - x$ which results in $a^{p^m} = a$ for all $a \in E$. Thus one sees that
    \begin{align*}
        (\sigma^k)(a) &= a^{p^{nk}}\\
        &= a^{p^m}\\
        &= a
    \end{align*}
    for all $a \in E$, meaning that $\sigma^k$ is the identity of $\Gal{E/F}$. We also see that $\sigma^r$ cannot be the identity for $1 \leq r < k$ since otherwise $x^{p^{nr}} - x$ would have $p^m$ roots while the polynomial itself has a degree of $p^{nr}$, which is less than $p^m$, contradicting \myref{thrm-polynomial-of-degree-n-has-at-most-n-zeroes}. Therefore $|\sigma| = k$ in $\Gal{E/F}$, showing that $\Gal{E/F}$ is cyclic with order $k$.
\end{proof}

\begin{example}
    Remember when we claimed that $\Gal{\Q(\sqrt3,\sqrt5)/\Q} \cong \Z_2^2$ in \myref{example-galois-group-of-Q-sqrt3-sqrt5-over-Q}? Certainly $H = \{\id, \sigma, \tau, \mu\}$ is a subgroup of $\Gal{\Q(\sqrt3,\sqrt5)/\Q}$. However one sees that $|H| = 4$ and
    \[
        |\Gal{\Q(\sqrt3,\sqrt5)/\Q}| = [\Q(\sqrt3,\sqrt5):\Q] = 4
    \]
    by \myref{thrm-order-of-galois-group-is-degree-of-field-extension}. Therefore $H = \Gal{\Q(\sqrt3,\sqrt5)/\Q}$.
\end{example}

\begin{exercise}
    Consider the field extension $\C/\R$.
    \begin{partquestions}{\roman*}
        \item Find the order of $\Gal{\C/\R}$.
        \item Find the element(s) of $\Gal{\C/\R}$.
    \end{partquestions}
\end{exercise}

\section{Fixed Fields}
\begin{definition}
    Let $F$ be a field and $E/F$ be a field extension. Let $H$ be a subgroup of $\Gal{E/F}$. Then the \textbf{fixed field}\index{fixed field} of $H$ is
    \[
        \Fix{E}{H} = \{x \in E \vert \phi(x) = x \text{ for all } \phi \in H\}.
    \]
\end{definition}
\begin{remark}
    Other authors may choose to denote the fixed field of $H$ by $E_H$ (e.g. \cite[p.~531]{gallian_2016} and \cite[p.~298]{judson_beezer_2022}) or by $E^H$ (e.g. \cite[p.~486]{artin_2011}).
\end{remark}

\begin{proposition}\label{prop-fixed-field-is-subfield}
    Let $F$ be a field, $E/F$ a field extension, and $H$ be a subgroup of $\Gal{E/F}$. Then $\Fix{E}{H}$ is a subfield of $E$.
\end{proposition}
\begin{proof}
    See \myref{exercise-fixed-field-is-subfield} (later).
\end{proof}

\begin{example}
    Consider the field $\Q(\sqrt3,\sqrt5)$, and the automorphism $\sigma$ on that field such that $\sqrt3 \mapsto -\sqrt3$. Then the fixed field of $\langle\sigma\rangle$ is $\Q(\sqrt5)$, i.e. $\Fix{E}{\langle\sigma\rangle} = \Q(\sqrt5)$.
\end{example}

\begin{exercise}\label{exercise-fixed-field-is-subfield}
    Prove \myref{prop-fixed-field-is-subfield}.
\end{exercise}

\begin{proposition}\label{prop-fixed-field-of-Gal-E/F-is-F}
    Let $F$ be a field. Let $E$ be the splitting field of a polynomial $f(x) \in F[x]$ over $F$. Then
    \[
        \Fix{E}{\Gal{E/F}} = F.
    \]
\end{proposition}
\begin{proof}[Proof (cf. {\cite[Proposition 23.17]{judson_beezer_2022}})]
    For brevity let $G = \Gal{E/F}$.
    
    Note that $\Fix{E}{G} = \{a \in E \vert \phi(a) = a \text{ for all } \phi \in \Gal{E/F}\}$. But clearly every element of $F$ is fixed by any $\phi \in \Gal{E/F}$. Hence it is safe to say that $F \subseteq \Fix{E}{G}$. It is also clear that $\Fix{E}{G} \subseteq E$ since $\Fix{E}{G}$ is a subfield of $E$.

    Now since $F \subseteq \Fix{E}{G}$, any automorphism of $E$ that fixes $\Fix{E}{G}$ must also fix $F$. Thus $\Gal{E/\Fix{E}{G}} \subseteq \Gal{E/F}$. On the other hand, suppose $\sigma \in \Gal{E/F} = G$. Note that
    \[
        \Fix{E}{G} = \{a \in E \vert \phi(a) = a \text{ for all } \phi \in \Gal{E/F}\}
    \]
    by definition, which means that $\sigma(a) = a$ for all $a \in \Fix{E}{G}$. Therefore, by definition of a Galois group, we see $\sigma \in \Gal{E/\Fix{E}{G}}$, meaning $\Gal{E/F} \subseteq \Gal{E/\Fix{E}{G}}$. Hence $\Gal{E/F} = \Gal{E/\Fix{E}{G}}$. Consequently, by \myref{thrm-order-of-galois-group-is-degree-of-field-extension} we see
    \[
        [E:F] = [E:\Fix{E}{G}].
    \]
    On the other hand, by Tower Law (\myref{thrm-tower-law}), one obtains
    \[
        [E:F] = [E:\Fix{E}{G}][\Fix{E}{G}:F].
    \]
    Therefore $[\Fix{E}{G}:F] = 1$ which yields $\Fix{E}{G} = F$ (\myref{prop-finite-extension-of-degree-1-means-extension-equals-base-field}), which is what was to be shown.
\end{proof}

\section{The Fundamental Theorem of Galois Theory}
% TODO: Add

\section{Solvability of Polynomials by Radicals}
% TODO: Add

\section{The Insolvability of the Quintic}
% TODO: Add

\newpage

\section{Problems}
% TODO: Add
