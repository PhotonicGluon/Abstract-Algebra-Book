\setpartpreamble[u][\textwidth]{
    \quoteattr{
        The significance of Galois's discovery far transcended the problem that inspired it. Today, ``Galois groups'' are ubiquitous in the literature, and the group idea has proven to be perhaps the most versatile in all mathematics, clarifying many a deep mystery. ``When in doubt,'' the great Andr\'e Weil advised, ``look for the group!''
    }
    {
        Jim Holt, 2018
    }
    {
        \cite{holt_2018}
    }

    In this final part, we tackle one of the most intimidating topics of introductory abstract algebra -- Galois theory. Introduced by \'Evariste Galois, the theory that now bears his name provides a connection between field theory and group theory.

    In the penultimate chapter, we explore Galois theory in depth. We first build up to what Galois groups are, then look at a similar concept called fixed fields. Along the way, we will build up the intuition for Galois extensions, which are intimately linked to the Fundamental Theorem of Galois Theory. Next, we motivate the exploration and definition of solvable groups and solvable series, and end the chapter with a proof of why there does not exist a quintic formula for polynomials.

    In the final chapter, it is only fitting to prove the Fundamental Theorem of Algebra. The claim of the theorem is simple -- every non-constant polynomial with complex coefficients must have at least one complex zero. Despite its name, the proof requires some parts of real analysis to produce a rigorous argument. Regardless, we will introduce these analytic definitions in the chapter, prove them, and then prove the main theorem.
}
\part{Galois}
