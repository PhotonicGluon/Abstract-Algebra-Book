\documentclass[
    a5paper,
    pagesize,
    11pt,
    bibtotoc,
    normalheadings,
    twoside,
    openany,
    chapterprefix,
    DIV=9
]{scrbook}

\usepackage[utf8]{inputenc}
\usepackage{tocloft}
\usepackage{mathtools}
\usepackage{amsfonts}
\usepackage{enumitem}
\usepackage{amsmath}
\usepackage{amsthm}
\usepackage{amssymb}
\usepackage[hmargin=2cm, vmargin=2.5cm]{geometry}
\usepackage{graphicx}
\usepackage{wrapfig}
\usepackage{parskip}
\usepackage{framed}
\usepackage{fancyhdr}
\usepackage{emptypage}
\usepackage{multicol}
\usepackage{imakeidx}
\usepackage[breaklinks]{hyperref}
\usepackage[capitalise, nameinlink]{cleveref}
\usepackage[x11names]{xcolor}
\usepackage{crossreftools}

\usepackage[
    backend=bibtex,
    style=alphabetic,
    sorting=ynt
]{biblatex}

%=========== Path to images ==============
\graphicspath{{./images/}}

%============== Resources ================
\addbibresource{../AbstractAlgebra.bib}

%============ Redefinitions ==============
\let\oldemptyset\emptyset
\let\emptyset\varnothing

\let\totient\varphi

\renewcommand{\vert}{ \ \vline \ }
\newcommand{\vertalt}{ \ | \ }

\newcommand{\myref}[1]{\textbf{\crthypercref{#1}}}
\newcommand{\myreffigures}[1]{\textbf{\cref{#1}}}

\renewcommand{\qedsymbol}{\ensuremath{\blacksquare}}

%=========== Theorem Styles ==============
\newtheoremstyle{theorem-style}
    {-12pt}      % Space above
    {-5pt}       % Space below
    {}           % Font to use in the theorem
    {0pt}        % Measure of space to indent
    {\bfseries}  % Name of the head font
    {.}          % Punctuation between head and body
    { }          % Space after theorem head; " " = normal inter-word space
    {\thmname{#1}\thmnumber{ #2}\textit{\thmnote{ (#3)}}}

\newtheoremstyle{definition-style}
    {-12pt}      % Space above
    {-5pt}       % Space below
    {}           % Font to use in the definition
    {0pt}        % Measure of space to indent
    {\bfseries}  % Name of the head font
    {.}          % Punctuation between head and body
    { }          % Space after theorem head; " " = normal inter-word space
    {\thmname{#1}\thmnumber{ #2}\textnormal{\thmnote{ (#3)}}}

\newtheoremstyle{exercise-style}
    {-5pt}       % Space above
    {\topsep}    % Space below
    {}           % Font to use in the exercise
    {0pt}        % Measure of space to indent
    {\bfseries}  % Name of the head font
    {.}          % Punctuation between head and body
    { }          % Space after theorem head; " " = normal inter-word space
    {\thmname{#1}\thmnumber{ #2}\textnormal{\thmnote{ (#3)}}}

%======== Theorem-Like Things ============
\theoremstyle{theorem-style}\newtheorem{theoremhidden}{Theorem}[section]
\renewcommand{\thetheoremhidden}{\Roman{part}.\arabic{chapter}.\arabic{section}.\arabic{theoremhidden}}

\theoremstyle{theorem-style}\newtheorem{lemmahidden}[theoremhidden]{Lemma}

\theoremstyle{theorem-style}\newtheorem{propositionhidden}[theoremhidden]{Proposition}

\theoremstyle{theorem-style}\newtheorem{corollaryhidden}[theoremhidden]{Corollary}

\theoremstyle{definition-style}\newtheorem{definitionhidden}[theoremhidden]{Definition}

\theoremstyle{exercise-style}\newtheorem{exercisehidden}{Exercise}[chapter]
\renewcommand{\theexercisehidden}{\Roman{part}.\arabic{chapter}.\arabic{exercisehidden}}

\theoremstyle{definition}\newtheorem{problem}{Problem}[chapter]
\renewcommand{\theproblem}{\Roman{part}.\arabic{chapter}.\arabic{problem}}

\theoremstyle{definition}\newtheorem*{remark}{Remark}
\theoremstyle{definition}\newtheorem{example}[theoremhidden]{Example}

%============ Environments ===============
\newenvironment{theorem}
{\definecolor{shadecolor}{named}{DarkSeaGreen2}\begin{shaded}\noindent\begin{theoremhidden}}
{\end{theoremhidden}\end{shaded}}

\newenvironment{lemma}
{\definecolor{shadecolor}{named}{Honeydew2}\begin{shaded}\noindent\begin{lemmahidden}}
{\end{lemmahidden}\end{shaded}}

\newenvironment{proposition}
{\definecolor{shadecolor}{named}{Honeydew1}\begin{shaded}\noindent\begin{propositionhidden}}
{\end{propositionhidden}\end{shaded}}

\newenvironment{corollary}
{\definecolor{shadecolor}{named}{DarkSeaGreen1}\begin{shaded}\noindent\begin{corollaryhidden}}
{\end{corollaryhidden}\end{shaded}}

\newenvironment{definition}
{\definecolor{shadecolor}{named}{LightCyan1}\begin{shaded}\noindent\begin{definitionhidden}}
{\end{definitionhidden}\end{shaded}}

\newenvironment{exercise}
{\begin{framed}\noindent\begin{exercisehidden}}
{\end{exercisehidden}\end{framed}}

%=========== Custom Commands =============
\newcommand{\code}[1]{\texttt{#1}}  % Code block
\makeatletter\newcommand*{\rom}[1]{\Ifstr{#1}{0}{0}{\expandafter\@slowromancap\romannumeral #1@}}\makeatother  % Roman numeral

\newcommand{\lcm}{\mathrm{lcm}}  % Lowest common multiple function
\newcommand{\sgn}{\mathrm{sgn}}  % Signum function

\newcommand{\im}{\mathrm{im}\;}  % Image of a function
\newcommand{\id}{\mathrm{id}}    % Identity function

%======== Custom Chapter Styling =========
\makeatletter
\renewcommand{\chaptermark}[1]{
    \markboth{\if@mainmatter\chapapp~\thechapter.\ \fi#1}{}
}

\renewcommand*{\chapterformat}{
  \MakeUppercase{\chapapp\nobreakspace\thechapter}
}

\renewcommand*{\chapterlineswithprefixformat}[3]{
    \Ifstr{#1}{chapter}{
        \vspace{-60px}
        \Ifstr{#2}{\empty}{\vspace{40px}}{\raggedleft#2}
        \vspace{-15px}
        \rule{\linewidth}{1pt}\par\nobreak
        \centering{#3}
        \vspace{-10px}
        \rule{\linewidth}{1pt}\par\nobreak
        \vspace{-10px}
    }{#2#3}
}
\makeatother

%======== Figure Caption Format ==========
\usepackage[labelfont=bf]{caption}
\DeclareCaptionLabelFormat{custom}{#1 \Roman{part}.#2.}
\captionsetup{labelformat=custom,labelsep=space}

%============ Custom Header ==============
\fancypagestyle{plain}{\fancyhf{}\renewcommand{\headrulewidth}{0pt}}  % To clear page numbers from footer, and header line at the start of every chapter

\pagestyle{fancy}
\fancyhf{}  % Clear header/footer

\fancyhead[LE,RO]{\thepage}
\fancyhead[LO,RE]{\textit{\nouppercase\leftmark}}

%========= Customise TOC Heading =========
\makeatletter
\def\createtoc{
    \renewcommand\tableofcontents{
        \chapter*{\contentsname}
        \@starttoc{toc}
    }
    \tableofcontents
}
\makeatother

%======= Customise Draft Watermark =======
\newcommand{\setasdraft}{
    \usepackage{draftwatermark}
    \SetWatermarkLightness{0.95}
    \SetWatermarkScale{5}
}

%========= Front Matter Pages ============
\def\volumetitle{Volume \rom{\volumenumber}: \volumename}

\def\frontmatterpages{
    \frontmatter  % Use lowercase roman numerals for page numbers

    % Title page
    \begin{titlepage}
        \centering{
            \selectfont
            \Huge
            \textbf{Abstract Algebra}\\
            \vspace{-0.2cm}
            
            \Large
            \textbf{A Simple Introduction}\\
            \vspace{0.5cm}
            
            \LARGE
            \volumetitle
            \vspace{2cm}
        }\\
        \centering{\Large{Overwrite}}
        \vspace{\fill}

        \includegraphics[width=5cm]{\volumeimage}
        \vspace{\fill}

        \centering \small{\textit{Version \version}}
    \end{titlepage}

    \newpage{}

    % Edition notice
    \clearpage\null\vfill
    \thispagestyle{empty}
    \begin{minipage}[b]{0.9\textwidth}
        \footnotesize\raggedright
        \setlength{\parskip}{0.5\baselineskip}

        Published by Kan Onn Kit\\
        Singapore
        \vspace{5cm}

        \textbf{Abstract Algebra: A Simple Introduction -- \volumetitle}\par
        Version \version
        \vspace{0.3cm}

        Copyright \copyright \ 2022 -- \the\year\ by Kan Onn Kit\par
        This work is licensed under a
        Creative Commons Attribution-NonCommercial-ShareAlike 4.0 International Licence.\par
        \includegraphics[width=2.5cm]{../Images/CC BY-NC-SA 4.0.png}\\  % With reference to the volumes' folders
        The full licence text is available at \url{http://creativecommons.org/licenses/by-nc-sa/4.0/}.\par    
        The source files for the project are available \href{https://github.com/PhotonicGluon/Abstract-Algebra-Book}{here}.
        \vspace{0.3cm}

        Typeset in 11pt Computer Modern Roman using PDF\LaTeX.
    \end{minipage}

    \vspace*{2\baselineskip}
    \cleardoublepage

    % "Quote" page
    \thispagestyle{empty}
    \vspace*{2cm}

    \begin{center}
        \Large{\parbox{10cm}{
            \begin{raggedright}
                \Large
                \quotepagetext
                \vspace{0.3cm}
                
                \hfill
                --- \quotepageattribution\\
                \vspace{-0.25cm}
                
                \hfill
                \normalsize
                (\quotepagecitation)
            \end{raggedright}
        }
    }
    \end{center}

    \newpage

    % Table of contents
    \createtoc
    \setcounter{part}{\volumenumber}

    % Acknowledgements
    \chapter{Acknowledgements}
    Undertaking such a monumental project is new to me, and I am indebted to the people who accompanied me on this journey.

    I am eternally grateful to my parents, who have spent countless hours and an ungodly amount of effort to raise me into who I am today. Their omnipresent kindness, patience, and love for me are something I certainly do not deserve, and I thank them for taking care of me.
    
    I would like to thank my tutor Leong Chong Ming, who got me interested in abstract algebra in the first place. His enthusiasm and eagerness in sharing his knowledge on the subject is the driving force behind my decision to write these books.

    I am grateful for the help of my friend Low Ji Yuan, who has assisted me with countless revisions of the content in these books and given me another pair of eyes in the vetting of content.

    I also sincerely appreciate the support from my mathematics tutors, Loke Weng Heng, Siow Yun Jie, and Teng Yen Ping, who has been there through my junior college years inspiring me with the wonders of mathematics. I am indebted to them for allowing me to excel in my final examinations.

    My close friends, Aidan Tay, Gabriel Fong, and Low Ji Yuan, accompanied me through two years of schooling (and math jokes). I offer infinite thanks to them for sticking with me and for encouraging this math nerd to pursue his wacky projects.

    A thousand thanks go out to my teachers at the School of Science and Technology, Singapore, and specifically my form teacher Lee Tsi Yew Samuel, who instilled important character values into me so I can excel in my future endeavours.

    % Preface
    \chapter{Preface}
    Although algebra has a long history, it has undergone quite striking changes in the past few decades. Abstract (or modern) algebra is widely recognised as an essential element of higher mathematical education. The results that it showcases, however, are often hard to grasp and understand without prerequisite knowledge or with a heavy background in mathematics. Most books on this subject are crafted for undergraduates at universities. They are not for a general mathematics enthusiast or one who seeks to understand more about the inner structure of algebra that mathematicians encounter frequently.

    The exploration of such structures is fundamental to the current underpinning of scientific inquiries. For example, groups are important as they describe the symmetries which the laws of physics seem to obey. Finite fields are also used in coding theory and combinatorics. I hope this series of books will inspire more people to learn more about abstract algebra, beyond the simple introduction presented here.

    This series of books serves to achieve several goals.
    \begin{itemize}
        \item Provide a step-by-step explanation of core results from abstract algebra, without ambiguity of the results discussed.
        \item Demystify the core steps that many textbooks skip over when writing proofs.
        \item Ensure that results from abstract algebra are as accessible, as approachable, and as understandable for as many people as possible.
    \end{itemize}
    I hope that these books can accomplish these goals and let readers enjoy the wonders of abstract algebra.

    \hfill{\textit{22 March, 2023}}

    \section*{Preface for Volume \rom{\volumenumber}}
    \prefacevolumetext
    
    \hfill{\textit{\prefacevolumedate}}

    % Suggestions on the use of this book
    \chapter{Suggestions on the Use of This Book}
    \section*{General Information}
    \begin{itemize}
        \item For most volumes, we include both exercises and problems.
        \begin{itemize}
            \item An exercise can be thought of as a simple ``self-review'' question. Exercises ensure that the content of a particular section is understood and should not be too hard to answer.
            \item A problem is a more holistic version of an exercise. Generally, solutions to problems require a thorough understanding of the current chapter and may require results from other chapters.
        \end{itemize}
        \item A consistent labelling system for all the results within and between volumes is necessary for a project as long as this one.
        \begin{itemize}
            \item All definitions, examples, lemmas, theorems, propositions, and corollaries are consecutively numbered, using the format
            \begin{quote}
                \code{[VOLUME].[CHAPTER].[SECTION].[NUMBER]}
            \end{quote}
            For example, the fourth statement in Volume I, chapter 2, section 3 is labelled \textbf{I.2.3.4}.
            \item Exercises and problems are also numbered consecutively, using the format
            \begin{quote}
                \code{[VOLUME].[CHAPTER].[NUMBER]}
            \end{quote}
            For example, the third exercise in Volume I, chapter 2 is labelled \textbf{I.2.3}. Likewise, the fourth exercise in Volume II, chapter 3 is labelled \textbf{II.3.4}.
        \end{itemize}
        \item Volume numbers are always written in Roman numerals, except for Volume 0 which will be written as a zero.
        \item The symbol ``$\qedsymbol$'' marks the end of a proof.
    \end{itemize}

    \section*{Chapter Interdependence}
    The diagram on the next page shows chapter interdependence. It should be used in conjunction with the table of contents and notes listed.

    \newpage
    \includegraphics[width=\linewidth]{Interdependence.png}
    
    \newpage

    \textbf{Notes}:
    \interdependencenotes

    \mainmatter  % Now use arabic numerals for page numbers
}

%============= Index Pages ===============
\usepackage[
    totoc,
    columnsep=20pt,
    hangindent=8pt,
    subindent=20pt,
    subsubindent=30pt
]{idxlayout}

\makeindex[options= -s ../index-style.ist]

%======= Bibliography Formatting =========
% These two lines are here to ensure that URLs do not exceed the page by too much
\setcounter{biburllcpenalty}{7000}
\setcounter{biburlucpenalty}{8000}

\usepackage{xr}

%=========== Global Variables ============
\newcommand{\version}{0.1}
\newcommand{\volumenumber}{2}
\newcommand{\volumename}{Rings}
\newcommand{\volumeimage}{cover/Integers Modulo n.png}

%============= Formatting ================
\linespread{1.05}

%============== Resources ================
\externaldocument{../Volume 0/Volume0}
\externaldocument{../Volume 1/Volume1}

%=========== Custom Commands =============
\newcommand{\Mn}[2]{\mathcal{M}_{#1\times#1}(#2)}  % The ring of n by n matrices with entries in the ring R

\newcommand{\C}{\mathbb{C}}                        % The ring of complex numbers
\newcommand{\Q}{\mathbb{Q}}                        % The ring of rational numbers
\newcommand{\R}{\mathbb{R}}                        % The ring of real numbers
\newcommand{\Z}{\mathbb{Z}}                        % The ring of integers
\newcommand{\Zn}[1]{\mathbb{Z}_{#1}}               % The ring of integers modulo n

%========= Front Matter Pages ============
% Quote page
\newcommand{\quotepagetext}{
    [Some] of the major discoveries in ring theory have helped shape the course of development of modern abstract algebra... A course in ring theory is an indispensable part of the education of any fledgling algebraist.
}
\newcommand{\quotepageattribution}{Tsit-Yuen Lam, 2001}
\newcommand{\quotepagecitation}{\cite{lam_2001}}

% Preface
\newcommand{\prefacevolumetext}{
    This volume covers the basics of ring theory. %TODO: Add
}
\newcommand{\prefacevolumedate}{}  %TODO: Add

% Suggestions of use
\newcommand{\interdependencenotes}{
    % TODO: Update
}

%=========================================
\begin{document}
\frontmatterpages

%=========================================
\chapter{Introduction to Rings}
In Volume I, we looked exclusively at groups and their operations. We discussed how groups are a generalisation of symmetry and looked at results related to groups. In this volume, we look at rings.

Before we introduce rings, we look at `simpler' algebraic structures and build our way up to them.

\section{Basic Algebraic Structures}
\begin{definition}
    A \textbf{magma}\index{magma} is a set $M$ together with a binary operation $\ast$ which is closed. That is, if $a$ and $b$ are in $M$, then $a \ast b \in M$. Such a magma is denoted $(M, \ast)$.
\end{definition}
\begin{example}
    Consider the set $M = \{1, 2, 3, 4\}$ with the operation $\ast$ such that $(M, \ast)$ has the Cayley table as shown below.
    \begin{table}[h]
        \centering
        \begin{tabular}{|l|l|l|l|l|}
            \hline
            $\ast$     & \textbf{1} & \textbf{2} & \textbf{3} & \textbf{4} \\ \hline
            \textbf{1} & 1          & 2          & 1          & 2          \\ \hline
            \textbf{2} & 2          & 3          & 4          & 1          \\ \hline
            \textbf{3} & 1          & 3          & 4          & 2          \\ \hline
            \textbf{4} & 2          & 1          & 2          & 1          \\ \hline
        \end{tabular}
    \end{table}
    
    Clearly for any $a, b\in M$ we have $a \ast b \in M$, so $(M, \ast)$ is a magma.
\end{example}

\begin{definition}
    A \textbf{semigroup}\index{semigroup} is a magma $(\mathcal{S}, \ast)$ where the operation $\ast$ is associative. That is, $a\ast(b\ast c) = (a\ast b)\ast c$.
\end{definition}
\begin{example}
    Consider the set $S = \{1, 2, 3, 4\}$ with the operation $\ast$ such that $(S, \ast)$ has the Cayley table as shown below.
    \begin{table}[h]
        \centering
        \begin{tabular}{|l|l|l|l|l|}
            \hline
            $\ast$     & \textbf{1} & \textbf{2} & \textbf{3} & \textbf{4} \\ \hline
            \textbf{1} & 1          & 1          & 1          & 1          \\ \hline
            \textbf{2} & 2          & 2          & 2          & 2          \\ \hline
            \textbf{3} & 3          & 3          & 3          & 3          \\ \hline
            \textbf{4} & 4          & 4          & 4          & 4          \\ \hline
        \end{tabular}
    \end{table}
    
    One sees that $(S, \ast)$ is closed under $\ast$. In addition, $\ast$ is associative. Hence $(S, \ast)$ is a semigroup.
\end{example}

\begin{definition}
    A \textbf{monoid}\index{monoid} is a semigroup $(M, \ast)$ with an element $e$, called the \textbf{identity}, such that
    \[
        e \ast m = m \ast e = m
    \]
    for all $m \in M$.
\end{definition}
\begin{example}
    Let $X$ be a set. Let the set
    \[
        S = \{f \vert  f: X \to X\}.
    \]
    and let $\circ$ denote function composition. Then $(S, \circ)$ forms a monoid:
    \begin{itemize}
        \item \textbf{Closure}: If $f, g \in S$ then $f\circ g \in S$ since function composition is closed.
        \item \textbf{Associative}: Function composition is associative.
        \item \textbf{Identity}: The identity function $\id: X \to X, x\mapsto x$ is inside $S$ and
        \[
            \id \circ f = f \circ \id = f
        \]
        for all $f \in S$.
    \end{itemize}
    Thus $(S, \circ)$ forms a monoid.
\end{example}

\begin{definition}
    A \textbf{group}\index{group} is a monoid $(G, \ast)$ where every element has an inverse. That is to say, for every $g \in G$, there exists $g^{-1} \in G$ such that
    \[
        g \ast g^{-1} = g^{-1} \ast g = e
    \]
    where $e$ is the identity in $G$.
\end{definition}

\section{Definition of a Ring}
With all that setup, we are ready to define what a ring is. Note that we follow \cite[p.~223]{dummit_foote_2004}, \cite[p.~115, Definition 1.1]{hungerford_1980}, and \cite{proofwiki_ringdefinition} for the definition of a ring.
\begin{definition}
    A \textbf{ring}\index{ring} is a set $R$ with two binary operations $+$ and $\cdot$ satisfying the following axioms.
    \begin{itemize}
        \item \textbf{Addition-Abelian}: $(R, +)$ is an abelian group.
        \item \textbf{Multiplication-Semigroup}: $(R, \cdot)$ is a semigroup.
        \item \textbf{Distributive}: $\cdot$ is distributive over $+$. That is,
        \begin{itemize}
            \item $a \cdot (b + c) = (a \cdot b) + (b \cdot c)$; and
            \item $(a + b) \cdot c = (a \cdot c) + (b \cdot c)$.
        \end{itemize}
    \end{itemize}
    One may denote such a ring by $(R, +, \cdot)$.
\end{definition}
\begin{remark}
    Other authors (e.g. \cite[p.~136]{cohn_1982}, \cite[pp.~145--146]{clark_1984}) may require that $(R, \cdot)$ is a monoid. In this book, any ring that satisfies the above condition is called a \textbf{ring with identity}\index{ring!with identity}.
\end{remark}
\begin{remark}
    A ring where $a \cdot b = b \cdot a$ for all $a$ and $b$ in $R$ is called a \textbf{commutative ring}\index{ring!commutative}.
\end{remark}

We end this chapter by introducing the \textbf{trivial ring}.
\begin{definition}
    The \textbf{trivial ring}\index{trivial ring} (or \textbf{zero ring}\index{zero ring}), denoted $\textbf{0}$, is the ring $(\{0\}, +, \cdot)$ where
    \[
        0 + 0 = 0 \text{ and } 0 \cdot 0 = 0.    
    \]
\end{definition}
\begin{exercise}
    Prove that the trivial ring is a commutative ring with identity.
\end{exercise}

%=========================================
\chapter{Basics of Rings}
With basic algebraic structures and the definition of a ring out of the way, we are now ready to tackle the basics of rings in this chapter.

\section{Basic Examples of Rings}
Before we introduce some examples of rings, we make some remarks for the notation that is used in Ring Theory.
\begin{itemize}
    \item The multiplication symbol $\cdot$ is usually omitted, so $x \cdot y$ is written as $xy$.
    \item The additive identity of $R$ will always be denoted by 0 and the multiplicative identity of $R$ (if it exists) will always be denoted by 1.
    \item The additive inverse of the element $x$ will be denoted by $-x$ and the multiplicative inverse of $x$ (if it exists) will be denoted by $x^{-1}$.
    \item $n$ applications of $+$ on an element $x$ will be denoted $nx$ (and will be denoted $-nx$ if the element is $-x$), while $n$ applications of $\cdot$ on an element $x$ will be denoted $x^n$ (and will be denoted $x^{-n}$ if the element is $x^{-1}$ and if it exists).
\end{itemize}

Let's look at some examples of rings.
\begin{example}
    We show that $(\Zn{n}, \oplus_n, \otimes_n)$, where $\oplus_n$ and $\otimes_n$ denote addition and multiplication modulo $n$ respectively, form a ring.
    \begin{itemize}
        \item \textbf{Addition-Abelian}: We know $(\Zn{n}, \oplus_n)$ is an abelian group from Volume I.
        \item \textbf{Multiplication-Semigroup}: We can see that $(\Zn{n}, \otimes_n)$ is a semigroup as
        \begin{itemize}
            \item $\Zn{n}$ is closed under $\otimes_n$ because $a \otimes_n b \in \{0, 1, 2, \dots, n-1\} = \Zn{n}$; and
            \item multiplication is associative, so multiplication modulo $n$ is associative.
        \end{itemize}
        \item \textbf{Distributive}: It is clear that $\oplus_n$ and $\otimes_n$ distribute.
    \end{itemize}
    Hence $(\Zn{n}, \oplus_n, \otimes_n)$ is a ring. Furthermore, $\otimes_n$ has an identity of 1 and is commutative, so in fact $(\Zn{n}, \otimes_n)$ is a commutative monoid. Therefore $(\Zn{n}, \oplus_n, \otimes_n)$ is a commutative ring with identity.

    Note that, in this volume, $\Zn{n}$ refers to the ring $(\Zn{n}, \oplus_n, \otimes_n)$.
\end{example}

\begin{example}
    We show that $(\Q, +, \times)$, where $+$ and $\times$ denote normal addition and multiplication, form a ring.
    \begin{itemize}
        \item \textbf{Addition-Abelian}: From Volume I, we know that $(\Q, +)$ is an abelian group.
        \item \textbf{Multiplication-Semigroup}: We note that $(\Q, \times)$ is a semigroup as
        \begin{itemize}
            \item $Q$ is closed under $\times$ because multiplying two rational numbers together produce a rational number; and
            \item multiplication is associative.
        \end{itemize}
        \item \textbf{Distributive}: It is clear that $+$ and $\times$ distribute.
    \end{itemize}
    Hence $(\Q, +, \times)$ is a ring. Furthermore, $\times$ has an identity of 1 and is commutative. So, just like $\Zn{n}$, we see that $(\Q, +, \times)$ is a commutative ring with identity.

    Note that, in this volume, $\Q$ refers to the ring $(\Q, +, \times)$.
\end{example}

\begin{exercise}
    Prove that $\Z$ is a ring under regular addition and multiplication.\newline
    (\textit{You do \textbf{not} need to prove the \textbf{Distributive} axiom.})
\end{exercise}
Note that, in this volume, $\Z$ refers to the ring $(\Z, +, \times)$.

We remark that both $\R$ and $\C$ are rings under regular addition and multiplication, although we will not prove it here.

We look at one more ring: the ring of $n \times n$ matrices.

Let $\Mn{n}{R}$ denote the set of $n\times n$ matrices with entries in the ring $(R, \oplus, \otimes)$.
\begin{itemize}
    \item We define \textbf{matrix addition}\index{matrix addition} within this set. Suppose we have $\textbf{A}, \textbf{B} \in \Mn{n}{R}$, and let their sum be $\textbf{C} = \textbf{A} + \textbf{B}$. Then
    \[
        c_{i,j} = a_{i,j} \oplus b_{i,j}    
    \]
    where $\oplus$ is the addition operation in the ring $R$.
    
    \item We define \textbf{matrix multiplication}\index{matrix multiplication} within this set. Suppose $\textbf{A}, \textbf{B} \in \Mn{n}{R}$, and let their product be $\textbf{C} = \textbf{AB}$. Then
    \begin{align*}
        c_{i,j} &= (a_{i,1}\otimes b_{1,j}) \oplus (a_{i,2}\otimes b_{2,j}) \oplus \cdots \oplus (a_{i,n}\otimes b_{n,j})\\
        &= \sum_{k=1}^n (a_{i,k}\otimes b_{i,k})
    \end{align*}
    where $\oplus$ and $\otimes$ are the addition and multiplication operations in the ring $R$ respectively.
\end{itemize}

\begin{proposition}
    $\Mn{n}{R}$ is a ring under matrix addition and matrix multiplication.
\end{proposition}
\begin{proof}
    We need to prove that the ring axioms hold.
    \begin{itemize}
        \item \textbf{Addition-Abelian}: We first prove that $(\Mn{n}{R}, +)$ is indeed an abelian group.
        \begin{itemize}
            \item \textbf{Closure}: Clearly the sum of any two matrices in $\Mn{n}{R}$ is also a square matrix with $n$ rows with elements inside $R$, meaning that $\Mn{n}{R}$ is closed under matrix addition.

            \item \textbf{Associative}: Let the matrices $\textbf{A}$, $\textbf{B}$, and $\textbf{C}$ belong inside $\Mn{n}{R}$. Let $\textbf{P} = \textbf{A} + (\textbf{B} + \textbf{C})$ and $\textbf{Q} = (\textbf{A} + \textbf{B}) + \textbf{C}$. We note that $\textbf{P} = \textbf{Q}$ as
            \begin{align*}
                p_{i,j} &= a_{i,j} \oplus (b_{i,j} \oplus c_{i,j})\\
                &= (a_{i,j} \oplus b_{i,j}) \oplus c_{i,j} & (\oplus\text{ is associative})\\
                &= q_{i,j}
            \end{align*}
            which proves that matrix addition is associative.
    
            \item \textbf{Identity}: Denote the $n \times n$ matrix with contains only the zero in $R$ by $\textbf{0}_n$. One sees clearly that this is the identity in $\Mn{n}{R}$ as $\textbf{M} + \textbf{0}_n = \textbf{M}$ for any matrix in $\Mn{n}{R}$.
            
            \item \textbf{Inverse}: Let $\textbf{A} \in \Mn{n}{R}$. Define the matrix $\textbf{B} = -\textbf{A}$ where
            \[
                b_{i,j} = -a_{i,j},    
            \]
            that is, $b_{i,j}$ contains the additive inverse of $a_{i,j}$ in the ring $R$. Then one sees that $\textbf{A} + \textbf{B} = \textbf{0}_n$.\newline
            (Note that we denote the additive inverse of a matrix $\textbf{M}$ by $-\textbf{M}$.)

            \item \textbf{Commutative}: Let $\textbf{A}, \textbf{B} \in \Mn{n}{R}$. Set $\textbf{C} = \textbf{A} + \textbf{B}$ and $\textbf{D} = \textbf{B} + \textbf{C}$. Then
            \begin{align*}
                c_{i,j} &= a_{i,j} \oplus b_{i,j}\\
                &= b_{i,j} \oplus a_{i,j} & (\oplus \text{ is commutative})\\
                &= d_{i,j}
            \end{align*}
            which means $\textbf{C} = \textbf{D}$.
        \end{itemize}

        \item \textbf{Multiplication-Semigroup}: We now need to prove that $(\Mn{n}{R}, \cdot)$ is a semigroup.
        \begin{itemize}
            \item \textbf{Closure}: In Volume I we showed that matrix multiplication produces another $n \times n$ matrix. Furthermore the entries of the new matrix are elements of $R$. Hence $\Mn{n}{R}$ is closed under matrix multiplication.
        
            \item \textbf{Associative}: We proved matrix multiplication is associative in Volume I.
        \end{itemize}
        
        \item \textbf{Distributive}: We prove only $\textbf{A}(\textbf{B} + \textbf{C}) = (\textbf{AB}) + (\textbf{AC})$ as the other case is proven similarly.
        
        Let $\textbf{R} = \textbf{A}(\textbf{B} + \textbf{C})$, $\textbf{G} = \textbf{AB}$, and $\textbf{H} = \textbf{AC}$. We note
        \begin{align*}
            r_{i,j} &= \sum_{k=1}^n \left(a_{i,k} \otimes \left(b_{k,j} \oplus c_{k,j}\right)\right)\\
            &= \sum_{k=1}^n \left((a_{i,k} \otimes b_{k,j}) \oplus (a_{i,k} \otimes c_{k,j})\right)\\
            &= \left(\sum_{k=1}^n (a_{i,k} \otimes b_{k,j})\right) \oplus \left(\sum_{k=1}^n (a_{i,k} \otimes c_{k,j})\right)\\
            &= g_{i,j}\oplus h_{i,j}
        \end{align*}
        which means $\textbf{R} = \textbf{G} + \textbf{H}$.
    \end{itemize}
    As all the ring axioms are satisfied, thus $\Mn{n}{R}$ is a ring.
\end{proof}
For brevity, we let $\Mn{n}{R}$ denote the ring under matrix addition and multiplication.

\section{General Properties of Rings}
We list some properties of rings here. For each of the propositions, assume $R$ is a ring.

\begin{proposition}\label{prop-multiplying-by-zero-is-zero}
    $0x = x0 = 0$ for all $x \in R$.
\end{proposition}
\begin{proof}
    We note that
    \begin{align*}
        0x &= (0 + 0)x & (0 \text{ is additive inverse})\\
        &= 0x + 0x & (\text{by \textbf{Distributive} axiom})
    \end{align*}
    so by `subtracting' $0x$ on both sides (i.e., adding $-0x$ on both sides) we see $0 = 0x$. A similar argument shows that $x0 = 0$.
\end{proof}

\begin{proposition}\label{prop-product-of-element-and-additive-inverse-is-additive-inverse-of-product}
    $(-a)b = a(-b) = -(ab)$ for any $a$ and $b$ in $R$.
\end{proposition}
\begin{proof}
    We show that $(-a)b = -(ab)$ and $a(-b) = -(ab)$ to complete the proof.
    \begin{itemize}
        \item Note $(-a)b + ab = (-a + a)b = 0b = 0$ by \textbf{Distributive} axiom. Hence by subtracting $ab$ on both sides we see $(-a)b = -(ab)$.
        \item Note also $a(-b) + ab = a(-b + b) = a0 = 0$ by \textbf{Distributive} axiom. Hence by subtracting $ab$ on both sides we see $a(-b) = -(ab)$.
    \end{itemize}
    Result follows.
\end{proof}

\begin{proposition}
    $(-a)(-b) = ab$ for any $a$ and $b$ in $R$.
\end{proposition}
\begin{proof}
    See \myref{exercise-product-of-additive-inverses} (later).
\end{proof}

\begin{proposition}
    If $R$ has an identity, it is unique.
\end{proposition}
\begin{proof}
    Suppose 1 and $1'$ are identities, and consider the sum $1 + 1'$. Then
    \begin{align*}
        1 + 1' &= 1'(1+1') & (\text{multiplying by identity }1')\\
        &= 1'1 + 1'1' & (\text{by \textbf{Distributive} axiom})\\
        &= 1' + 1'. & (1 \text{ and } 1' \text{ are identities})
    \end{align*}
    Subtracting $1'$ on both sides yields $1 = 1'$, meaning that the identity is unique.
\end{proof}

\begin{exercise}\label{exercise-product-of-additive-inverses}
    Show that $(-a)(-b) = ab$ for any $a$ and $b$ in $R$.
\end{exercise}

\section{More Definitions}
Suppose $R$ is a ring.
\begin{definition}
    We say that $a \neq 0$ is a \textbf{zero divisor}\index{zero divisor} if there exists a $b \neq 0$ such that $ab = 0$.
\end{definition}
\begin{example}
    Consider the ring $\Zn{12}$. Clearly 4 and 6 are in $\Zn{12}$, and their product is $24 = 2 \times 12 = 0$ in $\Zn{12}$. Hence 4 and 6 are zero divisors in $\Zn{12}$.
\end{example}
\begin{example}
    We claim that the ring
    \[
        R = \{f: [0, 1] \to [0, 1]\}
    \]
    has zero divisors. Consider the functions
    \begin{align*}
        f(x) &= x\\
        g(x) &= \begin{cases}
            0 & \text{ if } x \neq 0\\
            1 & \text{ if } x = 0
        \end{cases}
    \end{align*}
    Clearly neither of them are the zero function. However, consider $f(x)g(x)$.
    \begin{itemize}
        \item If $x \neq 0$, then $g(x) = 0$ which means $f(x)g(x) = 0$.
        \item If $x = 0$, then $f(x) = 0$ which means $f(x)g(x) = 0$.
    \end{itemize}
    Hence their product is the zero function, meaning that $R$ has zero divisors $f$ and $g$.
\end{example}
\begin{exercise}
    Does the ring $\Mn{2}{\mathbb{R}}$ have zero divisors?
\end{exercise}

\begin{definition}
    Suppose $0 \neq 1$. An element $u \in R$ is called a \textbf{unit}\index{unit} if there exists a $v \in R$ such that $uv=vu=1$.
\end{definition}
\begin{remark}
    Equivalently, we say that $u$ is a unit if it has a multiplicative inverse.
\end{remark}
\begin{example}
    3 and 7 are units in $\Zn{10}$ since $3 \times 7 = 7 \times 3 = 21 = 1$ in $\Zn{10}$.
\end{example}

\begin{definition}
    If every non-zero element $x \in R$ is a unit, then $R$ is said to be a \textbf{division ring}\index{division ring}.
\end{definition}

\begin{definition}
    A commutative division ring is called a \textbf{field}\index{field}.
\end{definition}
We look at fields in more detail in Volume III.

\section{Subrings}
We end this chapter off with an exploration about subrings.

\begin{definition}
    Let $R$ be a ring and $S$ be a subset of $R$. Then $S$ is a \textbf{subring}\index{subring} of $R$ if
    \begin{itemize}
        \item $(S, +) \leq (R, +)$, i.e. the set $S$ under addition is a subgroup of $R$ under addition; and
        \item For all $a$ and $b$ in $S$ we have $ab \in S$, i.e. $S$ is closed under multiplication.
    \end{itemize}
\end{definition}
\begin{remark}
    Alternatively, one may show that $S$ is a ring to prove that $S$ is a subring of $R$.
\end{remark}

\begin{example}
    We know that $\Z$ and $\Q$ are rings, and clearly $\Z \subseteq \Q$. Hence $\Z$ is a subring of $\Q$. Similarly, $\Q$ is a subring of $\R$, and $\R$ is a subring of $\C$.
\end{example}

\begin{example}
    Consider the \textbf{gaussian integers}\index{gaussian integers},
    \[
        \Z[i] = \{a + bi \vert a,b\in\mathbb{Z}\},
    \]
    where $i$ is the imaginary unit (i.e., $i^2 = -1$). We will show that $\Z[i]$ is a subring of $\C$.
    \begin{proof}
        Clearly $\Z[i] \subseteq \C$. We show that $(\Z[i], +) \leq (\C, +)$.
        \begin{itemize}
            \item Clearly the identity of $(\C, +)$, which is 0, is inside $(\Z[i], +)$ as $0 = 0 + 0i$.
            \item For any $a + bi, c+di \in \Z[i]$, we have $(a+bi) + (-(c+di)) = (a-c) + (b-d)i \in \Z[i]$.
        \end{itemize}
        Therefore by subgroup test we see $(\Z[i], +) \leq (\C, +)$.

        Now we show that $\Z[i]$ is closed under multiplication. Let $a + bi, c+di \in \Z[i]$. Then their product is
        \begin{align*}
            (a+bi)(c+di) &= ac+adi+bci+bdi^2\\
            &= (ac-bd) + (ad+bc)i
            &\in \Z[i]
        \end{align*}
        so $\Z[i]$ is closed under multiplication. Therefore $\Z[i]$ is a subring of $\C$.
    \end{proof}
\end{example}
\begin{exercise}
    Show that
    \[
        R = \left\{\begin{pmatrix}a&a\\a&a\end{pmatrix} \vert a \in \R\right\}
    \]
    is a subring of $\Mn{2}{\mathbb{R}}$.
\end{exercise}

\newpage

\section{Problems}
\begin{problem}
    Prove that the trivial ring is the unique ring with identity in which $0 = 1$.
\end{problem}
\begin{problem}
    Let $(R, \ast)$ be a magma. If $(R, \ast, \ast)$ is a ring, describe the elements in $R$.
\end{problem}
%TODO: Add problems

%=========================================
\appendix
\chapter{Exercise Solutions}
\section{Introduction to Rings}
\begin{enumerate}
    \item We note the following.
    \begin{itemize}
        \item \textbf{Addition-Abelian}: $(\{0\}, +)$ is an abelian group by Volume I.
        \item \textbf{Multiplication-Semigroup}: $(\{0\}, \cdot)$ is an abelian group by Volume I.
        \item \textbf{Distributive}: We know $+$ and $\cdot$ distribute.
    \end{itemize}
    Hence $(\{0\}, +, \cdot)$ is a commutative ring with identity.
\end{enumerate}

\section{Basics of Rings}
\begin{enumerate}
    \item We note the following.
    \begin{itemize}
        \item \textbf{Addition-Abelian}: $(\Z, +)$ is an abelian group by Volume I.
        \item \textbf{Multiplication-Semigroup}: $(\Z, \times)$ is a semigroup since
        \begin{itemize}
            \item multiplying two integers always results in an integer, so $\Z$ is closed under $\times$; and
            \item $\times$ is associative.
        \end{itemize}
        \item \textbf{Distributive}: We know $+$ and $\times$ distribute.
    \end{itemize}
    Hence $(\Z, +, \times)$ is a ring.

    \item Consider $(-a)(-b) + (-ab)$ and note
    \begin{align*}
        &(-a)(-b) + (-ab)\\
        &= (-a)(-b) + (-a)b & (\text{\myref{prop-product-of-element-and-additive-inverse-is-additive-inverse-of-product}})\\
        &= (-a)(-b + b) & (\text{by \textbf{Distributive} axiom})\\
        &= (-a)0\\
        &= 0 & (\myref{prop-multiplying-by-zero-is-zero})
    \end{align*}
    which means $(-a)(-b) = -(-ab) = ab$ as required.

    \item The ring $\Mn{2}{\mathbb{R}}$ indeed has zero divisors, as $\begin{pmatrix}0&1\\0&0\end{pmatrix} \neq \begin{pmatrix}0&0\\0&0\end{pmatrix}$ but $\begin{pmatrix}0&1\\0&0\end{pmatrix}^2 = \begin{pmatrix}0&0\\0&0\end{pmatrix}$ which means that $\begin{pmatrix}0&1\\0&0\end{pmatrix}$ is a zero divisor.
    
    \item We first show that $(R, +) \leq (\Mn{2}{\R}, +)$.
    \begin{itemize}
        \item Clearly the identity of $(\Mn{2}{\R}, +)$, the zero matrix $\begin{pmatrix}0&0\\0&0\end{pmatrix}$, is inside $R$.
        \item Consider $\begin{pmatrix}a&a\\a&a\end{pmatrix}, \begin{pmatrix}b&b\\b&b\end{pmatrix} \in R$. The additive inverse of $\begin{pmatrix}b&b\\b&b\end{pmatrix}$ is $\begin{pmatrix}-b&-b\\-b&-b\end{pmatrix}$, and so their sum is
        \[
            \begin{pmatrix}a&a\\a&a\end{pmatrix} + \begin{pmatrix}-b&-b\\-b&-b\end{pmatrix} = \begin{pmatrix}a-b&a-b\\a-b&a-b\end{pmatrix} \in R
        \]
        which means $R$ is closed under addition.
    \end{itemize}
    Hence $(R, +) \leq (\Mn{2}{\R}, +)$ by subgroup test.

    We now show that $R$ is closed under multiplication. Some calculation yields that
    \[
        \begin{pmatrix}a&a\\a&a\end{pmatrix}\begin{pmatrix}b&b\\b&b\end{pmatrix} = \begin{pmatrix}2ab&2ab\\2ab&2ab\end{pmatrix}
    \]
    which is clearly in $R$. Therefore $R$ is a subring of $\Mn{2}{\R}$.
\end{enumerate}

%=========================================
\chapter{Problem Solutions}
\section{Introduction to Rings}
No problems.

\section{Basics of Rings}
\begin{enumerate}
    \item Suppose $R$ is a ring where 0 = 1, and let $x \in R$. Then
    \begin{align*}
        x &= 1x & (1 \text{ is the multiplicative identity})\\
        &= 0x & (0 = 1)\\
        &= 0 & (0x = 0 \text{ for all }x)
    \end{align*}
    which means that $R$ only contains one element, namely the identity 0. Hence the trivial ring is the unique ring where 0 = 1.

    \item Since $(R, \ast, \ast)$ is a ring, we know that $(R, \ast)$ is an abelian group and that $a \ast(b\ast c) = (a \ast b) \ast (a \ast c)$ for all $a, b, c \in R$ (left distribution). Consider an element $x \in R$. Thus we must have
    \[
        x \ast (x \ast x) = (x \ast x) \ast (x \ast x)    
    \]
    which means $x^3 = x^4$. Since $(R, \ast)$ is a group, thus $x^{-3}$ exists. Applying that on both sides means $x = e$ where $e$ is the identity. Therefore, $R$ contains only one element, the identity, which means that $(R, \ast, \ast)$ is actually the trivial ring.
\end{enumerate}
%=========================================
\chapter{Image Acknowledgements}
Unless otherwise stated, all images are the author's own work, and are released under the same licence as this book.

%=========================================
\printbibliography[heading=bibintoc, title={References and Bibliography}]
\printindex

\end{document}
