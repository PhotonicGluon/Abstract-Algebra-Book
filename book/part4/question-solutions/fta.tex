\section{The Fundamental Theorem of Algebra}
\subsection*{Exercises}
\begin{questions}
    \item We note $\inf S = \frac12$ since $1 - \frac1{2^n} >
    \frac12$ for $n > 1$.

    We claim that $\sup S = 1$. Clearly 1 is an upper bound of $S$ since $1 - \frac1{2^n} < 1$ for all $n \in \mathbb{N}$. Now suppose $\epsilon > 0$. Choose $n \in \mathbb{N}$ such that $n > -\log_2(\epsilon)$, which means $\frac1{2^n} < \epsilon$. Then note that $1 - \frac1{2^n} \in S$ and also
    \[
        1 - \frac1{2^n} > 1 - \epsilon.
    \]
    Therefore by \myref{prop-identifying-suprema} we see that $\sup S = 1$.

    \item One sees
    \begin{align*}
        |f(x)| &= |f(x) - f(x_0) + f(x_0)|\\
        &\leq |f(x) - f(x_0)| + |f(x_0)| & (\myref{prop-triangle-inequality})\\
        &< 1 + |f(x_0)|. & (\text{By assumption})
    \end{align*}

    \item Note by Tower Law (\myref{thrm-tower-law}) that
    \[
        [L:\R] = [L:\C][\C:\R]
    \]
    and because $[\C:\R] = 2$ and $[L:\C]$ is finite, thus we see that $[L:\R] = 2k$ for some integer $k$. As $L/\R$ is a finite Galois extension we therefore have $\Gal{L/\R} = [L:\R] = 2k$ by \myref{corollary-galois-iff-galois-field-has-order-of-degree-of-field-extension}.

    \item Let
    \[
        u = \pm\sqrt{\frac{x + \sqrt{x^2+y^2}}{2}} \text{ and } v = \pm\sqrt{\frac{-x + \sqrt{x^2+y^2}}{2}}.
    \]
    where the signs of $u$ and $v$ should be chosen such that $uv$ has the same sign as $y$. Note that $x^2 + y^2 \geq 0$. By \myref{lemma-non-negative-real-number-has-square-root} we know that $\sqrt{x^2+y^2} \in \R$. Also $\sqrt{x^2+y^2} \geq \sqrt{x^2} = |x| \geq x$ so $-x + \sqrt{x^2+y^2} \geq 0$. Therefore by \myref{lemma-non-negative-real-number-has-square-root} again we see $u, v \in \R$. Without loss of generality let $u$ and $v$ both be positive. Let $w = u + vi \in \C$. Note that
    \begin{align*}
        uv &= \sqrt{\frac{x + \sqrt{x^2+y^2}}{2}}\sqrt{\frac{-x + \sqrt{x^2+y^2}}{2}}\\
        &= \sqrt{\left(\frac{x + \sqrt{x^2+y^2}}{2}\right)\left(\frac{-x + \sqrt{x^2+y^2}}{2}\right)}\\
        &= \sqrt{\frac{\left(x + \sqrt{x^2+y^2}\right)\left(-x + \sqrt{x^2+y^2}\right)}{4}}\\
        &= \sqrt{\frac{(x^2+y^2)-x^2}{4}}\\
        &= \sqrt{\frac{y^2}4}\\
        &= \frac y2.
    \end{align*}
    Thus we see
    \begin{align*}
        w^2 &= (u+vi)^2\\
        &= u^2 + 2uvi + (vi)^2\\
        &= u^2 + 2uvi - v^2\\
        &= \frac{x + \sqrt{x^2+y^2}}{2} + 2\left(\frac y2\right)i - \frac{-x + \sqrt{x^2+y^2}}{2}\\
        &= x + yi
    \end{align*}
    which therefore means that $\sqrt{x+yi} = \sqrt{z} \in \C$.
\end{questions}

\subsection*{Problems}
No problems.
