\section{Elementary Number Theory}
\begin{questions}
    \item $-210 = 11 - 13 \times 17$, so $a = 11$ and $b = 17$.

    \item $\gcd(-112, -35) = 7$ since $-112 = -16 \times 7$ and $-35 = -5 \times 7$, with 7 being the largest integer that achieves this.

    \item $\lcm(-112, -35) = 560$ since $560 = -5 \times -112$ and $-35 = -16 \times -35$, with 560 being the smallest \textit{positive} integer that achieves this.

    \item \begin{partquestions}{\roman*}
        \item $\gcd(42, 70) = 14$ since $42 = 3 \times 14$ and $70 = 5 \times 14$, and 14 is the largest integer achieving this.

        \item $\lcm(42, 70) = \frac{42 \times 70}{\gcd(m, n)} = \frac{2940}{14} = 210$ by \myref{prop-product-of-gcd-and-lcm}.

        \item Note that $x = 2$ and $y = -1$ works as $42 \times 2 + 70 \times (-1) = 84 - 70 = 14$.
    \end{partquestions}

    \item $44100 = 2^2 \times 3^2 \times 5^2 \times 7^2$.

    \item \begin{partquestions}{\alph*}
        \item $17 \mod 5 = 2$ since $17 = 3 \times 5 + 2$ by Euclid's division lemma (\myref{lemma-euclid-division}).
        \item As $19 = 3 \times 5 + 4$, thus $19 \equiv 4 \pmod 5$. Hence $x = 4$.
        \item $A \mod n$ equals $5 \mod 3$ which is 2.
    \end{partquestions}

    \item $-n = (-1) \times 2n + n$, which means that $-n \equiv n \pmod{2n}$.

    \item Finding the last two digits of a number is the same as finding the remainder of that number when divided by 100. We note $778899 \equiv 99 \pmod{100}$, so $778899^{112233} \equiv 99^{112233} \pmod{100}$. Furthermore, $99 \equiv -1 \pmod{100}$, so $99^{112233}\equiv (-1)^{112233} \equiv -1 \equiv 99 \pmod{100}$. Hence the last two digits of $778899^{112233}$ are both 9.

    \item Note $123 \equiv 3 \pmod 5$. One can easily find by trial and error that 2 is the multiplicative inverse of 3, since $3 \times 2 = 6 \equiv 1 \pmod 5$. Hence the multiplicative inverse of 123 is 2 modulo 5.
\end{questions}
