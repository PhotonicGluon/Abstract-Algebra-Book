\setpartpreamble[u][\textwidth]{
    \quoteattr{
        Aus dem Paradies, das Cantor uns geschaffen, soll uns niemand vertreiben k\"{o}nnen.\\
        \textnormal{(No one shall expel us from the Paradise that Cantor has created.)}
    }
    {
        David Hilbert, 1926
    }
    {
        \cite[p.~170]{hilbert_1926}
    }

    A firm foundation is required to build the core of abstract algebra. Without such a foundation, the claims made would be unwarranted and unjustified, and the proofs of those claims would also seem baffling to the uninitiated. This section provides the tools and techniques needed to understand the subject matter in later sections.

    We start with the fundamentals: sets, logic, and proof writing. Sets are a fundamental object in mathematics and will be used countless times in abstract algebra. We then introduce first-order logic notation and its meaning as an overview of how statements are formed in later parts. We also explore the properties of such statements and discuss numerous ways to prove them. We spend a lot of time on proof writing as we will employ various types of proofs in abstract algebra, and it is critical to understand why they are valid proofs and how they work.

    Although this book assumes an understanding of high-school algebra, it is helpful to recapitulate the symbols, terminology, and results used there. Although such algebraic skills will have limited use in group theory, they will be critical when we move on to later parts involving polynomials and algebraic manipulation.

    Afterwards, we examine relations and functions. Although we do not discuss relations in great detail, we highlight the important properties of equivalence relations and link functions to relations.

    Elementary number theory and modular arithmetic will also often arise, as these fields are highly integrated within abstract algebra. Readers are advised to look at the notation of divisibility, Euclid's division lemma, and congruence modulo $n$ in particular. Other results are essential but are used less often.

    Unlike the other parts, not all results will have a proof; most of these results are outside the scope of abstract algebra, and their proofs can be easily found online.
}
\part{Preliminaries}
