\section{Sets}
\subsection*{Exercises}
\begin{questions}
    \item \begin{partquestions}{\alph*}
        \item True, as both 1 and 2 appear in the set $\{1, 2, 3, 4\}$.
        \item False, 3 does not appear in $\{1, 2, 4\}$.
        \item True. Any set is a subset of itself, including the empty set.
        \item False, the set $S$ does not contain any element that is not in $S$. That is, $S \subseteq S$ but not $S \subset S$.
        \item True. $S$ is indeed an element of $\{S, \emptyset\}$.
        \item True. The set containing S is not an element of $\{S, \emptyset\}$.
        \item False, the set $S$ is not a subset of the set $\{S, \emptyset\}$.
        \item True. The set containing $S$ is a subset of the set containing $S$ and the empty set.
    \end{partquestions}

    \item \begin{partquestions}{\alph*}
        \item True. The set of elements that are in either $S$ or $R$ is indeed $\{1, 2, 3, 4, 5\}$.
        \item False, $S \cup U = \{1, 2, 3, 4, (2, 2), (3, 3), (5, 5)\}$.
        \item True. The set of elements that are in both $S$ and $T$ is indeed $\{2, 3\}$.
        \item True. $T$ and $U$ share no elements in common, so their intersection is empty.
        \item True. The elements that are in $S$ but not in $T$ are indeed 1 and 4.
        \item False, $S \setminus \{1, 4\} = \{2, 3\}$, not $T = \{2, 3, 5\}$.
        \item True.
        \item True. $(S \cup T)^2 = \{(1,1), (2,2), (3,3), (4,4), (5,5)\}$, so
        \[
            U = \{(2,2), (3,3), (5,5)\} \subset (S \cup T)^2.
        \]
    \end{partquestions}

    \item We note $S$ are all the non-positive rational numbers, and $T = \{-2, 0, 2, \dots, 8, 10\}$. Hence $S \cap T$ has only two elements, namely $-2$ and $0$.
\end{questions}

\subsection*{Problems}
\begin{questions}
    \item We can only be sure that $A = B$ if $n = 0$, i.e. $A = B = \emptyset$.

    To see this, suppose $n$ is non-zero, meaning $n \geq 1$. Then note that we can find the sets $A = \{1, 2, 3, \dots, n\}$ and $B = \{0, 2, 3, \dots, n\}$, which are two sets with cardinality $n$ but are not equal.

    Therefore the only value of $n$ that works is $n = 0$.

    \item We first express $A$ and $B$ in a `nicer' way to work with.

    For $A$, note that $x^2 - x - 2 \leq 0$ is the same as $(x+1)(x-2) \leq 0$, which means that $x \in [-1, 2]$. Therefore $A = \R \cap [-1, 2] = [-1, 2]$.

    For $B$, we see $12 - x - x^2 > 0$ is the same as $(x+4)(x-3) < 0$, meaning $x \in (-4, 3)$. Thus $B = [0, \infty) \cap (-4, 3) = [0, 3)$.

    \begin{partquestions}{\alph*}
        \item $A \cap B = [-1, 2] \cap [0, 3) = [0, 2]$.
        \item $A \cup B = [-1, 2] \cup [0, 3) = [-1, 3)$.
        \item $B \setminus A = [0, 3) \setminus [-1, 2] = (2, 3)$.
        \item Note $A \setminus B = [-1, 2] \setminus [0, 3) = [-1, 0)$. This is a subset of $[-1, 0)$, but not a \textit{proper} subset of $[-1, 0)$. Therefore $A\setminus B \subset [-1, 0)$ is false.
    \end{partquestions}

    \item Again, we express $A$ and $B$ in a `nicer' way to work with.

    For $A$, note we can rearrange $5x+2y+3=0$ to be $y=-\frac52x-\frac32$. For $B$, we can rearrange $2x^2+5x+2y+1=0$ to be $y=-x^2-\frac52x-\frac12$. Therefore
    \begin{align*}
        A &= \left\{\left(x, -\frac52x-\frac32\right) \vert x \in \Z \right\},\\
        B &= \left\{\left(x, -x^2-\frac52x-\frac12\right) \vert x \in [0, 1] \right\},
    \end{align*}
    which are both infinite sets.

    \begin{partquestions}{\alph*}
        \item $A \cap B$ represents the solutions to the equation $-\frac52x-\frac32 = -x^2-\frac52x-\frac12$. This equation has 2 solutions, namely $x = -1$ and $x = 1$, which have $y$ values of 1 and -4 respectively. Therefore $|A \cap B| = 2$, where $A \cap B = \{(-1, 1), (1, -4)\}$.
        \item Since $A$ and $B$ are both infinite sets, their union is also infinite. Therefore $|A \cup B| = \infty$.
    \end{partquestions}

    \item We work slowly:
    \begin{align*}
        A \cap (B \setminus C) &= \{x \vert x \in A \text{ and } x \in (B \setminus C)\}\\
        &= \{x \vert x \in A \text{ and } (x \in B \text{ and } x \notin C)\}\\
        &= \{x \vert (x \in A \text{ and } x \in B) \text{ and } x \notin C\}\\
        &= \{x \vert x \in A \cap B \text{ and } x \notin C\}\\
        &= (A \cap B) \setminus C.
    \end{align*}
\end{questions}
