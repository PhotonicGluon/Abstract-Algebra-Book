\section{Algebra}
\subsection*{Exercises}
\begin{questions}
    \item \begin{partquestions}{\alph*}
        \item $\displaystyle \sum_{i=3}^{5}(7ix+11) = (7\times3x + 11) + (7\times4x + 11) + (7\times5x + 11) = 84x + 33$.
        \item $\displaystyle \sum_{x=3}^{5}(7ix+11) = (7i\times3 + 11) + (7i\times4 + 11) + (7i\times5 + 11) = 84i + 33$.
        \item Since the upper bound is smaller than the lower bound, the sum evaluates to 0.
        \item $\displaystyle \sum_{i=4}^{8}ijk = 4jk + 5jk + 6jk + 7jk + 8jk = 30jk$.
        \item $\displaystyle \sum_{j=4}^{8}ijk = 4ik + 5ik + 6ik + 7ik + 8ik = 30ik$.
        \item $\displaystyle \sum_{n=4}^{8}ijk = ijk + ijk + ijk + ijk + ijk = 5ijk$.
        \item $\displaystyle \sum_{i=3}^{7}13 = 13 + 13 + 13 + 13 + 13 = 65$.
        \item $\displaystyle \sum_{i=1}^{3}i^2 + \sum_{j=4}^{6}j^2 = (1^2 + 2^2 + 3^2) + (4^2 + 5^2 + 6^2) = 91$.
        \item One notes
        \begin{align*}
            \sum_{i=1}^{3}\left(\sum_{j=5}^{7}(i+j)\right)\\
            &= \sum_{i=1}^{3}\left((i+5) + (i+6) + (i+7)\right)\\
            &= \sum_{i=1}^{3}\left(3i+18\right)\\
            &= (3\times1 + 18) + (3\times2 + 18) + (3\times3 + 18)\\
            &= 72.
        \end{align*}
    \end{partquestions}

    \item Note $\displaystyle \sum_{i=2}^5a_{i-1} = \sum_{i=1}^4a_i = 10$. Also
    \begin{align*}
        200 = \sum_{i=1}^5(a_i+2)^2 &= \sum_{i=1}^5(a_i^2 + 4a_i + 4)\\
        &= \sum_{i=1}^5a_i^2 + 4\sum_{i=1}^5a_i + \sum_{i=1}^54\\
        &= \sum_{i=1}^5a_i^2 + 4\left(\sum_{i=1}^4a_i + \sum_{i=5}^5a_i\right) + \sum_{i=1}^54\\
        &= 100 + 4\left(10 + a_5\right) + 20\\
        &= 160 + 4a_5
    \end{align*}
    which therefore means $a_5 = 10$.

    \item Rearrange $x^2 + 15 < 8|x|$ to become $x^2 - 8|x| + 15 < 0$, meaning $|x|^2 - 8|x| + 15 < 0$. Note $|x|^2 - 8|x| + 15 = (|x|-3)(|x|-5)$, so we are solving $(|x|-3)(|x|-5)<0$. Therefore $3 < |x| < 5$.
    \begin{itemize}
        \item If $|x| = -x$ (i.e., $x$ is negative), then $3 < -x < 5$, which means $-5 < x < -3$.
        \item If $|x| = x$ (i.e. $x$ is non-negative), then $3 < x < 5$.
    \end{itemize}
    Hence $-5 < x < -3$ or $3 < x < 5$.

    \item Note that $|x| + |y| \geq 0$ and $|x + y| \geq 0$ so their square roots exist. Also note
    \begin{align*}
        \left(|x| + |y|\right)^2 &= |x|^2 + 2|x||y| + |y|^2\\
        &= x^2 + 2|xy| + y^2\\
        &\geq x^2 + 2xy + y^2\\
        &= (x+y)^2\\
        &= \left(|x+y|\right)^2
    \end{align*}
    which therefore means $|x| + |y| \geq |x+y|$.

    \item The relevant term is
    \[
        {9 \choose 2}(7x)^7(-3)^{9-7} = 266827932x^7
    \]
    so its coefficient is 266827932.
\end{questions}

\subsection*{Problems}
\begin{questions}
    \item We cannot use the Binomial Theorem (\myref{thrm-binomial}) without commutativity. We just have to expand it slowly.
    \begin{align*}
        (x+y)^3 &= (x+y)(x+y)(x+y)\\
        &= xxx + xxy + xyx + xyy + yxx + yxy + yyx + yyy\\
        &= x^3 + x^2y + xyx + xy^2 + yx^2 + yxy + y^2x + y^3.
    \end{align*}

    \item First we rewrite the equation to become $x^6 - 6x^5 + 6x^4 + 16x^3 - 12x^2 - 24x + 19 = 0$. Now we try to rewrite it using the given substitution.
    \begin{align*}
        &x^6 - 6x^5 + 6x^4 + 16x^3 - 12x^2 - 24x + 19\\
        &=x^4(x^2-2x-2) - 4x^3(x^2-2x-2) + 0x^2(x^2-2x-2)\\
        &\quad\quad+ 8x(x^2-2x-2) + 4(x^2-2x-2) + 27\\
        &= (x^4-4x^3+8x+4)(x^2-2x-2) + 27\\
        &= \left(x^2(x^2-2x-2) - 2x(x^2-2x-2) - 2(x^2-2x-2)\right)(x^2\\
        &\quad\quad- 2x-2) + 27\\
        &= ((x^2-2x-2)(x^2-2x-2))(x^2-2x-2)+27\\
        &= (x^2-2x-2)^3 + 27\\
        &= u^3 + 3^3\\
        &= 0.
    \end{align*}
    Note $u^3 + 3^3 = (u+3)(u^2 - 3u + 9)$, so $u+3 = 0$ or $u^2 - 3u + 9 = 0$. One sees that the quadratic equation has no real solutions; so $u = -3$. Therefore $x^2 - 2x - 2 = -3$, meaning $x^2-2x+1 = 0$, Thus $x = 1$.

    \item \begin{partquestions}{\roman*}
        \item We see that $(x-2)(x^2-3x+1) = (x^3 - 3x^2 + x) - 2x^2 + 6x - 2 = x^3 - 5x^2 + 7x - 2$.

        \item Note $(x-2)(x^2-18) = x^3 - 2x^2 - 18x + 36$. Subtracting 3 on both sides we obtain
        \begin{align*}
            &\frac{4x^3 - 11x^2 - 47x + 106}{(x-2)(x^2-18)} - 3\\
            &= \frac{4x^3 - 11x^2 - 47x + 106 - 3(x^3 - 2x^2 - 18x + 36)}{(x-2)(x^2-18)}\\
            &= \frac{x^3 - 5x^2 + 7x - 2}{(x-2)(x^2-18)}\\
            &= \frac{(x-2)(x^2-3x+1)}{(x-2)(x^2-18)}\\
            &= \frac{(x-2)\left(x - \frac{3+\sqrt5}{2}\right)\left(x - \frac{3-\sqrt5}{2}\right)}{(x-2)(x-3\sqrt2)(x+3\sqrt2)}\\
            &\geq 0.
        \end{align*}
        Therefore $x < -3\sqrt2$ or $\frac{3-\sqrt5}{2} \leq x \leq \frac{3+\sqrt5}{2}$ with $x \neq 2$ or $x > 3\sqrt2$. Hence, when restricting to $x \in [0, 3]$, we see $\frac{3-\sqrt5}{2} \leq x \leq \frac{3+\sqrt5}{2}$ with $x \neq 2$.
    \end{partquestions}

    \item For easier computation later, let $u = |x-3|$. Note that
    \[
        x^2 - 6x + 5 = (x-3)^2 - 4 = |x-3|^2 - 4 = u^2 - 4
    \]
    so the original inequality becomes
    \[
        \frac{3u}{u^2 - 4} > 1.
    \]
    Thus
    \begin{align*}
        \frac{3u}{u^2-4} - \frac{u^2 - 4}{u^2 - 4} &= \frac{-u^2 + 3u + 4}{u^2-4}\\
        &= \frac{-(u+1)(u-4)}{(u+2)(u-2)}\\
        &> 0.
    \end{align*}
    Note $u \geq 0$, so $u+1 > 0$ and $u+2 > 0$. So $\frac{u+1}{u+2} > 0$, meaning that we are only concerned with $\frac{-(u-4)}{u-2} > 0$, i.e. $\frac{u-4}{u-2} < 0$. Therefore $2 < u < 4$, meaning $2 < |x - 3| < 4$.

    Note $2 < |x - 3| < 4$ means $(|x-3| > 2) \land (|x-3| < 4)$. We split the inequality into two parts.
    \begin{itemize}
        \item First we solve $|x-3| > 2$. This means that $x - 3 > 2$ or $x - 3 < -2$, so $x > 5$ or $x < 1$, i.e. $x \in (5, \infty) \cup (-\infty, 1)$.
        \item Now we solve $|x - 3| < 4$. Thus $-4 < x-3 < 4$, meaning $-1 < x < 7$, i.e. $x \in (-1, 7)$.
    \end{itemize}
    Thus the solution we are seeking is the intersection of the two intervals
    \[
        x \in ((5, \infty) \cup (-\infty, 1)) \cap (-1, 7),
    \]
    which therefore means $x \in (-1, 1) \cup (5, 7)$, i.e. $-1 < x < 1$ or $5 < x < 7$.

    \item \begin{partquestions}{\roman*}
        \item Setting $\frac{7x+4}{x^3+3x^2+2x} = \frac{A}{x} + \frac{B}{x+1} + \frac{C}{x+2}$ and multiplying both sides by $x^3+3x^2+2x = x(x+1)(x+2)$ we see that $7x+4 = A(x+1)(x+2) + Bx(x+2) + Cx(x+1)$.
        \begin{itemize}
            \item Setting $x = 0$ we see that $4 = A(0+1)(0+2) = 2A$, meaning $A = 2$.
            \item Setting $x = -1$ we see $7(-1) + 4 = B(-1)(-1+2)$, i.e. $-3 = -B$, meaning $B = 3$.
            \item Setting $x = -2$ we see $7(-2) + 4 = C(-2)(-2+1)$, i.e. $-10 = 2C$, meaning $C = -5$.
        \end{itemize}
        Therefore $\frac{7x+4}{x^3+3x^2+2x} = \frac{2}{x} + \frac{3}{x+1} - \frac{5}{x+2}$.

        \item Note that
        \begin{align*}
            \sum_{r=1}^N \frac{7r+4}{r^3+3r^2+2r} &= \sum_{r=1}^N \left(\frac{2}{r} + \frac{3}{r+1} - \frac{5}{r+2}\right)\\
            &= \frac21 + \frac32 - \frac53\\
            &+ \frac22 + \frac33 - \frac54\\
            &+ \frac23 + \frac34 - \frac55\\
            &+ \frac24 + \frac35 - \frac56\\
            &\cdots\\
            &+ \frac2{N-3} + \frac{3}{N-2} - \frac{5}{N-1}\\
            &+ \frac2{N-2} + \frac{3}{N-1} - \frac{5}{N}\\
            &+ \frac2{N-1} + \frac{3}{N} - \frac{5}{N+1}\\
            &+ \frac2{N} + \frac{3}{N+1} - \frac{5}{N+2}\\
            &= \frac21 + \frac32 + \frac22 - \frac{5}{N+1} + \frac{3}{N+1} - \frac{5}{N+2}\\
            &= \frac92 - \frac{2}{N+1} - \frac{5}{N+2}\\
            &= \frac92 - \frac{7N+9}{N^2+3N+2}
        \end{align*}
        so we see
        \begin{align*}
            &\frac{7N+9}{N^2+3N+2} + \sum_{r=1}^N \frac{7r+4}{r^3+3r^2+2r}\\
            &= \frac{7N+9}{N^2+3N+2} + \left(\frac92 - \frac{7N+9}{N^2+3N+2}\right)\\
            &= \frac92.
        \end{align*}
    \end{partquestions}

    \item We consider a proof by induction. We induct on $n$.

    When $n = 0$, it is clearly true that $0 \times 0! = 0 = (0+1)! - 1$. Thus the statement holds for $n = 0$.

    Now assume that the statement holds for some non-negative integer $k$, meaning
    \[
        \sum_{r=0}^k (r\times r!) = (k+1)! - 1.
    \]
    We are to show that it holds for $k+1$, i.e.
    \[
        \sum_{r=0}^{k+1} (r\times r!) = (k+2)! - 1.
    \]

    We see that
    \begin{align*}
        &\sum_{r=0}^{k+1} (r\times r!)\\
        &= \left(\sum_{r=0}^k (r\times r!)\right) + ((k+1) \times (k+1)!)\\
        &= ((k+1)! - 1) + ((k+1) \times (k+1!)) & (\text{induction hypothesis})\\
        &= (k+1)! + (k+1) \times (k+1)! - 1\\
        &= (1 + (k+1))\times(k+1)! - 1\\
        &= (k+2) \times (k+1)! - 1\\
        &= (k+2)! - 1
    \end{align*}
    proving the statement for $k + 1$.

    By mathematical induction we see
    \[
        \sum_{r=0}^n (r\times r!) = (n+1)! - 1
    \]
    for all non-negative integers $n$.

    \item \begin{partquestions}{\roman*}
        \item We consider three cases.
        \begin{itemize}
            \item The first case is when $x$ and $y$ are both non-negative. Then $|x| = x$, $|y| = y$, and $|xy| = xy$. Thus $|xy| = xy = |x||y|$.
            \item The second case is if $x$ and $y$ are both negative. So $|x| = -x$, $|y| = -y$, and $|xy| = xy$, since the product of two negative real numbers is a positive real number. Therefore $|xy| = xy = (-x)(-y) = |x||y|$.
            \item The last case is when one of them is non-negative and the other is negative. Without loss of generality, assume $x$ is non-negative and $y$ is negative. Then their product is negative. So $|x| = x$, $|y| = -y$, and $|xy| = -xy$. Hence $|xy| = -xy = (x)(-y) = |x||y|$.
        \end{itemize}
        Therefore $|xy| = |x||y|$ for all $x, y \in \R$.

        \item We induct on positive integer values for $n$.

        When $n = 1$ the statement holds trivially since $|x_1| = |x_1|$.

        Assume that the statement holds for some positive integer $k$, i.e. $|x_1\cdots x_k| = |x_1|\cdots|x_k|$. We show that the statement holds for $k+1$, i.e. $|x_1\cdots x_kx_{k+1}| = |x_1|\cdots|x_k||x_{k+1}|$

        Observe
        \begin{align*}
            &|x_1x_2\cdots x_kx_{k+1}|\\
            &= |(x_1x_2\cdots x_k)(x_{k+1})|\\
            &= |x_1x_2\cdots x_k||x_{k+1}| & (\text{by part (\textbf{i})})\\
            &= (|x_1||x_2|\cdots|x_k|)|x_{k+1}| & (\text{induction hypothesis})\\
            &= |x_1||x_2|\cdots|x_k||x_{k+1}|
        \end{align*}
        proving the statement for $k + 1$.

        Thus $|x_1x_2\cdots x_n| = |x_1||x_2|\cdots|x_n|$ for all $x_1, x_2, \dots, x_n \in \R$ by mathematical induction.
    \end{partquestions}
\end{questions}
