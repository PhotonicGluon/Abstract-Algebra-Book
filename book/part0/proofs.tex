\chapter{Proof Writing}
Proofs compel belief. Without proofs, statements are just meaningless words without justification. A series of logical steps must be provided in order to prove a claim. We explore the basics of proof writing here.

\section{Direct Proof}
In essence, a direct proof\index{proof!direct} for the statement ``if $p$ then $w$'' would begin by assuming $p$ is true and showing that this forces $q$ to be true. We don't need to worry about the case of $p$ being false, since $p \implies q$ is vacuously true in the case when $p$ is false.
\begin{example}
    We prove that ``if $n$ is an odd integer then $n^2$ is odd'' using direct proof. We first write out the proof step-by-step.
    \begin{proof}
        Suppose $n\in\Z$ is an odd integer.

        Then $n$ can be written in the form $n = 2k + 1$ where $k$ is an integer.

        Hence
        \begin{align*}
            n^2 &= (2k+1)^2\\
            &= 4k^2 + 4k + 1\\
            &= 2(2k^2 + 2k) + 1.
        \end{align*}

        We see $n^2$ is one more than a multiple of 2.

        Thus $n^2 = 2(2k^2 + 2k) + 1$ is odd.
    \end{proof}

    Of course, proofs are often not written this way; we often make it more succinct and remove unnecessary paragraph breaks. We produce a more `natural' proof below.
    \begin{proof}
        Suppose $n\in\Z$ is an odd integer. Then $n$ can be written in the form $n = 2k + 1$ where $k$ is an integer. Hence
        \begin{align*}
            n^2 &= (2k+1)^2\\
            &= 4k^2 + 4k + 1\\
            &= 2(2k^2 + 2k) + 1,
        \end{align*}
        so $n^2$ is one more than a multiple of 2, meaning $n^2 = 2(2k^2 + 2k) + 1$ is odd.
    \end{proof}
\end{example}
\begin{example}
    Let $x$ and $y$ be positive real numbers. We prove that ``if $x \leq y$ then $\sqrt x \leq \sqrt y$.''
    \begin{proof}
        Suppose $x \leq y$. This means $x - y \leq 0$. As $x$ and $y$ are positive real numbers, we may write that inequality as $(\sqrt x)^2 - (\sqrt y)^2 \leq 0$. We can factor the left hand side as $(\sqrt x + \sqrt y)(\sqrt x - \sqrt y) \leq 0$. Now $\sqrt x + \sqrt y$ is always positive, so if the entire inequality is less than or equal to 0 we must have $\sqrt x - \sqrt y \leq 0$. Therefore $\sqrt x \leq \sqrt y$.
    \end{proof}
\end{example}

\begin{exercise}
    Let $x$ be a positive real number. Prove that $x(1-x) > 0$ if $0 < x < 1$.
\end{exercise}

\section{Proof by Division Into Cases}
Division into cases can be considered as a variant of direct proof. We split a difficult statement into different cases which are individually easier to prove, and then conjoin them together in order to prove the full statement.

\begin{example}
    We look at a proof of the statement ``$1 + (-1)^n(2n-1)$ is a multiple of 4 if $n$ is an integer''.
    \begin{proof}[Proof (see {\cite[p.~124]{hammack_2018}})]
        Suppose $n$ is an integer. Then, $n$ is either odd or even. We look at two cases.
        \begin{itemize}
            \item If $n$ is odd, then $(-1)^n = -1$ and $n = 2k+1$ for some integer $k$. Thus
            \[
                1 + (-1)^n(2n-1) = 1 - (2(2k+1)-1) = 4(-k)
            \]
            which is a multiple of 4.
            \item If $n$ is even, then $(-1)^n = 1$ and $n = 2k$ for some integer $k$. Thus
            \[
                1 + (-1)^n(2n-1) = 1 + (2(2k)-1) = 4k
            \]
            which is a multiple of 4.
        \end{itemize}
        Hence, in both cases, $1 + (-1)^n(2n-1)$ is a multiple of 4.
    \end{proof}
\end{example}

\begin{example}
    We look at a proof of the statement ``if two integers have opposite parity, then their sum is odd''. Note that ``opposite parity'' means that one integer is odd and one integer is even.
    \begin{proof}
        Suppose $m$ and $n$ are two integers with opposite parity. We consider two cases.
        \begin{itemize}
            \item If $m$ is even and $n$ is odd, then we may write $m = 2a$ and $n = 2b + 1$ where $a$ and $b$ are some integers. Thus
            \begin{align*}
                m + n &= (2a) + (2b + 1)\\
                &= 2(a+b) + 1
            \end{align*}
            which is one more than a multiple of 2, so $m + n$ is odd.
            \item On the other hand, if $m$ is odd and $n$ is even, then we may write $m = 2a + 1$ and $n = 2b$ where $a$ and $b$ are some integers. Thus
            \begin{align*}
                m + n &= (2a + 1) + (2b)\\
                &= 2(a+b) + 1
            \end{align*}
            which is one more than a multiple of 2, so $m + n$ is odd.
        \end{itemize}
        Hence, in either case, $m + n$ is odd.
    \end{proof}

    Now, the two cases in the above proof are almost the same, except for whether $m$ or $n$ is the even integer. It is more productive to just do one case and indicate that the other case is nearly (or exactly) identical to the current case. The phrase ``without loss of generality'' (or WLOG for short) is a way to signpost to the reader that the proof is treating only one of several nearly identical cases. We produce a more succinct version of the proof above.

    \begin{proof}[Proof (see {\cite[p.~126]{hammack_2018}})]
        Suppose $m$ and $n$ are two integers with opposite parity. Without loss of generality, suppose $m$ is even and $n$ is odd. Therefore $m = 2a$ and $n = 2b + 1$ for some integers $a$ and $b$. Hence
        \begin{align*}
            m + n &= (2a) + (2b + 1)\\
            &= 2(a+b) + 1
        \end{align*}
        which is one more than a multiple of 2, so $m + n$ is odd.
    \end{proof}
\end{example}

\begin{exercise}
    Prove that $m + n$ is even if the integers $m$ and $n$ have the same parity (i.e., both odd or both even).
\end{exercise}

\section{Contrapositive Proof}
We now look at a \textbf{contrapositive proof}\index{proof!contrapositive}. Recall that $p \implies q$ is logically equivalent to $\lnot q \implies \lnot p$. Thus, a contrapositive proof for ``if $p$ then $q$'' would begin by assuming $\lnot q$ is true and deducing that this means that $\lnot p$ is true.

Generally, we would want to prove in the direction from simple to complex. So if $p$ is more complex than $q$, we may consider using a contrapositive proof.

\begin{example}\label{example-if-(n-1)(n-5)-is-even-then-n-is-odd}
    Suppose $n$ is an integer. We prove the statement ``if $n^2 - 6n + 5$ is even then $n$ is odd''. We note that a direct proof would be tedious and problematic. Using a contrapositive proof would be easier.

    We first note that the contrapositive statement that we want to prove is ``if $n$ is \textbf{not} odd, then $n^2 - 6n + 5$ is \textbf{not} even'', that is, ``if $n$ is even, then $n^2 - 6n + 5$ is odd''.
    \begin{proof}[Proof (see {\cite[p.~130]{hammack_2018}})]
        We consider a proof by contrapositive.

        Suppose $n$ is even. Then $n = 2k$ where $k$ is an integer. Note
        \begin{align*}
            n^2 - 6n + 5 &= (2k)^2 - 6(2k) + 5\\
            &= 4k^2 - 12k + 5\\
            &= (4k^2 - 12k + 4) + 1\\
            &= 2(2k^2 - 6k + 2) + 1
        \end{align*}
        which means $n^2 - 6n + 5$ is odd.
    \end{proof}
\end{example}

\begin{example}
    Suppose $x$ and $y$ are real numbers. We prove the statement ``$x \leq y$ if $x^3 + xy^2 \leq x^2y + y^3$'' using a contrapositive proof.

    We first note that the contrapositive statement that we want to prove is ``if $x > y$ then $x^3 + xy^2 > x^2y + y^3$''.

    \begin{proof}[Proof (cf. {\cite[p.~130]{hammack_2018}})]
        We consider a proof by contrapositive.

        Assume $x > y$. Then $x - y > 0$. Also, since $x > y$, thus $x$ and $y$ are not both zero. Hence $x^2 + y^2 > 0$.
        Observe
        \[
            (x-y)(x^2+y^2) > 0 \times (x^2+y^2) = 0
        \]
        so $(x-y)(x^2+y^2) = x^3 + xy^2 - x^2y - y^3 > 0$. Therefore $x^3 + xy^2 > x^2y + y^3$.
    \end{proof}
\end{example}

\begin{exercise}
    Suppose that $a$ and $b$ are integers. Prove that if $a(b^2 + 5)$ is even then either $a$ is even or $b$ is odd.
\end{exercise}

\section{Proof by Contradiction}
The third proof technique is called a \textbf{proof by contradiction}\index{proof!contradiction}. This method can be used to prove any kind of statement. The basic idea is to assume that the statement we want to prove is false, and then show that this assumption leads to a contradiction. A proof by contradiction for the statement ``$p$'' (yes, just $p$) would start by assuming that $p$ is false, and then showing that this assumption would lead to a contradiction, which means that $p$ is \textit{not} false, i.e. $p$ is true.
\begin{remark}
    In fact, what we are showing is that the statement ``$\lnot p \implies \textbf{false}$'' is true.
\end{remark}

Usually, when writing a proof by contradiction, we would like to inform the reader that a proof by contradiction is being employed. Language such as ``by way of contradiction'', ``towards a contradiction'', ``suppose for the sake of contradiction'' etc. may be used to signpost the use of a proof by contradiction.
\begin{remark}
    Some authors would also signal the use of contradiction by using the initialism ``BWOC'' (by way of contradiction).
\end{remark}

\begin{example}\label{example-sqrt2-is-irrational}
    We present the classic proof by contradiction for the statement ``$\sqrt 2$ is irrational''.
    \begin{proof}
        By way of contradiction, assume that $\sqrt2 = \frac ab$ for some integers $a$ and $b$. Furthermore let this fraction be fully reduced; in particular, this means that $a$ and $b$ are not both even. Squaring both sides yields $2 = \frac{a^2}{b^2}$, meaning  $a^2 = 2b^2$. Hence $a^2$ is even, so write $a = 2c$ where $c$ is an integer. This leads to $2b^2 = (2c)^2 = 4c^2$ which implies $b^2 = 2c^2$. Hence $b$ is even, which contradicts the fact that $a$ and $b$ are not both even.

        Hence, $\sqrt 2$ is irrational.
    \end{proof}
\end{example}
\begin{remark}
    It is not necessary to have the final statement that ``$\sqrt 2$ is irrational'' (or, more generally, ``$P$ is true'') as it is implied from the proof by contradiction.
\end{remark}

\begin{example}
    We prove the statement that ``for every positive rational number $x$, there exists a positive rational number $y$ such that $y < x$'' by way of contradiction.

    We note that the negation of the above statement is ``there exists a rational number $x$ such that for every positive rational number $y$ we have $y \geq x$''.
    \begin{proof}
        Suppose for the sake of contradiction that there exists a rational number $x$ such that for every positive rational number $y$ we have $y \geq x$. Write $x = \frac pq$ where $p$ and $q$ are positive integers.

        Now consider the rational number $\frac{p-1}{q}$. Clearly $\frac{p-1}{q} < \frac pq = x$. By assumption, every positive rational number $y$ satisfies $y \geq x$. Hence, $\frac{p-1}{q}$ is non-positive, meaning $\frac{p-1}{q} \leq 0$. Since $q$ is positive, hence $p - 1 \leq 0$ which means $p \leq 1$. But as $p$ is a positive integer, we conclude $p = 1$. Hence $x = \frac 1q$.

        We now consider the rational number $\frac{1}{q+1}$. Clearly $\frac{1}{q+1} < \frac{1}{q} = x$. By assumption we must conclude that $\frac{1}{q+1}$ is non-positive. However, $1 > 0$ and $q + 1 > 0$, so $\frac{1}{q+1}$ is positive. Hence we have the fact that $\frac{1}{q+1}$ is positive and non-positive simultaneously, leading to a contradiction.
    \end{proof}
\end{example}
\begin{remark}
    The statement above is one where a direct proof would be easier. We provide a direct proof of it below.
    \begin{proof}
        Since $x$ is a positive rational number write $x = \frac pq$ where $p$ and $q$ are positive integers. Then set $y = \frac{p}{q+1}$. Clearly $\frac{p}{q+1} < \frac{p}{q} = x$ and $\frac{p}{q+1}$ is positive, hence we have found a $y$ such that $y < x$.
    \end{proof}
\end{remark}

\begin{exercise}
    Prove that no integers $a$ and $b$ exist such that $2a + 4b = 1$.
\end{exercise}

We now look at a proof by contradiction for conditional statements. Recall that $\lnot(p \implies q) \equiv p \land \lnot q$. Hence, to prove the statement ``if $p$ then $q$'' via contradiction, we would start by assuming that $p \land \lnot q$, and then showing that this assumption would lead to a contradiction, which means that $p \implies q$ is \textit{not} false, i.e. $p \implies q$ is true.

\begin{example}
    Suppose $a$ and $b$ are real numbers. We prove the statement ``if $a$ is rational and $ab$ is irrational then $b$ is irrational'' using a proof by contradiction.

    We note that the statement we want to contradict is ``$a$ is rational and $ab$ is irrational \textbf{and} $b$ is \textbf{not} irrational'', i.e. ``$a$ is rational and $b$ is rational and $ab$ is irrational''.
    \begin{proof}
        By way of contradiction assume $a$ is rational, $b$ is rational, and $ab$ is irrational. We may then write $a = \frac mn$ and $b = \frac pq$ where $m, n, p, q \in \Z$. Hence $ab = \left(\frac mn\right)\left(\frac pq\right) = \frac{mp}{nq}$ which is clearly rational. Therefore we have that $ab$ is irrational (by assumption) and $ab$ is rational, a contradiction.
    \end{proof}
\end{example}

\begin{example}
    Suppose $a$, $b$, and $c$ are integers. We prove the statement that ``if $a^2 + b^2 = c^2$ then at least one of $a$ or $b$ is even'' using a proof by contradiction.

    We note that the statement we want to contradict is ``$a^2 + b^2 = c^2$ \textbf{and not} (at least one of $a$ or $b$ is even)'', i.e. ``$a^2 + b^2 = c^2$ \textbf{and} both $a$ and $b$ are odd''.
    \begin{proof}
        Seeking a contradiction, assume that $a^2 + b^2 = c^2$ and both $a$ and $b$ are odd. Thus we may write $a = 2m + 1$ and $b = 2n + 1$ where $m$ and $n$ are integers. Hence
        \begin{align*}
            a^2 + b^2 &= (2m+1)^2 + (2n+1)^2\\
            &= (4m^2+4m+1) + (4n^2+4n+1)\\
            &= 4m^2 + 4n^2 + 4m + 4n + 2\\
            &= 2(2m^2 + 2n^2 + 2m + 2n +1)
        \end{align*}
        which means that $c^2 = a^2 + b^2$ is even. Hence $c$ is even, which means we may write $c = 2k$ where $k$ is an integer. This leads to
        \[
            c^2 = 4k^2 = 2(2m^2 + 2n^2 + 2m + 2n + 1) = a^2 + b^2.
        \]
        Clearly $4k^2$ is a multiple of 4, while $2(2m^2 + 2n^2 + 2m + 2n + 1)$ is not. Yet, they are equal to each other, a contradiction.
    \end{proof}
\end{example}

\begin{exercise}
    Prove that $\frac{a+b}{2} \geq \sqrt{ab}$ if $a$ and $b$ are positive real numbers by way of contradiction.
\end{exercise}

Despite the power of proof by contradiction, it's best to use it only when the direct and contrapositive approaches do not seem to work.
\begin{example}
    Suppose $n$ is an integer. We prove the statement ``if $n^2 - 6n + 5$ is even then $n$ is odd'' using a proof by contradiction.
    \begin{proof}
        Working towards a contradiction, assume $n^2 - 6n + 5$ is even and $n$ is \textbf{not} odd, i.e. $n$ is even. Then $n = 2k$ for some integer $k$. Note that
        \begin{align*}
            n^2 - 6n + 5 &= (2k)^2 - 6(2k) + 5\\
            &= 4k^2 - 12k + 5\\
            &= (4k^2 - 12k + 4) + 1\\
            &= 2(2k^2 - 5k + 2) + 1
        \end{align*}
        which means that $n^2 - 6n + 5$ is odd. Hence, $n^2 - 6n + 5$ is even (by assumption) and $n^2 - 6n + 5$ is odd (as above), a contradiction.
    \end{proof}
    While there is nothing wrong with this proof, notice that part of it assumes that $n$ is even and concludes that  $n^2 - 6n + 5$ is odd, which is the contrapositive approach done in \myref{example-if-(n-1)(n-5)-is-even-then-n-is-odd}.
\end{example}

\section{Proof by Mathematical Induction}
Mathematical induction\index{proof!induction} is a method for proving that a predicate $P(n)$ is true for every positive integer $n$, that is, that the infinitely many statements $P(1), P(2), P(3), \dots$ all hold. A proof by induction consists of two steps.
\begin{itemize}
    \item The first, the \textbf{base case}\index{proof!induction!base case}, proves the statement for $n = 1$ without assuming any knowledge of other cases. In other words, the base case proves the statement $P(1)$.
    \item The second, the \textbf{induction step}\index{proof!induction!induction step}, proves that if the statement holds for any given case $n = k$, then it must also hold for the next case $n = k + 1$. In other words, $\forall k \in \mathbb{N}, (P(k) \implies P(k+1))$. The assumption that $P(k)$ being true is called the \textbf{induction hypothesis}.
\end{itemize}
These two steps establish that the predicate $P(n)$ holds for all positive integers $n$.

The base case does not necessarily need to begin with $n = 1$. Sometimes we may begin with $n = 0$, and possibly with any fixed natural number $n = N$, establishing the truth of the statement for all natural numbers $n \geq N$.

In summary, mathematical induction involves two steps:
\begin{itemize}
    \item \textbf{Base Case}: Prove the statement for the initial value.
    \item \textbf{Induction Step}: Prove that for every $n$, if the statement holds for $n$, then it holds for $n + 1$.
\end{itemize}

\begin{example}
    We prove the famous identity
    \[
        1 + 2 + 3 + \cdots + n = \frac{n(n+1)}2
    \]
    using mathematical induction.
    \begin{proof}
        When $n = 1$, the left hand side is 1; the right hand side is $\frac{1(1+1)}{2} = 1$. Thus the initial case is true.

        Now assume that the statement holds for some positive integer $k$, meaning
        \[
            1 + \cdots + k = \frac{k(k+1)}2.
        \]
        We need to show that the statement holds for $k+1$, meaning
        \[
            1 + \cdots + k + (k+1) = \frac{(k+1)(k+2)}2.
        \]
        We work slowly.
        \begin{align*}
            1 + \cdots + k + (k+1) &= (1 + \cdots + k) + (k+1)\\
            &= \frac{k(k+1)}{2} + (k+1) & (\text{by hypothesis})\\
            &= \frac{k(k+1)}2 + \frac{2(k+1)}{2}\\
            &= \frac{k(k+1) + 2(k+1)}2\\
            &= \frac{(k+1)(k+2)}2
        \end{align*}
        which proves the case for $k + 1$. Hence $1 + 2 + 3 + \cdots + n = \frac{n(n+1)}2$.
    \end{proof}
\end{example}

\begin{example}
    Suppose $x > -1$. We will prove that $(1+x)^n \geq 1+nx$ if $n$ is a positive integer.
    \begin{proof}[Proof (cf. {\cite[p.~186]{hammack_2018}})]
        When $n = 1$, the left hand side is $(1+x)^1 = 1+x$ which is exactly the right hand side. Thus the base case is true.

        Assume that the statement holds for some positive integer $k$, i.e. $(1+x)^k \geq 1+kx$. We show that the statement holds for $k+1$, i.e. $(1+x)^{k+1} \geq 1+(k+1)x$.

        We first note that since $x>-1$, thus $1+x > 0$. We start with our induction hypothesis.
        \begin{align*}
            (1+x)^k &\geq 1+kx\\
            (1+x)^k(1+x) &\geq (1+kx)(1+x) & (\text{since }1+x > 0)\\
            (1+x)^{k+1} &\geq 1 + x + kx + kx^2\\
            &= 1+(k+1)x + kx^2\\
            &> 1+(k+1)x
        \end{align*}
        Hence we see $(1+x)^{k+1} \geq 1+(k+1)x$, meaning that the statement is true for $k+1$.

        Therefore $(1+x)^n \geq 1+nx$ if $n$ is a positive integer.
    \end{proof}
\end{example}

\begin{exercise}
    Prove using mathematical induction that $a^2 - 1$ is a multiple of 8 for all positive odd integers $a$.
\end{exercise}

We now look at another form of mathematical induction, called \textbf{strong induction}\index{proof!induction!strong}. Unlike regular induction, strong induction assumes that all preceding cases are true, and proves the truth of the next case.

Strong mathematical induction involves two steps:
\begin{itemize}
    \item \textbf{Base Cases}: Prove the statement for the initial values.
    \item \textbf{Induction Step}: Prove that for every $n$, if the statement holds for all (positive) integers $m$ that are at most $n$, then it holds for $n + 1$.
\end{itemize}

\begin{example}
    We prove that every integer $n \geq 8$ can be expressed in the form $3a + 5b$ where $a$ and $b$ are non-negative integers.
    \begin{proof}
        We use strong induction on $n$.

        We show the base cases of 8, 9, and 10 hold:
        \begin{itemize}
            \item When $n = 8$, we have $8 = 3 + 5$.
            \item When $n = 9$, we have $9 = 3 \times 3 + 5 \times 0$.
            \item When $n = 10$, we have $10 = 3 \times 0 + 5 \times 2$.
        \end{itemize}

        Now assume that for some positive integer $k \geq 8$, every integer $m$ satisfying $8 \leq m \leq k$ results in the statement being true, i.e. $m$ can be written in the form $3a + 5b$. We are to show that the statement for $k+1$ is true, i.e. $k+1$ can be expressed in the form $3a + 5b$.

        By hypothesis, $k - 2$ can be expressed in the form $3a+5b$. Hence $k+1 = (k-2) + 3 = 3(a+1) + 5b$, proving the statement for $k+1$.

        Therefore by mathematical induction, every integer $n \geq 8$ can be expressed in the form $3a + 5b$ where $a$ and $b$ are non-negative integers.
    \end{proof}
\end{example}

\begin{example}
    Suppose we have a candy bar with $n \geq 1$ pieces. We prove that $n - 1$ breaks are required to break it into $n$ individual pieces.
    \begin{proof}
        We consider strong induction on $n$.

        When $n = 1$, the candy bar is already in individual pieces. Thus $1 - 1 = 0$ breaks are required.

        Now assume that for some integer $k \geq 1$, every candy bar with $m$ pieces (where $1 \leq m \leq k$) can be broken into individual pieces using $m - 1$ breaks. We are to show that a candy bar with $k + 1$ pieces can be broken into individual pieces using $k$ pieces.

        Consider any possible break in a $(k + 1)$-piece candy bar. This will break the candy bar into two pieces, one with $m$ pieces and one with $(k+1)-m$ pieces. Note $1 \leq m \leq k$ which means $1 \leq (k+1)-m \leq k$, so both pieces are non-empty and not the entire bar.

        Applying the induction hypothesis on both bars, we see that the $m$-piece bar can be broken into individual pieces using $m-1$ breaks; the $((k+1)-m)$-piece bar can be broke into individual pieces using $((k+1)-m)-1 = k-m$ breaks. Adding the first break needed to separate the candy bar into these two pieces, that means a total of $(m-1) + (k-m) + 1 = k$ breaks are needed, proving the statement for $k+1$.

        Therefore by mathematical induction, $n - 1$ breaks are required to break a candy bar of $n$ pieces into individual pieces.
    \end{proof}
\end{example}

\begin{exercise}
    Prove that every integer $n \geq 2$ is either prime or can be expressed as a product of primes.
\end{exercise}

\section{Proving Non-Conditional Statements}
\subsection{Biconditional Statements}\index{proof!biconditional}
Recall that a biconditional statement is a statement like ``$P \iff Q$'', i.e., ``$P$ if and only if $Q$''. We prove such a statement by proving that $P \implies Q$ (which we call the `forward direction') and $Q \implies P$ (which we call the `reverse direction'). Each of these statements may be proved using any of the proof techniques that we covered.

\begin{example}
    We will prove the biconditional statement ``the integer $n$ is even if and only if $n^2$ is even''.
    \begin{proof}
        We prove the forward direction ($n$ is even implies $n^2$ is even) first by using direct proof. Assume that $n$ is even. Then we may write $n = 2k$ where $k$ is an integer. Hence $n^2 = (2k)^2 = 4k^2 = 2(2k^2)$ which is even.

        We now prove the reverse direction ($n^2$ is even implies $n$ is even) via a proof by contrapositive. Suppose $n$ is \textbf{not} even, meaning $n$ is odd. Hence $n = 2k + 1$ where $k$ is an integer. Observe $n^2 = (2k+1)^2 = 4k^2 + 4k + 1 = 2(2k^2 + 2k) + 1$ which is odd.
    \end{proof}
\end{example}

\begin{example}
    Suppose $n$ is an integer. We will prove ``$n$ is a multiple of 6 if and only if $n$ is a multiple of 2 and 3''.
    \begin{proof}
        We prove the forward direction first by using direct proof. Assume $n$ is a multiple of 6, meaning $n = 6k$ for some integer $k$. Clearly $6k = 2(3k)$ and $6k = 3(2k)$, so $n$ is both a multiple of 2 and 3.

        We now prove the reverse direction, again using direct proof. Assume $n$ is a multiple of 2 and 3, so we may write $n = 2a$ and $n = 3b$ for some integers $a$ and $b$. Then $2a = 3b$. Hence $a = \frac 32 b$ and $b = \frac 23 a$. Since $a$ and $b$ are integers, hence we conclude $b$ is a multiple of 2 and $a$ is a multiple of 3. Write $a = 3p$ and $b = 2q$ where $p$ and $q$ are integers. Hence $n = 2(3p) = 6p$ and $n = 3(2q) = 6q$. In both cases we see $n$ is a multiple of 6.
    \end{proof}
\end{example}

\begin{exercise}
    Let $n$ be an integer. Prove that $n$ is one more than a multiple of 5 if and only if $n$ is of the form $5k - 4$ where $k$ is an integer.
\end{exercise}

\subsection{Existence Statements}
Some statements only assert the existence of something. These are called \textbf{existence statements} and one only has to provide a particular example that shows it is true.\index{proof!existence proof}
\begin{example}
    The statement ``there exists an even prime number'' is readily proven by noticing that 2 is an even prime number.
\end{example}

\begin{example}
    The statement ``an integer that can be expressed as the sum of two perfect cubes in two different ways exists'' is proven by giving the example 1729, since $1729 = 1^3 + 12^3 = 9^3 + 10^3$.
\end{example}

Note that while an example suffices to prove an existence statement, a single example does not prove a (universal) conditional statement.

\begin{exercise}
    Prove that there is a positive integer that is one less than a perfect cube and two less than a perfect square.
\end{exercise}

Existence proofs fall into two categories: \textbf{constructive}\index{proof!constructive} and \textbf{non-constructive}\index{proof!non-constructive} proofs.
\begin{itemize}
    \item Constructive proofs provide an explicit example that proves the statement.
    \item Non-constructive proofs prove that an example exists without providing it.
\end{itemize}
We have only seen constructive proofs so far, so let's look at an example of a non-constructive proof.

\begin{example}
    We prove the classic statement that ``there exist irrational numbers $x$ and $y$ such that $x^y$ is rational'' using a non-constructive proof.
    \begin{proof}
        We know that $\sqrt2$ is irrational from \myref{example-sqrt2-is-irrational}. Consider the number $\sqrt2^{\sqrt2}$. There are two possibilities for the rationality of this number.
        \begin{itemize}
            \item If $\sqrt2^{\sqrt2}$ is rational, then set $x = y = \sqrt2$. We have found two irrational numbers such that their exponentiation (i.e.,  $x = \sqrt2^{\sqrt2}$) is rational.
            \item If instead $\sqrt2^{\sqrt2}$ is irrational, set $x = \sqrt2^{\sqrt2}$ and $y = \sqrt2$. Observe that
            \[
                x^y = \left(\sqrt2^{\sqrt2}\right)^{\sqrt2} = (\sqrt2)^{\sqrt2 \times \sqrt2} = (\sqrt2)^2 = 2
            \]
            is a rational number. So we again found two irrational numbers such that their exponentiation is rational.
        \end{itemize}
        Hence, in both cases, we have an irrational number to an irrational power that results in a rational number.
    \end{proof}

    Notice that we did not explicitly prove whether $\sqrt2^{\sqrt2}$ is rational or irrational; we just showed that either case leads to a case where two irrational numbers, when exponentiated, results in a rational number.
\end{example}

\begin{exercise}
    Let $x = \sqrt2$ and $y = 2\log_2{3}$. It may be assumed that $\sqrt2$ is irrational.
    \begin{partquestions}{\roman*}
        \item Prove that $y$ is irrational.
        \item Produce a constructive proof that there exist irrational $x$ and $y$ such that $x^y$ is rational.
    \end{partquestions}
\end{exercise}
