\documentclass[
    b5paper,
    pagesize,
    10pt,
    bibtotoc,
    normalheadings,
    chapterprefix,
    DIV=9
]{scrbook}

%=========== Global Variables ============
\newcommand{\editionnumber}{1\textsuperscript{st}}
\newcommand{\buildnumber}{31}

\newcommand{\isebook}{true}

%=============== Preamble ================
%=============== Packages ================
%================= Fonts =================
\usepackage[utf8]{inputenc}
\usepackage[T1]{fontenc}
\usepackage{lmodern}
\usepackage{tgpagella}
\usepackage{mathpazo}

%============== Mathematics ==============
\usepackage{mathtools}
\usepackage{amsfonts}
\usepackage{amsmath}
\usepackage{amssymb}
\usepackage{amsthm}
\usepackage{thmtools}

%====== Formatting, Colour, Images =======
\usepackage[showframe]{geometry}
\usepackage{graphicx}
\usepackage{tocloft}
\usepackage[x11names]{xcolor}
\usepackage{wrapfig}
\usepackage{cutwin}
\usepackage{tikz}
\usepackage{fancyhdr}
\usepackage{emptypage}
\usepackage{imakeidx}
\usepackage{mdframed}
\usepackage{eso-pic}
\usepackage{float}

%======= Hyperlinks and References =======
\usepackage{hyperref}
\usepackage[capitalise, nameinlink]{cleveref}
\usepackage{crossreftools}
\usepackage[
    backend=bibtex,
    style=alphabetic,
    sorting=ynt
]{biblatex}
\usepackage[
    totoc,
    columnsep=20pt,
    hangindent=8pt,
    subindent=20pt,
    subsubindent=30pt
]{idxlayout}
\usepackage[record,symbols]{glossaries-extra}

%============= Miscellaneous =============
\usepackage{multicol}
\usepackage{multirow}
\usepackage{enumitem}


%============= PDF Metadata ==============
\hypersetup{
    pdftitle={A Complete Introduction To Abstract Algebra},
    pdfauthor={Kan Onn Kit},
    pdfsubject={Abstract Algebra},
    breaklinks=true
}

%=============== Geometry ================
% Notes on lengths and sizes used here:
% - 17.6cm x 25cm is B5 paper size.
% - 7" x 10" is executive paper size. We add a bleed margin of 0.25" for both height and width, so the final paper size for the normal geometry is 7.25" x 10.25".

\makeatletter
\geometry{
    layoutsize={17.6cm, 25cm},  % B5 paper size
    top=2.5cm,
    bottom=2cm
}

\Ifstr{\isebook}{true}{
    % eBook geometry
    \geometry{
        papersize={17.6cm, 25cm},
        inner=20mm,
        outer=20mm
    }
}{
    % Normal paper geometry
    \geometry{
        papersize={7in, 10in},
        inner=18mm,
        outer=22mm
    }
}

\makeatother

%============== Resources ================
% Bibliography
\addbibresource{abstract-algebra.bib}

% Graphics locations
\graphicspath{
    % Main location
    {images}
    % Part 0
    {part0/images}
    % Part 1
    {part1/images}
    {part1/images/intro-to-groups}
    {part1/images/basics-of-groups}
    {part1/images/homomorphisms-and-isomorphisms}
    {part1/images/symmetry-groups}
    {part1/images/further-homomorphisms}
    {part1/images/group-actions}
    % Part 2
    {part2/images}
    {part2/images/ring-homomorphisms}
    {part2/images/domains}
    % Part 3
    {part3/images}
    {part3/images/algebraic-extensions}
    {part3/images/finite-fields}
    {part3/images/geometric-constructions}
    % Part 4
    {part4/images}
    {part4/images/galois}
}

%============= Formatting ================
% Line format
\linespread{1.05}

% Indent
\setlength{\parindent}{1.5em}
\newlength{\normalparindent}
\setlength{\normalparindent}{\parindent}

% Heading format
\setkomafont{sectioning}{\sffamily\bfseries\boldmath}

%=========== (Re)definitions =============
% Part page configuration
\let\oldpart\part
\renewcommand*{\part}[1]{\cleardoubleoddpage\oldpart{#1}\cleardoubleemptypage}
\newcommand{\unnumberedpart}[1]{
    \cleardoubleoddpage
    \addcontentsline{toc}{part}{#1}
    \oldpart*{#1}
    \cleardoubleemptypage
}

\makeatletter
\patchcmd{\set@@@@preamble}{#6}{\setlength{\parindent}{\normalparindent}#6}{}{}  % Make paragraph indent correctly for part preambles
\makeatother

% Symbols redefinitions
\let\oldemptyset\emptyset
\let\emptyset\varnothing

\let\oldepsilon\epsilon
\let\epsilon\varepsilon

\let\totient\varphi

\renewcommand{\vert}{ \;\: \vline \;\: }
\newcommand{\vertalt}{ \;\: | \;\: }

\newcommand{\myref}[1]{\textbf{\crthypercref{#1}}}
\newcommand{\myreffigures}[1]{\textbf{\cref{#1}}}

\newcommand{\qedproof}{\ensuremath{\blacksquare}}
\newcommand{\qedsketch}{\ensuremath{\square}}
\renewcommand{\qedsymbol}{\qedproof}  % Actually changes the QED symbol

% Quote definitions
\newcommand{\quoteattr}[4][0.5cm]{
    \vspace{#1}
    \begin{center}
        \parbox{8cm}{
            {\itshape#2}\\
            \null\hfill--- #3\\
            \vspace{-0.1cm}
            \hfill\small(#4)
        }
    \end{center}
    \vspace{#1}
}

% PDF-TeX image definitions
\newcommand{\pdfteximg}[3][10pt]{
    {
        \fontsize{#1}{0pt}\selectfont
        \def\svgwidth{#2}
        \input{#3}
    }
}

\newcommand{\pdfteximgframed}[3][10pt]{
    {
        \fontsize{#1}{0pt}\selectfont
        \def\svgwidth{#2}
        \fbox{\input{#3}}
    }
}

%=========== Theorem Things ==============
% 'Results' declarations
\newcommand{\makenewresultstyle}[3]{
    \declaretheoremstyle[
        headfont=\normalfont\bfseries,
        bodyfont=\normalfont,
        notefont=\normalfont\bfseries\boldmath\itshape,
        spaceabove=0pt,  % Space between previous paragraph and current block
        spacebelow=0pt,  % Space between current block and next paragraph
        mdframed={
            linecolor=#3,
            skipabove=3pt,  % Space between top of block and beginning of coloured frame
            skipbelow=3pt,  % Space between bottom of block and beginning of coloured frame
            backgroundcolor=#2,
            usetwoside=false,  % Needed for `leftmargin` and `rightmargin` to work
            leftmargin=-5pt,
            rightmargin=-5pt,
            innerleftmargin=5pt,
            innerrightmargin=5pt
        }
    ]{#1-style}
}

\makenewresultstyle{theorem}{DarkSeaGreen2}{DarkSeaGreen4}
\declaretheorem[name=Theorem,style=theorem-style,within=section]{theorem}
\renewcommand*{\thetheorem}{\arabic{chapter}.\arabic{section}.\arabic{theorem}}

\makenewresultstyle{lemma}{AntiqueWhite1}{Bisque4}
\declaretheorem[style=lemma-style,sibling=theorem]{lemma}

\makenewresultstyle{proposition}{Honeydew1}{SpringGreen4}
\declaretheorem[style=proposition-style,sibling=theorem]{proposition}

\makenewresultstyle{corollary}{Cornsilk1}{Ivory4}
\declaretheorem[style=corollary-style,sibling=theorem]{corollary}

\makenewresultstyle{definition}{LightCyan1}{Cyan4}
\declaretheorem[style=definition-style,sibling=theorem]{definition}

\makenewresultstyle{axiom}{Thistle2}{Plum4}
\declaretheorem[style=axiom-style,sibling=theorem]{axiom}

% 'Questions' declarations
\declaretheoremstyle[
    headfont=\normalfont\bfseries,
    bodyfont=\normalfont,
    spaceabove=5pt,  % Space between previous paragraph and current block
    prefoothook={\vspace{5pt}},
    mdframed={
        skipabove=3pt,  % Space between top of block and beginning of frame
        skipbelow=3pt,  % Space between bottom of block and beginning of frame
        usetwoside=false,  % Needed for `leftmargin` and `rightmargin` to work
        leftmargin=-5pt,
        rightmargin=-5pt,
        innerleftmargin=5pt,
        innerrightmargin=5pt
    }
]{exercise-style}
\declaretheorem[style=exercise-style,within=chapter]{exercise}
\renewcommand*{\theexercise}{\arabic{chapter}.\arabic{exercise}}

\declaretheoremstyle[
    headfont=\normalfont\bfseries,
    bodyfont=\normalfont,
    notefont=\normalfont\bfseries\itshape,
    spaceabove=10pt,  % Space between previous paragraph and current block
    spacebelow=10pt,  % Space between current block and next paragraph
]{problem-style}
\declaretheorem[name=Problem,style=problem-style,within=chapter]{problem}
\renewcommand*{\theproblem}{\arabic{chapter}.\arabic{problem}}

% Miscellaneous declarations
\declaretheoremstyle[
    headfont=\normalfont\bfseries,
    bodyfont=\normalfont,
    spaceabove=10pt,  % Space between previous paragraph and current block
    spacebelow=10pt,  % Space between current block and next paragraph
]{general-style}
\declaretheorem[style=general-style,sibling=theorem]{example}
\declaretheorem[style=general-style,numbered=no]{remark}

%============ Environments ===============
% Questions environments
\newenvironment{questions}
{\begin{enumerate}[label=\textbf{\arabic*.}]}
{\end{enumerate}}

\newenvironment{partquestions}[1]
{\begin{enumerate}[label=\textbf{(#1)}]}
{\end{enumerate}}

% Proof environments
\newenvironment{proofsketch}
{\begin{proof}[Sketch of Proof.]}
{\renewcommand{\qedsymbol}{\qedsketch}\end{proof}\renewcommand{\qedsymbol}{\qedproof}}

% Miscellaneous environments
\newenvironment{examplewithcutout}[6]
{
    \vspace{\baselineskip}
    \begin{example}
    \renewcommand\windowpagestuff{#6}
    \Ifstr{#1}{left}{\opencutleft}{\Ifstr{#1}{right}{\opencutright}{\opencutcenter}}
    \begin{cutout}{#2}{#3}{#4}{#5}
}
{\end{cutout}\end{example}}

%=========== Custom Commands =============
\newcommand{\ZeroRoman}[1]{\ifcase\value{#1}\relax0\else\Roman{#1}\fi}  % Roman numeral that works with zero
\newcommand{\code}[1]{\texttt{#1}}  % Code block
\newcommand{\term}[1]{{\bfseries\boldmath\itshape #1}}  % Terminology

%=========== General Symbols =============
\newcommand{\lcm}{\mathrm{lcm}}  % Lowest common multiple function
\newcommand{\sgn}{\mathrm{sgn}}  % Signum function

\newcommand{\im}{\mathrm{im}\;}  % Image of a function
\newcommand{\id}{\mathrm{id}}    % Identity function

\newcommand{\Z}{\mathbb{Z}}  % The set of integers
\newcommand{\Q}{\mathbb{Q}}  % The set of rational numbers
\newcommand{\R}{\mathbb{R}}  % The set of real numbers
\newcommand{\C}{\mathbb{C}}  % The set of complex numbers

%================ Part I =================
\newcommand{\An}[1]{\mathrm{A}_{#1}}                  % Alternating group of degree n
\newcommand{\Aut}[1]{\mathrm{Aut}(#1)}                % Group of automorphisms of G
\newcommand{\Centralizer}[2]{\mathrm{C}_{#1}(#2)}     % Centralizer of an element in G
\newcommand{\Cl}[1]{\mathrm{Cl}(#1)}                  % Conjugacy class of the element x
\newcommand{\Cn}[1]{\mathrm{C}_{#1}}                  % Cyclic group of order n
\newcommand{\GL}[2]{\mathrm{GL}_{#1}\left(#2\right)}  % General Linear Group of degree n
\newcommand{\Inn}[1]{\mathrm{Inn}(#1)}                % Group of inner automorphisms of G
\newcommand{\N}[2]{\mathrm{N}_{#1}(#2)}               % Normalizer of S in G
\newcommand{\Out}[1]{\mathrm{Out}(#1)}                % Group of outer automorphisms of G
\newcommand{\SL}[2]{\mathrm{SL}_{#1}\left(#2\right)}  % Special Linear Group of degree n
\newcommand{\Sn}[1]{\mathrm{S}_{#1}}                  % Symmetric group of degree n
\newcommand{\Syl}[2]{\mathrm{Syl}_{#1}(#2)}           % Set of Sylow p-groups of G
\newcommand{\Sym}[1]{\mathrm{Sym}(#1)}                % Symmetric group of a set
\newcommand{\Un}[1]{\mathcal{U}_{#1}}                 % Group of units modulo n
\newcommand{\CenterGrp}[1]{\mathrm{Z}(#1)}            % Center of a group G

\newcommand{\Stab}[2]{\mathrm{Stab}_{#1}(#2)}         % Stabilizer of x by G
\newcommand{\Fix}[2]{\mathrm{Fix}_{#1}(#2)}           % Set of all elements in X which is fixed by g
\newcommand{\Orb}[2]{\mathrm{Orb}_{#1}(#2)}           % Orbit of x under G

%================ Part II ================
\newcommand{\ideal}[1]{\mathfrak{#1}}                % Ideal of a ring
\newcommand{\princ}[1]{\left\langle#1\right\rangle}  % Principal ideal generated by the element

\newcommand{\Mn}[2]{\mathcal{M}_{#1\times#1}(#2)}  % The ring of n by n matrices with entries in the ring R
\newcommand{\ZeroM}[1]{\textbf{0}_{#1}}      % Zero matrix
\newcommand{\IdentityM}[1]{\textbf{I}_{#1}}  % Identity matrix

% \renewcommand{\H}{\mathbb{H}}   % Quaternion Ring
% \newcommand{\qi}{\textbf{i}}    % Quaternion i
% \newcommand{\qj}{\textbf{j}}    % Quaternion j
% \newcommand{\qk}{\textbf{k}}    % Quaternion k

\newcommand{\Ann}[2]{\mathrm{Ann}_{#1}(#2)}  % Annihilator of a subset A of a ring R
\newcommand{\Char}[1]{\mathrm{char}(#1)}     % Characteristic of a ring R
\newcommand{\Nilr}[1]{\mathfrak{N}_{#1}}     % Nilradical of a ring R
\newcommand{\Frac}[1]{\mathrm{Frac}(#1)}     % Field of fractions of an integral domain D

%=============== Part III ================
\newcommand{\Span}[1]{\mathrm{span}(#1)}  % Span of a set S of vectors
\let\olddim\dim
\renewcommand*{\dim}[1]{\olddim(#1)}      % Dimension of a vector space V

%================ Part IV ================
\newcommand{\Gal}[1]{\mathrm{Gal}(#1)}  % Galois group of E/F


%======== Custom Chapter Styling =========
% Fix part numbering
\renewcommand*{\thepart}{\ZeroRoman{part}}

\makeatletter
% Set chapter page format
\renewcommand{\chaptermark}[1]{
    \markboth{\if@mainmatter\chapapp~\thechapter.\ \fi#1}{}
}

\renewcommand*{\chapterformat}{
    \MakeUppercase{\chapapp\nobreakspace\thechapter}
}

\renewcommand*{\chapterlineswithprefixformat}[3]{
    \Ifstr{#1}{chapter}{
        \vspace{-60px}
        \Ifstr{#2}{\empty}{\vspace{40px}}{\raggedleft#2}
        \vspace{-15px}
        \rule{\linewidth}{1pt}\par\nobreak
        \centering{#3}
        \vspace{-10px}
        \rule{\linewidth}{1pt}\par\nobreak
        \vspace{-10px}
    }{#2#3}
}

% Set part page format
\renewcommand*{\partformat}{
    {\fontsize{28pt}{0pt}\selectfont \MakeUppercase{\partname\nobreakspace\thepart}}
}

\renewcommand*{\partlineswithprefixformat}[3]{
    \Ifstr{#1}{part}{
        \Ifstr{#2}{}{  % Doesn't have a part title
            \begin{mdframed}[linewidth=2pt]
                \centering#3
            \end{mdframed}
        }{
            \vspace{-250px}
            \begin{mdframed}[linewidth=2pt]
                \begin{center}
                    \vspace{5px}
                    #2
                    \vspace{-25px}
                    #3
                \end{center}
            \end{mdframed}
        }
    }{#2#3}
}

\makeatother

%=========== Hyperlink Setup =============
\newcommand{\hyperrefcolour}[2]{
    \definecolor{hyperref-#1-colour}{HTML}{#2}
    \hypersetup{#1color=hyperref-#1-colour}
}

\makeatletter
\hypersetup{
    colorlinks=true,
    pdfborder={0 0 0}
}

\hyperrefcolour{link}{0077b3}
\hyperrefcolour{cite}{37992e}
\hyperrefcolour{url}{990099}

\makeatother

%======== Figure Caption Format ==========
\usepackage[labelfont=bf]{caption}
\DeclareCaptionLabelFormat{custom}{#1 #2.}
\captionsetup{labelformat=custom,labelsep=space}

%============ Custom Header ==============
\fancypagestyle{plain}{\fancyhf{}\renewcommand{\headrulewidth}{0pt}}  % To clear page numbers from footer, and header line at the start of every chapter

\pagestyle{fancy}
\fancyhf{}  % Clear header/footer

\fancyhead[EL,OR]{\thepage}
\fancyhead[OL,ER]{\textit{\nouppercase\leftmark}}

\newcommand{\draftstartmark}{\fancyfoot[EC,OC]{\textsc{Work In Progress (Build \buildnumber)}}}
\newcommand{\draftendmark}{\fancyfoot[EC,OC]{}}

%====== Customise Table of Contents ======
% Customise part styling in table of contents
\newlength\parttoclen
\renewcommand\cftpartpresnum{Part~}
\settowidth\parttoclen{\bfseries\cftpartpresnum\cftpartaftersnum}
\addtolength\cftpartnumwidth{\parttoclen+1em}  % +1em to make it nicer and more spaced out

% Heading customisation
\makeatletter
\def\createtoc{
    \renewcommand\tableofcontents{
        \chapter*{\contentsname}
        \@starttoc{toc}
    }
    \tableofcontents
}
\makeatother

%============= Index Pages ===============
\makeindex[options= -s index-style.ist]

%======= Bibliography Formatting =========
% These two lines are here to ensure that URLs do not exceed the page by too much
\setcounter{biburllcpenalty}{7000}
\setcounter{biburlucpenalty}{8000}


\usepackage{xr}
\externaldocument{book}

%=========================================
\title{A Complete Introduction To\\Abstract Algebra}
\subtitle{Exercises and Problems Only}
\date{\today}
\author{Kan Onn Kit}

\begin{document}
\maketitle

% START OF AUTOGENERATED CONTENT

\chapter{Sets}
\section*{Exercises}
\begin{mdframed}
    Let $S$ be a non-empty set. Determine whether the following statements are true or false.

    \begin{multicols}{2}
        \begin{partquestions}{\alph*}
            \item $\{1, 2\} \subset \{1, 2, 3, 4\}$
            \item $\{1, 2, 3\} \subseteq \{1, 2, 4\}$
            \item $\emptyset \subseteq \emptyset$
            \item $S \subset S$
            \item $S \in \{S, \emptyset\}$
            \item $\{S\} \notin \{S, \emptyset\}$
            \item $S \subseteq \{S, \emptyset\}$ where $S \neq \{\emptyset\}$
            \item $\{S\} \subseteq \{S, \emptyset\}$
        \end{partquestions}
    \end{multicols}
\end{mdframed}
\textbf{Solution}:\newline
 \begin{partquestions}{\alph*}
        \item True, as both 1 and 2 appear in the set $\{1, 2, 3, 4\}$.
        \item False, 3 does not appear in $\{1, 2, 4\}$.
        \item True. Any set is a subset of itself, including the empty set.
        \item False, the set $S$ does not contain any element that is not in $S$. That is, $S \subseteq S$ but not $S \subset S$.
        \item True. $S$ is indeed an element of $\{S, \emptyset\}$.
        \item True. The set containing S is not an element of $\{S, \emptyset\}$.
        \item False, the set $S$ is not a subset of the set $\{S, \emptyset\}$.
        \item True. The set containing $S$ is a subset of the set containing $S$ and the empty set.
    \end{partquestions}
\begin{mdframed}
    Let $S = \{1, 2, 3, 4\}$, $T = \{2, 3, 5\}$, $U = \{(2, 2), (3, 3), (5, 5)\}$. Determine whether the following statements are true or false.
    \begin{multicols}{2}
        \begin{partquestions}{\alph*}
            \item $S \cup T = \{1, 2, 3, 4, 5\}$
            \item $S \cup U = \{1, 2, 3, (5, 5)\}$
            \item $S \cap T = \{2, 3\}$
            \item $T \cap U = \emptyset$
            \item $S \setminus T = \{1, 4\}$
            \item $S \setminus \{1, 4\} = T$
            \item $T^2 = U$
            \item $U \subset (S \cup T)^2$
        \end{partquestions}
    \end{multicols}
\end{mdframed}
\textbf{Solution}:\newline
 \begin{partquestions}{\alph*}
        \item True. The set of elements that are in either $S$ or $R$ is indeed $\{1, 2, 3, 4, 5\}$.
        \item False, $S \cup U = \{1, 2, 3, 4, (2, 2), (3, 3), (5, 5)\}$.
        \item True. The set of elements that are in both $S$ and $T$ is indeed $\{2, 3\}$.
        \item True. $T$ and $U$ share no elements in common, so their intersection is empty.
        \item True. The elements that are in $S$ but not in $T$ are indeed 1 and 4.
        \item False, $S \setminus \{1, 4\} = \{2, 3\}$, not $T = \{2, 3, 5\}$.
        \item True.
        \item True. $(S \cup T)^2 = \{(1,1), (2,2), (3,3), (4,4), (5,5)\}$, so
        \[
            U = \{(2,2), (3,3), (5,5)\} \subset (S \cup T)^2.
        \]
    \end{partquestions}
\begin{mdframed}
    Let the sets
    \begin{align*}
        S &= \{x \in \Q \vert x \in (-\infty, 0]\}\\
        T &= \{y \in \Z \vert y \in [-2, 10] \text{ and } y \text{ is an even number} \}
    \end{align*}
    List the elements in the set $S \cap T$.
\end{mdframed}
\textbf{Solution}:\newline
 We note $S$ are all the non-positive rational numbers, and $T = \{-2, 0, 2, \dots, 8, 10\}$. Hence $S \cap T$ has only two elements, namely $-2$ and $0$.

\section*{Problems}
\begin{mdframed}
    Let $A$ and $B$ be finite sets with cardinality $n$. For what value(s) of $n$ can we be sure that $A = B$?
\end{mdframed}
\textbf{Solution}:\newline
 We can only be sure that $A = B$ if $n = 0$, i.e. $A = B = \emptyset$.

    To see this, suppose $n$ is non-zero, meaning $n \geq 1$. Then note that we can find the sets $A = \{1, 2, 3, \dots, n\}$ and $B = \{0, 2, 3, \dots, n\}$, which are two sets with cardinality $n$ but are not equal.

    Therefore the only value of $n$ that works is $n = 0$.
\begin{mdframed}
    Let the sets
    \begin{align*}
        A &= \{x \in \R \vert x^2 - x - 2 \leq 0\},\\
        B &= \{x \in [0, \infty) \vert 12 - x - x^2 > 0\}.
    \end{align*}
    \begin{partquestions}{\alph*}
        \item Express $A \cap B$ in interval notation.
        \item Express $A \cup B$ in interval notation.
        \item Express $B \setminus A$ in interval notation.
        \item Is $A \setminus B \subset [-1, 0)$? Explain.
    \end{partquestions}
\end{mdframed}
\textbf{Solution}:\newline
 We first express $A$ and $B$ in a `nicer' way to work with.

    For $A$, note that $x^2 - x - 2 \leq 0$ is the same as $(x+1)(x-2) \leq 0$, which means that $x \in [-1, 2]$. Therefore $A = \R \cap [-1, 2] = [-1, 2]$.

    For $B$, we see $12 - x - x^2 > 0$ is the same as $(x+4)(x-3) < 0$, meaning $x \in (-4, 3)$. Thus $B = [0, \infty) \cap (-4, 3) = [0, 3)$.

    \begin{partquestions}{\alph*}
        \item $A \cap B = [-1, 2] \cap [0, 3) = [0, 2]$.
        \item $A \cup B = [-1, 2] \cup [0, 3) = [-1, 3)$.
        \item $B \setminus A = [0, 3) \setminus [-1, 2] = (2, 3)$.
        \item Note $A \setminus B = [-1, 2] \setminus [0, 3) = [-1, 0)$. This is a subset of $[-1, 0)$, but not a \textit{proper} subset of $[-1, 0)$. Therefore $A\setminus B \subset [-1, 0)$ is false.
    \end{partquestions}
\begin{mdframed}
    Let the sets
    \begin{align*}
        A &= \{(x,y) \in \Z^2 \vert 5x+2y+3=0\},\\
        B &= \{(x,y) \in [0,1]^2 \vert 2x^2+5x+2y+1=0\}.
    \end{align*}
    Find the cardinality of the following sets. If the cardinality is finite, list all elements of the set.
    \begin{partquestions}{\alph*}
        \item $A \cap B$
        \item $A \cup B$
    \end{partquestions}
\end{mdframed}
\textbf{Solution}:\newline
 Again, we express $A$ and $B$ in a `nicer' way to work with.

    For $A$, note we can rearrange $5x+2y+3=0$ to be $y=-\frac52x-\frac32$. For $B$, we can rearrange $2x^2+5x+2y+1=0$ to be $y=-x^2-\frac52x-\frac12$. Therefore
    \begin{align*}
        A &= \left\{\left(x, -\frac52x-\frac32\right) \vert x \in \Z \right\},\\
        B &= \left\{\left(x, -x^2-\frac52x-\frac12\right) \vert x \in [0, 1] \right\},
    \end{align*}
    which are both infinite sets.

    \begin{partquestions}{\alph*}
        \item $A \cap B$ represents the solutions to the equation $-\frac52x-\frac32 = -x^2-\frac52x-\frac12$. This equation has 2 solutions, namely $x = -1$ and $x = 1$, which have $y$ values of 1 and -4 respectively. Therefore $|A \cap B| = 2$, where $A \cap B = \{(-1, 1), (1, -4)\}$.
        \item Since $A$ and $B$ are both infinite sets, their union is also infinite. Therefore $|A \cup B| = \infty$.
    \end{partquestions}
\begin{mdframed}
    Show that $A \cap (B \setminus C) = (A \cap B) \setminus C$ for any sets $A$, $B$, and $C$.
\end{mdframed}
\textbf{Solution}:\newline
 We work slowly:
    \begin{align*}
        A \cap (B \setminus C) &= \{x \vert x \in A \text{ and } x \in (B \setminus C)\}\\
        &= \{x \vert x \in A \text{ and } (x \in B \text{ and } x \notin C)\}\\
        &= \{x \vert (x \in A \text{ and } x \in B) \text{ and } x \notin C\}\\
        &= \{x \vert x \in A \cap B \text{ and } x \notin C\}\\
        &= (A \cap B) \setminus C.
    \end{align*}

\chapter{Logic}
\section*{Exercises}
\begin{mdframed}
    Let the statements
    \begin{align*}
        p: &\ \text{1 is a positive number}\\
        q: &\ -1 > 0\\
        r: &\ \text{1 is an odd number}
    \end{align*}
    Is the statement ``$\lnot((p\lor q)\land r)$ is false'' true?
\end{mdframed}
\textbf{Solution}:\newline
 We work from the inner-most bracket outwards. We note $p$ is true, $q$ is false, and $r$ is true.
    \begin{itemize}
        \item $p \lor q$ is ``1 is a positive number \textbf{or} $-1 > 0$'', which is true since $p$ is true.
        \item $(p \lor q) \land r$ is ``(1 is a positive number or $-1 > 0$) \textbf{and} 1 is an odd number'', which is true since $p \lor q$ is true and 1 is, indeed, an odd number.
        \item $\lnot((p \lor q) \land r)$ is false, since $(p \lor q) \land r$ is true.
    \end{itemize}
    Hence the statement ``$\lnot((p \lor q) \land r)$ is false'' is a true statement.
\begin{mdframed}
    Let $p$ and $q$ be statements. Draw the truth table for $p \land (\lnot q)$.
\end{mdframed}
\textbf{Solution}:\newline
 The truth table for $p \land \lnot q$ is given below.
    \begin{table}[H]
        \centering
        \begin{tabular}{|l|l||l|}
            \hline
            $\boldsymbol{p}$ & $\boldsymbol{q}$ & $\boldsymbol{p\land \lnot q}$ \\ \hline
            F   & F   & F                  \\ \hline
            F   & T   & F                  \\ \hline
            T   & F   & T                  \\ \hline
            T   & T   & F                  \\ \hline
        \end{tabular}
    \end{table}

    We will see that, in fact, this is the negation of $p \implies q$ later.
\begin{mdframed}
    Let $n$ be an integer. Let $p$ be the statement ``$n$ is a multiple of 5'' and $q$ be the statement ``the last digit of $n$ is 0 or 5''. Let the statement $r = (p \iff q)$.
    \begin{partquestions}{\roman*}
        \item Write the statement $r$ in English.
        \item Is the statement $r$ true? Justify your answer.
    \end{partquestions}
\end{mdframed}
\textbf{Solution}:\newline
 \begin{partquestions}{\roman*}
        \item $r$: $n$ is a multiple of 5 if and only if the last digit of $n$ is 0 or 5.
        \item If $n$ is a multiple of 5, then its last digit necessarily has to be 5 or 0, hence $p \implies q$. If the last digit is 5 or 0, then the number $n$ is a multiple of 5, hence $q \implies p$. Therefore $p \iff q$.
    \end{partquestions}
\begin{mdframed}
    Show that $(\lnot p \iff q) \equiv (p \implies \lnot q) \land (\lnot q \implies p)$ by drawing a truth table.
\end{mdframed}
\textbf{Solution}:\newline
 For brevity, let $r = p \implies \lnot q$ and $s = \lnot q \implies p$. So we want to show that $(\lnot p \iff q) \equiv r \land s$.
    \begin{table}[H]
        \centering
        \begin{tabular}{|l|l||l|l|l|l||l|l|}
            \hline
            $\boldsymbol{p}$ & $\boldsymbol{q}$ & $\boldsymbol{\lnot p}$ & $\boldsymbol{\lnot q}$ & $\boldsymbol{r}$ & $\boldsymbol{s}$ & $\boldsymbol{r \land s}$ & $\boldsymbol{\lnot p \iff q}$ \\ \hline
            F   & F   & T         & T         & T   & F   & F           & F                  \\ \hline
            F   & T   & T         & F         & T   & T   & T           & T                  \\ \hline
            T   & F   & F         & T         & T   & T   & T           & T                  \\ \hline
            T   & T   & F         & F         & F   & T   & F           & F                  \\ \hline
        \end{tabular}
    \end{table}

    From inspection, the truth tables of $\lnot p \iff q$ and $r \land s$ are the same, proving our required result.
\begin{mdframed}
    Simplify the statement
    \[
        ((p \lor \lnot q) \land \lnot r) \lor ((p \lor \lnot q) \land (p \lor r) \land (p \lor \lnot r))
    \]
    into a statement that uses only \textbf{three} operators in total.
\end{mdframed}
\textbf{Solution}:\newline
 We work slowly.
    \begin{align*}
        &((p \lor \lnot q) \land \lnot r) \lor ((p \lor \lnot q) \land (p \lor r) \land (p \lor \lnot r))\\
        &\equiv ((p \lor \lnot q) \land \lnot r) \lor ((p \lor \lnot q) \land ((p \lor r) \land (p \lor \lnot r))) & (\text{Associativity})\\
        &\equiv (p \lor \lnot q) \land (\lnot r \lor ((p \lor r) \land (p \lor \lnot r))) & (\text{Distributivity})\\
        &\equiv (p \lor \lnot q) \land (\lnot r \lor (p \lor (r \land \lnot r))) & (\text{Distributivity})\\
        &\equiv (p \lor \lnot q) \land (\lnot r \lor (p \lor \textbf{false}))\\
        &\equiv (p \lor \lnot q) \land (\lnot r \lor p)\\
        &\equiv (p \lor \lnot q) \land (p \lor \lnot r) & (\text{Commutativity})\\
        &\equiv p \lor (\lnot q \land \lnot r) & (\text{Distributivity})\\
        &\equiv p \lor \lnot(q \lor r) & (\text{De Morgan's Law})
    \end{align*}
\begin{mdframed}
    Convert the statement ``for all positive integers $n > 2$ there exist integers $a$ and $b$ such that $a^3 + b^4 = n^5$'' into symbolic notation using quantifiers and logical operators.
\end{mdframed}
\textbf{Solution}:\newline
 $\left((n \in \mathbb{Z}) \land (n > 2)\right) \implies \left(\exists a, b \in \mathbb{Z} \text{ s.t. } a^3 + b^4 = n^5\right)$
\begin{mdframed}
    Consider the statement ``there exists a real number $y$ such that $xy = 1$ for all non-zero real numbers $x$''.
    \begin{partquestions}{\roman*}
        \item Write down two predicates $P(x)$ and $Q(x)$ in symbols such that the above statement is $\forall x,(P(x) \implies Q(x))$.
        \item Negate the above statement, writing your answer in symbolic notation.
    \end{partquestions}
\end{mdframed}
\textbf{Solution}:\newline
 \begin{partquestions}{\roman*}
        \item Let
        \begin{align*}
            P(x):&\ (x \in \mathbb{R}) \land (x \neq 0)\\
            Q(x):&\ \exists y \in \mathbb{R} \text{ s.t. } xy = 1
        \end{align*}
        then the given statement is $\forall x, (P(x) \implies Q(x))$ as required.

        \item $\exists x \text{ s.t. } (x \in \mathbb{R}) \land (x \neq 0) \land (\forall y \in \mathbb{R}, xy \neq 1)$
    \end{partquestions}

\section*{Problems}
\begin{mdframed}
    Which of the following are statements? Which are predicates?
    \begin{partquestions}{\alph*}
        \item 4 equals 5.
        \item $n \geq 4$.
        \item $8 = 2 \times 4$.
        \item $x^2 - 1 = (x-1)(x+1)$.
        \item There is a prime number $p$ that is one less than a cube and two less than a square.
        \item $P(x) \equiv P(x)$.
        \item $P(x) \iff P(x)$ for all $x$.
    \end{partquestions}
\end{mdframed}
\textbf{Solution}:\newline
 \begin{partquestions}{\alph*}
        \item Statement.
        \item Predicate, since it depends on the value of $n$ for the truth or falsity of the expression.
        \item Statement.
        \item Predicate. Although it is true for all real $x$, the truth or falsity can only be determined when a value of $x$ is substituted.
        \item Statement. This is an existential statement.
        \item Predicate. The truth or falsity can only be determined when a value of $x$ is substituted.
        \item Statement. This is a universal statement.
    \end{partquestions}
\begin{mdframed}
    Assume $p$, $q$, and $r$ are statements. By considering a truth table,
    \begin{partquestions}{\alph*}
        \item is $(p \implies (q \implies r)) \equiv ((p \implies q) \implies r)$?
        \item is $(p \iff (q \iff r)) \equiv ((p \iff q) \iff r)$?
    \end{partquestions}
\end{mdframed}
\textbf{Solution}:\newline
 \begin{partquestions}{\alph*}
        \item We claim that they are not equivalent.
        \begin{table}[h]
            \centering
            \begin{tabular}{|l|l|l||l|l||l|l|}
                \hline
                $\boldsymbol{p}$ & $\boldsymbol{q}$ & $\boldsymbol{r}$ & $\boldsymbol{p \implies q}$ & $\boldsymbol{q \implies r}$ & $\boldsymbol{p \implies (q \implies r)}$ & $\boldsymbol{(p \implies q) \implies r}$ \\ \hline
                F & F & F & T & T & T & F \\ \hline
                F & F & T & T & T & T & T \\ \hline
                F & T & F & T & F & T & F \\ \hline
                F & T & T & T & T & T & T \\ \hline
                T & F & F & F & T & T & T \\ \hline
                T & F & T & F & T & T & T \\ \hline
                T & T & F & T & F & F & F \\ \hline
                T & T & T & T & T & T & T \\ \hline
            \end{tabular}
        \end{table}

        We can see from the truth table that if $p$ is false, $q$ is true, and $r$ is false, then $p \implies (q \implies r)$ is true while $(p \implies q) \implies r$ is false. Thus $(p \implies (q \implies r)) \not\equiv ((p \implies q) \implies r)$.

        \item We claim that they are equivalent.
        \begin{table}[H]
            \centering
            \begin{tabular}{|l|l|l||l|l||l|l|}
                \hline
                $\boldsymbol{p}$ & $\boldsymbol{q}$ & $\boldsymbol{r}$ & $\boldsymbol{p \iff q}$ & $\boldsymbol{q \iff r}$ & $\boldsymbol{p \iff (q \iff r)}$ & $\boldsymbol{(p \iff q) \iff r}$ \\ \hline
                F & F & F & T & T & F & F \\ \hline
                F & F & T & T & F & T & T \\ \hline
                F & T & F & F & F & T & T \\ \hline
                F & T & T & F & T & F & F \\ \hline
                T & F & F & F & T & T & T \\ \hline
                T & F & T & F & F & F & F \\ \hline
                T & T & F & T & F & F & F \\ \hline
                T & T & T & T & T & T & T \\ \hline
            \end{tabular}
        \end{table}

        By inspection, $(p \iff (q \iff r)) \equiv ((p \iff q) \iff r)$.
    \end{partquestions}
\begin{mdframed}
    For the following, assume $p$ and $q$ are statements. Which of the following are true statements?
    \begin{multicols}{2}
        \begin{partquestions}{\alph*}
            \item $\forall p\; \forall q\; (p \equiv q)$.
            \item $\forall p\; \exists q\; (p \equiv q)$.
            \item $\exists p\; \forall q\; (p \equiv q)$.
            \item $\exists p\; \exists q\; (p \equiv q)$.
            \item $\forall p\; \forall q\; (p \not\equiv q)$.
            \item $\forall p\; \exists q\; (p \not\equiv q)$.
            \item $\exists p\; \forall q\; (p \not\equiv q)$.
            \item $\exists p\; \exists q\; (p \not\equiv q)$.
        \end{partquestions}
    \end{multicols}
\end{mdframed}
\textbf{Solution}:\newline
 \begin{partquestions}{\alph*}
        \item False. Let $p$ be a false statement and $q$ be a true statement and the result is clear.
        \item True. Set $q$ to be the same as $p$ and result follows.
        \item False. For any statement $p$, if $p$ is true then set $q$ to be a false statement and set $q$ to be true otherwise. Result yields.
        \item True. Choose any two true statements and it will be clear.
        \item False. Let $p$ and $q$ be both the same statement and the result is clear.
        \item True. If $p$ is true then let $q$ be a false statement, and if $p$ is false then let $q$ be a true statement. The result follows.
        \item False. For any statement $p$, set $q$ to be the same as $p$ and result follows.
        \item True. Choose a true and false statement and result is clear.
    \end{partquestions}
\begin{mdframed}
    Simplify $\lnot p \land (\lnot p \implies (p \land q))$.
\end{mdframed}
\textbf{Solution}:\newline
 Recall from \myref{example-implication-law} that $(x \implies y) \equiv \lnot x \lor y$.
    \begin{align*}
        \lnot p \land (\lnot p \implies (p \land q)) &\equiv \lnot p \land (\lnot(\lnot p) \lor (p \land q)) & (\text{by } \myref{example-implication-law})\\
        &\equiv \lnot p \land (p \lor (p \land q))\\
        &\equiv \lnot p \land ((p \lor p) \land (p \lor q)) & (\text{Distributivity})\\
        &\equiv \lnot p \land (p \land (p \lor q))\\
        &\equiv (\lnot p \land p) \land (p \lor q) & (\text{Associativity})\\
        &\equiv \textbf{false} \land (p \lor q)\\
        &\equiv \textbf{false}.
    \end{align*}

\chapter{Proofs}
\section*{Exercises}
\begin{mdframed}
    Let $x$ be a positive real number. Prove that $x(1-x) > 0$ if $0 < x < 1$.
\end{mdframed}
\textbf{Solution}:\newline
 \begin{proof}
        Suppose $0 < x < 1$. Then $-1 < -x < 0$, meaning $0 < 1 - x < 1$. Therefore $x > 0$ and $1-x > 0$, so their product $x(1-x) > 0$.
    \end{proof}
\begin{mdframed}
    Prove that $m + n$ is even if the integers $m$ and $n$ have the same parity (i.e., both odd or both even).
\end{mdframed}
\textbf{Solution}:\newline
 \begin{proof}
        Suppose that $m$ and $n$ have the same parity. We split into two cases.
        \begin{itemize}
            \item If both $m$ and $n$ are even, then we may write $m = 2a$ and $n = 2b$ where $a$ and $b$ are integers. Hence
            \begin{align*}
                m + n &= (2a) + (2b) \\
                &= 2(a+b)
            \end{align*}
            which clearly means that $m + n$ is even.
            \item If both $m$ and $n$ are odd, then we may write $m = 2a + 1$ and $n = 2b + 1$ where $a$ and $b$ are integers. Hence
            \begin{align*}
                m + n &= (2a + 1) + (2b + 1)\\
                &= 2a + 2b + 2\\
                &= 2(a + b + 1)
            \end{align*}
            which clearly means that $m+n$ is even.
        \end{itemize}
    Hence, in both cases, $m + n$ is even.
    \end{proof}
\begin{mdframed}
    Suppose that $a$ and $b$ are integers. Prove that if $a(b^2 + 5)$ is even then either $a$ is even or $b$ is odd.
\end{mdframed}
\textbf{Solution}:\newline
 We consider a proof by contrapositive; the statement that we want to prove is ''if \textbf{not} ($a$ is even or $b$ is odd) then $a(b^2+5)$ is \textbf{not} even''. That is, ''if $a$ is \textbf{not} even \textbf{and} $b$ is \textbf{not} odd then $a(b^2+5)$ is \textbf{not} even'', meaning ''if $a$ is odd and $b$ is even then $a(b^2+5)$ is odd''.

    \begin{proof}
        Suppose that $a$ is odd and $b$ is even. Then we may write $a = 2m + 1$ and $b = 2n$ where $m$ and $n$ are integers. Hence
        \begin{align*}
            a(b^2+5) &= (2m+1)\left((2n)^2 + 5\right)\\
            &= (2m+1)(4n^2 + 5)\\
            &= 8mn^2 + 10m + 4n^2 + 5\\
            &= 8mn^2 + 10m + 4n^2 + 4 + 1\\
            &= 2(4mn^2 + 5m + 2n^2 + 2) + 1
        \end{align*}
    which clearly means that $a(b^2+5)$ is odd.
    \end{proof}
\begin{mdframed}
    Prove that no integers $a$ and $b$ exist such that $2a + 4b = 1$.
\end{mdframed}
\textbf{Solution}:\newline
 \begin{proof}
        By way of contradiction assume there exist integers $a$ and $b$ such that $2a + 4b = 1$. Then dividing both sides by 2 leads to $a + 2b = \frac12$. Note the left hand side is clearly an integer, while the right hand side is not an integer, a contradiction.
    \end{proof}
\begin{mdframed}
    Prove that $\frac{a+b}{2} \geq \sqrt{ab}$ if $a$ and $b$ are positive real numbers by way of contradiction.
\end{mdframed}
\textbf{Solution}:\newline
 \begin{proof}
        By way of contradiction assume that $a$ and $b$ are positive real numbers, and $\frac{a+b}{2} < \sqrt{ab}$. This means $a+b<2\sqrt ab$. Squaring both sides yields $(a+b)^2 < 4ab$. Note
        \[
            (a+b)^2 = a^2 + 2ab + b^2 < 4ab
        \]
        which implies $a^2 + b^2 < 2ab$, leading to $a^2 - 2ab + b^2 < 0$. However $a^2 - 2ab + b^2 = (a-b)^2 \geq 0$ for all positive real numbers $a$ and $b$. Hence we have $(a-b)^2 < 0$ and $(a-b)^2 \geq 0$, a contradiction.
    \end{proof}
\begin{mdframed}
    Prove using mathematical induction that $a^2 - 1$ is a multiple of 8 for all positive odd integers $a$.
\end{mdframed}
\textbf{Solution}:\newline
 We note that a positive odd number is of the form $2n - 1$ where $n$ is a positive integer.

    \begin{proof}
        Set $a = 2n - 1$; we induct on $n$.

        When $n = 1$, we have $a^2 - 1 = (2(1) - 1)^2 - 1 = 1 - 1 = 0$ which is clearly a multiple of 8.

        Assume that the statement holds for some positive integer $k$, i.e. $(2k-1)^2 - 1 = 8m$ for some integer $m$. We show that the statement holds for $k + 1$.

        We note that $(2k-1)^2 - 1 = 4k^2 - 4k$. Observe
        \begin{align*}
            (2(k+1)-1)^2 - 1 &= (2k+1)^2 - 1\\
            &= 4k^2 + 4k + 1 - 1\\
            &= (4k^2 - 4k) + 8k\\
            &= 8m + 8k & (\text{by hypothesis})\\
            &= 8(m+k)
        \end{align*}
        which means that $(2(k+1)-1)^2 - 1$ is a multiple of 8, proving that the statement holds for $k+1$. By mathematical induction, $a^2 - 1$ is a multiple of 8 for all positive odd integers $a$.
    \end{proof}
\begin{mdframed}
    Prove that every integer $n \geq 2$ is either prime or can be expressed as a product of primes.
\end{mdframed}
\textbf{Solution}:\newline
 \begin{proof}
        We use strong induction on $n$.

        When $n = 2$ the statement is true since 2 is prime.

        Now assume that for some positive integer $k \geq 2$, every integer $m$ satisfying $2 \leq m \leq k$ results in the statement being true, i.e. $m$ is either prime or can be expressed as a product of primes. We are to show that the statement is true for $k + 1$, i.e. $k+1$ is prime or can be expressed as a product of primes.

        Now if $k + 1$ is prime we are done. Otherwise $k + 1$ is composite, meaning that $k + 1 = ab$ for some integers $2 \leq a,b \leq k$. Applying the induction hypothesis on $a$ and $b$ means that $a$ and $b$ are primes or product of primes. Thus $k + 1$ is a product of primes.

        Therefore by mathematical induction, every integer $n \geq 2$ is either prime or can be expressed as a product of primes.
    \end{proof}
\begin{mdframed}
    Let $n$ be an integer. Prove that $n$ is one more than a multiple of 5 if and only if $n$ is of the form $5k - 4$ where $k$ is an integer.
\end{mdframed}
\textbf{Solution}:\newline
 \begin{proof}
        We prove the forward direction using direct proof. Assume $n$ is one more than a multiple of 5. Then we may write $n = 5a + 1$ where $a$ is an integer. Note $5a + 1 = 5a + (5 - 4) = (5a + 5) - 4 = 5(a+1) - 4$. Setting $k = a+1$ yields required result.

        We now prove the reverse direction, using direct proof as well. Assume $n = 5k - 4$. Observe $5k - 4 = 5k - 5 + 1 = 5(k-1) + 1$, meaning $n$ is one more than a multiple of 5.
    \end{proof}
\begin{mdframed}
    Prove that there is a positive integer that is one less than a perfect cube and two less than a perfect square.
\end{mdframed}
\textbf{Solution}:\newline
 \begin{proof}
        The number 7 satisfies this as $7 = 2^3 - 1$ and $7 = 3^2 - 2$.
    \end{proof}
\begin{mdframed}
    Let $x = \sqrt2$ and $y = 2\log_2{3}$. It may be assumed that $\sqrt2$ is irrational.
    \begin{partquestions}{\roman*}
        \item Prove that $y$ is irrational.
        \item Produce a constructive proof that there exist irrational $x$ and $y$ such that $x^y$ is rational.
    \end{partquestions}
\end{mdframed}
\textbf{Solution}:\newline
 \begin{partquestions}{\roman*}
        \item We use a proof by contradiction to prove this claim.

        \begin{proof}
            Seeking a contradiction, assume $y$ is rational. Write $y = \frac pq$ where $p$ and $q$ are integers. Note $2^y = 2^{\frac pq}$ and $2^y = 2^{2\log_2{3}} = 9$. Hence $2^{\frac pq} = 9$ meaning $2^p = 9^q$. However $2^p$ is always even and $9^q$ is always odd, a contradiction.
        \end{proof}

        \item \begin{proof}
            Note
            \[
                x^y = (\sqrt2)^{2\log_2{3}} = \left(\sqrt{2}^2\right)^{\log_2{3}} = 2^{\log_2{3}} = 3
            \]
            which is rational.
        \end{proof}
    \end{partquestions}

\section*{Problems}

\chapter{Algebra}
\section*{Exercises}
\begin{mdframed}
    Evaluate the following sums.
    \begin{multicols}{3}
        \begin{partquestions}{\alph*}
            \item $\displaystyle \sum_{i=3}^{5}(7ix+11)$
            \item $\displaystyle \sum_{x=3}^{5}(7ix+11)$
            \item $\displaystyle \sum_{i=-3}^{-5}(-7ix-11)$
            \item $\displaystyle \sum_{i=4}^{8}ijk$
            \item $\displaystyle \sum_{j=4}^{8}ijk$
            \item $\displaystyle \sum_{n=4}^{8}ijk$
            \item $\displaystyle \sum_{i=3}^{7}13$
            \item $\displaystyle \sum_{i=1}^{3}i^2 + \sum_{j=4}^{6}j^2$
            \item $\displaystyle \sum_{i=1}^{3}\left(\sum_{j=5}^{7}(i+j)\right)$
        \end{partquestions}
    \end{multicols}
\end{mdframed}
\textbf{Solution}:\newline
 \begin{partquestions}{\alph*}
        \item $\displaystyle \sum_{i=3}^{5}(7ix+11) = (7\times3x + 11) + (7\times4x + 11) + (7\times5x + 11) = 84x + 33$.
        \item $\displaystyle \sum_{x=3}^{5}(7ix+11) = (7i\times3 + 11) + (7i\times4 + 11) + (7i\times5 + 11) = 84i + 33$.
        \item Since the upper bound is smaller than the lower bound, the sum evaluates to 0.
        \item $\displaystyle \sum_{i=4}^{8}ijk = 4jk + 5jk + 6jk + 7jk + 8jk = 30jk$.
        \item $\displaystyle \sum_{j=4}^{8}ijk = 4ik + 5ik + 6ik + 7ik + 8ik = 30ik$.
        \item $\displaystyle \sum_{n=4}^{8}ijk = ijk + ijk + ijk + ijk + ijk = 5ijk$.
        \item $\displaystyle \sum_{i=3}^{7}13 = 13 + 13 + 13 + 13 + 13 = 65$.
        \item $\displaystyle \sum_{i=1}^{3}i^2 + \sum_{j=4}^{6}j^2 = (1^2 + 2^2 + 3^2) + (4^2 + 5^2 + 6^2) = 91$.
        \item $\displaystyle \sum_{i=1}^{3}\left(\sum_{j=5}^{7}(i+j)\right) = \sum_{i=1}^{3}\left((i+5) + (i+6) + (i+7)\right) = \sum_{i=1}^{3}\left(3i+18\right) = (3\times1 + 18) + (3\times2 + 18) + (3\times3 + 18) = 72$.
    \end{partquestions}
\begin{mdframed}
    Given $\displaystyle \sum_{i=1}^5a_i^2 = 100$, $\displaystyle \sum_{i=2}^5a_{i-1} = 10$, and $\displaystyle \sum_{i=1}^5(a_i+2)^2 = 200$, what is $a_5$?
\end{mdframed}
\textbf{Solution}:\newline
 Note $\displaystyle \sum_{i=2}^5a_{i-1} = \sum_{i=1}^4a_i = 10$. Also
    \begin{align*}
        200 = \sum_{i=1}^5(a_i+2)^2 &= \sum_{i=1}^5(a_i^2 + 4a_i + 4)\\
        &= \sum_{i=1}^5a_i^2 + 4\sum_{i=1}^5a_i + \sum_{i=1}^54\\
        &= \sum_{i=1}^5a_i^2 + 4\left(\sum_{i=1}^4a_i + \sum_{i=5}^5a_i\right) + \sum_{i=1}^54\\
        &= 100 + 4\left(10 + a_5\right) + 20\\
        &= 160 + 4a_5
    \end{align*}
    which therefore means $a_5 = 10$.
\begin{mdframed}
    Solve the inequality $x^2 + 15 < 8|x|$.
\end{mdframed}
\textbf{Solution}:\newline
 Rearrange $x^2 + 15 < 8|x|$ to become $x^2 - 8|x| + 15 < 0$, meaning $|x|^2 - 8|x| + 15 < 0$. Note $|x|^2 - 8|x| + 15 = (|x|-3)(|x|-5)$, so we are solving $(|x|-3)(|x|-5)<0$. Therefore $3 < |x| < 5$.
    \begin{itemize}
        \item If $|x| = -x$ (i.e., $x$ is negative), then $3 < -x < 5$, which means $-5 < x < -3$.
        \item If $|x| = x$ (i.e. $x$ is non-negative), then $3 < x < 5$.
    \end{itemize}
    Hence $-5 < x < -3$ or $3 < x < 5$.
\begin{mdframed}
    Find the coefficient of $x^7$ in $(7x-3)^9$.
\end{mdframed}
\textbf{Solution}:\newline
 The relevant term is
    \[
        {9 \choose 7}(7x)^7(-3)^{9-7} = -266827932x^6
    \]
    so its coefficient is -266827932.

\section*{Problems}
\begin{mdframed}
    Suppose that commutativity of multiplication (\myref{axiom-multiplication-is-commutative}) is no longer true. Expand $(x+y)^3$.
\end{mdframed}
\textbf{Solution}:\newline
 Without commutativity, we cannot use the binomial theorem (\myref{thrm-binomial}). We just have to expand it slowly.
    \begin{align*}
        (x+y)^3 &= (x+y)(x+y)(x+y)\\
        &= xxx + xxy + xyx + xyy + yxx + yxy + yyx + yyy\\
        &= x^3 + x^2y + xyx + xy^2 + yx^2 + yxy + y^2x + y^3.
    \end{align*}
\begin{mdframed}
    By using the substitution $u = x^2 - 2x - 2$, find all $x \in \R$ such that
    \[
        x^6 + 6x^4 + 16x^3 + 19 = 6x^5 + 12x^2 + 24x.
    \]
\end{mdframed}
\textbf{Solution}:\newline
 First we rewrite the equation to become $x^6 - 6x^5 + 6x^4 + 16x^3 - 12x^2 - 24x + 19 = 0$. Now we try to rewrite it using the given substitution.
    \begin{align*}
        &x^6 - 6x^5 + 6x^4 + 16x^3 - 12x^2 - 24x + 19\\
        &=x^4(x^2-2x-2) - 4x^3(x^2-2x-2) + 0x^2(x^2-2x-2) + 8x(x^2-2x-2)\\
        &\quad\quad+ 4(x^2-2x-2) + 27\\
        &= (x^4-4x^3+8x+4)(x^2-2x-2) + 27\\
        &= (x^2(x^2-2x-2) - 2x(x^2-2x-2) - 2(x^2-2x-2))(x^2-2x-2) + 27\\
        &= ((x^2-2x-2)(x^2-2x-2))(x^2-2x-2)+27\\
        &= (x^2-2x-2)^3 + 27\\
        &= u^3 + 3^3\\
        &= 0.
    \end{align*}
    Note $u^3 + 3^3 = (u+3)(u^2 - 3u + 9)$, so $u+3 = 0$ or $u^2 - 3u + 9 = 0$. One sees that the quadratic equation has no real solutions; so $u = -3$. Therefore $x^2 - 2x - 2 = -3$, meaning $x^2-2x+1 = 0$, Thus $x = 1$.
\begin{mdframed}
    Let $x \in [0,3]$.
    \begin{partquestions}{\roman*}
        \item Expand $(x-2)(x^2-3x+1)$.
        \item Hence find the range of values of $x$ such that
        \[
            \frac{4x^3 - 11x^2 - 47x + 106}{(x-2)(x^2-18)} \geq 3.
        \]
    \end{partquestions}
\end{mdframed}
\textbf{Solution}:\newline
 \begin{partquestions}{\roman*}
        \item $(x-2)(x^2-3x+1) = (x^3 - 3x^2 + x) - 2x^2 + 6x - 2 = x^3 - 5x^2 + 7x - 2$
        \item Note $(x-2)(x^2-18) = x^3 - 2x^2 - 18x + 36$. Subtracting both sides by 3 we see
        \begin{align*}
            &\frac{4x^3 - 11x^2 - 47x + 106}{(x-2)(x^2-18)} - 3\\
            &= \frac{4x^3 - 11x^2 - 47x + 106 - 3(x^3 - 2x^2 - 18x + 36)}{(x-2)(x^2-18)}\\
            &= \frac{x^3 - 5x^2 + 7x - 2}{(x-2)(x^2-18)}\\
            &= \frac{(x-2)(x^2-3x+1)}{(x-2)(x^2-18)}\\
            &= \frac{(x-2)\left(x - \frac{3+\sqrt5}{2}\right)\left(x - \frac{3-\sqrt5}{2}\right)}{(x-2)(x-3\sqrt2)(x+3\sqrt2)}\\
            &\geq 0.
        \end{align*}
        Therefore $x < -3\sqrt2$ or $\frac{3-\sqrt5}{2} \leq x \leq \frac{3+\sqrt5}{2}$ with $x \neq 2$ or $x > 3\sqrt2$. Hence, when restricting to $x \in [0, 3]$, we see $\frac{3-\sqrt5}{2} \leq x \leq \frac{3+\sqrt5}{2}$ with $x \neq 2$.
    \end{partquestions}
\begin{mdframed}
    Solve the inequality
    \[
        \frac{3|x-3|}{x^2-6x+5} > 1
    \]
    given $x \geq 0$.
\end{mdframed}
\textbf{Solution}:\newline
 For easier computation later, let $u = |x-3|$. Note that
    \[
        x^2 - 6x + 5 = (x-3)^2 - 4 = |x-3|^2 - 4 = u^2 - 4
    \]
    so the original inequality becomes
    \[
        \frac{3u}{u^2 - 4} > 1.
    \]
    Thus
    \begin{align*}
        \frac{3u}{u^2-4} - \frac{u^2 - 4}{u^2 - 4} &= \frac{-u^2 + 3u + 4}{u^2-4}\\
        &= \frac{-(u+1)(u-4)}{(u+2)(u-2)}\\
        &> 0.
    \end{align*}
    Note $u \geq 0$, so $u+1 > 0$ and $u+2 > 0$. So $\frac{u+1}{u+2} > 0$, meaning that we are only concerned with $\frac{-(u-4)}{u-2} > 0$, i.e. $\frac{u-4}{u-2} < 0$. Therefore $2 < u < 4$, meaning $2 < |x - 3| < 4$.

    Note $2 < |x - 3| < 4$ means $(|x-3| > 2) \land (|x-3| < 4)$. We split the inequality into two parts.
    \begin{itemize}
        \item First we solve $|x-3| > 2$. This means that $x - 3 > 2$ or $x - 3 < -2$, so $x > 5$ or $x < 1$, i.e. $x \in (5, \infty) \cup (-\infty, 1)$
        \item Now we solve $|x - 3| < 4$. Thus $-4 < x-3 < 4$, meaning $-1 < x < 7$, i.e. $x \in (-1, 7)$
    \end{itemize}
    Thus the solution we are seeking is the intersection of the two intervals,
    \[
        x \in ((5, \infty) \cup (-\infty, 1)) \cap (-1, 7)
    \]
    which therefore means $x \in (-1, 1) \cup (5, 7)$, i.e. $-1 < x < 1$ or $5 < x < 7$.
\begin{mdframed}
    Let $N$ be a positive integer.
    \begin{partquestions}{\roman*}
        \item Express $\frac{7x+4}{x^3+3x^2+2x}$ in the form $\frac{A}{x} + \frac{B}{x+1} + \frac{C}{x+2}$, where $A$, $B$, and $C$ are real constants to be determined.
        \item Hence simplify
        \[
            \frac{7N+9}{N^2+3N+2} + \sum_{r=1}^N \frac{7r+4}{r^3+3r^2+2r}.
        \]
    \end{partquestions}
\end{mdframed}
\textbf{Solution}:\newline
 \begin{partquestions}{\roman*}
        \item Setting $\frac{7x+4}{x^3+3x^2+2x} = \frac{A}{x} + \frac{B}{x+1} + \frac{C}{x+2}$ and multiplying both sides by $x^3+3x^2+2x = x(x+1)(x+2)$ we see that $7x+4 = A(x+1)(x+2) + Bx(x+2) + Cx(x+1)$.
        \begin{itemize}
            \item Setting $x = 0$ we see that $4 = A(0+1)(0+2) = 2A$, meaning $A = 2$.
            \item Setting $x = -1$ we see $7(-1) + 4 = B(-1)(-1+2)$, i.e. $-3 = -B$, meaning $B = 3$.
            \item Setting $x = -2$ we see $7(-2) + 4 = C(-2)(-2+1)$, i.e. $-10 = 2C$, meaning $C = -5$.
        \end{itemize}
        Therefore $\frac{7x+4}{x^3+3x^2+2x} = \frac{2}{x} + \frac{3}{x+1} - \frac{5}{x+2}$.

        \item Note that
        \begin{align*}
            \sum_{r=1}^N \frac{7r+4}{r^3+3r^2+2r} &= \sum_{r=1}^N \left(\frac{2}{r} + \frac{3}{r+1} - \frac{5}{r+2}\right)\\
            &= \frac21 + \frac32 - \frac53\\
            &+ \frac22 + \frac33 - \frac54\\
            &+ \frac23 + \frac34 - \frac55\\
            &+ \frac24 + \frac35 - \frac56\\
            &\cdots\\
            &+ \frac2{N-3} + \frac{3}{N-2} - \frac{5}{N-1}\\
            &+ \frac2{N-2} + \frac{3}{N-1} - \frac{5}{N}\\
            &+ \frac2{N-1} + \frac{3}{N} - \frac{5}{N+1}\\
            &+ \frac2{N} + \frac{3}{N+1} - \frac{5}{N+2}\\
            &= \frac21 + \frac32 + \frac22 - \frac{5}{N+1} + \frac{3}{N+1} - \frac{5}{N+2}\\
            &= \frac92 - \frac{2}{N+1} - \frac{5}{N+2}\\
            &= \frac92 - \frac{7N+9}{N^2+3N+2}
        \end{align*}
        so we see
        \begin{align*}
            \frac{7N+9}{N^2+3N+2} + \sum_{r=1}^N \frac{7r+4}{r^3+3r^2+2r} &= \frac{7N+9}{N^2+3N+2} + \left(\frac92 - \frac{7N+9}{N^2+3N+2}\right)\\
            &= \frac92.
        \end{align*}
    \end{partquestions}
\begin{mdframed}
    Prove that
    \[
        \sum_{r=0}^n (r\times r!) = (n+1)! - 1
    \]
    for all non-negative integers $n$.
\end{mdframed}
\textbf{Solution}:\newline
 We consider a proof by induction. We induct on $n$.

    When $n = 0$, it is clearly true that $0 \times 0! = 0 = (0+1)! - 1$. Thus the statement holds for $n = 0$.

    Now assume that the statement holds for some non-negative integer $k$, meaning
    \[
        \sum_{r=0}^k (r\times r!) = (k+1)! - 1.
    \]
    We are to show that it holds for $k+1$, i.e.
    \[
        \sum_{r=0}^{k+1} (r\times r!) = (k+2)! - 1.
    \]

    We see that
    \begin{align*}
        \sum_{r=0}^{k+1} (r\times r!) &= \left(\sum_{r=0}^k (r\times r!)\right) + ((k+1) \times (k+1)!)\\
        &= ((k+1)! - 1) + ((k+1) \times (k+1!)) & (\text{by induction hypothesis})\\
        &= (k+1)! + (k+1) \times (k+1)! - 1\\
        &= (1 + (k+1))\times(k+1)! - 1\\
        &= (k+2) \times (k+1)! - 1\\
        &= (k+2)! - 1
    \end{align*}
    proving the statement for $k + 2$.

    By mathematical induction we see
    \[
        \sum_{r=0}^n (r\times r!) = (n+1)! - 1
    \]
    for all non-negative integers $n$.
\begin{mdframed}
    Let $x_1, x_2, \dots, x_n$ be real numbers.
    \begin{partquestions}{\roman*}
        \item Prove that $|xy| = |x||y|$ for all real numbers $x$ and $y$.
        \item Prove that $|x_1x_2\cdots x_n| = |x_1||x_2|\cdots|x_n|$.
    \end{partquestions}
\end{mdframed}
\textbf{Solution}:\newline
 \begin{partquestions}{\roman*}
        \item We consider three cases.
        \begin{itemize}
            \item The first case is when $x$ and $y$ are both non-negative. Then $|x| = x$, $|y| = y$, and $|xy| = xy$. Thus $|xy| = xy = |x||y|$.
            \item The second case is if $x$ and $y$ are both negative. So $|x| = -x$, $|y| = -y$, and $|xy| = xy$, since the product of two negative real numbers is a positive real number. Therefore $|xy| = xy = (-x)(-y) = |x||y|$.
            \item The last case is when one of them is non-negative and the other is negative. Without loss of generality, assume $x$ is non-negative and $y$ is negative. Then their product is negative. So $|x| = x$, $|y| = -y$, and $|xy| = -xy$. Hence $|xy| = -xy = (x)(-y) = |x||y|$.
        \end{itemize}
        Therefore $|xy| = |x||y|$ for all real numbers $x$ and $y$.

        \item We induct on positive integer values for $n$.

        When $n = 1$ the statement holds trivially since $|x_1| = |x_1|$.

        Assume that the statement holds for some positive integer $k$, i.e. $|x_1\cdots x_k| = |x_1|\cdots|x_k|$. We show that the statement holds for $k+1$, i.e. $|x_1\cdots x_kx_{k+1}| = |x_1|\cdots|x_k||x_{k+1}|$

        Observe
        \begin{align*}
            |x_1x_2\cdots x_kx_{k+1}| &= |(x_1x_2\cdots x_k)(x_{k+1})|\\
            &= |x_1x_2\cdots x_k||x_{k+1}| & (\text{by part (\textbf{i})})\\
            &= (|x_1||x_2|\cdots|x_k|)|x_{k+1}| & (\text{by induction hypothesis})\\
            &= |x_1||x_2|\cdots|x_k||x_{k+1}|
        \end{align*}
        proving the statement for $k + 1$.

        Thus, by mathematical induction, $|x_1x_2\cdots x_n| = |x_1||x_2|\cdots|x_n|$ for all real values $x_1, x_2, \dots, x_n$.
    \end{partquestions}

\chapter{Relations And Functions}
\section*{Exercises}
\begin{mdframed}
    Let $A = \{0, 1, 2, 3\}$ and define the relation $T = \{(0, 1), (2, 3)\}$ on $A$.
    \begin{partquestions}{\alph*}
        \item Is $T$ reflexive?
        \item Is $T$ symmetric?
        \item Is $T$ transitive?
    \end{partquestions}
\end{mdframed}
\textbf{Solution}:\newline
 \begin{partquestions}{\alph*}
        \item Not reflexive since $0\not\mathrel{T}0$.
        \item Not symmetric since $0\mathrel{T}1$ but $1\not\mathrel{T}0$.
        \item Is transitive. The condition for transitivity is vacuously satisfied.
    \end{partquestions}
\begin{mdframed}
    Is the less-than-or-equal-to relation, $\leq$, an equivalence relation on $\Z$?
\end{mdframed}
\textbf{Solution}:\newline
 Not an equivalence relation, since it is not symmetric: $1 \leq 2$ but $2 \not\leq 1$.
\begin{mdframed}
    Let the function $f: \{1, 2, 3\} \to \{1, 4, 9, 16, 25\}$ be such that $f(x) = x^2$.
    \begin{partquestions}{\roman*}
        \item Use arrow notation to write a definition for $f$.
        \item State the domain, codomain, and range of $f$.
        \item What is the image of 2 under $f$?
        \item Is the function $g: \{1, 2, 3\} \to \{1, 8\}, x \mapsto x^3$ \textit{valid}?
    \end{partquestions}
\end{mdframed}
\textbf{Solution}:\newline
 \begin{partquestions}{\roman*}
        \item $f: \{1, 2, 3\} \to \{1, 4, 9, 16, 25\}, x \mapsto x^2$. (Or just $x \mapsto x^2$)
        \item Domain is $\{1, 2, 3\}$, codomain is $\{1, 4, 9, 16, 25\}$, range is $\{1, 4, 9\}$.
        \item The image of 2 under $f$ is $2^2 = 4$.
        \item No. The element 3 would map to 27, which is not in the codomain.
    \end{partquestions}
\begin{mdframed}
    Is $f: \Q \to \Z,\;\frac pq \mapsto p + q$ a well-defined function?
\end{mdframed}
\textbf{Solution}:\newline
 It is not well-defined. Note $\frac 12 = \frac 24$, but $f(\frac12) = 1 + 2 = 3$ and $f(\frac24) = 2 + 4 = 6$.
\begin{mdframed}
    Let $f: \mathbb{R} \to \mathbb{R}$ and $g: \mathbb{R} \to \mathbb{R}$. Write down the rule of the function $fg$ if $f(x) = x^2 - x + 1$ and $g(y) = \frac1{y^2+1}$.
\end{mdframed}
\textbf{Solution}:\newline
 $fg(x) = \left(\frac1{x^2+1}\right)^2 - \frac1{x^2+1} + 1$.
\begin{mdframed}
    Define the function $f: \mathbb{N} \to \Z$ such that
    \[
        f(x) = \begin{cases}
            \frac{x}{2} & \text{ if } x \text{ is even}\\
            \frac{1-x}{2} & \text{ if } x \text{ is odd}
        \end{cases}
    \]
    By considering $f$, prove that $|\mathbb{N}| = |\Z|$.
\end{mdframed}
\textbf{Solution}:\newline
 We prove the requirements of a bijection one by one.
    \begin{itemize}
        \item \textbf{Injective}: Suppose $x_1, x_2 \in \mathbb{N}$ such that $f(x_1) = f(x_2)$. We split into three cases.
        \begin{itemize}
            \item The first case is if $f(x_1) = f(x_2) = 0$. In this case, one sees clearly that $x_1 = x_2 = 1$.
            \item The second case is if $f(x_1) = f(x_2) > 0$. Now since $x_1 \neq 1$ and $x_2 \neq 1$ (as this case leads to $f(x_1) = 0$), the `valid' odd numbers are at least 3. Therefore one sees $\frac{1-x_1}{2} \leq \frac{1-3}{2} = -1 < 0$, so $x_1$ and $x_2$ cannot be odd. Hence, $x_1$ and $x_2$ are even, meaning $\frac{x_1}{2} = \frac{x_2}{2}$ which quickly implies $x_1 = x_2$.
            \item The third case is if $f(x_1) = f(x_2) < 0$. As argued above, this means that $x_1$ and $x_2$ must be odd numbers of at least 3. Hence, $\frac{1-x_1}{2} = \frac{1-x_2}{2}$ which quickly implies $x_1 = x_2$.
        \end{itemize}
        Thus, in all three cases, $f(x_1) = f(x_2)$ implies $x_1 = x_2$, meaning $f$ is injective.
        \item \textbf{Surjective}: Suppose $y \in \Z$. We split into three cases again.
        \begin{itemize}
            \item If $y = 0$, then setting $x = 1$ satisfies $f(x) = y$.
            \item Now suppose $y > 0$. We note $2y \in \mathbb{N}$, and clearly $2y$ is an even integer. So setting $x = 2y$ satisfies $f(x) = \frac{2y}{y} = y$.
            \item Suppose $y < 0$. Note $-2y > 0$, and $1 - 2y > 0 \in \mathbb{N}$. Furthermore $1 - 2y$ is clearly an odd integer. Hence setting $x = 1 - 2y$ satisfies $f(x) = \frac{1-(1-2y)}{2} = y$.
        \end{itemize}
        Therefore for every $y \in \Z$, there exists a pre-image $x \in \mathbb{N}$ such that $f(x) = y$. Hence $f$ is surjective.
    \end{itemize}
    Therefore, as $f$ is both injective and surjective, $f$ is bijective. Hence, $|\mathbb{N}| = |\Z|$.

\section*{Problems}
\begin{mdframed}
    Let the set $A = \{1, 2, 3, 4\}$. Find a function $f: A \to A$ that is bijective but is \textbf{not} equal to the function $g: A \to A, x \mapsto x$.
\end{mdframed}
\textbf{Solution}:\newline
 Consider the function $f: A \to A$ such that $1 \mapsto 2$, $2 \mapsto 3$, $3 \mapsto 4$, and $4 \mapsto 1$. Then clearly it is both injective and surjective, meaning that $f$ is bijective.
\begin{mdframed}
    Let the set $S = \{1, 2, 3\}$. Which of the following is/are relations on $S$? If it is a relation on $S$, is it an equivalence relation?
    \begin{partquestions}{\alph*}
        \item $A = \emptyset$
        \item $B = \{(1, 1), (2, 2), (3, 3)\}$
        \item $C = \{(1, 1), (1, 2), (2, 2), (3, 3)\}$
        \item $D = \{(1, 1), (1, 2), (2, 1), (2, 2), (2, 3), (3, 2), (3, 3)\}$
    \end{partquestions}
\end{mdframed}
\textbf{Solution}:\newline
 \begin{partquestions}{\alph*}
        \item Is a relation since $A = \emptyset \subseteq S^2$. $A$ is not an equivalence relation since it is not reflexive (e.g., $1\not\mathrel{A}1$).

        \item Is a relation since $B \subseteq S^2$. Note $B$ is an equivalence relation.
        \begin{itemize}
            \item Clearly it is reflexive.
            \item If $x = y$ then clearly $x\mathrel{B}y$ means $y\mathrel{B}x$. For $x \neq y$, the condition for symmetry is vacuously true. Therefore $B$ is symmetric.
            \item If $x = y = z$ then clearly if $x\mathrel{B}y$ and $y\mathrel{B}z$ then $x\mathrel{B}z$. Otherwise the condition for transitivity is vacuously true. Therefore $B$ is transitive.
        \end{itemize}
        Since $B$ is reflexive, symmetric, and transitive, therefore $B$ is an equivalence relation.

        \item Is a relation since $C \subseteq S^2$. $C$ is not an equivalence relation since it is not symmetric ($1\mathrel{C}2$ but $2\not\mathrel{C}1$).

        \item Is a relation since $D \subseteq S^2$. $D$ is not an equivalence relation since it is not transitive ($1\mathrel{D}2$ and $2\mathrel{D}3$ but $1\not\mathrel{D}3$).
    \end{partquestions}
\begin{mdframed}
    Let the functions
    \begin{align*}
        &f: [0, \infty) \to \mathbb{R},\; x\mapsto x^2+1,\\
        &g: (-1, 1] \to \mathbb{R},\; x\mapsto 1-x^2,\\
        &h: (-\infty, -1] \to \mathbb{R},\; x\mapsto \ln(-x).
    \end{align*}
    \begin{partquestions}{\alph*}
        \item Which of the given function(s) are injective? If they are, prove it. If not, provide a counterexample.
        \item Does the composite function $hf$ exist? If so, give a definition of $hf$ using arrow notation and state its image. Otherwise, explain why not.
        \item Does the composite function $fg$ exist? If so, give a definition of $fg$ using arrow notation and state its image. Otherwise, explain why not.
    \end{partquestions}
\end{mdframed}
\textbf{Solution}:\newline
 \begin{partquestions}{\alph*}
        \item We look at each function individually.
        \begin{itemize}
            \item $\boxed{f}$ Suppose $x_1,x_2 \in [0,\infty)$ such that $f(x_1) = f(x_2)$, i.e. $x_1^2 + 1 = x_2^2 + 1$. Therefore $x_1^2 = x_2^2$. Since $x_1$, $x_2$ are non-negative, we can directly take square root to yield $x_1 = x_2$, meaning $f$ is injective.
            \item $\boxed{g}$ Note $g(0.5) = 1-(0.5)^2 = 1 - (-0.5)^2 = g(-0.5)$ so $g$ is not injective.
            \item $\boxed{h}$ Suppose $x_1,x_2 \in (-\infty, -1]$ such that $h(x_1) = h(x_2)$, i.e. $\ln(-x_1) = \ln(-x_2)$. Exponentiating both sides by $e$ yields $-x_1 = -x_2$ which trivially means $x_1 = x_2$. Therefore $h$ is injective.
        \end{itemize}
        Therefore $f$ and $h$ are the two injective functions.

        \item We note that $\im f = [1, \infty)$. But $[1, \infty)$ is not a subset of $(-\infty, -1]$, which is the domain of $h$. Therefore $hf$ does not exist.

        \item We note that $\im g = [0, 1]$, and $[0, 1]$ is a subset of $[0, \infty)$ which is the domain of $f$. Therefore $fg$ does exist.

        One sees that
        \[
            fg: (-1, 1] \to \R, x \mapsto (1-x^2)^2 + 1.
        \]
        which has an image of $[1, 2]$.
    \end{partquestions}
\begin{mdframed}
    Let $\sim$ be an equivalence relation on a set $A$, and let $x$ and $y$ be elements in $A$. Prove that
    \begin{partquestions}{\alph*}
        \item if $[x] = [y]$ then $[x] \cap [y] \neq \emptyset$;
        \item if $[x] \cap [y] \neq \emptyset$ then $x \mathrel{\sim} y$; and
        \item if $x \mathrel{\sim} y$ then $[x] = [y]$.
    \end{partquestions}
\end{mdframed}
\textbf{Solution}:\newline
 \begin{partquestions}{\alph*}
        \item Suppose $[x] = [y]$. Then $[x] \cap [y] = [x]$. As $x \sim x$, therefore $x \in [x] \cap [y]$, which means $[x] \cap [y] \neq \emptyset$.

        \item Suppose $[x] \cap [y] \neq \emptyset$. Then there is an $a \in [x] \cap [y]$. So $a \in [x]$ and $a \in [y]$. By definition of equivalence classes, we see $x \mathrel{\sim} a$ and $y \mathrel{\sim} a$. Note that $y \mathrel{\sim} a$ means $a \mathrel{\sim} y$ by symmetry of $\sim$. Therefore, $x \mathrel{\sim} y$ by transitivity of $\sim$.

        \item Suppose $x \mathrel{\sim} y$. Then $y \mathrel{\sim} x$ since $\sim$ is a symmetric relation. Note that for any $a \in [x]$, we know $x \mathrel{\sim} a$ by definition of equivalence class. As $y \mathrel{\sim} x$ and $x \mathrel{\sim} a$, therefore $y \mathrel{\sim} a$ by transitivity of $\sim$. Therefore $a \in [y]$. Hence $[x] \subseteq [y]$. Switching the roles of $x$ and $y$ also yields $[y] \subseteq [x]$, which therefore leads to the conclusion that $[x] = [y]$.
    \end{partquestions}
\begin{mdframed}
    Let $X$ be any set, and let $f: X \to X$, $g: X \to X$, and $h: X \to X$ be functions. Suppose $h$ is injective. Prove or disprove the following statements.
    \begin{partquestions}{\alph*}
        \item If $hf = hg$ then $f = g$.
        \item If $fh = gh$ then $f = g$.
    \end{partquestions}
\end{mdframed}
\textbf{Solution}:\newline
 \begin{partquestions}{\alph*}
        \item Prove. Given $h(f(x)) = h(g(x))$ for all $x \in X$. Since $h$ is injective, thus $f(x) = g(x)$ for all $x \in X$. Therefore $f = g$.

        \item Disprove. Let $X = \Z$, and consider the functions
        \begin{align*}
            &f: n \mapsto n\\
            &g: n \mapsto \begin{cases}
                n & \text{if } n \text{ is even}\\
                0 & \text{otherwise}
            \end{cases}\\
            &h: n \mapsto 2n
        \end{align*}

        We first claim that $h$ is injective. Suppose $m,n\in\Z$ such that $h(m) = h(n)$. Then $2m = 2n$, which clearly means $m = n$. Thus $h$ is injective.

        Now we show that $fh = gh$. Let $n \in \Z$ be arbitrary; then
        \begin{align*}
            fh(n) &= f(h(n))\\
            &= f(2n) & (\text{definition of } h)\\
            &= 2n & (\text{definition of } f)\\
            &= g(2n) & (\text{since } 2n \text{ is even})\\
            &= g(h(n)) & (\text{definition of } h)
        \end{align*}
        which means $fh = gh$.

        But clearly $f \neq g$. Thus the statement is false.
    \end{partquestions}
\begin{mdframed}
    The McCarthy 91 function $M: \Z \to \Z$ is a recursive function created by computer scientist John McCarthy to test formal verification in computer science. It is defined as follows.
    \[
        M(n) = \begin{cases}
            n - 10 & \text{if } n > 100\\
            M(M(n+11)) & \text{if } n \leq 100
        \end{cases}
    \]
    \begin{partquestions}{\roman*}
        \item Show $M(101) = 91$.
        \item Show $M(n) = M(n+1)$ for all $90 \leq n \leq 100$.
        \item Deduce that $M(n) = 91$ for any $90 \leq n \leq 100$.
        \item Hence prove that $M(n) = 91$ for all $n \leq 100$.
    \end{partquestions}
\end{mdframed}
\textbf{Solution}:\newline
 \begin{partquestions}{\roman*}
        \item $M(101) = 101 - 10 = 91$.

        \item For $90 \leq n \leq 100$, we see that $101 \leq n+11 \leq 111$, meaning $n+11 > 100$. Therefore
        \begin{align*}
            M(n) &= M(M(n+11))\\
            &= M((n+11) - 10) & (\text{since } n+11 > 100)\\
            &= M(n+1).
        \end{align*}

        \item Note by part \textbf{(ii)} we see $M(90) = M(90 + 1) = M(91)$. Similarly, $M(91) = M(91 + 1) = M(92)$. Therefore
        \[
            M(90) = M(91) = M(92) = \cdots = M(99) = M(100) = M(101) = 91
        \]
        which means $M(n) = 91$ for all $90 \leq n \leq 100$.

        \item For clarity, set $n = 101 - x$. We induct on positive integer values of $x$.

        Base cases of $x = 1, 2, \dots, 11$ were proven in \textbf{(iii)}.

        Suppose for some positive integer $k$ every $1 \leq r \leq k$ satisfies $M(101 - r) = 91$. We show that the statement holds for $k + 11$, i.e. $M(101 - (k+11)) = M(90 - k) = 91$.

        Since $k$ is positive, therefore $90 - k < 90 < 100$. Hence
        \begin{align*}
            M(90 - k) &= M(M((90 - k) + 11))\\
            &= M(M(101 - k))\\
            &= M(91) & (\text{induction hypothesis})\\
            &= 91 & (\text{by part \textbf{(iii)}})
        \end{align*}
        which means that the statement holds for $k + 11$.

        Therefore by mathematical induction we have $M(101 - x) = 91$ for all positive integers $x$. Equivalently, this means that $M(n) = 91$ for all integers $n \leq 100$.
    \end{partquestions}
\begin{mdframed}
    Let the function $f: \left(\mathbb{N}\right)^2\to\mathbb{N}$ be defined such that
    \[
        f(m, n) =
        \begin{cases}
            n & \text{if } m = 1, \\
            m & \text{if } n = 1, \\
            f\left(n-1,f(n-1,m-1)\right) & \text{otherwise.}
        \end{cases}
    \]
    \begin{partquestions}{\roman*}
        \item Prove that $f(n,2) = n - 1$ for all integers $n > 1$.
        \item Prove that $f(n+1, n) = 2$ for all positive integers $n$.
        \item Prove that $f(n, 4) = 2$ for all integers $n > 1$.
    \end{partquestions}
\end{mdframed}
\textbf{Solution}:\newline
 \begin{partquestions}{\roman*}
        \item Note that
        \begin{align*}
            f(n,2) &= f(1, f(1, n-1))\\
            &= f(1, n-1)\\
            &= n-1
        \end{align*}
        for all integers $n > 1$.

        \item We induct on $n$.

        We show the base cases of 1 and 2 hold:
        \begin{itemize}
            \item When $n = 1$, we have $f(2, 1) = 2$, so the first case is true.
            \item When $n = 2$, we have
            \[
                f(3,2) = f(1, f(1, 2)) = f(1, 2) = 2
            \]
            so the second case is true.
        \end{itemize}

        Now suppose for some positive integer $k$, every integer $1 \leq m \leq k$ results in the statement being true, i.e. $f(m+1,m) = 2$. We want to show that the case for $k+1$ is true, i.e. $f(k+2, k+1) = 2$.
        \begin{align*}
            f(k+2, k+1) &= f(k, f(k, k+1))\\
            &= f(k, f(k, f(k, k-1)))\\
            &= f(k, f(k, 2)) & (\text{hypothesis on } k-1)\\
            &= f(k, f(1, f(1, k-1)))\\
            &= f(k, f(1, k-1))\\
            &= f(k, k-1) \\
            &= 2 & (\text{hypothesis on } k-1)
        \end{align*}
        which proves that the statement for $k+1$ holds. Hence by mathematical induction, $f(n+1, n) = 2$.

        \item We first prove the `general' case.
        \begin{align*}
            f(n, 4) &= f(3, f(3, n-1))\\
            &= f(3, f(n-2, f(n-2, 2)))\\
            &= f(3, f(n-2, n-1)) & (\text{by \textbf{(i)}})\\
            &= f(3, f(n-2, f(n-2, n-3)))\\
            &= f(3, f(n-2, f((n-3)+1, n-3)))\\
            &= f(3, f(n-2, 2)) & (\text{by \textbf{(ii)}})\\
            &= f(3, n-3) & (\text{by \textbf{(i)}})\\
            &= f(n-4, f(n-4, 2))\\
            &= f(n-4, n-5) & (\text{by \textbf{(i)}})\\
            &= f((n-5)+1, n-5)\\
            &= 2. & (\text{by \textbf{(ii)}})
        \end{align*}
        Note that this only works for $n > 5$. So we now have to independently verify the cases $n = 2$, $n = 3$, $n = 4$, and $n = 5$.
        \begin{itemize}
            \item If $n = 2$, then
            \begin{align*}
                f(2, 4) &= f(3, f(3, 1))\\
                &= f(3, 3)\\
                &= f(2, f(2, 2))\\
                &= f(2, 1) & (\text{by \textbf{(i)}})\\
                &= 2
            \end{align*}
            \item If $n = 3$, then
            \begin{align*}
                f(3, 4) &= f(3, f(3, 2))\\
                &= f(3, 2) & (\text{by \textbf{(ii)}})\\
                &= 2 & (\text{by \textbf{(ii)}})
            \end{align*}
            \item If $n = 4$, then
            \begin{align*}
                f(4, 4) &= f(3, f(3, 3))\\
                &= f(3, f(2, f(2, 2)))\\
                &= f(3, f(2, 1)) & (\text{by \textbf{(i)}})\\
                &= f(3, 2)\\
                &= 2 & (\text{by \textbf{(ii)}})
            \end{align*}
            \item Finally, if $n = 5$ we see $f(5, 4) = 2$ by \textbf{(ii)}.
        \end{itemize}
        Therefore, combining these cases with the general result for $n > 5$, we see that $f(n, 4)$ for all integers $n > 1$.
    \end{partquestions}
\begin{mdframed}
    Let $X$ be a non-empty finite set of cardinality $n$ and let $f: X \to X$ be a injective function. Prove that $f$ is bijective.\newline
    (\textit{Hint: consider strong induction on $n$.})
\end{mdframed}
\textbf{Solution}:\newline
 We consider strong induction on $n$.

    When $n = 1$, the set $X$ has only one element, say $X = \{a\}$. The only possible injective function is $f: X \to X, a \mapsto a$ which is clearly surjective. Hence $f$ is bijective, proving the base case.

    Assume that the statement holds for all $r \leq k$ for some positive integer $k$, i.e. for any set $S$ with cardinality $r \leq k$, every injective function $f: S \to S$ is injective. We prove the case for a set with cardinality $k + 1$.

    Let $X$ be a set with cardinality $k + 1$. Seeking a contradiction, suppose $f: X \to X$ is a injective function that is not surjective. Hence $|\im f| < k + 1$, meaning $|\im f| \leq k$. Define the function $g: \im f \to \im f$ where $g(t) = f(t)$. We note $g$ is injective since if $g(t_1) = g(t_2)$ this means $f(t_1) = f(t_2)$. As $f$ is injective thus $t_1 = t_2$, meaning $g$ is injective. By the Inductive Hypothesis we know $g$ is bijective, meaning $\im g = \im f$. So for any $y \in \im f$ (where $\im f$ is the codomain of $g$) there exists a $x \in \im f$ (where $\im f$ is the domain of $g$) such that $g(x) = y$. By definition of $g$ we see $f(x) = y$. Hence $f$ is surjective. But we assumed that $f$ is not surjective, a contradiction. Hence $f$ must be injective, proving the statement for $k + 1$.

    Therefore any injective function from a finite set to itself must be surjective.

\chapter{Number Theory}
\section*{Exercises}
\begin{mdframed}
    Express $-210$ in the form $a-13b$, where $a$ and $b$ are positive integers with $0 \leq a \leq 12$.
\end{mdframed}
\textbf{Solution}:\newline
 $-210 = 11 - 13 \times 17$, so $a = 11$ and $b = 17$.
\begin{mdframed}
    Find $\gcd(-112, -35)$.
\end{mdframed}
\textbf{Solution}:\newline
 $\gcd(-112, -35) = 7$ since $-112 = -16 \times 7$ and $-35 = -5 \times 7$, with 7 being the largest integer that achieves this.
\begin{mdframed}
    Find $\lcm(-112, -35)$.
\end{mdframed}
\textbf{Solution}:\newline
 $\lcm(-112, -35) = 560$ since $560 = -5 \times -112$ and $-35 = -16 \times -35$, with 560 being the smallest \textit{positive} integer that achieves this.
\begin{mdframed}
    Suppose $m = 42$ and $n = 70$.
    \begin{partquestions}{\roman*}
        \item Let $d = \gcd(m,n)$. Find $d$.
        \item Hence find $\lcm(m,n)$.
        \item Find a pair of integers $x$ and $y$ such that $mx + ny = d$.
    \end{partquestions}
\end{mdframed}
\textbf{Solution}:\newline
 \begin{partquestions}{\roman*}
        \item $\gcd(42, 70) = 14$ since $42 = 3 \times 14$ and $70 = 5 \times 14$, and 14 is the largest integer achieving this.

        \item $\lcm(42, 70) = \frac{42 \times 70}{\gcd(m, n)} = \frac{2940}{14} = 210$ by \myref{prop-product-of-gcd-and-lcm}.

        \item Note that $x = 2$ and $y = -1$ works as $42 \times 2 + 70 \times (-1) = 84 - 70 = 14$.
    \end{partquestions}
\begin{mdframed}
    Express 44100 as a product of primes.
\end{mdframed}
\textbf{Solution}:\newline
 $44100 = 2^2 \times 3^2 \times 5^2 \times 7^2$.
\begin{mdframed}
    Let $m = 5$ and $n = 3$.
    \begin{partquestions}{\alph*}
        \item State the value of $17 \mod m$.
        \item Find an $x$ where $0 \leq x < m$ and $19 \equiv x \pmod m$.
        \item If $A = 1234n + 5$, what is $A \mod n$?
    \end{partquestions}
\end{mdframed}
\textbf{Solution}:\newline
 \begin{partquestions}{\alph*}
        \item $17 \mod 5 = 2$ since $17 = 3 \times 5 + 2$ by Euclid's division lemma (\myref{lemma-euclid-division}).
        \item As $19 = 3 \times 5 + 4$, thus $19 \equiv 4 \pmod 5$. Hence $x = 4$.
        \item $A \mod n$ equals $5 \mod 3$ which is 2.
    \end{partquestions}
\begin{mdframed}
    Explain why $-n \equiv n \pmod{2n}$.
\end{mdframed}
\textbf{Solution}:\newline
 $-n = (-1) \times 2n + n$, which means that $-n \equiv n \pmod{2n}$.
\begin{mdframed}
    Find the last two digits of $778899^{112233}$.
\end{mdframed}
\textbf{Solution}:\newline
 Finding the last two digits of a number is the same as finding the remainder of that number when divided by 100. We note $778899 \equiv 99 \pmod{100}$, so $778899^{112233} \equiv 99^{112233} \pmod{100}$. Furthermore, $99 \equiv -1 \pmod{100}$, so $99^{112233}\equiv (-1)^{112233} \equiv -1 \equiv 99 \pmod{100}$. Hence the last two digits of $778899^{112233}$ are both 9.
\begin{mdframed}
    Find the multiplicative inverse of 123 modulo 5.
\end{mdframed}
\textbf{Solution}:\newline
 Note $123 \equiv 3 \pmod 5$. One can easily find by trial and error that 2 is the multiplicative inverse of 3, since $3 \times 2 = 6 \equiv 1 \pmod 5$. Hence the multiplicative inverse of 123 is 2 modulo 5.

\section*{Problems}
\begin{mdframed}
    Find the positive integer $a$ such that $\gcd(a, 50) = 5$ and $\lcm(a, 50) = 150$.
\end{mdframed}
\textbf{Solution}:\newline
 We know that $50a = 5 \times 50 = 750$ by \myref{prop-product-of-gcd-and-lcm}. Thus $a = 15$.
\begin{mdframed}
    Let $a$ and $b$ be positive integers such that $\lcm(a, b) = a^2$. What does this imply about $b$?
\end{mdframed}
\textbf{Solution}:\newline
 Given that $\lcm(a,b) = a^2$, we use \myref{prop-product-of-gcd-and-lcm} to see that
    \[
        ab = \gcd(a,b)\lcm(a,b) = \gcd(a,b)a^2.
    \]
    Let $d = \gcd(a,b)$, so $ab = da^2$ which means $b = da$. Note $\lcm(a,b) = \lcm(a, da) = da$, so $a^2 = da$ which means $d = a$. Therefore $b = a^2$.
\begin{mdframed}
    Prove $5^{2n+3} \equiv 5 \pmod 8$ for all non-negative integers $n$.
\end{mdframed}
\textbf{Solution}:\newline
 Note
    \begin{align*}
        5^{2n+3} &= 5^3 \times 5^{2n}\\
        &= 125 \times 25^n \\
        &\equiv (5) \times (1)^n & (\text{since } 25 \equiv 1 \pmod8 \text{ and } 125 \equiv 8 \pmod8)\\
        &= 5 \pmod8
    \end{align*}
    so $5^{2n+3} \equiv 5 \pmod8$ for all positive integers $n$.
\begin{mdframed}
    Show $n^2 \vert (1+n)^n - 1$ for all positive integers $n$.
\end{mdframed}
\textbf{Solution}:\newline
 Note that
    \begin{align*}
        (1+n)^n - 1 &= \left(1 + {n \choose 1}n + {n \choose 2}n^2 + \cdots + {n \choose n}n^n\right) - 1\\
        &= {n \choose 1}n + {n \choose 2}n^2 + \cdots + {n \choose n}n^n.
    \end{align*}
    From the second term onwards we see that they are all multiples of $n^2$; the first term is exactly $n^2$. Therefore $n^2 \vert (1+n)^n - 1$ for all positive integers $n$.
\begin{mdframed}
    Prove $7^{3^n} \equiv 1 \pmod 9$ for all positive integers $n$.
\end{mdframed}
\textbf{Solution}:\newline
 We induct on $n$.

    When $n = 1$, clearly $7^{3^1} = 7^3 = 343$ which is a multiple of 9.

    Now assume that $7^{3^k} \equiv 1 \pmod9$ for some positive integer $k$; we show that $7^{3^{k+1}} \equiv 1 \pmod9$ for $k + 1$.

    Note
    \begin{align*}
        7^{3^{k+1}} &= 7^{3\times3^k}\\
        &= \left(7^{3^k}\right)^3\\
        &\equiv 1^3 & (\text{by induction hypothesis})\\
        &= 1 \pmod9
    \end{align*}
    so the statement holds for $k+1$.

    Therefore by mathematical induction we see $7^{3^n} \equiv 1 \pmod9$ for all positive integers $n$.
\begin{mdframed}
    Find all even perfect squares that are not multiples of 4.
\end{mdframed}
\textbf{Solution}:\newline
 Consider any integer $x$; it has to have a remainder of 0, 1, 2, or 3 when divided by 4. Thus we consider 4 cases.
    \begin{itemize}
        \item If $x \equiv 0 \pmod4$, then $x^2 \equiv 0 \pmod4$.
        \item If $x \equiv 1 \pmod4$, then $x^2 \equiv 1 \pmod4$.
        \item If $x \equiv 2 \pmod4$, then $x^2 \equiv 4 \equiv 0 \pmod4$.
        \item If $x \equiv 3 \pmod4$, then $x^2 \equiv 9 \equiv 1 \pmod4$.
    \end{itemize}
    In any case, the remainder of any perfect square modulo 4 is either 0 or 1. If it is 1, then $x$ is odd; if it is 0 then $x$ is even. Thus if $x$ is an even perfect square then $x^2$ necessarily has to have a remainder 0 modulo 4, i.e. $x^2$ is a multiple of 4. Therefore no such even perfect square exists.
\begin{mdframed}
    Find the last 3 digits of $57^{2023}$.
\end{mdframed}
\textbf{Solution}:\newline
 We work modulo 1000 to find the last 3 digits of the number. We note that $57^1 = 57$, $57^2 = 3249 \equiv 249 \pmod{1000}$, $57^3 = 185193 \equiv 193 \pmod{1000}$, and $57^4 = 10556001 \equiv 1 \pmod{1000}$. We note that powers of 57 cycle between 57, 249, 193, and 1 when taking modulo 1000.

    Note $2023 = 4\times505+3$, so we see
    \begin{align*}
        57^{2023} &= 57^{4\times505+3}\\
        &= 57^3 \times \left(57^4\right)^{505}\\
        &\equiv 193 \times 1^{505}\\
        &= 193 \pmod{1000}
    \end{align*}
    which means $57^{2023}$ ends in 193.
\begin{mdframed}
    Let $n$ be an integer.
    \begin{partquestions}{\roman*}
        \item Prove that if $n$ is positive, then $6 \vert 2n^3 + 3n^2 + n$.
        \item Prove that $12 \vert n^4 - n^2$.
    \end{partquestions}
\end{mdframed}
\textbf{Solution}:\newline
 \begin{partquestions}{\roman*}
        \item We induct on $n$.

        When $n = 1$ we see that $6 \vert 2(1)^3 + 3(1)^2 + (1) = 6$ so the statement holds for $n = 1$.

        Assume that the statement is true for some positive integer $k$, i.e. $6 \vert 2k^3 + 3k^2 + k$, meaning $2k^3 + 3k^2 + k = 6a$ for some integer $a$. We show that the statement also holds for $k + 1$, i.e. $6 \vert 2(k+1)^3 + 3(k+1)^2 + (k+1)$, meaning $2(k+1)^3 + 3(k+1)^2 + (k+1) = 6b$ for some integer $b$.

        Note
        \begin{align*}
            &2(k+1)^3 + 3(k+1)^2 + (k+1)\\
            &= 2(k^3+3k^2+3k+1) + 3(k^2+2k+1) + (k+1)\\
            &= 2k^3 + 9k^2 + 13k + 6\\
            &= (2k^3 + 3k^2 + k) + (12k + 6)\\
            &= 6a + 6(2k + 1)\\
            &= 6(a + 2k + 1)
        \end{align*}
        so $6 \vert 2(k+1)^3 + 3(k+1)^2 + (k+1)$.

        Therefore by mathematical induction we see that $6 \vert 2n^3 + 3n^2 + n$ for all positive integers $n$.

        \item We induct on positive integer values of $n$.

        When $n = 1$ clearly $12 \vert 1^4 - 1^2 = 0$.

        Assume that for some positive integer $k$ we have $12 \vert k^4 - k^2$, i.e. $k^4 - k^2 = 12a$ for some integer $a$. We show that the statement holds for $k+1$, i.e. $12 \vert (k+1)^4 - (k+1)^2$, meaning $(k+1)^4 - (k+1)^2 = 12b$ for some integer $b$.

        Note
        \begin{align*}
            &(k+1)^4 - (k+1)^2\\
            &= (k^4 + 4k^3 + 6k^2 + 4k + 1) - (k^2 + 2k + 1)\\
            &= k^4 + 4k^3 + 5k^2 + 2k\\
            &= (k^4 - k^2) + (4k^3 + 6k^2 + 2k)\\
            &= 12a + 2(2k^3 + 3k^2 + 1k) & (\text{by induction hypothesis})\\
            &= 12a + 2(6m) & (\text{by \textbf{(i)}})\\
            &= 12a + 12m\\
            &= 12(a+m)
        \end{align*}
        so $12 \vert (k+1)^4 - (k+1)^2$.

        Therefore by mathematical induction we see that $12 \vert n^4 - n^2$ for all positive values of $n$.

        Now $12 \vert 0^4 - 0^2 = 0$, so it also works for $n = 0$.

        Finally, since $n^4 - n^2 = (-n)^4 - (-n)^2$, this also holds for negative values of $n$.
    \end{partquestions}
\begin{mdframed}
    Prove \myref{theorem-n-divides-ab-and-n-coprime-with-a-implies-n-divides-b} by considering B\'ezout's lemma (\myref{lemma-bezout}).
\end{mdframed}
\textbf{Solution}:\newline
 Since $n$ and $a$ are coprime, therefore there exist integers $\lambda$ and $\mu$ such that
    \[
        \lambda n + \mu a = 1
    \]
    by B\'ezout's lemma (\myref{lemma-bezout}). Multiplying both sides by $b$ yields
    \[
        \lambda nb + \mu ab = b.
    \]
    Since $n \vert ab$ by assumption, thus $ab = kn$ for some integer $k$. Therefore
    \[
        \lambda nb + \mu kn = n(\lambda b + \mu k) = b.
    \]
    which means that $b$ is divisible by $n$ as required.
\begin{mdframed}
    Let $a$ and $b$ be non-negative integers.
    \begin{partquestions}{\roman*}
        \item Prove that $\gcd(a^2, b^2) = \gcd(a,b)^2$.
        \item Find a similar expression for $\lcm(a^2, b^2)$.
    \end{partquestions}
\end{mdframed}
\textbf{Solution}:\newline
 \begin{partquestions}{\roman*}
        \item By B\'{e}zout's Lemma (\myref{lemma-bezout}) we can find integers $x$ and $y$ such that $ax + by = \gcd(a, b)$. For brevity let $d = \gcd(a,b)$, so $ax+by=d$.

        We take the non-obvious step of considering $d^3$. We see
        \[
            d^3 = (ax+by)^3 = a^3x^3 + 3a^2bx^2y + 3ab^2xy^2 + b^3y^3.
        \]
        By definition of the GCD, we know $d$ divides both $a$ and $b$, so $\frac ad$ and $\frac bd$ are integers. So by dividing both sides of the above equality by $d$ we see
        \begin{align*}
            d^2 &= \frac{a^3x^3}{d} + \frac{3a^2bx^2y}{d} + \frac{3ab^2xy^2}{d} + \frac{b^3y^3}{d}\\
            &= a^2\underbrace{\left(\frac{a}{d}x^3 + 3\frac{b}{d}x^2y\right)}_{\text{an integer}} + b^2\underbrace{\left(3\frac{a}{d}xy^2 + \frac{b}{d}y^3\right)}_{\text{an integer}}.
        \end{align*}
        Note B\'{e}zout's Lemma tells us that integers of the form $a^2\lambda + b^2\mu$ (where $\lambda$ and $\mu$ are integers) are multiples of $\gcd(a^2, b^2)$. Therefore $d^2$ is a multiple of $\gcd(a^2, b^2)$, i.e. $\gcd(a^2, b^2) \vert d^2$.

        Also as $d$ divides both $a$ and $b$, thus $d^2$ divides both $a^2$ and $b^2$. Therefore we see that $d^2 \vert \gcd(a^2, b^2)$ by \myref{prop-gcd-divides-common-divisor}.

        Therefore, as $\gcd(a^2, b^2) \vert d^2$ and $d^2 \vert \gcd(a^2, b^2)$, one obtains the fact that $d^2 = (\gcd(a,b))^2 = \gcd(a^2, b^2)$.

        \item We see that
        \begin{align*}
            \lcm(a^2,b^2) &= \frac{a^2b^2}{\gcd(a^2,b^2)} & (\text{by } \myref{prop-product-of-gcd-and-lcm})\\
            &= \frac{a^2b^2}{(\gcd(a,b))^2}\\
            &= \frac{ab}{\gcd(a,b)} \times \frac{ab}{\gcd(a,b)}\\
            &= \lcm(a,b) \times \lcm(a,b) & (\text{by } \myref{prop-product-of-gcd-and-lcm})\\
            &= (\lcm(a,b))^2
        \end{align*}
        so $\lcm(a^2,b^2) = (\lcm(a,b))^2$.
    \end{partquestions}

\chapter{Intro To Groups}
\section*{Exercises}
\begin{mdframed}
    How many symmetries are there in the symmetric group of degree 6? In other words, what is the order of the group above?\newline
    (\textit{Hint: Consider the number of permutations in the group.})
\end{mdframed}
\textbf{Solution}:\newline
 There are 6! = 720 possible permutations of 6 points, so there are 720 symmetries in the group given. That is, the order of the symmetric group of degree 6 is 720.

\section*{Problems}
\begin{mdframed}
Determine whether the following are groups. If they are, prove it. If not, explain why they are not groups.
\begin{partquestions}{\alph*}
    \item $(\Z, +)$.
    \item $(\Z \setminus \{0\}, \times)$ where $\times$ denotes regular multiplication.
    \item $(\R \setminus \{0\}, \times)$ where $\times$ denotes regular multiplication.
    \item $(\{0\}, \times)$ where $\times$ denotes regular multiplication.
    \item $(\{1\}, +)$ where $+$ denotes regular addition.
    \item $(\{1\}, \times)$ where $\times$ denotes regular multiplication.
\end{partquestions}
\end{mdframed}
\textbf{Solution}:\newline
 \begin{partquestions}{\alph*}
        \item This is a group. Addition is clearly closed and associative. The identity is 0. The inverse of any element $x$ is $-x$.
        \item This is not a group. Inverses do not exist. For example, the element $2$ does not have an inverse under multiplication.
        \item This is a group. Multiplication is clearly closed and associative. The identity is $1$. The inverse of any element $x$ is $\frac1x$.
        \item This is a group. Multiplication is clearly closed and associative. The identity is $0$ and the inverse is $0$.
        \item This is not a group. Addition is not closed: $1 + 1 = 2$ which is not in the group.
        \item This is a group. Multiplication is clearly closed and associative. The identity is $1$ and the inverse is $1$.
    \end{partquestions}
\begin{mdframed}
    Show that the \textbf{trivial group}\index{group!trivial} $(\{e\}, *)$ where $e \ast e = e$ is indeed a group.
\end{mdframed}
\textbf{Solution}:\newline
 We show that the trivial group is indeed a group by showing that the four group axioms hold.
    \begin{itemize}
        \item \textbf{Closure}: The only element in the underlying set is $e$, and $e \ast e = e \in \{e\}$. Thus the structure is closed under $\ast$.
        \item \textbf{Associativity}: Clearly $e \ast (e \ast e) = e \ast e = e$ and $(e \ast e) \ast e = e \ast e = e$ which means that $\ast$ is associative.
        \item \textbf{Identity}: The identity is clearly $e$.
        \item \textbf{Inverse}: The only element is $e$, and because $e \ast e = e$ hence $e$ is its own inverse.
    \end{itemize}
    Therefore $(\{e,\}, \ast)$ is a group.

\chapter{Basics Of Groups}
\section*{Exercises}
\begin{mdframed}
    By using a Cayley table, show that $(\Z_6, \otimes_6)$ does \textbf{not} form a group.
\end{mdframed}
\textbf{Solution}:\newline
 The Cayley table of $(\Z_6, \otimes_6)$ is as follows:
    \begin{table}[H]
        \centering
        \begin{tabular}{|l|l|l|l|l|l|l|}
        \hline
        \textbf{$\otimes_n$} & \textbf{0} & \textbf{1} & \textbf{2} & \textbf{3} & \textbf{4} & \textbf{5} \\ \hline
        \textbf{0}       & 0          & 0          & 0          & 0          & 0          & 0          \\ \hline
        \textbf{1}       & 0          & 1          & 2          & 3          & 4          & 5          \\ \hline
        \textbf{2}       & 0          & 2          & 4          & 0          & 2          & 4          \\ \hline
        \textbf{3}       & 0          & 3          & 0          & 3          & 0          & 3          \\ \hline
        \textbf{4}       & 0          & 4          & 2          & 0          & 4          & 2          \\ \hline
        \textbf{5}       & 0          & 5          & 4          & 3          & 2          & 1          \\ \hline
        \end{tabular}
    \end{table}

    Since the identity is $1$, and the row (and column) of 0 does not have a $1$, thus $0$ does not have an inverse. Therefore $(\Z_6, \oplus_6)$ is not a group.
\begin{mdframed}
Prove that for all $x$ in $G$, $\left(x^{-1}\right)^{-1} = x$.
\end{mdframed}
\textbf{Solution}:\newline
 Note that $(xx^{-1})^{-1} = (x^{-1})^{-1}x^{-1}$ by Shoes and Socks and $(xx^{-1})^{-1} = e^{-1} = e$. Thus $(x^{-1})^{-1}x^{-1} = e$. Multiplying both sides on the right by $x$ yields $(x^{-1})^{-1} = ex = x$, i.e. $(x^{-1})^{-1} = x$.
\begin{mdframed}
    Prove that $(x^{-1})^n = (x^n)^{-1}$ for all non-negative integers $n$.
\end{mdframed}
\textbf{Solution}:\newline
 We consider a proof by induction via inducting on $n$.

    The base case of $n = 0$ clearly holds true since
    \begin{align*}
        (x^{-1})^0 &= e & (\text{definition of }g^0 \text{ for any }g\in G)\\
        &= e^{-1} & (\myref{prop-inverse-of-identity-is-identity})\\
        &= (x^0)^{-1}. & (\text{definition of }x^0)
    \end{align*}

    Now assume that the statement holds for a non-negative integer $k$, i.e. $(x^{-1})^k = (x^k)^{-1}$. We are to show that the statement holds for $k+1$, i.e. $(x^{-1})^{k+1} = (x^{k+1})^{-1}$.

    Observe that
    \begin{align*}
        (x^{-1})^{k+1} &= (x^{-1})^k \ast x^{-1} & (\text{by statement 1})\\
        &= (x^k)^{-1} \ast x^{-1} & (\text{by hypothesis})\\
        &= (x\ast x^k)^{-1} & (\text{by Shoes and Socks})\\
        &= (x^{k+1})^{-1} & (\text{by statement 1})
    \end{align*}
    so the statement is true for $k+1$.

    Thus, by induction, we have $(x^{-1})^n = (x^n)^{-1}$ for any non-negative integer $n$.
\begin{mdframed}
    Let $i$ be a number such that $i^2 = -1$. Let $\mathcal{S} = \{1, -1, i, -i\}$.
    \begin{partquestions}{\roman*}
        \item Find the identity of the group $(\mathcal{S}, \times)$ where $\times$ denotes regular multiplication.
        \item Find the orders of the elements of the above group.
    \end{partquestions}
\end{mdframed}
\textbf{Solution}:\newline
 \begin{partquestions}{\roman*}
        \item The identity is $1$ since:
        \begin{itemize}
            \item $1 \times 1 = 1$;
            \item $1 \times (-1) = (-1) \times 1 = -1$;
            \item $1 \times i = i \times 1 = i$; and
            \item $1 \times (-i) = (-i) \times 1 = -i$.
        \end{itemize}
        \item The order of the identity $1$ is 1, so we look at the other elements:
        \begin{itemize}
            \item $|-1| = 2$ since $-1 \neq 1$ and $(-1)^2 = -1 \times -1 = 1$.
            \item $|i| = 4$ since $i \neq 1$, $i^2 = -1 \neq 1$, $i^3 = -i \neq 1$, but $i^4 = 1$.
            \item $|-i| = 4$ since $-i \neq 1$, $(-i)^2 = -1 \neq 1$, $(-i)^3 = i \neq 1$, but $(-i)^4 = 1$.
        \end{itemize}
    \end{partquestions}
\begin{mdframed}
    Using the set $\mathcal{S}$ from the above example, find the other generator of the group $(\mathcal{S}, \times)$.
\end{mdframed}
\textbf{Solution}:\newline
 $-i$ is the other generator since $(-i)^1 = -i$, $(-i)^2 = -1$, $(-i)^3 = i$, and $(-i)^4 = 1$.
\begin{mdframed}
    Simplify $rsr^4sr^3$ in the group $D_6$.
\end{mdframed}
\textbf{Solution}:\newline
 We work slowly:
    \begin{align*}
        rsr^4sr^3 &= r(sr^4)(sr^3)\\
        &= r(r^2s)(r^3s)\\
        &= r^3sr^3s\\
        &= r^3(sr^3)s\\
        &= r^3(r^3s)s\\
        &= r^6s^2\\
        &= e
    \end{align*}

\section*{Problems}
\begin{mdframed}
    Draw the Cayley table for $D_4$, the dihedral group of order 8, representing the symmetries of a square.\newline
    By referring to the Cayley table,
    \begin{partquestions}{\alph*}
        \item explain why $D_4$ is \textit{not} abelian;
        \item simplify $r^3srsr^3sr^3sr^2$.
    \end{partquestions}
\end{mdframed}
\textbf{Solution}:\newline
 The group table of $D_4$ is given as follows.
    \begin{table}[H]
        \centering
        \begin{tabular}{|l|l|l|l|l|l|l|l|l|}
        \hline
        $\ast$ & $e$    & $r$    & $r^2$  & $r^3$  & $s$    & $rs$   & $r^2s$ & $r^3s$ \\ \hline
        $e$    & $e$    & $r$    & $r^2$  & $r^3$  & $s$    & $rs$   & $r^2s$ & $r^3s$ \\ \hline
        $r$    & $r$    & $r^2$  & $r^3$  & $e$    & $rs$   & $r^2s$ & $r^3s$ & $s$    \\ \hline
        $r^2$  & $r^2$  & $r^3$  & $e$    & $r$    & $r^2s$ & $r^3s$ & $s$    & $rs$   \\ \hline
        $r^3$  & $r^3$  & $e$    & $r$    & $r^2$  & $r^3s$ & $s$    & $rs$   & $r^2s$ \\ \hline
        $s$    & $s$    & $r^3s$ & $r^2s$ & $rs$   & $e$    & $r^3$  & $r^2$  & $r$    \\ \hline
        $rs$   & $rs$   & $s$    & $r^3s$ & $r^2s$ & $r$    & $e$    & $r^3$  & $r^2$  \\ \hline
        $r^2s$ & $r^2s$ & $rs$   & $s$    & $r^3s$ & $r^2$  & $r$    & $e$    & $r^3$  \\ \hline
        $r^3s$ & $r^3s$ & $r^2s$ & $rs$   & $s$    & $r^3$  & $r^2$  & $r$    & $e$    \\ \hline
        \end{tabular}
    \end{table}
    \begin{partquestions}{\alph*}
        \item $D_4$ is not abelian because $rs \neq sr = r^3s$.
        \item We simplify $r^3srsr^3sr^3sr^2$.
        \begin{align*}
            r^3 sr sr^3 sr^3 sr^2 &= r^3srs(r^3s)(r^3s)r^2\\
            &= r^3 srs(e)r^2\\
            &= r^3 sr sr^2\\
            &= r^2(rs rs)r^2\\
            &= r^2(e)r^2\\
            &= r^4\\
            &= e
        \end{align*}
    \end{partquestions}
\begin{mdframed}
    Prove that $\Q$ under addition forms an abelian group.\newline
    (\textit{Note: addition is assumed to be associative and commutative by \myref{axiom-addition-is-associative} and \myref{axiom-addition-is-commutative} respectively, so you do not need to prove them.})
\end{mdframed}
\textbf{Solution}:\newline
 We need to prove each of the group axioms in order to prove that $(\Q, +)$ is indeed a group.
    \begin{itemize}
        \item \textbf{Closure}: Let $\frac ab$ and $\frac cd$ be rational numbers where $b, d \neq 0$. Their sum is $\frac{ad+bc}{bd}$, which is also rational. Therefore $\Q$ is closed under addition.

        \item \textbf{Associativity}: Addition is associative by \myref{axiom-addition-is-associative}.

        \item \textbf{Identity}: 0 is the identity since
        \[
            0 + \frac ab = \frac ab + 0 = \frac ab
        \]
        for any rational number $\frac ab$ (with $b \neq 0$).

        \item \textbf{Inverse}: For any rational number $\frac ab$, its inverse is $-\frac ab$ since
        \[
            \frac ab + \left(-\frac ab\right) = \left(-\frac ab\right) + \frac ab = 0
        \]
        for any rational number $\frac ab$ (with $b \neq 0$).
    \end{itemize}
    Furthermore addition is assumed to be commutative by \myref{axiom-addition-is-commutative}. Therefore $(\Q, +)$ is an abelian group.
\begin{mdframed}
    If every element in a group $G$ is its own inverse, show that $G$ is abelian.
\end{mdframed}
\textbf{Solution}:\newline
 If every element in $G$ is its own inverse, then for every element $g$ in $G$, $g^{-1} = g$. Consider $(gh)^{-1}$ where $g$ and $h$ are elements in $g$. On one hand, by Shoes and Socks, $(gh)^{-1} = h^{-1}g^{-1} = hg$ since each element is its own inverse. On the other hand, since $gh$ is an element in $G$, thus $(gh)^{-1} = gh$. Thus $gh = hg$ which means $G$ is abelian.
\begin{mdframed}
    Let $G$ be a group with identity $e$. Suppose an element $x$ in $G$ has finite order $n$. Prove that a positive integer $m$ is a multiple of $n$ if and only if $x^m = e$.\newline
    (\textit{Hint: consider Euclid's division lemma (\myref{lemma-euclid-division}) to prove one direction of the claim.})
\end{mdframed}
\textbf{Solution}:\newline
 Recall that $n = |x|$ is the smallest positive integer that satisfies $x^n = e$.

    We prove the forward direction first. Suppose $m$ is a multiple of $n$, say $m = qn$ for some integer $q$. Then
    \[
        x^m = x^{qn} = \left(x^n\right)^q = e^q = e
    \]
    which means $x^m = e$.

    We now prove the reverse direction. Suppose $x^m = e$. Using Euclid's division lemma (\myref{lemma-euclid-division}), we write $m = qn + r$ where $q$ and $r$ are integers with $0 \leq r < n$. Hence
    \[
        x^m = x^{qn + r} = x^{qn}x^r = \left(x^n\right)^qx^r = e^qx^r = x^r.
    \]
    Note that for all integers $k$ where $1 \leq k < n$, we have $x^k \neq e$ since $n$ is the smallest positive integer such that $x^n = e$. Hence, if $x^r = e$, we conclude $r = 0$. Therefore $m = qn$, meaning $m$ is a multiple of $n$.
\begin{mdframed}
    Let $G$ be a group.
    \begin{partquestions}{\alph*}
        \item Suppose $(gh)^2 = g^2h^2$ for all elements $g$ and $h$ in $G$. Prove that $G$ is abelian.
        \item Suppose $G$ is abelian. Prove that $(gh)^n = g^nh^n$ for all elements $g$ and $h$ in $G$ and for all positive integers $n$.
    \end{partquestions}
\end{mdframed}
\textbf{Solution}:\newline
 \begin{partquestions}{\alph*}
        \item Note that $(gh)^2 = ghgh$. Given that $(gh)^2 = g^2h^2 = gghh$. By cancellation law, $hg = gh$ which means $G$ is abelian.
        \item Suppose $G$ is abelian. Clearly $(gh)^1 = gh$. Suppose $(gh)^{k} = g^kh^k$ for some positive integer $k$. Then
        \begin{align*}
            (gh)^{k+1} &= (gh)(gh)^k\\
            &= (gh)(g^kh^k) & (\text{by assumption})\\
            &= ghg^kh^k\\
            &= g(hg^k)h^k\\
            &= g(g^kh)h^k & (\text{since } G \text{ is abelian})\\
            &= gg^khh^k\\
            &= g^{k+1}h^{k+1}
        \end{align*}
        so $(gh)^{k+1} = g^{k+1}h^{k+1}$ assuming $(gh)^k = g^kh^k$. Thus the claim is proven by mathematical induction.
    \end{partquestions}
\begin{mdframed}
    Let the group $G = (\Z_n, \oplus_n)$. Show that $G$ is cyclic with order $n$.
\end{mdframed}
\textbf{Solution}:\newline
 Note that $|1| = n$ since $1^2 = 1 \oplus_n 1 = 2$, $1^3 = 1 \oplus_n 1 \oplus_n 1 = 3$, $1^4 = 4$, ..., $1^{n-1} = n-1$ and $1^n = 0$ which is the identity. Since the group $(\Z_n, \oplus_n)$ has an element with the same order as the group, it is thus cyclic with order $n$ and generator 1.
\begin{mdframed}
    Let the set $S = \R^2$, that is,
    \[
        S = \{(x, y) \vert x, y \in \R\}.
    \]
    Let the transformation $T: S \to S$ be defined by
    \[
        T(x, y) = (-y, x+y).
    \]
    Define the set $A = \{T^r \vert r \in \Z \text{ and } r \geq 1\}$. Show that $A$ is a group under function composition ($\circ$), and state the order of this group.
\end{mdframed}
\textbf{Solution}:\newline
 We show that $(A, \circ)$ is a group.
    \begin{itemize}
            \item \textbf{Closure}: Function composition is closed by definition.
            \item \textbf{Associativity}: Function composition is associative.
            \item \textbf{Identity}: By performing brute-force computation, we find that $T^6(x, y) = (x, y)$. Hence $T^6$ is the identity of $A$.
            \item \textbf{Inverse}: If $r = 6$ then $T^r$ is its own inverse. Otherwise, $T^{6-r}$ is the inverse of $T^r$.
    \end{itemize}
    Thus, $(A, \circ)$ is a group, with order 6.

\chapter{Subgroups}
\section*{Exercises}
\begin{mdframed}
    Let $G$ be a group with identity $e$.
    \begin{partquestions}{\alph*}
        \item Prove that the \textbf{trivial subgroup}\index{subgroup!trivial}, $\{e\}$, is a subgroup of $G$.
        \item Prove that $G \leq G$.
    \end{partquestions}
\end{mdframed}
\textbf{Solution}:\newline
 \begin{partquestions}{\alph*}
        \item We will prove this claim by using the 3 axioms. Clearly $\{e\} \subseteq G$.
        \begin{itemize}
            \item The only element in $\{e\}$ is $e$, and $e \ast e = e \in \{e\}$. Hence $\{e\}$ is closed.
            \item The identity of the group $G$ is $e$ which is in $\{e\}$.
            \item The inverse of $e$ is $e$ which is in $\{e\}$.
        \end{itemize}
        Hence, $\{e\} \leq G$.
        \item We note that $G$ is a subset of $G$, and that $G$ is a group. Thus, $G$ is a subgroup of $G$ by definition of a subgroup.
    \end{partquestions}
\begin{mdframed}
    Let $G$ be a group, and let $g \in G$. Show that $\langle g \rangle = \{g^k \vert k \in \Z\}$ is a subgroup of $G$, called the \textbf{cyclic subgroup}\index{subgroup!cyclic} generated by $g$.
\end{mdframed}
\textbf{Solution}:\newline
 Since $G$ is closed, thus $g^n \in G$ for any integer $n$. Therefore $\langle g \rangle$ is a subset of $G$. Furthermore $e = g^0 \in \langle g\rangle$ by definition of $\langle g\rangle$, so $\langle g \rangle$ is non-empty.

    Now suppose $x$ and $y$ are in $\langle g \rangle$, meaning that we may write $x = g^m$ and $y = g^n$ for some integers $m$ and $n$. One sees clearly that
    \begin{align*}
        xy^{-1} &= g^m\left(g^n\right)^{-1}\\
        &= g^mg^{-n}\\
        &= g^{m-n}\\
        &\in \langle g\rangle.
    \end{align*}

    Therefore $\langle g \rangle \leq G$ by the subgroup test (\myref{thrm-subgroup-test}).
\begin{mdframed}
    Let $G$ be a group, $H \leq G$, and $g \in G$. Define the set $S = \{ghg^{-1} \vert h \in H\}$. Prove that $S \leq G$ under the group operation of $G$.
\end{mdframed}
\textbf{Solution}:\newline
 Note that $h \in G$ for any $h \in H$ (as $H \leq G$ so $H \subseteq G$). Since $G$ is closed, thus $ghg^{-1} \in G$ for any $h \in H$. Therefore $S$ is a subset of $G$. Also one sees clearly that $e$ is in $S$ since $geg^{-1} = gg^{-1} = e$ and $e \in H$, so $S$ is non-empty.

    Now suppose $x$ and $y$ are in $S$. Then there exist elements $h_x$ and $h_y$ in $H$ such that $x = gh_xg^{-1}$ and $y = gh_yg^{-1}$. Note that
    \begin{align*}
        xy^{-1} &= (gh_xg^{-1})(gh_yg^{-1})^{-1}\\
        &= (gh_xg^{-1})(g{h_y}^{-1}g^{-1}) & (\text{Shoes and Shocks})\\
        &= gh_xg^{-1}g{h_y}^{-1}g^{-1} & (\text{associativity})\\
        &= gh_x{h_y}^{-1}g^{-1} & (g^{-1}g = e).
    \end{align*}
    Note that since $H \leq G$, thus $h_x{h_y}^{-1} \in H$. Hence $xy^{-1} = g(h_x{h_y}^{-1})g^{-1} \in S$.

    Therefore $S \leq G$ by the subgroup test (\myref{thrm-subgroup-test}).
\begin{mdframed}
    Let $G$ be the group $(\Z_8, \oplus_8)$. Let $H = \{0, 4\} \leq G$.
    \begin{partquestions}{\alph*}
        % \item Explain why $gH = Hg$ for any $g \in G$.
        \item Explain why any left coset of $H$ in $G$ by an element $g$ is the same as the right coset of $H$ in $G$ by $g$.
        \item Find all distinct left cosets of $H$ in $G$.
    \end{partquestions}
\end{mdframed}
\textbf{Solution}:\newline
 \begin{partquestions}{\alph*}
        \item Since $\oplus_8$ is commutative, thus $gH = Hg$.\newline
        (Actually, since $G$ is an additive group, the better thing to write is $g \oplus_8 H = H \oplus_8 g$.)
        \item There are 4 distinct left cosets of $H$ in $G$.
        \begin{itemize}
            \item $0 \oplus_8 H = \{0, 4\} = H$
            \item $1 \oplus_8 H = \{1, 5\}$
            \item $2 \oplus_8 H = \{2, 6\}$
            \item $3 \oplus_8 H = \{3, 7\}$
        \end{itemize}
    \end{partquestions}
\begin{mdframed}
    Let $G$ be a group, $H \leq G$, and $g_1$ and $g_2$ be elements in $G$. Prove that if $g_1H \cap g_2H \neq \emptyset$ then $g_1H = g_2H$.
\end{mdframed}
\textbf{Solution}:\newline
 Let $x$ be in $g_1H \cap g_2H$. Then $x \in g_1H$ and $x \in g_2H$ simultaneously. Hence, $x = g_1h = g_2\hat{h}$ for some $h, \hat{h} \in H$. Thus, by rearrangement, $g_2^{-1}g_1 = \hat{h}h^{-1} \in H$. By Coset Equality (\myref{lemma-coset-equality}), statements 1 and 5, $g_1H = g_2H$.
\begin{mdframed}
    Let $G$ be the group $(\Z_{99}, \oplus_{99})$. It is given that $H = \{0, 33, 66\}$ is a subgroup of $G$. What is the index of $H$ in $G$?
\end{mdframed}
\textbf{Solution}:\newline
 Note that $|G| = 99$ and $|H| = 3$, so $[G:H] = \frac{99}{3} = 33$ by Lagrange's theorem (\myref{thrm-lagrange}).
\begin{mdframed}
    Let $G$ be a finite group with prime order $p$. Prove that any non-identity element in $G$ has order $p$.
\end{mdframed}
\textbf{Solution}:\newline
 Let $x \in G$ with $x \neq e$. Then $|x| > 1$. By \myref{corollary-order-of-group-multiple-of-order-of-element}, the order of $x$ is a factor of $|G| = p$. Since $p$ is prime, $|x| = 1$ (which is not possible) or $|x| = p$. Hence $|x| = p$.
\begin{mdframed}
    Let $G$ be a cyclic group and $H$ a subgroup of $G$.
    \begin{partquestions}{\roman*}
        \item Explain why $G/H$ is a quotient group.
        \item Show that $G/H$ is cyclic.
    \end{partquestions}
\end{mdframed}
\textbf{Solution}:\newline
 \begin{partquestions}{\roman*}
        \item By \myref{prop-subgroup-of-abelian-group-is-normal} every subgroup of $G$ is normal. Hence $H$ is a normal subgroup of $G$, meaning $G/H$ is a quotient group by \myref{thrm-quotient-group-requirement}.
        \item Let $g$ be the generator of $G$. Consider $xH \in G/H$. Since $x \in G$ and $G$ is cyclic, thus there exists an integer $k$ such that $x = g^k$. Hence, $xH = g^kH = (gH)^k$ which means that $gH$ generates any element in $G/H$. Therefore $gH$ is a generator of $G/H$, meaning $G/H$ is cyclic.
    \end{partquestions}

\section*{Problems}
\begin{mdframed}
    Let $G = D_4$, the dihedral group of order 8. By considering the subgroup axioms, determine if the following are subgroups of $G$.
    \begin{partquestions}{\alph*}
        \item $\{e\}$
        \item $\{e, r, s\}$
        \item $\{r, r^2, r^3\}$
        \item $\{r, r^3, r^4, r^6\}$
    \end{partquestions}
\end{mdframed}
\textbf{Solution}:\newline
 We note that $G$ contains $\{e, r, r^2, r^3, s, rs, r^2s, r^3s\}$.
    \begin{partquestions}{\alph*}
        \item Yes, this is the trivial subgroup.
        \item No, it is not closed. ($rs$ can be generated by $r \ast s$ but is not in the set)
        \item No, the identity $e$ is missing.
        \item Yes, $\{r, r^3, r^4, r^6\} = \{r, r^3, e, r^2\} = \langle r \rangle$ which is a cyclic subgroup of $G$.
    \end{partquestions}
\begin{mdframed}
    Let $G$ be a group and $H \leq G$. Define the set
    \[
        K = \{x \in G \vert x^2 \in H\}.
    \]
    Prove the following statements.
    \begin{partquestions}{\alph*}
        \item $K \leq G$
        \item $H \leq K$
    \end{partquestions}
\end{mdframed}
\textbf{Solution}:\newline
 \begin{partquestions}{\alph*}
        \item Clearly $K$ is a subset of $G$. Note $e \in K$ since $e^2 = e \in H$, so $K$ is non-empty.

        Let $x, y \in K$, so $x^2 \in H$ and $y^2 \in H$. We note $y^{-1} \in K$ since $(y^{-1})^2 = (y^2)^{-1} \in H$. Therefore $(xy^{-1})^2 = xy^{-1}xy^{-1} \in H$, so $xy^{-1} \in K$. Hence $K \leq G$ by subgroup test (\myref{thrm-subgroup-test}).

        \item We show that $H \subseteq K$. Note that for any $h \in H$ we have $h^2 \in H$ (since $H$ is a (sub)group and so it is closed). Thus $h \in K$. This shows that $H \subseteq K$. Now because $H \subseteq K$ and $H$ is a (sub)group, thus $H \leq K$ by definition of subgroup.
    \end{partquestions}
\begin{mdframed}
    Let $G$ be a group.
    \begin{partquestions}{\alph*}
        \item Prove that $\CenterGrp{G}$ is a normal subgroup of $G$.
        \item Prove that $\CenterGrp{G} = G$ if and only if $G$ is abelian.
        \item Find the center of the group $D_4$.\newline
        (You may assume $|\CenterGrp{G}| < \frac12 |G|$. The reason becomes apparent with \myref{problem-quotient-of-group-mod-center-is-cyclic-implies-abelian}'s solution.)
    \end{partquestions}
\end{mdframed}
\textbf{Solution}:\newline
 \begin{partquestions}{\alph*}
        \item We have proved that $\CenterGrp{G} \leq G$ so we only prove normality. Let $g$ and $z$ be arbitrary elements from $G$ and $\CenterGrp{G}$ respectively. Then
        \begin{align*}
            gzg^{-1} &= g(zg^{-1})\\
            &= g(g^{-1}z) & (\text{since }z \in \CenterGrp{G})\\
            &= (gg^{-1})z \\
            &= z\\
            &\in \CenterGrp{G}
        \end{align*}
        which proves that $\CenterGrp{G} \unlhd G$.

        \item We first work in the forward direction by assuming $G = \CenterGrp{G}$. Then for all $z \in \CenterGrp{G} = G$ we have $gz = zg$ for any $g \in G$ by definition, which means that $G$ is abelian.

        We now work in the reverse direction by assuming that $G$ is abelian. Note $\CenterGrp{G} = \{z \in G \vert gz = zg \text{ for all } g \in G\}$. But since $G$ is abelian, $gh = hg$ for all $g$ and $h$ in $G$. Thus every element in $G$ satisfies the condition to be in the center of $G$, meaning $\CenterGrp{G} = G$.

        \item We note that $D_4 = \{e, r, r^2, r^3, s, rs, r^2s, r^3s\}$. Since $\CenterGrp{D_4}$ is a subgroup of $D_4$ it has a maximum order of 2, by Lagrange's theorem (\myref{thrm-lagrange}) and the note given. Since 2 is prime the subgroups must be cyclic. Thus the non-trivial proper subgroups of $D_4$ are $\{e, r^2\}$ and $\{e, s\}$ (since $|r^2| = |s| = 2$). Now like how we proved that $\langle s \rangle = \{e, s\}$ is not a normal subgroup in $D_3$ in \myref{example-normal-subgroups-of-d3}, $\{e, s\}$ is not a normal subgroup of $D_4$. One verifies easily that $\{e, r^2\} = \langle r^2 \rangle$ is a normal subgroup of $D_4$. Thus $\CenterGrp{D_4} = \langle r^2 \rangle$ since $\CenterGrp{D_4}$ must be a normal subgroup of $D_4$ with order not exceeding 2.
    \end{partquestions}
\begin{mdframed}
    Let $G$ be a group, and $H, K \leq G$. Prove or disprove the following statements.
    \begin{partquestions}{\alph*}
        \item $H \cap K \leq G$
        \item $H \cap K \leq H$
        \item $H \cup K \leq G$
        \item $H \cup K \leq H$
    \end{partquestions}
\end{mdframed}
\textbf{Solution}:\newline
 \begin{partquestions}{\alph*}
        \item We will prove this statement.

        Clearly since $H \subseteq G$ and $K \subseteq G$ (as both are subgroups), thus $H \cap K \subseteq G$. Also $e \in H \cap K$ since $e \in H$ and $e \in K$ as both are subgroups of $G$. Therefore $H \cap K$ is non-empty.

        Let $x$ and $y$ be in $H \cap K$, meaning that $x, y \in H$ and $x, y \in K$. Thus $xy^{-1} \in H$ and $xy^{-1} \in K$ as both are subgroups of $G$. Hence $xy^{-1} \in H \cap K$.

        Therefore $H \cap K \leq G$ by the subgroup test (\myref{thrm-subgroup-test}).

        \item We will prove this statement. One sees that $H \cap K \subseteq H$. Since $H \cap K \leq G$, it is thus a group. Hence $H \cap K \leq H$ by definition of a subgroup.

        \item We will disprove this statement. Consider
        \begin{align*}
            &G = \Z_6 \text{ under }\oplus_6,\\
            &H = \{0, 2, 4\},\text{ and}\\
            &K = \{0, 3\}.
        \end{align*}
        Clearly $H \leq G$ and $K \leq G$. Note $H \cup K = \{0, 2, 3, 4\}$. But $H \cup K$ is not closed since $2 \oplus_6 3 = 5 \not \in H \cup K$. Hence $H \cup K \not\leq G$.

        \item We will disprove this statement. Since $H \cup K$ is not closed it is not a group, meaning it cannot be a subgroup.
    \end{partquestions}
\begin{mdframed}
    Let $G$ be a group of order 1024 and let $H$ be a proper subgroup of $G$. Determine the maximum order of $H$. Give an example of the groups $G$ and $H$ such that $H$ has this maximum order.
\end{mdframed}
\textbf{Solution}:\newline
 By Lagrange's Theorem (\myref{thrm-lagrange}), the order of a subgroup must divide the order of the group. Since $H \leq G$ is proper, and since $1024 = 2^{10}$, the largest order that $H$ can be is $512$ with $[G:H] = 2$. An example is $G = \Z_{1024}$ and $H = \langle 2 \rangle$, since $|H| = |2| = 512$ as $2 \times 512 = 1024 \equiv 0 \pmod{1024}$.
\begin{mdframed}
    Let $G$ be a finite group with even order. Show that there exists an element with order 2 in $G$.
\end{mdframed}
\textbf{Solution}:\newline
 Let $|G| = 2n$. The identity is its own inverse, leaving an odd number of non-identity elements.

    Suppose $x$ is an element of $G$ with $|x| > 2$; we cannot have $x^{-1} = x$ (otherwise $x^2 = e$). Thus $x^{-1}$ and $x$ are distinct. Pair every one of these $x$'s with its inverse $x^{-1}$.

    Remember that there is an odd number of non-identity elements. Hence, there must be at least one element which has not been paired off with any of the others, which is therefore its own self inverse.

    Since this element is not the identity, thus it has to have order 2 (as $g^{-1} = g$ implies $g^2 = e$).
\begin{mdframed}
    Let $G$ be a cyclic group with generator $g$. Prove that any subgroup of $G$ must also be cyclic.\newline
    (\textit{Hint: Consider the solution for \myref{example-subgroups-of-Z}.})
\end{mdframed}
\textbf{Solution}:\newline
 Suppose $G = \langle g \rangle$ and $H \leq G$. Then any element in $H$ is of the form $g^a$ where $a$ is an integer. Suppose $m$ is the smallest positive integer $m$ such that $g^m \in H$. Suppose now $g^n \in H$ for some $n$. By Euclid's division lemma (\myref{lemma-euclid-division}), $n = mq + r$ where $q$ and $r$ are non-negative integers such that $0 \leq r < m$. Hence,
    \[
        g^n = g^{mq}g^r = (g^m)^q g^r.
    \]
    Now, $m$ is the smallest positive integer such that $g^m \in H$. This means that if $r \neq 0$, $g^r \not\in H$ as $0 \leq r < m$. Hence, $r = 0$, which means
    \[
        g^n = (g^m)^q.
    \]
    Thus, every element in the subgroup $H$ can be formed by applying $g^m$ a certain number of times, meaning $H$ is cyclic with generator $g^m$.
\begin{mdframed}
    Let $G$ be a finite group. Suppose $H < G$ such that the index of $H$ in $G$ is 2. Prove that
    \begin{partquestions}{\roman*}
        \item $H \lhd G$;
        \item $H$ contains the squares of all elements of $G$; and
        \item an element $x \in G$ is in $H$ if $x$ has odd order.
    \end{partquestions}
\end{mdframed}
\textbf{Solution}:\newline
 \begin{partquestions}{\roman*}
        \item Let $xH$ be a coset in $G$. Since cosets partition $G$, either $xH = H$ or $xH = G \setminus H$ (since there are only two distinct cosets).
        \begin{itemize}
                \item If $xH = H$, then $xH = Hx$ by Element in Coset (\myref{corollary-equivalence-of-element-in-coset}).
                \item If $xH = G \setminus H$, then $xH \neq H$, meaning $Hx \neq H$ by contrapositive of Element in Coset. But if $Hx \neq H$, then $Hx = G \setminus H$ (since there are only two distinct cosets). So $xH = G \setminus H = Hx$.
        \end{itemize}
        Therefore $H$ is a normal subgroup of $G$, i.e. $H \lhd G$.

        \item Note that $G/H$ is a group since $H \lhd G$. Also since $[G:H] = 2$ thus $|G/H| = 2$.

        If $x \in G$ then $x^2H = (xH)^2 = H$ since the order of $G/H$ is 2, meaning that any non-identity element inside it (like $xH$) has an order of at most 2. Since $x^2H = H$, therefore $x^2 \in H$ by Element in Coset (\myref{corollary-equivalence-of-element-in-coset}).

        \item Suppose $x$ has odd order. Write $|x| = 2k - 1$ where $k$ is a positive integer. Hence $x^{2k-1} = e \in H$ since $H < G$. Therefore
        \[
            x = x^{2k} = \left(x^k\right)^2 \in H
        \]
        which means $x \in H$.
    \end{partquestions}
\begin{mdframed}
    Let $G$ be a finite group, $H \leq G$, and $K \leq G$. Suppose the order of $H$ and the order of $K$ are coprime.
    \begin{partquestions}{\alph*}
        \item Show that the intersection of the groups $H$ and $K$ contains only the identity.
        \item Show that, if $H$ and $K$ are normal subgroups of $G$, then for any $h \in H$ and $k \in K$ we have $hk = kh$.
    \end{partquestions}
\end{mdframed}
\textbf{Solution}:\newline
 \begin{partquestions}{\alph*}
        \item Suppose we have an element $x \in H \cap K$, meaning that $x \in H$ and $x \in K$. By a corollary of Lagrange's Theorem (\myref{corollary-order-of-group-multiple-of-order-of-element}), the order of $x$ must divide the order of its group. Hence, $|x|$ divides $|H|$ and $|x|$ divides $|K|$ simultaneously, meaning that $|x| = \gcd(|H|, |K|)$. But the GCD of the orders of both subgroups is 1. Hence, $|x| = 1$, meaning the only element in the intersection $H \cap K$ is the identity $e$.

        \item Consider $hkh^{-1}k^{-1}$.
        \begin{itemize}
            \item On one hand, note that $hkh^{-1}k^{-1} = h(kh^{-1}k^{-1})$. Clearly $h \in H$ and $kh^{-1}k^{-1} \in H$ by normality of $H$. Therefore $hkh^{-1}k^{-1} \in H$.
            \item On another hand, $hkh^{-1}k^{-1} = (hkh^{-1})k^{-1}$. Note $hkh^{-1} \in K$ by normality of $K$ and $k^{-1} \in K$, so $hkh^{-1}k^{-1} \in K$.
        \end{itemize}
        Therefore $hkh^{-1}k^{-1} \in H \cap K$. But by \textbf{(a)}, the only element in $H \cap K$ is the identity. Thus, $hkh^{-1}k^{-1} = e$ which the result follows quickly.
    \end{partquestions}
\begin{mdframed}
    Let $G$ be a finite group with order $m$.
    \begin{partquestions}{\alph*}
        \item State the smallest value of $m$ such that $G$ is non-abelian.
        \item Prove that the value of $m$ found in \textbf{(a)} is the smallest value that allows $G$ to be non-abelian.
        \item Hence prove that for all even integers $n \geq m$, there exists a non-abelian group of order $n$.
    \end{partquestions}
\end{mdframed}
\textbf{Solution}:\newline
 \begin{partquestions}{\alph*}
        \item $m = 6$.
        \item We first prove that all groups of order less than 6 are abelian, and then find a non-abelian group of order 6.

        We note that a group of order 1 is the trivial group which is abelian. The groups of order 2, 3, and 5 are groups of prime order, meaning that they are cyclic and hence abelian. We are left with a group of order 4.

        We note that the order of an element of a group of order 4 must divide 4 (\myref{corollary-order-of-group-multiple-of-order-of-element}). Hence the possible orders of an element in such a group is 1, 2, or 4. An element of order 1 is the identity. If an element with order 4 exists, then the group is cyclic and hence abelian. So we assume that all elements are either order 1 or order 2 (in fact, the orders must be 1, 2, 2, 2). This is precisely the group
        \[
            D_2 = \langle r, s \vert r^2 = s^2 = e, rs = sr\rangle
        \]
        which is abelian. Hence all groups of order 4 are abelian.

        We now show that a group of order 6 can be non-abelian. We note that the group
        \[
            D_3 =  \langle r, s \vert r^3 = s^2 = e, rs = sr^2\rangle
        \]
        has order 6 and because $rs = sr^2 \neq sr$, thus $D_3$ is non-abelian. Hence $m = 6$.

        \item For all even $n \geq 6$, the group $D_{\frac n2}$ has $n$ elements and $rs = sr^{\frac n2 - 1} \neq sr$, so $D_{\frac n2}$ is non-abelian.
    \end{partquestions}
\begin{mdframed}
    Let $G$ be a group, and suppose $G / \CenterGrp{G}$ is cyclic. Prove that $G$ is abelian.
\end{mdframed}
\textbf{Solution}:\newline
 Suppose $G / \CenterGrp{G}$ is cyclic. Then by definition, $G / \CenterGrp{G} = \langle g\CenterGrp{G}\rangle$ for some $g \in G$, and any element in $G/\CenterGrp{G}$ is of the form $g^n\CenterGrp{G}$.

    Now take $x, y \in G$. By \myref{lemma-left-coset-partition}, left cosets partition the group, so we may assume $x \in g^m\CenterGrp{G}$ and $y \in g^n\CenterGrp{G}$, meaning $x = g^mz_1$ and $y = g^nz_2$ for some $z_1, z_2 \in \CenterGrp{G}$. We note
    \begin{align*}
        xy &= (g^mz_1)(g^nz_2)\\
        &= g^m(z_1g^n)z_2\\
        &= g^m(g^nz_1)z_2 & (\text{since }z_1 \in \CenterGrp{G})\\
        &= (g^mg^n)(z_1z_2)\\
        &= g^{m+n}z_1z_2\\
        &= g^{n+m}z_2z_1\\
        &= g^ng^mz_2z_1\\
        &= g^n(g^mz_2)z_1\\
        &= g^n(z_2g^m)z_1\\
        &= (g^nz_2)(g^mz_1)\\
        &= yx
    \end{align*}
    which means that $xy = yx$ for any $x, y \in G$. Hence $G$ is abelian.

\chapter{Homomorphisms And Isomorphisms}
\section*{Exercises}
\begin{mdframed}
    Let $G = (\mathbb{N}, +)$ and $H = (\mathbb{N}, \times)$. Let $\phi: G \to H$. Determine if the following maps are homomorphisms.
    \begin{partquestions}{\alph*}
        \item $\phi(n) = n$
        \item $\phi(n) = 2^n$
    \end{partquestions}
\end{mdframed}
\textbf{Solution}:\newline
 \begin{partquestions}{\alph*}
        \item No, since $\phi(m+n) = m + n$ while $\phi(m)\phi(n) = mn \neq m+n$.
        \item Yes, since $\phi(m+n) = 2^{m+n} = 2^m2^n = \phi(m)\phi(n)$.
    \end{partquestions}
\begin{mdframed}
    Let $G$ and $H$ be groups with identities $e_G$ and $e_H$ respectively. Show that the \textbf{trivial homomorphism}\index{homomorphism!trivial} $\phi: G \to H, g \mapsto e_H$ is, indeed, a homomorphism.
\end{mdframed}
\textbf{Solution}:\newline
 One sees that
    \[
        \phi(xy) = e_H = e_He_H = \phi(x)\phi(y)
    \]
    so $\phi$ is indeed a homomorphism.
\begin{mdframed}
    Prove that $\phi(H_1) \leq G_2$.
\end{mdframed}
\textbf{Solution}:\newline
 The codomain of $\phi$ is $G_2$, so $\phi(H_1) \subseteq G_2$. Clearly $e_2 \in \phi(H_1)$ since $e_2 = \phi(e_1)$ and $e_1 \in H_1$. Now suppose $x$ and $y$ are in $\phi(H_1)$, meaning that $\phi(h_x) = x$ and $\phi(h_y) = y$ for some $h_x$ and $h_y$ in $H$. So $h_xh_y^{-1}$ is in $H$. Furthermore,
    \begin{align*}
        \phi(h_xh_y^{-1}) &= \phi(h_x)\phi(h_y^{-1})\\
        &= \phi(h_x)\left(\phi(h_y)\right)^{-1}\\
        &= xy^{-1},
    \end{align*}
    meaning that $xy^{-1}$ is in $\phi(H_1)$. Therefore $\phi(H_1) \leq G_2$ by subgroup test (\myref{thrm-subgroup-test}).
\begin{mdframed}
    Prove or disprove: if $H_1 \unlhd G_1$, then $\phi(H_1) \unlhd G_2$.
\end{mdframed}
\textbf{Solution}:\newline
 Disprove. Let $G_1 = H_1 = \Z$ be the additive group of integers and let $G_2 = H_2 = D_n$, the dihedral group of order $2n$. Consider the map $\phi: G_1 \to G_2$ where $\phi(m) = s^m$. Clearly, $H_1 \unlhd G_1$. Note that $\phi(H_1) = \{e, s\} = \langle s \rangle$. From \myref{example-normal-subgroups-of-d3}, we know that $\langle s \rangle$ is not a normal subgroup of $D_3 = G_2$, so $\phi(H_1)$ is not a normal subgroup of $G_2$.
\begin{mdframed}
    Let $G$ and $H$ be groups, and let $\phi: G \to H$ be a homomorphism. Prove that $|\phi(a)|$ divides $|a|$ for any $a \in G$.
\end{mdframed}
\textbf{Solution}:\newline
 Suppose $|a| = n$. Note that
    \[
        \left(\phi(a)\right)^n = \phi\left(a^n\right) = \phi(e_G) = e_H
    \]
    so $|\phi(a)|$ divides $n = |a|$ by \myref{lemma-order-of-an-element-that-is-equivalent-to-identity}.
\begin{mdframed}
    The \textbf{identity function}\index{identity function} (or \textbf{identity map}\index{identity map}) $\id: S \to S$, where $S$ is a set, is the function such that $\id(x) = x$ for all $x$ in $S$.
    \begin{partquestions}{\roman*}
        \item Show that $\id$ is a bijection.
        \item If $S$ is a group, show that $\id$ is an isomorphism.
    \end{partquestions}
\end{mdframed}
\textbf{Solution}:\newline
 \begin{partquestions}{\roman*}
        \item We show that $\id$ is both injective and surjective.
        \begin{itemize}
            \item \textbf{Injective}: Suppose $x, y \in S$ such that $\id(x) = \id(y)$. Then clearly $x = y$ by definition of the identity map.
            \item \textbf{Surjective}: Suppose $y \in S$. Clearly $\id(y) = y$, so $y$ is its own pre-image.
        \end{itemize}
        Thus $\id$ is a bijection.

        \item Clearly
        \[
            \id(xy) = xy = \id(x)\id(y)
        \]
        so $\id$ is a homomorphism. Coupled with \textbf{(i)}, this means that $\id$ is an isomorphism.
    \end{partquestions}
\begin{mdframed}
    Let the groups $G = (\{1, 2, 3, 4\}, \otimes_5)$ and $H = (\{1, 3, 7, 9\}, \otimes_{10})$.
    \begin{partquestions}{\roman*}
        \item Show that $G = \langle 3 \rangle$ and $H = \langle 7 \rangle$.
        \item Prove that $G \cong H$ by considering $\phi: G \to H, 3^k \mapsto 7^k$.
    \end{partquestions}
\end{mdframed}
\textbf{Solution}:\newline
 \begin{partquestions}{\roman*}
        \item Since $3^0 = 1$, $3^1 = 3$, $3^2 = 9 \equiv 4 \pmod{5}$, and $3^3 = 27 \equiv 2 \pmod{5}$, thus $G = \langle 3 \rangle$. Since $7^0 = 1$, $7^1 = 7$, $7^2 = 49 \equiv 9 \pmod{10}$, and $7^3 = 343 \equiv 3 \pmod{10}$, thus $H = \langle 7 \rangle$.
        \item We need to prove that it is a homomorphic bijection.
        \begin{itemize}
            \item \textbf{Homomorphism}:
            \begin{align*}
                \phi(3^m3^n) &= \phi(3^{m+n})\\
                &= 7^{m+n}\\
                &= 7^m7^n\\
                &= \phi(3^m)\phi(3^n)
            \end{align*}

            \item \textbf{Bijection}: Note that $1 \mapsto 1$, $3 \mapsto 7$, $4 \mapsto 9$, $2 \mapsto 3$ which clearly shows that $\phi$ is bijective.
        \end{itemize}
        Therefore $\phi$ is an isomorphism, meaning $G \cong H$.
    \end{partquestions}
\begin{mdframed}
    Let $A$, $B$, and $C$ be sets. Let $f: A \to B$ and $g: B \to C$ be bijections. Let $h = g\circ f$, where $\circ$ represents function composition. That is, $h: A \to C, x \mapsto g(f(x))$.
    \begin{partquestions}{\roman*}
        \item Show that $h$ is a bijection.
        \item If $A$, $B$, and $C$ are groups, and $f$ and $g$ are isomorphisms, show that $h$ is an isomorphism.
    \end{partquestions}
\end{mdframed}
\textbf{Solution}:\newline
 \begin{partquestions}{\roman*}
        \item We show that $h = g\circ f$ is both injective and surjective.
        \begin{itemize}
            \item \textbf{Injective}: Let $x, y \in A$. Then
            \begin{align*}
                &h(x) = h(y)\\
                \iff&g(f(x)) = g(f(y))\\
                \iff&f(x) = f(y) & (g \text{ is bijective})\\
                \iff&x = y & (f \text{ is injective})
            \end{align*}
            so $h$ is injective.
            \item \textbf{Surjective}: Suppose $c \in C$. Note that $f^{-1}: B \to A$ and $g^{-1}: C \to B$ both exist since both $f$ and $g$ are injective. Let $a = f^{-1}(g^{-1}(c)) \in A$. Then note
            \[
                h(a) = g(f(f^{-1}(g^{-1}(c)))) = c
            \]
            so $c$ has a pre-image of $a$, meaning $h$ is surjective.
        \end{itemize}
        Therefore $h$ is a bijection.

        \item Recall that an isomorphism is also a homomorphism.  Note that
        \begin{align*}
            h(xy) &= g(f(xy))\\
            &= g(f(x)f(y)) & (f \text( is a homomorphism))\\
            &= g(f(x))g(f(y)) & (g \text{ is a homomorphism})\\
            &= h(x)h(y)
        \end{align*}
        so $h$ is a homomorphism. As $h$ is also a bijection (from \textbf{(i)}), thus $h$ is an isomorphism.
    \end{partquestions}
\begin{mdframed}
    Let $\phi: G \to H$ be an isomorphism between the groups $G$ and $H$. Show that if $G$ has a normal subgroup with order $k$, then $H$ also has a normal subgroup of order $k$.
\end{mdframed}
\textbf{Solution}:\newline
 Suppose $N \unlhd G$ such that $|N| = k$. Then $\phi(N)$ is a subgroup of $H$ with order $k$ by \myref{thrm-isomorphism-consequences}, statement 5. All that remains to prove is that $\phi(N)$ is normal.

    Let $n \in N$ and $\hat{n} \in \phi(N)$ such that $\hat{n} = \phi(n)$. Let $h \in H$ be an arbitrary element. To prove that $h\hat{n}h^{-1}$ is in $\phi(N)$.

    Let $g$ be in $G$ such that $\phi(g) = h$. Then
    \begin{align*}
        h\hat{n}h^{-1} &= \phi(g)\phi(n)\phi(g^{-1})\\
        &= \phi(\underbrace{gng^{-1}}_{\text{In } N})\\
        &\in \phi(N)
    \end{align*}
    which proves that $\phi(N)$ is normal. Hence there exists a normal subgroup of order $k$, namely $\phi(N)$.
\begin{mdframed}
    Prove that the ``isomorphism'' relation $\cong$ is an equivalence relation. In particular, for the groups $G$, $H$, and $K$, prove that
    \begin{partquestions}{\alph*}
        \item $G \cong G$;
        \item if $G \cong H$ then $H \cong G$; and
        \item if $G \cong H$ and $H \cong K$ then $G \cong K$.
    \end{partquestions}
\end{mdframed}
\textbf{Solution}:\newline
 \begin{partquestions}{\alph*}
        \item We know that $\id: G \to G, g \mapsto g$ is an isomorphism (\myref{exercise-identity-map-is-isomorphism}), so $G \cong G$.
        \item Suppose $G \cong H$. This means that there is an isomorphism $\phi: G \to H$. \myref{thrm-isomorphism-consequences}, statement 2 tells us that $\phi^{-1}: H \to G$ is also an isomorphism. Therefore $H \cong G$.
        \item Suppose $G \cong H$ and $H \cong K$. Then there exist isomorphisms $\phi: G \to H$ and $\psi: H \to K$. We know that $f: G \to K, g \mapsto \psi(\phi(g))$ is an isomorphism by \myref{exercise-composition-of-isomorphisms-is-isomorphisms}, which means $G \cong K$.
    \end{partquestions}
\begin{mdframed}
    Let $G = (\{1, 3, 7, 9\}, \otimes_{10})$ be a group. Find the positive integer $n$ such that $G \cong \Z_n$.
\end{mdframed}
\textbf{Solution}:\newline
 Since $7^0 = 1$, $7^1 = 7$, $7^2 = 49 \equiv 9 \pmod{10}$, and $7^3 = 343 \equiv 3 \pmod{10}$, thus $G = \langle 7 \rangle$. Note $|7| = 4$ so $G \cong \Z_4$, i.e. $n = 4$.

\section*{Problems}
\begin{mdframed}
    Let $G$ be a group and $g \in G$. Define the map $f: G \to G, x \mapsto gxg^{-1}$. Prove that $f$ is an isomorphism.
\end{mdframed}
\textbf{Solution}:\newline
 We will prove that $f$ is a homomorphism, is injective, and is surjective.
    \begin{itemize}
        \item \textbf{Homomorphism}: Let $x, y \in G$. Then
        \begin{align*}
            f(xy) &= g(xy)g^{-1}\\
            &= (gxg^{-1})(gyg^{-1})\\
            &= f(x)f(y)
        \end{align*}
        which means that $f$ is a homomorphism.
        \item \textbf{Injective}: Let $x, y \in G$ be such that $f(x) = f(y)$. Then $gxg^{-1} = gyg^{-1}$. By cancellation law, $x = y$.
        \item \textbf{Surjective}: Suppose $y \in G$. Set $x = g^{-1}yg$. Since $G$ is closed, thus $x \in G$. Note $f(x) = g(g^{-1}yg)g^{-1} = y$. Hence $y$ has a pre-image of $x = g^{-1}yg$ in $G$.
    \end{itemize}
    Therefore $f$ is an isomorphism.
\begin{mdframed}
    Let $\Q_{>0}$ denote the set of positive rational numbers. Let the groups $G = (\Q, +)$ and $H = (\Q_{>0}, \times)$. Prove that $G \not\cong H$.
\end{mdframed}
\textbf{Solution}:\newline
 Suppose on the contrary there exists an isomorphism $\phi: G \to H$. Since $\phi$ is an isomorphism, it is surjective. Hence, there must exists a rational number $r \in G$ such that $\phi(r) = 2$. As $r$ is rational, so is $\frac r2$.

    Now consider $\phi\left(\frac r2 + \frac r2\right)$. On one hand, $\phi\left(\frac r2 + \frac r2\right) = \phi(r) = 2$. On another hand, $\phi(\frac r2 + \frac r2) = \left(\phi\left(\frac r2\right)\right)^2$ as $\phi$ is a homomorphism. Therefore, $\left(\phi\left(\frac r2\right)\right)^2 = 2$ which quickly implies $\phi\left(\frac r2\right) = \sqrt 2$ since $\phi\left(\frac r2\right)$ must be positive. However, $\sqrt 2 \notin H$ while $\phi\left(\frac r2\right) \in H$, a contradiction.

    Hence, $G \not\cong H$.
\begin{mdframed}
    Define a map $\phi: \Z \to \Z$ such that $\phi(n) = 2n$.
    \begin{partquestions}{\alph*}
        \item Prove that $\phi$ is a homomorphism.
        \item Prove that $\phi$ is injective.
        \item Prove that there does \textit{not} exist a homomorphism $\psi: \Z \to \Z$ where $\psi(\phi(n)) = n$.
    \end{partquestions}
\end{mdframed}
\textbf{Solution}:\newline
 \begin{partquestions}{\alph*}
        \item Let $m, n \in \Z$. Then
        \[
            \phi(m + n) = 2(m + n) = 2m + 2n = \phi(m) + \phi(n)
        \]
        which means $\phi$ is a homomorphism.

        \item Suppose $m, n \in G$ such that $\phi(m) = \phi(n)$. Then $2m = 2n$. Clearly this means that $m = n$. Thus $\phi$ is injective.

        \item Suppose on the contrary there existed a homomorphism $\psi: \Z \to \Z$ such that $\psi(\phi(n)) = n$. Then $\psi(2n) = n$ by definition of $\phi$. Note that
        \[
            \psi(2n) = \psi(n + n) = \psi(n) + \psi(n) = 2\psi(n)
        \]
        since $\psi$ is a homomorphism. Hence $2\psi(n) = n$ which implies that $\psi(n) = \frac n2$. But for the case of $n = 1$, $\psi(1) = \frac 12 \notin \Z$. Hence $\psi$ does not exist.
    \end{partquestions}
\begin{mdframed}
    Let $G$ be a group. Define a map $f: G \to G$ such that $f(g) = g^{-1}$ for all $g$ in $G$. Prove that $G$ is abelian if and only if $f$ is a homomorphism.
\end{mdframed}
\textbf{Solution}:\newline
 We prove the forward direction first: assume that $G$ is abelian. Then $f$ is a homomorphism since
    \[
        f(gh) = (gh)^{-1} = h^{-1}g^{-1} = g^{-1}h^{-1} = f(g)h(g).
    \]

    We now prove the reverse direction: assume that $f$ is a homomorphism, meaning $f(gh) = f(g)f(h) = g^{-1}h^{-1}$. But $f(gh) = (gh)^{-1} = h^{-1}g^{-1}$. Therefore we have $g^{-1}h^{-1} = h^{-1}g^{-1}$ which clearly shows that the group is abelian.
\begin{mdframed}
    Let $G$ and $H$ be groups. Suppose that we have a surjective homomorphism $\phi: G \to H$. Prove that if $G$ is abelian, then so is $H$.
\end{mdframed}
\textbf{Solution}:\newline
 Suppose $\phi: G \to H$ is a surjective homomorphism and $G$ is abelian. Since $\phi$ is surjective, thus $\im \phi = H$. Let $g_1, g_2 \in G$ and $h_1, h_2 \in H$ such that $\phi(g_1) = h_1$ and $\phi(g_2) = h_2$. Consider $\phi(g_1g_2)$.
    \begin{itemize}
        \item On one hand, $\phi(g_1g_2) = \phi(g_1)\phi(g_2) = h_1h_2$.
        \item On another hand, $\phi(g_1g_2) = \phi(g_2g_1) = \phi(g_2)\phi(g_1) = h_2h_1$.
    \end{itemize}
    Hence $h_1h_2 = h_2h_1$ which means that $H$ is abelian.
\begin{mdframed}
    Let $G$ and $H$ be groups. Suppose that we have a surjective homomorphism $\phi: G \to H$. Let $N \unlhd G$. Show that $\phi(N) \unlhd H$. That is, show that the image of $N$ under $\phi$ is a normal subgroup of $H$.
\end{mdframed}
\textbf{Solution}:\newline
 We first prove $\phi(N)$ is a subgroup of $H$ before proving normality.

    The codomain of $\phi$ is $H$, so $\phi(N) \subseteq H$. Note that $e_H \in \phi(N)$ since $e_G \in N$ and $\phi(e_G) = e_H$. Now let $x, y \in \phi(N)$. As $\phi$ is surjective, we know that there exists $n_x, n_y \in N$ where $\phi(n_x) = x$ and $\phi(n_y) = y$. Note that $\phi(n_y^{-1}) = y^{-1}$ and $n_xn_y^{-1} \in N$. Hence, $xy^{-1} = \phi(n_xn_y^{-1}) \in \phi(N)$. By subgroup test (\myref{thrm-subgroup-test}), $\phi(N) \leq H$.

    We now show that $\phi(N)$ is a normal subgroup of $H$. Take $g \in G$, $h \in H$, $n \in N$, and $x \in \phi(N)$, such that $\phi(g) = h$ and $\phi(n) = x$. Note that since $N \unlhd G$, thus $gng^{-1} \in N$. Therefore,
    \begin{align*}
        hxh^{-1} &= \phi(g)\phi(n)\phi(g^{-1})\\
        &= \phi(\underbrace{gng^{-1}}_{\text{In }N})\\
        &\in \phi(N)
    \end{align*}
    which means that $\phi(N) \unlhd H$.
\begin{mdframed}
    Let the groups $G = \Z_n$ and $H = \Z/(n\Z)$. Prove that $G \cong H$.
\end{mdframed}
\textbf{Solution}:\newline
 Consider the map $\phi: G \to H, a \mapsto a + n\Z$. We show that $\phi$ is an isomorphism:
    \begin{itemize}
        \item \textbf{Homomorphism}: Let $a$ and $b$ be in $G$. Then
        \begin{align*}
            \phi(a\oplus_n b) &= (a\oplus_n b) + n\Z\\
            &= \{(a \oplus_n b) + pn \vert p \in \Z\}\\
            &= \{a+b + pn \vert p \in \Z\}\\
            &= \{a+b + pn + qn\vert p, q \in \Z\}\\
            &= a+b+n\Z + n\Z\\
            &= (a+n\Z) + (b + n\Z)\\
            &= \phi(a) + \phi(b).
        \end{align*}
        \item \textbf{Injective}: Let $a$ and $b$ be in $G$ such that $\phi(a) = \phi(b)$. Thus
        \[
            \{a + pn \vert p \in \Z \} = \ \{b + qn \vert q \in \Z \}
        \]
        by definition of $\phi$. Hence $a \equiv b \pmod n$. But since $0 \leq a, b < n$, we must have $a = b$.
        \item \textbf{Surjective}: Let $x + n\Z \in H$. We use Euclid's division lemma (\myref{lemma-euclid-division}) on $x$ to yield
        \[
            x = qn + r, \text{ where } 0 \leq r < n.
        \]
        Note that
        \begin{align*}
            x + n\Z &= \{x + kn \vert k \in \Z\}\\
            &= \{(qn + r) + kn \vert k \in \Z\}\\
            &= \{r + n(\underbrace{q + k}_{\text{In }\Z}) \vertalt k \in \Z \}\\
            &= r + n\Z
        \end{align*}
        with $0 \leq r < n$, meaning $r \in G$. Now observe $\phi(r) = r+n\Z = x+n\Z$ which means that there is a pre-image for every element in $H$, hence proving that $\phi$ is surjective.
    \end{itemize}
    Therefore $\phi$ is an isomorphism, proving $G \cong H$.
\begin{mdframed}
    Let $G$ be a group and $N \unlhd G$. Let $B$ be a subgroup of the quotient group $G/N$. Prove that $B = A/N$, where $A$ is a subgroup of $G$ such that $N \subseteq A$.
\end{mdframed}
\textbf{Solution}:\newline
 Consider the map $\phi: G \to G/N$ such that $g \mapsto gN$. We note that $\phi$ is a homomorphism as
    \[
        \phi(gh) = (gh)N = (gN)(hN) = \phi(g)\phi(H).
    \]
    We note by \myref{prop-homomorphism-inverse-is-subgroup} that $A = \phi^{-1}(B) \leq G$. Thus
    \begin{align*}
        \phi^{-1}(N) &= \{g \in G \vert \phi(g) = N\}\\
        &= \{g \in G \vert gN = N\}\\
        &= \{g \in G \vert g \in N\}\\
        &= G \cap N\\
        &= N\\
        &\subseteq A
    \end{align*}
    by assumption. Since $N$ is a group, we know $N \leq A$. Furthermore $N \leq A \leq G$ and $N \unlhd G$, meaning $N \unlhd A$ (since $gN = Ng$ for all $g \in G$, including those in $A$). Hence $A/N$ is a group.

    Now clearly $\phi$ is surjective (since for any $gN \in G/N$ we know $\phi(g) = gN$), which means that $\phi(\phi^{-1}(B)) = B$. Since $\phi^{-1}(B) = A$, so $\phi(A) = B$. Finally,
    \begin{align*}
        \phi(A) &= \{\phi(a) \vert a \in A\}\\
        &= \{aN \vert a \in A\}\\
        &= A/N
    \end{align*}
    which means $B = A/N$.

\chapter{Symmetry Groups}
\section*{Exercises}
\begin{mdframed}
    Find the cycle decomposition of the following permutations.
    \begin{partquestions}{\alph*}
        \item $1 \mapsto 2$, $2 \mapsto 3$, $3 \mapsto 1$
        \item $1 \mapsto 3$, $2 \mapsto 2$, $3 \mapsto 1$
        \item $1 \mapsto 3$, $2 \mapsto 4$, $3 \mapsto 1$, $4 \mapsto 5$, $5 \mapsto 2$
    \end{partquestions}
\end{mdframed}
\textbf{Solution}:\newline
 \begin{partquestions}{\alph*}
        \item $\begin{pmatrix}1 & 2 & 3\end{pmatrix}$
        \item $\begin{pmatrix}1 & 3\end{pmatrix}$
        \item $\begin{pmatrix}1 & 3\end{pmatrix}\begin{pmatrix}2 & 4 & 5\end{pmatrix}$
    \end{partquestions}
\begin{mdframed}
    Find the inverse of the permutation $\pi$, which has cycle decomposition
    \[
        \begin{pmatrix}1 & 5 & 2\end{pmatrix}\begin{pmatrix}2 & 5 & 3 & 4\end{pmatrix}.
    \]
\end{mdframed}
\textbf{Solution}:\newline
 This exercise can be solved in two ways.
    \begin{enumerate}
        \item Notice that
        \[
            \pi = \begin{pmatrix}1 & 5 & 2\end{pmatrix}\begin{pmatrix}2 & 5 & 3 & 4\end{pmatrix} = \begin{pmatrix}1 & 5 & 3 & 4 \end{pmatrix}
        \]
        and so $\pi^{-1} = \begin{pmatrix}4 & 3 & 5 & 1\end{pmatrix} = \begin{pmatrix}1 & 4 & 3 & 5\end{pmatrix}$.
        \item Using Shoes and Socks,
        \begin{align*}
            \pi^{-1} &= \left(\begin{pmatrix}1 & 5 & 2\end{pmatrix}\begin{pmatrix}2 & 5 & 3 & 4\end{pmatrix}\right)^{-1}\\
            &= \begin{pmatrix}2 & 5 & 3 & 4\end{pmatrix}^{-1} \begin{pmatrix}1 & 5 & 2\end{pmatrix}^{-1}.
        \end{align*}
        Now, $\begin{pmatrix}2 & 5 & 3 & 4\end{pmatrix}^{-1} = \begin{pmatrix}4 & 3 & 5 & 2\end{pmatrix} = \begin{pmatrix}2 & 4 & 3 & 5\end{pmatrix}$ and $\begin{pmatrix}1 & 5 & 2\end{pmatrix}^{-1} = \begin{pmatrix}2 & 5 & 1\end{pmatrix} = \begin{pmatrix}1 & 2 & 5\end{pmatrix}$. Therefore
        \[
            \pi^{-1} = \begin{pmatrix}2 & 4 & 3 & 5\end{pmatrix}\begin{pmatrix}1 & 2 & 5\end{pmatrix} = \begin{pmatrix}1 & 4 & 3 & 5\end{pmatrix}.
        \]
    \end{enumerate}
\begin{mdframed}
    Explain why $|\Sn{n}| = n!$.
\end{mdframed}
\textbf{Solution}:\newline
 To see why, consider the fact that elements of $\Sn{n}$ are permutations. Each element is only able to permute the elements of the set $X = \{1, 2, 3, \dots, n\}$. For a set of $n$ elements, there are $n!$ permutations. Thus, $|\Sn{n}| = n!$ since $\Sn{n}$ is the set of all permutations of $n$ letters.

\section*{Problems}
\begin{mdframed}
    Let the permutations
    \begin{align*}
        &\alpha = \begin{pmatrix}1 & 5 & 2 & 3\end{pmatrix},\\
        &\beta  = \begin{pmatrix}1 & 5 & 2\end{pmatrix}\begin{pmatrix}3 & 4\end{pmatrix},\\
        &\gamma = \begin{pmatrix}1 & 2 & 5\end{pmatrix}\begin{pmatrix}3 & 4\end{pmatrix}, \text{ and}\\
        &\delta = \begin{pmatrix}1 & 3 & 2 & 5\end{pmatrix}.
    \end{align*}
    What is the cycle decomposition of $\alpha\beta\gamma\delta$?
\end{mdframed}
\textbf{Solution}:\newline
 We work from the right to the left.
    \begin{itemize}
        \item $\gamma \delta$ has cycle notation
        \begin{align*}
            &\begin{pmatrix}1 & 2 & 5\end{pmatrix}\begin{pmatrix}3 & 4\end{pmatrix}\begin{pmatrix}1 & 3 & 2 & 5\end{pmatrix}\\
            &= \begin{pmatrix}1 & 4 & 3 & 5 & 2\end{pmatrix};
        \end{align*}
        \item $\beta \gamma \delta$ has cycle notation
        \begin{align*}
            &\begin{pmatrix}1 & 5 & 2\end{pmatrix}\begin{pmatrix}3 & 4\end{pmatrix}\begin{pmatrix}1 & 4 & 3 & 5 & 2\end{pmatrix}\\
            &= \begin{pmatrix}1 & 3 & 2 & 5\end{pmatrix}\\
            &= \delta;
        \end{align*}
        and
        \item $\alpha \beta \gamma \delta$ has cycle notation
        \begin{align*}
            &\begin{pmatrix}1 & 5 & 2 & 3\end{pmatrix}\begin{pmatrix}1 & 3 & 2 & 5\end{pmatrix}\\
            &= \id,
        \end{align*}
        the identity.
    \end{itemize}
\begin{mdframed}
    Prove that the symmetric group of degree 3, $\Sn{3}$, is isomorphic to the dihedral group of order 6, $D_3$.
\end{mdframed}
\textbf{Solution}:\newline
 Recall that $D_3$ has presentation
    \[
        \langle r, s \vert r^3 = s^2 = e, rs = sr^2 \rangle.
    \]

    Let the map $\phi: D_3 \to \Sn{3}$ be given such that $r \mapsto \begin{pmatrix}1 & 2 & 3\end{pmatrix}$ and $s \mapsto \begin{pmatrix}1 & 2\end{pmatrix}$. We show that $\begin{pmatrix}1 & 2 & 3\end{pmatrix}$ and $\begin{pmatrix}1 & 2\end{pmatrix}$ satisfy the two rules above. For brevity let $\sigma = \begin{pmatrix}1 & 2 & 3\end{pmatrix}$ and $\tau = \begin{pmatrix}1 & 2\end{pmatrix}$.
    \begin{itemize}
        \item We check that $\phi(r^3) = \phi(s^2) = \phi(e)$.
        \begin{itemize}
            \item $\sigma^2 = \begin{pmatrix}1 & 2 & 3\end{pmatrix}\begin{pmatrix}1 & 2 & 3\end{pmatrix} = \begin{pmatrix}1 & 3 & 2\end{pmatrix} \neq \id$;
            \item $\sigma^3 = \begin{pmatrix}1 & 2 & 3\end{pmatrix}\begin{pmatrix}1 & 3 & 2\end{pmatrix} = \id$; and
            \item $\tau^2 = \begin{pmatrix}1 & 2\end{pmatrix}\begin{pmatrix}1 & 2\end{pmatrix} = \id$.
        \end{itemize}
        \item We check that $\phi(rs) = \phi(sr^2)$.
        \begin{itemize}
            \item $rs \mapsto \sigma\tau = \begin{pmatrix}1 & 2 & 3\end{pmatrix}\begin{pmatrix}1 & 2\end{pmatrix} = \begin{pmatrix}1 & 3\end{pmatrix}$; and
            \item $sr^2 \mapsto \tau\sigma^2 = \begin{pmatrix}1 & 2\end{pmatrix}\begin{pmatrix}1 & 3 & 2\end{pmatrix} = \begin{pmatrix}1 & 3\end{pmatrix}$.
        \end{itemize}
    \end{itemize}
    Thus $\phi$ is an isomorphism and so $D_3 \cong \Sn{3}$.
\begin{mdframed}
    State the number of elements in $\Sn{4}$.
    \begin{partquestions}{\alph*}
        \item Let $G$ be the cyclic group of order 4. Cayley's Theorem says that it is isomorphic to a subgroup of $\Sn{4}$. Find one such subgroup of $\Sn{4}$ and prove that it is, indeed, isomorphic to $G$.
        \item Let $G$ be the group with presentation
        \[
            \langle a, b \vert a^2 = b^2 = (ab)^2 = e \rangle.
        \]
        Cayley's Theorem says that it is isomorphic to a subgroup of $\Sn{4}$. Find one such subgroup of $\Sn{4}$ and prove that it is, indeed, isomorphic to $G$.
    \end{partquestions}
\end{mdframed}
\textbf{Solution}:\newline
 We note that $|\Sn{4}| = 4! = 24$.
    \begin{partquestions}{\alph*}
        \item Consider $H = \left\langle \begin{pmatrix}1 & 2 & 3 & 4\end{pmatrix} \right\rangle$. For brevity, let $\sigma = \begin{pmatrix}1 & 2 & 3 & 4\end{pmatrix}$. Note that
        \begin{itemize}
            \item $\sigma^2 = \begin{pmatrix}1 & 3\end{pmatrix}\begin{pmatrix}2 & 4\end{pmatrix} \neq \id$;
            \item $\sigma^3 = \begin{pmatrix}1 & 4 & 3 & 2\end{pmatrix} \neq \id$; and
            \item $\sigma^4 = \id$.
        \end{itemize}
        Thus, $|\sigma| = 4$ which means $|H| = 4$. Therefore, $G \cong H \leq \Sn{4}$.

        \item Let $\sigma = \begin{pmatrix}1 & 2\end{pmatrix}$ and $\tau = \begin{pmatrix}3 & 4\end{pmatrix}$. Let $H$ have presentation $\langle \sigma, \tau \rangle$. Notice that
        \begin{itemize}
            \item $\sigma^2 = \id$;
            \item $\tau^2 = \id$; and
            \item $(\sigma\tau)^2 = \id$.
        \end{itemize}
        Therefore $H = \{\id, \sigma, \tau, \sigma\tau\}$, so $G \cong H \leq \Sn{4}$.
    \end{partquestions}

\chapter{Products}
\section*{Exercises}
\begin{mdframed}
    Prove \myref{prop-external-direct-product-is-group}.
\end{mdframed}
\textbf{Solution}:\newline
 We prove the group axioms.
    \begin{itemize}
        \item \textbf{Closure}: Let $(g_1, h_1), (g_2, h_2) \in G \times H$. So $g_1, g_2 \in G$ and $h_1, h_2 \in H$. Thus $g_1g_2 \in G$ and $h_1h_2 \in H$ by closure of groups, and therefore $(g_1g_2, h_1h_2) \in G\times H$. Note that $(g_1, h_1)(g_2, h_2) = (g_1g_2, h_1h_2)$ by definition of the group operation in the external direct product, so $G \times H$ is closed under component-wise application of the group operations.

        \item \textbf{Associativity}: Let $(g_1, h_1), (g_2, h_2), (g_3, h_3) \in G \times H$. Then we see
        \begin{align*}
            (g_1, h_1)\left((g_2, h_2)(g_3, h_3)\right) &= (g_1, h_1)(g_2g_3, h_2h_3)\\
            &= (g_1(g_2g_3), h_1(h_2h_3))\\
            &= ((g_1g_2)g_3, (h_1h_2)h_3)\\
            &= (g_1g_2, h_1h_2)(g_3,h_3)\\
            &= \left((g_1,h_1)(g_2,h_2)\right)(g_3,h_3)
        \end{align*}
        which proves the associativity of the group operation.

        \item \textbf{Identity}: Let $e_G \in G$ and $e_H \in H$ be the identities of $G$ and $H$ respectively. Then we see for any $(g, h) \in G \times H$ that
        \[
            (e_G, e_H)(g, h) = (e_Gg, e_Hh) = (g, h)
        \]
        and
        \[
            (g, h)(e_G, e_H) = (ge_G, he_H) = (g, h),
        \]
        so $(e_G, e_H)$ is indeed the identity of $G \times H$.

        \item \textbf{Inverse}: Let $(g, h) \in G \times H$. Note that
        \[
            (g, h)(g^{-1}, h^{-1}) = (gg^{-1}, hh^{-1}) = (e_G, e_H)
        \]
        and
        \[
            (g^{-1}, h^{-1})(g, h) = (g^{-1}g, h^{-1}h) = (e_G, e_H)
        \]
        so $\left((g, h)\right)^{-1} = (g^{-1}, h^{-1})$.
    \end{itemize}
    Therefore $G \times H$ is a group.
\begin{mdframed}
    Simplify $(s, rs)(r^2s, r^3)$ in $D_3 \times D_4$.
\end{mdframed}
\textbf{Solution}:\newline
 We work component-wise:
    \begin{align*}
        (s, rs)(r^2s, r^3) &= (sr^2s, rsr^3)\\
        &= (s(r^2s), r(sr^3))\\
        &= (s(sr), r(rs))\\
        &= ((ss)r, (rr)s)\\
        &= (r, r^2s)
    \end{align*}
\begin{mdframed}
    Find all positive integers $m$ and $n$, where $1 < m < n$, such that $\Z_m \times \Z_n \cong \Z_{180}$.
\end{mdframed}
\textbf{Solution}:\newline
 Note that $180 = 2^2 \times 3^2 \times 5$. By \myref{thrm-Zm-cross-Zn-isomorphic-to-Zmn-condition}, we must have $mn = 180$ and $\gcd(m, n) = 1$. Thus, the possible values for $m$ and $n$ are
    \begin{itemize}
        \item $m = 4$ and $n = 45$;
        \item $m = 5$ and $n = 36$; and
        \item $m = 9$ and $n = 20$.
    \end{itemize}
\begin{mdframed}
    Let $G = \{1, 5\}$ and $H = \{1, 7\}$ be groups under $\otimes_{12}$. Find the internal direct product of $G$ and $H$.
\end{mdframed}
\textbf{Solution}:\newline
 Note that $5 \otimes_{12} 7 = 11$. Hence $GH = \{1, 5, 7, 11\}$.
\begin{mdframed}
    Let $\mathcal{S} = \{1, 5, 7, 11\}$, $G = \{1, 5\}$, and $H = \{1, 7\}$ be groups under $\otimes_{12}$. Find the value of $n$ such that $\mathcal{S} \cong (\Z_n)^2$.
\end{mdframed}
\textbf{Solution}:\newline
 From above exercise, $GH = \mathcal{S}$. Now $G = \langle 5 \rangle \cong \Z_2$ and $H \langle 7 \rangle \cong \Z_2$. Thus, $\mathcal{S} = GH = \cong G \times H \cong \Z_2 \times \Z_2 = (\Z_2)^2$, meaning $n = 2$.

\section*{Problems}
\begin{mdframed}
    Let $G$ and $H$ be abelian groups. Prove that $G \times H$ is also an abelian group.
\end{mdframed}
\textbf{Solution}:\newline
 Let $g_1, g_2 \in G$ and $h_1, h_2 \in H$. Then for $(g_1, h_1), (g_2, h_2) \in G\times H$ we see that
    \begin{align*}
        (g_1, h_1)(g_2, h_2) &= (g_1g_2, h_1h_2)\\
        &= (g_2g_1, h_2h_1)\\
        &= (g_2,h_2)(g_1,h_1)
    \end{align*}
    which means that $G \times H$ is abelian.
\begin{mdframed}
    Let $G$ and $H$ be groups. Prove that $G \times H \cong H \times G$.
\end{mdframed}
\textbf{Solution}:\newline
 Let the map $\phi: G\times H \to H \times G, (g, h) \mapsto (h, g)$. We prove that $\phi$ is an isomorphism:
    \begin{itemize}
        \item \textbf{Homomorphism}: Let $(g_1, h_1), (g_2, h_2) \in G \times H$. We note that
        \begin{align*}
            \phi((g_1, h_1)(g_2, h_2)) &= \phi((g_1g_2, h_1h_2))\\
            &= (h_1h_2, g_1g_2)\\
            &= (h_1, g_1)(h_2, g_2)\\
            &= \phi((g_1, h_1))\phi((g_2, h_2))
        \end{align*}
        which proves that $\phi$ is a homomorphism.
        \item \textbf{Injective}: Let $(g_1, h_1), (g_2, h_2) \in G \times H$ be such that $\phi((g_1, h_1)) = \phi((g_2, h_2))$. Then by definition of $\phi$ we have $(h_1, g_1) = (h_2, g_2)$. Clearly by comparing component parts of each ordered pair, we have $g_1 = g_2$ and $h_1 = h_2$, meaning $(g_1, h_1) = (g_2, h_2)$. Hence $\phi$ is injective.
        \item \textbf{Surjective}: Let $(h, g) \in H \times G$. Clearly $(g, h) \in G \times H$ and $\phi((g, h)) = (h, g)$, meaning that $(h, g)$ has a pre-image of $(g, h)$. Therefore $\phi$ is surjective.
    \end{itemize}
    Therefore $\phi$ is an isomorphism, meaning $G \times H \cong H \times G$.
\begin{mdframed}
    Let $G = \Z_6$, and consider the subgroups $H = \{0, 2, 4\}$ and $K = \{0, 3\}$ of $G$. Determine whether $G$ is the internal direct product of $H$ and $K$, justifying your answer.
\end{mdframed}
\textbf{Solution}:\newline
 We claim that $G$ is the internal direct product of $H$ and $K$. We need to check 3 things.
    \begin{itemize}
        \item $\boxed{G = HK}$ We note that
        \begin{align*}
            HK &= \{h \oplus_6 k \vert h \in H, k \in K\}\\
            &= \{0 \oplus_6 0, 0 \oplus_6 3, 2 \oplus_6 0, 2 \oplus_6 3, 4 \oplus_3 0, 4 \oplus_3 3\}\\
            &= \{0, 3, 2, 5, 4, 1\}\\
            &= \Z_6\\
            &= G
        \end{align*}
        so in fact $G = HK$.

        \item $\boxed{H \cap K = \{e\}}$ Clearly $H \cap K = \{0\}$.

        \item $\boxed{hk = kh}$ Since $\oplus_6$ is commutative, thus $h \oplus_6 k = k \oplus_6$.
    \end{itemize}
    Thus $G$ is the internal direct product of $H$ and $K$.
\begin{mdframed}
    Consider the \textbf{Klein four-group}\index{Klein four-group} $\mathrm{V}$ with presentation
    \[
        \langle a, b \vert a^2 = b^2 = (ab)^2 = e \rangle.
    \]
    Show that $\mathrm{V} \cong (\Z_2)^2$.
\end{mdframed}
\textbf{Solution}:\newline
 Define the subgroups $H = \{e, a\}$ and $K = \{e, b\}$. We show the $\mathrm{V}$ is the internal direct product of $H$ and $K$.
    \begin{itemize}
        \item $\boxed{\mathrm{V} = HK}$ Observe that
        \begin{align*}
            HK &= \{hk \vert h \in H, k \in K\}\\
            &= \{ee, eb, ae, ab\}\\
            &= \{e, b, a, ab\}\\
            &= \mathrm{V}
        \end{align*}
        so in fact $\mathrm{V} = HK$.

        \item $\boxed{H \cap K = \{e\}}$ Clearly $H \cap K = \{e\}$.

        \item $\boxed{hk = kh}$ Clearly if one of the elements is the identity then result follows. So assume that $h$ and $k$ are both non-identity elements, so $h = a$ and $k = b$. Note
        \begin{align*}
            kh &= ba\\
            &= (ba)\left((ab)(ab)\right) & (\text{since }(ab)^2 = e)\\
            &= (ba ab)(ab)\\
            &= (bb)(ab) & (\text{since }a^2 = e)\\
            &= ab & (\text{since }b^2 = e)\\
            &= hk
        \end{align*}
        so in fact $hk = kh$ for all $h \in H$, $k \in K$.
    \end{itemize}
    Therefore $\mathrm{V}$ is the internal direct product of $H$ and $K$.

    We note $H = \langle a\rangle \cong \Z_2$ and $K = \langle b \rangle \cong \Z_2$. By direct product equivalence (\myref{thrm-direct-product-equivalence}) we know $\mathrm{V} \cong H \times K \cong \Z_2 \times \Z_2 = (\Z_2)^2$.

\chapter{Further Homomorphisms}
\section*{Exercises}
\begin{mdframed}
    Consider the map $\phi: \Z_3 \to \Z_6, n \mapsto 2n$. Determine whether $\phi$ is a homomorphism and, if so, find its image.
\end{mdframed}
\textbf{Solution}:\newline
 $\phi$ is a homomorphism since
    \begin{align*}
        \phi(a \oplus_3 b) &= 2(a\oplus_3 b)\\
        &= (2a) \oplus_6 (2b)\\
        &= \phi(a) \oplus_6 \phi(b).
    \end{align*}
    The image is $\{0, 2, 4\}$.
\begin{mdframed}
    Let $i$ be the imaginary unit, i.e. $i^2 = -1$. Let the group $G = \Z$ be under addition and $H = \langle i \rangle$ be under multiplication. Let the map $\phi: G \to H, n \mapsto i^n$.  Show that $\phi$ is a homomorphism and hence find $\ker\phi$.
\end{mdframed}
\textbf{Solution}:\newline
 $\phi$ is a homomorphism since
    \begin{align*}
        \phi(a+b) &= i^{a+b}\\
        &=i^ai^b\\
        &=\phi(a)\phi(b).
    \end{align*}
    The kernel is the set of values which map to the identity of $H$, i.e. $\{n \in \Z \vert \phi(n) = 1\}$. Now note $H$ is a cyclic group and $|i| = 4$. Thus $i^4 = 1$. Furthermore $i^8 = (i^4)^2 = 1, i^{12} = 1, \dots, i^{4k} = 1$. Thus $\ker\phi = \{4n \vert n \in \Z\} = 4\Z$ (using coset notation).
\begin{mdframed}
    Prove that a homomorphism $\phi:G\to H$ is injective if and only if $\ker \phi$ is trivial, i.e. $\ker \phi = \{e_G\}$.
\end{mdframed}
\textbf{Solution}:\newline
 We prove the forward direction first. Suppose that $\phi$ is injective. Clearly $\phi(e_G) = e_H$. Let $x$ be an element which is in the kernel of $\phi$, meaning $\phi(x) = e_H$. Then, $\phi(x) = \phi(e_G) = e_H$ which means $x = e_G$ by injectivity of $\phi$. Hence the kernel is trivial.

    Now we prove the reverse direction. Suppose the kernel of $\phi$ is trivial, i.e. $\ker \phi = \{e_G\}$. Suppose now there exists elements $x$ and $y$ in $G$ such that $\phi(x) = \phi(y)$. This means that $(\phi(x))^{-1} = \phi(x^{-1}) = \phi(y^{-1}) = (\phi(y))^{-1}$. Hence,
    \[
        \phi(xy^{-1})
        = \phi(x)\phi(y^{-1})
        = \phi(x)\left(\phi(y)\right)^{-1}
        = e_H.
    \]
    Now since the kernel is trivial, this must mean that $xy^{-1} = e_G$ which immediately leads $x=y$. Hence $\phi$ is injective.
\begin{mdframed}
    Let $\phi: G \to H$ be a homomorphism between finite groups $G$ and $H$. Prove that
    \[
        |G| = |\im \phi|\times|\ker \phi|.
    \]
\end{mdframed}
\textbf{Solution}:\newline
 Note $G / \ker \phi \cong \im \phi$ by the Fundamental Homomorphism Theorem (\myref{thrm-isomorphism-1}). Furthermore, we note that $|G / \ker \phi| = \frac{|G|}{|\ker\phi|}$ by Lagrange's Theorem (\myref{thrm-lagrange}). Hence, $\frac{|G|}{|\ker\phi|} = |\im\phi|$ which leads to the result quickly.
\begin{mdframed}
    Let $G$ be a finite group, $H \leq G$, and $N \lhd G$. Prove that
    \[
        |HN| = \frac{|H||N|}{|H \cap N|}.
    \]
\end{mdframed}
\textbf{Solution}:\newline
 The Diamond Isomorphism Theorem (\myref{thrm-isomorphism-2}), statement 6, states that $H / (H\cap N) \cong HN / N$. Taking orders on both sides yields $\frac{|H|}{|H \cap N|} = \frac{|HN|}{|N|}$. Rearranging yields required result.
\begin{mdframed}
    Suppose $x$ and $y$ are positive integers such that $y = mx$ for some integer $m$. Let $H = x\Z$ and $N = y\Z$ be groups under addition.
    \begin{partquestions}{\roman*}
        \item Explain why $N \subseteq H$.
        \item Find a group $G$ such that $H \lhd G$ and $N \lhd G$.
        \item Hence find the order of $H/N$.
    \end{partquestions}
\end{mdframed}
\textbf{Solution}:\newline
 \begin{partquestions}{\roman*}
        \item Note $H = x\Z = \{ax \vert a \in \Z\}$ and $N = mx\Z = \{a(mx) \vert a \in \Z\}$, which necessarily means $N \subseteq H$.
        \item Let $G = \Z$. Then both $H$ and $N$ are clearly subgroups of $G$. Now since $G$ is abelian (since addition is commutative), therefore $H$ and $N$ are normal by \myref{prop-subgroup-of-abelian-group-is-normal}.
        \item The Third Isomorphism Theorem (\myref{thrm-isomorphism-3}) tells us that
        \[
            (G/N)/(H/N) \cong G/H.
        \]
        Now we know from \myref{problem-Zn-isomorphic-to-Z-by-nZ} that $|G/H| = |\Z/(x\Z)| = x$ and $|G/N| = |\Z/(y\Z)| = y$. Hence
        \[
            \frac{x}{|H/N|} = y
        \]
        which quickly implies $|H/N| = \frac yx$.
    \end{partquestions}

\section*{Problems}
\begin{mdframed}
    Let $G$ be a group. Prove that $G/G$ is isomorphic to the trivial group.
\end{mdframed}
\textbf{Solution}:\newline
 Construct the map $\phi: G \to \{e\}$ where $\phi(g) = e$. Clearly $\phi$ is a homomorphism as
    \[
        \phi(gh) = e = ee = \phi(g)\phi(h).
    \]
    Also, one sees that $\im\phi = \{e\}$ and $\ker\phi = G$. The Fundamental Homomorphism Theorem (\myref{thrm-isomorphism-1}) tells us that
    \[
        G / \ker\phi \cong \im\phi
    \]
    which immediately implies $G/G \cong \{e\}$.
\begin{mdframed}
    Let $R = (\R, +)$. Also let $G = R^2$ and $H = \left\{(r\times\sqrt2, r\times\sqrt3) \vert r\in R\right\}$ be groups under component-wise addition. Prove that $G/H \cong R$.
\end{mdframed}
\textbf{Solution}:\newline
 Consider $\phi: G \to R$ where $(x, y) \mapsto x\sqrt3 - y\sqrt2$. We show that $\phi$ is a homomorphism, find its image, and find its kernel.
    \begin{itemize}
        \item \textbf{Homomorphism}: Let $(x_1, y_1), (x_2, y_2) \in G$. Then
        \begin{align*}
            &\phi((x_1,y_1)(x_2,y_2))\\
            &= \phi((x_1+x_2,y_1+y_2))\\
            &= (x_1+x_2)\sqrt3 - (y_1+y_2)\sqrt2\\
            &= (x_1\sqrt3 - y_1\sqrt2) + (x_2\sqrt3 - y_2\sqrt2)\\
            &= \phi((x_1, y_1)) + \phi((x_2, y_2)).
        \end{align*}

        \item \textbf{Image}: We show that $\phi$ is surjective to show that $\im\phi = R$. For any $r \in R$, we have $\phi((\frac{r}{\sqrt3}, 0)) = \frac{r}{\sqrt3} \times \sqrt3 + 0 = r$ and $(\frac{r}{\sqrt3}, 0) \in G$, so $\phi$ is surjective.

        \item \textbf{Kernel}:
        \begin{align*}
            \ker\phi &= \{(x, y) \in G \vert \phi((x, y)) = 0\}\\
            &= \{(x, y) \in G \vert x\sqrt3-y\sqrt2 = 0\}\\
            &= \left\{(x, y) \in G \vert y = \frac{\sqrt{3}}{\sqrt{2}}x\right\}\\
            &= \left\{(x, \frac{\sqrt{3}}{\sqrt{2}}x) \vert x \in \R\right\}\\
            &= \left\{(r\sqrt2, \frac{\sqrt{3}}{\sqrt{2}}(r\sqrt2)) \vert r \in \R\right\}\\
            &= \{(r\sqrt2, r\sqrt3) \vert r \in \R\}\\
            &= H.
        \end{align*}
    \end{itemize}
    Thus $G / H \cong R$ by the Fundamental Homomorphism Theorem (\myref{thrm-isomorphism-1}).
\begin{mdframed}
    Let $G$ be a group. Let $H$ and $K$ be subgroups of $G$ such that $K \subseteq H$. Prove that $HK = H$.
\end{mdframed}
\textbf{Solution}:\newline
 We are given that $K \subseteq H$. Hence
    \begin{align*}
        HK &= \{hk \vert h \in H, k \in K \subseteq H\}\\
        &\subseteq \{hk \vert h \in H, k \in H\}\\
        &= \{h_1h_2 \vert h_1, h_2 \in H\}\\
        &= H. & (H \leq G \text{ so } H \text{ is closed})
    \end{align*}
    Therefore $HK \subseteq H$. Also, we know that $H \leq HK$ by the Diamond Isomorphism Theorem (\myref{thrm-isomorphism-2}), statement 3, so $H \subseteq HK$. Hence we obtain the fact that $H \subseteq HK \subseteq H$ which means $HK = H$ as required.
\begin{mdframed}
    Let $G$ be an abelian group with operation $\ast$. Let $I = \{(g, g^{-1}) \vert g \in G\}$ be a group under component-wise application of $\ast$.
    \begin{partquestions}{\roman*}
        \item Show that $I \cong G$.
        \item Hence prove that $G^2/I \cong G$.
    \end{partquestions}
\end{mdframed}
\textbf{Solution}:\newline
 \begin{partquestions}{\roman*}
        \item Consider the map $\phi: I \to G, (g, g^{-1}) \mapsto g$. We show that $\phi$ is an isomorphism:
        \begin{itemize}
            \item \textbf{Homomorphism}: Recall that $G$ is abelian, so $gh = hg$ for any $g, h \in G$. Let $(g, g^{-1}), (h, h^{-1}) \in I$. Then
            \begin{align*}
                \phi((g, g^{-1})(h, h^{-1})) &= \phi((gh, g^{-1}h^{-1}))\\
                &= \phi((gh, h^{-1}g^{-1}))\\
                &= \phi((gh, (gh)^{-1}))\\
                &= gh\\
                &= \phi((g, g^{-1}))\phi((h, h^{-1})).
            \end{align*}

            \item \textbf{Injective}: Suppose $(g, g^{-1}), (h, h^{-1}) \in I$ such that we have $\phi((g, g^{-1})) = \phi((h, h^{-1}))$. Then $g = h$ by definition of $\phi$ which clearly means $(g, g^{-1}) = (h, h^{-1})$.

            \item \textbf{Surjective}: Suppose $g \in G$. Then $(g, g^{-1}) \in I$ and $\phi((g, g^{-1})) = g$. Thus $g \in G$ has a pre-image $(g, g^{-1}) \in I$, so $\phi$ is surjective.
        \end{itemize}
        Hence $\phi$ is an isomorphism, meaning $I \cong G$.

        \item Consider the map $\psi: G^2 \to G, (g_1, g_2) \mapsto g_1g_2$. We show that $\psi$ is a homomorphism, then find its image and kernel.
        \begin{itemize}
            \item \textbf{Homomorphism}: Let $(g_1, g_2), (h_1, h_2) \in G^2$, so
            \begin{align*}
                \psi((g_1, g_2)(h_1, h_2)) &= \psi((g_1h_1, g_2h_2))\\
                &= g_1h_1g_2h_2\\
                &= g_1g_2h_1h_2 & (G \text{ is abelian})\\
                &= (g_1g_2)(h_1h_2)\\
                &= \psi((g_1, g_2))\psi((h_1, h_2))
            \end{align*}
            which means $\psi$ is a homomorphism.

            \item \textbf{Image}: We show that $\psi$ is surjective to show $\im \psi = G$. Consider any $g \in G$. Clearly we have $\psi((g, e)) = ge = g$, so $\psi$ is surjective.

            \item \textbf{Kernel}:
            \begin{align*}
                \ker\psi &= \{(g, h) \in G^2 \vert \psi((g, h)) = e\}\\
                &= \{(g, h) \in G^2 \vert gh = e\}\\
                &= \{(g, h) \in G^2 \vert h = g^{-1}\}\\
                &= \{(g, g^{-1}) \ | g \in G\}\\
                &= I.
            \end{align*}
        \end{itemize}

        Thus we have $G^2 / I \cong G$ by the Fundamental Homomorphism Theorem (\myref{thrm-isomorphism-1}).
    \end{partquestions}
\begin{mdframed}
    Let $G = m\Z$ and $H = n\Z$ be groups under addition, where $m\vert n$ and $m \neq n$. Let the map $\phi: G \to \Z/({\frac nm}\Z)$ be defined such that
    \[
        \phi(am) = a + \frac nm \Z.
    \]
    Prove that $G/H \cong \Z_{\frac nm}$.
\end{mdframed}
\textbf{Solution}:\newline
 We show that $\phi$ is an isomorphism.
    \begin{itemize}
        \item \textbf{Homomorphism} Let $am, bm \in m\Z$. Then
        \begin{align*}
            \phi(am + bm) &= \phi((a+b)m)\\
            &= (a+b) + \frac nm \Z\\
            &= \left(a + \frac nm \Z\right) + \left(b + \frac nm \Z\right)\\
            &= \phi(am) + \phi(bm)
        \end{align*}
        which means $\phi$ is a homomorphism.

        \item \textbf{Image}: Suppose $k + \frac nm \Z \in \Z/(\frac nm \Z)$. Clearly $\phi(k) = k + \frac nm \Z$ which means $\phi$ is surjective. Hence $\im\phi = \Z/(\frac nm \Z)$.

        \item \textbf{Kernel}:
        \begin{align*}
            \ker\phi &= \left\{am \vert \phi(am) = \frac nm \Z\right\}\\
            &= \left\{am \vert a + \frac nm \Z = \frac nm \Z\right\}\\
            &= \left\{am \vert a  = k\left(\frac nm\right),\; k \in \Z\right\}\\
            &= \left\{k\left(\frac nm\right)m \vert k \in \Z\right\}\\
            &= \{kn \vert k \in \Z\}\\
            &= n\Z
        \end{align*}
        so the kernel of $\phi$ is $n\Z$.
    \end{itemize}
    Hence $G/H \cong \Z/(\frac nm \Z)$ by the Fundamental Homomorphism Theorem (\myref{thrm-isomorphism-1}). But \myref{problem-Zn-isomorphic-to-Z-by-nZ} tells us that $\Z/(\frac nm \Z) \cong \Z_{\frac nm}$. Hence $G/H \cong \Z_{\frac nm}$.

\chapter{More Groups}
\section*{Exercises}
\begin{mdframed}
    Prove that
    \[
        \Z_{mn} / \Z_m \cong \Z_n.
    \]
    for any positive integers $m$ and $n$.
\end{mdframed}
\textbf{Solution}:\newline
 Let $G = \Z_{mn}$ and $H = \{0, n, 2n, \dots, (m-1)n\}$. Clearly $H$ is a subgroup of $G$ of order $m$. By \myref{problem-subgroup-of-cyclic-group-is-cyclic} we know $H$ is normal and cyclic with order $m$ and by \myref{exercise-quotient-group-of-cyclic-group-is-cyclic} we know $G/H$ is cyclic. The order of $G/H$ is $\frac{|G|}{|H|} = \frac{mn}{m} = n$ by Lagrange's Theorem (\myref{thrm-lagrange}), meaning that $G/H \cong \Z_n$. Hence, $\Z_{mn}/\Z_m \cong G/H \cong \Z_n$.
\begin{mdframed}
    The number 12 is equivalent to 0 in the group $\Z_n$. What are the possible value(s) of $n$?
\end{mdframed}
\textbf{Solution}:\newline
 Note 0 is the identity in $\Z_n$. By \myref{lemma-order-of-an-element-that-is-equivalent-to-identity}, we know that if $12$ is equivalent to the identity in $\Z_n$, then $12 = mn$ for some integer $m$. Since $n > 0$ we restrict $m$ to positive integers. Now $12 = 2^2 \times 3$. Thus the possible cases are:
    \begin{itemize}
        \item $n = 1$ with $m = 12$;
        \item $n = 2$ with $m = 6$;
        \item $n = 3$ with $m = 4$;
        \item $n = 4$ with $m = 3$;
        \item $n = 6$ with $m = 2$; and
        \item $n = 12$ with $m = 1$.
    \end{itemize}
\begin{mdframed}
    In the group $\Z_{210}$, find the order of 10, 42, 75, and 140.
\end{mdframed}
\textbf{Solution}:\newline
 $|10| = \frac{210}{\gcd(10, 210)} = \frac{210}{10} = 21$, $|42| = \frac{210}{\gcd(42, 210)} = \frac{210}{42} = 5$, $|75| = \frac{210}{\gcd(75, 210)} = \frac{210}{15} = 14$, and $|140| = \frac{210}{\gcd(140, 210)} = \frac{210}{70} = 3$.
\begin{mdframed}
    Find all the generators of the following groups.
    \begin{partquestions}{\alph*}
        \item $\Z_{10}$
        \item $\Z_{101}$
    \end{partquestions}
\end{mdframed}
\textbf{Solution}:\newline
 \begin{partquestions}{\alph*}
        \item Note that $10 = 2 \times 5$. Generators of the group $\Z_n$ (which has order 10) has to satisfy $\gcd(m,n) = 1$ by \myref{corollary-element-in-cyclic-group-is-generator-iff-gcd-is-1}. The positive integers that satisfy this requirement (and which are less than 10) are 1, 3, 7, 9. Thus they are the generators of $\Z_{10}$.
        \item Note that 101 is prime. Hence all positive integers from 1 to 100 (inclusive) are generators. (Note that 0 is not as 0 is the identity.)
    \end{partquestions}
\begin{mdframed}
    Find all the normal subgroups of the quaternion group $\mathrm{Q}$.\newline
    (\textit{Hint: consider \myref{problem-subgroup-of-index-2}.})
\end{mdframed}
\textbf{Solution}:\newline
 We show that all subgroups of $\mathrm{Q}$ are, in fact, normal. We consider the first definition of the quaternion group.
    \begin{itemize}
        \item Clearly $\{1\} \lhd \mathrm{Q}$ and $\mathrm{Q} \unlhd \mathrm{Q}$.
        \item The subgroups $\langle i\rangle$, $\langle j\rangle$, and $\langle k\rangle$ have order 4. Therefore, Lagrange's Theorem (\myref{thrm-lagrange}) tells us that they have index 2. Hence these subgroups are normal by \myref{problem-subgroup-of-index-2}.
        \item Consider the subgroup $\langle -1 \rangle = \{1, -1\}$. \begin{itemize}
            \item $1\langle -1 \rangle = \langle -1 \rangle1$, since 1 is the identity;
            \item $-1\langle -1 \rangle = \{1, -1\} = \langle -1 \rangle(-1)$;
            \item $i\langle -1 \rangle = \{-i, i\} = \langle -1 \rangle i$;
            \item $-i\langle -1 \rangle = \{i, -i\} = \langle -1 \rangle (-i)$;
            \item $j\langle -1 \rangle = \{-j, j\} = \langle -1 \rangle j$;
            \item $-j\langle -1 \rangle = \{j, -j\} = \langle -1 \rangle (-j)$;
            \item $k\langle -1 \rangle = \{-k, k\} = \langle -1 \rangle k$; and
            \item $-k\langle -1 \rangle = \{k, -k\} = \langle -1 \rangle (-k)$.
        \end{itemize}
        Thus $\langle -1 \rangle$ is normal.
    \end{itemize}
    Hence all subgroups of $\mathrm{Q}$ are normal.
\begin{mdframed}
    Write the transposition $\begin{pmatrix}2&6\end{pmatrix}$ into a composition of adjacent transpositions.
\end{mdframed}
\textbf{Solution}:\newline
 (2 6) = (2 3)(3 4)(4 5)(5 6)(4 5)(3 4)(2 3).
\begin{mdframed}
    Find $\sgn\left(\begin{pmatrix}1&3&2&5&4\end{pmatrix}\right)$.
\end{mdframed}
\textbf{Solution}:\newline
 Note that (1 3 2 5 4) = (1 4)(1 5)(1 2)(1 3). Thus by \myref{thrm-parity-of-permutation}, (1 3 2 5 4) is even and thus has a sign of $+1$.
\begin{mdframed}
    List all elements of $\An{3}$.
\end{mdframed}
\textbf{Solution}:\newline
 Note that $\An{3}$ has order $\frac{3!}{2} = 3$ so we should expect 3 permutations. Clearly the identity is one such permutation. Looking at \myref{example-symmetric-group-of-degree-3} we can find two more: (1 2 3) and (1 3 2).
\begin{mdframed}
    List the elements of $\Un{10}$.
\end{mdframed}
\textbf{Solution}:\newline
 $\Un{10} = \{1, 3, 7, 9\}$.
\begin{mdframed}
    Let $a \in \Un{n}$. Prove that $|a|$ divides $\totient(n)$.
\end{mdframed}
\textbf{Solution}:\newline
 By a corollary of Lagrange's Theorem (\myref{corollary-order-of-group-multiple-of-order-of-element}), the order of $a$ dives the order of the group $\Un{n}$. Now the order of $\Un{n} = \totient(n)$. Thus the order of $a$ divides $\totient(n)$.
\begin{mdframed}
    Find the matrix given by the product
    \[
        \begin{pmatrix}1&1&0\\0&1&0\\0&1&1\end{pmatrix}\begin{pmatrix}1&1&1\\1&0&1\\1&1&1\end{pmatrix}.
    \]
\end{mdframed}
\textbf{Solution}:\newline
 The matrix product should be $\begin{pmatrix}2&1&2\\1&0&1\\2&1&2\end{pmatrix}$.
\begin{mdframed}
    Let $G$ be a group. Prove that $\Inn{G} \unlhd \Aut{G}$.
\end{mdframed}
\textbf{Solution}:\newline
 We already proved that $\Inn{G} \leq \Aut{G}$ so we only need to prove normality.

    Let $\phi \in \Aut{G}$ and $\iota_g \in \Inn{G}$. For brevity let $f = \phi\iota_g\phi^{-1}$. We note that $f \in \Aut{G}$; we need to prove that $f \in \Inn{G}$.

    Suppose $x \in G$ such that $w = \phi^{-1}(x)$ (as $\phi$ is an isomorphism, there exists a $w \in G$). Then
    \begin{align*}
        f(x) &= \left(\phi\iota_g\phi^{-1}\right)(x)\\
        &= \phi(\iota_g(\phi^{-1}(x)))\\
        &= \phi(\iota_g(w))\\
        &= \phi(gwg^{-1})\\
        &= \phi(g)\phi(w)\phi(g^{-1})\\
        &= \phi(g)x\left(\phi(g)\right)^{-1}
    \end{align*}
    which shows that $f \in \Inn{G}$. Hence, $\Inn{G} \unlhd \Aut{G}$.

\section*{Problems}
\begin{mdframed}
    Consider the group $\Z_{10101}$.
    \begin{partquestions}{\alph*}
        \item Find the smallest positive integer $a$ such that $1870a$ is a multiple of 10101.
        \item Find the smallest positive integer $b$ such that $3774b$ is a multiple of 10101.
    \end{partquestions}
\end{mdframed}
\textbf{Solution}:\newline
 We note that the two questions are equivalent to finding the orders of 3774 and 1870 in the group $\Z_{10101}$. We note that
    \begin{align*}
        1870 &= 2 \times 5 \times 11 \times 17,\\
        3774 &= 2 \times 3 \times 17 \times 37, \text{ and}\\
        10101 &= 3 \times 7 \times 13 \times 37.
    \end{align*}
    Therefore, $\gcd(1870, 10101) = 1$ and $\gcd(3774, 10101) = 3 \times 37 = 111$. Hence $|1870| = 10101$ and $|3774| = \frac{10101}{111} = 91$. Therefore, $a = 10101$ and $b = 91$.
\begin{mdframed}
    Find the largest integer $n$ such that $\An{n}$ is an abelian group, proving your claim. Hence find all integers $k$ such that $\An{k}$ is cyclic.
\end{mdframed}
\textbf{Solution}:\newline
 We claim that $\An{n}$ is non-abelian for $n > 3$. Note that $\pi = \begin{pmatrix}1 & 2 & 3\end{pmatrix}$ and $\sigma = \begin{pmatrix}2 & 3 & 4\end{pmatrix}$ are both even permutations, and hence are in $\An{n}$ for $n > 3$. We note
    \begin{itemize}
        \item $\pi\sigma = \begin{pmatrix}1 & 2 & 3\end{pmatrix}\begin{pmatrix}2 & 3 & 4\end{pmatrix} = \begin{pmatrix}1 & 2\end{pmatrix}\begin{pmatrix}3 & 4\end{pmatrix}$; and
        \item $\sigma\pi = \begin{pmatrix}2 & 3 & 4\end{pmatrix}\begin{pmatrix}1 & 2 & 3\end{pmatrix} = \begin{pmatrix}1 & 3\end{pmatrix}\begin{pmatrix}2 & 4\end{pmatrix}$.
    \end{itemize}
    Hence $\pi\sigma \neq \sigma\pi$ for $\An{n}$ where $n > 3$, meaning that $\An{n}$ is non-abelian for $n > 3$. We note that
    \begin{itemize}
        \item $\An{2}$ has order 1 so $\An{2}$ is the trivial group, which is abelian (and cyclic); and
        \item $\An{3}$ has order 3 so $\An{3}$ is cyclic and thus abelian.
    \end{itemize}
    Thus the largest integer $n$ for which $\An{n}$ is abelian is $n = 3$. Furthermore $\An{k}$ is cyclic if $k = 2$ ore $k = 3$.
\begin{mdframed}
    Suppose $r$ is an odd primitive root modulo $p^k$, where $p$ is an odd prime and $k \geq 1$. Prove that $r$ is also a primitive root modulo $2p^k$.
\end{mdframed}
\textbf{Solution}:\newline
 We first note that
    \[
        \totient(2p^k) = 2p^k\left(1-\frac12\right)\left(1-\frac1p\right) = p^k\left(1-\frac1p\right) = \totient(p^k).
    \]

    Now we are given that $r$ is an odd primitive root of $p^k$. Then $\gcd(r, 2p^k) = 1$ since $\gcd(r, p^k) = 1$ (as $r \in \Un{p^k}$). Now because $r$ is odd, thus $r \in \Un{2p^k}$. Let $n = |r|$ in $\Un{2p^k}$. Then by \myref{exercise-order-of-a-divides-phi-a}, $n$ divides $\totient(2p^k)$. But at the same time, $r$ is a generator in $\Un{p^k} \cong \Z_{\phi(p^k)}$, so $\totient(p^k) = \totient(2p^k)$ divides $n$ by \myref{lemma-order-of-an-element-that-is-equivalent-to-identity}. Since $n$ divides $\totient(2p^k)$ and $\totient(2p^k)$ divides $n$ simultaneously, therefore $n = \totient(2p^k) = |\Un{2p^k}|$ which means that $r$ is a primitive root modulo $2p^k$.
\begin{mdframed}
    Let $G = \Cn{n}$ with generator $g$.
    \begin{partquestions}{\roman*}
        \item Suppose $f_1: G \to G$ and $f_2: G \to G$ are homomorphisms. Prove that $f_1 = f_2$ if and only if $f_1(g) = f_2(g)$.

        \item Let $f: G \to G$ be a homomorphism. Explain why $f(g) = g^{m_f}$ for some $m_f \in \Z_n$.

        \item Show that the $m_f$ obtained in \textbf{(ii)} is unique to $f$.

        \item Suppose $f_1: G \to G$ and $f_2: G \to G$ are homomorphisms. Prove that
        \[
            m_{f_1\circ f_2} = m_{f_1} \otimes_n m_{f_2},
        \]
        where $\circ$ denotes function composition and $\otimes_n$ denotes multiplication modulo $n$.

        \item Let $f: G \to G$ be a homomorphism. Prove that $f$ is an automorphism if and only if $m_f$ has a multiplicative inverse modulo $n$. That is, there exists $k \in \Un{n}$ such that $m_fk \equiv 1 \pmod n$ if and only if $f$ is an automorphism.\newline
        (\textit{Hint: consider \myref{prop-multiplicative-inverse-exists-iff-coprime}, where we have $ab \equiv 1 \pmod m$ if and only if $\gcd(a, m) = 1$.})

        \item Hence, by considering the map $\phi: \Aut{G} \to \Un{n}$ where $\phi(f) = m_f$, prove that
        \[
            \Aut{G} \cong \Un{n}.
        \]
    \end{partquestions}
\end{mdframed}
\textbf{Solution}:\newline
 \begin{partquestions}{\roman*}
        \item The forward direction is clearly true since if $f_1 = f_2$, then $f_1(x) = f_2(x)$ for all $x \in G$, including $g \in G$. For the reverse direction, assume $f_1(g) = f_2(g)$. Note that
        \[
            f_1(g^k) = (f_1(g))^k = (f_2(g))^k = f_2(g^k)
        \]
        for any integer $k$. Since $g$ is a generator, thus we have $f_1(x) = f_2(x)$ for all $x \in G$, meaning $f_1 = f_2$.

        \item We note $f(g) \in G$. Since $g$ is a generator hence $f(g) = g^k$ for some $k$. Hence any homomorphism from $G$ to $G$ is of the form $f(g) = g^{m_f}$ where $0 \leq m_f \leq n-1$, which means $m_f \in \Z_n$.

        \item Suppose $f_2: G \to G$ is another homomorphism where $f_2(g) = g^{m_f}$. Then
        \[
            f(g) = g^{m_f} = f_2(g)
        \]
        which means $f = f_2$ by \textbf{(i)}. Hence the value of $m_f$ is unique to $f$.

        \item Consider $f_1(f_2(g))$. On one hand,
        \[
            f_1(f_2(g)) = f_1(g^{m_{f_2}}) = (f_1(g))^{m_{f_2}} = g^{m_{f_1}m_{f_2}},
        \]
        while on the other,
        \[
            f_1(f_2(g)) = (f_1 \circ f_2)(g) = g^{m_{f_1\circ f_2}}
        \]
        by definition of $m_f$ as introduced in \textbf{(ii)}. Therefore $m_{f_1\circ f_2} \equiv m_{f_1}m_{f_2} \pmod n$. In other words, $m_{f_1\circ f_2} = m_{f_1} \otimes_n m_{f_2}$.

        \item We prove the forward direction first by assuming that $f$ is an automorphism. Hence $f$ is surjective, meaning that there exists an $a \in G$ such that $f(a) = g$. Since $a \in G$ thus $a = g^k$ for some $k \in \Z_n$ (we will show $k \in \Un{n}$ later). Observe
        \[
            g = f(a) = f(g^k) = (f(g))^k = g^{m_fk}
        \]
        which means $m_fk \equiv 1 \pmod n$. By \myref{prop-multiplicative-inverse-exists-iff-coprime}, this means that we have $\gcd(m_f, n) = 1$ and $\gcd(k, n) = 1$. Therefore, $m_f, k \in \Un{n}$. Hence $k$ is the multiplicative inverse of $m_f$.

        We now prove the reverse direction. Assume $m_f$ has a multiplicative inverse (say $k$), meaning $m_fk \equiv 1 \pmod n$. As above this means $m_f, k \in \Un{n}$. We show that $f$ is a bijection.
        \begin{itemize}
            \item \textbf{Injective}: Suppose $x, y \in G$ such that $f(x) = f(y)$. Since $g$ is a generator we may take $x = g^p$ and $y = g^q$. Hence we have $g^{m_fp} = g^{m_fq}$. Then
            \[
                \left(g^{m_fp}\right)^k = g^{km_fp} = \left(g^{km_f}\right)^p = g^p
            \]
            and $\left(g^{m_fq}\right)^k = g^q$. Hence this implies $g^p = g^q$ which means $x = y$.
            \item \textbf{Surjective}: Suppose $x \in G$. Since $g$ is a generator we may take $x = g^p$. Then $f(g^{kp}) = g^{m_fkp} = g^p = x$.
        \end{itemize}
        Also $f$ is given to be a homomorphism. Hence $f$ is an isomorphism. Since $f: G \to G$, it is thus an automorphism.

        \item We prove that $\phi$ is an isomorphism.
        \begin{itemize}
            \item \textbf{Homomorphism}: Let $f_1, f_2 \in \Aut{G}$. Then
            \begin{align*}
                \phi(f_1\circ f_2) &= m_{f_1\circ f_2} & (\text{definition of } m_f \text{ in }\textbf{(ii)})\\
                &= m_{f_1} \otimes_n m_{f_2} & (\text{by \textbf{(iv)}})\\
                &= \phi(f_1)\otimes_n\phi(f_2),
            \end{align*}
            which means $\phi$ is a homomorphism.

            \item \textbf{Injective}: Suppose we have $f_1, f_2 \in \Aut{G}$ such that $\phi(f_1) = \phi(f_2)$. Thus $m_{f_1} = m_{f_2}$ by definition of $\phi$. However, we know that the value of $m$ uniquely defines a homomorphism from $G$ to $G$ from \textbf{(iii)}. Hence $f_1 = f_2$, which shows that $\phi$ is injective.

            \item \textbf{Surjective}: Suppose $r \in \Un{n}$. Define $f: G \to G$ where $f(g) = g^r$. Since $r \in \Un{n}$ it has a multiplicative inverse, which means that $f$ is an automorphism by \textbf{(v)}. Clearly $\phi(f) = r$, so $r$ has a pre-image. So $\phi$ is surjective.
        \end{itemize}
        Hence $\phi$ is an isomorphism, meaning $\Aut{G} \cong \Un{n}$.
    \end{partquestions}

\chapter{Group Actions}
\section*{Exercises}
\begin{mdframed}
    Let $G$ be a group and the function $\alpha: G\times G \to G$ be defined such that $\alpha(g, x) = gxg^{-1}$. Show that $\alpha$ is a group action of $G$ on $G$.
\end{mdframed}
\textbf{Solution}:\newline
 We prove the two group action axioms.
    \begin{itemize}
        \item \textbf{Identity}: $\alpha(e, x) = exe^{-1} = x$.
        \item \textbf{Compatibility}: Note
        \begin{align*}
            \alpha(g, \alpha(h, x)) &= \alpha(g, hxh^{-1})\\
            &= gh x h^{-1}g^{-1}\\
            &= (gh)x(gh)^{-1}\\
            &= \alpha(gh, x).
        \end{align*}
    \end{itemize}
    Therefore $\alpha$ is a group action of $G$ on $G$.
\begin{mdframed}
    Let $X = \{1, 2, 3\}$, and let $\Sn{3}$ act on $X$. What are the fixed point(s) of each of the 6 actions in $\Sn{3}$?
\end{mdframed}
\textbf{Solution}:\newline
 Recall there are 6 elements in $\Sn{3}$: $\id$, $\begin{pmatrix}1 & 2 & 3\end{pmatrix}$, $\begin{pmatrix}1 & 3 & 2\end{pmatrix}$, $\begin{pmatrix}1 & 2\end{pmatrix}$, $\begin{pmatrix}1 & 3\end{pmatrix}$, and $\begin{pmatrix}2 & 3\end{pmatrix}$. Clearly the identity has all elements of $X$ as fixed points. It is also clear that $\begin{pmatrix}1 & 2 & 3\end{pmatrix}$ and $\begin{pmatrix}1 & 3 & 2\end{pmatrix}$ have no fixed points since they permute all elements. For the rest, the fixed points are the missing element from the cycle notation, i.e. $\begin{pmatrix}1 & 2\end{pmatrix}$ has fixed point 3, $\begin{pmatrix}1 & 3\end{pmatrix}$ has fixed point 2, and $\begin{pmatrix}2 & 3\end{pmatrix}$ has fixed point 1.
\begin{mdframed}
    Let $X = \{1, 2, 3\}$, and let $\Sn{3}$ act on $X$. What are the stabilizers of each of the 3 elements in $X$?
\end{mdframed}
\textbf{Solution}:\newline
 For 1, it is $\{\id, \begin{pmatrix}2 & 3\end{pmatrix}\}$. For 2, it is $\{\id, \begin{pmatrix}1 & 3\end{pmatrix}\}$. For 3, it is $\{\id, \begin{pmatrix}1 & 2\end{pmatrix}\}$.
\begin{mdframed}
    Let $G$ be a group that acts on a set $X$. Prove that $g \cdot x = h \cdot x$ if and only if $g^{-1}h \in \Stab{G}{x}$.
\end{mdframed}
\textbf{Solution}:\newline
 We work from the statement forwards. Note that each of these statements are ``if and only if'' statements.
    \begin{align*}
        g \cdot x = h \cdot x &\iff g^{-1} \cdot (g \cdot x) = g^{-1} \cdot (h \cdot x)\\
        &\iff (g^{-1}g) \cdot x = (g^{-1}h) \cdot x\\
        &\iff e \cdot x = (g^{-1}h) \cdot x\\
        &\iff x = (g^{-1}h) \cdot x\\
        &\iff (g^{-1}h) \cdot x = x\\
        &\iff g^{-1}h \in \Stab{G}{x}
    \end{align*}
\begin{mdframed}
    Let $G$ be a group that acts on a non-empty set $X$. Prove that
    \begin{partquestions}{\alph*}
        \item every orbit is non-empty;
        \item every element in $X$ is in some orbit; and
        \item if $\Orb{G}{x_1} \cap \Orb{G}{x_2} \neq \emptyset$ where $x_1, x_2 \in X$, then $\Orb{G}{x_1} = \Orb{G}{x_2}$.
    \end{partquestions}
\end{mdframed}
\textbf{Solution}:\newline
 \begin{partquestions}{\alph*}
        \item An orbit takes the form $\Orb{G}{x}$. Clearly $e \cdot x = x$ so $x \in \Orb{G}{x}$ and thus $\Orb{G}{x}$ is non-empty.
        \item Let $x \in X$. Since $e \cdot x = x$, so $x \in \Orb{G}{x}$.
        \item Suppose $x \in \Orb{G}{x_1} \cap \Orb{G}{x_2}$ (as their intersection is non-empty). Then there exists $g_1, g_2 \in G$ such that $g_1\cdot x_1 = x = g_2\cdot x_2$. Thus,
        \begin{align*}
            x_1 &= e \cdot x_1\\
            &= (g_1^{-1}g_1)\cdot x_1\\
            &= g_1^{-1} \cdot (g_1 \cdot x_1)\\
            &= g_1^{-1} \cdot (g_2 \cdot x_2)\\
            &= (g_1^{-1}g_2) \cdot x_2.
        \end{align*}
        Now suppose $y \in \Orb{G}{x_1}$. Then $y = g\cdot x_1$ for some $g \in G$. Hence,
        \begin{align*}
            y &= g\cdot x_1 \\
            &= g \cdot \left((g_1^{-1}g_2) \cdot x_2\right)\\
            &= (\underbrace{gg_1^{-1}g_2}_{\text{In } G})\cdot x_2\\
            &\in \Orb{G}{x_2}
        \end{align*}
        which means any element in $\Orb{G}{x_1}$ is also in $\Orb{G}{x_2}$. Hence, $\Orb{G}{x_1}$ is a subset of $\Orb{G}{x_2}$. A similar argument can be used to show that $\Orb{G}{x_2}$ is a subset of $\Orb{G}{x_1}$. Hence $\Orb{G}{x_1} = \Orb{G}{x_2}$.
    \end{partquestions}
\begin{mdframed}
    Let $G$ be a group that acts on a non-empty set $X$. Show that if the group action is transitive if and only if $\Orb{G}{x} = X$ for all $x \in X$.
\end{mdframed}
\textbf{Solution}:\newline
 We prove the forward direction first: suppose the action is transitive. So there exists $x \in X$ such that $\Orb{G}{x} = X$. Now recall that distinct orbits partition $X$ (\myref{prop-distinct-orbits-partition-set}); since $X$ is an orbit, therefore the only partition using orbits of $X$ is $\{X\}$. In particular, any other $y \in X$ must also have an orbit of $X$.

    The reverse direction is trivial: suppose $\Orb{G}{x} = X$ for all $x \in X$. Then certainly there exists an element $x \in X$ such that $\Orb{G}{x} = X$, meaning that the group action is transitive.
\begin{mdframed}
    Let $G = \Sn{n}$ be a transformation group and $X = \{1, 2, 3, \dots, n\}$ be a $G$-set.
    \begin{partquestions}{\roman*}
        \item Show that the group action is transitive.
        \item Find the order of the stabilizer of $x \in X$ by $G$.
    \end{partquestions}
\end{mdframed}
\textbf{Solution}:\newline
 \begin{partquestions}{\alph*}
        \item Consider $x = n$. The orbit of $n$ is all of $X$. Consider the permutation $\sigma = \begin{pmatrix}k & n\end{pmatrix}$ where $1 \leq k \leq n$. Clearly $\sigma \in \Sn{n}$. Note that $\sigma \cdot n = \sigma(n) = k$. Thus, $\Orb{G}{n} = X$, meaning that the group action ``$\cdot$'' given by $g \cdot x \mapsto g(x)$ is transitive.
        \item Note that $|X| = n$ and $|\Sn{n}| = n!$. By Orbit-Stabilizer theorem (\myref{thrm-orbit-stabilizer}), the stabilizer of $x$ by $G$ must have order $\frac{n!}{n} = (n-1)!$.
    \end{partquestions}
\begin{mdframed}
    Let $G$ be a group and $x \in G$. Prove that $|\Cl{x}| = [G : \Centralizer{G}{x}]$.
\end{mdframed}
\textbf{Solution}:\newline
 By the Orbit-Stabilizer theorem (\myref{thrm-orbit-stabilizer}),
    \[
        |\Orb{G}{x}| = \frac{|G|}{|\Stab{G}{x}|} = [G : \Stab{G}{x}].
    \]
    Under the group action of conjugation, $\Orb{G}{x} = \Cl{x}$ and $\Stab{G}{x} = \Centralizer{G}{x}$. Hence, $|\Cl{x}| = [G : \Centralizer{G}{x}]$ as required.
\begin{mdframed}
    The Cayley table of $D_3$ is in \myref{example-presentation-of-D3}.
    \begin{partquestions}{\alph*}
        \item Find $\CenterGrp{D_3}$, the center of $D_3$.
        \item For each conjugacy class of $D_3$ with more than 1 element, find a representative.
        \item Find the class equation of $D_3$.
    \end{partquestions}
\end{mdframed}
\textbf{Solution}:\newline
 \begin{partquestions}{\alph*}
        \item One sees that $\CenterGrp{D_3} = \{e\}$ based on the group table of $D_3$.
        \item Recall that every element in $D_3$ can be expressed in the form $r^as^b$ where $a \in \{0, 1, 2\}$ and $b \in \{0, 1\}$. One finds that $\Cl{r} = \{r, r^2\}$ and $\Cl{s} = \{s, rs, rs^2\}$.
        \item The class equation is $6 = 1 + 2 + 3$.
    \end{partquestions}
\begin{mdframed}
    For a finite group $G$ with order $np$ where $p$ is prime and $n$ is a positive integer, show that $G$ has a subgroup of order $p$.
\end{mdframed}
\textbf{Solution}:\newline
 By the first part of Cauchy's Theorem (\myref{thrm-cauchy}) there exists an element (say $x$) with order $p$. Consider $H = \langle x \rangle$. Note that $|H| = p$ and $H \leq G$. Hence we found a subgroup of $G$ of order $p$.

\section*{Problems}
\begin{mdframed}
    Let $G = D_5$, the dihedral group of order 10.
    \begin{partquestions}{\roman*}
        \item What are the possible orders of a non-trivial proper subgroup of $G$?
        \item For each of the possible order(s) identified in \textbf{(i)}, find such a subgroup.
    \end{partquestions}
\end{mdframed}
\textbf{Solution}:\newline
 \begin{partquestions}{\roman*}
        \item Since $G = D_5$ has order $10 = 2 \times 5$, by \myref{thrm-cauchy}, $G$ must have non-trivial proper subgroups of orders 2 and 5.
        \item For the subgroup with order 2, $\{e, s\} \leq G$. For the subgroup with order 5, $\{e, r, r^2, r^3, r^4\} \leq G$.
    \end{partquestions}
\begin{mdframed}
    Suppose $G$ is a finite group with order $n > 1$ where $g^2 = e$ for all $g \in G$. Prove that $n = 2^k$ where $k$ is a positive integer.
\end{mdframed}
\textbf{Solution}:\newline
 Suppose that $n = mp$ where $m$ is a positive integer and $p$ is an odd prime. Then there must exist an element $x$ such that $x^p = e$ by Cauchy's Theorem (\myref{thrm-cauchy}). But all elements in $G$ satisfy $x^2 = e$. Since $p$ is odd, thus $x^p \neq e$ which is a contradiction. Hence, $n$ cannot be a multiple of an odd prime, meaning that $n = 2^k$ where $k$ is a positive integer.
\begin{mdframed}
    A group action is said to be \textbf{free}\index{group action!free} if $g\cdot x = x$ implies that $g$ is the identity (i.e., only the identity fixes any $x$).

    Let $G$ be a group and $S$ be a non-empty $G$-set. Suppose $G$ acts on $S$ freely and transitively. Prove that $G$ and $S$ have the same number of elements.
\end{mdframed}
\textbf{Solution}:\newline
 We define the map $\phi: G \to S, g \mapsto g \cdot x$ where $x \in S$ is a fixed element. We show that $\phi$ is a bijection.
    \begin{itemize}
        \item \textbf{Injective}: Suppose $g, h \in G$ are such that $\phi(g) = \phi(h)$, meaning that $g\cdot x = h\cdot x$. Hence $(g^{-1}h) \cdot x = x$ which quickly implies that $g^{-1}h = e$ since the group action is free. Therefore $g = h$ which proves that $\phi$ is injective.
        \item \textbf{Surjective}: Suppose $y \in S$. Then since the group action is transitive, there must exist an element $g \in G$ such that $g \cdot x = y$. Hence, $\phi(g) = g\cdot x = y$, meaning that the pre-image of $y$ is $g$. Therefore $\phi$ is surjective.
    \end{itemize}
    Thus $\phi$ is bijective, which means that $|G| = |S|$.
\begin{mdframed}
    Let $G$ be a group of order 25 and $X$ be a $G$-set of 24 elements. Show that there exists an element in $G$ with a fixed point.
\end{mdframed}
\textbf{Solution}:\newline
 Recall that $\Orb{G}{x} = \{y \in X \vert g\cdot x = y \text{ for some } g \in G\}$.

    Let $x \in X$. Then by the Orbit-Stabilizer theorem (\myref{thrm-orbit-stabilizer}), $|\Orb{G}{x}| = \frac{|G|}{|\Stab{G}{x}|}$. Since $\Stab{G}{x} \leq G$ thus it has order of either 1, 5, or 25 by Lagrange's Theorem (\myref{thrm-lagrange}). Hence, the number of elements in $\Orb{G}{x}$ is either 1, 5, or 25.

    Now $X$ has 24 elements. Since $\Orb{G}{x}$ can, at most, be the entire set $X$ which has 24 elements, thus $|\Orb{G}{x}| \neq 25$. Hence $\Orb{G}{x}$ has either 1 or 5 elements. Now by \myref{exercise-distinct-orbits-partition-set}, distinct orbits must partition the set $X$. Let the number of orbits of size 1 be $a$ and the number of orbits of size 5 be $b$. Hence, $1a + 5b = 24$. Since $b$ is an integer, thus $5b$ must be a multiple of 5, which means that $a \geq 1$. Hence, there exists an orbit of size 1, which means that there is a $g \in G$ with a fixed point.
\begin{mdframed}
    A bracelet consists of 3 beads that each can be one of $n$ colours. Two bracelets are considered to be identical if the rotation of one yields the other, or if one can be obtained via reflecting about a line, or any combination of these two actions. How many distinct bracelets are there?
\end{mdframed}
\textbf{Solution}:\newline
 We note that the group in question that acts upon the bracelet is the group $D_3$. We consider Burnside's Lemma (\myref{lemma-burnside}) to answer this question. There are 6 actions to consider:
    \begin{itemize}
        \item $\boxed{e}$: The number of fixed points is the total number of colourings, $n^3$.
        \item $\boxed{r}$: Rotating a bracelet $120^\circ$ results in all points affecting one another, so the only fixed points would be colourings of the same colour. There are $n$ such arrangements.
        \item $\boxed{r^2}$: Similar argument as $r$ yields $n$ arrangements.
        \item $\boxed{s}$: This `fixes' one bead and flips the other two about a line. A fixed point thus requires the two beads that flipped about the line to be of the same colour, while the third bead is free. Hence, there are $n^2$ possible colourings.
        \item $\boxed{rs}$: We note that $rs$ is yet another reflection. Thus a similar argument as $s$ yields $n^2$ arrangements.
        \item $\boxed{r^2s}$: Similar argument as $s$ yields $n^2$ arrangements.
    \end{itemize}
    Note that $|D_3| = 6$, so by Burnside's Lemma,
    \[
        |X/G| = \frac16\left(n^3 + n + n + n^2 + n^2 + n^2\right) = \frac16 n(n+1)(n+2)
    \]
    meaning that the total number of distinct braces of 3 beads with $n$ colours is $\frac16 n(n+1)(n+2)$.
\begin{mdframed}
    Let $p$ be a prime number. Prove that a group of order $p^2$ must be abelian.\newline
    (\textit{Hint: Consider \myref{problem-center-of-G} and \myref{problem-quotient-of-group-mod-center-is-cyclic-implies-abelian}.})
\end{mdframed}
\textbf{Solution}:\newline
 Let $G$ be a group of order $p^2$. We note that $\CenterGrp{G} \leq G$, so by Lagrange's Theorem (\myref{thrm-lagrange}) the order of $\CenterGrp{G}$ must divide the order of $G$, meaning $|\CenterGrp{G}|$ divides $p^2$. Hence $|\CenterGrp{G}|$ is 1, $p$, or $p^2$.

    We note that $|\CenterGrp{G}| \neq 1$ by \myref{example-group-with-prime-power-order-has-non-trivial-center}. If instead $|\CenterGrp{G}| = p$, note
    \[
        |G/\CenterGrp{G}| = \frac{|G|}{|\CenterGrp{G}|} = \frac{p^2}{p} = p
    \]
    so $G/\CenterGrp{G}$ is a group of prime order. Hence by a corollary of Lagrange's Theorem (\myref{corollary-group-with-prime-order-is-cyclic}), $G/\CenterGrp{G}$ is cyclic. But by \myref{problem-quotient-of-group-mod-center-is-cyclic-implies-abelian}, $G = \CenterGrp{G}$, meaning $p^2 = |G| = |\CenterGrp{G}| = p$, a contradiction.

    Hence $|\CenterGrp{G}| = p^2$. Since $\CenterGrp{G} \leq G$ and $|G| = |\CenterGrp{G}| = p^2$, therefore $G = \CenterGrp{G}$, meaning $G$ is abelian by \myref{problem-center-of-G}.
\begin{mdframed}
    Let $G$ be a finite $p$-group, and $X$ be a $G$-set. Denote the set of points of $X$ that are fixed under the action of $G$ by $\Omega$. Prove that $|X| \equiv |\Omega| \pmod p$.\newline
    (\textit{Hint: $\Omega = \{x \in X \vert g\cdot x = x \textrm{ for all } g \in G\}$.})
\end{mdframed}
\textbf{Solution}:\newline
 We first look at elements inside $\Omega$. Suppose $x \in \Omega$. Then $g \cdot x = x$ for any $g \in G$. Recall that $\Orb{G}{x} = \{y \in X \vert g \cdot x = y \text{ for some } x \in X\}$. Hence $x \in \Omega$ if and only if $\Orb{G}{x} = \{x\}$, which means $|\Orb{G}{x}| = 1$.

    Now consider $x \notin \Omega$, meaning $|\Orb{G}{x}| \neq 1$. Recall $|G| = p^n$ for some $n \geq 1$ and prime $p$. By Orbit-Stabilizer theorem (\myref{thrm-orbit-stabilizer}), one obtains
    \[
        |\Stab{G}{x}| = \frac{|G|}{|\Orb{G}{x}|} = \frac{p^n}{|\Orb{G}{x}|}.
    \]
    Since $|\Stab{G}{x}|$ is an integer, thus $\frac{p^n}{|\Orb{G}{x}|}$ must be an integer, meaning $|\Orb{G}{x}|$ divides $p^n$. Therefore if $x \notin \Omega$ then $|\Orb{G}{x}| \equiv 0 \pmod p$.

    Finally, recall that by \myref{exercise-distinct-orbits-partition-set} distinct orbits partition $X$. Hence the number of elements in $X$ is the sum of the number of elements in each of the distinct orbits of $X$. Now for each orbit $\Orb{G}{x}$ where $x \notin \Omega$, the number of elements in it is a multiple of $p$, while for $x \in \Omega$ there is only one element in its orbit. Hence, $|X| \equiv |\Omega| \pmod p$ since there are $|\Omega|$ orbits with only one element.

\chapter{Sylow Theorems}
\section*{Exercises}
\begin{mdframed}
    Find the Sylow 2-subgroup of $\Z_{12}$.
\end{mdframed}
\textbf{Solution}:\newline
 We note that $12 = 2^2 \times 3$. Thus a Sylow 2-subgroup must have order 4. Clearly $|3| = 4$ so $\langle 3 \rangle = \{0, 3, 6, 9\}$ is the Sylow 2-subgroup of $\Z_{12}$.
\begin{mdframed}
    Find all primes $p$ such that $\Syl{p}{\Sn{5}}$ is non-empty.
\end{mdframed}
\textbf{Solution}:\newline
 Recall that $|\Sn{5}| = 120 = 2^3 \times 3 \times 5$. By a corollary of the First Sylow Theorem (\myref{corollary-sylow-p-subgroup-exists}), $\Syl{p}{G} \neq \emptyset$ if $p$ is 2, 3, or 5.
\begin{mdframed}
    Let $G$ be a group and $H \leq G$. Prove that $gHg^{-1} \cong H$ for any $g \in G$.
\end{mdframed}
\textbf{Solution}:\newline
 We prove this by constructing the map $\phi: H \to gHg^{-1}$ where $h \mapsto ghg^{-1}$. We note that $\phi$ is an isomorphism.
    \begin{itemize}
        \item \textbf{Homomorphism}: Let $x, y \in H$. Then
        \[
            \phi(xy) = g(xy)g^{-1} = (gxg^{-1})(gyg^{-1}) = \phi(x)\phi(y)
        \]
        which clearly means that $\phi$ is an isomorphism.
        \item \textbf{Injective}: Suppose $x, y \in H$ such that $\phi(x) = \phi(y)$. Then $gxg^{-1} = gyg^{-1}$ which quickly implies $x = y$ by cancellation law.
        \item \textbf{Surjective}: Suppose $ghg^{-1} \in gHg^{-1}$. Clearly we have $\phi(h) = ghg^{-1}$, so any element in $gHg^{-1}$ has a pre-image inside $H$.
    \end{itemize}
    Hence $H \cong gHg^{-1}$.
\begin{mdframed}
    Let $G$ be a group. Prove that $|gh| = |hg|$ for any $g, h \in G$.
\end{mdframed}
\textbf{Solution}:\newline
 By \myref{prop-order-of-conjugate-element-equals-order-of-element} we know that $|xyx^{-1}| = |y|$ for all $x, y \in G$. Substituting $x = g$, and $y = hg$ yields
    \[
        |xyx^{-1}| = |gh gg^{-1}| = |gh| \text{ and } |y| = |hg|
    \]
    so the result follows.
\begin{mdframed}
    Let $G$ be a group and $S$ be a subset of $G$. Prove that $\N{G}{S} \leq G$.
\end{mdframed}
\textbf{Solution}:\newline
 Clearly $e \in \N{G}{S}$ since $eSe^{-1} = S$. Consider $x, y \in \N{G}{S}$, meaning that $xSx^{-1} = S$ and $ySy^{-1} = S$. Note that $y^{-1} \in \N{G}{S}$ since
    \begin{align*}
        y^{-1}S\left(y^{-1}\right)^{-1} &= y^{-1}Sy\\
        &= y^{-1}\left(ySy^{-1}\right)y & (y \in \N{G}{S})\\
        &= (y^{-1}y)S(y^{-1}y)\\
        &= S.
    \end{align*}
    Therefore
    \begin{align*}
        \left(xy^{-1}\right)S\left(xy^{-1}\right)^{-1} &= \left(xy^{-1}\right)S\left(yx^{-1}\right)\\
        &= x\left(y^{-1}Sy\right)x^{-1}\\
        &= xSx^{-1} & (y^{-1} \in \N{G}{S})\\
        &= S & (x \in \N{G}{S})
    \end{align*}
    which means that $xy^{-1} \in \N{G}{S}$. Hence, by the subgroup test, we have $\N{G}{S} \leq G$.
\begin{mdframed}
    Let $G$ be a finite group, $p$ be a prime number, and $H$ and $K$ be distinct Sylow $p$-subgroups of $G$. Prove that $H \cong K$.
\end{mdframed}
\textbf{Solution}:\newline
 By the Second Sylow Theorem (\myref{thrm-sylow-2}), we know that $gHg^{-1} = K$. Since $H \cong gHg^{-1}$ by \myref{exercise-conjugate-subgroup-isomorphic-to-subgroup} thus $H \cong gHg^{-1} = K$ as required.
\begin{mdframed}
    Let $G$ be a group of order 784, and let $P$ be a Sylow 7-subgroup that is \textbf{not} a normal subgroup of $G$. Find the order of $\N{G}{P}$.
\end{mdframed}
\textbf{Solution}:\newline
 We note $784 = 2^4 \times 7^2$, so $m = 16$, $p = 7$, and $k = 2$. By the Third Sylow Theorem (\myref{thrm-sylow-3}), we know that
    \begin{itemize}
        \item $n_7 = [G : \N{G}{P}] = \frac{|G|}{|\N{G}{P}|}$;
        \item $n_7 \mid 16$, which implies $n_7 \in \{1, 2, 4, 8, 16\}$; and
        \item $n_7 \equiv 1 \pmod 7$, which implies $n_7 \in \{1, 8, 15, 22, \dots\}$.
    \end{itemize}
    Hence $n_7 = 1$ or $n_7 = 8$. But since $P$ is not a normal subgroup of $G$, by \myref{corollary-sylow-subgroup-is-normal-if-it-is-unique}, $P$ cannot be the only Sylow 7-subgroup, meaning $n_7 \neq 1$. Hence $n_7 = 8$, so
    \[
        8 = n_7 = \frac{|G|}{|\N{G}{P}|} = \frac{784}{|\N{G}{P}|}
    \]
    which means that $|\N{G}{P}| = 98$.
\begin{mdframed}
    Show that any group of order $130$ is non-simple.
\end{mdframed}
\textbf{Solution}:\newline
 Note $130 = 2 \times 5 \times 13$. Consider the number of Sylow 13-subgroups, $n_{13}$. The Third Sylow Theorem (\myref{thrm-sylow-3}) tells us that
    \begin{itemize}
        \item $n_{13} \mid 2 \times 5 = 10$, so $n_{13} \in \{1, 2, 5, 10\}$, and
        \item $n_{13} \equiv 1 \pmod{13}$ so $n_{13} \in \{1, 14, 27, \dots\}$.
    \end{itemize}
    Hence $n_{13} = 1$. But by \myref{corollary-sylow-subgroup-is-normal-if-it-is-unique} this means that the only Sylow 13-subgroup is normal. Hence a group of order 130 is non-simple.

\section*{Problems}
\begin{mdframed}
    Show that a group of order 200 has a normal Sylow 5-subgroup.
\end{mdframed}
\textbf{Solution}:\newline
 Note $200 = 2^3 \times 5^2$. Note that for $p = 5$ we have $m = 8$ and the factors of 8 are 1, 2, 4, and 8. Furthermore by the Third Sylow Theorem (\myref{thrm-sylow-3}) we must have $n_5 \equiv 1 \pmod 5$. Hence $n_5 = 1$. By a corollary of the Second Sylow Theorem (\myref{corollary-sylow-subgroup-is-normal-if-it-is-unique}) this means that the only Sylow 5-subgroup is normal.
\begin{mdframed}
    Show that any Sylow $p$-subgroup of a group of order 33 must be normal.
\end{mdframed}
\textbf{Solution}:\newline
 Note $33 = 3 \times 11$,
    \begin{itemize}
        \item when $p = 3$ we have $m = 11$ and the factors of 11 are 1 and 11; and
        \item when $p = 11$ we have $m = 3$ and the factors of 3 are 1 and 3.
    \end{itemize}
    The Third Sylow Theorem (\myref{thrm-sylow-3}) tells us that $n_p \equiv 1 \pmod p$. Hence we must have $n_3 = n_{11} = 1$. A corollary of the Second Sylow Theorem (\myref{corollary-sylow-subgroup-is-normal-if-it-is-unique}) tells us that the only Sylow 3-subgroup and Sylow 11-subgroup are normal.
\begin{mdframed}
    A perfect number is a positive integer that is equal to the sum of its positive divisors, excluding the number itself. All even perfect numbers are of the form $2^{p-1}\left(2^p-1\right)$ where both $p$ and $2^p-1$ are primes. Prove that any group with an even perfect number order is not simple.
\end{mdframed}
\textbf{Solution}:\newline
 For brevity let $q = 2^p - 1$, and we are given that $q$ is a prime. By the Third Sylow Theorem (\myref{thrm-sylow-3}), $n_q \mid 2^{p-1}$ and $n_q \equiv 1 \pmod p$. The factors of $2^{p-1}$ are $1, 2, 4, 8, \dots, 2^{p-1}$. We note $2^{p-1} < 2^p - 1 = q$ for any prime $p$ since
    \[
        2^{p-1} + 1 < 2^{p-1} + 2^{p-1} = 2(2^{p-1}) = 2^p
    \]
    which result immediately follows by subtracting 1 on both sides. Hence, the only possible value that satisfies both conditions is $n_q = 1$. By a corollary of the Second Sylow Theorem (\myref{corollary-sylow-subgroup-is-normal-if-it-is-unique}) this means that the only Sylow $q$-subgroup is normal, hence showing that a group with an even perfect number order is non-simple.
\begin{mdframed}
    Let $p$ and $q$ be primes such that $p < q$. Let $G$ be a group of order $pq$.
    \begin{partquestions}{\roman*}
        \item Prove that there is only one subgroup $H$ of $G$ of order $q$. Deduce that $H \lhd G$.
        \item Prove also that if $q \not\equiv 1 \pmod p$ then $G$ is cyclic.
    \end{partquestions}
\end{mdframed}
\textbf{Solution}:\newline
 \begin{partquestions}{\roman*}
        \item The divisors of $p$ are 1 and $p$ itself. By the Third Sylow Theorem (\myref{thrm-sylow-3}), $n_q$ divides $p$ and $n_q \equiv 1 \pmod q$. Since $p < q$ hence $p \not\equiv 1 \pmod q$ meaning that $n_q = 1$. By a corollary of the Second Sylow Theorem (\myref{corollary-sylow-subgroup-is-normal-if-it-is-unique}) the only Sylow $q$-subgroup is normal.

        \item The divisors of $q$ are 1 and $q$ itself. By the Third Sylow Theorem (\myref{thrm-sylow-3}), $n_p$ divides $q$ and $n_p \equiv 1 \pmod p$. Since $q \not\equiv 1 \pmod p$ by assumption, we must have $n_p = 1$.

        Recall that the order of an element in a group of order $pq$ must divide $pq$ (\myref{corollary-order-of-group-multiple-of-order-of-element}). Hence the possible orders of an element in such a group are 1, $p$, $q$, or $pq$.
        \begin{itemize}
            \item There is only one element of order 1, the identity.
            \item There are $p - 1$ elements of order $p$, all belonging in the single Sylow $p$-subgroup. Note that we subtract 1 because one element in the Sylow $p$-subgroup is the identity.
            \item There are $q - 1$ elements of order $q$, all in the single Sylow $q$-subgroup.
        \end{itemize}
        Hence, since the total number of elements in a group of order $pq$ is $pq$, the number of elements of order $pq$ is
        \begin{align*}
            pq - \left((p-1)+(q-1)+1\right) &= pq - (p+q - 1)\\
            &= pq - p - q + 1\\
            &> 2q - 2 - q + 1\\
            &= 2q - q - 1\\
            &= q - 1\\
            &> 0
        \end{align*}
        which means that there is at least one element of order $pq$. By \myref{thrm-cyclic-group-has-element-with-same-order} this means that such a group is cyclic.
    \end{partquestions}
\begin{mdframed}
    Let $G$ be a finite group, and write the order of $G$ as $p^km$ where $k \geq 0$ and $p \nmid m$. Let $N \lhd G$ such that $p$ does not divide the index of $N$ in $G$.
    \begin{partquestions}{\roman*}
        \item Prove that any Sylow $p$-subgroup of $N$ is also in $G$.
        \item Prove that any Sylow $p$-subgroup of $G$ is also in $N$.
    \end{partquestions}
    (That is, prove that $N$ contains all Sylow $p$-subgroups of $G$ and vice versa.)
\end{mdframed}
\textbf{Solution}:\newline
 \begin{partquestions}{\roman*}
        \item Let $P$ be a Sylow $p$-subgroup of $N$. Lagrange's Theorem (\myref{thrm-lagrange}) tells us that $|G| = [G:N]|N|$. Since $p$ does not divide $[G:N]$ we must have $|N| = p^ka$ where $a$ divides $m$. Hence $|P| = p^k$ as $P$ is a Sylow $p$-subgroup of $N$. Since $P$ has order $p^k$ and $P \leq N \leq G$, thus $P$ is also a Sylow $p$-subgroup of $G$.
        \item Let $Q$ be a Sylow $p$-subgroup of $G$. The Second Sylow Theorem (\myref{thrm-sylow-2}) tells us there exist  a $g \in G$ such that $Q = gPg^{-1}$. Recall by definition of normality that $gNg^{-1} = N$ for any $g \in G$. Note also that $P \leq N$. Hence,
        \[
            Q = gPg^{-1} \leq gNg^{-1} = N
        \]
        which means that $Q$ is also a Sylow $p$-subgroup of $N$.
    \end{partquestions}
\begin{mdframed}
    Show that any group of order 3325 is abelian.
\end{mdframed}
\textbf{Solution}:\newline
 We note $3325 = 5^2 \times 7 \times 19$. Let the group of order 3325 be $G$. We know that
    \begin{itemize}
        \item for $p = 5$ we have $m = 7 \times 19 = 133$ and so the possible divisors of $m$ are $\{1, 7, 19, 133\}$;
        \item for $p = 7$ we have $m = 5^2 \times 19 = 475$ and so the possible divisors of $m$ are $\{1, 5, 19, 25, 95, 475\}$; and
        \item for $p = 19$ we have $m = 5^2 \times 7 = 175$ and so the possible divisors of $m$ are $\{1, 5, 7, 25, 35, 175\}$.
    \end{itemize}
    The Third Sylow Theorem (\myref{thrm-sylow-3}) tells us that $n_p \equiv 1 \pmod p$. Thus $n_5 = n_7 = n_{19} = 1$. Let $P$, $Q$, and $R$ be the Sylow 5-subgroup, the Sylow 7-subgroup, and the Sylow 19-subgroup respectively. We note that $P$, $Q$, and $R$ are all normal subgroups of $G$ by \myref{corollary-sylow-subgroup-is-normal-if-it-is-unique}.

    Denote the group $QR$ by $H$. Since $Q$ and $R$ are of prime order, their intersection is the identity (\myref{problem-intersection-of-coprime-subgroups}). Furthermore, as $Q$ and $R$ are normal subgroups of $G$, thus they commute by \myref{problem-intersection-of-coprime-subgroups}. Therefore $H$ is the internal direct product of $Q$ and $R$, meaning $H \cong Q \times R$ by \myref{thrm-direct-product-equivalence}. Hence $|H| = |Q||R| = 7 \times 19 = 133$. Now because as $Q$ and $R$ are of prime order, thus $Q$ and $R$ are abelian and so is $H$. Hence $H$ is an abelian group of order 133.

    Now consider the group $PH$. Since 5 and 133 are coprime, thus $P \cap H = \{e\}$. In addition, since $P \lhd G$ thus $PH \leq G$ by Diamond Isomorphism Theorem (\myref{thrm-isomorphism-2}), statement 3. Also,
    \[
        |PH| = \frac{|P||H|}{|P \cap H|} = |P||H| = 5^2 \times 133 = 3325 = |G|
    \]
    which means that $G = PH$. Since $P \lhd G$, thus $ph = hp$ for any element $h \in H$, meaning elements in $P$ and $H$ commute. Hence, $G$ is the internal direct product of $P$ and $H$, meaning $G \cong P \times H$. As the external direct product of two abelian groups is also abelian (\myref{problem-external-direct-product-of-abelian-groups-is-abelian}) thus $G$ is abelian.
\begin{mdframed}
    Let $G$ be a finite group such that $|G| = p^km$ where $k \geq 1$, $m > 1$, and $p \nmid m$. Prove that if $m! < |G|$ then $G$ is non-simple.
\end{mdframed}
\textbf{Solution}:\newline
 Let $P$ be a Sylow $p$-subgroup of $G$. We note that $|G/P| = \frac{p^km}{p^k} = m$. Let $G$ act on the set of cosets $G/P$ by left multiplication, meaning $g\cdot (xP) = (gx)P$. We know by \myref{thrm-group-action-definition-equivalence} that this induces a homomorphism $\phi: G \to \Sn{m}$ where $\phi(g) = \sigma_g$ such that $\sigma_g(xP) = g\cdot (xP) = (gx)P$. By \myref{example-using-kernel-to-show-non-simple}, $\ker\phi = \bigcap_{x \in G}xPx^{-1}$.

    We note $\ker\phi \neq \{e\}$ since otherwise it would imply that $\phi$ is injective (\myref{exercise-trivial-kernel-means-injective}), which is impossible as that would mean $p^km = |G| \leq |\Sn{m}| = m!$ which is a contradiction. Also $\ker\phi \neq G$ as otherwise
    \[
        p^km = |G| = |\ker\phi| = \left|\bigcap_{x \in G} xPx^{-1}\right| \leq |xPx^{-1}| = |P| = p^k,
    \]
    which would mean $m = 1$, a contradiction. Hence $\ker\phi$ is a non-trivial proper subgroup of $G$. We note that $\ker\phi \lhd G$, so we have found a non-trivial proper normal subgroup of $G$, meaning that $G$ is non-simple.
\begin{mdframed}
    Prove that a group of order 30 has a normal subgroup of order 5.
\end{mdframed}
\textbf{Solution}:\newline
 Let $G$ be a group of order 30. Note $30 = 2 \times 3 \times 5$, and consider $n_5$. The Third Sylow Theorem (\myref{thrm-sylow-3}) tells us that
    \begin{itemize}
        \item $6 \vert n_5$, so $n_5 \in \{1, 2, 3, 6\}$; and
        \item $n_5 \equiv 1 \pmod 5$, so $n_5 \in \{1, 6, 11, 16, \dots\}$.
    \end{itemize}
    Hence $n_5$ is 1 or 6. Seeking a contradiction, assume $n_5 = 6$, and let $P_5$ be a Sylow 5-subgroup.

    Since $|P_5| = 5$, which is prime, each non-identity element of $P_5$ is a generator. Hence, no two Sylow 5-subgroups can share any non-identity elements (otherwise they will be the same group), thereby meaning any two Sylow 5-subgroups intersect in the identity only. Thus there exists $6(5-1) = 24$ elements of order 5, meaning there must be 6 elements of order not equal to 5.

    Now consider $n_3$. Note by the Third Sylow Theorem again,
    \begin{itemize}
        \item $10 \vert n_3$, so $n_3 \in \{1, 2, 5, 10\}$; and
        \item $n_3 \equiv 1 \pmod 3$, so $n_3 \in \{1, 4, 7, 10, 13, \dots\}$.
    \end{itemize}
    Thus $n_3$ is 1 or 10. Now if $n_3 = 10$ then there must be $10(3-1) = 20$ elements of order 3, a contradiction to the fact there exists only 6 elements with order not 5. Hence $n_3 = 1$, meaning the only Sylow 3-subgroup (call it $P_3$) is normal in $G$.

    As $P_5 \leq G$ and $P_3 \lhd G$, by the Diamond Isomorphism Theorem (\myref{thrm-isomorphism-2}), statement 3, we have $P_5P_3 \leq G$. Note $P_5 \cap P_3 = \{e\}$ by \myref{problem-intersection-of-coprime-subgroups}. So \myref{exercise-order-of-subgroup-product} tells us
    \[
        |P_5P_3| = \frac{|P_5||P_3|}{|P_5 \cap P_3|} = \frac{5\times3}{1} = 15.
    \]
    One sees that $[G:P_5P_3] = \frac{30}{15} = 2$, so $P_5P_3 \lhd G$ by \myref{problem-subgroup-of-index-2}.

    Now \myref{problem-group-of-order-pq-has-normal-subgroup-of-order-q} tells us there exists a unique $H \lhd P_5P_3$ with $|H| = 5$. But since $P_5P_3 \lhd G$, \myref{problem-normal-subgroup-of-G-contains-all-sylow-p-subgroups} tells us that $P_5P_3$ contains all Sylow 5-subgroups of $G$, meaning $G$ has only 1 Sylow 5-subgroup, i.e. $n_5 = 1$, a contradiction to our assumption that $n_5 = 6$.

    Hence $n_5 = 1$. Therefore the unique Sylow 5-subgroup is a normal subgroup of $G$ by \myref{corollary-sylow-subgroup-is-normal-if-it-is-unique}.
\begin{mdframed}
    Let $p, q$, and $r$ be distinct primes such that $p < q < r$. Let $G$ be a group of order $pqr$. Prove that $G$ is non-simple.\newline
    (\textit{Hint: show that a normal subgroup of order $p$, $q$, or $r$ must exist.})
\end{mdframed}
\textbf{Solution}:\newline
 We prove that $G$ has a normal subgroup of order $p$, $q$, or $r$. By \myref{corollary-sylow-subgroup-is-normal-if-it-is-unique}, subgroups of order $p$, $q$, or $r$ are normal if they are unique. By way of contradiction, assume that they are not unique, meaning $n_p, n_q, n_r > 1$.

    By the Third Sylow Theorem (\myref{thrm-sylow-3}), $n_r \equiv 1 \pmod r$ and $n_r \mid pq$. The divisors of $pq$ are 1, $p$, $q$, and $pq$. We note that since both $p$ and $q$ are less than $r$, thus $p \not\equiv 1 \pmod r$ and $q \not\equiv 1 \pmod r$. The only possibility that is left is $n_r = pq$ as we assume $n_r \neq 1$. Similarly, $n_q \equiv 1 \pmod q$ and $n_q \mid pr$. The divisors of $pr$ are 1, $p$, $r$, and $pr$. Since $p < q$ thus $p \not\equiv 1 \pmod q$. Hence $n_q \geq r$ as we assume $n_q \neq 1$. Similarly, $n_p \geq q$.

    We now consider the number of elements with order $p$, $q$, and $r$.
    \begin{itemize}
        \item $\boxed{p}$ With $n_p \geq q$, there are at least $q(p-1)$ elements of order $p$. We minus 1 because one of the elements in a Sylow $p$-subgroup is the identity with order 1.
        \item $\boxed{q}$ With $n_q \geq r$, there are at least $r(q-1)$ elements of order $q$.
        \item $\boxed{r}$ We know $n_r = pq$ so there are exactly $pq(r-1)$ elements of order $r$.
    \end{itemize}
    Since the total number of elements, $pqr$, must be at least the sum of the numbers of these elements, thus
    \begin{align*}
        pqr &\geq q(p-1) + r(q-1) + pq(r-1)\\
        &= pq - q + qr - r + pqr - pq\\
        &= pqr + qr - q - r
    \end{align*}
    which means $qr - q - r \leq 0$. Rearranging, we see
    \[
        q \leq \frac{r}{r-1} = 1 + \frac{1}{r-1}.
    \]
    Since $p < q$ and they are both primes, we must have $q \geq 3$. Hence one sees
    \[
        3 \leq q \leq 1 + \frac{1}{r-1} \leq 2
    \]
    which is a clear contradiction. Hence, at least one of $n_p$, $n_q$, or $n_r$ is 1, meaning that there exists a non-trivial proper normal subgroup in $G$ by \myref{corollary-sylow-subgroup-is-normal-if-it-is-unique}. Therefore $G$ is non-simple.

\chapter{Abelian Groups}
\section*{Exercises}
\begin{mdframed}
    Classify all abelian groups of order 100.
\end{mdframed}
\textbf{Solution}:\newline
 Note $100 = 2^2 \times 5^2$, so the only 4 possibilities are
    \begin{itemize}
        \item $\Cn{2} \times \Cn{2} \times \Cn{5} \times \Cn{5}$;
        \item $\Cn{2} \times \Cn{2} \times \Cn{25}$;
        \item $\Cn{4} \times \Cn{5} \times \Cn{5}$; and
        \item $\Cn{4} \times \Cn{25}$,
    \end{itemize}
    by the Fundamental Theorem of Finite Abelian Groups (\myref{thrm-fundamental-theorem-of-finite-abelian-groups}).
\begin{mdframed}
    Prove \myref{lemma-fundamental-theorem-of-finite-abelian-groups-3}.
\end{mdframed}
\textbf{Solution}:\newline
 \begin{partquestions}{\roman*}
        \item Note that $e \in G^n$ since $e = e^n$.

        Now let $a, b \in G^n$, which means $a = x^n$ and $b = y^n$ for some $x, y \in G$. Then note
        \begin{align*}
            ab^{-1} &= (x^n)(y^n)^{-1}\\
            &= x^n(y^{-1})^n\\
            &= (xy^{-1})^n & (\text{rewriting possible as }G \text{ is abelian})\\
            &\in G^n.
        \end{align*}

        Thus $G^n \leq G$ by subgroup test.

        \item By Cauchy's Theorem (\myref{thrm-cauchy}) tells us that an element of order $p$ must exist within $G$. Let this element be $a$. If $G = G^p$ then $\phi$ must be at least injective. But note $\phi(a) = a^p = e = \phi(e)$ and $a \neq e$, so $\phi$ is not injective and hence $G \neq G^p$. Therefore $G^p < G$.
    \end{partquestions}
\begin{mdframed}
    Let $G$ be an abelian group.
    \begin{partquestions}{\roman*}
        \item Prove that $G^n = \{x^n \vert x \in G\}$, where $n$ is a positive integer, is a subgroup of $G$.
        \item Suppose also that $G$ is finite and $p$ is a prime that divides $|G|$. Using the map $\phi: G \to G^p$ where $\phi(x) = x^p$, prove that $G^p$ is a proper subgroup of $G$.\newline
        (\textit{Hint: consider Cauchy's Theorem (\myref{thrm-cauchy}).})
    \end{partquestions}
\end{mdframed}
\textbf{Solution}:\newline
 Let a finite abelian group of prime-power order have order $p^n$ where $p$ is prime and $n$ is a non-negative integer. We use strong induction on $n$.

    When $n = 0$ then $G$ is just the trivial group, which is itself cyclic.

    Assume that the lemma holds for all finite abelian groups of order $p^r$ where $0 \leq r \leq k$ for some positive integer $k$. We show that a finite abelian $p$-group of order $p^{k+1}$ is also an internal direct product of cyclic groups.

    Let $G$ be a finite abelian $p$-group of order $p^{k+1}$. Let $g$ be an element of maximal order in $G$. If $\langle g \rangle = G$ then we are done; otherwise \myref{lemma-fundamental-theorem-of-finite-abelian-groups-2} tells us that $G \cong \langle g \rangle \times H$ for some subgroup $H$ in $G$. Note that $|H| < |G|$ as otherwise $g = e$ which clearly does not have maximal order in $G$. Therefore we may use the Induction Hypothesis on $H$ to write it as an internal direct product of cyclic groups; thus $G$ itself is an internal direct product of cyclic groups.

    Therefore the lemma is proven by mathematical induction.

\section*{Problems}
\begin{mdframed}
    Suppose $p_1, p_2, \dots, p_n$ are distinct primes. How many distinct abelian groups of order $p_1p_2\cdots p_n$ are there, up to isomorphism?
\end{mdframed}
\textbf{Solution}:\newline
 There is only one way, namely $\Cn{p_1}\times\Cn{p_2}\times\cdots\times\Cn{p_n}$.
\begin{mdframed}
    What is the smallest positive integer $n$ such that there exist two non-isomorphic abelian groups of order $n$? Write down the isomorphism class of these two groups.
\end{mdframed}
\textbf{Solution}:\newline
 The smallest $n$ is 4, since anything smaller is just a cyclic group of prime order (or the trivial group). The two groups required are $\Cn{4}$ and $\Cn{2} \times \Cn{2}$ (which is actually isomorphic to $D_2$, the dihedral group of degree 2).
\begin{mdframed}
    Write out all isomorphism classes of an abelian group of order $p^4$, where $p$ is a prime.
\end{mdframed}
\textbf{Solution}:\newline
 First note that we can write 4 as a sum in 5 distinct ways:
    \begin{itemize}
        \item 4;
        \item 3 + 1;
        \item 2 + 2;
        \item 2 + 1 + 1; and
        \item 1 + 1 + 1 + 1.
    \end{itemize}
    Consequently the distinct isomorphism classes for an abelian group of order $p^4$ are
    \begin{itemize}
        \item $\Cn{p^4}$;
        \item $\Cn{p^3}\times\Cn{p}$;
        \item $\Cn{p^2}\times\Cn{p^2}$;
        \item $\Cn{p^2}\times\Cn{p}\times\Cn{p}$; and
        \item $\Cn{p}\times\Cn{p}\times\Cn{p}\times\Cn{p}$.
    \end{itemize}
\begin{mdframed}
    Let $G$ be an abelian group of order 16. Suppose there exist elements $a$ and $b$ in $G$ such that $|a| = |b| = 4$ and $a^2 \neq b^2$. Determine the isomorphism class of $G$.
\end{mdframed}
\textbf{Solution}:\newline
 Immediately we may conclude that $a \neq b$, since otherwise $a^2 = b^2$. Because $|a| = 4$ and $|b| = 4$, we know that $G$ contains two distinct subgroups of order 4, say $A$ and $B$. So ``$\Cn{4} \times \Cn{4}$'' must be a part of the isomorphism class of $G$. But $\Cn{4} \times \Cn{4}$ has an order of 16. Therefore the isomorphism class of $G$ must be $\Cn{4} \times \Cn{4}$.
\begin{mdframed}
    How many abelian groups of order 16 have the property that $x^4 = e$ for all $x$ in the group, up to isomorphism?
\end{mdframed}
\textbf{Solution}:\newline
 Since $x^4 = e$, therefore there cannot be an element of order 8 or order 16 inside the group. So the maximum order of a `component' of the isomorphism class is 4. Consequently, the only 3 choices are
    \begin{itemize}
        \item $\Cn{4}\times\Cn{4}$;
        \item $\Cn{4}\times\Cn{2}\times\Cn{2}$; and
        \item $\Cn{2}\times\Cn{2}\times\Cn{2}\times\Cn{2}$.
    \end{itemize}

\chapter{Composition Series}
\section*{Exercises}
\begin{mdframed}
    Let the group $G = \Z_4$.
    \begin{partquestions}{\roman*}
        \item Find a subnormal series of length 2 for $G$.
        \item Hence find the factor groups of that subnormal series.
        \item Explain whether the subnormal series found in \textbf{(i)} is a normal series.
    \end{partquestions}
\end{mdframed}
\textbf{Solution}:\newline
 \begin{partquestions}{\roman*}
        \item One sees clearly that $\{0, 2\}$ is the only non-trivial proper normal subgroup of $G$, so the subnormal series of length 2 is $1 \lhd \{0, 2\} \lhd G$.
        \item There are 2 factor groups of the above subnormal series. The first is $\{0, 2\} / 1 = \{0, 2\} \cong \Z_2$ and the second is
        \begin{align*}
            G / \{0, 2\} &= \{g \oplus_4 \{0, 2\} \vert g \in G\}\\
            &= \{\{0, 2\}, \{1, 3\}, \{2, 0\}, \{3, 1\}\}\\
            &= \{\{0, 2\}, \{1, 3\}\}\\
            &= \langle \{1, 3\} \rangle\\
            &\cong \Z_2.
        \end{align*}
        \item Since $1 \lhd G$ and $\{0, 2\} \lhd G$ thus the subnormal series in \textbf{(i)} is also a normal series of $G$.
    \end{partquestions}
\begin{mdframed}
    Find the order of the maximal normal subgroup of $\Z_{120}$.
\end{mdframed}
\textbf{Solution}:\newline
 By Lagrange's Theorem (\myref{thrm-lagrange}) the order of a subgroup must divide the order of the group. Furthermore $\Z_{120}$ is abelian, so any subgroup of it is normal. Now the subgroup $N = \{0, 2, 4, \dots, 118\}$ has 60 elements which is the maximum possible guaranteed by Lagrange. Hence $N$ is the maximal normal subgroup of $\Z_{120}$, which has order 60.
\begin{mdframed}
    Find the two composition series for $\Cn{6}$ (up to isomorphism), their composition lengths, and their composition factors (up to isomorphism).
\end{mdframed}
\textbf{Solution}:\newline
 $\Cn{6}$ has these two composition series up to isomorphism
    \begin{align*}
        &1 \lhd \Cn{2} \lhd \Cn{6} \text{ and }\\
        &1 \lhd \Cn{3} \lhd \Cn{6}.
    \end{align*}
    In both cases, their composition length is 2. Their respective composition factors are:
    \begin{itemize}
        \item $\Cn{2} / 1 \cong \Cn{2}$ and $\Cn{6} / \Cn{2} \cong \Cn{3}$ by \myref{exercise-Zmn-mod-Zn-cong-Zn}; and
        \item $\Cn{3} / 1 \cong \Cn{3}$ and $\Cn{6} / \Cn{3} \cong \Cn{2}$ by \myref{exercise-Zmn-mod-Zn-cong-Zn},
    \end{itemize}
    up to isomorphism.
\begin{mdframed}
    Let $p$ be a prime. Show that any group with order $p^2$ has only one composition series.
\end{mdframed}
\textbf{Solution}:\newline
 Let the group in question be $G$. We know by Cauchy's Theorem (\myref{thrm-cauchy}) and \myref{exercise-group-of-order-multiple-of-prime-has-subgroup-of-prime-order}, and by writing $p^2$ as $p \times p$, that $G$ has a subgroup of order $p$ (call this $H$).

    Lagrange's Theorem (\myref{thrm-lagrange}) tells us that the possible orders of the subgroups of $G$ are 1, $p$, and $p^2$. These subgroups are $\{e\}$, $H$, and $G$ respectively. Furthermore, by \myref{problem-group-of-order-prime-squared-is-abelian}, $G$ must be abelian, thereby its subgroups are all normal (\myref{prop-subgroup-of-abelian-group-is-normal}). Finally, a corollary of Lagrange's Theorem (\myref{corollary-group-with-prime-order-subgroups}) says that the only subgroups of $H$ are the trivial group and the group itself. Hence, $G$ has only one composition series, namely $1 \lhd H \lhd G$.

\section*{Problems}
\begin{mdframed}
    Consider the Klein four-group\index{Klein four-group} $\mathrm{V}$ with presentation
    \[
        \langle a, b \vert a^2 = b^2 = (ab)^2 = e \rangle.
    \]
    Recall that $\mathrm{V} \cong (\Z_2)^2$.
    \begin{partquestions}{\roman*}
        \item Find the unique composition series for $\mathrm{V}$ up to isomorphism.
        \item Let $\mathrm{Q}$ denote the quaternion group. Find the two unique composition series for $\mathrm{Q}$ up to isomorphism.\newline
        (\textit{Hint: consider the alternate definition of the quaternion group.})
        \item Let $H$ be the subgroup of $\mathrm{Q}$ that is isomorphic to $\Cn{4}$, and $K$ be a subgroup of $\mathrm{Q}$ that is isomorphic to $\mathrm{V}$. Prove that
        \[
            \mathrm{Q}/H \cong \mathrm{Q}/K.
        \]
        (\textit{Hint: consider \myref{problem-cartesian-product-of-group-by-group-isomorphic-to-group}.})
    \end{partquestions}
\end{mdframed}
\textbf{Solution}:\newline
 \begin{partquestions}{\roman*}
        \item We note $\mathrm{V}$ has order 4. By writing 4 as $2 \times 2$ we know that $\mathrm{V}$ has a subgroup of order 2 (which is cyclic) by Cauchy's Theorem (\myref{thrm-cauchy}). Now $\mathrm{V}$ is abelian (\myref{problem-group-of-order-prime-squared-is-abelian}) which means that the subgroup of order 2 is normal in $\mathrm{V}$ (\myref{prop-subgroup-of-abelian-group-is-normal}). Finally, the only possible order for a non-trivial proper subgroup of $\mathrm{V}$ is 2 by Lagrange's Theorem (\myref{thrm-lagrange}). Hence, the only composition series for $\mathrm{V}$ is $1 \lhd \Cn{2} \lhd \mathrm{V}$ up to isomorphism.\newline
        (Note that this analysis applies for \textit{any} group of order 4.)

        \item Recall that $\mathrm{Q} = \langle \alpha, \beta \vert \alpha^4 = e, \alpha^2 = \beta^2, \text{ and } \beta\alpha = \alpha^3\beta \rangle$. From the solution of \myref{exercise-normal-subgroups-of-quarternion-group}, the maximal subgroups of $\mathrm{Q}$ are $G_1 = \langle \alpha \rangle$, $G_2 = \langle \beta \rangle$, and $G_3 = \langle \alpha\beta \rangle$ (by setting $\alpha = i$ and $\beta = j$). We note the following.
        \begin{itemize}
            \item $G_1 = \{e, \alpha, \alpha^2, \alpha^3\} \cong \Cn{4}$.
            \item $G_2 = \{e, \beta, \beta^2, \beta^3\} = \{e, \beta, \alpha^2, \alpha^2\beta\} \cong \mathrm{V}$ where $a = \alpha^2$ and $b = \beta$.
            \item $G_3 = \{e, \alpha\beta, (\alpha\beta)^2, (\alpha\beta)^3\} = \{e, \alpha\beta, \alpha^2, \alpha^3\beta\} \cong \mathrm{V}$ with $a = \alpha\beta$ and $b = \alpha^2$.
        \end{itemize}
        Also, note that $\Cn{2} \cong \langle \alpha^2 \rangle \lhd G_1$, $\Cn{2} \cong \langle \beta^2 \rangle \lhd G_2$, $\Cn{2} \cong \langle (\alpha\beta)^2 \rangle \lhd G_3$. Hence, the two series up to isomorphism are
        \begin{align*}
            1 \lhd \Cn{2} \lhd \Cn{4} \lhd \mathrm{Q} & \text{ and }\\
            1 \lhd \Cn{2} \lhd \mathrm{V} \lhd \mathrm{Q}
        \end{align*}

        \item By Jordan-H\"older theorem (\myref{thrm-jordan-holder}), the composition factors are isomorphic to each other. We note
        \begin{itemize}
            \item $\Cn{2} / 1 \cong \Cn{2}$;
            \item $\Cn{4} / \Cn{2} \cong \Cn{2}$ by \myref{exercise-Zmn-mod-Zn-cong-Zn}; and
            \item $\mathrm{V} / \Cn{2} \cong (\Cn{2})^2 / \Cn{2} \cong \Cn{2}$ by \myref{problem-cartesian-product-of-group-by-group-isomorphic-to-group}.
        \end{itemize}
        The only unaccounted set of factors is $\mathrm{Q}/\mathrm{V}$ and $\mathrm{Q}/\Cn{4}$. So, either $\mathrm{Q}/\mathrm{V} \cong \Cn{2}$ and $\mathrm{Q}/\Cn{4} \cong \Cn{2}$, or $\mathrm{Q}/\mathrm{V} \cong \mathrm{Q}/\Cn{4}$. Hence $\mathrm{Q}/H \cong \mathrm{Q}/K$.
    \end{partquestions}
\begin{mdframed}
    Find the unique composition series of $\Sn{4}$ (up to isomorphism).\newline
    (\textit{Hint: consider the orders of elements in $\An{4}$ and \myref{problem-subgroup-of-index-2} to prove why a particular subgroup with a particular order cannot exist.})
\end{mdframed}
\textbf{Solution}:\newline
 We know that $\An{4} \lhd \Sn{4}$ by \myref{prop-An-normal-subgroup-of-Sn}. Note $\An{4}$ is a maximal normal subgroup since $|\An{4}| = \frac{4!}2 = 12$ by \myref{prop-order-of-An}, and a subgroup's order must divide the order of the group by Lagrange's Theorem (\myref{thrm-lagrange}).

    Now applying that theorem on $\An{4}$, we see that the possible orders of a subgroup of $\An{4}$ are 6, 4, 3, 2, and 1. We claim that a subgroup of order 6 does not exist. Note that $\An{4}$ contains
    \begin{itemize}
        \item 1 element of order 1;
        \item 3 elements of order 2; and
        \item 8 elements of order 3.
    \end{itemize}
    If a subgroup of order 6 exists (say, $H$), then its index would be $\frac{12}{6} = 2$ (Lagrange), meaning $H$ contains all odd order elements (\myref{problem-subgroup-of-index-2}). However, there are $1 + 8 = 9$ odd order elements, meaning that $H$ has an order of at least 9, a contradiction. Hence a subgroup of $\An{4}$ of order 6 is impossible.

    Now we note that a subgroup of order $4 = 2^2$ exists by a corollary of the First Sylow Theorem (\myref{corollary-sylow-p-subgroup-exists}) as it is a Sylow 2-subgroup. The Third Sylow Theorem (\myref{thrm-sylow-3}) tells us how many Sylow 2-subgroups there are:
    \begin{itemize}
        \item $n_2 \vert 3$, so $n_2$ is 1 or 3; and
        \item $n_2 \equiv 1 \pmod2$, so $n_2 \in \{1, 3, 5, \dots\}$.
    \end{itemize}
    Hence $n_2$ is either 1 or 3. Now if $n_2 = 3$, then the number of elements of order of 1, 2, or 4 is
    \[
        3 \times (4 - 1) + 1 = 10
    \]
    (where the 3 is $n_2$, the $4-1$ is the number of non-identity elements in each Sylow 2-subgroup, and the $+1$ is to add the identity element). However, as noted above, there are only 4 elements of order 1, 2, or 4, a contradiction. Hence $n_2 = 1$, meaning the Sylow 2-subgroup (which is a subgroup of order 4) is normal (\myref{corollary-sylow-subgroup-is-normal-if-it-is-unique}). Therefore the subgroup of order 4 is the maximal normal subgroup of $\An{4}$.

    We note that the subgroup of order 4 of $\An{4}$ is not $\Cn{4}$ (as this would imply that $\An{4}$ has an element of order 4, which it does not). Hence, from \myref{problem-smallest-nonabelian-group}, the subgroup of order 4 must be isomorphic to the Klein-4 group, $\mathrm{V}$.

    Note that a group of order 4 has a subgroup of order 2 by Cauchy's Theorem (\myref{thrm-cauchy}). Clearly such a subgroup is cyclic (since 2 is prime), and has index $\frac42 = 2$, meaning that it is normal in the group of order 4. Furthermore the trivial group is always a subgroup of any group.

    Hence, the composition series for $\Sn{4}$, up to isomorphism, is
    \[
        1 \lhd \Cn{2} \lhd \mathrm{V} \lhd \An{4} \lhd \Sn{4}.
    \]
    \begin{remark}
        We list the actual subgroups that are isomorphic to the above terms in the composition series here.
        \begin{itemize}
            \item $\Cn{2}$: $\{e, \begin{pmatrix}1&2\end{pmatrix}\begin{pmatrix}3&4\end{pmatrix}\}$
            \item V: $\{e, \begin{pmatrix}1&2\end{pmatrix}\begin{pmatrix}3&4\end{pmatrix}, \begin{pmatrix}1&3\end{pmatrix}\begin{pmatrix}2&4\end{pmatrix}, \begin{pmatrix}1&4\end{pmatrix}\begin{pmatrix}2&3\end{pmatrix}\}$
            \item $\An{4}$ is an actual subgroup of $\Sn{4}$
        \end{itemize}
    \end{remark}

\chapter{Simple Groups}
\section*{Exercises}
\begin{mdframed}
    Prove that a group of order 12 either has a normal subgroup of order 3, or a normal subgroup of order 4, or both.
\end{mdframed}
\textbf{Solution}:\newline
 We find the number of Sylow 2- and Sylow 3-subgroups (denoted $n_2$ and $n_3$ respectively) of the group of order 12 (call it $G$) using the Third Sylow Theorem (\myref{thrm-sylow-3}):
    \begin{itemize}
        \item For $n_2$, note $12 = 2^2 \times 3$. So,
        \begin{itemize}
            \item $3 \vert n_2$ meaning $n_2 \in \{1, 3\}$; and
            \item $n_2 \equiv 1 \pmod 2$ meaning $n_2 \in \{1, 3, 5, \dots\}$.
        \end{itemize}
        Hence $n_2 = 1$ or $n_2 = 3$.

        \item For $n_3$, note $12 = 3 \times 2^2$. So,
        \begin{itemize}
            \item $4 \vert n_3$ meaning $n_3 \in \{1, 2, 4\}$; and
            \item $n_3 \equiv 1 \pmod 3$ meaning $n_3 \in \{1, 4, 7, \dots\}$.
        \end{itemize}
        Hence $n_3 = 1$ or $n_3 = 4$.
    \end{itemize}

    Now by way of contradiction suppose both $n_2$ and $n_3$ are not 1. Thus $n_2 = 3$ and $n_3 = 4$. We consider the number of elements of a certain order.
    \begin{itemize}
        \item Number of elements with order 2 or 4 is $3(4-1) = 9$, since each of the 3 Sylow 2-subgroups has 4 elements, 1 of which is the identity.
        \item Number of elements with order 3 is $4(3-1) = 8$, since each of the 4 Sylow 3-subgroups has 3 elements, 1 of which is the identity.
    \end{itemize}
    Therefore, the number of elements in $G$ must be at least $9 + 8 = 17 > 12$, a contradiction.

    Thus, at least one of $n_2$ and $n_3$ must be 1, meaning that there must exist a normal subgroup of order 4 or 3 (or both) by \myref{corollary-sylow-subgroup-is-normal-if-it-is-unique}.
\begin{mdframed}
    Prove that a group of each of the following orders has a normal subgroup of order 5.
    \begin{partquestions}{\alph*}
        \item 15
        \item 20
    \end{partquestions}
\end{mdframed}
\textbf{Solution}:\newline
 \begin{partquestions}{\alph*}
        \item Note $15 = 3 \times 5$, so \myref{problem-group-of-order-pq-has-normal-subgroup-of-order-q} means that there exists a unique (and hence normal) subgroup of order 5.
        \item Note $20 = 2^2 \times 5$. Then the Third Sylow Theorem (\myref{thrm-sylow-3}) tells us that
        \begin{itemize}
            \item $2^2 \vert n_5$, so $n_5 \in \{1, 2, 4\}$; and
            \item $n_5 \equiv 1 \pmod 5$, so $n_5 \in \{1, 6, 11, 16, \dots\}$.
        \end{itemize}
        Hence $n_5 = 1$, so there exists a unique (and hence normal) subgroup of order 5.
    \end{partquestions}
\begin{mdframed}
    Consider the permutation $\sigma = \begin{pmatrix}1&3&2&4&5\end{pmatrix}$.
    \begin{partquestions}{\roman*}
        \item Explain why $\sigma \in \An{5}$.
        \item Find the order of the subgroup $\langle \sigma \rangle$.
        \item Find another subgroup of $\An{5}$ with order 5.
    \end{partquestions}
\end{mdframed}
\textbf{Solution}:\newline
 \begin{partquestions}{\alph*}
        \item Note that $\sigma = \begin{pmatrix}1&3\end{pmatrix}\begin{pmatrix}2&3\end{pmatrix}\begin{pmatrix}2&4\end{pmatrix}\begin{pmatrix}4&5\end{pmatrix}$, so we see $\sigma$ is an even permutation (\myref{thrm-parity-of-permutation}), and since the highest integer that appears in $\sigma$ is 5, thus $\sigma \in \An5$.

        \item Observe that
        \begin{itemize}
            \item $\sigma^0 = \id$;
            \item $\sigma^1 = \sigma \neq \id$;
            \item $\sigma^2 = \begin{pmatrix}1&2&5&3&4\end{pmatrix} \neq \id$;
            \item $\sigma^3 = \begin{pmatrix}1&4&3&5&2\end{pmatrix} \neq \id$;
            \item $\sigma^4 = \begin{pmatrix}1&5&4&2&3\end{pmatrix} \neq \id$; and
            \item $\sigma^5 = \id$.
        \end{itemize}
        Hence the order of $\langle \sigma \rangle$ is 5 since that cyclic subgroup contains 5 distinct elements.

        \item Consider the other permutation $\pi = \begin{pmatrix}1&2&3&4&5\end{pmatrix}$. We note
        \[
            \pi = \begin{pmatrix}1&2\end{pmatrix}\begin{pmatrix}2&3\end{pmatrix}\begin{pmatrix}3&4\end{pmatrix}\begin{pmatrix}4&5\end{pmatrix} \in \An5
        \]
        Also,
        \begin{itemize}
            \item $\pi^0 = \id$;
            \item $\pi^1 = \pi \neq \id$;
            \item $\pi^2 = \begin{pmatrix}1&3&5&2&4\end{pmatrix} \neq \id$;
            \item $\pi^3 = \begin{pmatrix}1&4&2&5&3\end{pmatrix} \neq \id$;
            \item $\pi^4 = \begin{pmatrix}1&5&4&3&2\end{pmatrix} \neq \id$; and
            \item $\pi^5 = \id$,
        \end{itemize}
        meaning $|\langle \pi \rangle| = 5$ with $\langle \pi \rangle \neq \langle \sigma \rangle$, so we have found another subgroup of $\An5$.
    \end{partquestions}
\begin{mdframed}
    Let $S$ be a non-empty set and let $G \leq \Sym{S}$ act on $S$. Show that $\sigma\Stab{G}{x}\sigma^{-1} = \Stab{G}{\sigma(x)}$ for any $\sigma \in G$ and $x \in S$.
\end{mdframed}
\textbf{Solution}:\newline
 Note that $\sigma \in G$, and
    \begin{align*}
        \sigma\Stab{G}{x}\sigma^{-1} &= \{\underbrace{\sigma\pi\sigma^{-1}}_{\text{Set as }\pi'} \vertalt \pi \in G,\; \pi(x) = x\}\\
        &= \{\pi' \vertalt \underbrace{\sigma^{-1}\pi'\sigma \in G}_{\text{True if } \pi' \in G},\; \sigma^{-1}\pi'\sigma(x) = x\}\\
        &= \{\pi' \vert \pi' \in G,\; \sigma^{-1}\pi'(\sigma(x)) = x\}\\
        &= \{\pi' \vert \pi' \in G,\; \pi'(\sigma(x)) = \sigma(x)\}\\
        &= \{\pi \vert \pi \in G,\; \pi(\sigma(x)) = \sigma(x)\}\\
        &= \Stab{G}{\sigma(x)}
    \end{align*}
    which proves the claim.
\begin{mdframed}
    Let $\sigma, \pi \in \Sn{n}$. Suppose $\sigma$ has cycle decomposition
    \[
        \begin{pmatrix}a_1&a_2&\cdots&a_{k_1}\end{pmatrix} \begin{pmatrix}b_1&b_2&\cdots&b_{k_2}\end{pmatrix}\cdots,
    \]
    where $a_1, a_2, \dots, a_{k_1}, b_1, b_2, \dots, b_{k_2}, \dots$ are all distinct. Show that
    \[
        \pi\sigma\pi^{-1} = \begin{pmatrix}\pi(a_1)&\cdots&\pi(a_{k_1})\end{pmatrix} \begin{pmatrix}\pi(b_1)&\cdots&\pi(b_{k_2})\end{pmatrix}\cdots,
    \]
    that is, $\pi\sigma\pi^{-1}$ is obtained from $\sigma$ by replacing each entry $i$ by $\pi(i)$.
\end{mdframed}
\textbf{Solution}:\newline
 Observe that if $\sigma(i) = j$, then we must have
    \[
        \pi\sigma\pi^{-1}(\pi(i)) = \pi\sigma(i) = \pi(j).
    \]
    Thus if the ordered pair $(i, j)$ appears in the cycle decomposition of $\sigma$, then the ordered pair $(\pi(i), \pi(j))$ appears in the cycle decomposition of $\pi\sigma\pi^{-1}$, completing the proof of the claim.

\section*{Problems}

\chapter{Intro To Rings}
\section*{Exercises}
\begin{mdframed}
    Prove that the trivial ring is a commutative ring with identity.
\end{mdframed}
\textbf{Solution}:\newline
 We note the following.
    \begin{itemize}
        \item \textbf{Addition-Abelian}: $(\{0\}, +)$ is an abelian group.
        \item \textbf{Multiplication-Semigroup}: $(\{0\}, \cdot)$ is an abelian group.
        \item \textbf{Distributive}: We know $+$ and $\cdot$ distribute.
    \end{itemize}
    Hence $(\{0\}, +, \cdot)$ is a commutative ring with identity.

\section*{Problems}

\chapter{Basics Of Rings}
\section*{Exercises}
\begin{mdframed}
    Prove that $\Z$ is a ring under regular addition and multiplication.\newline
    (\textit{You do \textbf{not} need to prove the \textbf{Distributive} axiom.})
\end{mdframed}
\textbf{Solution}:\newline
 We note the following.
    \begin{itemize}
        \item \textbf{Addition-Abelian}: $(\Z, +)$ is an abelian group.
        \item \textbf{Multiplication-Semigroup}: $(\Z, \times)$ is a semigroup since
        \begin{itemize}
            \item multiplying two integers always results in an integer, so $\Z$ is closed under $\times$; and
            \item $\times$ is associative.
        \end{itemize}
        \item \textbf{Distributive}: We know $+$ and $\times$ distribute.
    \end{itemize}
    Hence $(\Z, +, \times)$ is a ring.
\begin{mdframed}
    Show that $(-a)(-b) = ab$ for any $a$ and $b$ in $R$.
\end{mdframed}
\textbf{Solution}:\newline
 Consider $(-a)(-b) + (-ab)$ and note
    \begin{align*}
        &(-a)(-b) + (-ab)\\
        &= (-a)(-b) + (-a)b & (\text{\myref{prop-product-of-element-and-additive-inverse-is-additive-inverse-of-product}})\\
        &= (-a)(-b + b) & (\text{by \textbf{Distributive} axiom})\\
        &= (-a)0\\
        &= 0 & (\myref{prop-multiplying-by-zero-is-zero})
    \end{align*}
    which means $(-a)(-b) = -(-ab) = ab$ as required.
\begin{mdframed}
    Does the ring $\Mn{2}{\mathbb{R}}$ have zero divisors?
\end{mdframed}
\textbf{Solution}:\newline
 The ring $\Mn{2}{\mathbb{R}}$ indeed has zero divisors, as $\begin{pmatrix}0&1\\0&0\end{pmatrix} \neq \begin{pmatrix}0&0\\0&0\end{pmatrix}$ but $\begin{pmatrix}0&1\\0&0\end{pmatrix}^2 = \begin{pmatrix}0&0\\0&0\end{pmatrix}$ which means that $\begin{pmatrix}0&1\\0&0\end{pmatrix}$ is a zero divisor.
\begin{mdframed}
    Which of the following rings, if any, are fields?
    \begin{partquestions}{\alph*}
        \item $\Z$
        \item $\Q$
    \end{partquestions}
\end{mdframed}
\textbf{Solution}:\newline
 \begin{partquestions}{\alph*}
        \item $\Z$ is not a field. Note that the multiplicative inverse of 2 is $\frac12$ which is not an integer. Hence not all non-zero elements in $\Z$ has a multiplicative inverse, meaning that not all non-zero elements are units.

        \item $\Q$ is a field. Note for any rational number $\frac ab$ (where $b \neq 0$) it has an inverse of $\frac ba$. Thus any non-zero rational number is a unit, which means $\Q$ is a division ring. Since $\Q$ is also a commutative ring, therefore $\Q$ is a field.
    \end{partquestions}
\begin{mdframed}
    Prove \myref{prop-product-of-units-is-unit}.
\end{mdframed}
\textbf{Solution}:\newline
 Let $u$ and $v$ be units, meaning that $u^{-1}$ and $v^{-1}$ exist. Then one sees that $(uv)(v^{-1}u^{-1}) = (v^{-1}u^{-1})(uv) = 1$, which means that $uv$ is also a unit.
\begin{mdframed}
    Show that
    \[
        R = \left\{\begin{pmatrix}a&a\\a&a\end{pmatrix} \vert a \in \R\right\}
    \]
    is a subring of $\Mn{2}{\mathbb{R}}$.
\end{mdframed}
\textbf{Solution}:\newline
 We first show that $(R, +) \leq (\Mn{2}{\R}, +)$.
    \begin{itemize}
        \item Clearly the identity of $(\Mn{2}{\R}, +)$, the zero matrix $\begin{pmatrix}0&0\\0&0\end{pmatrix}$, is inside $R$.
        \item Consider $\begin{pmatrix}a&a\\a&a\end{pmatrix}, \begin{pmatrix}b&b\\b&b\end{pmatrix} \in R$. The additive inverse of the matrix $\begin{pmatrix}b&b\\b&b\end{pmatrix}$ is the matrix $\begin{pmatrix}-b&-b\\-b&-b\end{pmatrix}$, and so their sum is
        \[
            \begin{pmatrix}a&a\\a&a\end{pmatrix} + \begin{pmatrix}-b&-b\\-b&-b\end{pmatrix} = \begin{pmatrix}a-b&a-b\\a-b&a-b\end{pmatrix} \in R
        \]
        which means $R$ is closed under addition.
    \end{itemize}
    Hence $(R, +) \leq (\Mn{2}{\R}, +)$ by subgroup test.

    We now show that $R$ is closed under multiplication. Some calculation yields that
    \[
        \begin{pmatrix}a&a\\a&a\end{pmatrix}\begin{pmatrix}b&b\\b&b\end{pmatrix} = \begin{pmatrix}2ab&2ab\\2ab&2ab\end{pmatrix}
    \]
    which is clearly in $R$. Therefore $R$ is a subring of $\Mn{2}{\R}$.

\section*{Problems}
\begin{mdframed}
    Let $R$ be a ring. Prove that if $u \in R$ is a unit then so is $-u$.
\end{mdframed}
\textbf{Solution}:\newline
 Since $u$ is a unit, thus $u^{-1}$ exists. Consider the element $(-u)(-u^{-1})$. We know from an earlier proposition that this equals $uu^{-1} = 1$. Similarly, $(-u^{-1})(-u) = 1$. Hence $-u$ is a unit.
\begin{mdframed}
    Prove that the trivial ring is the unique ring with identity in which $0 = 1$.
\end{mdframed}
\textbf{Solution}:\newline
 Suppose $R$ is a ring where 0 = 1, and let $x \in R$. Then
    \begin{align*}
        x &= 1x & (1 \text{ is the multiplicative identity})\\
        &= 0x & (0 = 1)\\
        &= 0 & (0x = 0 \text{ for all }x)
    \end{align*}
    which means that $R$ only contains one element, namely the identity 0. Hence the trivial ring is the unique ring where 0 = 1.
\begin{mdframed}
    Let $R$ be a set with an operation $\ast$ such that for all elements $x$ and $y$ in $R$ we have $x \ast y \in R$. If $(R, \ast, \ast)$ is a ring, describe the elements in $R$.
\end{mdframed}
\textbf{Solution}:\newline
 Since $(R, \ast, \ast)$ is a ring, we know that $(R, \ast)$ is an abelian group and that $a \ast(b\ast c) = (a \ast b) \ast (a \ast c)$ for all $a, b, c \in R$ (left distribution). Consider an element $x \in R$. Thus we must have
    \[
        x \ast (x \ast x) = (x \ast x) \ast (x \ast x)
    \]
    which means $x^3 = x^4$. Since $(R, \ast)$ is a group, thus $x^{-3}$ exists. Applying that on both sides means $x = e$ where $e$ is the identity. Therefore, $R$ contains only one element, the identity, which means that $(R, \ast, \ast)$ is actually the trivial ring.
\begin{mdframed}
    Let $R$ be a ring with identity 1, and let $x$ be an element from that ring.
    \begin{partquestions}{\roman*}
        \item Find \textbf{four} closed forms for the geometric series $1 + x + x^2 + x^3 + \cdots + x^n$.
        \item What are the condition(s) such that the closed forms are valid?
        \item Factor 112 into primes, and thus evaluate 112 in $\Z_{37}$.
        \item Hence, using the result(s) above, evaluate
        \[
            1 + 2^3 + 2^6 + 2^9 + \cdots + 2^{72}
        \]
        in the ring $\Z_{37}$.
    \end{partquestions}
\end{mdframed}
\textbf{Solution}:\newline
 By way of contradiction assume that the element $x$ is both a zero divisor and a unit. Since $x$ is a zero divisor, we know $x \neq 0$ and there exists a non-zero $y \in R$ such that $xy = 0$. Since $x$ is a unit, thus $x^{-1}$ exists such that $xx^{-1} = x^{-1}x = 1$. However we note that
    \begin{align*}
        y &= (x^{-1}x)y & (\text{as }x^{-1}x = 1)\\
        &= x^{-1}(xy) & (\text{associativity})\\
        &= x^{-1}0 & (\text{as }x \text{ is a zero divisor})\\
        &= 0
    \end{align*}
    which contradicts the fact that $y \neq 0$. Therefore it is impossible for an element to be both a zero divisor and a unit.
\begin{mdframed}
    Show that
    \[
        \Q[\sqrt2] = \{a + b\sqrt2 \vert a,b \in \Q\}
    \]
    is a ring. Hence show it is a field.
\end{mdframed}
\textbf{Solution}:\newline
 \begin{partquestions}{\roman*}
        \item Let $1 + x + \cdots + x^n = y$, which is an element of $R$ since $R$ is closed under addition and multiplication. Note $xy = x + x^2 + \cdots + x^n + x^{n+1}$ and $yx = x + x^2 + \cdots + x^n + x^{n+1}$ as well. Thus we have $1 - x^{n+1} = y - xy = y - yx$ and $x^{n+1} = xy - y = yx - y$, so
        \[
            (1-x)^{-1}(1-x^{n+1}) = (1-x^{n+1})(1-x)^{-1} = y
        \]
        and
        \[
            (x-1)^{-1}(x^{n+1}-1) = (x^{n+1}-1)(x-1)^{-1} = y
        \]
        which are the four closed forms that we require.
        \item The condition is that either $1-x$ is a unit or $x-1$ is a unit in $R$, i.e. $(1-x)^{-1}$ or $(x-1)^{-1}$ exists.
        \item $112 = 1$ in $\Z_{37}$ since $112 = 37\times3 + 1$.
        \item Let $x = 2^3 = 8$. Note $(x-1)^{-1} = (8-1)^{-1} = 7^{-1} = 16$ by \textbf{(iii)}. Thus
        \[
            1+2+\cdots+2^{72} = (8-1)^{-1}(8^{25}-1) = 16(8^{25}-1).
        \]
        We also see that
        \begin{align*}
            8^{25}-1 &= \left(8^{5}\right)^{5} - 1\\
            &= 32768^5 - 1\\
            &= (37\times885 + 23)^5 - 1\\
            &= (23)^5 - 1\\
            &= 6436342\\
            &= 37\times173955 + 7\\
            &= 7
        \end{align*}
        which means $1+2^3+\cdots+2^{72} = 16 \times 7 = 112 = 1$.
    \end{partquestions}
\begin{mdframed}
    Let
    \[
        R = \left\{\begin{pmatrix}a&b\\0&0\end{pmatrix} \vert a,b \in \R\right\}
    \]
    be a ring under matrix addition and multiplication.
    \begin{partquestions}{\roman*}
        \item Show that $R$ has no identity.
        \item Show that $R$ contains a non-trivial subring $S$ with identity.
    \end{partquestions}
\end{mdframed}
\textbf{Solution}:\newline
 For brevity let $R = \Q[\sqrt2]$. We first show that $(R,+)$ is an abelian group.
    \begin{itemize}
        \item \textbf{Closure}: Take $a+b\sqrt2, c+d\sqrt2 \in R$. Clearly
        \[
            (a+b\sqrt2) + (c+d\sqrt2) = (a+c) + (b+d)\sqrt2 \in R
        \]
        which means that $R$ is closed under addition.
        \item \textbf{Associativity}: + is associative.
        \item \textbf{Identity}: The identity is $0 = 0 + 0\sqrt2$.
        \item \textbf{Inverse}: The inverse of $a+b\sqrt2 \in R$ is $(-a) + (-b)\sqrt2$ which is in $R$.
        \item \textbf{Commutative}: + is commutative.
    \end{itemize}
    Hence $(R, +)$ is an abelian group.

    Now we show that $(R, \cdot)$ is a semigroup.
    \begin{itemize}
        \item \textbf{Closure}: Take $a+b\sqrt2, c+d\sqrt2 \in R$. Then
        \begin{align*}
            (a+b\sqrt2)(c+d\sqrt2) &= ac + ad\sqrt2 + bc\sqrt2 + 2bd\\
            &= (ac+2bd) + (ad+bc)\sqrt2\\
            &\in R.
        \end{align*}
        \item \textbf{Associative}: $\cdot$ is associative.
    \end{itemize}
    So $(R, \cdot)$ is an abelian group.

    Clearly $+$ and $\cdot$ distribute, so we have shown that $R$ is in fact a ring.

    Furthermore,
    \begin{itemize}
        \item $\cdot$ is commutative, meaning that $R$ is a commutative ring;
        \item $1 = 1 + 0\sqrt2 \in R$ is the multiplicative identity, so $R$ is a ring with identity; and
        \item for any non-zero element $a+b\sqrt2$ we have its multiplicative inverse as
        \begin{align*}
            \frac{1}{a+b\sqrt2} &= \frac{a-b\sqrt2}{(a-b\sqrt2)(a+b\sqrt2)}\\
            &= \frac{a-b\sqrt2}{a^2-2b}\\
            &= \frac{a}{a^2-2b} + \left(-\frac{b}{a^2-2b}\right)\sqrt2\\
            &\in R.
        \end{align*}
    \end{itemize}
    Therefore $R$ is a field.
\begin{mdframed}
    A ring $R$ is called a \textbf{Boolean ring}\index{Boolean ring} if $r^2 = r$ for all $r \in R$.
    \begin{partquestions}{\roman*}
        \item Show that $r = -r$ for all $r \in R$.
        \item Prove that every Boolean ring is commutative.
    \end{partquestions}
\end{mdframed}
\textbf{Solution}:\newline
 \begin{partquestions}{\roman*}
        \item Suppose $R$ has an identity $E = \begin{pmatrix}e_1&e_2\\0&0\end{pmatrix}$. Take any matrix $M = \begin{pmatrix}a&b\\0&0\end{pmatrix} \in R$, so we must have $ME = EM = M$.
        \begin{itemize}
            \item Note $EM = \begin{pmatrix}e_1&e_2\\0&0\end{pmatrix}\begin{pmatrix}a&b\\0&0\end{pmatrix} = \begin{pmatrix}e_1a&e_1b\\0&0\end{pmatrix}$.
            \item Also, $ME = \begin{pmatrix}a&b\\0&0\end{pmatrix}\begin{pmatrix}e_1&e_2\\0&0\end{pmatrix} = \begin{pmatrix}ae_1&ae_2\\0&0\end{pmatrix}$.
        \end{itemize}
        Now as $EM = M$ we must have $e_1 = 1$. But as $ME = M$ this means that
        \[
            \begin{pmatrix}a&ae_2\\0&0\end{pmatrix} = \begin{pmatrix}a&b\\0&0\end{pmatrix}
        \]
        which implies that $ae_2 = b$, hence meaning $e_2 = \frac{b}{a}$ which is not a constant. Hence there does not exist a fixed identity $E$ in $R$.

        \item Consider the subset
        \[
            S = \left\{\begin{pmatrix}a&0\\0&0\end{pmatrix} \vert a \in \R\right\}
        \]
        of $R$. We first show that $S$ is a subring of $R$. Clearly $(S, +) \leq (R, +)$ so we only show that $S$ is closed under matrix multiplication.
        \[
            \begin{pmatrix}a&0\\0&0\end{pmatrix}\begin{pmatrix}b&0\\0&0\end{pmatrix} = \begin{pmatrix}ab&0\\0&0\end{pmatrix} \in S.
        \]

        One sees clearly based on the above calculation that $\begin{pmatrix}1&0\\0&0\end{pmatrix}$ is the identity in $S$. Hence $S$ is a subring of $R$ with identity.
    \end{partquestions}
\begin{mdframed}
    Let $R$ be a commutative ring with identity. We say that an element $x$ in $R$ is \textbf{nilpotent}\index{nilpotent} if there exists a positive integer $n$ such that $x^n = 0$.
    \begin{partquestions}{\roman*}
        \item Let $u \in R$ be a unit and $x \in R$ be nilpotent. Show that $ux$ is nilpotent.
        \item Show that $u - x$ is a unit.
    \end{partquestions}
\end{mdframed}
\textbf{Solution}:\newline
 \begin{partquestions}{\roman*}
        \item Consider $(r+r)^2$. CLearly $(r+r)^2 = r+r$ by definition of a Boolean ring. On the other hand, one may expand $(r+r)^2$ to yield
        \begin{align*}
            (r+r)^2 &= r^2 + r^2 + r^2 + r^2 \\
            &= r + r + r + r & (r^2 = r \text{ for all }r \in R)
        \end{align*}
        which means $r+r = r+r+r+r$. Thus $r+r = 0$ which hence means $r=-r$.

        \item Let $x,y\in R$. Then
        \begin{align*}
            x+y &= (x+y)^2 & (r = r^2 \text{ for all }r \in R)\\
            &= x^2 + xy + yx + y^2\\
            &= x + xy + yx + y. & (r = r^2 \text{ for all }r \in R)
        \end{align*}
        Subtracting $x+y$ on both sides yields $xy +yx = 0$. Hence $xy = -yx$ which means $xy = yx$ by \textbf{(i)}. Therefore every Boolean ring is commutative.
    \end{partquestions}

\chapter{Integral Domains}
\section*{Exercises}
\begin{mdframed}
    Prove that the ring
    \[
        \Z[\sqrt2] = \left\{a + b\sqrt2 \vert a,b \in \Z\right\}
    \]
    is an integral domain.
\end{mdframed}
\textbf{Solution}:\newline
 Clearly multiplication is commutative and $1 = 1 + 0\sqrt2 \in \Z[\sqrt2]$. All that is needed is to show that there are no zero divisors in $\Z[\sqrt2]$.

    Take $a+b\sqrt2, c+d\sqrt2 \in \Z[\sqrt2]$ such that $a+b\sqrt2 \neq 0$ and $(a+b\sqrt2)(c+d\sqrt2) = 0$. We want to show that the only way this is possible is if $c = d = 0$. Now consider
    \[
        \left((a+b\sqrt2)\underbrace{(a-b\sqrt2)}_{\neq 0}\right)\left((c+d\sqrt2)\underbrace{(c-d\sqrt2)}_{\neq 0}\right) = 0.
    \]
    This simplifies to $(a^2-2b^2)(c^2-2d^2) = 0$. Hence either $a^2-2b^2 = 0$ or $c^2-2d^2 = 0$, implying $a = \sqrt2b$ or $c = \sqrt2d$. Now we cannot have $a = \sqrt2b$ as $\sqrt2$ is not an integer; the only case that is possible is if $c = \sqrt2d$ which finally means that $c = d = 0$.
\begin{mdframed}
    Let $n$ be an arbitrary composite number. Prove that $\Z_n$ is not an integral domain.
\end{mdframed}
\textbf{Solution}:\newline
 Let $n = ab$ where $a,b \in Z$ and, without loss of generality, assume $1 < a \leq b < n$ (we exclude 1 and $n$ because we want $a$ and $b$ to be `proper' factors). Now clearly $a, b \in \Z_n$ with them both being non-zero but $ab = n = 0$ in $\Z_n$. Hence $a$ and $b$ are zero divisors in $\Z_n$, meaning $\Z_n$ is not an integral domain.
\begin{mdframed}
    Is the ring
    \[
        \Z_2[\alpha] = \{a+b\alpha \vert a,b\in\Z_2\}
    \]
    where $\alpha^2 = 1 + \alpha$ under regular addition and multiplication a field?
\end{mdframed}
\textbf{Solution}:\newline
 We show that that $\Z_2[\alpha]$ is indeed a field. We note that multiplication is commutative with identity 1. The multiplication table in $\Z_2[\alpha]$ is provided below.
    \begin{table}[H]
        \centering
        \resizebox{\textwidth}{!}{
            \begin{tabular}{|l|l|l|l|}
                \hline
                $\boldsymbol{\times}$   & \textbf{1} & $\boldsymbol{\alpha}$                 & $\boldsymbol{1+\alpha}$                     \\ \hline
                \textbf{1}          & 1          & $\alpha$                          & $1+\alpha$                              \\ \hline
                $\boldsymbol{\alpha}$   & $\alpha$   & $\alpha^2 = 1+\alpha$             & $\alpha+\alpha^2 = 1+2\alpha = 1$       \\ \hline
                $\boldsymbol{1+\alpha}$ & $1+\alpha$ & $\alpha+\alpha^2 = 1+2\alpha = 1$ & $1+2\alpha+\alpha^2 = 2+3\alpha=\alpha$ \\ \hline
            \end{tabular}
        }
    \end{table}

    What we see from this table is that no non-zero elements multiply together to form zero, meaning that there are no zero divisors in $\Z_2[\alpha]$. Therefore $\Z_2[\alpha]$ is an integral field. Furthermore as $\Z_2[\alpha]$ is finite thus $\Z_2[\alpha]$ is a field by \myref{thrm-finite-integral-domain-is-field}.
\begin{mdframed}
    Is the trivial ring an integral domain?
\end{mdframed}
\textbf{Solution}:\newline
 The trivial ring $\{0\}$ is not an integral domain. Seeking a contradiction, if $\{0\}$ is indeed an integral domain, then by \myref{prop-zero-or-prime-characteristic-if-integral-domain} it has to have either 0 or prime characteristic. However, one sees clearly that the characteristic of the trivial ring is 1, which is neither 0 nor prime. Therefore $\{0\}$ is not an integral domain.
\begin{mdframed}
    What is the characteristic of the ring $\Z_2[\alpha]$ in \myref{exercise-Zn2[alpha]}?
\end{mdframed}
\textbf{Solution}:\newline
 The additive identity in $\Z_2[\alpha]$ is 1. Clearly $1 + 1 = 0$, so the order of 1 in $(\Z_2[\alpha],+)$ is 2. Now \myref{exercise-Zn2[alpha]} tells us that $\Z_2[\alpha]$ is an integral domain, so by the previous proposition this means that $\Char{\Z_2[\alpha]} = 2$.

\section*{Problems}
\begin{mdframed}
    Find two zero divisors in the ring
    \[
        \Z_5[i] = \{a + bi \vert a,b\in\Z_5\}.
    \]
\end{mdframed}
\textbf{Solution}:\newline
 To find a $a+bi \in \Z_5[i]$ such that there exists a $c+di \in \Z_5[i]$ where $(a+bi)(c+di) = 0$ but both $a+bi$ and $c+di$ are non-zero. Expanding $(a+bi)(c+di)$ yields $(ac-bd)+(ad+bc)i = 0$. Therefore we must have $ac-bd = 0$ and $ad+bc = 0$. For simplicity let's choose $a=c=1$. Using second equation we have $d+b = 0$ which means $d = -b$. Hence $(1 - b(-b))+(-b + b)i = 1+b^2 = 0$. Therefore choosing $b = 2$ would make it work. Therefore one solution is $a = 1, b = 2, c = 1, d = -2 = 3$; i.e. two zero divisors are $1+2i$ and $1+3i$.
\begin{mdframed}
    Let the integer $n$ be such that $\sqrt{|n|}$ is not an integer. Let the ring
    \[
        R = \Z[\sqrt{n}] = \{a + b\sqrt{n} \vert a,b\in \Z\}.
    \]
    \begin{partquestions}{\alph*}
        \item Show that $R$ is an integral domain.
        \item Is $R$ a field for all integers $n$?
    \end{partquestions}
\end{mdframed}
\textbf{Solution}:\newline
 \begin{partquestions}{\alph*}
        \item Note that multiplication is commutative with identity $1 = 1 + 0\sqrt{n} \in R$. We just need to show that there are no zero divisors in $R$.

        Take $a+b\sqrt n, c+d\sqrt n \in R$ such that $a+b\sqrt n \neq 0$ but $(a+b\sqrt n)(c+d\sqrt n) = 0$. We want to show $c = d = 0$. Consider
        \[
            \left((a+b\sqrt n)(\underbrace{a-b\sqrt n}_{\neq 0})\right)\left((c+d\sqrt n)(\underbrace{c-d\sqrt n}_{\neq 0})\right) = 0.
        \]
        This means that $(a^2-nb^2)(c^2-nd^2) = 0$, so either $a^2-nb^2 = 0$ or $c^2-nd^2 = 0$.

        Now if $n < 0$ then clearly we have to have $c = d = 0$. Otherwise we have $a = b\sqrt n$ or $c = d\sqrt n$. But $\sqrt n$ is not an integer, so the only way for equality is if $c = d = 0$. Thus $\Z[\sqrt n]$ has no zero divisors, meaning $\Z[\sqrt n]$ is an integral domain.

        \item Consider $2 + \sqrt 2 \in \Z[\sqrt 2]$. Its multiplicative inverse is
        \begin{align*}
            \frac{1}{2+\sqrt2} &= \frac{2-\sqrt2}{(2+\sqrt2)(2-\sqrt2)}\\
            &= \frac{2-\sqrt2}{4-2}\\
            &= 1 - \frac12\sqrt2 \notin \Z[\sqrt2].
        \end{align*}
        This means that $2+\sqrt2$, a non-zero element in $\Z[\sqrt2]$, does not have an inverse in $\Z[\sqrt2]$. Therefore $\Z[\sqrt2]$ is not a field, meaning $R$ is not a field in the general case.
    \end{partquestions}
\begin{mdframed}
    Show that
    \[
        R = \left\{\begin{pmatrix}0&0\\0&0\end{pmatrix},\begin{pmatrix}1&0\\0&1\end{pmatrix},\begin{pmatrix}1&1\\1&0\end{pmatrix},\begin{pmatrix}0&1\\1&1\end{pmatrix}\right\}
    \]
    with entries in $\Z_2$ is
    \begin{partquestions}{\roman*}
        \item a subring of $\Mn{2}{\Z_2}$; and
        \item a field.
    \end{partquestions}
\end{mdframed}
\textbf{Solution}:\newline
 For brevity let O$ = \begin{pmatrix}0&0\\0&0\end{pmatrix}$, I$ = \begin{pmatrix}1&0\\0&1\end{pmatrix}$, A$ = \begin{pmatrix}1&1\\1&0\end{pmatrix}$, and B$ = \begin{pmatrix}0&1\\1&1\end{pmatrix}$.

    \begin{partquestions}{\roman*}
        \item Clearly one sees that $R$ is a subset of $\Mn{2}{\Z_2}$.
        \begin{itemize}
            \item We show $(R, +)\leq(\Mn{2}{\Z_2},+)$.
            \begin{table}[H]
                \centering
                \begin{tabular}{|l|l|l|l|l|}
                    \hline
                    \textbf{+} & \textbf{O} & \textbf{I} & \textbf{A} & \textbf{B} \\ \hline
                    \textbf{O} & O          & I          & A          & B          \\ \hline
                    \textbf{I} & I          & O          & B          & A          \\ \hline
                    \textbf{A} & A          & B          & O          & I          \\ \hline
                    \textbf{B} & B          & A          & I          & O          \\ \hline
                \end{tabular}
            \end{table}

            From the Cayley table, clearly the identity of the ring $\Mn{2}{\Z_2}$ is in $R$ and $R$ is closed under addition. Hence $(R, +)\leq(\Mn{2}{\Z_2},+)$

            \item We show $R$ is closed under multiplication.
            \begin{table}[H]
                \centering
                \begin{tabular}{|l|l|l|l|l|}
                    \hline
                    $\boldsymbol{\cdot}$ & \textbf{O} & \textbf{I} & \textbf{A} & \textbf{B} \\ \hline
                    \textbf{O}           & O          & O          & O          & O          \\ \hline
                    \textbf{I}           & O          & I          & A          & B          \\ \hline
                    \textbf{A}           & O          & A          & B          & I          \\ \hline
                    \textbf{B}           & O          & B          & I          & A          \\ \hline
                \end{tabular}
            \end{table}

            From the Cayley table, clearly $R$ is closed under multiplication.
        \end{itemize}
        Therefore $R$ is a subring of $\Mn{2}{\Z_2}$.

        \item Since $R$ is a subring of $\Mn{2}{\Z_2}$, it is a ring. Furthermore, by the Cayley table of $(R, \cdot)$, we see that $R$ is commutative with identity I. Finally, one sees that $\mathrm{A}^{-1} = \mathrm{B}$, $\mathrm{B}^{-1} = \mathrm{A}$, and $\mathrm{I}^{-1} = \mathrm{I}$. Therefore all non-zero elements of $R$ have inverses. Hence $R$ is a field.
    \end{partquestions}

\chapter{Ideals And Quotient Rings}
\section*{Exercises}
\begin{mdframed}
    Let $I$ be an ideal of a ring $R$. Show that $I$ is a subring of $R$.
\end{mdframed}
\textbf{Solution}:\newline
 We are given that $(I, +) \leq (R,+)$, so all that remains to show is that $I$ is closed under multiplication. Take any two elements $x$ and $y$ in $I$. Since $I$ is an ideal, thus we have $ri \in I$ for any $r \in R$ and $i \in I$. Viewing $x$ as an element of $R$ and $y$ as an element of $I$, we see that $xy \in I$, meaning $I$ is closed under multiplication. Hence $I$ is a subring of $R$.
\begin{mdframed}
    Let the sets
    \begin{align*}
        R &= \left\{\begin{pmatrix}x&y\\0&z\end{pmatrix} \vert x,y,z \in \R\right\},\\
        I &= \left\{\begin{pmatrix}x&y\\0&0\end{pmatrix} \vert x,y \in \R\right\}.
    \end{align*}
    It is given that $R$ is a subring of $\Mn{2}{\R}$.
    \begin{partquestions}{\roman*}
        \item Show that $I$ is a subring of $R$.
        \item Show that $I$ is an ideal of $R$.
        \item Simplify
        \[
            \left(\begin{pmatrix}1&2\\0&1\end{pmatrix} + I\right)\left(\begin{pmatrix}1&-2\\0&1\end{pmatrix} + I\right)
        \]
        in $R/I$.
    \end{partquestions}
\end{mdframed}
\textbf{Solution}:\newline
 \begin{partquestions}{\roman*}
        \item Clearly the zero matrix, the additive identity of $R$, is inside $I$. Also,
        \[
            \begin{pmatrix}a&b\\0&0\end{pmatrix} + (-\begin{pmatrix}c&d\\0&0\end{pmatrix}) = \begin{pmatrix}a-c&b-d\\0&0\end{pmatrix} \in I
        \]
        so $I$ is a subring of $R$.

        \item Let $\begin{pmatrix}a&b\\0&0\end{pmatrix} \in I$ and $\begin{pmatrix}x&y\\0&z\end{pmatrix} \in R$. We need to show that $I$ is both a left and right ideal.
        \begin{itemize}
            \item \textbf{Left Ideal}:
            \[
                \begin{pmatrix}x&y\\0&z\end{pmatrix}\begin{pmatrix}a&b\\0&0\end{pmatrix} = \begin{pmatrix}xa&xb\\0&0\end{pmatrix} \in I;
            \]
            and
            \item \textbf{Right Ideal}: \[
                \begin{pmatrix}a&b\\0&0\end{pmatrix}\begin{pmatrix}x&y\\0&z\end{pmatrix} = \begin{pmatrix}ax&ay+bz\\0&0\end{pmatrix} \in I.
            \]
        \end{itemize}
        Therefore $I$ is an ideal of $R$.

        \item $\begin{pmatrix}1&0\\0&1\end{pmatrix} + I$
    \end{partquestions}
\begin{mdframed}
    Let $R$ be a ring with identity 1, and let $I$ be an ideal of $R$.
    \begin{partquestions}{\roman*}
        \item If $1 \in I$, what does this imply about $I$?
        \item Prove $I$ contains a unit if and only if $I = R$.
    \end{partquestions}
\end{mdframed}
\textbf{Solution}:\newline
 \begin{partquestions}{\roman*}
        \item If $1 \in I$, then for any element $r \in R$, we must have $r = 1r \in I$ since $I$ is an ideal of $R$. Therefore $R \subseteq I$. But by definition of an ideal, $I \subseteq R$. Therefore $I = R$.

        \item For the forward direction, if $I$ contains a unit $u$, then there exists a $v \in R$ such that $uv = 1$. Note $uv \in I$ since $u \in I$ and $I$ is an ideal, so $1 \in I$. By \textbf{(i)} we have $I = R$.

        For the reverse direction, note that $1 \in R$ and so $1 \in I$ since $I = R$. Clearly 1 is a unit since $1\times1 = 1$. Therefore $I$ contains a unit.
    \end{partquestions}
\begin{mdframed}
    Let $R$ be a ring and let $\ideal{a}$ and $\ideal{b}$ be ideals of $R$. Prove that $\ideal{a} \cap \ideal{b}$ is an ideal of $R$.
\end{mdframed}
\textbf{Solution}:\newline
 We consider the test for ideal (\myref{thrm-test-for-ideal}) to prove that $\ideal{a}\cap\ideal{b}$ is an ideal. We note that as $\ideal{a}$ and $\ideal{b}$ are ideals, they are therefore subrings of $R$. Thus, 0 is in both $\ideal{a}$ and $\ideal{b}$, meaning $0 \in \ideal{a}\cap\ideal{b}$. Hence $\ideal{a}\cap\ideal{b}$ is non-empty.

    Suppose $i,j\in\ideal{a}\cap\ideal{b}$, so $i,j \in \ideal{a}$ and $i,j \in \ideal{b}$. Note $\ideal{a}$ and $\ideal{b}$ are ideals and so are subrings, which means that $i-j \in \ideal{a}$ and $i-j \in \ideal{b}$, which hence means $i-j \in \ideal{a}\cap\ideal{b}$, satisfying the first statement for the test for ideals.

    Now suppose $r \in R$ and $i \in \ideal{a}\cap\ideal{b}$. This means that $i \in \ideal{a}$ and $i \in \ideal{b}$. So we have $ri, ir \in \ideal{a}$ (since $\ideal{a}$ is an ideal) and $ri, ir \in \ideal{b}$ (since $\ideal{b}$ is an ideal). Therefore $ri,ir \in \ideal{a}\cap\ideal{b}$, so by the test for ideal we have $\ideal{a}\cap\ideal{b}$ is an ideal.
\begin{mdframed}
    Show that all principal ideals are ideals.
\end{mdframed}
\textbf{Solution}:\newline
 Let $R$ be a commutative ring with identity and $\princ{a}$ be a principal ideal of $R$. We note any element in $\princ{a}$ takes the form $ar$ for some element $r \in R$. We consider the test for ideal (\myref{thrm-test-for-ideal}) to prove this. Clearly $\princ{a}$ is non-empty as $0 = a0 \in \princ{a}$.

    Let $ar_1, ar_2 \in \princ{a}$. Clearly $ar_1 - ar_2 = a(r_1-r_2) \in \princ{a}$, so the first condition for the test for ideal is satisfied. Now take any $r \in R$ and let $ax \in \princ{a}$. Then $r(ax) = (ax)r = a(xr) \in \princ{a}$ since $R$ is commutative. Therefore by the test for ideal we have $\princ{a}$ is an ideal of $R$.
\begin{mdframed}
    Let $R$ be a commutative ring with identity. Show that the following ideals are principal in $R$.
    \begin{partquestions}{\alph*}
        \item The trivial ideal.
        \item The ring itself.
    \end{partquestions}
\end{mdframed}
\textbf{Solution}:\newline
 \begin{partquestions}{\alph*}
        \item Clearly $\{0\} = \princ{0}$ since $0r = 0$ for any $r \in R$.
        \item Let 1 be the identity of $R$. Then $R = \{r \vert r \in R\} = \{1r \vert r \in R\} = \princ{1}$.
    \end{partquestions}
\begin{mdframed}
    Is the principal ideal $\princ{2} = \{0, 2, 4, 6\}$ prime in $\Z_8$?
\end{mdframed}
\textbf{Solution}:\newline
 We show that $\princ2$ is indeed a prime ideal of $\Z_8$. Without loss of generality, assume that $a \leq b$.
    \begin{itemize}
        \item If $ab = 0 \in \princ{2}$, then clearly $a = b = 0$ which is in $\princ{2}$.
        \item If $ab = 2 \in \princ{2}$, then $a = 1$ and $b = 2$. Note $b = 2 \in \princ{2}$.
        \item If $ab = 4 \in \princ{2}$, then $a = 1$ and $b = 4$ or $a = b = 2$. Note $2 \in \princ{2}$ and $4 \in \princ{2}$.
        \item If $ab = 6 \in \princ{2}$, then $a = 1$ and $b = 6$ or $a = 2$ and $b = 3$. Note $2 \in \princ{2}$ and $6 \in \princ{2}$.
    \end{itemize}
    In all cases, we see that if $ab \in \princ{2}$, then at least one of $a$ or $b$ is also in $\princ{2}$. Thus $\princ{2}$ is a prime ideal of $\Z_8$.
\begin{mdframed}
    What conditions must be placed on the positive integer $n$ so that $n\Z$ is a maximal ideal in $\Z$?
\end{mdframed}
\textbf{Solution}:\newline
 Note that $\Z$ is a PID (\myref{prop-Z-is-PID}). We claim that $n$ has to be prime. By way of contradiction suppose $n$ is composite, meaning $n = ab$ where $2 \leq a,b < n$. Note $\princ{n} = \princ{ab} = \{\dots, -ab, 0, ab, \dots\}$. Observe that
    \begin{align*}
        \princ{a} &= \{\dots, -a(b+1), -ab, -a(b-1), \dots, -a,\\
        &\quad\quad0, a, \dots, a(b-1), ab, a(b+1), \dots\}
    \end{align*}
    so $\princ{n} \subset \princ{a}$. Similarly, $\princ{n} \subset \princ{b}$. However, as $\princ{n}$ is a maximal ideal, there does not exist a positive integer $k$ such that $\princ{n} \subset \princ{k} \subset \Z$. This contradicts the fact that we have both $\princ{n} \subset \princ{a}$ and $\princ{n} \subset \princ{b}$. Therefore, $n$ has to be prime.
\begin{mdframed}
    Is $\princ{3 - i}$ a prime ideal in the Gaussian integers?
\end{mdframed}
\textbf{Solution}:\newline
 No. Let $I = \princ{3-i}$. Observe that
    \begin{align*}
        ((1+i)+I)((1-2i)+I) &= (1+i)(1-2i) + I\\
        &= (1-2i+i-2i^2) + I\\
        &= \underbrace{(3 - i)}_{\text{In }I} + I\\
        &= 0 + I
    \end{align*}
    so $((1+i)+I)$ and $((1-2i)+I)$ are zero divisors in $\Z[i]/I$. Therefore $\Z[i]/I$ is not an integral domain, meaning that $I$ is not a prime ideal (\myref{thrm-prime-ideal-iff-quotient-ring-is-integral-domain}).
\begin{mdframed}
    Let $R$ be a finite commutative ring with identity, and let $P$ be a prime ideal in $R$. Show that $P$ is maximal in $R$.
\end{mdframed}
\textbf{Solution}:\newline
 If $P$ is a prime ideal of $R$, then $R/P$ is an integral domain by \myref{thrm-prime-ideal-iff-quotient-ring-is-integral-domain}. Now $R$ is finite, meaning that $R/P$ is finite. Therefore, by \myref{thrm-finite-integral-domain-is-field}, $R/P$ is a field which therefore means that $P$ is maximal by \myref{thrm-maximal-ideal-iff-quotient-ring-is-field}.
\begin{mdframed}
    Let $R$ be a commutative ring and $A$ a non-empty subset of $R$. Show that $\Ann{R}{A}$ is an ideal of $R$.
\end{mdframed}
\textbf{Solution}:\newline
 We consider the test for ideal (\myref{thrm-test-for-ideal}) to prove that $\Ann{R}{A}$ is an ideal of $R$. We note that $\Ann{R}{A}$ is non-empty since 0 is in $\Ann{R}{A}$ (because $0a = 0$ for any $a \in A$).

    Take any $r, s \in \Ann{R}{A}$, and an $a \in A$. Then one sees clearly that $(r-s)a = rs - sa = 0 - 0 = 0$ so $r-s \in \Ann{R}{A}$.

    Now take an $r \in \Ann{R}{A}$, an $a \in A$, and a $x \in R$. Note $(rx)a = (xr)a = x(ra) = x0 = 0$ since $R$ is commutative, which means that $rx, xr \in \Ann{R}{A}$.

    By the test for ideal, $\Ann{R}{A}$ is an ideal of $R$.

\section*{Problems}
\begin{mdframed}
    Find $\Ann{\Z_{36}}{\{15\}}$.
\end{mdframed}
\textbf{Solution}:\newline
 Note that $36 = 2^2 \times 3^2$. So
    \begin{align*}
        \Ann{\Z_{36}}{\{15\}} &= \{r \in \Z_{36} \vert 15r = 0\}\\
        &= \{r \in \Z_{15} \vert 3(5r) = 0\}\\
        &= \{r \in \Z_{15} \vert r \text{ is a multiple of }2^2\times3 = 12\}\\
        &= \{0,12,24\}.
    \end{align*}
\begin{mdframed}
    Let $S = \{a + 2bi \vert a, b \in \Z\}$. Show that $S$ is a subring of $\Z[i]$ but not an ideal of $\Z[i]$.
\end{mdframed}
\textbf{Solution}:\newline
 We first show that $S$ is a subring of $\Z[i]$.
    \begin{itemize}
        \item The identity of $\Z[i]$ is $0 = 0 + 2(0)i \in S$.
        \item For any $a+2bi, c+2di \in S$, clearly $a+2bi + (-(c + 2di)) = (a-c) + 2(b-d)i \in S$.
        \item For any $a+2bi, c+2di \in S$, one sees that
        \begin{align*}
            (a+2bi)(c+2di) &= ac + 2adi + 2bci + 4bdi^2\\
            &= (ac-4bd) + 2(ad+bc)i\\
            &\in S.
        \end{align*}
    \end{itemize}
    Therefore $S$ is a subring of $\Z[i]$.

    We now show that $S$ is not an ideal of $\Z[i]$. Consider $1+2i \in \S$ and $1+i \in \Z[i]$. Then
    \begin{align*}
        (1+2i)(1+i) &= 1+i+2i+2i^2\\
        &= -1 + 3i\\
        &\notin S
    \end{align*}
    so there exists a $s \in S$ and a $r \in \Z[i]$ such that $rs\notin S$, meaning that $S$ is not a left ideal (and hence is not an ideal).
\begin{mdframed}
    Consider
    \[
        I = \left\{\begin{pmatrix}2a&2b\\2c&2d\end{pmatrix} \vert a,b,c,d \in \Z\right\}.
    \]
    Show that $I$ is an ideal of $\Mn{2}{\Z}$.
\end{mdframed}
\textbf{Solution}:\newline
 We consider the test for ideal (\myref{thrm-test-for-ideal}).
    \begin{itemize}
        \item Note that $\begin{pmatrix}0&0\\0&0\end{pmatrix}=\begin{pmatrix}2(0)&2(0)\\2(0)&(0)\end{pmatrix}$ is in $I$ so $I$ is non-empty.
        \item $\begin{pmatrix}2a&2b\\2c&2d\end{pmatrix}-\begin{pmatrix}2e&2f\\2g&2h\end{pmatrix} = \begin{pmatrix}2(a-e)&2(b-f)\\2(c-g)&2(d-h)\end{pmatrix} \in I$.
        \item To show left ideal, take $\begin{pmatrix}2a&2b\\2c&2d\end{pmatrix} \in I$ and $\begin{pmatrix}e&f\\g&h\end{pmatrix} \in \Mn{2}{\Z}$. Then
        \begin{align*}
            \begin{pmatrix}2a&2b\\2c&2d\end{pmatrix}\begin{pmatrix}e&f\\g&h\end{pmatrix} &= \begin{pmatrix}2ae+2bg&2af+2bh\\2ce+2dg&2cf+2dh\end{pmatrix}\\
            &= \begin{pmatrix}2(ae+bg)&2(af+bh)\\2(ce+dg)&2(cf+dh)\end{pmatrix}\\
            &\in I
        \end{align*}
        so $I$ is a left ideal of $\Mn{2}{\Z}$.
        \item To show right ideal, take $\begin{pmatrix}a&b\\c&d\end{pmatrix} \in \Mn{2}{\Z}$ and $\begin{pmatrix}2e&2f\\2g&2h\end{pmatrix} \in I$. Then
        \begin{align*}
            \begin{pmatrix}a&b\\c&d\end{pmatrix}\begin{pmatrix}2e&2f\\2g&2h\end{pmatrix} &= \begin{pmatrix}2ae+2bg&2af+2bh\\2ce+2dg&2cf+2dh\end{pmatrix}\\
            &= \begin{pmatrix}2(ae+bg)&2(af+bh)\\2(ce+dg)&2(cf+dh)\end{pmatrix}\\
            &\in I
        \end{align*}
        so $I$ is a right ideal of $\Mn{2}{\Z}$.
    \end{itemize}
    Therefore by the test for ideal we have $I$ is an ideal of $\Mn{2}{\Z}$.
\begin{mdframed}
    Let $R$ be a commutative ring with identity and $I$ be an ideal of $R$.
    \begin{partquestions}{\alph*}
        \item Prove that if $R$ is a field then $I$ is either the trivial ring or $R$ (i.e., $R$ has no proper ideals). Hence prove that any field is a PID.
        \item Prove that if $R$ has no proper ideals, then $R$ is a field.
    \end{partquestions}
\end{mdframed}
\textbf{Solution}:\newline
 \begin{partquestions}{\alph*}
        \item Suppose $I$ is not the trivial ring; we want to show that $I = R$. Since $I$ is non-trivial there there exists a non-zero element $a$ in $I$. Note that $a^{-1}$ exists since $R$ is a field, so $a$ is a unit. By \myref{prop-ideal-contains-unit-iff-ideal-is-whole-ring} this means $I = R$. Note that $\{0\} = \princ{0}$ and $R = \princ{1}$ by \myref{exercise-trivial-ideal-and-whole-ring-are-principal-ideals}, so $R$ is indeed a PID.

        \item Take a non-zero $x \in R$ and note that $\princ{x}$ is a non-trivial ideal. Since there are no proper ideals in $R$, thus $\princ{x} = R$. This means that $1 \in \princ{x}$ (since $\princ{x} = R$ is a ring with identity), meaning that there exists an element $r \in R$ such that $xr = 1$. Therefore $x$ is a unit.

        Since $x$ is an arbitrary non-zero element in $R$, this thus shows that all non-zero elements of the ring $R$ are units, meaning $R$ is a division ring.

        Finally, because $R$ is commutative, thus $R$ is a field.
    \end{partquestions}
\begin{mdframed}
    Let $R$ be a commutative ring and let $I$ and $J$ be ideals in $R$. Prove that
    \begin{partquestions}{\alph*}
        \item $\sqrt{\sqrt{I}} = \sqrt{I}$; and
        \item $\sqrt{I \cap J} = \sqrt{I} \cap \sqrt{J}$.
    \end{partquestions}
\end{mdframed}
\textbf{Solution}:\newline
 \begin{partquestions}{\alph*}
        \item Suppose $r \in \sqrt{\sqrt{I}}$, meaning that $r^m \in \sqrt{I}$ for some positive integer $m$, further meaning that $(r^m)^n \in I$ for some positive integer $n$. Note $(r^m)^n = r^{mn} \in I$, so $r \in \sqrt{I}$. Therefore $\sqrt{\sqrt{I}} \subseteq \sqrt{I}$.

        Now suppose $r \in \sqrt{I}$, meaning that $r^n \in I$ for some positive integer $n$. Note that $r = r^1 \in \sqrt{I}$, so $r \in \sqrt{\sqrt{I}}$. Hence $\sqrt{I} \subseteq \sqrt{\sqrt{I}}$.

        Therefore, since $\sqrt{\sqrt{I}} \subseteq \sqrt{I}$ and $\sqrt{I} \subseteq \sqrt{\sqrt{I}}$, thus $\sqrt{\sqrt{I}} = \sqrt{I}$.

        \item Suppose $r \in \sqrt{I\cap J}$, so $r^n \in I \cap J$ for some positive integer $n$. This means that $r^n \in I$ and $r^n \in J$. Hence $r \in \sqrt{I}$ and $r \in \sqrt{J}$ by definition of the radical, so $r \in \sqrt{I}\cap\sqrt{J}$. Thus $\sqrt{I\cap J} \subseteq \sqrt{I}\cap\sqrt{J}$.

        Now suppose $r \in \sqrt{I}\cap\sqrt{J}$, meaning that $r \in \sqrt{I}$ and $r \in \sqrt{J}$. Thus $r^m \in I$ and $r^n \in J$ for some positive integers $m$ and $n$. Note that
        \[
            (\underbrace{r^m}_{\text{In }I})^n \in I \text{ and } (\underbrace{r^n}_{\text{In }J})^m \in J
        \]
        so $r^{mn} \in I$ and $r^{mn} \in J$, meaning $r^{mn} \in I \cap J$. Thus $r \in \sqrt{I \cap J}$, showing that $\sqrt{I}\cap\sqrt{J} \subseteq \sqrt{I\cap J}$.

        Therefore $\sqrt{I}\cap\sqrt{J} = \sqrt{I\cap J}$.
    \end{partquestions}
\begin{mdframed}
    Let $m$ and $n$ be positive integers, and let $d = \gcd(m,n)$ and $l = \lcm(m,n)$. Prove the following.
    \begin{partquestions}{\alph*}
        \item $m\Z \cap n\Z = l\Z$
        \item $m\Z + n\Z = d\Z$
    \end{partquestions}
\end{mdframed}
\textbf{Solution}:\newline
 \begin{partquestions}{\alph*}
        \item Suppose $a \in m\Z\cap n\Z$. Thus $a \in m\Z$ and $a \in n\Z$, meaning that $a = mx = ny$ for some integers $x$ and $y$. Therefore $a = \lcm(m,n)z = lz$ for some integer $z$, meaning $a \in l\Z$. Hence $m\Z \cap n\Z \subseteq l\Z$.

        Now suppose $a \in l\Z$, so $a = lx$ for some integer $x$. Write $l = m\alpha = n\beta$ for some integers $\alpha$ and $\beta$. Note that
        \begin{align*}
            a &= (m\alpha)x = m(\alpha x) \in m\Z\\
            a &= (n\beta)x = n(\beta x) \in n\Z
        \end{align*}
        so $a \in m\Z \cap n\Z$. Thus $l\Z \subseteq m\Z \cap n\Z$.

        Therefore $m\Z\cap n\Z = l\Z$.

        \item Suppose $a \in m\Z + n\Z$, meaning that there exist integers $x$ and $y$ such that $a = mx + ny$. By definition of the GCD, write $m = d\alpha$ and $n = d\beta$ for some integers $\alpha$ and $\beta$. Hence
        \begin{align*}
            a &= (d\alpha)x + (d\beta)y\\
            &= d(\alpha x + \beta y)\\
            &\in d\Z
        \end{align*}
        so $m\Z + n\Z \subseteq d\Z$.

        On the other hand, suppose $a \in d\Z$, meaning $a = dt$ for some integer $t$. By B\'{e}zout's Lemma (\myref{lemma-bezout}), we may write $d = mx + ny$ for some integers $x$ and $y$. Hence
        \begin{align*}
            a &= (mx + ny)t\\
            &= m(xt) + n(yt)\\
            &\in m\Z + n\Z
        \end{align*}
        which means $d\Z \subseteq m\Z + n\Z$.

        Therefore $m\Z + n\Z = d\Z$.
    \end{partquestions}
\begin{mdframed}
    Let $R$ be a commutative ring. Prove that $R / \Nilr{R}$ has no non-zero nilpotents.
\end{mdframed}
\textbf{Solution}:\newline
 Let $r \in R$, and suppose $x = r + \Nilr{R} \in R/\Nilr{R}$ is nilpotent, i.e. there is a positive integer $n$ such that
    \[
        x^n = (r + \Nilr{R})^n = r^n + \Nilr{R} = 0 + \Nilr{R}.
    \]
    Coset Equality (\myref{lemma-coset-equality}) thus tells us that $r^n \in \Nilr{R}$. Note that $\Nilr{R}$ contains all the nilpotents of $R$. Thus $r^n$ is a nilpotent of $R$, i.e. there exists a positive integer $m$ such that $(r^n)^m = 0$. But clearly $(r^n)^m = r^{mn} = 0$, so $r$ is nilpotent, meaning $r \in \Nilr{R}$. Hence $x = r + \Nilr{R} = 0 + \Nilr{R}$, meaning that the only nilpotent of $R/\Nilr{R}$ is the zero element. Therefore $R/\Nilr{R}$ has no non-zero nilpotents.
\begin{mdframed}
    Prove that every non-trivial prime ideal is a maximal ideal in a PID.
\end{mdframed}
\textbf{Solution}:\newline
 Suppose $R$ is a PID and $I$ is a non-zero prime ideal. Let $J$ be an ideal such that $I \subseteq J \subseteq R$. Since $R$ is a PID, write $I = \princ{a}$ and $J = \princ{b}$ for some elements $a$ and $b$ in $R$. Note $a \in \princ{a} = I \subseteq J = \princ{b}$, so there exists an $r \in R$ such that $a = rb$. Now since $a = rb \in \princ{a} = I$ and $I$ is prime, therefore $r \in I$ or $b \in I$.
    \begin{itemize}
        \item If $r \in I$, write $r = sa$ for some $s \in R$. Then
        \[
            a = rb = (sa)b = (as)b = a(sb)
        \]
        since an integral domain is commutative. Thus $a - a(sb) = a(1-sb) = 0$. Now as $R$ is an integral domain thus either $a = 0$ (impossible since $a \neq 0$) or $1-sb = 0$. So $1-sb = 0$, meaning $sb = 1 \in J$ since $b \in J$. By \myref{prop-ideal-contains-unit-iff-ideal-is-whole-ring} we have $J = R$.
        \item If instead $b \in I$, take any $x \in J = \princ{b}$, so $x = rb$ for some $r \in R$. Thus $x = rb \in I$ since $b \in I$, so $J \subseteq I$. But $I \subseteq J$, so $J = I$.
    \end{itemize}
    Therefore we have shown that $I$ is maximal.
\begin{mdframed}
    Prove \myref{prop-principal-ideals-equal-iff-associates}.
\end{mdframed}
\textbf{Solution}:\newline
 First we work in the forward direction. Suppose $\princ{a} = \princ{b}$. As $a \in \princ{a} = \princ{b}$, thus $a = bx$ for some $x \in R$. Also, as $b \in \princ{b} = \princ{a}$, thus $b = ay$ for some $y \in R$. Therefore
    \[
        b = ay = (bx)y = b(xy)
    \]
    which means $xy = 1$. Thus $x$ and $y$ are units, meaning $a = bx$ with $x$ being a unit.

    Now we work in the reverse direction; suppose $a = bu$ for some unit $u$ in $D$.
    \begin{itemize}
        \item Take $r \in \princ{a}$, so $r = ax$ for some $x$ in $D$. Thus $r = (bu)x = b(ux) \in \princ{b}$, so $\princ{a} \subseteq \princ{b}$.
        \item Note $b = au^{-1}$ since $u$ is a unit. Take $s \in \princ{b}$, so $s = by$ for some $y$ in $D$. But as $b = au^{-1}$, hence $s = (au^{-1})y = a(u^{-1}y) \in \princ{a}$, so $\princ{b} \subseteq \princ{a}$.
    \end{itemize}
    Therefore we see that $\princ{a} = \princ{b}$.

\chapter{Ring Homomorphisms}
\section*{Exercises}
\begin{mdframed}
    Let $\phi: \Mn{2}{\Z} \to \Z$ be such that
    \[
        \phi\left(\begin{pmatrix}a&b\\c&d\end{pmatrix}\right) = a+d.
    \]
    Is $\phi$ a ring homomorphism?
\end{mdframed}
\textbf{Solution}:\newline
 We show that $\phi$ is not a ring homomorphism. Consider the matrices
    \[
        \begin{pmatrix}1&0\\0&0\end{pmatrix} \text{ and } \begin{pmatrix}0&0\\0&1\end{pmatrix}.
    \]
    Note that $\phi\left(\begin{pmatrix}1&0\\0&0\end{pmatrix}\right) = 1 + 0 = 1$ and $\phi\left(\begin{pmatrix}0&0\\0&1\end{pmatrix}\right) = 0 + 1 = 1$, so
    \[
        \phi\left(\begin{pmatrix}1&0\\0&0\end{pmatrix}\right)\phi\left(\begin{pmatrix}0&0\\0&1\end{pmatrix}\right) = 1 \times 1 = 1.
    \]
    However, note
    \[
        \begin{pmatrix}1&0\\0&0\end{pmatrix}\begin{pmatrix}0&0\\0&1\end{pmatrix} = \begin{pmatrix}0&0\\0&0\end{pmatrix}
    \]
    so $\phi\left(\begin{pmatrix}1&0\\0&0\end{pmatrix}\begin{pmatrix}0&0\\0&1\end{pmatrix}\right) = 0$. Thus $\phi$ is not a homomorphism.
\begin{mdframed}
    Let $R$ and $S$ be rings with additive identities $0_R$ and $0_S$ respectively. Show that the \textbf{trivial homomorphism}\index{homomorphism!trivial} $\phi: R \to S, r \mapsto 0_S$ is, indeed, a ring homomorphism.
\end{mdframed}
\textbf{Solution}:\newline
 Note
    \[
        \phi(a+b) = 0 = 0 + 0 = \phi(a) + \phi(b)
    \]
    and
    \[
        \phi(ab) = 0 = 0\times0 = \phi(a)\phi(b)
    \]
    so $\phi$ is indeed a ring homomorphism.
\begin{mdframed}
    Let $R$ be a ring. Show that the \textbf{identity homomorphism}\index{homomorphism!identity} $\id: R \to R, r \mapsto r$ is a ring endomorphism.
\end{mdframed}
\textbf{Solution}:\newline
 Note
    \[
        \id(a+b) = a + b = \id(a) + \id(b)
    \]
    and
    \[
        \id(ab) = ab = \id(a)\id(b)
    \]
    so $\id$ is a ring endomorphism.
\begin{mdframed}
    Show that the identity homomorphism is an automorphism.
\end{mdframed}
\textbf{Solution}:\newline
 We have shown that the identity homomorphism is a homomorphism, so we just need to prove that it is a bijection.
    \begin{itemize}
        \item \textbf{Injective}: Suppose $a, b \in R$ are such that $\phi(a) = \phi(b)$. But since $\phi(x) = x$ thus $a = b$.
        \item \textbf{Surjective}: As $\phi(x) = x$ thus any element is its own pre-image.
    \end{itemize}
    Therefore the identity homomorphism is an isomorphism. As it is also an endomorphism, therefore it is an automorphism.
\begin{mdframed}
    Let $R_1$ and $R_2$ be rings, and $\phi: R_1 \to R_2$ be a ring homomorphism.
    \begin{partquestions}{\alph*}
        \item Show $\phi(0_1) = 0_2$, where $0_1$ and $0_2$ is the additive identity of $R_1$ and $R_2$ respectively.
        \item If $R_1$ and $R_2$ are division rings, then show $\phi(1_1) = 1_2$, where $1_1$ and $1_2$ is the multiplicative identity of $R_1$ and $R_2$ respectively.
    \end{partquestions}
\end{mdframed}
\textbf{Solution}:\newline
 Note that
    \[
        \phi(0_1) = \phi(0_1 + 0_1) = \phi(0_1) + \phi(0_1)
    \]
    so by 'adding' $-\phi(0_1)$ on both sides we see that $\phi(0_1) = 0_2$.
\begin{mdframed}
    Let $R_1$ and $R_2$ be rings, $x \in R_1$, and $\phi: R_1 \to R_2$ be a ring homomorphism.
    \begin{partquestions}{\alph*}
        \item Show that $\phi(-x) = -\phi(x)$.
        \item If $R_1$ and $R_2$ are division rings, show that $\phi(x^{-1}) = (\phi(x))^{-1}$.
    \end{partquestions}
\end{mdframed}
\textbf{Solution}:\newline
 Note that
    \[
        \phi(1_1) = \phi(1_1 \times 1_1) = \phi(1_1)\phi(1_1).
    \]
    Since $R_1$ and $R_2$ are division rings, we may apply $\phi(1_1)^{-1}$ on both sides to yield $\phi(1_1) = 1_2$.
\begin{mdframed}
    Let $R_1$ and $R_2$ be rings, and let $\phi: R_1 \to R_2$ be a ring homomorphism. Prove that $\ker\phi$ is an ideal of $R_1$.
\end{mdframed}
\textbf{Solution}:\newline
 \begin{partquestions}{\alph*}
        \item Notice that
        \[
            \phi(x + (-x)) = \phi(x) + \phi(-x)
        \]
        and
        \[
            \phi(x + (-x)) = \phi(0_1) = 0_2
        \]
        so subtracting $-\phi(x)$ on both sides yields $\phi(-x) = -\phi(x)$.

        \item Notice that
        \[
            \phi(xx^{-1}) = \phi(x)\phi(x^{-1})
        \]
        and
        \[
            \phi(xx^{-1}) = \phi(1_1) = 1_2
        \]
        so applying $\phi(x)^{-1}$ on the left on both sides $\phi(x^{-1}) = \phi(x)^{-1}$.
    \end{partquestions}
\begin{mdframed}
    Show that $\Z_n \cong \Z/n\Z$ by considering the ring homomorphism
    \[
        \phi: \Z \to \Z_n, m \mapsto m \mod n
    \]
    and by using the FRIT.
\end{mdframed}
\textbf{Solution}:\newline
 Recall that $\{0_2\}$, the trivial ideal, is an ideal of $R_2$, where $0_2$ is the additive identity of $R_2$. Therefore $\ker\phi = \phi^{-1}(\{0\})$ is an ideal of $R_1$ by \myref{prop-inverse-homomorphism-on-ideal-is-ideal}.
\begin{mdframed}
    In the above example, show that $\phi(n) = n$ for all positive integers $n$.
\end{mdframed}
\textbf{Solution}:\newline
 We show that $\phi$ is surjective. Note that for any $k \in \Z_n$, we have $k \leq n$. Thus, $\phi(k) = k$, so $\phi$ is surjective.

    We now find the kernel of $\phi$.
    \begin{align*}
        \ker\phi &= \{m \in \Z \vert \phi(m) = 0\}\\
        &= \{m \in \Z \vert m \cong 0 \pmod n\}\\
        &= \{kn \vert k \in \Z\}\\
        &= n\Z.
    \end{align*}

    The FRIT (\myref{thrm-ring-isomorphism-1}) on $\phi$ tells us that
    \[
        \Z/n\Z \cong \Z_n.
    \]
\begin{mdframed}
    Suppose $R$ and $R'$ are rings such that $\Q$ is a subring of both $R$ and $R'$. Let $\phi: R \to R'$ be a ring homomorphism such that $\phi(1) = 1$. Show that for any $q \in \Q$ we have $\phi(q) = q$.
\end{mdframed}
\textbf{Solution}:\newline
 Note that $\phi(1) = 1$ is given. Now suppose $\phi(k) = k$ for some positive integer $k$. Then
    \[
        \phi(k+1) = \phi(k) + \phi(1) = k + 1
    \]
    by induction hypothesis and by the base case. Thus by mathematical induction we prove the statement.

\section*{Problems}
\begin{mdframed}
    Let $R$ be a ring.
    \begin{partquestions}{\roman*}
        \item Show that $R/\{0\} \cong R$.
        \item Prove that $R$ is an integral domain if and only if $\{0\}$ is a prime ideal.
        \item Prove that $R$ is a field if and only if $\{0\}$ is a maximal ideal.
    \end{partquestions}
\end{mdframed}
\textbf{Solution}:\newline
 \begin{partquestions}{\roman*}
        \item Consider the identity homomorphism $\phi: R \to R, r \mapsto r$. Clearly $\phi$ is surjective. Note that
        \begin{align*}
            \ker\phi &= \{r \in R \vert \phi(r) = 0\}\\
            &= \{r \in R \vert r = 0\}\\
            &= \{0\}.
        \end{align*}
        By the FRIT (\myref{thrm-ring-isomorphism-1}),
        \[
            R / \{0\} \cong R
        \]
        which is what we wanted to show.

        \item By \myref{thrm-prime-ideal-iff-quotient-ring-is-integral-domain}, $\{0\}$ is prime if and only if $R/\{0\}$ is an integral domain. But since $R/\{0\} \cong R$, thus $\{0\}$ is prime if and only if $R$ is an integral domain.

        \item By \myref{thrm-maximal-ideal-iff-quotient-ring-is-field}, $\{0\}$ is maximal if and only if $R/\{0\} \cong R$ is a field.
    \end{partquestions}
\begin{mdframed}
    Find all ring endomorphisms of $\Q$.
\end{mdframed}
\textbf{Solution}:\newline
 Let $\phi: \Q \to Q$ be a ring endomorphism. From \myref{prop-homomorphism-on-multiplicative-identity-is-idempotent}, we know that $\phi(1)$ is an idempotent in $\Q$, meaning that $\phi(1) = 0$ or $\phi(1) = 1$.

    Clearly if $\phi(1) = 0$ then
    \[
        \phi(x) = \phi(1x) = \phi(1)\phi(x) = 0\phi(x) = 0
    \]
    so $\phi(x)$ is the trivial homomorphism.

    If instead $\phi(1) = 1$, then from \myref{exercise-homomorphism-over-Q-fixes-elements-of-Q} we know that $\phi(x) = x$ for all $x \in Q$, i.e. $\phi$ is the identity homomorphism.

    Therefore the only two endomorphisms in $\Q$ are the trivial and identity homomorphisms.
\begin{mdframed}
    Show $\Z^2 \not\cong \Q$.
\end{mdframed}
\textbf{Solution}:\newline
 By way of contradiction, suppose $\phi: \Z^2 \to \Q$ is an isomorphism. Then \myref{prop-ring-image-of-additive-identity-is-additive-identity} tells us that $\phi((0,0)) = 0$. Also \myref{prop-homomorphism-on-multiplicative-identity-is-idempotent} tells us that $\phi((1,1))$ is an idempotent in $\Q$, which means that $\phi((1,1)) = 0$ or $\phi((1,1)) = 1$.

    If $\phi((1,1)) = 0$ then we have $\phi((0,0)) = \phi((1,1)) = 0$, so $\phi$ is not injective, which thus means that $\phi$ is not an isomorphism, a contradiction.

    Thus $\phi((1,1)) = 1$. Note that
    \begin{align*}
        \phi((1,1)) &= \phi((0,1)) + \phi((1,0)) = 1,\\
        \phi((0,0)) &= \phi((0,1)) \times \phi((1,0)) = 0
    \end{align*}
    so this means that either $\phi((0,1)) = 0$ or $\phi((1,0)) = 0$. But this means that either $\phi((0,0)) = \phi((0,1)) = 0$ or $\phi((0,0)) = \phi((1,0)) = 0$ which again contradicts the fact that $\phi$ is injective and hence an isomorphism.

    Therefore $\Z^2 \not\cong \Q$.
\begin{mdframed}
    Show $\Q[\sqrt2] \not\cong \Q[\sqrt3]$.
\end{mdframed}
\textbf{Solution}:\newline
 BWOC, suppose $\phi: \Q[\sqrt2] \to \Q[\sqrt3]$ is an isomorphism.

    Suppose $\phi(\sqrt2) = a + b\sqrt3$ where $a$ and $b$ are rational numbers. Note that $\phi(2) = 2$ by \myref{exercise-homomorphism-over-Q-fixes-elements-of-Q}, so
    \[
        2 = \phi(2) = \phi({\sqrt2}^2) = \left(\phi(\sqrt2)\right)^2 = a^2 + 2\sqrt3ab + 3b^2.
    \]
    This means that $2ab = 0$, which means that $a = 0$ or $b = 0$.

    If $a = 0$ then $2 = 3b^2$ which means $b = \pm\sqrt{\frac23}$. But $\sqrt{\frac23}$ is not a rational number, contradicting the fact that $b$ is a rational number.

    Thus $b = 0$. But this means that
    \[
        \phi(\sqrt2) = a = \phi(a),
    \]
    since $a \in \Q$, which means $\phi$ is not injective, contradicting the fact that $\phi$ is an isomorphism.

    Therefore $\Q[\sqrt2] \not\cong \Q[\sqrt3]$.
\begin{mdframed}
    Let
    \[
        R = \left\{\begin{pmatrix}a&0\\0&b\end{pmatrix}\vert a,b \in \Z\right\},
    \]
    which is a subring of $\Mn{2}{\Z}$. Show that $R \cong \Z^2$.
\end{mdframed}
\textbf{Solution}:\newline
 Let $\phi: \Z^2 \to R, (a,b)\mapsto \begin{pmatrix}a&0\\0&b\end{pmatrix}$. We show that $\phi$ is a bijective ring homomorphism.
    \begin{itemize}
        \item \textbf{Homomorphism}:
        \begin{align*}
            \phi((a,b) + (x,y)) &= \phi((a+x,b+y))\\
            &= \begin{pmatrix}a+x&0\\0&b+y\end{pmatrix}\\
            &= \begin{pmatrix}a&0\\0&b\end{pmatrix} + \begin{pmatrix}x&0\\0&y\end{pmatrix}\\
            &= \phi((a,b)) + \phi((x,y))
        \end{align*}
        and
        \begin{align*}
            \phi((a,b)(x,y)) &= \phi((ax,by))\\
            &= \begin{pmatrix}ax&0\\0&by\end{pmatrix}\\
            &= \begin{pmatrix}a&0\\0&b\end{pmatrix}\begin{pmatrix}x&0\\0&y\end{pmatrix}\\
            &= \phi((a,b))\phi((x,y))
        \end{align*}
        so $\phi$ is a ring homomorphism.

        \item \textbf{Injective}: Let $(a, b), (x, y) \in \Z^2$ such that $\phi((a,b)) = \phi((x,y))$. Therefore $\begin{pmatrix}a&0\\0&b\end{pmatrix} = \begin{pmatrix}x&0\\0&y\end{pmatrix}$. Thus $a = x$ and $b = y$, which means $(a,b) = (x,y)$. Therefore $\phi$ is injective.

        \item \textbf{Surjective}: Let $\begin{pmatrix}a&0\\0&b\end{pmatrix} \in R$. Clearly one sees that $\phi((a, b)) = \begin{pmatrix}a&0\\0&b\end{pmatrix}$ so $\phi$ is surjective.
    \end{itemize}

    Therefore $\phi$ is a ring isomorphism, which means $R \cong \Z^2$.
\begin{mdframed}
    Let $R$ and $R'$ be rings, and let $\phi: R \to R'$ be a ring isomorphism. Prove or disprove the following statements.
    \begin{partquestions}{\alph*}
        \item $\phi^{-1}: R' \to R$ is a ring isomorphism.
        \item If $R$ has a subring with $n$ elements, then so does $R'$.
        \item If $R$ has an ideal, then so does $R'$.
    \end{partquestions}
\end{mdframed}
\textbf{Solution}:\newline
 \begin{partquestions}{\alph*}
        \item We prove the statement. Note that since $\phi$ is bijective, thus $\phi^{-1}$ is bijective. We just need to show that $\phi^{-1}$ is a homomorphism.

        Let $u,v\in R'$. Then there is a $x,y \in R$ such that $\phi(x) = u$ and $\phi(y) = v$. Note
        \begin{align*}
            \phi^{-1}(u + v) &= \phi^{-1}(\phi(x) + \phi(y))\\
            &= \phi^{-1}(\phi(x + y))\\
            &= x + y\\
            &= \phi^{-1}(u) + \phi^{-1}(v)
        \end{align*}
        and
        \begin{align*}
            \phi^{-1}(uv) &= \phi^{-1}(\phi(x)\phi(y))\\
            &= \phi^{-1}(\phi(xy))\\
            &= xy\\
            &= \phi^{-1}(u)\phi^{-1}(v)
        \end{align*}
        so $\phi^{-1}$ is a ring homomorphism. Therefore $\phi^{-1}$ is a ring isomorphism.

        \item We prove the statement. Suppose $S$ is a subring of $R$ with $n$ elements. Consider $\phi(S)$; by \myref{prop-homomorphism-on-subring-is-subring} we know $\phi(S)$ is a subring of $R'$. Also, since $\phi$ is a bijection, thus $S$ and $\phi(S)$ are equinumerous. Therefore $\phi(S)$ is a subring of $R'$ with $n$ elements.

        \item Suppose $I$ is an ideal of $R$. Consider $\phi(I)$; by \myref{prop-homomorphism-on-subring-is-subring} we know $\phi(I)$ is a subring of $R'$. We just need to check that $\phi(I)$ is an ideal of $R'$.

        Let $r' \in R'$ and $i' \in \phi(I)$. Since $\phi$ is surjective, there is an $r \in R$ and an $i \in I$ such that $r' = \phi(r)$ and $i' = \phi(i)$. Note
        \begin{align*}
            r'i' = \phi(r)\phi(i) = \phi(\underbrace{ri}_{\text{In }I}) \in \phi(I)\\
            i'r' = \phi(i)\phi(r) = \phi(\underbrace{ir}_{\text{In }I}) \in \phi(I)
        \end{align*}
        so $\phi(I)$ is an ideal of $R'$.
    \end{partquestions}
\begin{mdframed}
    Find all ring endomorphisms of $\Z_{10}$.\newline
    Hence find all ring automorphisms $\psi: \Z_{10} \to \Z_{10}$.
\end{mdframed}
\textbf{Solution}:\newline
 Suppose $\phi: \Z_{10}\to\Z_{10}$ is a ring homomorphism. Let $a = \phi(1)$. By \myref{prop-homomorphism-on-multiplicative-identity-is-idempotent} we know that $a^2 = \phi(1)^2 = \phi(1) = a$. Once again, we cannot assume that 0 and 1 are the only idempotents in $\Z_{10}$; by exhaustion we see that
    \begin{itemize}
        \item $0^2 = 0$;
        \item $1^2 = 1$;
        \item $5^2 = 25 = 5$; and
        \item $6^2 = 36 = 6$
    \end{itemize}
    so the idempotents in $\Z_{10}$ are 0, 1, 5, and 6.

    Recall from \myref{example-homomorphisms-from-Z12-to-Z28} that $|\phi(1)|_+$ divides $|1|_+$ (\myref{exercise-order-of-homomorphism-divides-order}) so $|\phi(1)|_+$ divides 10, and $|k|_+ = \frac{n}{\gcd(k,10)}$ (\myref{thrm-order-of-element-in-cyclic-group}). Observe that
    \begin{itemize}
        \item $|0|_+ = 1$ which divides 10;
        \item $|1|_+ = 10$ which divides 10;
        \item $|5|_+ = \frac{10}{\gcd(5,10)} = \frac{10}{5} = 2$ which divides 10; and
        \item $|6|_+ = \frac{10}{\gcd(6,10)} = \frac{10}{2} = 5$ which divides 10.
    \end{itemize}
    Therefore the possible values of $\phi(1)$ are 0, 1, 5, and 6.

    Note that
    \begin{align*}
        \phi(n) &= \phi(\underbrace{1+1+\cdots+1}_{n \text{ times}})\\
        &= \underbrace{\phi(1)+\phi(1)+\cdots+\phi(1)}_{n \text{ times}}\\
        &= \underbrace{a + a + \cdots + a}_{n \text{ times}}\\
        &= na
    \end{align*}
    so the only homomorphisms in $\phi(n) = 0$, $\phi(n) = n$, $\phi(n) = 5n$, and $\phi(n) = 6n$.

    Now consider the possibility that $\psi: \Z_{10} \to \Z_{10}$ is an automorphism. We require that $\psi$ is an isomorphism.
    \begin{itemize}
        \item Clearly $\psi(n) = 0$ is not a valid isomorphism since $\psi(0) = \psi(1) = 0$ which means that $\psi$ is not an injective.
        \item $\psi(n) = n$, the identity homomorphism, is an isomorphism by \myref{exercise-identity-homomorphism-is-an-isomorphism}.
        \item $\psi(n) = 5n$ is not possible since $\psi(0) = 0$ and $\psi(2) = 10 = 0$, so $\psi$ is not injective.
        \item $\psi(n) = 6n$ is not possible since $\psi(0) = 0$ and $\psi(5) = 30 = 0$, so $\psi$ is not injective.
    \end{itemize}
    Therefore the only automorphism $\psi:\Z_{10} \to \Z_{10}$ is $\psi(n) = n$.
\begin{mdframed}
    Find all ring endomorphisms of $\Q[\sqrt3]$.\newline
    Hence find all ring automorphisms $\psi: \Q[\sqrt3] \to \Q[\sqrt3]$.
\end{mdframed}
\textbf{Solution}:\newline
 Let $\phi: \Q[\sqrt3] \to \Q[\sqrt3]$ be an endomorphism. By \myref{prop-homomorphism-on-multiplicative-identity-is-idempotent} we know that $\phi(1)^2 = \phi(1)$, so $\phi(1) = 0$ or $\psi(1) = 1$ (since 0 and 1 are the only idempotents in $\Q[\sqrt3]$).

    If $\phi(1) = 0$ then
    \[
        \phi(a+b\sqrt3) = \phi(1)\phi(a+b\sqrt3) = 0\phi(a+b\sqrt3) = 0
    \]
    so $\phi$ is the trivial homomorphism.

    If instead $\phi(1) = 1$, then $\phi(q) = q$ for all $q \in \Q$ (\myref{exercise-homomorphism-over-Q-fixes-elements-of-Q}). Let $\phi(\sqrt3) = a+b\sqrt3$ where $a$ and $b$ are rational numbers. Note
    \[
        3 = \phi(3) = \phi((\sqrt3)^3) = \left(\phi(\sqrt3)\right)^2 = a^2+2\sqrt3ab+3b^2
    \]
    which means $2ab = 0$, so $a = 0$ or $b = 0$. If $b = 0$ then $3 = a^2$ which means $a = \pm\sqrt3$, a contradiction since $\sqrt3$ is not a rational number. Therefore $a = 0$, so $3 = 3b^2$ which means $b = \pm1$. Thus $\phi(\sqrt3) = \sqrt3$ or $\phi(\sqrt3) = -\sqrt3$.

    Thus we have 3 possibilities:
    \begin{itemize}
        \item $\phi(a+b\sqrt3) = 0$;
        \item $\phi(a+b\sqrt3) = a+b\sqrt3$; and
        \item $\phi(a+b\sqrt3) = a-b\sqrt3$.
    \end{itemize}
    The first two possibilities are the trivial and identity homomorphism respectively, so we just need to check whether the last possibility is a homomorphism. Because
    \begin{align*}
        \phi((a+b\sqrt3) + (x+y\sqrt3)) &= \phi((a+x)+(b+y)\sqrt3)\\
        &= (a+x)-(b+y)\sqrt3\\
        &= (a-b\sqrt3) + (x-y\sqrt3)\\
        &= \phi(a+b\sqrt3) + \phi(x+y\sqrt3)
    \end{align*}
    and
    \begin{align*}
        \phi((a+b\sqrt3)(x+y\sqrt3)) &= \phi((ax+3by)+(ay+bx)\sqrt3)\\
        &= (ax+3by)-(ay+bx)\sqrt3\\
        &= (a-b\sqrt3)(x-y\sqrt3)\\
        &= \phi(a+b\sqrt3)\phi(x+y\sqrt3)
    \end{align*}
    so $\phi(a+b\sqrt3) = a-b\sqrt3$ is indeed a homomorphism.

    Therefore the 3 possible homomorphisms are $\phi(a+b\sqrt3) = 0$, $\phi(a+b\sqrt3) = a+b\sqrt3$, and $\phi(a+b\sqrt3) = a-b\sqrt3$.

    Now consider the possibility that $\psi: \Q[\sqrt3] \to \Q[\sqrt3]$ is an automorphism, meaning that $\psi$ has to be at least an isomorphism. Clearly the trivial homomorphism is not, while the identity homomorphism is (\myref{exercise-identity-homomorphism-is-an-isomorphism}). We consider $\psi(a+b\sqrt3) = a-b\sqrt3$.
    \begin{itemize}
        \item \textbf{Injective}: Suppose $a+b\sqrt3, x+y\sqrt3 \in \Q[\sqrt3]$ such that $\phi(a+b\sqrt3) = \phi(x+y\sqrt3)$. Thus $a - b\sqrt3 = x - y\sqrt3$, which clearly means $a = x$ and $b = y$. Therefore $a+b\sqrt3 = x+y\sqrt3$ which means that $\psi$ is injective.
        \item \textbf{Surjective}: For any $x+y\sqrt3 \in \Q[\sqrt3]$, note that $x-y\sqrt3 \in \Q[\sqrt3]$ and $\psi(x-y\sqrt3) = x+y\sqrt3$, so $\psi$ is surjective.
    \end{itemize}
    Therefore $\psi(a+b\sqrt3) = a-b\sqrt3$ is also an isomorphism. Hence, the two automorphisms $\psi: \Q[\sqrt3] \to \Q[\sqrt3]$ are $\psi(a+b\sqrt3) = a+b\sqrt3$ and $\psi(a+b\sqrt3) = a-b\sqrt3$.
\begin{mdframed}
    Let $R$ and $R'$ be commutative rings, $I$ be an ideal of $R$, and $\phi: R\to R'$ be a ring homomorphism.
    \begin{partquestions}{\roman*}
        \item Show that $\phi(\sqrt I) \subseteq \sqrt{\phi(I)}$.
        \item If $\phi$ is surjective with $\ker\phi \subseteq I$, prove that $\phi(\sqrt I) = \sqrt{\phi(I)}$.
    \end{partquestions}
\end{mdframed}
\textbf{Solution}:\newline
 \begin{partquestions}{\roman*}
        \item Let $x \in \phi(\sqrt I)$. Therefore there is an $a \in \sqrt{I}$ such that $\phi(a) = x$. Now by definition of $\sqrt{I}$, this means that there is a positive integer $n$ such that $a^n \in I$. Note that
        \[
            x^n = \left(\phi(a)\right)^n = \phi(a^n) \in \phi(I)
        \]
        which means $x \in \sqrt{\phi(I)}$. Therefore $\phi(\sqrt I) \subseteq \sqrt{\phi(I)}$.

        \item Let $y \in \sqrt{\phi(I)}$, so there is a positive integer $n$ such that $y^n \in \phi(I)$. Thus there exists an $a \in I$ such that $\phi(a) = y^n$.

        Note $y \in R'$, and since $\phi$ is surjective, therefore there is an $x \in R$ such that $\phi(x) = y$. Therefore
        \[
            \phi(a) = y^n = \left(\phi(x)\right)^n = \phi\left(x^n\right)
        \]
        which thus means $\phi(a-x^n) = 0$. Hence $a-x^n \in \ker\phi \subseteq I$, so $a-x^n \in I$. Since $a \in I$, this means that $x^n \in I$, so $x \in \sqrt{I}$.

        So,
        \[
            y = \phi(x) = \phi(\sqrt I)
        \]
        which thus means $\sqrt{\phi(I)} \subseteq \phi(\sqrt I)$. But by (i), we found out that $\phi(\sqrt I) \subseteq \sqrt{\phi(I)}$, so $\phi(\sqrt I) = \sqrt{\phi(I)}$ as needed.
    \end{partquestions}
\begin{mdframed}
    Let $R$ be a ring with subring $S$ and ideal $I$.
    \begin{partquestions}{\roman*}
        \item Prove $S+I$ is a subring of $R$.
        \item Prove $S \cap I$ is an ideal of $S$.
        \item Prove $S/(S\cap I)\cong (S+I)/I$.
    \end{partquestions}
\end{mdframed}
\textbf{Solution}:\newline
 \begin{partquestions}{\roman*}
        \item We first show $(S+I, +) \leq (R,+)$.
        \begin{itemize}
            \item $0 = 0 + 0 \in S + I$ so $S + I \neq \emptyset$.
            \item For any $s_1, s_2 \in S$ and $i_1, i_2 \in I$ we see that
            \[
                (s_1+i_1) - (s_2 + i_2) = (\underbrace{s_1 - s_2}_{\text{In }S}) + (\underbrace{i_1 + i_2}_{\text{In }I}) \in S + I
            \]
        \end{itemize}
        Therefore $(S+I, +) \leq (R,+)$ by the subgroup test (\myref{thrm-subgroup-test}).

        We now show $S+I$ is closed under multiplication. Let $s_1, s_2 \in S$ and $i_1, i_2 \in I$. Then
        \[
            (s_1 + i_1)(s_2 + i_2) = s_1s_2 + s_1i_2 + i_1s_2 + i_1i_2.
        \]
        Note $s_1i_2 \in I$ since $I$ is a left ideal, and $i_1s_2 \in I$ since $I$ is a right ideal, so
        \[
            (s_1 + i_1)(s_2 + i_2) = \underbrace{s_1s_2}_{\text{In }S} + \underbrace{s_1i_2 + i_1s_2 + i_1i_2}_{\text{In }I} \in S + I.
        \]

        Therefore $S+I$ is a subring of $R$.

        \item We consider the test for ideal.
        \begin{itemize}
            \item $S \cap I \neq \emptyset$ since $0 \in S$ and $0 \in I$ so $0 \in S \cap I$.
            \item Let $a$ and $b$ are in $S \cap I$. Thus $a, b \in S$ and $a, b \in I$, which means $a - b \in S$ and $a - b \in I$ as $S$ and $I$ are both subrings. Therefore $a - b \in S \cap I$.
            \item Let $s \in S$ and $a \in S \cap I$, which means $a \in S$ and $a \in I$.
            \begin{itemize}
                \item Note $sa \in S$ (because $S$ is a subring) and $sa \in I$ (because $I$ is a left ideal), so $sa \in S \cap I$.
                \item Note also $as \in S$ (because $S$ is a subring) and $as \in I$ (because $I$ is a right ideal), so $as \in S \cap I$.
            \end{itemize}
        \end{itemize}
        By the test for ideal (\myref{thrm-test-for-ideal}), $S \cap I$ is an ideal of $S$.

        \item Define $\phi: S \to (S+I)/I, s \mapsto s+I$. We show that $\phi$ is a homomorphism and then find its image and kernel.
        \begin{itemize}
            \item \textbf{Homomorphism}: Let $s_1, s_2 \in S$. Then
            \begin{align*}
                \phi(s_1 + s_2) &= (s_1 + s_2) + I\\
                &= (s_1 + I) + (s_2 + I)\\
                &= \phi(s_1) + \phi(s_2)
            \end{align*}
            and
            \begin{align*}
                \phi(s_1s_2) &= (s_1s_2) + I\\
                &= (s_1+I)(s_2+I)\\
                &= \phi(s_1)\phi(s_2)
            \end{align*}
            so $\phi$ is a ring homomorphism.

            \item \textbf{Image}: We show that $\phi$ is surjective. Let $(s+i) + I \in S+I$. Note since $i \in I$,
            \[
                (s+i)+I = (s+I) + (i+I) = (s+I) + (0+I) = s+I.
            \]
            Clearly $\phi(s) = s+I = (s+i)+I$, so for any $(s+i)\in (S+I)/I$ has a pre-image in $S$. Therefore $\im\phi = (S+I)/I$.

            \item \textbf{Kernel}: We find the kernel of $\phi$.
            \begin{align*}
                \ker\phi &= \{s \in S \vert \phi(s) = 0 + I\}\\
                &= \{s \in S \vert s + I = 0 + I\}\\
                &= \{s \in S \vert s \in I\}\\
                &= S \cap I,
            \end{align*}
            where $s + I = 0 + I$ implies $s \in I$ due to Coset Equality, \myref{lemma-coset-equality}.
        \end{itemize}

        By the FRIT (\myref{thrm-ring-isomorphism-1}),
        \[
            S/(S\cap I) \cong (S+I)/I
        \]
        which is what we needed to prove for the Second Ring Isomorphism Theorem.
    \end{partquestions}
\begin{mdframed}
    Let $R$ be a ring with ideals $I$ and $J$ such that $I$ is a subset of $J$.
    \begin{partquestions}{\roman*}
        \item Prove that $J/I$ is an ideal of $R/I$.
        \item Prove that $\frac{R/I}{J/I} \cong R/J$.\newline
        (\textit{Note: remember to prove that the map is well-defined.})
    \end{partquestions}
\end{mdframed}
\textbf{Solution}:\newline
 \begin{partquestions}{\roman*}
        \item We consider the test for ideal.
        \begin{itemize}
            \item $J/I$ is non-empty since $0+I \in J/I$.
            \item Let $a+I,b+I \in J/I$. This means that $a,b \in J$, so $a - b \in J$ (since $J$ is a subring), which means $(a+I) - (b+I) = (a-b) + I \in J/I$.
            \item Let $r+I \in R/I$ and $j+I \in J/I$. Clearly $rj \in J$ and $jr \in J$ since $J$ is an ideal. Thus $(r+I)(j+I) = rj + I \in J/I$ and $(j+I)(r+I) = jr + I \in J/I$.
        \end{itemize}
        By the test for ideal (\myref{thrm-test-for-ideal}), $J/I$ is an ideal of $R/I$.

        \item Consider $\phi: R/I \to R/J, r+I\mapsto r+J$. We show that $\phi$ is a well-defined homomorphism and then find its image and kernel.
        \begin{itemize}
            \item \textbf{Well-Defined}: Suppose $r_1 + I = r_2 + I \in R/I$. This means $r_1 - r_2 + I = I$, so $r_1 - r_2 \in I$ by Coset Equality (\myref{lemma-coset-equality}). Since $I \subset J$ thus $r_1 - r_2 \in J$, meaning $r_1 + J = r_2 + J$ by Coset Equality. Therefore
            \[
                \phi(r_1 + I) = r_1 + J = r_2 + J = \phi(r_2 + I)
            \]
            so $\phi$ is well defined.

            \item \textbf{Homomorphism}: Let $r_1 + I, r_2 + I \in R/I$. Note
            \begin{align*}
                \phi(r_1+r_2) &= (r_1+r_2) + I\\
                &= (r_1+I) + (r_2+I)\\
                &= \phi(r_1) + \phi(r_2)
            \end{align*}
            and
            \begin{align*}
                \phi(r_1r_2) = &= (r_1r_2) + I\\
                &= (r_1+I)(r_2+I)\\
                &= \phi(r_1)\phi(r_2)
            \end{align*}
            so $\phi$ is a ring homomorphism.

            \item \textbf{Image}: We show $\phi$ is surjective. Suppose $r + J \in R/J$. Then clearly $\phi(r+I) = r+J$ so any $r+J$ has a pre-image in $R/I$. Thus $\im\phi = R/J$.

            \item \textbf{Kernel}: We find the kernel of $\phi$.
            \begin{align*}
                \ker\phi &= \{r+I \in R/I \vert \phi(r+I) = 0+J\}\\
                &= \{r+I \in R/I \vert r+J = J\}\\
                &= \{r+I \in R/I \vert r \in J\}\\
                &= J/I
            \end{align*}
            where $r+J=J$ implies $r \in J$ due to Coset Equality, \myref{lemma-coset-equality}.
        \end{itemize}
        Finally, by FRIT (\myref{thrm-ring-isomorphism-1}),
        \[
            \frac{R/I}{J/I} \cong R/J.
        \]
    \end{partquestions}

\chapter{Polynomial Rings}
\section*{Exercises}
\begin{mdframed}
    Let $R$ be a commutative ring. The \textbf{evaluation homomorphism}\index{evaluation homomorphism} is $\phi_a: R[x] \to R$ where $\phi_a(p(x)) = p(a)$ and $a \in R$. Prove that $\phi_a$ is indeed a ring homomorphism.
\end{mdframed}
\textbf{Solution}:\newline
 Let $p(x), q(x) \in R$. Note
    \[
        \phi_a(p(x)+q(x)) = p(a) + q(a) = \phi_a(p(x)) + \phi_a(q(x))
    \]
    and
    \[
        \phi_a(p(x)q(x)) = p(a)q(a) = \phi_a(p(x))\phi_a(q(x))
    \]
    so $\phi_a$ is indeed a homomorphism.
\begin{mdframed}
    Prove that polynomial multiplication is associative.\newline
    (\textit{Hint: $\displaystyle c_k = \sum_{i+j=k} a_ib_j$, which is the sum over all non-negative integers $i$ and $j$ with the property that $i+j=k$.})
\end{mdframed}
\textbf{Solution}:\newline
 For simplicity let
    \begin{align*}
        f(x)(g(x)h(x)) &= \sum_{k=0}^{m+n+l}u_kx^k\\
        (f(x)g(x))h(x) &= \sum_{k=0}^{m+n+l}v_kx^k
    \end{align*}
    for some $u_k, v_k \in \R$.
    \begin{itemize}
        \item On one hand,
        \begin{align*}
            u_k &= \sum_{r+s=k}\left(a_r\left(\sum_{p+q=s}b_pc_q\right)\right) & (\text{Definition of polynomial multiplication})\\
            &= \sum_{r+s=k}\left(\sum_{p+q=s}a_rb_pc_q\right)\\
            &= \sum_{p+q+r=k}a_rb_pc_q\\
            &= \sum_{p+q+r=k}a_pb_qc_r & (\text{Order of }p,\;q,\;r \text{ is arbitrary})
        \end{align*}

        \item On another hand,
        \begin{align*}
            v_k &= \sum_{r+s=k}\left(\left(\sum_{p+q=r}a_pb_q\right)c_s\right) & (\text{Definition of polynomial multiplication})\\
            &= \sum_{r+s=k}\left(\sum_{p+q=r}a_pb_qc_s\right)\\
            &= \sum_{p+q+s=k}a_pb_qc_s\\
            &= \sum_{p+q+r=k}a_pb_qc_r & (s\text{ is dummy variable})
        \end{align*}
    \end{itemize}
    Therefore $u_k = v_k$ for all $k$, meaning that $f(x)(g(x)h(x)) = (f(x)g(x))h(x)$.
\begin{mdframed}
    Show that $\Q[\sqrt[3]{2}] = \left\{a + b\sqrt[3]{2} + c\sqrt[3]4 \vert a,b,c \in \Q\right\}$.
\end{mdframed}
\textbf{Solution}:\newline
 We note
    \begin{align*}
        \Q[\sqrt[3]{2}] &= \left\{a_0 + a_1\sqrt[3]{2} + a_2\left(\sqrt[3]{2}\right)^2 + a_3\left(\sqrt[3]{2}\right)^3 + \cdots + a_n\left(\sqrt[3]{2}\right)^n \vert a_i \in \Q\right\}\\
        &= \left\{a_0 + a_1\sqrt[3]{2} + a_2\sqrt[3]{4} + 2a_3 + \cdots + a_n\left(\sqrt[3]{2}\right)^n \vert a_i \in \Q\right\}\\
        &= \left\{(a_0 + 2a_3 + \cdots) + (a_1 + 2a_4 + \cdots)\sqrt[3]{2} + (a_2 + 2a_5 + \cdots)\sqrt[3]{4} \vert a_i \in \Q\right\}\\
        &= \left\{a + b\sqrt[3]{2} + c\sqrt[3]{4} \vert a,b,c \in \Q\right\}
    \end{align*}
    which establishes the required result.
\begin{mdframed}
    Give an example of a degree 5 polynomial in the ring $\Z_2[x]$.
\end{mdframed}
\textbf{Solution}:\newline
 One example would be $x^5$. Essentially any polynomial where the highest term is $x^5$ would work.
\begin{mdframed}
    Find the zeroes of the polynomial $x^2-1$ in $\Z_4[x]$.
\end{mdframed}
\textbf{Solution}:\newline
 Let $f(x) = x^2 - 1 \in \Z_4[x]$. Note that
    \begin{align*}
        f(0) &= 0^2 - 1 = -1 = 3 \neq 0,\\
        f(1) &= 1^2 - 1 = 0,\\
        f(2) &= 2^2 - 1 = 3 \neq 0, \text{ and}\\
        f(3) &= 3^3 - 1 = 8 = 0
    \end{align*}
    so the zeroes of $f(x)$ are 1 and 3.
\begin{mdframed}
    Let $R$ be a commutative ring.
    \begin{partquestions}{\alph*}
        \item Show that any element in $R$ is a constant polynomial of $R[x]$.
        \item If $f(x)$ is a constant polynomial in $R[x]$, show that $f(x) \in R$.
    \end{partquestions}
\end{mdframed}
\textbf{Solution}:\newline
 \begin{partquestions}{\alph*}
        \item Suppose $r \in R$. If $r = 0$, then it is immediately a constant polynomial of $R[x]$ by definition. Otherwise, may interpret $r$ as a degree 0 polynomial in $R[x]$, which means $r$ is a constant polynomial of $R[x]$.

        \item Suppose $f(x)$ is a constant polynomial in $R[x]$. If $f(x) = 0$ then clearly $f(x) \in R$. Otherwise it takes the form $f(x) = a_0$ for some $a_0 \in R$. Thus clearly $f(x) \in R$.
    \end{partquestions}
\begin{mdframed}
    Let $R$ be a ring.
    \begin{partquestions}{\alph*}
        \item Prove that $R$ is a ring with identity if and only if $R[x]$ is a ring with identity.
        \item Prove that $R$ is a commutative ring if and only if $R[x]$ is a commutative ring.
    \end{partquestions}
\end{mdframed}
\textbf{Solution}:\newline
 \begin{partquestions}{\alph*}
        \item We first suppose $D$ is a ring with identity 1. We may see 1 as a degree 0 polynomial in $D[x]$. Now for any polynomial $f(x) = a_0+a_1x+a_2x^2+\cdots+a_nx^n$ in $D[x]$ we have
        \begin{align*}
            (1)(f(x)) &= (1)(a_0+a_1x+\cdots+a_nx^n)\\
            &= (1a_0)+(1a_1)x+\cdots+(1a_n)x^n\\
            &= a_0+a_1x+\cdots+a_nx^n\\
            &= f(x)
        \end{align*}
        and
        \begin{align*}
            (f(x))(1) &= (a_0+a_1x+\cdots+a_nx^n)(1)\\
            &= (a_{0}1)+(a_{1}1)x+\cdots+(a_{n}1)x^n\\
            &= a_0+a_1x+\cdots+a_nx^n\\
            &= f(x)
        \end{align*}
        so 1 is the identity in $D[x]$.

        Now suppose $D[x]$ is a ring with identity $\id(x)$. We see that $\id(x)f(x) = f(x)\id(x) = f(x)$ for any $f(x) \in D[x]$, meaning $\deg(\id(x)f(x)) = \deg(f(x))$. We note that $\deg(\id(x)f(x)) = \deg(\id(x)) + \deg(f(x))$, this means that $\id$ has degree 0, meaning we may write $\id(x) = e$ for some $e \in D$. Now if $f(x) = a$ for some $a \in D$, we must have $\id(x)f(x) = ae = a$ and $f(x)\id(x) = ea = a$, meaning that $e$ is the identity of $D$.

        \item Suppose first that $D$ is a commutative ring. Let
        \[
            f(x) = \sum_{i=0}^ma_ix^i \text{ and } g(x) = \sum_{j=0}^nb_jx^j
        \]
        be polynomials in $D[x]$. Then
        \begin{align*}
            f(x)g(x) &= \left(\sum_{i=0}^ma_ix^i\right)\left(\sum_{j=0}^nb_jx^j\right)\\
            &= \sum_{k=0}^{m+n}\left(\sum_{i=0}^k a_{i}b_{k-i}\right)x^k\\
            &= \sum_{k=0}^{m+n}\left(\sum_{i=0}^k b_{k-i}a_{i}\right)x^k\\
            &= \sum_{k=0}^{m+n}\left(\sum_{i=0}^k b_{i}a_{k-i}\right)x^k\\
            &= \left(\sum_{j=0}^nb_jx^j\right)\left(\sum_{i=0}^ma_ix^i\right)\\
            &= g(x)f(x)
        \end{align*}
        which therefore means that $D[x]$ is commutative.

        Now suppose $D[x]$ is commutative. Consider the polynomials $f(x) = a$ and $g(x) = b$ where $a$ and $b$ are non-zero. We thus have $ab = f(x)g(x) = g(x)f(x) = ba$ for all $a,b \in D$ which means $D$ is commutative.
    \end{partquestions}
\begin{mdframed}
    Show that $\phi$ in \myref{prop-polynomial-ring-quotient-ideal-polynomial-ring-cong-quotient-polynomial-ring} is a ring homomorphism.
\end{mdframed}
\textbf{Solution}:\newline
 For brevity, let
    \begin{align*}
        f(x) &= \sum_{i=0}^ma_ix^i,\\
        g(x) &= \sum_{i=0}^nb_ix^i
    \end{align*}
    be polynomials in $R[x]$. Without loss of generality assume $m \geq n$, and define $b_i = 0$ for $i > n$. Then
    \begin{align*}
        \phi(f(x) + g(x)) &= \phi\left(\sum_{i=0}^m (a_i+b_i)x^i\right)\\
        &= \sum_{i=0}^m (a_i+b_i + I)x^i\\
        &= \sum_{i=0}^m ((a_i + I) + (b_i + I))x^i\\
        &= \sum_{i=0}^m (a_i + I)x^i + \sum_{i=0}^m (b_i + I)x^i\\
        &= \sum_{i=0}^m (a_i + I)x^i + \sum_{i=0}^n (b_i + I)x^i & (\text{since } b_i = 0\text{ for } i > n)\\
        &= \phi(f(x)) + \phi(g(x))
    \end{align*}
    and
    \begin{align*}
        \phi(f(x)g(x)) &= \phi\left(\sum_{i=0}^{m+n}\left(\sum_{j=0}^i a_jb_{i-j}\right)x^i\right)\\
        &= \sum_{i=0}^{m+n}\left(\left(\sum_{j=0}^i a_jb_{i-j}\right) + I\right)x^i\\
        &= \sum_{i=0}^{m+n}\left(\sum_{j=0}^i (a_jb_{i-j} + I)\right)x^i\\
        &= \sum_{i=0}^{m+n}\left(\sum_{j=0}^i ((a_j+I)(b_{i-j}+I))\right)x^i\\
        &= \left(\sum_{i=0}^m(a_i+I)x^i\right)\left(\sum_{i=0}^n(b_i+I)x^i\right)\\
        &= \phi(f(x))\phi(g(x))
    \end{align*}
    so $\phi$ is indeed a ring homomorphism.
\begin{mdframed}
    Prove \myref{thrm-prime-ideal-iff-prime-ideal-in-polynomial-ring}.\newline
    (\textit{Hint: most of the statements required are if and only if statements.})
\end{mdframed}
\textbf{Solution}:\newline
 We note
    \begin{align*}
        &P \text{ is a prime ideal of }R\\
        \iff&R/P \text{ is an integral domain} & (\myref{thrm-prime-ideal-iff-quotient-ring-is-integral-domain})\\
        \iff&(R/P)[x] \text{ is an integral domain} & (\myref{thrm-integral-domain-iff-polynomial-ring-is-integral-domain})\\
        \iff&R[x]/P[x] \text{ is an integral domain} & (\myref{prop-polynomial-ring-quotient-ideal-polynomial-ring-cong-quotient-polynomial-ring})\\
        \iff&P[x]\text{ is a prime ideal of }R[x] & (\myref{thrm-prime-ideal-iff-quotient-ring-is-integral-domain})
    \end{align*}
    proving the theorem.
\begin{mdframed}
    Divide $3x^4 + 3x^3 + 4x^2 + 3x + 3$ by $x^2 + 2x + 3$ in $\Z_5[x]$ and hence state two factors of $3x^4 + 3x^3 + 4x^2 + 3x + 3$ in $\Z_5[x]$.
\end{mdframed}
\textbf{Solution}:\newline
 Note that
    \begin{align*}
        3x^4 + 3x^3 + 4x^2 + 3x + 3 &= 3x^2(x^2+2x+3) - 3x(x^2+2x+3)\\
        &\quad\quad+ 1(x^2+2x+3) + 10x\\
        &= (3x^2-3x+1)(x^2+2x+3) + 10x\\
        &= (3x^2+2x+1)(x^2+2x+3) + 0x\\
        &= (3x^2+2x+1)(x^2+2x+3)
    \end{align*}
    so dividing $3x^4 + 3x^3 + 4x^2 + 3x + 3$ by $x^2+2x+3$ yields $3x^2+2x+1$. Thus two factors of $3x^4 + 3x^3 + 4x^2 + 3x + 3$ are $x^2+2x+3$ and $3x^2+2x+1$.

\section*{Problems}
\begin{mdframed}
    Find a degree 4 polynomial that is equal to the polynomial $1 - 2x + 3x^2 - 4x^3 + 5x^4 - 6x^5$ in $\Z_3[x]$.
\end{mdframed}
\textbf{Solution}:\newline
 Note
    \begin{align*}
        1 - 2x + 3x^2 - 4x^3 + 5x^4 - 6x^5 &= 1 + x + 0x^2 + 2x^3 + 2x^4 + 0x^5\\
        &= 1 + x + 2x^3 + 2x^4
    \end{align*}
    so a degree 4 polynomial that is equal to the given polynomial is $1 + x + 2x^3 + 2x^4$.
\begin{mdframed}
    Let $I$ be a principal ideal of $\Z[x]$ generated by the polynomial $x^2 + 3x - 1$. Simplify $\left((x + 3) + I\right)\left((2x^2 + 3x - 1) + I\right)$ in the quotient ring $\Z[x]/I$.
\end{mdframed}
\textbf{Solution}:\newline
 We work step-by-step.
    \begin{align*}
        &\left((x + 3) + I\right)\left((2x^2 + 3x - 1) + I\right)\\
        &= \left((x + 3)(2x^2+3x-1)\right) + I\\
        &= \left(2x^3 + 9x^2 + 8x - 3\right) + I\\
        &= \left((2x+3)\underbrace{(x^2+3x-1)}_{\text{In }I} + x\right) + I\\
        &= x + I
    \end{align*}
\begin{mdframed}
    Determine if the following statements are true or false, and justify your answer.
    \begin{partquestions}{\alph*}
        \item All polynomials $f(x) \in \Z[x]$ has at least one integer zero.
        \item There exists a polynomial $f(x) \in \Z[x]$ with non-integer zero(es).
        \item There exists a polynomial $f(x) \in \R[x]$ with integer coefficients with non-integer zero(es).
    \end{partquestions}
\end{mdframed}
\textbf{Solution}:\newline
 \begin{partquestions}{\alph*}
        \item False. $2x-1$ is a polynomial with integer coefficients with no integer zeroes.

        \item False. By definition, zeroes must always come from the ground ring, so if $f(x) \in \Z[x]$ has zero(es) they must be integers.

        \item True. $2x-1 \in \R[x]$ is a polynomial with integer coefficients and has a non-integer zero of $\frac12$.
    \end{partquestions}
\begin{mdframed}
    For the polynomials
    \begin{partquestions}{\alph*}
        \item $4x^2 + 2x + 1 \in \Z_8[x]$ and
        \item $x^2 + 2x + 1 \in \Z_7[x]$,
    \end{partquestions}
    find another polynomial $f(x)$ such that their product is 1 for all $x$ in the ground ring. If such an $f(x)$ does not exist, explain why.
\end{mdframed}
\textbf{Solution}:\newline
 \begin{partquestions}{\alph*}
        \item We guess the form of $f(x)$ for $4x^2 + 2x + 1$.

        Guess first that $f(x)$ is a degree 0 polynomial, i.e. $f(x) = a$. Then $4ax^2 + 2ax + a = 1$, meaning $a = 1$. But that is the same as the original polynomial, so we conclude $f(x)$ cannot be a polynomial of degree 0.

        Guess next that $f(x)$ is a degree 1 polynomial, i.e. $f(x) = ax + b$. Then $(4x^2+2x+1)(ax+b) = 1$, meaning $4ax^3 + 2(a+2b)x^2 + (a+2b)x + b = 1$. Thus $b = 1$, otherwise it would not be possible for its product to be 1. Thus it reduces to $4ax^3 + 2(a+2)x^2 + (a+2)x + 1$, which means that $a+2 = 0$. Therefore $a = -2 = 6$. We check that
        \begin{align*}
            (4x^2+2x+1)(6x+1) &= 24x^3 + 16x^2 + 8x + 1\\
            &= 0x^3 + 0x^2 + 0x + 1\\
            &= 1
        \end{align*}
        so $f(x) = 6x+1$ works.

        \item We note $\Z_7$ is a field (\myref{example-Zp-is-field}) and so $\Z_7$ is an integral domain. This means $\Z_7[x]$ is an integral domain (\myref{thrm-integral-domain-iff-polynomial-ring-is-integral-domain}). So a unit of $\Z_7[x]$ is a unit of $\Z_7$ (\myref{prop-unit-of-ring-iff-unit-of-polynomial-ring}), which means only constants can be units. Thus $x^2 + 2x + 1$ has no corresponding $f(x)$ such that $(x^2+2x+1)f(x) = 1$.
    \end{partquestions}
\begin{mdframed}
    Prove the remainder theorem (\myref{corollary-remainder-theorem}).
\end{mdframed}
\textbf{Solution}:\newline
 Using polynomial long division (\myref{thrm-polynomial-long-division}) we write
    \[
        f(x) = q(x)(x-a) + r(x) \text{ with } r(x) \text{ or } \deg r(x) < \deg(x-a) = 1.
    \]
    Thus $r(x) = b \in F$. Evaluating $f(x)$ at a yields
    \[
        f(a) = q(x)(a-a) + b = b
    \]
    which means $f(a)$ is the remainder of the division of $f(x)$ by $x-a$.
\begin{mdframed}
    Let $I = \{f(x) \in \Z[x] \vert f(-2) = 0\}$ be a subset of $\Z[x]$, and let the homomorphism $\phi:\Z[x]\to\Z, f(x) \mapsto f(-2)$.
    \begin{partquestions}{\roman*}
        \item Show that $I$ is an ideal of $\Z[x]$.
        \item Hence determine if the ideal $I$ is prime, maximal, or both.
    \end{partquestions}
\end{mdframed}
\textbf{Solution}:\newline
 \begin{partquestions}{\roman*}
        \item Note that
        \begin{align*}
            \ker\phi &= \{f(x) \in \Z[x] \vert \phi(f(x)) = 0\}\\
            &= \{f(x) \in \Z[x] \vert f(-2) = 0\}\\
            &= I.
        \end{align*}
        \myref{prop-kernel-is-an-ideal} tells us that $\ker\phi$ is an ideal of $\Z[x]$, so $I$ is an ideal of $\Z[x]$.

        \item We first show that $\phi$ is surjective. Let $n \in \Z$, note that $n$ is a degree zero polynomial, so $n \in \Z[x]$. Clearly $\phi(n) = n$ so $n$ is its own pre-image. Therefore $\im\phi = \Z$.

        By FRIT (\myref{thrm-ring-isomorphism-1}),
        \[
            \Z[x]/I \cong \Z.
        \]
        Note that $\Z$ is an integral domain but not a field. Thus $I$ is prime but not maximal.
    \end{partquestions}
\begin{mdframed}
    Show that $\princ{x}$ is a prime ideal in $\Z[x]$.
\end{mdframed}
\textbf{Solution}:\newline
 For brevity let $I = \princ{x} = \{xP(x) \vert P(x) \in \Z[x]\}$. This means that $I$ is the set of polynomials with integer coefficients and with constant term 0. Now suppose $f(x), g(x) \in \Z[x]$; write
    \begin{align*}
        f(x) &= a_0 + a_1x + \cdots + a_mx^m\\
        g(x) &= b_0 + b_1x + \cdots + b_nx^n
    \end{align*}
    where $a_i, b_i \in \Z$ and $m$ and $n$ are positive integers. Note that
    \[
        f(x)g(x) = a_0b_0 + (a_1b_0+a_0b_1)x + \cdots.
    \]
    Now if $f(x)g(x) \in I$, this means that $a_0b_0 = 0$. Hence either $a_0 = 0$ or $b_0 = 0$, meaning that either $f(x)$ has zero constant term (so $f(x) \in I$) or $g(x)$ has zero constant term (so $g(x) \in I$). Thus $I$ is prime.
\begin{mdframed}
    Prove that $\Z[x] / \princ{x} \cong \Z$.
\end{mdframed}
\textbf{Solution}:\newline
 We show that $\Z[x] / \princ{x} \cong \Z$ by using the FRIT.

    Let $\phi: \Z[x] \to \Z$ be defined such that $p(x) \mapsto p(0)$.

    We claim $\phi$ is surjective. Let $n \in \Z$. Note that
    \[
        n = n + 0x + 0x^2 + \cdots \in \Z[x]
    \]
    so
    \begin{align*}
        \phi(n) &= \phi(n + 0x + 0x^2 + \cdots)\\
        &= n + 0(0) + 0(0)^2 + \cdots\\
        &= n
    \end{align*}
    which means that $n$ is its own pre-image. Thus $\im\phi = \Z$.

    Now we find the kernel of $\phi$. Suppose $p(x) \in \ker\phi$, i.e. $\phi(p(x)) = 0$. This means that $p(0) = 0$. The general form for a univariate polynomial $p(x)$ is $a_0 + a_1x + a_2x^2 + \cdots$, so if $p(0) = 0$ then $a_0 = 0$. Thus,
    \begin{align*}
        p(x) &= 0 + a_1x + a_2x^2 + \cdots\\
        &= x(a_1 + a_2x + \cdots)\\
        &= xq(x)
    \end{align*}
    where $q(x) \in \Z[x]$. Therefore
    \[
        \ker\phi = \{xq(x) \vert q(x) \in \Z[x]\} = \princ{x}.
    \]

    Thus, the FRIT (\myref{thrm-ring-isomorphism-1}) tells us that
    \[
        \Z[x] / \princ{x} \cong \Z.
    \]
\begin{mdframed}
    Find an infinite set of polynomials $S \subseteq \Z_5[x]$ such that any two distinct polynomials in $S$ have different degrees and all of $\Z_5$ are zeroes of a polynomial in $S$.
\end{mdframed}
\textbf{Solution}:\newline
 We first note that the polynomial functions $f: x \mapsto x$ and $g: x \mapsto x^5$ are equal since
    \begin{align*}
        g(0) &= 0^5 = 0 = f(0),\\
        g(1) &= 1^5 = 1 = f(1),\\
        g(1) &= 2^5 = 32 = 2 = f(2),\\
        g(3) &= 3^5 = 243 = 3 = f(3), \text{ and}\\
        g(4) &= 4^5 = 1024 = 4 = f(4).
    \end{align*}
    One sees thereafter that the polynomial $h(x) = x^5 + 4x$ has the entirety of $\Z_5$ as zeroes, since
    \begin{align*}
        h(0) &= 0^5 + 4(0) = 0,\\
        h(1) &= 1^5 + 4(1) = 5 = 0,\\
        h(2) &= 2^5 + 4(2) = 40 = 0,\\
        h(3) &= 3^5 + 4(3) = 255 = 0, \text{ and}\\
        h(4) &= 4^5 + 4(4) + 1040 = 0.
    \end{align*}
    We note that the polynomial $(x^n)(x^5 + 4x)$ where $n$ is a non-negative integer would also have the entirety of $\Z_5$ as zeroes. Therefore a set $S$ that works is
    \[
        S = \{x^n(x^5+4x) \vert n \geq 0\}
    \]
    and we note that this is an infinite set where distinct polynomials in $S$ have different degrees.
\begin{mdframed}
    Let $F$ be a field and $I$ a non-zero ideal in $F[x]$.
    \begin{partquestions}{\roman*}
        \item If $g(x) \in I$ is a non-zero polynomial of minimum degree, prove $I = \princ{g(x)}$.
        \item If $I = \princ{g(x)}$ for some $g(x) \in I$, prove that $g(x)$ is non-zero and of minimum degree.
    \end{partquestions}
\end{mdframed}
\textbf{Solution}:\newline
 \begin{partquestions}{\roman*}
        \item Suppose $g(x) \in I$ is a non-zero polynomial of minimum degree.
        \begin{itemize}
            \item Since $g(x) \in I$ thus $\princ{g(x)} \subseteq I$.
            \item Now suppose $f(x) \in I$. By polynomial long division (\myref{thrm-polynomial-long-division}) write $f(x) = q(x)g(x) + r(x)$ where $r(x) = 0$ or $\deg r(x) < \deg g(x)$. Note $r(x) = f(x) - \underbrace{g(x)q(x)}_{\text{In }I} \in I$, so the minimality of $\deg g(x)$ means that $\deg r(x) \not< \deg g(x)$, i.e. $r(x) = 0$. Thus $f(x) = q(x)g(x) \in \princ{g(x)}$, meaning $I \subseteq \princ{g(x)}$.
        \end{itemize}
        This shows that $I = \princ{g(x)}$.

        \item If $g(x) = 0$ then $I = \princ{0} = \{0\}$, which is the zero ideal. Thus $g(x) \neq 0$. Let $g(x)$ have degree $n$. Now suppose there exists a $h(x) \in I$ that has a smaller degree than $g(x)$. By \textbf{(i)} we know $I = \princ{h(x)} = \{f(x)h(x) \vert f(x) \in F[x]\}$. Let $h(x)$ have degree $m$, so $g(x) = h(x) + x^{n-m}k(x)$ for some polynomial $k(x)$ with $\deg k(x) < m$ (since if $\deg k(x) = m$ this means $g(x)$ also has degree $m$, a contradiction). We also know
        \begin{align*}
            I &= \princ{g(x)}\\
            &= \{f(x)g(x) \vert f(x) \in F[x]\}\\
            &= \{f(x)(h(x) + x^{n-m}k(x)) \vert f(x) \in F[x]\}\\
            &= \{f(x)h(x) + x^{n-m}f(x)k(x) \vert f(x) \in F[x]\}\\
            &\subset \{f(x)h(x) \vert f(x) \in F[x]\}\\
            &= \princ{h(x)}\\
            &= I,
        \end{align*}
        a contradiction.
    \end{partquestions}
\begin{mdframed}
    Prove $R[x] \cong R[x^k]$ for any positive integer $k$ and commutative ring $R$.
\end{mdframed}
\textbf{Solution}:\newline
 Consider the map $\phi: R[x] \to R[x^k]$ where
    \[
        \sum_{i=0}^n a_ix^i \mapsto \sum_{i=0}^na_ix^{ki}.
    \]
    We show that $\phi$ is a ring isomorphism. For brevity let $f(x), g(x) \in R[x]$ where, without loss of generality, assume $m \geq n$,
    \begin{align*}
        f(x) &= \sum_{i=0}^m a_ix^i,\\
        g(x) &= \sum_{i=0}^n b_ix^i,
    \end{align*}
    and set $b_i = 0$ for any $i > n$.
    \begin{itemize}
        \item \textbf{Homomorphism}: Note that
        \begin{align*}
            \phi(f(x) + g(x)) &= \phi\left(\sum_{i=0}^m(a_i+b_i)x^i\right)\\
            &= \sum_{i=0}^m(a_i+b_i)x^{ki}\\
            &= \sum_{i=0}^ma_ix^{ki} + \sum_{i=0}^mb_ix^{ki}\\
            &= \sum_{i=0}^ma_ix^{ki} + \sum_{i=0}^mb_ix^{ki} & (\text{since } b_i = 0 \text{ for } i > n)\\
            &= \phi(f(x)) + \phi(g(x))
        \end{align*}
        and
        \begin{align*}
            \phi(f(x)g(x)) &= \phi\left(\sum_{r=0}^{m+n}\left(\sum_{i=0}^ra_ib_{r-i}\right)x^r\right)\\
            &= \sum_{r=0}^{m+n}\left(\sum_{i=0}^ra_ib_{r-i}\right)x^{kr}\\
            &= \left(\sum_{i=0}^ma_ix_{ki}\right)\left(\sum_{i=0}^nb_ix_{ki}\right)\\
            &= \phi(f(x))\phi(g(x))
        \end{align*}
        so $\phi$ is a ring homomorphism.

        \item \textbf{Injective}: Suppose $\phi(f(x)) = \phi(g(x))$. Thus
        \[
            \sum_{i=0}^m a_ix^{ki} = \sum_{i=0}^n b_ix^{ki}
        \]
        which clearly means that $a_i = b_i$ for all $0 \leq i \leq m$ (where, again, $b_i = 0$ if $i > n$). Thus
        \[
            f(x) = \sum_{i=0}^m a_ix^i = \sum_{i=0}^n b_ix^i = g(x)
        \]
        which shows that $\phi$ is injective.

        \item \textbf{Surjective}: Suppose $h(x) = c_0 + c_kx^k + c_{2k}x^{2k} + \cdots + c_{pk}x^{pk} \in R[x^k]$. Set $a_i = c_{ki}$ for all $0 \leq i \leq p$ and one sees that
        \[
            \phi\left(\sum_{i=0}^pa_ix^i\right) = \sum_{i=0}^pa_ix^{ki} = \sum_{i=0}^pc_{ki}x^{ki} = h(x)
        \]
        which means that any $h(x) \in R[x^k]$ has a pre-image in $R[x]$. Thus $\phi$ is surjective.
    \end{itemize}
    Hence $\phi$ is a ring isomorphism, meaning $R[x] \cong R[x^k]$.
\begin{mdframed}
    Let $R$ and $S$ be commutative rings, and let $\phi: R \to S$ be a ring homomorphism. Define the map $\psi: R[x] \to S[x]$ where
    \[
        \psi\left(\sum_{i=0}^na_ix^i\right) = \sum_{i=0}^n\phi(a_i)x^i.
    \]
    \begin{partquestions}{\roman*}
        \item Show that $\psi$ is a ring homomorphism.
        \item If $\phi$ is an isomorphism, prove that $R[x] \cong S[x]$.
    \end{partquestions}
\end{mdframed}
\textbf{Solution}:\newline
 \begin{partquestions}{\roman*}
        \item Let $f(x), g(x) \in R[x]$ where, without loss of generality, assume $m \geq n$,
        \begin{align*}
            f(x) &= \sum_{i=0}^m a_ix^i,\\
            g(x) &= \sum_{i=0}^n b_ix^i,
        \end{align*}
        and set $b_i = 0$ for any $i > n$. Note that
        \begin{align*}
            \psi(f(x) + g(x)) &= \psi\left(\sum_{i=0}^m(a_i+b_i)x_i\right)\\
            &= \sum_{i=0}^m\phi(a_i + b_i)x_i\\
            &= \sum_{i=0}^m\left(\phi(a_i) + \phi(b_i)\right)x_i\\
            &= \sum_{i=0}^m\phi(a_i)x_i + \sum_{i=0}^m\phi(b_i)x_i\\
            &= \sum_{i=0}^m\phi(a_i)x_i + \sum_{i=0}^n\phi(b_i)x_i & (\text{since } b_i = 0 \text{ for } i > n)\\
            &= \psi(f(x)) + \psi(g(x))
        \end{align*}
        and
        \begin{align*}
            \psi(f(x)g(x)) &= \psi\left(\sum_{k=0}^{m+n}\left(\sum_{i=0}^ka_ib_{k-i}\right)x^k\right)\\
            &= \sum_{k=0}^{m+n}\left(\phi\left(\sum_{i=0}^ka_ib_{k-i}\right)x^k\right)\\
            &= \sum_{k=0}^{m+n}\left(\left(\sum_{i=0}^k\phi(a_ib_{k-i})\right)x^k\right)\\
            &= \sum_{k=0}^{m+n}\left(\left(\sum_{i=0}^k\phi(a_i)\phi(b_{k-i})\right)x^k\right)\\
            &= \left(\sum_{i=0}^m\phi(a_i)x_i\right)\left(\sum_{i=0}^n\phi(b_i)x_i\right)\\
            &= \psi(f(x))\psi(g(x))
        \end{align*}
        so $\psi$ is a ring homomorphism.

        \item We prove that $\psi$ is both injective and surjective. We use the same functions and notation as in \textbf{(i)}.
        \begin{itemize}
            \item \textbf{Injective}: Suppose $\psi(f(x)) = \psi(g(x))$. Thus
            \[
                \sum_{i=0}^m \phi(a_i)x^i = \sum_{i=0}^n \phi(b_i)x^i
            \]
            by definition of $\psi$, which means $\phi(a_i) = \phi(b_i)$ for all $0 \leq i \leq m$ (where $b_i = 0$ for $i > n$). Now as $\phi$ is an isomorphism, thus we see $a_i = b_i$ for all $0 \leq i \leq m$ (where, again, $b_i = 0$ for $i > n$). Therefore
            \[
                \sum_{i=0}^m a_ix^i = \sum_{i=0}^n b_ix^i
            \]
            which therefore means $f(x) = g(x)$. So $\psi$ is an injective function.

            \item \textbf{Surjective}: Suppose $F(x) = a_0 + a_1x + a_2x^2 + \cdots + a_mx^m \in S[x]$. Let $G(x) = \phi^{-1}(a_0) + \phi^{-1}(a_1)x + \cdots + \phi^{-1}(a_m)x^m \in R[x]$. Then note
            \[
                \psi(G(x)) = \sum_{i=0}^m\phi(\phi^{-1}(a_i))x_i = \sum_{i=0}^ma_ix_i = F(x)
            \]
            so any $F(x) \in S[x]$ has a pre-image in $R[x]$. Therefore $\psi$ is a surjective function.
        \end{itemize}

        So $\psi$ is a bijection. Furthermore by \textbf{(i)} we know that $\psi$ is a homomorphism. Therefore $\psi$ is a ring isomorphism, meaning $R[x] \cong S[x]$.
    \end{partquestions}
\begin{mdframed}
    Prove that $\Z[x]$ is \textit{not} a PID.\newline
    (\textit{Hint: consider the set $\{2f(x) + xg(x) \vert f(x),g(x) \in \Z[x]\}$.})
\end{mdframed}
\textbf{Solution}:\newline
 Let $\ideal{a} = \princ{2}$ and $\ideal{b} = \princ{x}$ be principal ideals. Note that
    \[
        I = \ideal{a} + \ideal{b} = \{2f(x) + xg(x) \vert f(x),g(x) \in \Z[x]\}
    \]
    is a sum of ideals and thus is an ideal (\myref{prop-sum-of-ideals-is-ideal}). We show that $I$ is not a principal ideal.

    Suppose on the contrary that $I = \princ{F(x)}$ for some $F(x) \in \Z[x]$. Then $2 \in I$, meaning that $2 = f(x)F(x)$ for some $f(x) \in \Z[x]$. We note that $\Z$ is an integral domain, so $\Z[x]$ is an integral domain (\myref{thrm-integral-domain-iff-polynomial-ring-is-integral-domain}) which means that
    \[
        \deg(f(x)F(x)) = \deg f(x) + \deg F(x) = \deg 2 = 0,
    \]
    so $f(x)$ and $F(x)$ are both constant polynomials. Thus $f(x) = m \in \Z$ and $F(x) = n \in \Z$. Actually, since $2 \in I$, one sees that the only possibilities for $n$ are $\pm1$ and $\pm2$.

    Note that an element of $I$ takes the form $2a_0 + a_1x + a_2x^2 + \cdots + a_nx^n$.
    \begin{itemize}
        \item We note $\princ{1} = \princ{-1}$. But this means that $1 \in I$ which is impossible (since the constant term must be an even integer).
        \item We note $\princ{2} = \princ{-2}$. But this means that $I = \princ{2}$ is the set of all polynomials of even coefficients. But $2 + x \in I$ although $2 + x \notin \princ{2}$, a contradiction.
    \end{itemize}

    Therefore $I$ is non-principal, which means that $\Z[x]$ is not a PID.

\chapter{Polynomial Factorization}
\section*{Exercises}
\begin{mdframed}
    Let the polynomial $f(x) = 3x^2 - 6$. Which of the following integral domains is $f(x)$ irreducible?
    \begin{partquestions}{\alph*}
        \item $\Z$
        \item $\Q$
        \item $\R$
    \end{partquestions}
\end{mdframed}
\textbf{Solution}:\newline
 \begin{partquestions}{\alph*}
        \item Not irreducible since $3x^2 - 6 = 3(x^2-2)$ and both 3 and $x^2-2$ are non-units in $\Z$.
        \item Irreducible in $\Q$ since we cannot express $3x^2-6$ as a product of two polynomial of smaller degree that are in $\Q[x]$ (by \myref{thrm-irreducible-iff-not-expressable-as-product-of-smaller-polynomials}).
        \item Reducible in $\R$ since $3x^2 - 6 = (3x-3\sqrt2)(x+\sqrt2)$ and both $3x-3\sqrt2, x+\sqrt2 \in \R[x]$.
    \end{partquestions}
\begin{mdframed}
    Let the polynomial $f(x) = 2x^3 + 4x + 9$. For which field(s) listed below is $f(x)$ irreducible?
    \begin{multicols}{4}
        \begin{partquestions}{\alph*}
            \item $\Z_2$
            \item $\Z_3$
            \item $\Z_5$
            \item $\Z_7$
        \end{partquestions}
    \end{multicols}
\end{mdframed}
\textbf{Solution}:\newline
 For brevity we compute all possible values of $f(x)$ for $x \in \Z_7$ and then check for zeroes.
    \begin{table}[H]
        \centering
        \begin{tabular}{|l|l|l|l|l|l|}
            \hline
            $\boldsymbol{x}$ & $\boldsymbol{f(x)}$ & $\boldsymbol{f(x) \mod2}$ & $\boldsymbol{f(x) \mod3}$ & $\boldsymbol{f(x) \mod5}$ & $\boldsymbol{f(x) \mod7}$ \\ \hline
            \textbf{0} & 9 & 1 & 0 & 4 & 2 \\ \hline
            \textbf{1} & 15 & 1 & 0 & 0 & 1 \\ \hline
            \textbf{2} & 33 &  & 0 & 3 & 5 \\ \hline
            \textbf{3} & 75 &  &  & 0 & 5 \\ \hline
            \textbf{4} & 153 &  &  & 3 & 6 \\ \hline
            \textbf{5} & 279 &  &  &  & 6 \\ \hline
            \textbf{6} & 465 &  &  &  & 3 \\ \hline
        \end{tabular}
    \end{table}
    We therefore see that $f(x)$ has zeroes in $\Z_3$ and $\Z_5$ and no zeroes in $\Z_2$ and $\Z_7$. Thus $f(x)$ is irreducible over $\Z_2$ and $\Z_7$ and reducible in $\Z_3$ and $\Z_5$.
\begin{mdframed}
    Let $f(x) \in \Z[x]$. Prove or disprove the following statements.
    \begin{partquestions}{\alph*}
        \item If $f(x)$ is irreducible over $\Z$ then it is irreducible over $\Q$.
        \item If $f(x)$ is irreducible over $\Q$ then it is irreducible over $\Z$.
    \end{partquestions}
\end{mdframed}
\textbf{Solution}:\newline
 \begin{partquestions}{\alph*}
        \item Statement is the converse of \myref{thrm-irreducible-over-Z-means-irreducible-over-Q}, so it is true.
        \item Disprove. $2(x^2+1)$ has no zeroes in $\Q$ and so is irreducible over $\Q$ (\myref{thrm-degree-2-or-3-irreducible-iff-has-no-zeroes}). But $2(x^2+1)$ is reducible in $\Z$ as $2 \times (x^2 + 1)$ where 2 and $x^2 + 1$ are both non-units.
    \end{partquestions}
\begin{mdframed}
    Prove that, for all prime numbers $p$, there exists a non-negative integer $n < p$ such that $n^2 + 2n + 1 \equiv 0 \pmod{p}$.
\end{mdframed}
\textbf{Solution}:\newline
 Clearly $1^2 + 2(1) + 1 = 4 \equiv 0 \pmod2$, so we consider $p > 3$. Consider the polynomial $f(x) = x^2 + 2x + 1$. Clearly $f(-1) = 0$ so $f(x)$ is reducible in $\Q$ (\myref{thrm-degree-2-or-3-irreducible-iff-has-no-zeroes}). Thus the converse of the Mod $p$ Irreducibility Test (\myref{thrm-mod-p-irreducibility-test}) tells us that $f(x)$ is reducible in $\Z_p$ (since reducing the coefficients modulo $p$ yields the same polynomial). Therefore $f(x) \in \Z_p[x]$ must have a zero (\myref{thrm-degree-2-or-3-irreducible-iff-has-no-zeroes}), i.e. there is an integer $n \in \Z_p$ such that $f(n) = n^2 + 2n + 1 = 0$ when working modulo $p$. That same $n$ satisfies $n^2 + 2n + 1 \equiv 0 \pmod{p}$ as required.
\begin{mdframed}
    Let $n$ be an arbitrary positive integer. Find a polynomial $f(x) \in \Z[x]$ of degree $n$ that is irreducible over $\Q$.
\end{mdframed}
\textbf{Solution}:\newline
 We claim that $f(x) = x^n + 2$ is irreducible over $\Q$ for any $n \geq 1$. We note that 2 does not divide 1, 2 divides 2, and $2^2 = 4$ does not divide 2. Thus $f(x)$ is irreducible over $\Q$ by Eisenstein's Criterion (\myref{thrm-eisenstein-criterion}) with the prime 2.
\begin{mdframed}
    Let $f(x) = x^4 + 3$.
    \begin{partquestions}{\roman*}
        \item Show that $f(x)$ is irreducible over $\Q$.
        \item Explain why $x^4 + 4x^3 + 6x^2 + 4x + 4$ has no integer zeroes.
        \item Show that $x^4 - 8x^3 + 24x^2 - 32x + 19$ is irreducible over $\Q$.
    \end{partquestions}
\end{mdframed}
\textbf{Solution}:\newline
 \begin{partquestions}{\roman*}
        \item Choose the prime 3. Clearly 3 does not divide 1, 3 divides 3, but $3^2 = 9$ does not divide 3. Thus Eisenstein's Criterion (\myref{thrm-eisenstein-criterion}) tells us that $f(x)$ is irreducible over $\Q$.

        \item One may use Eisenstein's Criterion (with the prime 2) on the given polynomial. We instead note that
        \[
            x^4 + 4x^3 + 6x^2 + 4x + 4 = (x+1)^4 + 3
        \]
        and since $x^4 + 3$ is irreducible thus the above polynomial is also irreducible over $\Q$. Note that the given polynomial is primitive, so it is also irreducible over $\Z$ (\myref{thrm-irreducible-over-Z-means-irreducible-over-Q}). Hence $x^4 + 4x^3 + 6x^2 + 4x + 4$ has no zeroes in $\Z$ by contrapositive of \myref{thrm-degree-above-1-reducible-if-has-zero}.

        \item We note we cannot use Eisenstein's Criterion directly, since the only prime that works on the constant term is 19, and $19 \nmid 32$. We need to consider a substitution.

        One may see that, after some trial and error, that
        \begin{align*}
            x^4 - 8x^3 + 24x^2 - 32x + 19 &= (x^4 - 8x^3 + 24x^2 - 32x + 16) + 3\\
            &= (2-x)^4 + 3
        \end{align*}
        and since $x^4 + 3$ is irreducible by \textbf{(i)}, thus $x^4 - 8x^3 + 24x^2 - 32x + 19$ is irreducible by transformation.
    \end{partquestions}
\begin{mdframed}
    Show that the map $\phi$ given in \myref{corollary-irreducible-iff-constant-factor-multiple-is-irreducible} is indeed a ring automorphism.
\end{mdframed}
\textbf{Solution}:\newline
 We prove the three requirements for a ring isomorphism.
    \begin{itemize}
        \item \textbf{Homomorphism}: One sees that
        \begin{align*}
            \phi(f(x) + g(x)) &= \phi\left(\sum_{i=0}^m(a_i+b_i)x^i\right)\\
            &= \sum_{i=0}^m(a_i+b_i)(kx)^i\\
            &= \left(\sum_{i=0}^ma_i(kx)^i\right) + \left(\sum_{i=0}^mb_i(kx)^i\right)\\
            &= \left(\sum_{i=0}^ma_i(kx)^i\right) + \left(\sum_{i=0}^nb_i(kx)^i\right) & (\because b_i = 0 \text{ for } i > n)\\
            &= \phi(f(x)) + \phi(g(x))
        \end{align*}
        and
        \begin{align*}
            \phi(f(x)g(x)) &= \phi\left(\sum_{r=0}^{m+n}\left(\sum_{i=0}^ra_ib_{r-i}\right)x^r\right)\\
            &= \sum_{r=0}^{m+n}\left(\sum_{i=0}^ra_ib_{r-i}\right)(kx)^r\\
            &= \left(\sum_{r=0}^ma_r(kx)^r\right)\left(\sum_{r=0}^nb_r(kx)^r\right)\\
            &= \phi(f(x))\phi(g(x))
        \end{align*}
        so $\phi$ is a ring homomorphism.

        \item \textbf{Injective}: Suppose $\phi(f(x)) = \phi(g(x))$. Then
        \begin{align*}
            \sum_{i=0}^m(a_ik^i)x^i &= \sum_{i=0}^ma_i(kx)^i\\
            &= \sum_{i=0}^nb_i(kx)^i\\
            &= \sum_{i=0}^n(b_ik^i)x^i
        \end{align*}
        which, by comparing coefficients, we see $a_i = b_i$ for all $0 \leq i \leq m$. Therefore $f(x) = g(x)$, meaning $\phi$ is injective.

        \item \textbf{Surjective}: Let $p(x) = c_0 + c_1x + \cdots + c_rx^r$ be a polynomial in $D[x]$. Since $k$ is a unit, therefore $k^{-1}$ exists. Note that $q(x) = c_0 + c_1(k^{-1}x) + \cdots + c_r(k^{-1}x)^r$ is also a polynomial in $D[x]$. Observe
        \begin{align*}
            \phi(q(x)) &= c_0 + c_1(k(k^{-1}x)) + \cdots + c_r(k(k^{-1}x))^r\\
            &= c_0 + c_1x + \cdots + c_rx^r\\
            &= p(x)
        \end{align*}
        so any $p(x) \in D[x]$ has a pre-image under $\phi$.
    \end{itemize}
    Therefore $\phi$ is an isomorphism.
\begin{mdframed}
    Prove \myref{corollary-irreducible-polynomial-division-rule}.
\end{mdframed}
\textbf{Solution}:\newline
 Since $p(x)$ is irreducible we know $D[x]/\princ{p(x)}$ is a field (\myref{corollary-polynomial-quotient-by-principal-ideal-is-field-iff-polynomial-irreducible}), which is a integral domain (\myref{prop-field-is-integral-domain}), and so $\princ{p(x)}$ is a prime ideal (\myref{thrm-prime-ideal-iff-quotient-ring-is-integral-domain}). As $p(x) = a(x)b(x)$ therefore $a(x)b(x) \in \princ{p(x)}$. Hence $a(x) \in \princ{p(x)}$ or $b(x) \in \princ{p(x)}$ by definition of prime ideal. So $a(x) = k(x)p(x)$ or $b(x) = k(x)p(x)$ for some $k(x) \in D[x]$, meaning $p(x) \vert a(x)$ or $p(x) \vert b(x)$.
\begin{mdframed}
    Let $p(x) = x^2 + 2$, $f(x) = 2x+3$, and $g(x) = 4x^2+3x+2$ be polynomials in $\Z_5[x]$.
    \begin{partquestions}{\roman*}
        \item Show that $p(x)$ is irreducible over $\Z_5$.
        \item Hence construct a field, $F$, of 25 elements.
        \item Let $I = \princ{p(x)}$ be an ideal of $F$.  Simplify the following cosets.
        \begin{partquestions}{\alph*}
            \item $(f(x) + I) + (g(x) + I)$
            \item $(f(x) + I)(g(x) + I)$
        \end{partquestions}
    \end{partquestions}
\end{mdframed}
\textbf{Solution}:\newline
 \begin{partquestions}{\roman*}
        \item Let $p(x) = x^2 + 2 \in \Z_5[x]$. One sees that
        \begin{itemize}
            \item $p(0) = 0^2 + 2 = 2 \neq 0$;
            \item $p(1) = 1^2 + 2 = 3 \neq 0$;
            \item $p(2) = 2^2 + 2 = 6 = 1 \neq 0$;
            \item $p(3) = 3^2 + 2 = 11 = 1 \neq 0$; and
            \item $p(4) = 4^2 + 2 = 18 = 3 \neq 0$,
        \end{itemize}
        so $p(x)$ is irreducible in $\Z_5$ by \myref{thrm-degree-2-or-3-irreducible-iff-has-no-zeroes}.

        \item $F = Z_5[x]/\princ{p(x)}$ is a field of 25 elements.

        \item \begin{partquestions}{\alph*}
            \item We see
            \begin{align*}
                \left(2x+3 + I\right) + \left(4x^2+3x+2 + I\right) &= (4x^2+5x+5) + I\\
                &= 4(x^2+2) + 5x - 3 + I\\
                &= 5x - 3 + I\\
                &= 0x + 2 + I
            \end{align*}
            so $(f(x) + I) + (g(x) + I) = 2 + I$.

            \item We note
            \begin{align*}
                (2x+3 + I)(4x^2+3x+2 + I) &= (8 x^3 + 18 x^2 + 13 x + 6) + I\\
                &= \left((8x + 18)(x^2+2) - 3x - 30\right) + I\\
                &= -3x - 30 + I\\
                &= 2x + I
            \end{align*}
            so $(f(x) + I)(g(x) + I) = 2x + I$.
        \end{partquestions}
    \end{partquestions}
\begin{mdframed}
    Fill in the rest of the multiplication table in \myref{example-Z3/<x^2+1>}.
\end{mdframed}
\textbf{Solution}:\newline
 \begin{partquestions}{\alph*}
        \item $x \times x = x^2 = 1(x^2+1) - 1 = -1 = 2$.

        \item We note
        \begin{align*}
            (x+1)(2x+1) &= 2x^2 + 3x + 1\\
            &= 2x^2 + 0x + 1\\
            &= 2x^2 + 1\\
            &= 2(2) + 1 & (\text{since } x^2 = 2)\\
            &= 5\\
            &= 2.
        \end{align*}

        \item We see
        \begin{align*}
            (2x+2)(x+2) &= 2\left((x+1)(x+2)\right)\\
            &= 2(1) & (\text{since} (x+1)(x+2) = 1 \text{ from table})\\
            &= 2.
        \end{align*}

        \item Note
        \begin{align*}
            (2x+2)(2x+1) &= 2(x+1)(2x+1)\\
            &= 2(2) & (\text{from }\textbf{(b)})\\
            &= 4\\
            &= 1.
        \end{align*}
    \end{partquestions}

\section*{Problems}
\begin{mdframed}
    Determine which of the following polynomials, if any, is/are irreducible over $\Q$.
    \begin{multicols}{2}
        \begin{partquestions}{\alph*}
            \item $f(x) = x^3 + x^2 + x + 1$
            \item $g(x) = 6x^3 + x^2 + x + 1$
            \item $h(x) = x^4 + 5$
            \item $F(x) = x^4 + 9$
            \item $G(x) = x^4 - 2x^3 + x^2 - x + 1$
            \item $H(x) = x^4 + 12x^3 + 54x^2 + 108x + 86$
        \end{partquestions}
    \end{multicols}
\end{mdframed}
\textbf{Solution}:\newline
 \begin{partquestions}{\alph*}
        \item Reducible. Since
        \[
            f(-1) = (-1)^3 + (-1)^2 + (-1) + 1 = 0
        \]
        thus $f(x)$ has a zero in $\Q$, meaning that it is reducible over $\Q$ (\myref{thrm-degree-above-1-reducible-if-has-zero}).

        \item Reducible. Since
        \[
            g\left(-\frac12\right) = 6\left(-\frac12\right)^3 + \left(-\frac12\right)^2 + \left(-\frac12\right) + 1 = 0
        \]
        thus $g(x)$ has a zero in $\Q$, meaning that it is reducible over $\Q$ (\myref{thrm-degree-above-1-reducible-if-has-zero}).

        \item Irreducible. Note that 5 does not divide 1, 5 divides 5, and $5^2 = 25$ does not divide 5, so $h(x)$ is irreducible over $\Q$ by Eisenstein's Criterion (\myref{thrm-eisenstein-criterion}) with the prime 5.

        \item Irreducible. Note that reducing the coefficients of $F(x)$ modulo 7 results in the polynomial $\bar{F}(x) = x^4 + 2$, and that
        \begin{itemize}
            \item $\bar{F}(0) = 0^4 + 2 = 2 \neq 0$;
            \item $\bar{F}(1) = 1^4 + 2 = 3 \neq 0$;
            \item $\bar{F}(2) = 2^4 + 2 = 18 = 4 \neq 0$;
            \item $\bar{F}(3) = 3^4 + 2 = 83 = 6 \neq 0$;
            \item $\bar{F}(4) = 4^4 + 2 = 258 = 6 \neq 0$;
            \item $\bar{F}(5) = 5^4 + 2 = 627 = 4 \neq 0$; and
            \item $\bar{F}(6) = 6^4 + 2 = 1928 = 3 \neq 0$,
        \end{itemize}
        so $F(x)$ is irreducible over $\Q$ by Mod 7 Irreducibility Test (\myref{thrm-mod-p-irreducibility-test}).

        \item Reducible. Since
        \[
            G(1) = (1)^4 - 2(1)^3 + (1)^2 - (1) + 1 = 0
        \]
        thus $G(x)$ has a zero in $\Q$ and so is reducible over $\Q$ (\myref{thrm-degree-above-1-reducible-if-has-zero}).

        \item Irreducible. As
        \begin{align*}
            H(x) &= x^4 + 12x^3 + 54x^2 + 108x + 86\\
            &= (x^4 + 12x^3 + 54x^2 + 108x + 81) + 5\\
            &= (x+3)^4 + 5,
        \end{align*}
        and since $x^4 + 5$ is irreducible by \textbf{(c)}, therefore $H(x)$ is also irreducible by a corollary of the Transformation Rule (\myref{corollary-irreducible-iff-translation-is-irreducible}).
    \end{partquestions}
\begin{mdframed}
    Assuming $\sqrt2$ is a real number, and by considering the polynomial $x^2 - 2$, prove that $\sqrt2$ is irrational.
\end{mdframed}
\textbf{Solution}:\newline
 Let $f(x) = x^2 - 2$. Reducing coefficients of $f(x)$ modulo 3 yields $\bar{f}(x) = x^2 + 1$. Note that in $\Z_3$ we have
    \begin{itemize}
        \item $\bar{f}(0) = 1 \neq 0$;
        \item $\bar{f}(1) = 1^2 + 1 = 2 \neq 0$; and
        \item $\bar{f}(2) = 2^2 + 1 = 5 = 2 \neq 0$.
    \end{itemize}
    Therefore $f(x)$ is irreducible over $\Q$ by Mod 3 Irreducibility Test (\myref{thrm-mod-p-irreducibility-test}), which means $f(x)$ has no zeroes in $\Q$ (\myref{thrm-degree-2-or-3-irreducible-iff-has-no-zeroes}). But in $\R$ one sees that $f(x)$ has the zeroes $\pm\sqrt2$. Therefore $\sqrt2 \notin \Q$.
\begin{mdframed}
    Prove that any primitive degree 1 polynomial in $\Z[x]$ is irreducible.
\end{mdframed}
\textbf{Solution}:\newline
 Suppose $f(x)$ is a primitive, degree 1 polynomial in $\Z[x]$. By way of contradiction, suppose $f(x)$ is reducible, meaning $f(x) = p(x)q(x)$ for some non-zero non-unit polynomials $p(x), q(x) \in \Z[x]$. Since $\Z[x]$ is an integral domain, we must have
    \[
        1 = \deg f(x) = \deg(p(x)q(x)) = \deg p(x) + \deg q(x)
    \]
    by \myref{thrm-polynomial-degree-properties}. So exactly one of $p(x)$ or $q(x)$ is constant; without loss of generality assume $p(x)$ is the constant polynomial. Set $p(x) = k \in \Z$ and we see $f(x) = kq(x)$. But $f(x)$ is primitive, meaning that a factorization of $kq(x)$ is only possible if $k = \pm1$, which are both units, a contradiction. Therefore, $f(x)$ is irreducible.
\begin{mdframed}
    Let $f(x) = x^3 + 6$.
    \begin{partquestions}{\roman*}
        \item Show that $f(x)$ is reducible over $\Z_7$.
        \item Write $f(x)$ as a product of irreducible polynomials over $\Z_7$.
    \end{partquestions}
\end{mdframed}
\textbf{Solution}:\newline
 \begin{partquestions}{\roman*}
        \item Note $f(1) = 1^3 + 6 = 7 = 0$ so $f(x)$ has a zero in $\Z_7$.

        \item As 1 is a zero of $f(x)$, thus $x-1 = x+6$ is a factor of $f(x)$ (\myref{corollary-factor-theorem}). Performing long division on $x+6$ we see
        \begin{align*}
            x^3 + 6 &= x^2(x+6) - 6x(x+6) + 36(x+6) - 210\\
            &= (x^2-6x+36)(x+6) - 210\\
            &= (x^2+x+1)(x+6) + 0 & (\text{Evaluating in }\Z_7)\\
            &= (x^2+x+1)(x+6).
        \end{align*}
        One can see that 2 is a zero of $x^2 + x + 1$ in $\Z_7$, so $x - 2 = x+5$ is a factor of $x^2 + x + 1$. Performing division on it yields
        \begin{align*}
            x^2 + x + 1 &= x(x+5) - 4(x+5) + 21\\
            &= (x-4)(x+5) + 21\\
            &= (x+3)(x+5) + 0 & (\text{Evaluating in }\Z_7)\\
            &= (x+3)(x+5)
        \end{align*}
        which means that
        \[
            x^3 + 6 = (x+3)(x+5)(x+6).
        \]
        We note that $x+3$, $x+5$, and $x+6$ are all irreducible polynomials in $\Z_7$, so we have accomplished our goal.
    \end{partquestions}
\begin{mdframed}
    Let $R = \Z_2[x]$.
    \begin{partquestions}{\alph*}
        \item Find all irreducible polynomials of degree 2 in $R$.
        \item Find all irreducible polynomials of degree 3 in $R$.
    \end{partquestions}
\end{mdframed}
\textbf{Solution}:\newline
 \begin{partquestions}{\alph*}
        \item We note that there are only 4 degree 2 polynomials in $\Z_2[x]$. We evaluate the reducibility of each of the polynomials.
        \begin{itemize}
            \item $\boxed{x^2}$ Reducible since 0 is a zero of $x^2$.
            \item $\boxed{x^2 + 1}$ Reducible since 1 is a zero of $x^2 + 1$.
            \item $\boxed{x^2+x}$ Reducible since 0 is a zero of $x^2 + x$.
            \item $\boxed{x^2+x+1}$ Irreducible since neither 0 nor 1 are zeroes of the polynomial (\myref{thrm-degree-2-or-3-irreducible-iff-has-no-zeroes}).
        \end{itemize}
        Thus the only degree 2 polynomial that is irreducible over $\Z_2[x]$ is $x^2+x+1$.

        \item We do the same for degree 3 polynomials. We note that there are 8 such polynomials.
        \begin{itemize}
            \item $\boxed{x^3}$ Reducible since 0 is a zero.
            \item $\boxed{x^3 + 1}$ Reducible since 1 is a zero.
            \item $\boxed{x^3 + x}$ Reducible since 0 is a zero.
            \item $\boxed{x^3 + x + 1}$ Irreducible since Irreducible since neither 0 nor 1 are zeroes of the polynomial.
            \item $\boxed{x^3 + x^2}$ Reducible since 0 is a zero.
            \item $\boxed{x^3 + x^2 + 1}$ Irreducible since neither 0 nor 1 are zeroes of the polynomial.
            \item $\boxed{x^3 + x^2 + x}$ Reducible since 0 is a zero.
            \item $\boxed{x^3 + x^2 + x + 1}$ Reducible since 1 is a zero.
        \end{itemize}
        Therefore the only irreducible degree 3 polynomials in $\Z_2[x]$ are $x^3+x+1$ and $x^3+x^2+1$.
    \end{partquestions}
\begin{mdframed}
    Let $p$ be a prime. Show that the number of reducible polynomials over $\Z_p$ of the form $x^2 + ax + b$, where $a,b \in \Z_p$, is $\frac{p(p+1)}2$.
\end{mdframed}
\textbf{Solution}:\newline
 A reducible polynomial of the required form must have the factorization $(x+\alpha)(x+\beta)$. We split into two cases.
    \begin{itemize}
        \item If $\alpha \neq \beta$, then there are $p$ possibilities for $\alpha$ and $p - 1$ possibilities for $\beta$. However, we need to account for commutativity of the two factors, so we divide the total by 2. This leaves us with a total of $\frac{p(p-1)}{2}$ distinct polynomials for this case.
        \item If instead $\alpha = \beta$, then there are just $p$ choses for $\alpha = \beta$.
    \end{itemize}
    All in all, there are $\frac{p(p-1)}{2} + p = \frac{p(p+1)}{2}$ distinct polynomials of the required form.
\begin{mdframed}
    We show a failure case of the Mod $p$ Irreducibility Test (\myref{thrm-mod-p-irreducibility-test}). Let $f(x) = x^4 + 1$.
    \begin{partquestions}{\alph*}
        \item Prove that $f(x)$ is irreducible over $\Q$.
        \item Show that $f(x)$ is reducible over $\Z_2$ via factorization.\newline
        For the rest of this part, assume $p > 2$.
        \begin{partquestions}{\roman*}
            \item Suppose there is an $r \in \Z_p$ satisfying $r^2 = k$ for some $k$ to be specified below. Deduce a factorization of $f(x)$ in the following cases.
            \begin{partquestions}{\alph*}
                \item $k = 2$
                \item $k = -1$
                \item $k = -2$
            \end{partquestions}
            \item Let $\Z_p^\ast = \Z_p \setminus \{0\}$. Explain why $\Z_p^\ast$ is a group under multiplication modulo $n$. In particular, explain why $\Z_p^\ast = \Un{p}$, the group of units modulo $p$.
            \item Explain why $\Un{p}$ is a cyclic group of even order.
            \item Hence, prove that one of the cases in \textbf{(b)(i)} must occur.
        \end{partquestions}
    \end{partquestions}
\end{mdframed}
\textbf{Solution}:\newline
 \begin{partquestions}{\alph*}
        \item Note that
        \begin{align*}
            (x+1)^4 + 1 &= (x^4 + 4x^3 + 6x^2 + 4x + 1) + 1\\
            &= x^4 + 4x^3 + 6x^2 + 4x + 2.
        \end{align*}
        We see that the prime 2 does not divide the leading coefficient 1, divides all other coefficients, and $2^2 = 4$ does not divide the constant term 2. Therefore, by Eisenstein's Criterion (\myref{thrm-eisenstein-criterion}) with the prime 2, we know $(x+1)^4 + 1$ is irreducible. Hence $x^4 + 1$ is irreducible by a corollary of the Transformation Rule (\myref{corollary-irreducible-iff-translation-is-irreducible}).

        \item Note that in $\Z_2[x]$ we have
        \begin{align*}
            x^4 + 1 &= x^4 + 2x + 1\\
            &= (x^2+1)^2
        \end{align*}
        so $f(x) = x^4+1$ is reducible over $\Z_2$.

        \begin{partquestions}{\roman*}
            \item \begin{partquestions}{\alph*}
                \item If $r^2 = 2$, then note
                \begin{align*}
                    x^4 + 1 &= (x^4 + 2x^2 +1) - 2x^2\\
                    &= (x^2+1)^2 - 2x^2\\
                    &= (x^2+1)^2 - r^2x^2 & (\text{since } r^2 = 2)\\
                    &= (x^2+1+rx)(x^2+1-rx)
                \end{align*}
                so $f(x)$ is reducible.

                \item If instead $r^2 = -1$, then one sees
                \begin{align*}
                    x^4 + 1 &= x^4 - (-1)\\
                    &= x^4 - r^2 & (\text{since } r^2 = -1)\\
                    &= (x^2+r)(x^2-r)
                \end{align*}
                so, once again, $f(x)$ is reducible.

                \item In the third case, if $r^2 = -2$, then
                \begin{align*}
                    x^4 + 1 &= (x^4 - 2x^2 +1) + 2x^2\\
                    &= (x^2-1)^2 - (-2)x^2\\
                    &= (x^2-1)^2 - r^2x^2 & (\text{since } r^2 = -2)\\
                    &= (x^2-1+rx)(x^2-1-rx)
                \end{align*}
                so $f(x)$ is, again, reducible.
            \end{partquestions}

            \item We note that, as sets, we obtain
            \begin{align*}
                \Z_p^\ast &= \{0, 1, 2, \dots, p-1\} \setminus \{0\}\\
                &= \{1, 2, \dots, p-1\}\\
                &= \left\{m \vert 1 \leq m < p \text{ and } \gcd(m,p) = 1\right\}\\
                &= \Un{p},
            \end{align*}
            where the third line is justified since any prime is coprime to any positive integer that is smaller than it. Now because the group operations on both sets is the same (i.e., multiplication modulo $p$), they must be the same group.

            \item There are $p - 1$ numbers from 1 to $p - 1$ inclusive. As $p > 2$, thus $p$ is odd and therefore $p - 1$ is even, meaning $\Un{p}$ is a group with even order.

            Also, as $p$ is an odd integer, there exists a primitive root modulo $p$ (\myref{axiom-primitive-root-modulo-p}). Therefore $\Un{p}$ is cyclic (\myref{prop-Un-cyclic-only-if-exists-primitive-root}), which therefore means that $\Un{p}$ is a cyclic group with even order.

            \item Let $g$ be the generator of $\Un{p}$. We consider again the three cases.
            \begin{itemize}
                \item If $2 = g^{2k}$ for some positive integer $k$, then the integer $r$ in question is $g^k$. The case in \textbf{(b)(i)(a)} applies.
                \item Otherwise, if $-1 = g^{2k}$ for some positive integer $k$, then the integer $r$ in question is $g^k$. The case in \textbf{(b)(i)(b)} applies.
                \item Otherwise, we must have $2 = g^{2m - 1}$ and $-1 = g^{2n - 1}$ for some positive integers $m$ and $n$. So one sees that
                \begin{align*}
                    -2 &= (2)(-1)\\
                    &= \left(g^{2m-1}\right)\left(g^{2n-1}\right)\\
                    &= g^{2m+2n-2}\\
                    &= g^{2(m+n-1)},
                \end{align*}
                so the integer $r$ in question is $g^{m+n-1}$ and the case in \textbf{(b)(i)(c)} applies.
            \end{itemize}
            In all cases, we obtain a factorization of $f(x)$.
        \end{partquestions}
    \end{partquestions}

\chapter{Domains}
\section*{Exercises}
\begin{mdframed}
    Let $D$ be an integral domain and $a,b \in D$. Prove that $a \vert b$ and $b \vert a$ if and only if $a$ and $b$ are associates.
\end{mdframed}
\textbf{Solution}:\newline
 For the forward direction, suppose $a \vert b$ and $b \vert a$. Then there exist elements $u, v \in D$ such that $b = au$ and $a = bv$. So
    \[
        a = bv = (au)v = auv
    \]
    which means $uv = 1$ by cancellation law (\myref{prop-domain-cancellation-law}). Thus $u$ (and also $v$) is a unit, which shows that $a$ and $b$ are associates.

    For the reverse direction, if $a$ and $b$ are associates, then there exists a unit $u \in D$ such that $a = bu$, which also means that $b = au^{-1}$. Therefore one clearly sees that $a \vert b$ and $b \vert a$.
\begin{mdframed}
    Prove \myref{prop-associates-of-irreducible-is-irreducible}.
\end{mdframed}
\textbf{Solution}:\newline
 Suppose $r$ is an irreducible and $q$ an associate of $r$, meaning there exists a unit $u$ in the integral domain (say $D$) such that $q = ru$. Suppose on the contrary that $q$ is reducible, meaning $q = ab$ where neither $a$ nor $b$ are units in $D$. Then $ru = ab$ which means $r = (u^{-1}a)b$. As $r$ is irreducible we know that either $b$ is a unit (contradicting the assumption) or $u^{-1}a$ is a unit. Since the product of units is a unit, thus $u(u^{-1}a) = a$ is a unit, contradicting the assumption. Therefore $q = ru$ is irreducible.
\begin{mdframed}
    Let $d$ be a square-free integer that is not 1. Prove that $x \in \Z[\sqrt{d}]$ is irreducible if $N(x) = p$, where $p$ is a prime number.
\end{mdframed}
\textbf{Solution}:\newline
 \begin{partquestions}{\alph*}
        \item Note $2 = (1+i)(1-i)$ and $5 = (2+i)(2-i)$, and one can easily verify that each of these factors do not have a norm of $\pm1$. Therefore 2 and 5 are reducible in $\Z[i]$, so they cannot be prime (contrapositive of \myref{thrm-in-integral-domain-primes-are-irreducibles}).

        \item We first note $N(x) \geq 0$ for all $x \in \Z[i]$.
        \begin{partquestions}{\roman*}
            \item Without loss of generality we consider only $a$. We note that
            \begin{itemize}
                \item if $a \equiv 0 \pmod4$ then $a^2 \equiv 0 \pmod4$;
                \item if $a \equiv 1 \pmod4$ then $a^2 \equiv 1 \pmod4$;
                \item if $a \equiv 2 \pmod4$ then $a^2 \equiv 4 \equiv 0 \pmod4$; and
                \item if $a \equiv 3 \pmod4$ then $a^2 \equiv 9 \equiv 1 \pmod4$.
            \end{itemize}
            So the sum of two squares modulo 4 can only be 0, 1, or 2. Thus it is impossible for $a^2+b^2 \equiv 3\pmod4$.

            \item Seeking a contradiction, suppose $3 = pq$ where $p,q\in\Z[i]$ are both non-units. Then
            \[
                9 = N(3) = N(pq) = N(p)N(q).
            \]
            Since $p$ and $q$ are non-units we must have $N(p) = N(q) = 3$ (since otherwise $N(p) = 1$ which means $p$ is a unit by \myref{prop-properties-of-quadratic-integer-norm}). So if $p = a+bi$ then $a^2+b^2 = 3$. But we just proved in \textbf{(b)(i)} that this is impossible. Thus 3 is irreducible.

            \item Let $x = m+ni$, so $N(x) = m^2 + n^2$. Given $9 \vert N(x)$, we know $m^2 + n^2 \equiv 0 \pmod9$. We note the possibilities of squares modulo 9.
            \begin{table}[H]
                \centering
                \begin{tabular}{|l|l|l|}
                    \hline
                    $k\mod9$ & $k^2$ & $k^2\mod9$ \\ \hline
                    0 & 0 & 0 \\ \hline
                    1 & 1 & 1 \\ \hline
                    2 & 4 & 4 \\ \hline
                    3 & 9 & 0 \\ \hline
                    4 & 16 & 7 \\ \hline
                    5 & 25 & 7 \\ \hline
                    6 & 36 & 0 \\ \hline
                    7 & 49 & 4 \\ \hline
                    8 & 64 & 1 \\ \hline
                \end{tabular}
            \end{table}
            One may then verify that the set of pairs of $(m,n)$ that results in $m^2 + n^2 \equiv 0 \pmod9$ is
            \[
                \left\{(0,0),(0,3),(0,6),(3,0),(3,3),(3,6),(6,0),(6,3),(6,6)\right\}.
            \]
            In each case, both $m$ and $n$ are divisible by 3. Say $m = 3p$ and $n = 3q$ for some integers $p$ and $q$. Then $x = m + ni = 3p + (3q)i = 3(p+qi)$ which therefore means $3 \vert x$.

            \item Suppose $3 \vert uv$ for some $u,v\in D$. This means $uv = 3w$ for some $w \in D$. Note
            \[
                N(u)N(v) = N(uv) = N(3w) = 9N(w)
            \]
            and $9 = 3\times3$. We have 3 cases.
            \begin{enumerate}[label=\arabic*.]
                \item If $9 \vert N(u)$ we may use \textbf{(b)(iii)} to see that $3 \vert u$.
                \item Otherwise, if $9 \vert N(v)$ we may again use \textbf{(b)(iii)} to see that $3 \vert v$.
                \item Otherwise, we must have $3 \vert N(u)$ and $3 \vert N(v)$. Let's focus on $3 \vert N(u)$ only; let $u = x+yi$ for some integers $x$ and $y$. Then $3 \vert x^2+y^2$. Note for any integer $n$,
                \begin{itemize}
                    \item $n \equiv 0 \pmod3$ means $n^2 \equiv 0 \pmod3$;
                    \item $n \equiv 1 \pmod3$ means $n^2 \equiv 1 \pmod3$; and
                    \item $n \equiv 2 \pmod3$ means $n^2 \equiv 4 \equiv 1 \pmod3$.
                \end{itemize}
                So for $x^2+y^2$ to be divisible by 3, both $x$ and $y$ must be multiples of 3. Let $x = 3m$ and $y = 3n$ for some integers $n$; we see
                \[
                    N(u) = x^2+y^2 = (3m)^2 + (3n)^2 = 9(m^2+n^2)
                \]
                which means $9 \vert N(u)$. Finally use \textbf{(b)(iii)} to see that $3 \vert u$.
            \end{enumerate}
            Hence in any case, $3 \vert uv$ means $3 \vert u$ or $3 \vert v$, meaning 3 is prime.
        \end{partquestions}
    \end{partquestions}
\begin{mdframed}
    Recall that the Gaussian integers is the set $\Z[i] = \Z[\sqrt{-1}]$.
    \begin{partquestions}{\alph*}
        \item Show that the numbers 2 and 5, which are primes in the positive integers, are no longer primes in the Gaussian integers.
        \item \begin{partquestions}{\roman*}
            \item Show that, for any two integers $a$ and $b$, it is impossible to have $a^2 + b^2 \equiv 3 \pmod4$.
            \item Prove that 3 is irreducible in the Gaussian integers.
            \item Let $x \in \Z[i]$. Show that if $9 \vert N(x)$ then $3 \vert x$.\newline
            (\textit{Hint: consider all possibilities of squares modulo 9.})
            \item Prove that 3 is a prime in the Gaussian integers, which is called a \textbf{Gaussian prime}\index{Gaussian prime}.
        \end{partquestions}
    \end{partquestions}
\end{mdframed}
\textbf{Solution}:\newline
 Suppose $N(x) = p$ where $p$ is a prime number, and $x = ab$ where $a,b \in \Z[\sqrt{d}]$. Taking norms yields $N(x) = N(ab) = N(a)N(b) = p$. But since $p$ is prime, one of $N(a)$ or $N(b)$ is 1, which by \myref{prop-properties-of-quadratic-integer-norm}, statement 3, we know that $a$ or $b$ is a unit. Therefore $x$ is irreducible.
\begin{mdframed}
    Prove that multiplication over the field of fractions is a well-defined operation, and that it is commutative.
\end{mdframed}
\textbf{Solution}:\newline
 Suppose $\frac{a_1}{b_1} = \frac{a_2}{b_2}$ and $\frac{c_1}{d_1} = \frac{c_2}{d_2}$. We show that $\frac{a_1}{b_1} \times \frac{c_1}{d_1} = \frac{a_2}{b_2} \times \frac{c_2}{d_2}$, i.e. $\frac{a_1c_1}{b_1d_1} = \frac{a_2c_2}{b_2c_2}$, which is equivalent to showing that $(a_1c_1, b_1d_1) \mathrel{\sim} (a_2c_2, b_2d_2)$ by \myref{thrm-equivalence-class-equivalence}, i.e.
    \[
        a_1c_1b_2d_2 = b_1d_1a_2c_2.
    \]
    As $\frac{a_1}{b_1} = \frac{a_2}{b_2}$, we know that $(a_1, b_1) \mathrel{\sim} (a_2, b_2)$ (\myref{thrm-equivalence-class-equivalence}) which means $a_1b_2 = b_1a_2$. Similarly, because $\frac{c_1}{d_1} = \frac{c_2}{d_2}$ thus $c_1d_2 = d_1c_2$. Hence
    \begin{align*}
        a_1c_1b_2d_2 &= a_1b_2c_1d_2\\
        &= (a_1b_2)(c_1d_2)\\
        &= (b_1a_2)(d_1c_2)\\
        &= b_1d_1a_2c_2.
    \end{align*}
    so we have shown that $\frac{a_1}{b_1} \times \frac{c_1}{d_1} = \frac{a_2}{b_2} \times \frac{c_2}{d_2}$, meaning multiplication is well-defined.

    We now show that multiplication is commutative. Let $\frac ab, \frac cd \in \Frac{D}$. Then
    \begin{align*}
        \frac ab \times \frac cd &= \frac{ac}{bd}\\
        &= \frac{ca}{db} & (\text{since } D \text{ is commutative})\\
        &= \frac cd \times \frac ab
    \end{align*}
    so multiplication is commutative.
\begin{mdframed}
    Prove or disprove the following statements.
    \begin{partquestions}{\alph*}
        \item $\Frac{\Z[x]} \cong \Q[x]$.
        \item $\Frac{k\Z} \cong \Q$ for all $k \in \Z$.
    \end{partquestions}
\end{mdframed}
\textbf{Solution}:\newline
 \begin{partquestions}{\alph*}
        \item Disprove. One sees $1 \in \Z[x]$ and $x \in \Z[x]$ but $\frac1x \notin \Q[x]$.

        \item Prove. Define $\phi: \Frac{k\Z} \to \Q$ by $\phi([(ka, kb)]) = \frac ab$. We show that this is a well-defined isomorphism.
        \begin{itemize}
            \item \textbf{Well-defined}: Suppose $[(ka, kb)], [(kc, kd)] \in \Frac{k\Z}$ where $[(ka,kb)]=[(kc,kd)]$. Then $(ka, kb) \mathrel{\sim} (kc, kd)$ by \myref{thrm-equivalence-class-equivalence}, i.e. $k^2ad = k^2bc$. Hence one sees clearly that $\frac ab = \frac cd$. Therefore,
            \[
                \phi([(ka,kb)]) = \frac ab = \frac cd = \phi([(kc, kd)])
            \]
            and so $\phi$ is well-defined.

            \item \textbf{Homomorphism}: Let $[(ka, kb)], [(kc, kd)] \in \Frac{k\Z}$. Note
            \begin{align*}
                \phi([(ka,kb)] + [(kc,kd)]) &= \phi([(k^2(ad+bc), k^2bd)])\\
                &= \frac{k(ad+bc)}{kbd}\\
                &= \frac{k^2(ad+bc)}{k^2bd}\\
                &= \frac {ka}{kb} + \frac {kc}{kd}\\
                &= \phi([(ka,kb)]) + \phi([(kc,kd)])
            \end{align*}
            and
            \begin{align*}
                \phi([(ka,kb)] \times [(kc,kd)]) &= \phi([(k^2ac, k^2bd)])\\
                &= \frac{kac}{kbd}\\
                &= \frac{k^2ac}{k^2bd}\\
                &= \frac {ka}{kb} \times \frac {kc}{kd}\\
                &= \phi([(ka,kb)]) \times \phi([(kc,kd)])
            \end{align*}
            which shows that $\phi$ is a homomorphism.

            \item \textbf{Injective}: Let $[(ka, kb)], [(kc, kd)] \in \Frac{k\Z}$ such that $\phi([(ka, kb)]) = \phi([(kc, kd)])$. Then $\frac ab = \frac cd$, meaning $ad = bc$. Hence $(ka, kb) \mathrel{\sim} (kc, kd)$ by definition of the equivalence relation on $\Frac{k\Z}$, and so $[(ka, kb)] = [(kc, kd)]$ by \myref{thrm-equivalence-class-equivalence} again. Hence $\phi$ is injective.

            \item \textbf{Surjective}: Suppose $\frac ab \in \Q$ where $a, b \in \Z$ such that $b \neq 0$. Then note that $[(ka, kb)]$ is its pre-image since
            \[
                \phi([(ka, kb)]) = \frac ab
            \]
            which proves that $\phi$ is surjective.
        \end{itemize}
        Therefore $\phi$ is a well-defined ring isomorphism, meaning $\Frac{k\Z} \cong \Q$.
    \end{partquestions}
\begin{mdframed}
    Let $D$ be an integral domain, $a,b\in D$, and $b \neq 0$. Prove that $\princ{ab} \subset \princ{b}$ if and only if $a$ is not a unit.\newline
    (\textit{Hint: consider contrapositives.})
\end{mdframed}
\textbf{Solution}:\newline
 For the forward direction, we prove by contrapositive. If $a$ is a unit then $\princ{b} = \princ{ab}$ by \myref{prop-principal-ideals-equal-iff-associates}, so $\princ{ab} \not\subset \princ{b}$.

    For the reverse direction, we again consider contrapositive. Assume $\princ{ab}$ is not a proper subset of $\princ{b}$, meaning $\princ{b} \subseteq \princ{ab}$. But one clearly sees that $\princ{ab} \subseteq \princ{b}$. Thus $\princ{b} = \princ{ab}$ which means $a$ is a unit, again by \myref{prop-principal-ideals-equal-iff-associates}.
\begin{mdframed}
    Prove \myref{corollary-primitive-polynomial-irreducible-iff-polynomial-irreducible-in-field-of-fractions}.
\end{mdframed}
\textbf{Solution}:\newline
 We prove the contrapositive of the forward direction. Suppose $f(x)$ is reducible in $D[x]$. Then, since $f(x)$ is primitive, $f(x) = p(x)q(x)$ for some non-unit polynomials $p(x), q(x) \in D[x]$. Therefore $p(x), q(x) \in \Frac{D}$, which are non-units in $\Frac{D}$. Thus $f(x) = p(x)q(x)$ for some non-unit polynomials in $p(x), q(x) \in \Frac{D}$, so $f(x)$ is reducible in $\Frac{D}$.

    We now prove the contrapositive of the reverse direction. Suppose $f(x)$ is reducible in $\Frac{D}$, so there exists non-constant polynomials $p(x), q(x) \in \Frac{D}$ such that $f(x) = p(x)q(x)$. Then there exists polynomials $P(x), Q(x) \in D[x]$ such that $f(x) = P(x)Q(x)$ and $\deg P(x) = \deg p(x)$ and $\deg Q(x) = \deg q(x)$ by \myref{lemma-reducible-in-field-of-fractions-means-reducible-in-UFD}. Hence $f(x)$ is reducible in $D[x]$.
\begin{mdframed}
    Prove \myref{thrm-field-is-euclidean-domain}.
\end{mdframed}
\textbf{Solution}:\newline
 For a field $F$, choose the norm to be $N(x) = 1$ for any $x \in F$. Clearly \textbf{EF1} is satisfied since $1 = N(x) \leq N(xy) = 1$ for any $x,y\in F$. For \textbf{EF2}, for any $n, d \in F$, we can write $n = (nd^{-1})d + 0$. Note $nd^{-1}$ exists since we are in a field, so \textbf{EF2} is satisfied. Therefore $F$ is a Euclidean domain.
\begin{mdframed}
    Explain the rest of the class inclusions in \myref{figure-domain-class-inclusion} by citing the appropriate theorems.
\end{mdframed}
\textbf{Solution}:\newline
 We note
    \begin{itemize}
        \item all fields are Euclidean domains by \myref{thrm-field-is-euclidean-domain};
        \item all Euclidean domains are PIDs by \myref{thrm-euclidean-domain-is-PID}; and
        \item all PIDs are UFDs by \myref{thrm-PID-is-UFD}.
    \end{itemize}

\section*{Problems}
\begin{mdframed}
    Prove that if a prime $p$ can be written as $a^2+b^2$ where $a$ and $b$ are integers, then $a+bi$ is a Gaussian prime.
\end{mdframed}
\textbf{Solution}:\newline
 Given that we have integers $a$ and $b$ such that $a^2 + b^2$ is a prime, therefore $N(a+bi) = a^2+b^2$ is prime. Therefore $a+bi$ is irreducible (\myref{prop-properties-of-quadratic-integer-norm}). Since $\Z[i]$ is a Euclidean domain (\myref{example-gaussian-integers-is-euclidean-domain}) it is thus a PID (\myref{thrm-euclidean-domain-is-PID}) and so any irreducible in $\Z[i]$ is prime (\myref{thrm-in-PID-prime-iff-irreducible}).
\begin{mdframed}
    Let $D$ be an integral domain and $p$ a prime in $D$. Suppose $a_1, a_2, a_3, \dots, a_n$ are elements in $D$ such that $p \vert a_1a_2a_3\dots a_n$, where $n \geq 2$. Prove that $p \vert a_i$ for some $i \in \{1, 2, \dots, n\}$.
\end{mdframed}
\textbf{Solution}:\newline
 We induct on $n$. When $n = 2$, this reduces down to the definition of a prime element in an integral domain. Now suppose $p \vert a_1a_2\cdots a_k$ for some $k \geq 2$. We are to show that this works for $k + 1$.

    Note $p \vert a_1a_2\cdots a_ka_{k+1}$ is the same as $p \vert (a_1a_2\cdots a_k)a_{k+1}$. By definition of a prime element we know that $p \vert a_1a_2\cdots a_k$ or $p \vert a_{k+1}$. Therefore, using induction hypothesis on the former, we see $p\vert a_1$, or $p\vert a_2$, or $p \vert a_3$, etc., or $p \vert a_k$, or $p \vert a_{k+1}$, which proves the statement for $k + 1$.

    Therefore by induction we establish the required result.
\begin{mdframed}
    Determine whether the following Gaussian integers are Gaussian primes, proving your claims.
    \begin{multicols}{3}
        \begin{partquestions}{\alph*}
            \item 6
            \item 7
            \item 13
            \item $1+2i$
            \item $3+4i$
            \item $5+i$
        \end{partquestions}
    \end{multicols}
\end{mdframed}
\textbf{Solution}:\newline
 Since $\Z[i]$ is a PID (as shown earlier), any irreducible in $\Z[i]$ is prime (\myref{thrm-in-PID-prime-iff-irreducible}). So we just need to show whether the following are reducible or irreducible.
    \begin{partquestions}{\alph*}
        \item $6 = 2 \times 3$. As $N(2) = 4 \neq 1$ and $N(3) = 9 \neq 1$, neither 2 nor 3 are units in $\Z[i]$. Therefore 6 is reducible, and so it is not prime.

        \item We claim 7 is irreducible. Suppose $7 = wz$ for some $w,z \in \Z[i]$. Then $49 = N(wz) = N(w)N(z)$. Suppose neither $w$ nor $z$ are units, which means $N(w) = N(z) = 7$.

        If $w = a+bi$ for some integers $a$ and $b$, then $N(w) = a^2+b^2 = 7$. This means $a^2 = 7 - b^2$. For a solution to exist the right hand side cannot be negative, so we note that $b$ can only be -2, -1, 0, 1, or 2. For each one of these possibilities we see that $7 - b^2$ is not a square, so there is no solution for $w$ (and likewise for $z$).

        Hence one of $N(w)$ or $N(z)$ is 1, meaning that one of $w$ or $z$ is a unit. Hence 7 is irreducible and therefore prime.

        \item 13 is not since $13 = (3+2i)(3-2i)$, and one sees that $N(3+2i) = N(3-2i) = 13 \neq 1$, so neither $3+2i$ nor $3-2i$ are units. Hence 13 is reducible and therefore not prime.

        \item $1+2i$ is irreducible since $N(1+2i) = 5$ is prime (\myref{prop-properties-of-quadratic-integer-norm}). Hence $1+2i$ is prime.

        \item $3+2i$ is reducible since $3+4i = (2+i)^2$ and $N(2+i) = 5 \neq 1$, so $2+i$ is not a unit. Hence $3+2i$ is not prime.

        \item $5+i$ is reducible since $5+i = (1+i)(3-2i)$ and $N(1+i) = 2 \neq 1$ and $N(3-2i) = 13 \neq 1$, so neither $1+i$ nor $3-2i$ are units. Hence $5+i$ is not a prime.
    \end{partquestions}
\begin{mdframed}
    Let $D$ be an integral domain. Define a relation $R$ on $D$ such that $a\mathrel{R}b$ if and only if $a$ and $b$ are associates. Prove that $R$ is an equivalence relation.
\end{mdframed}
\textbf{Solution}:\newline
 To prove that $R$ is an equivalence relation, we need to show that $R$ is reflexive, symmetric, and transitive.
    \begin{itemize}
        \item \textbf{Reflexive}: Clearly for any $a \in D$ we see $a = 1a$. 1 is a unit since it is its own inverse. Hence $a$ is its own associate, so $a\mathrel{R}a$.

        \item \textbf{Symmetric}: Suppose $a,b \in D$ such that $a\mathrel{R}b$. Then $a = ub$ for some unit $u \in D$. Then $b = u^{-1}a$, and since $u^{-1}$ is a unit, thus $b$ and $a$ are associates. Hence $b\mathrel{R}a$.

        \item \textbf{Transitive}: Suppose $a,b,c\in D$ such that $a\mathrel{R}b$ and $b\mathrel{R}c$. Then there exist units $u,v \in D$ such that $a = ub$ and $b = vc$. Thus $a = uvc$. As the product of units is a unit \myref{prop-product-of-units-is-unit}, therefore $uv$ is a unit and so $a$ and $c$ are associates. Hence $a\mathrel{R}c$.
    \end{itemize}
    Therefore $R$ is an equivalence relation.
\begin{mdframed}
    Expand the polynomials $(3x+2)(x+4)$ and $(4x+1)(2x+3)$ in $\Z_5[x]$. Does this result contradict \myref{corollary-polynomial-ring-over-field-is-UFD}?
\end{mdframed}
\textbf{Solution}:\newline
 We see
    \begin{align*}
        (3x+2)(x+4) &= 3x^2 + 12x + 2x + 8\\
        &= 3x^2 + 14x + 8\\
        &= 3x^2 + 4x + 3
    \end{align*}
    in $\Z_5[x]$, and we also see
    \begin{align*}
        (4x+1)(2x+3) &= 8x^2 + 12x + 2x + 3\\
        &= 8x^2 + 14x + 3\\
        &= 3x^2 + 4x + 3
    \end{align*}
    in $\Z_5[x]$, which is the same polynomial as before.

    We note that this does not contradict \myref{corollary-polynomial-ring-over-field-is-UFD} since the factors, up to reordering, are associates. Note
    \begin{align*}
        3x+2 = 8x+2 = 2(4x+1)\\
        x+4 = 6x+9 = 3(2x3)
    \end{align*}
    and since $\Z_5$ is a field, so 2 and 3 are units as all non-zero elements in $\Z_5$ are units. Therefore $3x+2$ and $4x+1$ are associates, as are $x+4$ adn $2x+3$.
\begin{mdframed}
    Prove \myref{thrm-prime-element-iff-generates-prime-ideal}.
\end{mdframed}
\textbf{Solution}:\newline
 For the forward direction, suppose $p$ is a prime. Let $a, b \in D$ such that $ab \in \princ{p}$, which means there exists a $k \in D$ such that $ab = kp$. So $p \vert ab$. As $p$ is a prime, this means that either $p \vert a$ or $p \vert b$. Without loss of generality assume $p \vert a$, meaning $a = qp$ for some $q \in D$. Hence $a \in \princ{p}$, which shows that $\princ{p}$ is a prime ideal.

    For the reverse direction, suppose $\princ{p}$ is prime and $p \vert ab$. Thus $ab = kp$ for some $k \in D$, which means $ab \in \princ{p}$. So $a \in \princ{p}$ or $b \in \princ{p}$ by definition of prime ideal. Without loss of generality assume $a \in \princ{p}$; we see $a = qp$ for some $q \in D$, meaning $p \vert a$. Hence $p$ is a prime element.
\begin{mdframed}
    Let $F$ be a field. Prove that $\Frac{F} \cong F$.
\end{mdframed}
\textbf{Solution}:\newline
 Define $\phi: \Frac{F} \to F$ such that $[(a,b)]\mapsto ab^{-1}$. We show that $\phi$ is a well-defined ring isomorphism.
    \begin{itemize}
        \item \textbf{Well-defined}: Let $[(a,b)], [(c,d)] \in \Frac{F}$ and suppose $[(a,b)] = [(c,d)]$. Then $(a,b) \mathrel{\sim} (c,d)$ by \myref{thrm-equivalence-class-equivalence}, which means $ad = bc$ by definition of the equivalence relation on the field of fractions. Hence $ab^{-1} = cd^{-1}$ (remembering that the operation is commutative in a field), so
        \[
            \phi([(a,b)]) = ab^{-1} = cd^{-1} = \phi([(c,d)]),
        \]
        meaning that $\phi$ is well-defined.

        \item \textbf{Homomorphism}: Let $[(a,b)], [(c,d)] \in \Frac{F}$. Note
        \begin{align*}
            \phi([(a,b)] + [(c,d)]) &= \phi([(ad+bc, bd)])\\
            &= (ad+bc)(bd)^{-1}\\
            &= (ad)(bd)^{-1} + (bc)(bd)^{-1}\\
            &= add^{-1}b^{-1} + bcd^{-1}b^{-1}\\
            &= ab^{-1} + cd^{-1}\\
            &= \phi([(a,b)]) + \phi([(c,d)])
        \end{align*}
        and
        \begin{align*}
            \phi([(a,b)][(c,d)]) &= \phi([(ac,bd)])\\
            &= (ac)(bd)^{-1}\\
            &= (ab^{-1})(cd^{-1})\\
            &= \phi([(a,b)])\phi([(c,d)])
        \end{align*}
        so $\phi$ is a ring homomorphism.

        \item \textbf{Injective}: Let $[(a,b)], [(c,d)] \in \Frac{F}$ such that $\phi([(a,b)]) = \phi([(c,d)])$. Then $ab^{-1} = cd^{-1}$, so $ad = bc$. Hence $(a,b) \mathrel{\sim} (c,d)$ by definition of the equivalence relation on the field of fractions, so $[(a,b)] = [(c,d)]$ by \myref{thrm-equivalence-class-equivalence}.

        \item \textbf{Surjective}: Suppose $r \in F$. Note $r1^{-1} = r$. Therefore $\phi([(r, 1)]) = r1^{-1} = r$, meaning that every $r \in F$ has a pre-image under $\phi$.
    \end{itemize}
    Therefore $\phi$ is a well-defined ring isomorphism, meaning that $\Frac{F} \cong F$.
\begin{mdframed}
    Prove that $\Z[\sqrt{-6}]$ is not a UFD.\newline
    (\textit{Hint: show that some specific elements are irreducible, and then factor 6 in two ways.})
\end{mdframed}
\textbf{Solution}:\newline
 We note that the only units of $\Z[\sqrt{-6}]$ are $\pm1$. We claim that 2, 3, and $\sqrt{-6}$ are all irreducible in $\sqrt{-6}$.

    We first show that 2 is irreducible in $\Z[\sqrt{-6}]$. Suppose $2 = (a+b\sqrt{-6})(c+d\sqrt{-6})$ for some integers $a,b,c,d\in \Z$. So
    \[
        4 = N(2) = N((a+b\sqrt{-6})(c+d\sqrt{-6})) = (a^2+6b^2)(c+6d^2)
    \]
    which means that $a^2+6b^2$ must be 1, 2, or 4.
    \begin{itemize}
        \item If $a^2+6b^2 = 1$, then one sees that we must have $a = \pm1$ and $b = 0$, which means $a+b\sqrt{-6} = \pm1$ is a unit.
        \item If $a^2+6b^2 = 2$ then $a^2 = 2 - 6b^2$. As $b^2 \geq 0$, we see that $b = 0$ is the only way for the right hand side to not be negative. But $a^2 = 2$ has no solution. Thus $a^2+6b^2 = 2$ is impossible.
        \item If $a^2+6b^2 = 4$ then $c^2+6d^2 = 1$. We may then use the argument for the first case to show that $c+d\sqrt{-6} = \pm1$ which is a unit.
    \end{itemize}
    Therefore 2 is an irreducible.

    We next show that 3 is irreducible in $\Z[\sqrt{-6}]$. Suppose $3 = (a+b\sqrt{-6})(c+d\sqrt{-6})$ for some integers $a,b,c,d\in \Z$. So
    \[
        9 = N(3) = N((a+b\sqrt{-6})(c+d\sqrt{-6})) = (a^2+6b^2)(c+6d^2)
    \]
    which means that $a^2+6b^2$ must be 1, 3, or 9.
    \begin{itemize}
        \item If $a^2+6b^2 = 1$, then $a+b\sqrt{-6} = \pm1$ is a unit by previous working.
        \item If $a^2+6b^2 = 3$ then $a^2 = 3 - 6b^2$. As $b^2 \geq 0$, we see that $b = 0$ is the only way for the right hand side to not be negative. But $a^2 = 3$ has no solution. Thus $a^2+6b^2 = 3$ is impossible.
        \item If $a^2+6b^2 = 4$ then $c^2+6d^2 = 1$ which means $c+d\sqrt{-6} = \pm1$ which is a unit.
    \end{itemize}
    Therefore 3 is an irreducible.

    We finally show that $\sqrt{-6}$ is an irreducible in $\Z[\sqrt{-6}]$. Suppose $\sqrt{-6} = (a+b\sqrt{-6})(c+d\sqrt{-6})$ for some integers $a,b,c,d\in \Z$. So
    \[
        6 = N(\sqrt{-6}) = N((a+b\sqrt{-6})(c+d\sqrt{-6})) = (a^2+6b^2)(c+6d^2)
    \]
    which means that $a^2+6b^2$ must be 1, 2, 3, or 6.
    \begin{itemize}
        \item If $a^2+6b^2 = 1$, then $a+b\sqrt{-6} = \pm1$ is a unit.
        \item If $a^2+6b^2 = 2$ is impossible by earlier working.
        \item If $a^2+6b^2 = 3$ is impossible by earlier working.
        \item If $a^2+6b^2 = 6$ then $c^2+6d^2 = 1$ so $c+d\sqrt{-6} = \pm1$ which is a unit.
    \end{itemize}
    Therefore $\sqrt{-6}$ is an irreducible.

    Clearly 2, 3, and $\sqrt{-6}$ are not associates of each other.

    Finally, notice $6 = 2 \times 3 = (-\sqrt{-6})(\sqrt{-6})$ which is two factorizations of 6 into irreducibles. Therefore $\Z[\sqrt{-6}]$ is not a UFD.
\begin{mdframed}
    Let $D$ be a UFD and let $p \in D$. Prove that $p$ is irreducible if and only if $p$ is prime.
\end{mdframed}
\textbf{Solution}:\newline
 For the forward direction, assume $p$ is an irreducible, and let $p \vert ab$ for some $a,b \in D$. Therefore $ab = pq$ for some $q \in D$. As $D$ is a UFD, we may factorise $a$, $b$, and $q$ into irreducibles, say $a_1, a_2, \dots, a_r$, $b_1, b_2, \dots, b_s$, and $q_1, q_2, \dots, q_t$ respectively. Therefore
    \[
        a_1a_2\cdots a_r b_1b_2\cdots b_s = p q_1q_2\cdots q_t.
    \]
    As $p$ is also an irreducible, $p$ must be a $a_i$ or $b_i$, which means $p$ divides $a$ or $b$.

    For the reverse direction, since UFDs are integral domains, therefore any prime is an irreducible (\myref{thrm-in-integral-domain-primes-are-irreducibles}).
\begin{mdframed}
    Let $D$ be a Euclidean domain with norm $N$. Prove the following statements.
    \begin{partquestions}{\alph*}
        \item $a,b \in D$ are associates if $N(a) = N(b)$.
        \item $u$ is a unit in $D$ if and only if $N(u) = N(1)$.
    \end{partquestions}
\end{mdframed}
\textbf{Solution}:\newline
 \begin{partquestions}{\alph*}
        \item Suppose $a$ and $b$ are associates in $D$. So there is a unit $u \in D$ such that $a = ub$. Note
        \begin{align*}
            N(a) &= N(ub) \leq N(b) \text{ and}\\
            N(b) &= N(u^{-1}a) \leq N(a)
        \end{align*}
        by \textbf{EF1}. Hence $N(a) = N(b)$.

        \item For the forward direction, suppose $u \in D$ is a unit. Then there exists a $v \in D$ such that $uv = 1$. Therefore $N(uv) = N(1)$. Note that
        \begin{align*}
            N(u) &\leq N(uv) = N(1) \text{ and}\\
            N(1) &\leq N(u1) = N(u)
        \end{align*}
        by \textbf{EF1}, so $N(u) = N(1)$.

        For the reverse direction, suppose $N(u) = N(1)$. By \textbf{EF2}, there exists $q, r \in D$ such that $1 = uq + r$ with $r = 0$ or $N(r) < N(u)$. As $N(u) = N(1)$, so $r = 0$ or $N(r) < N(1)$. But $N(1) \leq N(1x) = N(x)$ for all $x \in D$, so $N(1) > N(r)$ is impossible. Hence $r = 0$, meaning $uq = 1$, which shows that $u$ is a unit.
    \end{partquestions}
\begin{mdframed}
    Suppose $D$ is an integral domain and $\phi$ is a non-constant function from $D$ to $\mathbb{N} \cup \{0\}$ such that $\phi(xy) = \phi(x)\phi(y)$ for all $x,y\in D$. Prove that if $u$ is a unit in $D$ then $\phi(u) = 1$.
\end{mdframed}
\textbf{Solution}:\newline
 We first show that $\phi(1) = 1$. Let $\phi(1) = a$ for some $a \in \mathbb{N} \cup \{0\}$. Since $1 = 1 \times 1$, therefore
    \[
        a = \phi(1) = \phi(1 \times 1) = \phi(1)\phi(1) = a^2
    \]
    which shows $a^2 = a$. As we are in $\mathbb{N} \cup \{0\}$, thus $a = 0$ or $a = 1$. But if $\phi(1) = 0$ then we see
    \[
        \phi(x) = \phi(1x) = \phi(1)\phi(x) = 0\phi(x) = 0
    \]
    but $\phi$ is non-constant, a contradiction. Hence $\phi(1) = 1$.

    Now let $u$ be a unit in $D$, meaning that there exists a $v \in D$ such that $uv = 1$. So one sees
    \[
        1 = \phi(1) = \phi(uv) = \phi(u)\phi(v).
    \]
    As we working with the non-negative integers, the only way to a product to equal 1 we must have $\phi(u) = \phi(v) = 1$.
\begin{mdframed}
    Prove that, for any field $F$, there exist infinitely many irreducible polynomials in $F[x]$.
\end{mdframed}
\textbf{Solution}:\newline
 Suppose there are only finitely many irreducible polynomials within $F[x]$, say $p_1(x), p_2(x), \dots, p_n(x)$. As $F[x]$ is a PID (\myref{thrm-polynomial-ring-over-field-is-a-PID}) thus these polynomials are also prime (\myref{thrm-in-PID-prime-iff-irreducible}).

    Consider the polynomial
    \[
        q(x) = (p_1(x) + 1)(p_2(x) + 1)\cdots(p_n(x) + 1).
    \]
    By construction we see that none of $p_1(x), p_2(x), \dots, p_n(x)$ divide $q(x)$. Therefore, either
    \begin{itemize}
        \item $q(x)$ is prime and thus irreducible (\myref{thrm-in-integral-domain-primes-are-irreducibles}), meaning that the list of irreducible polynomials is not complete; or
        \item there is another prime (and thus irreducible) polynomial that divides $q(x)$, which also means that the above list of irreducible polynomials is not complete.
    \end{itemize}
    Either way, this contradicts that there are a finite number of prime (and this irreducible) polynomials, meaning that there are infinitely many irreducible polynomials in $F[x]$.
\begin{mdframed}
    Prove that for every non-trivial ideal $I$ of the Gaussian integers, the quotient ring $\Z[i]/I$ is finite.
\end{mdframed}
\textbf{Solution}:\newline
 Let $I$ be a non-trivial ideal in $\Z[i]$. As $\Z[i]$ is a PID (it is a Euclidean domain which means it is a PID), suppose $I = \princ{a+bi}$ for some $a,b \in \Z$. Note that
    \[
        a^2 + b^2 = (a+bi)(a-bi) \in I.
    \]

    Now let $x+yi \in \Z[i]$. Euclid's division lemma (\myref{lemma-euclid-division}) tells us that there exist $q_1, q_2, r_1, r_2 \in \Z$ such that
    \begin{align*}
        x &= q_1(a^2+b^2) + r_1\\
        y &= q_2(a^2+b^2) + r_2
    \end{align*}
    where $0 \leq r_1, r_2 \leq a^2+b^2$. So
    \begin{align*}
        (x+yi)+I &= \left(\left(q_1(a^2+b^2) + r_1\right) + \left(q_2(a^2+b^2) + r_2\right)i\right) + I\\
        &= \left((a^2+b^2)(q_1+q_2i) + r_1 + r_2i\right) + I\\
        &= (r_1 + r_2i) + I & (\text{as } a^2+b^2 \in I)
    \end{align*}
    Notice that $(x+yi)+I = (r_1+r_2i)+I$ is an element of $\Z[i]/I$ and that there are only finitely many integers within the interval $[0, a^2+b^2]$. Therefore $\Z[i]/I$ is finite, since $0 \leq r_1, r_2 \leq a^2+b^2$.
\begin{mdframed}
    Consider
    \[
        P(x) = (x + x^2 + x^3 + x^4 + x^5 + x^6)^2
    \]
    in the polynomial ring $\Z[x]$.
    \begin{partquestions}{\roman*}
        \item Explain why the coefficient of the term of degree $k$ (where $2 \leq k \leq 12$) gives exactly the number of ways that a pair of two dice could sum to $k$.

        \item Factor $P(x)$ into a product of irreducible polynomials. Prove that the factorization, indeed, consists only of irreducible polynomials.

        \item Define the polynomials
        \begin{align*}
            f(x) &= x^{a_1} + x^{a_2} + x^{a_3} + x^{a_4} + x^{a_5} + x^{a_6} \text{ and}\\
            g(x) &= x^{b_1} + x^{b_2} + x^{b_3} + x^{b_4} + x^{b_5} + x^{b_6},
        \end{align*}
        where each $a_i$ and $b_i$ are positive integers. Suppose $P(x) = f(x)g(x)$. Based on \textbf{(ii)}, explain why
        \[
            f(x) = x^q(x+1)^r(x^2+x+1)^s(x^2-x+1)^t
        \]
        where $0 \leq q,r,s,t \leq 2$.

        \item Show $r = s = 1$, and $q \neq 0$. Note that we define $0^0 = 1$ for this part only.\newline
        (\textit{Hint: compute $f(1)$ in two ways. Then compute $f(0)$ in two ways.})

        \item Explain why $q \neq 2$.

        \item Hence, show that one possible choice of $a_1, a_2, \dots, a_6$ is 1, 2, 2, 3, 3, and 4. Find the other choice(s).

        \item Deduce the labels of a pair of dice (that are not the standard labels of 1 to 6 on both dice) that will lead to the same likelihood of rolling a sum of, say, $k$ (where $2 \leq k \leq 12$), as regular dice.
    \end{partquestions}
\end{mdframed}
\textbf{Solution}:\newline
 As $\Z[x]$ is a UFD (\myref{corollary-Z-is-UFD}) we note that the factorizations mentioned below are unique.
    \begin{partquestions}{\roman*}
        \item We note that if one die shows $n$, then the other die must have a value of $k - n$ in order for both dice's sum to be $k$. Similarly, in order to obtain a term with degree $k$, and if the first term to multiply is $x^n$, then the other has to be $x^{k-n}$ in order for their product to have degree $k$.

        \item $P(x) = x^2(x+1)^2(x^2+x+1)^2(x^2-x+1)^2$.

        We know that both $x$ and $x+1$ are irreducible by \myref{problem-primitive-degree-1-polynomial-in-Z[x]-is-irreducible}.

        $x^2+x+1$ is irreducible since it is equal to $(x+\frac12)^2 + \frac34$ which is never zero for all integers $x$, which means that $x^2+x+1$ has no zeroes in $\Z$ and so it is irreducible (\myref{thrm-degree-2-or-3-irreducible-iff-has-no-zeroes}).

        $x^2-x+1$ is irreducible since it is equal to $(x-\frac12)^2 + \frac34$ which is never zero for all integers $x$, which means that $x^2-x+1$ has no zeroes in $\Z$ and so it is irreducible (\myref{thrm-degree-2-or-3-irreducible-iff-has-no-zeroes}).

        Hence the factorization given here is indeed the unique factorization into irreducibles.

        \item Since the factorization in \textbf{(ii)} is into irreducibles, $f(x)$ must consist of the same irreducibles. Thus $f(x) = x^q(x+1)^r(x^2+x+1)^s(x^2-x+1)^t$.

        Note that $0 \leq q,r,s,t \leq 2$, since otherwise it would exceed the number of factors of the appropriate irreducible in the factorization of $P(x)$ obtained in \textbf{(ii)}.

        \item We see
        \[
            f(1) = 1^{a_1} + 1^{a_2} + \cdots + 1^{a_6} = 6
        \]
        and
        \[
            f(1) = 1^q(1+1)^r(1+1+1)^s(1-1+1)^t = 2^r3^s
        \]
        so $6 = 2^r3^s$. But as $6 = 2 \times 3$, therefore $r = s = 1$.

        We also note that
        \[
            f(0) = 0^{a_1} + 0^{a_2} + \cdots + 0^{a_6} = 0
        \]
        and
        \[
            f(0) = 0^q(0+1)^r(0+0+1)^s(0+0+1)^t = 0^q
        \]
        so $0^q = 0$. As we defined $0^0 = 1$, we see $q \neq 0$.

        \item So far we have $f(x) = x^q(x+1)(x^2+x+1)(x^2-x+1)^t$. If $q = 2$ then $f(x) = x^2(x+1)(x^2+x+1)(x^2-x+1)^t$. But since $P(x) = f(x)g(x)$ and the smallest sum of two dice is 2, that means one of the $b_i$ in $g(x)$ must be zero. But in \textbf{(iii)} we said that every $b_i$ is a positive integer, a contradiction. Hence $q \neq 2$.

        \item We deduce from \textbf{(iv)} and \textbf{(v)} that $q = 1$, meaning $f(x) = x(x+1)(x^2+x+1)(x^2-x+1)^t$. We now list all possibilities for $f(x)$.
        \begin{itemize}
            \item If $t = 0$ then $f(x) = x + 2x^2 + 2x^3 + x^4$, so one choice is 1, 2, 2, 3, 3, and 4.
            \item If $t = 1$ then $f(x) = x + x^2 + x^3 + x^4 + x^5 + x^6$, so one choice is 1, 2, 3, 4, 5, and 6 (which is the normal choice for a dice).
            \item If $t = 2$ then $f(x) = x + x^3 + x^4 + x^5 + x^6 + x^8$, so another choice is 1, 3, 4, 5, 6, and 8.
        \end{itemize}

        \item The labels of the other pair of dice are
        \[
            1,2,2,3,3,4 \quad\text{and}\quad 1,3,4,5,6,8
        \]
        which are known as Sicherman dice.
    \end{partquestions}

\chapter{Encryption}
\section*{Exercises}
\begin{mdframed}
    Find an inverse of $f(x) = x + 2$ when $N = 2$ and $p = 4$.
\end{mdframed}
\textbf{Solution}:\newline
 We claim that $x+2$ is its own inverse when $N = 2$ and $p = 4$, since
    \begin{align*}
        (x+2)^2 &= x^2 + 4x + 4\\
        &= x^0 + 4x + 4 & (\text{reduce powers modulo } N = 2)\\
        &= 4x + 5\\
        &= 1 & (\text{reduce coefficients modulo } p = 4)
    \end{align*}
\begin{mdframed}
    Show that a polynomial in $\mathcal{L}(k,k)$, where $k$ is some positive integer smaller than $\frac N2$, does not have an inverse modulo $p$ for any $p$.
\end{mdframed}
\textbf{Solution}:\newline
 Suppose $f(x) \in \mathcal{L}(k,k)$. Then $f(x)$ has exactly $k$ coefficients of 1 and $k$ coefficients of -1. Therefore, evaluating $f(1)$ yields a result of 0.

    The equivalent requirement for $f(x)$ to have an inverse modulo $p$, say $F_p(x)$, is that $F_p(x)f(x) = 1$ within the ring $\Z_p[x]$. However $f(1) = 0$, meaning that $f(x)$ has a zero in $\Z_p$, while the constant polynomial 1 does not. Therefore, there cannot be a polynomial $F_p(x)$ such that $F_p(x)f(x) = 1$ in $\Z_p[x]$, i.e. $f(x)$ does not have an inverse.
\begin{mdframed}
    Let $N = 6$, $p = 5$, and $q = 16$, and suppose Bob chooses $f(x) = x^3 + x^2 - 1$ and $g(x) = x^4 - x^3 - x + 1$. Given
    \begin{align*}
        F_5(x) &= 3x^5 + 4x^4 + x^3 + 2x^2 + 2x + 4,\\
        F_{16}(x) &= 7x^5 + 5x^4 + 14x^3 + 9x^2 + 12x + 2,
    \end{align*}
    what is Bob's public key?
\end{mdframed}
\textbf{Solution}:\newline
 Bob's public key is
    \begin{align*}
        h(x) &= F_{16}g(x)\\
        &= (7x^5 + 5x^4 + 14x^3 + 9x^2 + 12x + 2)(x^4 - x^3 - x + 1)\\
        &= 7x^9 - 2x^8 + 9x^7 - 12x^6 + 5x^5 - 19x^4 + 3x^3 - 3x^2 + 10x + 2\\
        &= 7x^3 - 2x^2 + 9x - 12 + 5x^5 - 19x^4 + 3x^3 - 3x^2 + 10x + 2\\
        &= 5x^5 - 19x^4 + 10x^3 - 5x^2 + 19x - 10\\
        &\equiv 5x^5 + 13x^4 + 10x^3 + 11x^2 + 3x + 6 \pmod{16}.
    \end{align*}
\begin{mdframed}
    Convert the character `U' into a representative polynomial.
\end{mdframed}
\textbf{Solution}:\newline
 $x^6 + x^4 + x^2 + 1$
\begin{mdframed}
    Let $N = 8$, $p = 3$, $q = 64$. Let Bob's public key be
    \[
        h(x) = 24x^7 + 15x^6 + 50x^5 + 48x^4 + 40x^3 + 46x^2 + 14x + 19.
    \]
    Alice wants to transmit the character `U' to Bob. Suppose Alice encodes `U' using the ASCII conversion table (\myref{table-ASCII-conversion-table}) and obtains an `encoding fuzz' polynomial of $x - 1$. What is the encrypted message?
\end{mdframed}
\textbf{Solution}:\newline
 For this case, $\phi(x) = x - 1$ and $m(x) = x^6 + x^4 + x^2 + 1$. Thus
    \begin{align*}
        e(x) &= 3\phi(x)h(x) + m(x) \\
        &= 3(x-1)(24x^7 + 15x^6 + 50x^5 + 48x^4 + 40x^3 + 46x^2 + 14x + 19)\\
        &\quad\quad + (x^6 + x^4 + x^2 + 1)\\
        &= 72x^8 - 27x^7 + 106x^6 - 6x^5 - 23x^4 + 18x^3 - 95x^2 + 15x - 56\\
        &= 72 - 27x^7 + 106x^6 - 6x^5 - 23x^4 + 18x^3 - 95x^2 + 15x - 56\\
        &= -27x^7 + 106x^6 - 6x^5 - 23x^4 + 18x^3 - 95x^2 + 15x + 16\\
        &\equiv 37x^7 + 42x^6 + 58x^5 + 41x^4 + 18x^3 + 33x^2 + 15x + 16 \pmod{64}
    \end{align*}
    which means the encrypted message polynomial is $37x^7 + 42x^6 + 58x^5 + 41x^4 + 18x^3 + 33x^2 + 15x + 16$.
\begin{mdframed}
    Suppose $N = 8$, $p = 3$, and $q = 64$. Bob received an encrypted message from Alice,
    \[
        e(x) = 8x^7 + 12x^6 + 52x^5 + 26x^4 +x^3 + 62x^2 + 4x + 29,
    \]
    that was encrypted using his private key
    \[
        f(x) = x^7 +x^6 +x^5 -x^4 -x^3 -x + 1
    \]
    which has an inverse modulo 3 of
    \[
        F_3(x) = 2x^6 +x^5 +x^4 + 2x^2 + 2x + 2.
    \]
    What was the \textit{character} that Alice sent to him?\newline
    (\textit{Note: although the private key and its inverse look familiar, we used a different $g(x)$ and $\phi(x)$ from the previous exercises to encrypt the message, so the answer from the previous exercises may not match.})
\end{mdframed}
\textbf{Solution}:\newline
 We first compute $a(x)$,
    \begin{align*}
        a(x) &= f(x)e(x)\\
        &= (x^7 +x^6 +x^5 -x^4 -x^3 -x + 1)(8x^7 + 12x^6 + 52x^5 + 26x^4 +x^3 +\\
        &\quad\quad62x^2 + 4x + 29)\\
        &= 8x^{14} + 20x^{13} + 72x^{12} + 82x^{11} + 59x^{10} + 25x^9 - 19x^8 + 64x^7 - 70x^6 - 11x^5\\
        &\quad\quad - 8x^4 - 90x^3 + 58x^2 - 25x + 29\\
        &= 8x^6 + 20x^5 + 72x^4 + 82x^3 + 59x^2 + 25x - 19 + 64x^7 - 70x^6 - 11x^5\\
        &\quad\quad - 8x^4 - 90x^3 + 58x^2 - 25x + 29\\
        &= 64x^7 - 62x^6 + 9x^5 + 64x^4 - 8x^3 + 117x^2 + 10\\
        &\equiv 2x^6 + 9x^5 - 8x^3 - 11x^2 + 10 \pmod{64},
    \end{align*}
    remembering that we choose coefficients of $a(x)$ to be between -32 and 32. Then we compute $b(x)$ by reducing the coefficients of $a(x)$ modulo 3,
    \[
        b(x) \equiv -x^6 + x^3 + x^2 + 1 \pmod{3},
    \]
    again remembering that we choose coefficients of $b(x)$ to be between -1 and 1. Finally we retrieve the original message polynomial,
    \begin{align*}
        F_3(x)b(x) &= (2x^6 +x^5 +x^4 + 2x^2 + 2x + 2)(-x^6 + x^3 + x^2 + 1)\\
        &= -2x^{12} - x^{11} - x^{10} + 2x^9 + x^8 + x^6 + 3x^5 + 5x^4 + 4x^3 + 4x^2 + 2x + 2\\
        &= -2x^4 - x^3 - x^2 + 2x + 1 + x^6 + 3x^5 + 5x^4 + 4x^3 + 4x^2 + 2x + 2\\
        &= x^6 + 3x^5 + 3x^4 + 3x^3 + 3x^2 + 4x + 3\\
        &\equiv x^6 + x \pmod{3},
    \end{align*}
    which is the character `B' encoded using \myref{table-ASCII-conversion-table}.

\section*{Problems}

\chapter{Basics Of Fields}
\section*{Exercises}
\begin{mdframed}
    Prove the following statements \textit{without} using any results from the previous chapters.
    \begin{partquestions}{\alph*}
        \item $-0 = 0$.
        \item $1^{-1} = 1$.
    \end{partquestions}
\end{mdframed}
\textbf{Solution}:\newline
 \begin{partquestions}{\alph*}
        \item Clearly $0 = 0 - 0$ since $x - x = 0$ for all elements $x$ in $F$, including $x = 0$. But since 0 is the additive identity, this means $0 + x = x$ for all $x$, including $x = -0$. Hence $0 = -0$.
        \item Clearly $1 = 1\times 1^{-1}$ since $xx^{-1} = 1$ for all $x \in F^\ast$, including $x = 1$. But since 1 is the multiplicative identity, this means $1x = x$ for all $x$, including $x = 1^{-1}$. So $1 = 1^{-1}$.
    \end{partquestions}
\begin{mdframed}
    Let $d$ be a square-free integer that is not 1. Show that $\Q[\sqrt{d}]$ is a subfield of $\C$.
\end{mdframed}
\textbf{Solution}:\newline
 We already proved that $\Q[\sqrt{d}]$ is a field (\myref{prop-quadratic-field-is-a-field}). Now if $d \geq 0$ then $\sqrt{d} \in \R$, meaning that $\Q[\sqrt{d}] \subset \R \subset \C$. If instead $d < 0$, we may write $d = -r$, so $\Q[\sqrt{d}] = \Q[\sqrt{-r}] = \Q[i\sqrt{r}] \subset \C$. Hence $\Q[\sqrt{d}]$ is a subfield of $\C$, by using the definition of a subfield.
\begin{mdframed}
    Prove or disprove: $\R$ is a prime field.
\end{mdframed}
\textbf{Solution}:\newline
 Disprove. $\Q$ is a proper subfield of $\R$.

\section*{Problems}
\begin{mdframed}
    Find a ring without identity that is contained in a field.
\end{mdframed}
\textbf{Solution}:\newline
 Note $2\Z$ is a ring (as it is a principal ideal) that is contained in the field $\Q$.
\begin{mdframed}
    Prove or disprove: $\R \cong \C$ as fields.
\end{mdframed}
\textbf{Solution}:\newline
 Disprove. Suppose $\phi: \C \to \R$ is an isomorphism. Then there must exist a $w \in \C$ such that $\phi(w) = -1$. Note that $\sqrt{w} \in \C$; let $z = \sqrt{w}$. Then
    \[
        \left(\phi(z)\right)^2 = \phi\left(z^2\right) = \phi(w) = -1.
    \]
    So $\phi(z) = \sqrt{-1} = i$, the imaginary unit. But $\phi(z) \in \R$ and $\phi(z) = i \notin \R$, a contradiction. Therefore $\R \not\cong \C$.
\begin{mdframed}
    Show that $\Z_3[x]/\princ{x^2+x+1}$ is \textit{not} a field.
\end{mdframed}
\textbf{Solution}:\newline
 Let $f(x) = x^2 + x + 1$. Note $f(1) = 1^2 + 1 + 1 = 3 = 0$ in $\Z_3$, so $f(x)$ is reducible over $\Z_3$ (\myref{thrm-degree-above-1-reducible-if-has-zero}). Thus $\princ{x^2+x+1}$ is not a maximal ideal (\myref{thrm-irreducible-iff-principal-ideal-is-maximal}), so $\Z_3[x]/\princ{x^2+x+1}$ is not a field (\myref{thrm-maximal-ideal-iff-quotient-ring-is-field}).
\begin{mdframed}
    Find an irreducible polynomial in $\Z_2[x]$. Hence construct a field $F$ of order 4. Find the prime subfield of $F$.
\end{mdframed}
\textbf{Solution}:\newline
 Let $f(x) = x^2 + x + 1$. Note $f(x)$ is irreducible over $\Z_2$ since $f(0) = 1 \neq 0$ and $f(1) = 3 = 1 \neq 0$ in $\Z_2$ (\myref{thrm-degree-2-or-3-irreducible-iff-has-no-zeroes}).

    $F = \Z_2[x]/\princ{x^2+x+1}$.

    Let $I = \princ{x^2+x+1}$, and let the subset $K = \{0 + I, 1 + I\}$. We claim that $K$ is a subfield of $F$.
    \begin{itemize}
        \item $K^\ast = \{1 + I\} \neq \emptyset$.
        \item One sees clearly that
        \begin{itemize}
            \item $(0 + I) - (0 + I) = (0 + I) \in K$;
            \item $(0 + I) - (1 + I) = (-1 + I) = (1 + I) \in K$;
            \item $(1 + I) - (0 + I) = (1 + I) \in K$; and
            \item $(1 + I) - (1 + I) = (0 + I) \in K$,
        \end{itemize}
        so for any $x, y \in K$ we have $x - y \in K$.
        \item One sees that $(1+I)^{-1} = 1+I$ since $(1+I)(1+I) = (1+I)$. Thus
        \begin{itemize}
            \item $(0+I)(1+I)^{-1} = (0+I)(1+I) = (0+I) \in K$; and
            \item $(1+I)(1+I)^{-1} = (1+I)(1+I) = (1+I) \in K$,
        \end{itemize}
        so for any $x \in K$ and $y \in K^\ast$ we have $xy^{-1} \in K$.
    \end{itemize}
    Therefore $K$ is a subfield of $F$ by subfield test (\myref{thrm-subfield-test}). Clearly $K$ cannot contain a smaller subfield, since a field must contain at least 2 elements. Therefore, $K$ is the smallest subfield of $F$, meaning $K$ is the prime subfield.
\begin{mdframed}
    Consider the ring $R = \Z[i] / \princ{2-i}$.
    \begin{partquestions}{\roman*}
        \item Show that $5 \in \princ{2-i}$ and $2 + \princ{2-i} = i + \princ{2-i}$.
        \item Show that $R = \{k + \princ{2-i} \vert k \in \Z_5\}$.
        \item Hence prove, via an isomorphism, that $R$ is a field with order 5.
    \end{partquestions}
\end{mdframed}
\textbf{Solution}:\newline
 \begin{partquestions}{\roman*}
        \item Note $5 = (2-i)(2+i) \in \princ{2-i}$ and
        \[
            i + \princ{2-i} = i + (2-i) + \princ{2-i} = 2 + \princ{2-i}.
        \]

        \item Suppose $a + bi + \princ{2-i} \in \Z[i]/\princ{2-i}$. Note that
        \begin{align*}
            a + bi + \princ{2-i} &= (a+\princ{2-i}) + b(i + \princ{2-i})\\
            &= (a+\princ{2-i}) + b(2 + \princ{2-i}) & (\text{by }\textbf{(i)})\\
            &= (a + 2b) + \princ(2-i)\\
            &= n + \princ{2-i}
        \end{align*}
        for some $n \in \Z$. Now using Euclid's division lemma (\myref{lemma-euclid-division}), write $n = 5q + r$ where $r \in \Z_5$. Then we see
        \begin{align*}
            a + bi + \princ{2-i} &= 5q + r + \princ{2-i}\\
            &= (r + \princ{2-i}) + q(5 + \princ{2-i})\\
            &= (r + \princ{2-i}) + q(0 + \princ{2-i}) & (\text{since } 5 \in \princ{2-i})\\
            &= r + \princ{2-i}.
        \end{align*}
        Therefore any element of $\Z[i]/\princ{2-i}$ can be written as $r + \princ{2-i}$, where $r \in \Z_5$, as required.

        \item Define $\phi: \Z_5 \to R$ where $r \mapsto r + \princ{2-i}$. We show that $\phi$ is a surjective homomorphism.
        \begin{itemize}
            \item \textbf{Homomorphism}: We note that for $m,n \in \Z_5$ that
            \begin{align*}
                \phi(m+n) &= (m+n) + \princ{2-i}\\
                &= (m + \princ{2-i}) + (n + \princ{2-i})\\
                &= \phi(m) + \phi(n)
            \end{align*}
            and
            \begin{align*}
                \phi(mn) &= (mn) + \princ{2-i}\\
                &= (m + \princ{2-i})(n + \princ{2-i})\\
                &= \phi(m)\phi(n)
            \end{align*}
            and so $\phi$ is a homomorphism.

            \item \textbf{injective}: Since $\phi$ is non-trivial it is thus injective by \myref{thrm-homomorphism-from-field-is-injective-or-trivial}.

            \item \textbf{Surjective}: Let $a+bi + \princ{2-i} \in R$. Using \textbf{(ii)}, find the $r \in \Z_5$ such that $a+bi + \princ{2-i} = r + \princ{2-i}$. Then $\phi(r) = r + \princ{2-i} = a+bi + \princ{2-i}$, meaning $\phi$ is surjective.
        \end{itemize}
        Therefore we see $\phi$ is an isomorphism, meaning $R \cong \Z_5$, which shows that $R$ is a field of order 5.
    \end{partquestions}
\begin{mdframed}
    Suppose $F$ is a subfield of $\Q$. Prove $F = \Q$.
\end{mdframed}
\textbf{Solution}:\newline
 Since $F$ is a subfield, we know both the additive and multiplicative identities of $\Q$ are in $F$, i.e. $0 \in F$ and $1 \in F$.

    We claim that any $n \in \mathbb{N} \in F$, via induction on $n$.
    \begin{itemize}
        \item Clearly $1 \in F$.
        \item Since $n \in F$, $1 \in F$, and $F$ is closed under addition, therefore $n + 1 \in F$.
    \end{itemize}
    So all positive integers are in $F$. We also have $0 \in F$ by above result, and note $-n \in F$ for any $n \in \mathbb{N}$ since $F$ is closed under additive inverses. So all integers are in $F$.

    Note also that for all $x \in F^\ast$, we have $x^{-1} \in F^\ast$.

    So suppose $q \in \Q$, i.e. $q = \frac ab$ where $a, b \in \Z$ with $b \neq 0$. Then $b^{-1} \in F^\ast$ by above observation and so $q = \frac ab = ab^{-1} \in F$ since $F$ is closed under multiplication. We therefore see $\Q \subseteq F$.

    However, we also know $F \subseteq \Q$ since $F$ is a subfield of $\Q$. Therefore, as $\Q \subseteq F$ and $F \subseteq \Q$, we must have $F = \Q$.
\begin{mdframed}
    Prove \myref{thrm-homomorphism-from-field-is-injective-or-trivial}.
\end{mdframed}
\textbf{Solution}:\newline
 Note that $\ker\phi$ is an ideal of $F$. But $F$ has no proper ideals (\myref{prop-ring-is-field-iff-no-proper-ideals}). So $\ker\phi = F$ or $\ker\phi = \{0\}$.

    If $\ker\phi = F$ then this means $\phi(x) = 0$ for all $x\in R$, i.e. $\phi$ is trivial.

    Now suppose $\ker\phi = \{0\}$. Let $a, b \in F$ such that $\phi(a) = \phi(b)$. Then $\phi(a) - \phi(b) = 0$, which means $\phi(a - b) = 0$. Hence $a - b \in \ker\phi$ and so $a - b = 0$, i.e. $a = b$. Hence $\phi$ is injective.
\begin{mdframed}
    Let $F$ be a finite field with characteristic $p$, a prime.
    \begin{partquestions}{\roman*}
        \item Explain why $p$ divides $|F|$.
        \item Suppose $q$ is another prime dividing $|F|$. Show there exist integers $\lambda$ and $\mu$ such that $(\lambda p + \mu q)x = x$ for all elements $x \in F$.
        \item Deduce that the supposition in \textbf{(ii)} leads to a contradiction. Hence explain why $|F|$ must have order $p^n$, where $n$ is a positive integer.
    \end{partquestions}
\end{mdframed}
\textbf{Solution}:\newline
 \begin{partquestions}{\roman*}
        \item Note that as a field is a ring with identity, we have $|1|_+ = p$ by \myref{prop-characteristic-of-ring-with-identity} and since $\Char{F} = p$. As the order of an element divides the order of the group (\myref{corollary-order-of-group-multiple-of-order-of-element}), therefore $p$ divides the order of the additive group of $F$. As the order of the additive group of $F$ is precisely the order of $F$, thus $p$ divides $|F|$.

        \item Since $q$ is another prime dividing $|F|$, it is not $p$. Thus $p$ and $q$ are coprime, meaning that there are integers $\lambda$ and $\mu$ such that $\lambda p + \mu q = 1$ by B\'ezout's lemma (\myref{lemma-bezout}). Hence $(\lambda p + \mu q)x = x$ for any element $x \in F$.

        \item Since $q$ divides $|F|$, then there is an element $x$ in the additive group of $F$ with order $q$ by Cauchy's theorem (\myref{thrm-cauchy}). This means $qx = 0$. But we also have $px = 0$ since $\Char{F} = p$. Thus,
        \begin{align*}
            (\lambda p + \mu q)x &= \lambda (px) + \mu (qx) \\
            &= \lambda 0 + \mu 0\\
            &= 0\\
            &= x, & (\text{by \textbf{(ii)}})
        \end{align*}
        which shows $x = 0$. Note $|0|_+ = 1$ since 0 is the identity in the additive group. But $x$ has order $q$, a prime number, so $q \geq 2$, i.e. $|0|_+ \geq 2$, a contradiction.

        Therefore no other prime other than $p$ divides $|F|$, meaning that $|F| = p^n$ where $n$ is a \textit{non-negative} integer. But $n \neq 0$ since otherwise $F$ has order 1, which means that the multiplicative group has order 0, an impossibility. Thus $|F| = p^n$ where $n$ is a \textit{positive} integer.
    \end{partquestions}
\begin{mdframed}
    Consider the field $F = \Frac{\Z_p[x]}$.
    \begin{partquestions}{\alph*}
        \item Prove $F$ is infinite.\newline
        (\textit{Hint: it would be easier to prove $\Z_p[x]$ is infinite.})
        \item Prove $\Char{F} \neq 0$.
    \end{partquestions}
\end{mdframed}
\textbf{Solution}:\newline
 \begin{partquestions}{\alph*}
        \item We will just show that there are infinitely many equivalence classes of the form $\frac{f(x)}1$.

        Suppose, on the contrary, that there are only finitely many equivalence classes of the form $\frac{f(x)}1$; in particular, they are $\frac{f_1(x)}1, \frac{f_2(x)}1, \dots, \frac{f_n(x)}1$.

        Consider the equivalence class
        \[
            \frac{x^nf_1(x)f_2(x)\cdots f_n(x)}{1} = \left(\frac{xf_1(x)}{1}\right)\left(\frac{xf_2(x)}{1}\right)\cdots\left(\frac{xf_n(x)}{1}\right).
        \]
        Clearly none of the equivalence classes $\frac{f_1(x)}1, \frac{f_2(x)}1, \dots, \frac{f_n(x)}1$ are equal to this class. This contradicts the fact that we have listed all of such equivalence classes.

        Therefore there are infinitely many equivalence classes of of the form $\frac{f(x)}1$. As these are elements of $\Frac{\Z_p[x]}$, this means that $\Frac{\Z_p[x]}$ is infinite.

        \item Let $\frac{f(x)}{g(x)} \in \Frac{\Z_p[x]}$. Write $f(x) = a_0 + a_1x + a_2x^2 + \cdots + a_nx^n$ where $a_i \in \Z_p$. Note
        \begin{align*}
            p\left(\frac{f(x)}{g(x)}\right) &= \underbrace{\frac{f(x)}{g(x)} + \frac{f(x)}{g(x)} + \cdots + \frac{f(x)}{g(x)}}_{p \text{ times}}\\
            &= \frac{pf(x)}{g(x)}\\
            &= \frac{(pa_0) + (pa_1)x + (pa_2)x^2 + \cdots + (pa_n)x^n}{g(x)}\\
            &= \frac{0 + 0x + 0x^2 + \cdots + 0x^n}{g(x)} & (\text{as }\Char{\Z_p} = p)\\
            &= \frac0{g(x)}\\
            &= 0.
        \end{align*}
        Thus $p\left(\frac{f(x)}{g(x)}\right) = 0$ for any $\frac{f(x)}{g(x)} \in \Frac{\Z_p[x]}$, meaning that $\Char{\Z_p[x]} \neq 0$.
    \end{partquestions}

\chapter{Vector Spaces}
\section*{Exercises}
\begin{mdframed}
    Prove that $\R^n$ under addition as defined in \myref{example-R^n-is-vector-space} is an abelian group.
\end{mdframed}
\textbf{Solution}:\newline
 We first need to prove the four group axioms.
    \begin{itemize}
        \item \textbf{Closure}: We clearly see
        \[
            (a_1, a_2, \dots, a_n) + (b_1, b_2, \dots, b_n) = (a_1 + b_1, a_2 + b_2, \dots, a_n + b_n) \in \R^n
        \]
        so $\R^n$ is closed under vector addition.

        \item \textbf{Associativity}: Let $(a_1, a_2, \dots, a_n), (b_1, b_2, \dots, b_n), (c_1, c_2, \dots, c_n) \in \R^n$. Note
        \begin{align*}
            &(a_1, a_2, \dots, a_n) + \left((b_1, b_2, \dots, b_n) + (c_1, c_2, \dots, c_n)\right)\\
            &= (a_1, a_2, \dots, a_n) + (b_1 + c_1, b_2 + c_2, \dots, b_n + c_n)\\
            &= (a_1 + (b_1 + c_1), a_2 + (b_2 + c_2), \dots, a_n + (b_n + c_n))\\
            &= ((a_1 + b_1) + c_1, (a_2 + b_2) + c_2, \dots, (a_n + b_n) + c_n)\\
            &= (a_1 + b_1, a_2 + b_2, \dots, a_n + b_n) + (c_1, c_2, \dots, c_n)\\
            &= \left((a_1, a_2, \dots, a_n) + (b_1, b_2, \dots, b_n)\right) + (c_1, c_2, \dots, c_n)
        \end{align*}
        so vector addition is associative.

        \item \textbf{Identity}: One sees that $(0, 0, \dots, 0) \in \R^n$ is the identity.

        \item \textbf{Inverse}: One sees that $(-a_1, -a_2, \dots, -a_n) \in \R^n$ is the inverse of the element $(a_1, a_2, \dots, a_n)$.
    \end{itemize}

    Thus $(\R^n, +)$ is a group. Also, for any $(a_1, a_2, \dots, a_n), (b_1, b_2, \dots, b_n) \in \R^n$ we see
    \begin{align*}
        (a_1, a_2, \dots, a_n) + (b_1, b_2, \dots, b_n) &= (a_1 + b_1, a_2 + b_2, \dots, a_n + b_n)\\
        &= (b_1+a_1, b_2+a_2, \dots, b_n+a_n)\\
        &= (b_1, b_2, \dots, b_n) + (a_1, a_2, \dots, a_n)
    \end{align*}
    so vector addition is also commutative. Therefore $(\R^n, +)$ is an abelian group.
\begin{mdframed}
    Let
    \[
        W = \{f(x) \in \R[x] \vert \deg f(x) = 5\}.
    \]
    Is $W$ a vector space?
\end{mdframed}
\textbf{Solution}:\newline
 No. Note $x^5 + 1 \in W$ and $-x^5 + 1 \in W$ but $(x^5+1) + (-x^5+1) = 2 \notin W$ since 2 has a degree of 0.
\begin{mdframed}
    Prove \myref{prop-zero-vector-scaled-by-constant-is-zero-vector}.
\end{mdframed}
\textbf{Solution}:\newline
 Note
    \[
        \alpha\textbf{0} = \alpha(\textbf{0} + \textbf{0}) = \alpha\textbf{0} + \alpha\textbf{0}
    \]
    by \textbf{Distributivity-Addition}. Then, by adding $-(\alpha\textbf{0})$ on both sides we see
    \begin{align*}
        \alpha\textbf{0} + (-(\alpha\textbf{0})) &= (\alpha\textbf{0} + \alpha\textbf{0}) + (-(\alpha\textbf{0}))\\
        &= \alpha\textbf{0} + (\alpha\textbf{0} + (-(\alpha\textbf{0}))) & (\textbf{Addition-Abelian-Associativity})\\
    \end{align*}
    which means
    \[
        \textbf{0} = \alpha\textbf{0} + \textbf{0},
    \]
    by \textbf{Addition-Abelian-Inverses}, and hence $\alpha\textbf{0} = \textbf{0}$, by \textbf{Addition-Abelian-Identity}.
\begin{mdframed}
    Let $F$ be a field and $n$ be a non-negative integer. Prove that
    \[
        V = \{f(x) \in F[x] \vert \deg f(x) \leq n\} \cup \{0\}
    \]
    is a vector space over $F$ using the definitions of polynomial addition and scalar multiplication.\newline
    (\textit{Note: be careful with the zero polynomial!})
\end{mdframed}
\textbf{Solution}:\newline
 We know from \myref{example-polynomial-ring-over-field-is-vector-space} that $F[x]$ is a vector space, so we just need to prove that $V$ is a subspace of $F[x]$.
    \begin{itemize}
        \item Clearly $0 \in V$.
        \item Let $f(x), g(x) \in V$. We consider two cases.
        \begin{itemize}
            \item Suppose that $f(x)$ or $g(x)$ is the zero polynomial; without loss of generality assume $g(x) = 0$. Then $f(x) + g(x) = f(x) + 0 = f(x) \in V$.
            \item Otherwise, $\deg f(x), \deg g(x) \leq n$. By \myref{thrm-polynomial-degree-properties} we thus see that $\deg (f(x) + g(x)) \leq n$, meaning $f(x) + g(x) \in V$.
        \end{itemize}
        In either case, $f(x) + g(x) \in V$.
        \item Let $\alpha \in F$ and $f(x) \in V$. We again consider two cases.
        \begin{itemize}
            \item If $f(x) = 0$, then $\alpha f(x) = 0$ which is in $V$.
            \item Otherwise, we know $\deg f(x) \leq n$ and multiplication by a constant does not change the degree. Hence $\deg (\alpha f(x)) \leq n$ and so $\alpha f(x) \in V$.
        \end{itemize}
        In either case, $\alpha f(x) \in V$.
    \end{itemize}
    Therefore, by subspace test (\myref{thrm-subspace-test}), we see $V$ is a subspace of $F[x]$ over $F$, meaning that $V$ is a subspace over $F$.
\begin{mdframed}
    Let the vector space
    \[
        V = \{f(x) \in \Q[x] \vert \deg f(x) \leq 2\} \cup \{0\}.
    \]
    Let $S = \{x, x + 2, x^2 + 2, x^2 + 2x + 3\}$. Prove or disprove: $S$ is a spanning set for $V$.
\end{mdframed}
\textbf{Solution}:\newline
 We show that $S$ is a spanning set of $V$.

    Let $a + bx + cx^2 \in V$. Consider
    \[
        \alpha x + \beta(x + 2) + \gamma(x^2 + 2) + \delta(x^2 + 2x + 3) = a + bx + cx^2,
    \]
    where $\alpha, \beta, \gamma, \delta \in \Q$, i.e.
    \[
        (\gamma + \delta)x^2 + (\alpha + \beta + 2\delta)x + (2\beta + 2\gamma + 3\delta) = a + bx + cx^2.
    \]
    So we see
    \begin{align*}
        \gamma + \delta = a\\
        \alpha + \beta + 2\delta = b\\
        2\beta + 2\gamma + 3\delta = c
    \end{align*}
    which we can solve for $\alpha, \beta, \gamma, \delta$, yielding
    \begin{align*}
        \alpha &= a + b - \frac{c}{2} - \frac{3\delta}{2}\\
        \beta &= \frac{c}{2} - a - \frac{\delta}{2}\\
        \gamma &= a - \delta
    \end{align*}
    and $\delta$ is a free variable. Therefore any element of $V$ can be expressed as a linear combination of vectors in $S$, meaning $\Span{S} = V$.
\begin{mdframed}
    For the vector space
    \[
        V = \{f(x) \in \Q[x] \vert \deg f(x) \leq 2\} \cup \{0\}
    \]
    over $\Q$, which of the following sets of vectors of $V$ is/are linearly independent?
    \begin{partquestions}{\alph*}
        \item $S = \{1, x + 2\}$
        \item $T = \{1, x + 2, x^2 + 2x + 3\}$
        \item $U = \{1, x + 2, x^2 + 2x + 3, x^2 - 2x - 3\}$
    \end{partquestions}
\end{mdframed}
\textbf{Solution}:\newline
 \begin{partquestions}{\alph*}
        \item Linearly independent since the only solution to
        \[
            \alpha + \beta(x+2) = 0,
        \]
        which yields the system of equations
        \begin{align*}
            \alpha + 2\beta &= 0,\\
            \beta &= 0,
        \end{align*}
        is the trivial solution $\alpha = \beta = 0$.

        \item Linearly independent since the only solution to
        \[
            \alpha + \beta(x+2) + \gamma(x^2 + 2x + 3) = 0,
        \]
        which yields the system of equations
        \begin{align*}
            \alpha + 2\beta + 3\gamma &= 0,\\
            \beta + 2\gamma &= 0,\\
            \gamma = 0
        \end{align*}
        is the trivial solution $\alpha = \beta = \gamma = 0$.

        \item Not linearly independent (i.e., linearly dependent) since
        \[
            1 - 2(x+2) + \frac12(x^2 + 2x + 3) - \frac12(x^2 - 2x - 3) = 0.
        \]
    \end{partquestions}
\begin{mdframed}
    Prove that if $S$ is a basis for $V$, then every vector in $V$ can be expressed uniquely as a linear combination of the vectors in $S$.
\end{mdframed}
\textbf{Solution}:\newline
 TODO: Add
\begin{mdframed}
    For the vector space $V$ given in \myref{exercise-polynomial-of-degree-at-most-2-vector-space}, which of the following sets of vectors of $V$ is/are a basis for $V$?
    \begin{partquestions}{\alph*}
        \item $S = \{1, x + 2\}$
        \item $T = \{1, x + 2, x^2 + 2x + 3\}$
        \item $U = \{1, x + 2, x^2 + 2x + 3, x^2 - 2x - 3\}$
    \end{partquestions}
\end{mdframed}
\textbf{Solution}:\newline
 \begin{partquestions}{\alph*}
        \item Not a basis since $x^2$ cannot be expressed as a linear combination of the polynomials in $S$, i.e. $\Span{S} \neq V$.

        \item Is a basis. We found in \myref{exercise-polynomial-of-degree-at-most-2-vector-space} that the vectors in $T$ are linearly independent. Furthermore, any polynomial $ax^2 + bx + c \in V$ can be written as
        \[
            a(x^2+2x+3) + (-2a+b)(x+2) + (a-2b+c)
        \]
        which shows that $T$ spans $V$. Therefore $T$ is a basis for $V$.

        \item As $U$ is not linearly independent (from \myref{exercise-polynomial-of-degree-at-most-2-vector-space}), $U$ cannot be a basis for $V$.
    \end{partquestions}
\begin{mdframed}
    What is the dimension of the vector space $V$ in \myref{exercise-polynomial-of-degree-at-most-2-vector-space}?
\end{mdframed}
\textbf{Solution}:\newline
 We earlier found that $T = \{1, x + 2, x^2 + 2x + 3\}$ is a basis for $V$. Since $|T| = 3$ therefore $\dim{V} = 3$.

\section*{Problems}
\begin{mdframed}
    Let $U$ and $W$ be subspaces of a vector space $V$ over a field $F$.
    \begin{partquestions}{\alph*}
        \item Prove that $U \cap W$ is a subspace of $V$.
        \item Define the set
        \[
            U + W = \{\textbf{u} + \textbf{w} \vert \textbf{u} \in U \text{ and } \textbf{w} \in W\}.
        \]
        Prove that $U + W$ is a subspace of $V$.
    \end{partquestions}
\end{mdframed}
\textbf{Solution}:\newline
 \begin{partquestions}{\alph*}
        \item Since $\textbf{0} \in U$ (as $U$ is a subspace of $V$) and since $\textbf{0} \in W$ (as $W$ is a subspace of $V$), therefore $\textbf{0} \in U \cap W$, meaning $U \cap W$ is non-empty.

        Now suppose $\textbf{u}, \textbf{v} \in U \cap W$. So $\textbf{u}, \textbf{v} \in U$ and $\textbf{u}, \textbf{v} \in W$. Hence $\textbf{u} + \textbf{v} \in U$ and $\textbf{u} + \textbf{v} \in W$ by closure of vector spaces by addition. Therefore $\textbf{u} + \textbf{v} \in U \cap W$.

        Now let $\alpha \in F$ and $\textbf{u} \in U \cap W$. So $\textbf{u} \in U$ and $\textbf{u} \in W$. This means $\alpha\textbf{u} \in U$ and $\alpha\textbf{u} \in W$ by closure of vector spaces by scalar multiplication. Hence $\alpha\textbf{u} \in U \cap W$.

        So we see $U \cap W$ is a subspace of $V$ by the subspace test (\myref{thrm-subspace-test}).

        \item Note that since $\textbf{0} \in U$ and $\textbf{0} \in W$, we see $\textbf{0} = \textbf{0} + \textbf{0} \in U + W$, meaning $U + W$ is non-empty.

        Now suppose $\textbf{a}, \textbf{b} \in U + W$, meaning $\textbf{a} = \textbf{u}_1 + \textbf{w}_1$ and $\textbf{b} = \textbf{u}_2 + \textbf{w}_2$ for some $\textbf{u}_1, \textbf{u}_2 \in U$ and $\textbf{w}_1, \textbf{w}_2 \in W$. Hence one sees $\textbf{u}_1 + \textbf{u}_2 \in U$ and $\textbf{w}_1 + \textbf{w}_2 \in W$, meaning
        \begin{align*}
            \textbf{a} + \textbf{b} &= (\textbf{u}_1 + \textbf{w}_1) + (\textbf{u}_2 + \textbf{w}_2)\\
            &= (\textbf{u}_1 + \textbf{u}_2) + (\textbf{w}_1 + \textbf{w}_2)\\
            &\in U + W.
        \end{align*}
        Also, for any $\alpha \in F$ we see $\alpha\textbf{u}_1 \in U$ and $\alpha\textbf{w}_1 \in W$, which means
        \begin{align*}
            \alpha\textbf{a} &= \alpha(\textbf{u}_1 + \textbf{w}_1)\\
            &= (\alpha\textbf{u}_1) + (\alpha\textbf{w}_1)\\
            &\in U + W.
        \end{align*}
        Therefore $U + W$ is a subspace of $V$ by the subspace test (\myref{thrm-subspace-test}).
    \end{partquestions}
\begin{mdframed}
    Suppose $V$ is a vector space over the field $F$ with basis $\{\textbf{v}_1, \textbf{v}_2, \textbf{v}_3\}$. Let
    \begin{align*}
        \textbf{u}_1 &= \textbf{v}_1,\\
        \textbf{u}_2 &= \textbf{v}_1 + \textbf{v}_2, \text{ and}\\
        \textbf{u}_3 &= \textbf{v}_1 + \textbf{v}_2 + \textbf{v}_3.
    \end{align*}
    Prove that $\{\textbf{u}_1, \textbf{u}_2, \textbf{u}_3\}$ is also a basis for $V$.
\end{mdframed}
\textbf{Solution}:\newline
 Consider $S = \{\textbf{u}_1, \textbf{u}_2, \textbf{u}_3\} = \{\textbf{v}_1, \textbf{v}_1 + \textbf{v}_2, \textbf{v}_1 + \textbf{v}_2 + \textbf{v}_3\}$. Since $\{\textbf{v}_1,\textbf{v}_2,\textbf{v}_3\}$ is a basis for $V$, meaning that any $\textbf{a} \in V$ can be expressed as
    \[
        \alpha_1\textbf{v}_1 + \alpha_2\textbf{v}_2 + \alpha_3\textbf{v}_3
    \]
    where $\alpha_1, \alpha_2, \alpha_3 \in F$. One therefore sees
    \begin{align*}
        \textbf{a} &= \alpha_1\textbf{v}_1 + \alpha_2\textbf{v}_2 + \alpha_3\textbf{v}_3\\
        &= (\alpha_1 - \alpha_2)\textbf{v}_1 + (\alpha_2 - \alpha_3)(\textbf{v}_1 + \textbf{v}_2) + \alpha_3(\textbf{v}_1 + \textbf{v}_2 + \textbf{v}_3)\\
        &= (\alpha_1 - \alpha_2)\textbf{u}_1 + (\alpha_2 - \alpha_3)\textbf{u}_2 + \alpha_3\textbf{u}_3
    \end{align*}
    which means $\Span{S} = V$.

    Now we show that the vectors in $S$ are linearly independent. Suppose that
    \[
        \beta_1\textbf{u}_1 + \beta_2\textbf{u}_2 + \beta_3\textbf{u}_3 = \textbf{0}
    \]
    for some $\beta_1, \beta_2, \beta_3 \in F$. This means
    \begin{align*}
        \beta_1\textbf{u}_1 + \beta_2\textbf{u}_2 + \beta_3\textbf{u}_3 &= \beta_1\textbf{v}_1 + \beta_2(\textbf{v}_1 + \textbf{v}_2) + \beta_3(\textbf{v}_1 + \textbf{v}_2 + \textbf{v}_3)\\
        &= (\beta_1+\beta_2+\beta_3)\textbf{v}_1 + (\beta_2+\beta_3)\textbf{v}_2 + \beta_3\textbf{v}_3\\
        &= \textbf{0}.
    \end{align*}
    Since $\{\textbf{v}_1, \textbf{v}_2, \textbf{v}_3\}$ is a basis, the only way for this to occur is when
    \begin{align*}
        \beta_1 + \beta_2 + \beta_3 &= 0,\\
        \beta_2 + \beta_3 &= 0, \text{ and}\\
        \beta_3 & = 0,
    \end{align*}
    which clearly means $\beta_1 = \beta_2 = \beta_3 = 0$. Therefore the vectors in $S$ are linearly independent, which shows that $S$ is a basis for $V$.
\begin{mdframed}
    Let $V = \R^4$ be a vector space over $\R$.
    \begin{partquestions}{\alph*}
        \item Let the set
        \[
            R = \{(a, b, c, d) \in V \vert a + b + c + d = 1\}.
        \]
        Show that $R$ is \textit{not} a subspace of $V$.

        \item Let the set
        \[
            S = \{(a, b, c, d) \in V \vert a + b + c + d = 0\}.
        \]
        Show that $S$ is a subspace of $V$. Find a basis for $S$ and hence state the value of $\dim{S}$.
    \end{partquestions}
\end{mdframed}
\textbf{Solution}:\newline
 \begin{partquestions}{\alph*}
        \item Note $(1, 0, 0, 0), (0, 1, 0, 0) \in R$ (since $1 + 0 + 0 + 0 = 1$ and $0 + 1 + 0 + 0 = 1$) but
        \[
            (1, 0, 0, 0) + (0, 1, 0, 0) = (1, 1, 0, 0)
        \]
        is not since $1 + 1 + 0 + 0 = 2 \neq 1$. Therefore $R$ is not closed under vector addition, meaning that it is not a vector space (and hence not a subspace).

        \item We first show that $S$ is a subspace of $V$.

        Note $(0, 0, 0, 0) \in S$ since $0 + 0 + 0 + 0 = 0$, which means $S$ is non-empty.

        Now suppose $(a_1, a_2, a_3, a_4), (b_1, b_2, b_3, b_4) \in V$. So
        \begin{align*}
            a_1 + a_2 + a_3 + a_4 &= 0\\
            b_1 + b_2 + b_3 + b_4 &= 0
        \end{align*}
        which therefore means
        \begin{align*}
            &(a_1 + b_1) + (a_2 + b_2) + (a_3 + b_3) + (a_4 + b_4)\\
            &= (a_1 + a_2 + a_3 + a_4) + (b_1 + b_2 + b_3 + b_4)\\
            &= 0
        \end{align*}
        Hence
        \[
            (a_1, a_2, a_3, a_4) + (b_1, b_2, b_3, b_4) = (a_1 + b_1, a_2 + b_2, a_3 + b_3, a_4 + b_4) \in S.
        \]

        Finally suppose $\alpha \in \R$ and $(a_1, a_2, a_3, a_4) \in V$. One sees that
        \[
            \alpha a_1 + \alpha a_2 + \alpha a_3 + \alpha a_4 = \alpha(a_1 + a_2 + a_3 + a_4) = 0
        \]
        and so
        \[
            \alpha(a_1, a_2, a_3, a_4) = (\alpha a_1, \alpha a_2 , \alpha a_3, \alpha a_4) \in S.
        \]

        Therefore, by subspace test (\myref{thrm-subspace-test}), we see $S$ is a subspace of $V$.

        Now note that any $(a, b, c, d) \in S$ is expressible in the form
        \[
            (a, b, c, -a-b-c) = a(1, 0, 0, -1) + b(0, 1, 0, -1) + c(0, 0, 1, -1)
        \]
        and so the vectors in the set $A = \{(1, 0, 0, -1), (0, 1, 0, -1), (0, 0, 1, -1)\}$ span $V$. We also see that if
        \[
            \alpha(1, 0, 0, -1) + \beta(0, 1, 0, -1) + \gamma(0, 0, 1, -1) = (0, 0, 0, 0)
        \]
        we must have
        \begin{align*}
            \alpha &= 0\\
            \beta &= 0\\
            \gamma &= 0\\
            -\alpha - \beta - \gamma &= 0
        \end{align*}
        and so the only solution to the above equation is the trivial one, where $\alpha = \beta = \gamma = 0$. This shows that the vectors in $A$ are linearly independent.

        Therefore, $A$ is a basis for $S$. Since $|A| = 3$, therefore $S$ has dimension 3, i.e. $\dim{S} = 3$.
    \end{partquestions}
\begin{mdframed}
    Let $F$ be a field and $n$ be a positive integer. Define addition and scalar multiplication by
    \[
        (u_1, u_2, \dots, u_n) + (v_1, v_2, \dots, v_n) = (u_1 + v_1, u_2 + v_2, \dots, u_n + v_n)
    \]
    and
    \[
        \alpha(u_1, u_2, \dots, u_n) = (\alpha u_1, \alpha u_2, \dots, \alpha u_n).
    \]
    respectively, for any $(u_1, u_2, \dots, u_n), (v_1, v_2, \dots, v_n) \in F^n$ and $\alpha \in F$. Prove that $F^n$ is a vector space over $F$ under the above definitions of addition and scalar multiplication.
\end{mdframed}
\textbf{Solution}:\newline
 We are to prove the vector space axioms for $F^n$ over $F$.
    \begin{itemize}
        \item \textbf{Addition-Abelian}: We are to show that $(F^n, +)$ is an abelian group.
        \begin{itemize}
            \item \textbf{Closure}: From the definition of vector addition it is clear that closure is satisfied.

            \item \textbf{Associativity}: For any $(u_1, u_2, \dots, u_n), (v_1, v_2, \dots, v_n), (w_1, w_2, \dots, w_n) \in F^n$ we see
            \begin{align*}
                &(u_1, u_2, \dots, u_n) + ((v_1, v_2, \dots, v_n) + (w_1, w_2, \dots, w_n))\\
                &= (u_1, u_2, \dots, u_n) + (v_1 + w_1, v_2 + w_2, \dots, v_n + w_n)\\
                &= (u_1 + (v_1 + w_1), u_2 + (v_2 + w_2), \dots, u_n + (v_n + w_n))\\
                &= ((u_1 + v_1) + w_1, (u_2 + v_2) + w_2, \dots, (u_n + v_n) + w_n)\\
                &= (u_1 + v_1, u_2 + v_2, \dots, u_n + v_n) + (w_1, w_2, \dots, w_n)\\
                &= ((u_1, u_2, \dots, u_n) + (v_1, v_2, \dots, v_n)) + (w_1, w_2, \dots, w_n)
            \end{align*}
            which proves that vector addition is associative.

            \item \textbf{Identity}: $(0, 0, \dots, 0) \in F^n$ is the identity since, for any $(u_1, u_2, \dots, u_n) \in F^n$, we see
            \begin{align*}
                (0, 0, \dots, 0) + (u_1, u_2, \dots, u_n) &= (0 + u_1, 0 + u_2, \dots, 0 + u_n)\\
                &= (u_1, u_2, \dots, u_n).
            \end{align*}

            \item \textbf{Inverse}: For any $(u_1, u_2, \dots, u_n) \in F^n$, we note that $(-u_1, -u_2, \dots, -u_n)$ is its additive inverse since
            \begin{align*}
                (u_1, u_2, \dots, u_n) + (-u_1, -u_2, \dots, -u_n) &= (u_1 - u_1, u_2 - u_2, \dots, u_n - u_n)\\
                &= (0, 0, \dots, 0).
            \end{align*}

            \item \textbf{Commutativity}: For any $(u_1, u_2, \dots, u_n), (v_1, v_2, \dots, v_n) \in F^n$ we note
            \begin{align*}
                (u_1, u_2, \dots, u_n) + (v_1, v_2, \dots, v_n) &= (u_1 + v_1, u_2 + v_2, \dots, u_n + v_n)\\
                &= (v_1 + u_1, v_2 + u_2, \dots, v_n + u_n)\\
                &= (v_1, v_2, \dots, v_n) + (u_1, u_2, \dots, u_n)
            \end{align*}
            so addition is commutative.
        \end{itemize}
        Therefore $(F^n, +)$ is an abelian group.

        \item \textbf{Multiplication-Identity}: Note for any $(u_1, u_2, \dots, u_n) \in F^n$ we have
        \begin{align*}
            1(u_1, u_2, \dots, u_n) &= (1u_1, 1u_2, \dots, 1u_n)\\
            &= (u_1, u_2, \dots, u_n)
        \end{align*}
        so this axiom is satisfied.

        \item \textbf{Multiplication-Compatibility}: Let $\alpha, \beta \in F$ and $(u_1, u_2, \dots, u_n) \in F^n$. Then
        \begin{align*}
            \alpha(\beta(u_1, u_2, \dots, u_n)) &= \alpha(\beta u_1, \beta u_2, \dots, \beta u_n)\\
            &= (\alpha\beta u_1, \alpha\beta u_2, \dots, \alpha\beta u_n)\\
            &= (\alpha\beta)(u_1, u_2, \dots, u_n).
        \end{align*}
        Thus this axiom is satisfied.

        \item \textbf{Distributivity-Addition}: Let $\alpha \in F$ and $(u_1, u_2, \dots, u_n), (v_1, v_2, \dots, v_n) \in F^n$. Then
        \begin{align*}
            &\alpha((u_1, u_2, \dots, u_n) + (v_1, v_2, \dots, v_n))\\
            &= \alpha(u_1 + v_1, u_2 + v_2, \dots, u_n + v_n)\\
            &= (\alpha(u_1 + v_1), \alpha(u_2 + v_2), \dots, \alpha(u_n + v_n))\\
            &= (\alpha u_1 + \alpha v_1, \alpha u_2 + \alpha v_2, \dots, \alpha u_n + \alpha v_n)\\
            &= (\alpha u_1, \alpha u_2, \dots, \alpha u_n) + (\alpha v_1, \alpha v_2, \dots, \alpha v_n)\\
            &= \alpha(u_1, u_2, \dots, u_n) + \alpha(v_1, v_2, \dots, v_n)
        \end{align*}
        which shows that the axiom is satisfied.

        \item \textbf{Distributivity-Scalar}: Let $\alpha, \beta \in F$ and $(u_1, u_2, \dots, u_n) \in F^n$. Note
        \begin{align*}
            (\alpha+\beta)(u_1, u_2, \dots, u_n) &= ((\alpha+\beta)u_1, (\alpha+\beta)u_2, \dots, (\alpha+\beta)u_n)\\
            &= (\alpha u_1 + \beta u_1, \alpha u_2 + \beta u_2, \dots, \alpha u_n + \beta u_n)\\
            &= (\alpha u_1, \alpha u_2, \dots, \alpha u_n) + (\beta u_1, \beta u_2, \dots, \beta u_n)\\
            &= \alpha(u_1, u_2, \dots, u_n) + \beta(u_1, u_2, \dots, u_n)
        \end{align*}
        so this axiom is satisfied.
    \end{itemize}
    Since all the vector space axioms are satisfied, this finally means that $F^n$ is a vector space over $F$.
\begin{mdframed}
    The vector space analogue of a group/ring homomorphism is called a \textbf{linear transformation}\index{linear transformation}. If $V$ and $W$ are vector spaces over a field $F$, then the map $T: V \to W$ is a linear transformation if it preserves vector addition and scalar multiplication, i.e.
    \begin{align*}
        T(\textbf{u} + \textbf{v}) &= T(\textbf{u}) + T(\textbf{v}) \text{ and}\\
        T(\alpha\textbf{u}) &= \alpha T(\textbf{u})
    \end{align*}
    for all $\textbf{u}, \textbf{v} \in V$ and $\alpha \in F$.

    To prove that two vector spaces are isomorphic, one needs to find a bijective linear transformation.

    With this information, prove \myref{thrm-vector-space-of-dimension-n-isomorphic-to-F^n}.
\end{mdframed}
\textbf{Solution}:\newline
 Suppose $V$ has a basis of $\{\textbf{u}_1, \textbf{u}_2, \dots, \textbf{u}_n\}$. This means any $\textbf{a} \in V$ can be expressed as
    \[
        \alpha_1\textbf{u}_1 + \alpha_2\textbf{u}_2 + \cdots + \alpha_n\textbf{u}_n
    \]
    where $\alpha_1,\alpha_2,\dots,\alpha_n \in F$.

    Consider the map $T: V \to F^n$ given by
    \[
        T( \alpha_1\textbf{u}_1 + \alpha_2\textbf{u}_2 + \cdots + \alpha_n\textbf{u}_n) = (\alpha_1, \alpha_2, \dots, \alpha_n).
    \]
    We first show that $T$ is a linear transformation.
    \begin{itemize}
        \item For all $\alpha_1\textbf{u}_1 + \alpha_2\textbf{u}_2 + \cdots + \alpha_n\textbf{u}_n, \beta_1\textbf{u}_1 + \beta_2\textbf{u}_2 + \cdots + \beta_n\textbf{u}_n \in V$ we see
        \begin{align*}
            &T((\alpha_1\textbf{u}_1 + \alpha_2\textbf{u}_2 + \cdots + \alpha_n\textbf{u}_n) + (\beta_1\textbf{u}_1 + \beta_2\textbf{u}_2 + \cdots + \beta_n\textbf{u}_n))\\
            &= T((\alpha_1 + \beta_1)\textbf{u}_1 + (\alpha_2 + \beta_2)\textbf{u}_2 + \cdots + (\alpha_n + \beta_n)\textbf{u}_n)\\
            &= (\alpha_1 + \beta_1, \alpha_2 + \beta_2, \dots, \alpha_n + \beta_n)\\
            &= (\alpha_1, \alpha_2, \dots, \alpha_n) + (\beta_1, \beta_2, \dots, \beta_n)\\
            &= T(\alpha_1\textbf{u}_1 + \alpha_2\textbf{u}_2 + \cdots + \alpha_n\textbf{u}_n) + T(\beta_1\textbf{u}_1 + \beta_2\textbf{u}_2 + \cdots + \beta_n\textbf{u}_n).
        \end{align*}

        \item For all $\lambda \in F$ and $\alpha_1\textbf{u}_1 + \alpha_2\textbf{u}_2 + \cdots + \alpha_n\textbf{u}_n \in V$ we see
        \begin{align*}
            T(\lambda(\alpha_1\textbf{u}_1 + \alpha_2\textbf{u}_2 + \cdots + \alpha_n\textbf{u}_n)) &= T((\lambda\alpha_1)\textbf{u}_1 + (\lambda\alpha_2)\textbf{u}_2 + \cdots + (\lambda\alpha_n)\textbf{u}_n)\\
            &= (\lambda\alpha_1, \lambda\alpha_2, \dots, \lambda\alpha_n)\\
            &= \lambda(\alpha_1, \alpha_2, \dots, \alpha_n)\\
            &= \lambda T(\alpha_1\textbf{u}_1 + \alpha_2\textbf{u}_2 + \cdots + \alpha_n\textbf{u}_n).
        \end{align*}
    \end{itemize}
    Therefore $T$ is a linear transformation.

    We now prove that $T$ is a bijection.
    \begin{itemize}
        \item \textbf{Injective}: Suppose $\alpha_1\textbf{u}_1 + \alpha_2\textbf{u}_2 + \cdots + \alpha_n\textbf{u}_n, \beta_1\textbf{u}_1 + \beta_2\textbf{u}_2 + \cdots + \beta_n\textbf{u}_n \in V$ are such that $T(\alpha_1\textbf{u}_1 + \alpha_2\textbf{u}_2 + \cdots + \alpha_n\textbf{u}_n) = T(\beta_1\textbf{u}_1 + \beta_2\textbf{u}_2 + \cdots + \beta_n\textbf{u}_n)$. This therefore means $(\alpha_1, \alpha_2, \dots, \alpha_n) = (\beta_1, \beta_2, \dots, \beta_n)$, which shows that $\alpha_i = \beta_i$ for all $i \in \{1, 2, \dots, n\}$. It follows quickly from this result that $\alpha_1\textbf{u}_1 + \alpha_2\textbf{u}_2 + \cdots + \alpha_n\textbf{u}_n = \beta_1\textbf{u}_1 + \beta_2\textbf{u}_2 + \cdots + \beta_n\textbf{u}_n$, proving that $T$ is injective.

        \item \textbf{Surjective}: Suppose $(r_1, r_2, \dots, r_n) \in F^n$. Then we see $r_1\textbf{u}_1 + r_2\textbf{u}_2 + \cdots + r_n\textbf{u}_n \in V$, and
        \[
            T(r_1\textbf{u}_1 + r_2\textbf{u}_2 + \cdots + r_n\textbf{u}_n) = (r_1, r_2, \dots, r_n).
        \]
        Therefore any $(r_1, r_2, \dots, r_n) \in F^n$ has a pre-image in $V$, meaning $T$ is surjective.
    \end{itemize}
    Hence $T$ is a bijective linear transformation, which means $V$ and $F^n$ are isomorphic as vector spaces, as required.

\chapter{Field Extensions}
\section*{Exercises}
\begin{mdframed}
    Verify that the field $F$ in \myref{example-R[x]-mod-x^2+1-is-isomorphic-to-C} is indeed isomorphic to $\C$ by using the map $\phi$ given. For completeness, also verify that $\phi$ is well-defined.
\end{mdframed}
\textbf{Solution}:\newline
 We need to prove that $\phi$ is a well-defined ring isomorphism.
    \begin{itemize}
        \item \textbf{Well-defined}: Suppose $ax + b + \princ{x^2+1}, cx + d + \princ{x^2+1} \in F$ such that $ax + b + \princ{x^2+1} = cx + d + \princ{x^2+1}$. Then $(a - c)x + (b - d) + \princ{x^2 + 1} = 0 + \princ{x^2+1}$ by Coset Equality, which therefore means that $(a-c)x + (b-d) \in \princ{x^2+1}$. Now elements of $\princ{x^2+1}$ have degree of at least 2, or is the zero polynomial. Since $(a-c)x + (b-d)$ has degree less than 2, we must have $(a-c)x + (b-d) = 0$ which means $a - c = 0$ and $b - d = 0$. Hence $a = c$ and $b = d$, and so
        \begin{align*}
            \phi\left(ax + b + \princ{x^2+1}\right) &= b + ai\\
            &= d + ci\\
            &= \phi\left(cx + d + \princ{x^2+1}\right)
        \end{align*}
        which proves that $\phi$ is well-defined.

        \item \textbf{Homomorphism}. Let $ax + b + \princ{x^2+1}, cx + d + \princ{x^2+1} \in F$. Then
        \begin{align*}
            &\phi\left(\left(ax + b + \princ{x^2+1}\right) + \left(cx + d + \princ{x^2+1}\right)\right)\\
            &= \phi\left((a+c)x + (b+d) + \princ{x^2+1}\right)\\
            &= (b+d) + (a+c)i\\
            &= (b+ai) + (d+ci)\\
            &= \phi\left(ax + b + \princ{x^2+1}\right) + \phi\left(cx + d + \princ{x^2+1}\right)
        \end{align*}
        and
        \begin{align*}
            &\phi\left(\left(ax + b + \princ{x^2+1}\right)\left(cx + d + \princ{x^2+1}\right)\right)\\
            &= \phi\left((acx^2 + (ad + bc)x + bd) + \princ{x^2+1}\right)\\
            &= \phi\left(((ad + bc)x + (bd - ac)) + \princ{x^2+1}\right) & (\text{since } x^2 = -1 \text{ in } F)\\
            &= (bd - ac) + (ad + bc)i\\
            &= (b + ai)(d + ci)\\
            &= \phi\left(ax + b + \princ{x^2+1}\right)\phi\left(cx + d + \princ{x^2+1}\right)
        \end{align*}
        which shows that $\phi$ is a ring homomorphism.

        \item \textbf{Injective}: Suppose $ax + b + \princ{x^2+1}, cx + d + \princ{x^2+1} \in F$ such that
        \[
            \phi\left(ax + b + \princ{x^2+1}\right) = \phi\left(cx + d + \princ{x^2+1}\right).
        \]
        This means that $b + ai = d + ci$. Hence $b = d$ and $a = c$, which shows that $ax + b + \princ{x^2+1} = cx + d + \princ{x^2+1}$. Hence $\phi$ is injective.

        \item \textbf{Surjective}: Suppose $u + vi \in \C$. Then clearly
        \[
            \phi\left(vx + u + \princ{x^2+1}\right) = u + vi
        \]
        which means that $\phi$ is surjective.
    \end{itemize}
    Therefore $\phi$ is a well-defined ring isomorphism, proving that $F \cong \C$.
\begin{mdframed}
    Let $f(x) = 2x + 1$ be a polynomial in $\Z_4[x]$. Prove that no ring that contains a subring that is isomorphic to $\Z_4$ has a zero of $f(x)$.
\end{mdframed}
\textbf{Solution}:\newline
 By way of contradiction, suppose a ring $R$ has an subring that is isomorphic to $\Z_4$, and that $f(x)$ has a zero in such a ring. Let this zero be denoted $\alpha$, which means $2\alpha + 1 = 0$ in $R$. Note that this means
    \begin{align*}
        0 &= 2\times0\\
        &= 2(2\alpha + 1)\\
        &= 2(2\alpha) + 2\\
        &= (2 \times 2)\alpha + 2\\
        &= 0\alpha + 2\\
        &= 2
    \end{align*}
    but $0 \neq 2$ in $\Z_4$.
\begin{mdframed}
    Find two extension fields that contains a zero of the polynomial $x^4 + 3x^2 + 1 \in \Q[x]$.
\end{mdframed}
\textbf{Solution}:\newline
 We note that $x^4 + 3x^2 + 1 = (x^2+1)(x^2+2)$, and that $x^2+1$ and $x^2 + 2$ are irreducible over $\Q$. By the Fundamental Theorem of Field Theory (\myref{thrm-fundamental-theorem-of-field-theory}) we may choose
    \[
        \Q[x]/\princ{x^2+1} \quad\text{or}\quad \Q[x]/\princ{x^2+2}
    \]
    as the possible extension fields.
\begin{mdframed}
    Prove \myref{prop-field-generated-by-S-inductive-definition}.
\end{mdframed}
\textbf{Solution}:\newline
 For brevity let $K = F(a_1, a_2, \dots, a_{n-1})(a_n)$. Note $K$ contains $F(a_1, a_2, \dots, a_{n-1})$ and the element $a_n$, and that $F(a_1, a_2, \dots, a_{n-1})$ contains the field $F$ and the elements $a_1, a_2, \dots, a_{n-1}$. Thus $K$ contains $F$ and the elements $a_1, a_2, \dots, a_{n-1}, a_n$, which means that $K$ is included in the intersection that generates $F(a_1, a_2, \dots, a_n)$, so $K \subseteq F(a_1, a_2, \dots, a_n)$. But $F(a_1, a_2, \dots, a_n)$ is the smallest subfield of $E$ that contains $F$ and the elements $a_1, a_2, \dots, a_n$. Therefore $K = F(a_1, a_2, \dots, a_n)$ as required.
\begin{mdframed}
    Let $f(x) = x^4 - 5x^2 + 6 \in \Q[x]$.
    \begin{partquestions}{\roman*}
        \item Show that a splitting field of $f(x)$ over $\Q$ is $\Q(\sqrt2)(\sqrt3)$.
        \item Show that $p(x) = x^4 - 10x + 1$ is an irreducible polynomial over $\Q$ with a zero of $\sqrt2 + \sqrt3$.
        \item Hence show that $\Q(\sqrt2 + \sqrt3)$ is a splitting field of $f(x)$ over $\Q$.
    \end{partquestions}
\end{mdframed}
\textbf{Solution}:\newline
 \begin{partquestions}{\roman*}
        \item One sees that
        \[
            f(x) = (x-\sqrt2)(x+\sqrt2)(x-\sqrt3)(x+\sqrt3)
        \]
        and so a splitting field of $f(x)$ over $\Q$ is
        \begin{align*}
            &\Q(\sqrt2, -\sqrt2, \sqrt3, -\sqrt3)\\
            &=\Q(\sqrt2, \sqrt3)\\
            &=\Q(\sqrt2)(\sqrt3).
        \end{align*}

        \item Reducing $p(x)$ modulo 5 yields $\bar{p}(x) = x^4 + 1$. One sees then that
        \begin{itemize}
            \item $\bar{p}(0) = 1 \neq 0$;
            \item $\bar{p}(1) = 2 \neq 0$;
            \item $\bar{p}(2) = 17 = 2 \neq 0$;
            \item $\bar{p}(3) = 82 = 2 \neq 0$; and
            \item $\bar{p}(4) = 257 = 2 \neq 0$.
        \end{itemize}
        Therefore $p(x)$ is irreducible over $\Q$ by Mod 5 Irreducibility Test (\myref{thrm-mod-p-irreducibility-test}). One also sees that
        \begin{align*}
            &p(\sqrt2 + \sqrt3)\\
            &= (\sqrt2 + \sqrt3)^4 - 10(\sqrt2 + \sqrt3)^2 + 1\\
            &= \left((\sqrt2)^4 + 4(\sqrt2)^3(\sqrt3) + 6(\sqrt2)^2(\sqrt3)^2 + 4(\sqrt2)(\sqrt3)^3 + (\sqrt3)^4\right)\\
            &\quad\quad- 10\left((\sqrt2)^2 + 2\sqrt2\sqrt3 + (\sqrt3)^2\right) + 1\\
            &=\left(49 + 20\sqrt6\right) - 10\left(5 + 2\sqrt6\right) + 1\\
            &= 0
        \end{align*}
        so $\sqrt2 + \sqrt3$ is indeed a zero of $p(x)$ in $\R$.

        \item We work backwards. Since $p(x) = x^4 - 10x + 1$ is an irreducible polynomial of degree 4, therefore $\Q(\sqrt2 + \sqrt3)$ has 4 basis vectors. Therefore
        \begin{align*}
            &\Q(\sqrt2 + \sqrt3)\\
            &= \{s + t(\sqrt2 + \sqrt3) + u(\sqrt2 + \sqrt3)^2 + v(\sqrt2 + \sqrt3)^3 \vert s,t,u,v \in \Q\}\\
            &= \{(s + 5u) + (t + 11v)\sqrt2 + (t + 9v)\sqrt3 + 2u\sqrt6 \vert s, t, u, v \in \Q\}\\
            &= \{a + b\sqrt2 + c\sqrt3 + d\sqrt6 \vert a, b, c, d \in \Q\}\\
            &=\{(a+b\sqrt2) + (c+d\sqrt2)\sqrt3 \vert a, b, c, d \in \Q\}\\
            &=\{u + v\sqrt3 \vert u, v \in \Q(\sqrt2)\}\\
            &= \Q(\sqrt2)(\sqrt3)
        \end{align*}
        and so $\Q(\sqrt2 + \sqrt3)$ is a splitting field of $f(x)$ over $\Q$.
    \end{partquestions}
\begin{mdframed}
    Prove for any field $F$ and any two polynomials $f(x), g(x) \in F[x]$ that $(f(x)+g(x))' = f'(x) + g'(x)$.
\end{mdframed}
\textbf{Solution}:\newline
 Let $f(x) = a_0 + a_1x + \cdots + a_mx^m$ and $g(x) = b_0 + b_1x + \cdots + b_nx^n$. Without loss of generality assume $m \geq n$, and set $b_i = 0$ for all $i > m$. Thus
    \begin{align*}
        (f(x) + g(x))' &= \left(\sum_{i=0}^m(a_i + b_i)x^i\right)'\\
        &= \sum_{i=1}^m i(a_i+b_i)x^{i-1}\\
        &= \sum_{i=1}^m ia_ix^{i-1} + \sum_{i=1}^m ib_ix^{i-1}\\
        &= f'(x) + g'(x)
    \end{align*}
    as required.
\begin{mdframed}
    Prove for any field $F$ and any polynomial $f(x) \in F[x]$ that
    \[
        \left((f(x))^n\right)' = n(f(x))^{n-1}f'(x),
    \]
    where $n$ is a positive integer.
\end{mdframed}
\textbf{Solution}:\newline
 We induct on $n$.

    When $n = 1$ clearly $((f(x))^1)' = (f(x))' = f'(x) = 1(f(x))^{1-1}f'(x)$.

    Assume $((f(x))^k)' = k(f(x))^{k-1}f'(x)$ for some positive integer $k$. We show that the $k+1$ case also works.

    Note
    \begin{align*}
        ((f(x))^{k+1})' &= ((f(x))(f(x))^k)'\\
        &= f(x)((f(x))^k)' + f'(x)(f(x))^k\\
        &= f(x)(k(f(x))^{k-1}f'(x)) + f'(x)(f(x))^k & (\text{Induction Hypothesis})\\
        &= k(f(x))^kf'(x) + (f(x))^kf'(x)\\
        &= (k+1)(f(x))^kf'(x)
    \end{align*}
    so the $k+1$ case also holds.

    Therefore by mathematical induction we have $((f(x))^n)' = n(f(x))^{n-1}f'(x)$ for all positive integers $n$.
\begin{mdframed}
    Let $F$ be a field and let $f(x) = a_0 + a_1x + \cdots + a_nx^n$. Prove that $f'(x) = 0$ implies $ra_r = 0$ for all $r \in \{1, 2, \dots, n\}$.
\end{mdframed}
\textbf{Solution}:\newline
 If $f'(x) = 0$, we see
    \[
        a_1 + 2a_2x + 3a_3x^2 + \cdots + na_nx^{n-1} = 0.
    \]
    This means that all coefficients of $f'(x)$ must be zero; hence $ra_r = 0$ for all $r \in \{1, 2, \dots, n\}$.
\begin{mdframed}
    Let $F$ be a field of prime characteristic $p$. Prove that $(x + y)^{p^n} = x^{p^n} + y^{p^n}$ for all $x,y \in F$ and for all positive integers $n$.
\end{mdframed}
\textbf{Solution}:\newline
 We prove this by induction on $n$. The base case of $n = 1$ was already proven. Assume that the statement holds for some positive integer $k$, i.e. $(x+y)^{p^k} = x^{p^k} + y^{p^k}$. Note that
    \begin{align*}
        (x+y)^{p^{k+1}} &= \left((x+y)^{p^k}\right)^p\\
        &= (x^{p^k}+y^{p^k})^p & (\text{By Induction Hypothesis})\\
        &= \left(x^{p^k}\right)^p + \left(y^{p^k}\right)^p & (\text{By Freshman's Dream})\\
        &= x^{p^{k+1}} + y^{p^{k+1}}
    \end{align*}
    so the statement holds for $k + 1$. Thus by mathematical induction we have proven that $(x+y)^{p^n} = x^{p^n} + y^{p^n}$.

\section*{Problems}
\begin{mdframed}
    Let $F$ be a field, and let $p, q \in F$ where $p \neq 0$. Suppose $E/F$ is a field extension and $a \in E$. Prove that $F(pa + q) = F(a)$.
\end{mdframed}
\textbf{Solution}:\newline
 Note that since $p, q \in F$, thus $pa + q \in F(a)$. Hence $F(pa+q) \subseteq F(a)$. On the other hand, note that because $p \neq 0$ thus $p^{-1}$ exists, so
    \[
        a = p^{-1}(pa+q) - p^{-1}q
    \]
    and thus $a \in F(pa+q)$. Hence $F(a) \subseteq F(pa+q)$ also. Therefore $F(pa+q) = F(a)$.
\begin{mdframed}
    Find the splitting fields of the following polynomial. If the splitting field found is a field extension, express it as a simple extension if possible.
    \begin{partquestions}{\alph*}
        \item $x^3 - 3x + 2$ over $\Q$.
        \item $x^3 - 1$ over $\Q$.
        \item $x^4 + x^2 + 1$ over $\Q$.\newline
        (\textit{Hint: $x^4+x^2+1 = (x^2-x+1)(x^2+x+1)$.})
        \item $x^2 - 2\sqrt{2}x + 3$ over $\Q(\sqrt2)$.
    \end{partquestions}
\end{mdframed}
\textbf{Solution}:\newline
 \begin{partquestions}{\alph*}
        \item One sees that $x^3-3x+2 = (x-1)^2(x+2)$, so $x^3-3x+2$ splits in $\Q$. Hence $\Q$ is the splitting field of $x^3-3x+2$ over $\Q$.

        \item We see $x^3 - 1 = (x-1)(x^2+x+1)$. Note $x^2 + x + 1$ has zeroes of $\frac{1\pm\sqrt{-3}}{2} = \frac12 \pm \frac12\sqrt{-3}$ by the quadratic formula. Thus the splitting field of $x^3 - 1$ over $\Q$ is
        \begin{align*}
            &\Q\left(1, \frac12 + \frac12\sqrt{-3}, \frac12 - \frac12\sqrt{-3}\right)\\
            &=\Q(1)\left(\frac12 + \frac12\sqrt{-3}\right)\left(\frac12 - \frac12\sqrt{-3}\right)\\
            &=\Q(\sqrt{-3})(\sqrt{-3}) & (\myref{problem-simple-extension-absorbs-field-elements})\\
            &=\Q(\sqrt{-3}).
        \end{align*}

        \item Given $x^4 + x^2 + 1 = (x^2 - x + 1)(x^2 + x + 1)$. The zeroes of $x^2 + x + 1$ are $\frac12 \pm \frac12\sqrt{-3}$; the zeroes of $x^2 - x + 1$ are $-\frac12 \pm \frac12\sqrt{-3}$. Therefore the splitting field of $x^4 + x^2 + 1$ over $\Q$ is
        \begin{align*}
            &\Q\left(\frac12 + \frac12\sqrt{-3}, \frac12 - \frac12\sqrt{-3}, -\frac12 + \frac12\sqrt{-3}, -\frac12 - \frac12\sqrt{-3}\right)\\
            &=\Q\left(\frac12 + \frac12\sqrt{-3}\right)\left(\frac12 - \frac12\sqrt{-3}\right)\left(-\frac12 + \frac12\sqrt{-3}\right)\left(-\frac12 - \frac12\sqrt{-3}\right)\\
            &=\Q(\sqrt{-3})(\sqrt{-3})(\sqrt{-3})(\sqrt{-3})\\
            &= \Q(\sqrt{-3}).
        \end{align*}

        \item The zeroes of $x^2 + 2\sqrt2x + 3$ are $\sqrt2 \pm i$ by quadratic formula. Hence the splitting field of $x^2 + 2\sqrt2x + 3$ over $\Q(\sqrt2)$ is
        \begin{align*}
            \Q(\sqrt2)(\sqrt2+i, \sqrt2-i) &= \Q(\sqrt2)(\sqrt2 + i)(\sqrt2 - i)\\
            &= Q(\sqrt2, i).
        \end{align*}
    \end{partquestions}
\begin{mdframed}
    Determine which of the following polynomials, if any, have multiple zeroes over the field $F$ specified. If the polynomial has a multiple zero, state it.
    \begin{partquestions}{\alph*}
        \item $x^4 + 2x + 7$ over $F = \Z_2$.
        \item $x^{19} + x^8 + 1$ over $F = \Z_3$.
        \item $2x^6 + x^4 + 2x^3 + 2$ over $F = \Z_3$.
        \item $x^8 + 3x^5 + x^3 + 5$ over $F = \Z_7$.
    \end{partquestions}
\end{mdframed}
\textbf{Solution}:\newline
 \begin{partquestions}{\alph*}
        \item Let $f(x) = x^4 + x + 7$. Note that $f(0) = 7 = 1 \neq 0$ and $f(1) = 1 + 1 + 7 = 9 = 1 \neq 0$, so $f(x)$ does not have a zero in $\Z_2$. Hence $f(x)$ does not have a multiple zero.

        \item Let $f(x) = x^{19} + x^8 + 1$, so $f'(x) = 19x^{18} + 8x^7$. In $\Z_3[x]$, $f'(x) = x^{18} + 2x^7$. Note that $f(1) = 1 + 1 + 1 = 3 = 0$ and $f'(1) = 1 + 2 = 3 = 0$, so both $f(x)$ and $f'(x)$ share a common factor of positive degree of $x-1$ by Factor Theorem (\myref{corollary-factor-theorem}). Hence 1 is a multiple zero by \myref{thrm-criterion-for-multiple-zeroes}.

        \item Let $f(x) = 2x^6 + x^4 + 2x^3 + 2$, so $f'(x) = 12x^5 + 4x^3 + 6x^2$. In $\Z_3[x]$, $f'(x) = x^3$. Note that $f(0) = 2 \neq 0$, $f(1) = 2 + 1 + 2 + 2 = 7 = 1 \neq 0$, but $f(2) = 2(2)^6 + 2^4 + 2(2)^3 + 2 = 162 = 0$, so the only zero of $f(x)$ is 2. But $f'(2) = 2^3 = 8 = 2 \neq 0$. Hence $f(x)$ and $f'(x)$ do not share a common factor of positive degree, meaning that $f(x)$ has no multiple zeroes (\myref{thrm-criterion-for-multiple-zeroes}).

        \item Let $f(x) = x^8 + 3x^5 + x^3 + 5$, so $f'(x) = 8x^7 + 15x^4 + 3x^2$. In $\Z_7[x]$, $f'(x) = x^7 + x^4 + 3x^2$. Note $f(4) = 68677 = 0$ and $f'(4) = 16688 = 0$, so $f(x)$ and $f'(x)$ share a common factor of positive degree of $x-4$ by Factor Theorem (\myref{corollary-factor-theorem}). Hence 4 is a multiple zero by \myref{thrm-criterion-for-multiple-zeroes}.
    \end{partquestions}
\begin{mdframed}
    Let $p(x) \in \Z_2$ be a polynomial of degree 7 that is irreducible in $\Z_2$. Prove or disprove the following statements.
    \begin{partquestions}{\roman*}
        \item $\Z_2[x]/\princ{p(x)}$ is a field.
        \item $\left(\Z_2[x]/\princ{p(x)}\right)^\ast$ is a group.
        \item Every non-identity element in $\left(\Z_2[x]/\princ{p(x)}\right)^\ast$ is a generator.
    \end{partquestions}
\end{mdframed}
\textbf{Solution}:\newline
 \begin{partquestions}{\roman*}
        \item Prove. Since $p(x)$ is irreducible, thus $\princ{p(x)}$ is maximal (\myref{thrm-irreducible-iff-principal-ideal-is-maximal}). Therefore $\Z_2[x]/\princ{p(x)}$ is a field by \myref{thrm-maximal-ideal-iff-quotient-ring-is-field}.

        \item Prove. Since $\Z_2[x]/\princ{p(x)}$ is a field, thus $\left(\Z_2[x]/\princ{p(x)}\right)^\ast$ is in fact the multiplicative group of the field.

        \item Prove. Note that $\left|\Z_2[x]/\princ{p(x)}\right| = 2^7 = 128$, so $\left|\left(\Z_2[x]/\princ{p(x)}\right)^\ast\right| = 2^7 - 1 = 127$, since the multiplicative group omits the additive identity. One sees that 127 is prime. Hence every non-identity element of $\left(\Z_2[x]/\princ{p(x)}\right)^\ast$ is a generator (\myref{exercise-prime-order-element}).
    \end{partquestions}
\begin{mdframed}
    Let $F$ be a field, $a \in F$, and $f(x) \in F[x]$. Show that $f(x)$ and $f(x+a)$ have the same splitting field over $F$.
\end{mdframed}
\textbf{Solution}:\newline
 Suppose $\alpha_1, \alpha_2, \dots, \alpha_n$ are the zeroes of $f(x)$. Then $\alpha_1 - a, \alpha_2 - a, \dots, \alpha_n - a$ are the zeroes of $f(x+a)$. Thus the splitting field of $f(x+a)$ over $F$ is
    \begin{align*}
        F(\alpha_1 - a, \alpha_2 - a, \dots, \alpha_n - a) &= F(\alpha_1 - a)(\alpha_2 - a)\dots(\alpha_n - a)\\
        &= F(\alpha_1)(\alpha_2)\dots(\alpha_n) & (\myref{problem-simple-extension-absorbs-field-elements})\\
        &= F(\alpha_1, \alpha_2, \dots, \alpha_n)
    \end{align*}
    which is the splitting field of $f(x)$ over $F$.
\begin{mdframed}
    Find all subfields of $\Q(\sqrt2)$.
\end{mdframed}
\textbf{Solution}:\newline
 By definition of a simple extension, $\Q(\sqrt2)$ contains $\Q$. But $\Q$ is a prime field (\myref{thrm-Q-is-prime-field}), so it contains no smaller subfields. Therefore the only subfields of $\Q(\sqrt2)$ are $\Q$ and $\Q(\sqrt2)$.
\begin{mdframed}
    Let $F$ be a field of prime characteristic $p$. Prove that $f(x) = x^{p^n} - x \in F[x]$, where $n$ is a positive integer, only has simple zeroes.
\end{mdframed}
\textbf{Solution}:\newline
 Let $f(x) = x^{p^n} - x$. Thus $f'(x) = p^nx^{p^n-1}-1$. Note that since $\Char{F} = p$, therefore $p^nx^{p^n-1} = 0$. Hence $f'(x) = -1$; clearly $f'(x)$ shares no common factor of positive degree with $f(x)$. Hence, there are no multiple zeroes of $f(x)$ (\myref{thrm-criterion-for-multiple-zeroes}), meaning that any zero of $f(x)$ must be simple.
\begin{mdframed}
    Let $F$ be a field of prime characteristic $p$, and let $a \in F$. Prove that $f(x) = x^p - a \in F[x]$ is either irreducible or splits over $F$.\newline
    (\textit{Hint: consider two cases -- either $f(x)$ has a zero in $F$ or it does not.})
\end{mdframed}
\textbf{Solution}:\newline
 We consider two cases.

    If $f(x)$ has a zero in $F$, say $\alpha$, then that must mean $\alpha^p - a = 0$ by definition of a zero. Hence $a = \alpha^p$, meaning
    \[
        f(x) = x^p - \alpha^p = (x-\alpha)^p
    \]
    by the Freshman's Dream (\myref{prop-freshman-dream}). This clearly means that $f(x)$ splits over $F$.

    Otherwise $f(x)$ has no zeroes in $F$. Let $E/F$ be the splitting field of $f(x)$, and suppose $\beta \in E$ is a zero of $f(x)$. Using the above working we know that $f(x) = (x-\beta)^p$.

    Now, seeking a contradiction, suppose $f(x)$ is reducible over $F$, meaning that there exist non-constant polynomials $g(x), h(x) \in F[x]$ such that $f(x) = g(x)h(x)$. In particular, $g(x) = (x-\beta)^n$ for some positive integer $n$ where $1 \leq n \leq p$. Expanding $g(x)$ using the Binomial Theorem yields
    \[
        g(x) = x^n + n\beta x^{n-1} + {n\choose2}\beta^2x^{n-2} + \cdots + n\beta^{n-1}x + \beta^n \in F[x],
    \]
    and in particular this means that $\beta^n \in F$. Also, since $f(x) = (x-\beta)^p$, we draw a similar conclusion that $\beta^p \in F$.

    Now since $1 \leq n \leq p$ thus $n$ and $p$ are coprime, i.e. $\gcd(n,p) = 1$. By B\'ezout's lemma (\myref{lemma-bezout}) we are able to find integers $s$ and $t$ such that $sn + tp = 1$. In particular, by the closure of fields, we know $\beta^{sn} = \left(\beta^n\right)^s \in F$ and $\beta^{tp} = \left(\beta^p\right)^t \in F$, so
    \[
        \beta = \beta^1 = \beta^{sn+tp} = \beta^{sn}\beta^{tp} \in F.
    \]
    Thus $f(x)$ has a zero, namely $\beta$, in $F$. But this contradicts the fact that $f(x)$ has no zeroes in $F$. Therefore, $f(x)$ is irreducible over $F$.

\chapter{Algebraic Extensions}
\section*{Exercises}
\begin{mdframed}
    Is $\sqrt{2 - \sqrt{i}}$ a transcendental number?
\end{mdframed}
\textbf{Solution}:\newline
 It is not transcendental, i.e. it is algebraic. Let $\alpha = \sqrt{2-\sqrt{i}}$; one sees $\alpha^2 = 2 - \sqrt{i}$. Thus $\alpha^2 - 2 = -\sqrt{i}$ and so $(\alpha^2-2)^2 = i$. Note that $(\alpha^2 - 2)^2 = \alpha^4 - 4\alpha^2 + 4$ and so $\alpha^4 - 4\alpha^2 + 4 = i$. Squaring one more time we see
    \[
        x^8 - 8x^6 + 24x^4 - 32x^2 + 16 = i^2 = -1
    \]
    and so $\alpha$ is a zero of the polynomial $x^8 - 8x^6 + 24x^4 - 32x^2 + 17 \in \Q$. Therefore $\alpha$ is algebraic.
\begin{mdframed}
    Find $[\Q(\sqrt2,\sqrt[3]3):\Q]$ and hence find $[\Q(\sqrt2, \sqrt[3]3, \sqrt[6]{72}):\Q]$.
\end{mdframed}
\textbf{Solution}:\newline
 We know by Tower Law (\myref{thrm-tower-law}) that
    \begin{align*}
        [\Q(\sqrt2, \sqrt[3]3): \Q] &= [\Q(\sqrt2, \sqrt[3]3): \Q(\sqrt2)][\Q(\sqrt2):\Q]\\
        &= [\Q(\sqrt2, \sqrt[3]3): \Q(\sqrt[3]3)][\Q(\sqrt[3]3):\Q].
    \end{align*}
    Since $[\Q(\sqrt2):\Q] = 2$ and $[\Q(\sqrt[3]3):\Q] = 3$, thus $[\Q(\sqrt2, \sqrt[3]3): \Q]$ is a multiple of $\lcm(2, 3) = 6$. On the other hand, note that $x^3 - 3 \in \Q(\sqrt2)[x]$ has a zero of $\sqrt[3]3$, so $[\Q(\sqrt2, \sqrt[3]3): \Q(\sqrt2)]$ is at most 3. Hence $[\Q(\sqrt2, \sqrt[3]3): \Q] = [\Q(\sqrt2, \sqrt[3]3): \Q(\sqrt2)][\Q(\sqrt2):\Q] \leq 3 \times 2 = 6$. Hence $[\Q(\sqrt2, \sqrt[3]3): \Q] = 6$.

    Now, note that $\sqrt[6]{72} = \sqrt2 \times \sqrt[3]3$, so $\sqrt[6]{72} \in \Q(\sqrt2, \sqrt[3]3)$. Consequently, we see that $\Q(\sqrt2, \sqrt[3]3, \sqrt[6]{72}) = \Q(\sqrt2, \sqrt[3]3)$ and so $[\Q(\sqrt2, \sqrt[3]3, \sqrt[6]{72}):\Q] = [\Q(\sqrt2, \sqrt[3]3): \Q] = 6$.
\begin{mdframed}
    Suppose $\alpha$ is an element of some extension of $\Q$ with minimal polynomial $p(x) = x^5 + 2x + 1 \in \Q[x]$. Prove that $\sqrt2 \notin \Q(\alpha)$.
\end{mdframed}
\textbf{Solution}:\newline
 Suppose otherwise, that $\sqrt2 \in \Q(\alpha)$. So $\Q \subset \Q(\sqrt2) \subseteq \Q(\alpha)$. Thus we see
    \[
        [\Q(\alpha): \Q] = [\Q(\alpha):\Q(sqrt2)][\Q(\sqrt2):\Q].
    \]
    But $[\Q(\alpha): \Q] = 5$ (since $\alpha$ is a zero of the irreducible polynomial $p(x)$) and $[\Q(\sqrt2):\Q] = 2$, so we see that 2 divides 5, a contradiction. Therefore $\sqrt2 \notin \Q(\alpha)$.
\begin{mdframed}
    Consider $\Q(\sqrt3,\sqrt5)$.
    \begin{partquestions}{\roman*}
        \item Prove that $\Q(\sqrt3,\sqrt5) = \Q(\sqrt3+\sqrt5)$.
        \item Deduce the degree of the minimal polynomial of $\sqrt3+\sqrt5$.
    \end{partquestions}
\end{mdframed}
\textbf{Solution}:\newline
 \begin{partquestions}{\roman*}
        \item One clearly sees $\Q(\sqrt3 + \sqrt5) \subseteq \Q(\sqrt3, \sqrt5)$ as $\sqrt3 + \sqrt5 \in \Q(\sqrt3, \sqrt5)$. Now observe
        \begin{align*}
            \left(\sqrt3+\sqrt5\right)^{-1} &= \frac{\sqrt3-\sqrt5}{\left(\sqrt3+\sqrt5\right)\left(\sqrt3-\sqrt5\right)}\\
            &= -\frac12\left(\sqrt3-\sqrt5\right)\\
        \end{align*}
        so $\sqrt3-\sqrt5 \in \Q(\sqrt3 + \sqrt5)$. It follows then $\frac12\left(\sqrt3 + \sqrt5\right) + \frac12\left(\sqrt3 - \sqrt5\right) = \sqrt3$ and $\frac12\left(\sqrt3 + \sqrt5\right) - \frac12\left(\sqrt3 - \sqrt5\right) = \sqrt5$ and so $\sqrt3,\sqrt5\in\Q(\sqrt3+\sqrt5)$. Hence $\Q(\sqrt3,\sqrt5)\subseteq\Q(\sqrt3+\sqrt5)$, and therefore $\Q(\sqrt3, \sqrt5) = \Q(\sqrt3 + \sqrt5)$.

        \item We see
        \[
            [\Q(\sqrt3, \sqrt5): \Q] = \underbrace{[\Q(\sqrt3, \sqrt5): \Q(\sqrt3)]}_2\underbrace{[\Q(\sqrt3):\Q]}_2 = 4
        \]
        by Tower Law (\myref{thrm-tower-law}), which means that the minimal polynomial of $\sqrt3+\sqrt5$ has degree 4.
    \end{partquestions}
\begin{mdframed}
    Prove \myref{corollary-field-is-algebraically-closed-iff-no-proper-algebraic-extension}.
\end{mdframed}
\textbf{Solution}:\newline
 For the forward direction, suppose $F$ is an algebraically closed field and let $E$ be an algebraic extension of $F$, which means $F \subseteq E$. Now let $\alpha \in E$. Since $F$ is algebraically closed, by \myref{corollary-field-is-algebraically-closed-iff-irreducible-polynomials-are-of-degree-1}, every irreducible polynomial is of degree 1. Consequently the minimal polynomial of $\alpha$ in $F$ is $x - \alpha$, which clearly means $\alpha \in F$. So $E \subseteq F$, meaning $F = E$.

    For the reverse direction, suppose a field $F$ has no proper algebraic extension. Suppose $f(x) \in F[x]$ is an irreducible polynomial. Then \myref{corollary-polynomial-quotient-by-principal-ideal-is-field-iff-polynomial-irreducible} tells us that $E = F[x]/\princ{f(x)}$ is an extension field of $F$ where $[E:F] = \deg f(x)$, which is finite and thus algebraic (\myref{thrm-finite-extension-is-algebraic}). However since $F$ has no proper algebraic extension, thus $E = F$, meaning $[E:F] = 1$, and thus $\deg f(x) = 1$. The proof is complete by \myref{corollary-field-is-algebraically-closed-iff-irreducible-polynomials-are-of-degree-1}.

\section*{Problems}
\begin{mdframed}
    Prove that $E = \Q(2^{\frac12}, 2^{\frac13}, 2^{\frac14}, \dots, 2^{\frac1n}, \dots)$ is an algebraic extension of $\Q$ but not a finite extension of $\Q$.
\end{mdframed}
\textbf{Solution}:\newline
 Note that each of the $2^{\frac1n}$ is algebraic since they are a zero of the polynomial $x^n - 2$ over $\Q$. Therefore $E$ is an algebraic extension.

    However $E$ is not finite, since by Tower Law (\myref{thrm-tower-law}) we have
    \begin{align*}
        [E: F] &= \cdots[\Q(2^{\frac12},2^{\frac13},2^{\frac14}):\Q(2^{\frac12},2^{\frac13})]
        [\Q(2^{\frac12},2^{\frac13}):\Q(2^{\frac12})][\Q(2^{\frac12}):\Q]\\
        &= \cdots \times 4 \times 3 \times 2
    \end{align*}
    which is clearly not finite.
\begin{mdframed}
    Find $[\Q(\sqrt3+\sqrt5):\Q(\sqrt{15})]$, and hence find
    \begin{partquestions}{\alph*}
        \item the minimal polynomial of $\sqrt3 + \sqrt5$ over $\Q(\sqrt{15})$; and
        \item a basis for $\Q(\sqrt3+\sqrt5)$ over $\Q(\sqrt{15})$.
    \end{partquestions}
    (\textit{Hint: consider \myref{example-Q-sqrt3-sqrt5}.})
\end{mdframed}
\textbf{Solution}:\newline
 Earlier in \myref{exercise-Q-sqrt3-sqrt5} we have shown $\Q(\sqrt3 + \sqrt5) = \Q(\sqrt3,\sqrt5)$. In \myref{example-Q-sqrt3-sqrt5} we have also shown that $[\Q(\sqrt3,\sqrt5):\Q] = 4$. Furthermore we see $[\Q(\sqrt{15}):\Q] = 2$ and
    \[
        [\Q(\sqrt3,\sqrt5):\Q] = [\Q(\sqrt3,\sqrt5):\Q(\sqrt{15})][\Q(\sqrt{15}):\Q]
    \]
    by Tower Law (\myref{thrm-tower-law}). Therefore
    \[
        4 = [\Q(\sqrt3,\sqrt5):\Q(\sqrt{15})]\times2
    \]
    and so $[\Q(\sqrt3,\sqrt5):\Q(\sqrt{15})] = 2$.

    \begin{partquestions}{\alph*}
        \item Let $\alpha = \sqrt3 + \sqrt5$. Then $\alpha^2 = 8 + 2\sqrt{15}$, so we see that the minimal polynomial of $\sqrt3+\sqrt5$ over $\Q(\sqrt{15})$ is $x^2 - (8 + 2\sqrt{15})$.

        \item A basis for $\Q(\sqrt3 + \sqrt5)$ over $\Q(\sqrt{15})$ is $\{1, \sqrt3 + \sqrt5\}$, since we have shown that $(\sqrt3 + \sqrt5)^2 \in \Q(\sqrt{15})$ in \textbf{(a)}.
    \end{partquestions}
\begin{mdframed}
    Prove that $\Q(\sqrt2,\sqrt[3]2) = \Q(\sqrt[6]2)$.
\end{mdframed}
\textbf{Solution}:\newline
 We see $\sqrt[6]2 = (\sqrt2)(\sqrt[3]2)^{-1} \in \Q(\sqrt2, \sqrt[3]2)$ so $\Q(\sqrt[6]2) \subseteq \Q(\sqrt2,\sqrt[3]2)$. But also $\sqrt[3]2 = \left(\sqrt[6]2\right)^2$ and $\sqrt2 = \left(\sqrt[6]2\right)^3$, so $\sqrt2,\sqrt[3]2 \in \Q(\sqrt[6]2)$ and thus $\Q(\sqrt2,\sqrt[3]2) \subseteq \Q(\sqrt[6]2)$. Therefore $\Q(\sqrt2,\sqrt[3]2) = \Q(\sqrt[6]2)$.
\begin{mdframed}
    Let $a, b \in \Q$. Show that $\Q(\sqrt a, \sqrt b) = \Q(\sqrt a + \sqrt b)$.
\end{mdframed}
\textbf{Solution}:\newline
 Note clearly $\Q(\sqrt a + \sqrt b) \subseteq \Q(\sqrt a, \sqrt b)$ since $\sqrt a + \sqrt b \in \Q(\sqrt a, \sqrt b)$.

    Note also
    \[
        (\sqrt a + \sqrt b)^{-1} = \frac{\sqrt a - \sqrt b}{a^2 - b^2} = \frac1{a^2-b^2}\left(\sqrt a - \sqrt b\right)
    \]
    so $\sqrt a - \sqrt b \in \Q(\sqrt a + \sqrt b)$. Consequently we see
    \begin{align*}
        \sqrt a &= \frac12(\sqrt a + \sqrt b) + \frac12(\sqrt a - \sqrt b) \in \Q(\sqrt a + \sqrt b)\\
        \sqrt b &= \frac12(\sqrt a + \sqrt b) - \frac12(\sqrt a - \sqrt b) \in \Q(\sqrt a + \sqrt b)
    \end{align*}
    and therefore $\Q(\sqrt a, \sqrt b) \subseteq \Q(\sqrt a + \sqrt b)$.

    Hence because $\Q(\sqrt a + \sqrt b) \subseteq \Q(\sqrt a, \sqrt b)$ and $\Q(\sqrt a, \sqrt b) \subseteq \Q(\sqrt a + \sqrt b)$ we have $\Q(\sqrt a, \sqrt b) = \Q(\sqrt a + \sqrt b)$ as required.
\begin{mdframed}
    Suppose $F$ is a field and $E$ is a finite extension where $[E:F] = p$ where $p$ is a prime. Prove that for every $\alpha \in E$ we have $F(\alpha) = F$ or $F(\alpha) = E$.
\end{mdframed}
\textbf{Solution}:\newline
 Note that $F \subseteq F(\alpha) \subseteq E$, so by Tower Law (\myref{thrm-tower-law}) we have
    \[
        p = [E:F] = [E:F(\alpha)][F(\alpha):F].
    \]
    Since $p$ is a prime we must have either $[E:F(\alpha)] = 1$ or $[F(\alpha):F] = 1$, meaning $E \cong F(\alpha)$ or $F(\alpha) \cong F$ by \myref{prop-finite-extension-of-degree-1-means-extension-equals-base-field}. But as $F \subseteq F(\alpha) \subseteq E$ we see that $F(\alpha) = E$ or $F(\alpha) = F$.
\begin{mdframed}
    Let $F$ be a field and $E/F$ be a field extension. Prove that if $\alpha \in E$ is transcendental over $F$, then $\alpha^n$ is transcendental over $F$ for all positive integers $n$.
\end{mdframed}
\textbf{Solution}:\newline
 Seeking a contradiction, suppose $\alpha^n$ is algebraic. Thus $\alpha^n$ is a zero of some polynomial $f(x) \in F[x]$, i.e. $f(\alpha^n) = 0$. Consequently $\alpha$ is a zero of the polynomial $g(x) \in F[x]$ where $g(x) = f(x^n)$, which implies that $\alpha$ is algebraic over $F$, a contradiction. Therefore $\alpha^n$ is transcendental for all positive integers $n$.
\begin{mdframed}
    Let $F$ be a field and $E$ be an extension field of $F$.
    \begin{partquestions}{\roman*}
        \item Let $a_0, a_1, \dots, a_n \in E$ be algebraic over $F$. Prove that the zero(es) of the polynomial
        \[
            f(x) = a_0 + a_1x + a_2x^2 + \cdots + a_nx^n
        \]
        are also algebraic over $F$.

        \item Deduce that if $\alpha$ and $\beta$ are both transcendental over $F$, then either $\alpha+\beta$ or $\alpha\beta$ is also transcendental.
    \end{partquestions}
\end{mdframed}
\textbf{Solution}:\newline
 \begin{partquestions}{\roman*}
        \item Let $\alpha \in E$ be a zero of $f(x)$. We know that $F(a_0, a_1, \dots, a_n)$ is a finite extension over $F$. Furthermore we know that $\alpha$ is algebraic over $F(a_0, a_1, \dots, a_n)$ since $\alpha$ is a zero of $f(x) \in F(a_0, a_1, \dots, a_n)[x]$. Therefore we see
        \[
            [F(a_0, \dots, a_n, \alpha):F] = \underbrace{[F(a_0, \dots, a_n, \alpha):F(a_0, \dots, a_n)]}_{\leq n}\underbrace{[F(a_0, \dots, a_n):F]}_{\text{Finite}}
        \]
        by Tower Law (\myref{thrm-tower-law}), and so $[F(a_0, a_1, \dots, a_n, \alpha):F]$ is an algebraic extension over $F$ (\myref{thrm-finite-extension-is-algebraic}). Hence $\alpha$ is algebraic over $F$.

        \item Suppose not, that both $\alpha + \beta$ and $\alpha\beta$ are algebraic. Then note that
        \[
            x^2 - (\alpha+\beta)x + \alpha\beta
        \]
        is a polynomial of algebraic coefficients over $F$ with zeroes of $\alpha$ and $\beta$, which means that $\alpha$ and $\beta$ are algebraic over $F$, a contradiction. Therefore at least one of $\alpha + \beta$ and $\alpha\beta$ is transcendental over $F$.
    \end{partquestions}
\begin{mdframed}
    Suppose $\alpha \in \C$ is a zero of $x^2 + x + 1$ over $\Q$. Prove that $\Q(\alpha) = \Q(\sqrt{\alpha})$.
\end{mdframed}
\textbf{Solution}:\newline
 Clearly $\Q(\alpha) \subseteq \Q(\sqrt\alpha)$ since $\alpha = (\sqrt\alpha)^2 \in \Q(\sqrt\alpha)$.

    For brevity let $\omega = \sqrt\alpha$. Note that $\omega$ is a zero of the polynomial $x^4+x^2+1$. But we may factor that polynomial since
    \begin{align*}
        x^4 + x^2 + 1 &= (x^4 + 2x^2 + 1) - x^2\\
        &= (x^2+1)^2 - x^2\\
        &= (x^2+x+1)(x^2-x+1)
    \end{align*}
    and so we see that
    \[
        (\omega^2+\omega+1)(\omega^2-\omega+1) = 0.
    \]
    THerefore we have $\omega^2+\omega+1 = 0$ or $\omega^2-\omega+1 = 0$, which one may rearrange appropriately to yield $\omega = -\omega^2-1$ or $\omega = \omega^2 + 1$, i.e. $\sqrt\alpha = \pm(\alpha+1)$. Therefore we see $\sqrt\alpha \in \Q(\alpha)$, meaning $\Q(\sqrt\alpha) \subseteq \Q(\alpha)$.

    Since $\Q(\alpha) \subseteq \Q(\sqrt\alpha)$ and $\Q(\sqrt\alpha) \subseteq \Q(\alpha)$ thus $\Q(\alpha) = \Q(\sqrt\alpha)$.
\begin{mdframed}
    Prove that $\sqrt2 \notin \Q(\pi)$. You may assume that $\pi$ is a transcendental number and that $\sqrt2$ is irrational.\newline
    (\textit{Hint: refer to the proof of \myref{thrm-characterisation-of-extensions} on the form of an element in $\Q(\pi)$.})
\end{mdframed}
\textbf{Solution}:\newline
 From \myref{thrm-characterisation-of-extensions} we know that an element of $\Q(\pi)$ takes the form
    \[
        \frac{a_0 + a_1\pi + a_2\pi^2 + \cdots + a_m\pi^m}{b_0 + b_1\pi + b_2\pi^2 + \cdots + b_n\pi^n}
    \]
    where each $a_i, b_j \in \Q$.

    By way of contradiction suppose $\sqrt2 \in \Q(\pi)$, meaning
    \[
        \sqrt2 = \frac{a_0 + a_1\pi + a_2\pi^2 + \cdots + a_m\pi^m}{b_0 + b_1\pi + b_2\pi^2 + \cdots + b_n\pi^n}
    \]
    for some $a_i, b_j \in \Q$. Rearranging we see
    \[
        \left(a_0 + a_1\pi + a_2\pi^2 + \cdots + a_m\pi^m\right)^2 = 2\left(b_0 + b_1\pi + b_2\pi^2 + \cdots + b_n\pi^n\right)^2.
    \]
    Equating the leading term on both sides we see $a_m^2\pi^{2m} = 2b_n^2\pi^{2n}$. Thus $m = n$ and $a_m^2 = 2b_n^2$ which means $\sqrt2 = \frac{a_m}{b_n}$, a rational number, which is a contradiction. Therefore $\sqrt2 \notin \Q(\pi)$.
\begin{mdframed}
    Let $F$ be a field and $f(x)$ is a non-constant polynomial in $F[x]$ with degree $n$. Let $E$ be the splitting field of $f(x)$ over $F$.
    \begin{partquestions}{\roman*}
        \item By considering the binomial coefficient ${a+b\choose a}$, explain why $a!b!$ divides $(a+b)!$ for all non-negative integers $a$ and $b$.
        \item Prove that $[E:F]$ divides $n!$.\newline
        (\textit{Hint: induct on $n$; break into two cases, where $f(x)$ is irreducible or $f(x)$ is reducible.})
    \end{partquestions}
\end{mdframed}
\textbf{Solution}:\newline
 \begin{partquestions}{\roman*}
        \item Note that ${a+b\choose a}$ is a positive integer. Also we have ${a+b\choose a} = \frac{(a+b)!}{a!b!}$ by definition of the binomial coefficient. Therefore we see that $a!b!$ divides $(a+b)!$ for all non-negative integers $a$ and $b$.

        \item We use strong induction on $n$.

        When $n = 1$, then the splitting field of $f(x)$ over $F$ is just $F$, so $[E:F] = [F:F] = 1$ which clearly divides $1! = 1$.

        Assume that the claim holds for all polynomials $g(x) \in F[x]$ of degree of at most $k$ over all fields $F'$, for some positive integer $k$. We show that the claim holds for a polynomial of degree $k+1$ over $F$.

        Let $f(x)$ be such a polynomial. We split into two cases.
        \begin{itemize}
            \item The first case is if $f(x)$ is irreducible over $F$. Let $E$ be the splitting field of $f(x)$ over $F$. Let $\alpha \in E$ be a zero of $f(x)$. Then by Factor Theorem (\myref{corollary-factor-theorem}) we may write $f(x) = (x-\alpha)g(x)$ where $g(x) \in F(\alpha)[x]$ has degree $k$. Note $E$ is a splitting field of $g(x)$ over $F(\alpha)$, so by Induction Hypothesis we know that $[E:F(\alpha)]$ divides $k!$. As $[F(\alpha):F] = k+1$ (since $f(x)$ is an irreducible polynomial of degree $k+1$ with a zero of $\alpha$), it follows from Tower Law (\myref{thrm-tower-law}) that $[E:F]$ divides $(k+1)k! = (k+1)!$.

            \item The second case is if $f(x)$ is reducible over $F$. Write $f(x) = g(x)h(x)$ where $g(x), h(x) \in F[x]$. Let $a = \deg g(x)$ and $b = \deg h(x)$. Let $K \subseteq E$ be the splitting field of $g(x)$ over $F$. Then $[K:F]$ divides $a!$ by Induction Hypothesis. Also we must have $E$ being the splitting field of $h(x) \in K[x]$, meaning $[E:K]$ divides $b!$. By Tower Law we see that $[E:F] = [E:K][K:F] = a!b!$ which divides $(a+b)! = (k+1)!$.
        \end{itemize}
        Therefore the statement holds for any polynomial of degree $k+1$ over $F$, proving this claim by induction.
    \end{partquestions}

\chapter{Finite Fields}
\section*{Exercises}
\begin{mdframed}
    In \myref{thrm-finite-field-is-unique}, prove that $S$, the set of zeroes of $f(x)$ in $F$, is a subfield of $F$.
\end{mdframed}
\textbf{Solution}:\newline
 Note for any $\alpha \in S$, because $f(\alpha) = \alpha^{p^n} - \alpha = 0$, thus $\alpha^{p^n} = \alpha$. This fact will be useful later.
    \begin{itemize}
        \item Clearly $f(1) = 1^{p^n} - 1 = 0$ so $1 \in S^\ast$.

        \item Now let $\alpha,\beta \in S$. Then we see
        \begin{align*}
            f(\alpha - \beta) &= (\alpha-\beta)^{p^n} - (\alpha - \beta)\\
            &= \alpha^{p^n} + (-1)^{p^n}\beta^{p^n} - \alpha + \beta & (\myref{prop-freshman-dream})\\
            &= (\alpha^{p^n} - \alpha) + ((-1)^{p^n}\beta^{p^n} + \beta)\\
            &= 0 + (-1)^{p^n}\beta^{p^n} + \beta & (\text{as } \alpha \in S)\\
            &= (-1)^{p^n}\beta^{p^n} + \beta.
        \end{align*}
        We further split into two cases.
        \begin{itemize}
            \item If $p$ is odd then $(-1)^{p^n}\beta^{p^n} + \beta = -(\beta^{p^n} - \beta) = 0$.
            \item If $p$ is even then $p = 2$. So $(-1)^{p^n}\beta^{p^n} + \beta = \beta^{2^n} + \beta = 2\beta = 0$ since $\Char{F} = 2$.
        \end{itemize}
        In either case, we see that $f(\alpha-\beta) = 0$, meaning $\alpha - \beta \in S$.

        \item Finally, for $\alpha \in S$ and $\beta \in S^\ast$ we see
        \begin{align*}
            f(\alpha\beta^{-1}) &= (\alpha\beta^{-1})^{p^n} - (\alpha\beta^{-1})\\
            &= \alpha^{p^n}\left(\beta^{-1}\right)^{p^n} - \alpha\beta^{-1}\\
            &= \alpha^{p^n}\left(\beta^{p^n}\right)^{-1} - \alpha\beta^{-1}\\
            &= \alpha\beta^{-1} - \alpha\beta^{-1}\\
            &= 0
        \end{align*}
        and so $\alpha\beta^{-1} \in S$.
    \end{itemize}
    Therefore by the subfield test (\myref{thrm-subfield-test}) we see that $S$, the set of zeroes of $f(x)$ in $F$, is a subfield of $F$.
\begin{mdframed}
    Find all the zeroes of $f(x) = x^4 + x + 1$ over the field $F$ described in \myref{example-GF16-analysis}, using $\alpha$ as one possible zero of $f(x)$ in $F$.
\end{mdframed}
\textbf{Solution}:\newline
 After some work, we notice that
    \begin{align*}
        f(\alpha^2) &= (\alpha^2)^4 + (\alpha^2) + 1\\
        &= \alpha^8 + \alpha^2 + 1\\
        &= (\alpha^2 + 1) + \alpha^2 + 1\\
        &= 2\alpha^2 + 2\\
        &= 0\\
        f(\alpha^4) &= (\alpha^4)^4 + (\alpha^4) + 1\\
        &= \alpha^{16} + \alpha^4 + 1\\
        &= \alpha + (\alpha + 1) + 1\\
        &= 2\alpha + 2\\
        &= 0\\
        f(\alpha^8) &= (\alpha^8)^4 + (\alpha^8) + 1\\
        &= \alpha^{32} + \alpha^8 + 1\\
        &= \alpha^2 + (\alpha^2 + 1) + 1\\
        &= 2\alpha^2 + 2\\
        &= 0
    \end{align*}
    so the other 3 zeroes of $f(x)$ in $F$ are $\alpha^2$, $\alpha^4$, and $\alpha^8$.
\begin{mdframed}
    Prove that $K = \{x \in \GF{p^n} \vert x^{p^m} = x\}$ is a subfield of $\GF{p^n}$.
\end{mdframed}
\textbf{Solution}:\newline
 Clearly $1 \in K$ since $1^{p^m} = 1$, so $K^\ast \neq \emptyset$. Let $x, y \in K$, meaning $x^{p^m} = x$ and $y^{p^m} = y$. Then
    \begin{align*}
        (x-y)^{p^m} &= x^{p^m} - y^{p^m} & (\myref{prop-freshman-dream})\\
        &= x - y\\
        &\in K
    \end{align*}
    and if $y \neq 0$ then
    \begin{align*}
        (xy^{-1})^{p^m} &= x^{p^m}\left(y^{-1}\right)^{p^m}\\
        &= x^{p^m}\left(y^{p^m}\right)^{-1}\\
        &= xy^{-1}\\
        &\in K
    \end{align*}
    and so by the subfield test (\myref{thrm-subfield-test}) we have shown that $K$ is a subfield of $\GF{p^n}$.
\begin{mdframed}
    Prove \myref{corollary-divisibility-of-finite-field-degree}.
\end{mdframed}
\textbf{Solution}:\newline
 By \myref{thrm-subfields-of-finite-field} we know that $\GF{p^n}$ indeed has a subfield isomorphic to $\GF{p^m}$. So one sees by Tower Law (\myref{thrm-tower-law}) that
    \[
        [\GF{p^n}:\GF{p}] = [\GF{p^n}:\GF{p^m}][\GF{p^m}:\GF{p}]
    \]
    which means
    \[
        n = [\GF{p^n}:\GF{p^m}]m
    \]
    by \myref{corollary-degree-of-finite-field-to-prime-power}. Therefore $[\GF{p^n}:\GF{p^m}] = \frac nm$ as required.
\begin{mdframed}
    Consider the finite field $F$ in \myref{example-GF8-analysis}. What are its subfields?
\end{mdframed}
\textbf{Solution}:\newline
 Note that $8 = 2^3$ so the only subfields of $F$ must have order $2^1 = 2$ and $2^3 = 8$. In particular, the subfield of order 2 is $\{0, 1\}$ and the subfield of order 8 is the whole field $F$.

\section*{Problems}
\begin{mdframed}
    Is $\Z_2[x]/\princ{x^3+x+1} \cong \Z_2[x]/\princ{x^3+x^2+1}$? Justify your answer using a concise argument.
\end{mdframed}
\textbf{Solution}:\newline
 Yes. One sees that both $x^3 + x + 1$ and $x^3 + x^2 + 1$ are irreducible over $\Z_2$ since they both do not have zeroes in $\Z_2$ (\myref{thrm-degree-2-or-3-irreducible-iff-has-no-zeroes}). Hence both $\Z_2[x]/\princ{x^3+x+1}$ and $\Z_2/\princ{x^3+x^2+1}$ are fields of order $2^3 = 8$, which means that they are isomorphic.
\begin{mdframed}
    Let $\alpha$ be a zero of $x^3 + x^2 + 1$ in some extension field of $\Z_2$.
    \begin{partquestions}{\roman*}
        \item Find $(\alpha^2 + \alpha + 1)^{-1}$ in $\Z_2(\alpha)$.
        \item Hence solve the equation
        \[
            (\alpha^2 + \alpha + 1)x + \alpha^2 = \alpha
        \]
        for $x$ over $\Z_2(\alpha)$.
    \end{partquestions}
    (\textit{Hint: refer to the conversion formulae given in \myref{example-GF8-analysis}.})
\end{mdframed}
\textbf{Solution}:\newline
 \begin{partquestions}{\roman*}
        \item From \myref{example-GF8-analysis} we see that $\alpha^2 + \alpha + 1 = \alpha^4$. Since $\alpha^7 = 1$, thus $(\alpha^2 + \alpha + 1)^{-1} = (\alpha^4)^{-1} = \alpha^3$, which is $\alpha^2 + 1$.

        \item Note $(\alpha^2 + \alpha + 1)x + \alpha^2 = \alpha$ means $(\alpha^2 + \alpha + 1)x = \alpha^2 + \alpha$. Hence we see
        \begin{align*}
            x &= (\alpha^2+\alpha+1)^{-1}(\alpha^2 + \alpha)\\
            &= (\alpha^2 + 1)(\alpha^2 + \alpha)\\
            &= \alpha^4 + \alpha^3 + \alpha^2 + \alpha\\
            &= (\alpha^2 + \alpha + 1) + (\alpha^2 + 1) + \alpha^2 + 1\\
            &= \alpha^2 + \alpha + 1.
        \end{align*}
    \end{partquestions}
\begin{mdframed}
    Explain why every element of a finite field of characteristic $p$ can be written in the form $a^p$, where $a$ is an element of the same finite field.
    (\textit{Hint: Consider the proof of \myref{thrm-finite-field-is-perfect}.})
\end{mdframed}
\textbf{Solution}:\newline
 In \myref{thrm-finite-field-is-perfect} we have shown that the Frobenius endomorphism $\phi: F \to F, x \mapsto x^p$ is actually an automorphism. The result yields immediately since an automorphism is an isomorphism, which means that every $a \in F$ has a unique $b \in F$ such that $a = b^p$.
\begin{mdframed}
    Consider the irreducible polynomial $f(x) = x^2 + 2x + 2$ over $\Z_3$. Let $\alpha$ be a zero of $f(x)$ in some extension field of $\Z_3$. Find the other zero(es) of $f(x)$ in $\Z_3(\alpha)$.
\end{mdframed}
\textbf{Solution}:\newline
 Since $\deg f(x) = 2$ and $\alpha$ is given to be a zero of $f(x)$, there must only be one other zero in $\Z_3(\alpha)$. Note that $\Z_3(\alpha) \cong \Z_3[x]/\princ{f(x)}$ which is a finite field of $3^2 = 9$ elements (\myref{thrm-characterisation-of-extensions}).

    One sees that $\alpha$ is a generator of $\Z_3(\alpha)$ since
    \begin{multicols}{2}
        \begin{itemize}
            \item $\alpha^1 = \alpha$;
            \item $\alpha^2 = \alpha + 1$;
            \item $\alpha^3 = \alpha^2 + \alpha = 2\alpha + 1$;
            \item $\alpha^4 = 2\alpha^2 + 1 = 2$;
            \item $\alpha^5 = 2\alpha$;
            \item $\alpha^6 = 2\alpha^2 = 2\alpha + 2$;
            \item $\alpha^7 = 2\alpha^2 + 2\alpha = \alpha + 2$; and
            \item $\alpha^8 = \alpha^2 + 2\alpha = 1$.
        \end{itemize}
    \end{multicols}
    One consequently sees that
    \begin{align*}
        f(\alpha^3) &= (\alpha^3)^2 + 2(\alpha^3) + 2\\
        &= \alpha^6 + 2\alpha^3 + 2\\
        &= (2\alpha + 2) + 2(2\alpha + 1) + 2\\
        &= 6\alpha + 6\\
        &= 0
    \end{align*}
    so $\alpha^3 = 2\alpha + 1$ is the other zero of $f(x)$ in $\Z_3(\alpha)$.
\begin{mdframed}
    Let $p(x) = x^3 + 2x + 2 \in \Z_3[x]$.
    \begin{partquestions}{\roman*}
        \item Show that $p(x)$ is irreducible over $\Z_3$.
        \item Let $\alpha$ be a zero of $p(x)$ in the field $F = \Z_3[x]/\princ{p(x)}$. Show that $\alpha$ is \textit{not} a generator of $F^\ast$.
        \item Find a generator of $F^\ast$, and prove that your answer is indeed a generator of $F^\ast$. Express your answer in terms of $\alpha$.
    \end{partquestions}
\end{mdframed}
\textbf{Solution}:\newline
 \begin{partquestions}{\roman*}
        \item One sees that
        \begin{itemize}
            \item $p(0) = 0^3 + 2(0) + 2 = 2 \neq 0$;
            \item $p(1) = 1^3 + 2(1) + 2 = 5 = 2 \neq 0$; and
            \item $p(2) = 2^3 + 2(2) + 2 = 14 = 2 \neq 0$,
        \end{itemize}
        so $p(x)$ has no zeroes in $\Z_3$ and thus is irreducible over $\Z_3$ by \myref{thrm-degree-2-or-3-irreducible-iff-has-no-zeroes}.

        \item Note $|F| = 3^3 = 27$ which means $|F^\ast| = 27 - 1 = 26$. So we require $|\alpha| = 26$ in order for $\alpha$ to be a generator of $F^\ast$. However we see
        \begin{align*}
            \alpha^{13} &= \alpha(\alpha^3)^4\\
            &= \alpha(\alpha + 1)^4 & (\text{as }\alpha^3 + 2\alpha + 2 = 0)\\
            &= \alpha(\alpha^4 + 4\alpha^3 + 6\alpha^2 + 4\alpha + 1)\\
            &= \alpha(\alpha^4 + \alpha^3 + \alpha + 1)\\
            &= \alpha(\alpha+1)(\alpha^3+1)\\
            &= \alpha(\alpha+1)(\alpha+2)\\
            &= \alpha^3 + 3\alpha^2 + 2\alpha\\
            &= \alpha^3 + 2\alpha\\
            &= (\alpha + 1) + 2\alpha\\
            &= 3\alpha + 1\\
            &= 1
        \end{align*}
        which means that $|\alpha| \leq 13$. Therefore $\alpha$ is not a generator of $F^\ast$.

        \item We claim that $2\alpha$ is a generator of $F^\ast$. Since the order of an element must divide the order of a group (\myref{corollary-order-of-group-multiple-of-order-of-element}) we thus know that the order of $2\alpha$ can only be 1, 2, 13, or 26.

        Clearly $2\alpha \neq 1$ and so $|2\alpha| \neq 1$. Also we clearly see $(2\alpha)^2 = 4\alpha^2 = \alpha^2 \neq 1$ so $|2\alpha| \neq 2$. Finally, we see
        \begin{align*}
            (2\alpha)^{13} &= 2^{13}\alpha^{13}\\
            &= 2^{13} & (\text{since, from above, we know }\alpha^{13} = 1)\\
            &= 8192\\
            &= 2\\
            &\neq 1
        \end{align*}
        which means $|2\alpha| \neq 13$. Therefore we conclude $|2\alpha| = 26$, meaning that $2\alpha$ is a generator of $F^\ast$.
    \end{partquestions}
\begin{mdframed}
    Prove that any irreducible factor of $x^{32} - x$ over $\Z_2$ has a degree of 1 or 5.
\end{mdframed}
\textbf{Solution}:\newline
 Suppose $p(x)$ is an irreducible factor of $x^{32} - x$, where $\deg p(x) = n$. Then we know that $\Z_2[x]/\princ{p(x)}$ is a field of order $2^n$. But the splitting field of $x^{32} - x$ over $\Z_2$ has order 32 (\myref{thrm-finite-field-is-unique}), which means that $\Z_2[x]/\princ{p(x)}$ must be a subfield of $\GF{32}$.

    Since $32 = 2^5$, the only possible subfields are of order $2^1 = 2$ and $2^5 = 32$ as 1 and 5 are the only divisors of 5. Hence, an irreducible factor of $x^{32} - x$ must have a degree of 1 or 5.
\begin{mdframed}
    Prove that every finite extension of a finite field is simple.
\end{mdframed}
\textbf{Solution}:\newline
 Let $E$ be a finite extension of $F = \GF{p^n}$. In particular, let $[E:F] = m$. So by virtue of \myref{corollary-divisibility-of-finite-field-degree} we see $E \cong \GF{p^{mn}}$. In particular, $E^\ast$ is cyclic (\myref{thrm-structure-of-finite-field}) and so it contains a generator, say $\alpha$. By \myref{corollary-generator-of-finite-field-is-algebraic-over-subfield} we then see that $\alpha$ is algebraic over $F$ and $\alpha$ has degree $m$. In particular we obtain
    \[
        \underbrace{[\GF{p^{mn}}:\GF{p^n}]}_m = [\GF{p^{mn}}:\GF{p}(\alpha)]\underbrace{[\GF{p}(\alpha):\GF{p}]}_{m}
    \]
    by Tower Law (\myref{thrm-tower-law}), which means $[\GF{p^{mn}}:\GF{p}(\alpha)] = 1$. Hence $\GF{p^{mn}} = \GF{p}(\alpha)$ by \myref{prop-finite-extension-of-degree-1-means-extension-equals-base-field}.
\begin{mdframed}
    Prove that no finite field is algebraically closed.
\end{mdframed}
\textbf{Solution}:\newline
 Let $\GF{p^n}$ be a finite field. In particular, since $\GF{p^n}$ is finite, we may write out all $p^n$ elements, i.e. $\GF{p^n} = \{a_1, a_2, \dots, a_{p^n}\}$ where each $a_i$ is distinct. Now we construct the polynomial
    \[
        f(x) = (x-a_1)(x-a_2)\cdots(x-a_{p^n}) + 1.
    \]
    It is clear that none of the elements in $\GF{p^n}$ are zeroes of $f(x)$. Hence $f(x)$ has no zeroes in $\GF{p^n}$, meaning that $\GF{p^n}$ is not algebraically closed.
\begin{mdframed}
    Let $F$ be a field. We say that $a \in F$ is a \textit{square} if and only if there exists a $b \in F$ such that $a = b^2$. If $a \in F$ is not a square it is called a \textit{non-square}.

    Now let $p$ be an odd prime. Show that if any non-square in $\GF{p}$ is still a non-square in $\GF{p^n}$, then $n$ is an odd positive integer.\newline
    (\textit{Hint: consider the polynomial $f(x) = x^2-a$, where $a$ is a non-square.})
\end{mdframed}
\textbf{Solution}:\newline
 Let $a$ be a non-square in $\GF{p}$. Consider the polynomial $f(x) = x^2 - a$, which has no zeroes in $\GF{p}$ by definition of a non-square. Thus $f(x)$ is irreducible (\myref{thrm-degree-2-or-3-irreducible-iff-has-no-zeroes}).

    Now consider the field $F = \GF{p}/\princ{f(x)}$. Note that $|F| = p^2$, which means $F \cong \GF{p^2}$. This is an extension field of $\GF{p}$ that contains a zero of $f(x)$, i.e. $a$ is a square in $\GF{p^2}$.

    Furthermore, note that an extension field of $\GF{p^2}$ must take the form $\GF{p^{2m}}$ by \myref{thrm-subfields-of-finite-field}. In particular, we see that if $n$ is even, then $a$ becomes a square in $\GF{p^n}$. Taking the contrapositive means that if $a$ remains a non-square in $\GF{p^n}$, then $n$ must be odd.

\chapter{Geometric Constructions}
\section*{Exercises}
\begin{mdframed}
    Let $\theta \in [0, \frac\pi2]$. Prove that $\sin\theta$ is constructible if and only if $\cos\theta$ is constructible.
\end{mdframed}
\textbf{Solution}:\newline
 We know that $\sin^2\theta + \cos^2\theta = 1$, which means that $\cos\theta = \sqrt{1-\sin^2\theta}$. So if $\sin\theta$ is constructible, so is $\sin^2\theta$, $1 - \sin^2\theta$, and $\sqrt{1-\sin^2\theta} = \cos\theta$ by the results about constructible numbers. Similarly we see $\sin\theta = \sqrt{1-\cos^2\theta}$, so if $\cos\theta$ is constructible so is $\sin\theta$.
\begin{mdframed}
    Prove \myref{corollary-field-of-all-constructible-numbers-is-algebraic-extension}.
\end{mdframed}
\textbf{Solution}:\newline
 Let $\gamma$ be a constructible number. Then by \myref{thrm-condition-for-constructability} we know that $[\Q(\gamma):\Q] = 2^k$ for some integer $k$, meaning that $\Q(\gamma)$ is a finite extension of $\Q$ and is hence algebraic (\myref{thrm-finite-extension-is-algebraic}), which hence means that $\gamma$ is algebraic over $\Q$. Therefore any constructible number is algebraic over $\Q$, proving that $\mathfrak{C}$ is an algebraic extension of $\Q$.
\begin{mdframed}
    Prove that $8x^3 - 6x - 1$ is irreducible over $\Q$.\newline
    (\textit{Hint: transformation before checking irreducibility.})
\end{mdframed}
\textbf{Solution}:\newline
 We note that $8x^3 - 6x - 1$ is irreducible if and only if $8\left(\frac12x\right)^3 - 6\left(\frac12x\right) - 1 = x^3 - 3x - 1$ is irreducible by a corollary of the transformation rule (\myref{corollary-irreducible-iff-constant-factor-multiple-is-irreducible}). Reducing the coefficients of the new polynomial modulo 2 yields $f(x) = x^3 + x + 1$. Note $f(0) = 1 \neq 0$ and $f(1) = 3 = 1 \neq 0$ so $f(x)$ has no zeroes in $\Z_2$ which means that it is irreducible over $\Z_2$ (\myref{thrm-degree-2-or-3-irreducible-iff-has-no-zeroes}). Hence $f(x)$ is irreducible over $\Q$ by Mod 2 Irreducibility Test (\myref{thrm-mod-p-irreducibility-test}) which therefore means $8x^3 - 6x - 1$ is irreducible over $\Q$.

\section*{Problems}

\chapter{Galois}
\section*{Exercises}
\begin{mdframed}
    Consider the field extension $\C/\R$.
    \begin{partquestions}{\roman*}
        \item Find the order of $\Gal{\C/\R}$.
        \item Find the element(s) of $\Gal{\C/\R}$.
    \end{partquestions}
\end{mdframed}
\textbf{Solution}:\newline
 \begin{partquestions}{\roman*}
        \item We note that $[\C:\R] = 2$ since $\C = \R(i)$ and $i$ is a zero of the irreducible polynomial $x^2 + 1$ over $\R$. Therefore, by \myref{thrm-order-of-galois-group-is-degree-of-field-extension}, we see $|\Gal{\C/\R}| = [\C:\R] = 2$.

        \item Certainly $\id \in \Gal{\C/\R}$. We claim that the other element in $\Gal{\C/\R}$ is $\phi: \C \to \C$ where $\phi(a+bi) = a-bi$ for all $a+bi \in \C$. We first need to check that $\phi$ is an automorphism.
        \begin{itemize}
            \item \textbf{Homomorphism}: One sees clearly for any $a+bi, c+di \in \C$ that
            \begin{align*}
                \phi((a+bi)+(c+di)) &= \phi((a+c)+(b+d)i)\\
                &= (a+c)-(b+d)i\\
                &= (a-bi) + (c-di)\\
                &= \phi(a+bi) + \phi(c+di)
            \end{align*}
            and
            \begin{align*}
                \phi((a+bi)(c+di)) &= \phi((ac-bd) + (ad+bc)i)\\
                &= (ac-bd) - (ad+bc)i\\
                &= (a-bi)(c-di)\\
                &= \phi(a+bi)\phi(c+di)
            \end{align*}
            which proves that $\phi$ is indeed a homomorphism.

            \item \textbf{Injective}: Since $\phi$ is non-trivial it is thus injective by \myref{thrm-homomorphism-from-field-is-injective-or-trivial}.

            \item \textbf{Surjective}: For any $a + bi \in \C$ we note that $a - bi \in \C$ and that $\phi(a - bi) = a - (-b)i = a+bi$, proving that $\phi$ is surjective.
        \end{itemize}
        Therefore $\phi$ is a bijective homomorphism from $\C$ to $\C$, i.e. an automorphism. One also sees that $\phi(r) = r$ for all $r \in \R$, so $\phi$ fixes $\R$. Thus $\phi \in \Gal{\C/\R}$. But as $\Gal{\C/\R}$ has order 2, thus $\id$ and $\phi$ are the only two elements in $\Gal{\C/\R}$.
    \end{partquestions}
\begin{mdframed}
    Prove \myref{prop-algebraic-extension-of-field-of-characteristic-0-is-separable}.
\end{mdframed}
\textbf{Solution}:\newline
 Let $\alpha \in E$. Since $E$ is algebraic, it is the zero of a polynomial $f(x) \in F[x]$. In particular we may choose $f(x)$ to be the minimal polynomial of $\alpha$. Since the minimal polynomial is irreducible (\myref{corollary-minimal-polynomial-is-irreducible}), thus \myref{thrm-zeroes-of-an-irreducible} tells us that $f(x)$ is separable. Therefore any $\alpha \in E$ is the zero of a separable polynomial in $F[x]$, which shows that $E$ is a separable extension of $F$.
\begin{mdframed}
    Prove \myref{prop-fixed-field-is-subfield}.
\end{mdframed}
\textbf{Solution}:\newline
 Let $\phi \in H$. One sees that $1 \in \Fix{E}{H}^\ast$ since $\phi(1) = 1$ by properties of homomorphism. For any $a, b \in \Fix{E}{H}$ we also see that
    \begin{align*}
        \phi(a - b) &= \phi(a) - \phi(b)\\
        &= a - b,
    \end{align*}
    which means $a - b \in \Fix{E}{H}$. Finally, for any $a \in \Fix{E}{H}$ and $b \in \Fix{E}{H}^\ast$ one notes
    \begin{align*}
        \phi(ab^{-1}) &= \phi(a)\phi(b^{-1})\\
        &= \phi(a)\left(\phi(b)\right)^{-1}\\
        &= ab^{-1}
    \end{align*}
    which means $ab^{-1} \in \Fix{E}{H}$. Therefore by subfield test (\myref{thrm-subfield-test}) we see $\Fix{E}{H}$ is a subfield of $E$.
\begin{mdframed}
    Let $E$ be a field, $G$ be a subgroup of $\Aut{E}$, and $F = \Fix{E}{G}$. Show $G \leq \Gal{E/F}$.
\end{mdframed}
\textbf{Solution}:\newline
 All we need to show is that $G \subseteq \Gal{E/F}$ since $G$ is already a group. Suppose $\sigma \in G$. Note that
    \[
        \Fix{E}{G} = \{a \in E \vert \phi(a) = a \text{ for all } \phi \in G\}
    \]
    by definition, which means that $\sigma(a) = a$ for all $a \in \Fix{E}{G}$. Therefore, by definition of a Galois group, we see $\sigma \in \Gal{E/\Fix{E}{G}}$, meaning $G \subseteq \Gal{E/\Fix{E}{G}}$. Therefore $G \leq \Gal{E/\Fix{E}{G}}$.
\begin{mdframed}
    Prove that the group action defined in \myref{thrm-degree-of-element-under-fixed-field-action} is indeed a group action.
\end{mdframed}
\textbf{Solution}:\newline
 We need to prove the group action axioms.
    \begin{itemize}
        \item \textbf{Identity}: One sees that for any $x \in E$ we have
        \begin{align*}
            \id\cdot x &= \id(x)\\
            &= x.
        \end{align*}
        \item \textbf{Compatibility}: Let $\sigma, \mu \in G$ and $x \in E$. Then
        \begin{align*}
            \sigma \cdot (\mu \cdot x) &= \sigma \cdot (\mu(x))\\
            &= \sigma(\mu(x))\\
            &= (\sigma\circ \mu)(x) & (\text{where } \circ \text{ denotes function composition})\\
            &= (\sigma\mu) \cdot x.
        \end{align*}
    \end{itemize}
    Therefore the group action defined in \myref{thrm-degree-of-element-under-fixed-field-action} is, indeed, a group action.
\begin{mdframed}
    Prove that the degree of $\alpha_1$ over $F$ in \myref{thrm-degree-of-element-under-fixed-field-action} divides $|G|$.
\end{mdframed}
\textbf{Solution}:\newline
 By the Orbit-Stabilizer theorem (\myref{thrm-orbit-stabilizer}) we know that
    \[
       |G| = |\Orb{G}{x}||\Stab{G}{x}|
    \]
    for all $x \in E$. Since the degree of $\alpha_1$ over $F$ is $|\Orb{G}{\alpha_1}| = r$ by \myref{thrm-degree-of-element-under-fixed-field-action}, therefore one sees clearly that the degree of $\alpha_1$ divides the order of $G$.
\begin{mdframed}
    Prove \myref{prop-intermediate-field-of-galois-extension-is-galois-extension}.
\end{mdframed}
\textbf{Solution}:\newline
 Clearly if every element of $E$ is a zero of a separable polynomial in $F[x]$, say $f(x)$, we may just view $f(x)$ as a polynomial in $K[x]$ and we would obtain the fact that every element of $E$ is a zero of a separable polynomial in $K[x]$. Therefore $E$ is a separable extension of $K$. Coupled with \myref{prop-intermediate-field-of-normal-extension-is-normal-extension} proves that $E/K$ is a Galois extension.
\begin{mdframed}
    Prove that all abelian groups are solvable.
\end{mdframed}
\textbf{Solution}:\newline
 For any abelian group $G$, it has a subnormal series
    \[
        1 \lhd G.
    \]
    Clearly $G/1 \cong G$ is abelian, which means that all factor groups are abelian. Therefore, by definition of a solvable group, $G$ is solvable.
\begin{mdframed}
    Prove \myref{thrm-abel-ruffini}.
\end{mdframed}
\textbf{Solution}:\newline
 The degree 5 polynomial $f(x) = x^5 - 4x - 2$ is not solvable by radicals over $\Q$. For $n > 5$ the existence of a solution by radicals of the general polynomial of degree $n$ yields a solution by radicals of the equation $f(x)x^{n-5} = 0$, contradicting the above.

\section*{Problems}
\begin{mdframed}
    Let $E$ be the splitting field of a polynomial of degree $n$ over the field $F$. Prove that $|\Gal{E/F}|$ divides $n!$.
\end{mdframed}
\textbf{Solution}:\newline
 Recall that $\Gal{E/F}$ permutes the $n$ zeroes of the polynomial (\myref{prop-galois-field-automorphism-permutes-zeroes-of-polynomial}). Thus Cayley's theorem (\myref{thrm-cayley}) tells us that $\Gal{E/F}$ is isomorphic to a permutation group; in fact it is isomorphic to a subgroup of $\Sn{n}$. Since the order of a subgroup must divide the order of the group by Lagrange's theorem (\myref{thrm-lagrange}), thus $|\Gal{E/F}|$ must divide the order of $\Sn{n}$, which is $n!$.
\begin{mdframed}
    Let $f(x) = x^2 + 3$ and $F$ be the splitting field of $f(x)$ over $\R$. Determine $F$ and $\Gal{F/\R}$.
\end{mdframed}
\textbf{Solution}:\newline
 Clearly the zeroes of $x^2 + 3$ over $\R$ are $\pm\sqrt{-3} = \pm\sqrt3i$. Therefore
    \[
        F = \R(\sqrt3i, -\sqrt3i) = \R(i) = \C.
    \]
    Note that $[\C:\R] = 2$ which means that $\Gal{\C/\R} = 2$ (\myref{thrm-order-of-galois-group-is-degree-of-field-extension}). One thus sees that the elements of $\Gal{F/\R} = \Gal{\C/\R}$ are $\id$ and $\sigma$ where $\sigma(a + bi) = a - bi$ for any $a + bi \in \C$.
\begin{mdframed}
    Let $f(x)$ be an irreducible quadratic polynomial, and let $E$ be its splitting field over a field $F$. Prove that $\Gal{E/F} \cong \Cn{2}$.
\end{mdframed}
\textbf{Solution}:\newline
 Since $f(x)$ is irreducible, it must be separable over its splitting field. In fact $[E:F] = 2$, which means $|\Gal{E/F}| = 2$. But by \myref{corollary-group-with-prime-order-is-cyclic} this means that $\Gal{E/f} \cong \Cn{2}$.
\begin{mdframed}
    Let $\sigma \in \Aut{\R}$. Prove that $\sigma(x) > 0$ for all $x > 0$.
\end{mdframed}
\textbf{Solution}:\newline
 Since $x > 0$ thus $\sqrt x \in \R$. In particular one sees that
    \begin{align*}
        \sigma(x) &= \sigma\left((\sqrt x)^2\right)\\
        &= \left(\sigma(\sqrt x)\right)^2\\
        &\geq 0
    \end{align*}
    which means that $\sigma(x) \geq 0$ for all $x > 0$. Now, seeking a contradiction, suppose $\sigma(a) = 0$ for some $a > 0$. But note that $\sigma(0) = 0$ by properties of homomorphism, which shows that $\sigma$ is not injective and hence not an automorphism, a contradiction. Therefore $\sigma(x) \neq 0$ for all $x > 0$, i.e. $\sigma(x) > 0$ for all $x > 0$.
\begin{mdframed}
    Suppose $E$ is the splitting field of some polynomial over $\GF{p}$, where $p$ is a prime. If $|\Gal{E/\GF{p}}| = q^6$ where $q$ is also a prime, how many fields are strictly between $E$ and $\GF{p}$? That is, how many fields $F$ are there such that $\GF{p} \subset F \subset E$?
\end{mdframed}
\textbf{Solution}:\newline
 Given that $|\Gal{E/\GF{p}}| = q^6$, this means that $[E:\GF{p}] = q^6$ (\myref{thrm-order-of-galois-group-is-degree-of-field-extension}). Therefore by \myref{thrm-subfields-of-finite-field} we see that $E = \GF{p^{q^6}}$. Note by the same theorem that the number of subfields of $E$ is the number of distinct divisors of $q^6$, which is 7 ($q^0 = 1$, $q$, $q^2$, $q^3$, $q^4$, $q^5$, and $q^6$). But what we want are subfields that are strictly between $\GF{p}$ and $E$. The case of $q^0 = 1$ just gives us $\GF{p}$ and the case of $q^6$ gives us $E$, so the number of subfields that are strictly between $\GF{p}$ and $E$ is $7 - 2 = 5$.
\begin{mdframed}
    Consider the polynomial $f(x) = x^3 - 2$. Let $E$ be the splitting field of $f(x)$ over $\Q$.
    \begin{partquestions}{\roman*}
        \item Find a field $F$ such that $F$ is a simple extension of $\Q$, its primitive element is a real number, and $\Q \subset F \subset E$.
        \item Show that $F$ is \textit{not} a splitting field of any polynomial in $\Q[x]$.
    \end{partquestions}
\end{mdframed}
\textbf{Solution}:\newline
 \begin{partquestions}{\roman*}
        \item Note that $x^3 - 2 = (x-\sqrt[3]2)(x-\sqrt[3]2\omega)(x-\sqrt[3]2\omega^2)$ where $\omega$ is a primitive 3rd root of unity. So
        \begin{align*}
            E &= \Q(\sqrt[3]2, \sqrt[3]2\omega, \sqrt[3]2\omega^2)\\
            &= \Q(\sqrt[3]2,\omega).
        \end{align*}
        So the only possible field $F$ that fits the requirements is $F = \Q(\sqrt[3]2)$.

        \item Suppose $\Q(\sqrt[3]2)$ is the splitting field of some polynomial $g(x) \in \Q[x]$, which means that $g(\sqrt[3]2) = 0$ for that polynomial. By \myref{corollary-minimal-polynomial-divides-polynomial-with-same-root} we know that the minimal polynomial of $\sqrt[3]2$ must divide $g(x)$. In fact note that the minimal polynomial of $\sqrt[3]2$ is exactly $f(x) = x^3 - 2$, which means that $\Q(\sqrt[3]2)$ has to have \textit{all} the zeroes of $f(x)$. However the zeroes of $f(x)$ are $\sqrt[3]2$, $\sqrt[3]2\omega$, and $\sqrt[3]2\omega^2$, of which the latter two are not in $\Q(\sqrt[3]2)$. Hence $\Q(\sqrt[3]2)$ cannot be the splitting field of any polynomial.
    \end{partquestions}
\begin{mdframed}
    Prove that $D_n$ is solvable for all positive integers $n$.
\end{mdframed}
\textbf{Solution}:\newline
 If $n < 3$ we see that $D_n$ is abelian and hence solvable. So let's assume that $n \geq 3$. We find a subnormal series for $D_n$ and show that it fits the requirements for a solvable series.

    Note that $\langle r \rangle = \{e, r, r^2, \dots, r^{n-1}\}$ is a subgroup of $D_n$. In fact one sees clearly that $\langle r \rangle \cong \Cn{n}$. Thus by Lagrange's theorem (\myref{thrm-lagrange}) we see that $[D_n:\langle r \rangle] = \frac{2n}{n} = 2$. Therefore by \myref{problem-subgroup-of-index-2} we see that $\langle r \rangle \lhd D_n$. So a subnormal series of $D_n$ is
    \[
        1 \lhd \Cn{n} \lhd D_n,
    \]
    where one sees that
    \begin{itemize}
        \item $\Cn{n}/1 \cong \Cn{n}$ is abelian; and
        \item $D_n/\Cn{n} \cong \C_2$ is also abelian.
    \end{itemize}
    Therefore $D_n$ is solvable.
\begin{mdframed}
    Prove that $\Sn{n}$ is \textit{not} solvable for all $n \geq 5$.
\end{mdframed}
\textbf{Solution}:\newline
 By \myref{prop-solvable-equivalence-for-finite-groups} we just need to show that the composition series for $\Sn{n}$ fails the requirements to allow $\Sn{n}$ to be solvable.

    Note that $\An{n} \lhd \Sn{n}$ (\myref{prop-An-normal-subgroup-of-Sn}), $|\Sn{n}| = n!$ (\myref{exercise-order-of-Sn}), and $|\An{n}| = \frac{n!}2$ (\myref{prop-order-of-An}). Note that a subgroup of $\Sn{n}$ must divide the order of the group (namely $n!$), and a proper subgroup can have an order of at most $\frac{n!}2$. Therefore $\An{n}$ is the largest subgroup of $\Sn{n}$. In fact, $\An{n}$ is the maximal normal subgroup of $\Sn{n}$. However, note that $\An{n}$ is simple for $n \geq 5$ (\myref{thrm-An-is-simple-for-n>=5}) which means that it has no proper normal subgroups. Hence the composition series for $\Sn{n}$ is
    \[
        1 \lhd \An{n} \lhd \Sn{n}.
    \]
    One sees clearly that although $\Sn{n}/\An{n} \cong \Cn{2}$ works, $\An{n}/1 \cong \An{n}$ does not. Therefore $\Sn{n}$ is not solvable.
\begin{mdframed}
    Let $\theta \in \R$. \textbf{De Moivre's formula}\index{De Moivre's formula} states that
    \[
        (\cos\theta + i\sin\theta)^n = \cos(n\theta) + i\sin(n\theta)
    \]
    for all non-negative integers $n$.
    \begin{partquestions}{\roman*}
        \item Prove De Moivre's formula.\newline
        (\textit{Hint: you will need \myref{thrm-sine-sum-rule} and \myref{thrm-cosine-sum-rule}.})
        \item Deduce a primitive $n$th root of unity in $\C$, proving your answer.
    \end{partquestions}
\end{mdframed}
\textbf{Solution}:\newline
 \begin{partquestions}{\roman*}
        \item We induct on $n$.

        When $n = 0$, one trivially sees that
        \begin{align*}
            (\cos\theta + i\sin\theta)^0 &= 1\\
            &= \cos(0\theta) + i\sin(0\theta)
        \end{align*}
        which proves this case.

        Assume that the statement holds for some non-negative integer $k$, i.e. $(\cos\theta + i\sin\theta)^k = \cos(k\theta) + i\sin(k\theta)$. We show that the statement holds for $k+1$ too.

        Note that
        \begin{align*}
            &(\cos\theta + i\sin\theta)^{k+1}\\
            &= (\cos\theta + i\sin\theta)^k(\cos\theta + i\sin\theta)\\
            &= (\cos(k\theta) + i\sin(k\theta))(\cos\theta + i\sin\theta) & (\text{Induction Hypothesis})\\
            &= \cos(k\theta)\cos\theta + i\cos(k\theta)\sin\theta\\
            &\quad\quad+ i\cos\theta\sin(k\theta) + i^2\sin(k\theta)\sin\theta\\
            &= (\cos(k\theta)\cos\theta - \sin(k\theta)\sin\theta)\\
            &\quad\quad+ i(\cos(k\theta)\sin\theta + \cos\theta\sin(k\theta))\\
            &= \cos(k\theta+\theta) + i\sin(k\theta+\theta) & (\myref{thrm-sine-sum-rule}, \myref{thrm-cosine-sum-rule})\\
            &= \cos((k+1)\theta) + i\sin((k+1)\theta)
        \end{align*}
        which proves the statement for the $k + 1$ case.

        \item We show that $\omega = \cos\left(\frac{2\pi}n\right) + i\sin\left(\frac{2\pi}n\right)$ is a primitive $n$th root of unity.

        Note that
        \begin{align*}
            \omega^n &= (\cos\left(\frac{2\pi}n\right) + i\sin\left(\frac{2\pi}n\right))^n\\
            &= \cos\left(\frac{2\pi}n\times n\right) + i\sin\left(\frac{2\pi}n\times n\right) & (\text{by }\textbf{(i)})\\
            &= \cos(2\pi) + i\sin(2\pi)\\
            &= 1 + 0i\\
            &= 1
        \end{align*}
        so $\omega$ is a $n$th root of unity. Also one sees clearly that $\omega^k \neq 1$ for all $1 \leq k < n$, which shows that $\omega$ is indeed a primitive $n$th root of unity.
    \end{partquestions}
\begin{mdframed}
    Let $F$ be a field and $f(x) \in F[x]$ be a separable polynomial of degree $n$. Suppose $E$ is the splitting field of $f(x)$ over $F$. Let $\alpha_1, \alpha_2, \dots, \alpha_n$ be the zeroes of $f(x)$ in $E$. Let
    \[
        D = \prod_{i<j}(\alpha_i - \alpha_j).
    \]
    For example, for two distinct zeroes, $D = \alpha_1 - \alpha_2$. For three distinct zeroes, $D = (\alpha_1-\alpha_2)(\alpha_1-\alpha_3)(\alpha_2-\alpha_3)$. The \textbf{discriminant}\index{discriminant}\index{polynomial!discriminant} of $f(x)$ is defined to be $\Delta = D^2$.
    \begin{partquestions}{\roman*}
        \item Verify that if $F = \Q$ then the discriminant of $x^4 - 6x^3 + 11x^2 - 6x$ is 144.
        \item Prove that $\Delta \in F$.\newline
        (\textit{Hint: consider what an automorphism from $\Gal{E/F}$ does to $D$.})
        \item If $\sigma \in \Gal{E/F}$ is the transposition of two zeroes of $f(x)$, show that $\sigma(D) = -D$.
        \item Show that $\sigma \in \Gal{E/F}$ is an even permutation of the zeroes of $f(x)$ if and only if $\sigma(D) = D$.
        \item Hence prove that $\Gal{E/F}$ is isomorphic to a subgroup of $\An{n}$ if and only if $D \in F$.
        \item It is known that if $f(x) = x^3 + px + q$ for some constants $p, q \in F$, then
        \[
            \Delta = -4p^3 - 27q^2.
        \]
        Hence determine $\Gal{E/F}$ if $F = \R$ and $f(x) = x^3 + 2x - 4$.
    \end{partquestions}
\end{mdframed}
\textbf{Solution}:\newline
 \begin{partquestions}{\roman*}
        \item Note that the zeroes of $x^4 - 6x^3 + 11x^2 - 6x$ are 0, 1, 2, and 3. So
        \begin{align*}
            D &= (0-1)(0-2)(0-3)(1-2)(1-3)(2-3)\\
            &= (-1)(-2)(-3)(-1)(-2)(-1)\\
            &= 12
        \end{align*}
        and thus $\Delta = D^2 = 144$.

        \item Let $\sigma \in \Gal{E/F}$. Note that
        \begin{align*}
            \sigma(D) &= \sigma\left(\prod_{i<j}(\alpha_i - \alpha_j)\right)\\
            &= \prod_{i<j}(\sigma(\alpha_i) - \sigma(\alpha_j)).
        \end{align*}
        Note that \myref{prop-galois-field-automorphism-permutes-zeroes-of-polynomial} tells us that $\sigma(\alpha_r)$ is simply another zero of $f(x)$ for all $r \in \{1, 2, \dots, n\}$. This means that the overall effect of $\sigma$ on $D$ is that $\sigma(D) = \pm D$, where the change in sign could occur due to the swapping of zeroes. However one sees that
        \[
            \sigma(\Delta) = \sigma(D^2) = (\sigma(D))^2 = (\pm D)^2 = D^2 = \Delta.
        \]
        Now this is true for any automorphism $\sigma \in \Gal{E/F}$, which means that $\Delta \in \Fix{E}{\Gal{E/F}}$. But by \myref{prop-fixed-field-of-Gal-E/F-is-F} we see that $\Fix{E}{\Gal{E/F}} = F$ and thus $\Delta \in F$.

        \item Say $\sigma(\alpha_k) = \alpha_r$ and $\sigma(\alpha_r) = \alpha_k$ for some $k$ and $r$. Without loss of generality we may assume $k < r$. So when we apply $\sigma$ to $D$, the only thing that changes is the term $(\alpha_k - \alpha_r)$. In particular, $\sigma(\alpha_k - \alpha_r) = \alpha_r - \alpha_k = -(\alpha_k - \alpha_r)$ and so $\sigma(D) = -D$.

        \item Let $\sigma \in \Gal{E/F}$. Since $\sigma$ is a permutation it is made up of transpositions (\myref{lemma-permutations-as-product-of-transpositions}), say $\tau_1, \tau_2, \dots, \tau_k$. This means that $\sigma(D) = \tau_1\tau_2\cdots\tau_k(D)$. Now by \textbf{(iii)} each of these transpositions changes the sign of the resulting output, which therefore means that $\sigma(D) = (-1)^kD$.

        Now if $\sigma$ is an odd permutation, this means that $\sigma$ is made up of an odd number of transpositions (\myref{thrm-parity-of-permutation}). Therefore $k$ is odd and so $\sigma(D) = -D$. On the other hand if $\sigma$ is even then $\sigma(D) = D$.

        \item For the forward direction, suppose $\Gal{E/F} \cong H \leq \An{n}$. Thus any $\sigma \in \Gal{E/F}$ is an even permutation, which therefore means $\sigma(D) = D$ for all $\sigma \in \Gal{E/F}$ by \textbf{(iv)}. Thus $D \in \Fix{E}{\Gal{E/F}} = F$ by \myref{prop-fixed-field-of-Gal-E/F-is-F}.

        For the reverse direction, suppose $D \in F$. Then $\sigma(D) = D$ for all $\sigma \in \Gal{E/F}$. By \textbf{(iv)} again this means that $\sigma$ is even for all $\sigma \in \Gal{E/F}$. This means that $\Gal{E/F}$ contains only even permutations, and so must be isomorphic to a subgroup of $\An{n}$.

        \item One sees that $\Delta = -4(2)^3 - 27(-4)^2 = -464$. Clearly $D = \sqrt{\Delta} \notin \R$ and so $\Gal{E/F}$ is \textit{not} isomorphic to a subgroup of $\An{n}$ (in particular $\An{3}$) by \textbf{(v)}. However, we note $\Gal{E/F}$ is isomorphic to a subgroup of $\Sn{3}$ and that $\An{3}$ is the largest subgroup of $\Sn{3}$. Since $\Gal{E/F}$ is not a subgroup of $\An{3}$, this leaves the case where $\Gal{E/F} \cong \Sn{3}$ as the only possibility.
    \end{partquestions}

\chapter{Fta}
\section*{Exercises}
\begin{mdframed}
    Let the set $S = \{1 - \frac1{2^n} \vert n \in \mathbb{N}\}$. Find $\inf S$ and $\sup S$, proving that the values found are indeed the infimum and supremum of $S$ respectively.
\end{mdframed}
\textbf{Solution}:\newline
 We note $\inf S = \frac12$ since $1 - \frac1{2^n} >
    \frac12$ for $n > 1$.

    We claim that $\sup S = 1$. Clearly 1 is an upper bound of $S$ since $1 - \frac1{2^n} < 1$ for all $n \in \mathbb{N}$. Now suppose $\epsilon > 0$. Choose $n \in \mathbb{N}$ such that $n > -\log_2\epsilon$, which means $\frac1{2^n} < \epsilon$. Then note that $1 - \frac1{2^n} \in S$ and also
    \[
        1 - \frac1{2^n} > 1 - \epsilon.
    \]
    Therefore by \myref{prop-identifying-suprema} we see that $\sup S = 1$.
\begin{mdframed}
    In the proof of \myref{thrm-fundamental-theorem-of-algebra}, explain why $\Gal{L/\R}$ has even order.
\end{mdframed}
\textbf{Solution}:\newline
 Note by Tower Law (\myref{thrm-tower-law}) that
    \[
        [L:\R] = [L:\C][\C:\R]
    \]
    and because $[\C:\R] = 2$ and $[L:\C]$ is finite, thus we see that $[L:\R] = 2k$ for some integer $k$. As $L/\R$ is a finite Galois extension we therefore have $\Gal{L/\R} = [L:\R] = 2k$ by \myref{corollary-galois-iff-galois-field-has-order-of-degree-of-field-extension}.
\begin{mdframed}
    Let $z = x + yi \in \C$ where $x, y \in \R$. Prove that $\sqrt{z} \in \C$ rigorously using \myref{lemma-non-negative-real-number-has-square-root}. That is, prove that there is a $w \in \C$ such that $w^2 = z$.
\end{mdframed}
\textbf{Solution}:\newline
 Let
    \[
        u = \pm\sqrt{\frac{x + \sqrt{x^2+y^2}}{2}} \text{ and } v = \pm\sqrt{\frac{-x + \sqrt{x^2+y^2}}{2}}.
    \]
    where the signs of $u$ and $v$ are chosen such that $uv$ has the same sign as $y$. Note that $x^2 + y^2 \geq 0$. By \myref{lemma-non-negative-real-number-has-square-root} we know that $\sqrt{x^2+y^2} \in \R$. Also $\sqrt{x^2+y^2} \geq \sqrt{x^2} = |x| \geq x$ so $-x + \sqrt{x^2+y^2} \geq 0$. Therefore by \myref{lemma-non-negative-real-number-has-square-root} again we see $u, v \in \R$. Without loss of generality let $u$ and $v$ both be positive. Let $w = u + vi \in \C$. Note that
    \begin{align*}
        uv &= \sqrt{\frac{x + \sqrt{x^2+y^2}}{2}}\sqrt{\frac{-x + \sqrt{x^2+y^2}}{2}}\\
        &= \sqrt{\left(\frac{x + \sqrt{x^2+y^2}}{2}\right)\left(\frac{-x + \sqrt{x^2+y^2}}{2}\right)}\\
        &= \sqrt{\frac{\left(x + \sqrt{x^2+y^2}\right)\left(-x + \sqrt{x^2+y^2}\right)}{4}}\\
        &= \sqrt{\frac{(x^2+y^2)-x^2}{4}}\\
        &= \sqrt{\frac{y^2}4}\\
        &= \frac y2.
    \end{align*}
    Thus we see
    \begin{align*}
        w^2 &= (u+vi)^2\\
        &= u^2 + 2uvi + (vi)^2\\
        &= u^2 + 2uvi - v^2\\
        &= \frac{x + \sqrt{x^2+y^2}}{2} + 2\left(\frac y2\right)i - \frac{-x + \sqrt{x^2+y^2}}{2}\\
        &= x + yi
    \end{align*}
    which therefore means that $\sqrt{x+yi} = \sqrt{z} \in \C$.

\section*{Problems}

% END OF AUTOGENERATED CONTENT

\end{document}
