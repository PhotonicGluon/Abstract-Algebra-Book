\setpartpreamble[u][\textwidth]{
    \quoteattr{
        [Some] of the major discoveries in ring theory have helped shape the course of development of modern abstract algebra... A course in ring theory is an indispensable part of the education of any fledgling algebraist.
    }
    {
        Tsit-Yuen Lam, 2001
    }
    {
        \cite{lam_2001}
    }

    The invention of rings was motivated by mathematicians' desire to generalise number systems. What resulted was a structure that has numerous connections to other fields in mathematics, such as algebraic geometry, algebraic number theory and commutative algebra, as well as practical applications such as cryptography.

    We begin this part by looking at the historical motivation for rings and how they act as a generalisation of number systems. We define the axioms of rings and then get into the main types of rings and their properties. We then focus on integral domains, the class of rings that most number systems follow. These three chapters underlie the exploration of rings in subsequent chapters; it is vital to have a firm grasp of the concepts there.

    Ideals are commonly discussed objects in ring theory, as they are essential to the properties and structures held by a ring. We discuss a myriad of types of ideals and end with two prominent examples of ideals present in every ring. Following that, we look at ring homomorphisms and isomorphisms. Such maps are similar to homomorphisms and isomorphisms for groups, but their restriction to rings leads to some interesting results.

    The next chapter focuses on polynomials, which one with knowledge of high school algebra should be familiar with. We explore polynomials within rings, their terminology, and polynomial long division. Polynomials are essential in algebra, much more so in abstract algebra, so one must fully understand this chapter.

    We then look at factorisation, first in polynomial rings and then in domains. We circle back to the idea of primes and irreducibles and develop some tests determining the reducibility of polynomials and results about the factorisation of elements in integral domains in general. We end this part by examining how rings, specifically polynomial rings, can be used in encryption. Although we only present a simple explanation here, we ensure that the core of the process is provided in full.
}
\part{Rings}
