\chapter{Factorization of Polynomials}
One of the underappreciated facts about working with polynomials with real coefficients is that they can be factored simply. The idea of an `irreducible' polynomial is simple to understand within the reals -- it simply cannot be `broken down' into `smaller' polynomials. We explore how polynomials are factored in a more general setting -- polynomial rings.

\section{Irreducible Polynomials}
We formally define what irreducible polynomials are in a polynomial ring that is an integral domain.
\begin{definition}
    Let $D$ be an integral domain and $f(x) \in D[x]$ be a non-zero, non-unit polynomial. Then $f(x)$ is \term{irreducible over $D$}\index{polynomial!irreducible}\index{irreducible!polynomial} if and only if, whenever we write $f(x) = p(x)q(x)$ for some polynomials $p(x)$ and $q(x)$ in $D[x]$, then either $p(x)$ is a unit or $q(x)$ is a unit in $D[x]$.

    A non-zero, non-unit polynomial that is not irreducible over $D$ is said to be \term{reducible over $D$}\index{polynomial!reducible}.
\end{definition}

\begin{example}
    The polynomial $x + 1$ is irreducible over $\Z$ since we are unable to express $x+1$ as a product of non-unit polynomials in $\Z[x]$.
\end{example}

\begin{example}
    The polynomial $2x + 2$ is reducible over $\Z$ since $2x+2 = 2(x+1)$ and both 2 and $x+1$ are not units in $\Z[x]$.
\end{example}

It is often more useful to talk about irreducible polynomials over a field. In this case, we have this useful theorem to simplify the task of determining the irreducibility of a polynomial.
\begin{theorem}\label{thrm-irreducible-iff-not-expressable-as-product-of-smaller-polynomials}
    Let $F$ be a field. A non-constant polynomial $f(x) \in F[x]$ with degree $n$ is irreducible if and only if $f(x)$ cannot be expressed as a product of two polynomials each of degree lower than $n$.
\end{theorem}
\begin{proof}
    For the forward direction, assume that a non-constant polynomial $f(x)$ with degree $n$ is irreducible. So there does not exist two non-unit polynomials $p(x)$ and $q(x)$ such that $f(x) = p(x)q(x)$, which thus means that it cannot be expressed as two polynomials of degree lower than $n$.

    For the reverse direction, assume that $f(x)$ with degree $n$ cannot be expressed as a product of two polynomials of degree lower than $n$. This means that it is possible for $f(x) = p(x)q(x)$ where at least one of the polynomials $p(x)$ and $q(x)$ have degree $n$. Without loss of generality assume $p(x)$ has degree $n$, thereby meaning $q(x)$ has degree $n - n = 0$ (\myref{thrm-polynomial-degree-properties}). Thus $q(x)$ is a constant polynomial, so $q(x) = u \in F$ (\myref{prop-constant-polynomial-iff-ring-element}). Note $u \neq 0$ since otherwise $f(x) = 0$, a constant polynomial (a contradiction). But every non-zero element in $F$ is a unit. Therefore the only factorization of $f(x)$ is as a product of a unit and a polynomial of equal degree. Thus $f(x)$ is irreducible.
\end{proof}

\begin{example}
    One sees that $2x^2 + 4$ is reducible over $\Z$ since $2x^2 + 4 = 2(x^2 + 2)$ and neither 2 nor $x^2 + 2$ are units in $\Z[x]$. However, in $\Q$, the given polynomial is irreducible by \myref{thrm-irreducible-iff-not-expressable-as-product-of-smaller-polynomials}.
\end{example}

\begin{example}
    The polynomial $2x^2 + 4$ is irreducible over $\Q$ and $\R$, but reducible over $\C$ since $2x^2 + 4 = (2x-4i)(x+2i)$.
\end{example}

\begin{exercise}
    Consider the polynomial $f(x) = 3x^2 - 6$. Which of the following integral domains is $f(x)$ irreducible over?
    \begin{partquestions}{\alph*}
        \item $\Z$
        \item $\Q$
        \item $\R$
    \end{partquestions}
\end{exercise}

\section{Tests for Irreducibility}
Trying to see if a polynomial is irreducible by definition is not a trivial task. For example, it is hard to see that $x^2 + 1$ is irreducible over $\Z_3$ but reducible over $\Z_5$ by just using the definition directly. Therefore we develop some tests for irreducibility to more easily see which polynomials are reducible or irreducible.

\subsection{Zeroes of a Polynomial}
The first test is a consequence of the factor theorem.
\begin{theorem}\label{thrm-degree-above-1-reducible-if-has-zero}
    Let $F$ be a field and $f(x) \in F[x]$ have degree above 1. If $f(x)$ has a zero then $f(x)$ is reducible.
\end{theorem}
\begin{proof}
    Since $f(x)$ has a zero, say $\alpha \in F$, we may write $f(x) = (x-\alpha)q(x)$ by Factor Theorem (\myref{corollary-factor-theorem}). Note $\deg f(x) = \deg(x-\alpha) + \deg q(x) = 1 + \deg q(x)$ (\myref{thrm-polynomial-degree-properties}), and clearly $1 \leq \deg q(x) < \deg f(x)$ (since the degree of $f(x)$ is above 1), so $f(x)$ is expressible as a product of two polynomials of lower degree in $F[x]$. Thus $f(x)$ is reducible over $F$ (\myref{thrm-irreducible-iff-not-expressable-as-product-of-smaller-polynomials}).
\end{proof}

The next test provides a partial converse to the above theorem for polynomials of degree 2 or 3.
\begin{theorem}\label{thrm-degree-2-or-3-irreducible-iff-has-no-zeroes}
    Let $F$ be a field and $f(x) \in F[x]$ have degree 2 or 3. Then $f(x)$ is irreducible over $F$ if and only if $f(x)$ has no zeroes.
\end{theorem}
\begin{proof}
    The forward direction follows immediately as it is the contrapositive of \myref{thrm-degree-above-1-reducible-if-has-zero}, so we only work in the reverse direction.

    We consider a contrapositive proof. Assume that $f(x) \in F[x]$ has degree $n \in \{2, 3\}$ and is reducible. Then $f(x) = p(x)q(x)$ where both $p(x)$ and $q(x)$ have degrees less than $n$ (\myref{thrm-irreducible-iff-not-expressable-as-product-of-smaller-polynomials}). Note $n = \deg p(x) + \deg q(x)$, if $n = 2$ then both $p(x)$ and $q(x)$ must have degree 1, and if $n = 3$ then at one of $p(x)$ and $q(x)$ has degree 1. So we know that at least one of $p(x)$ or $q(x)$ has degree 1. Without loss of generality assume $p(x)$ has degree 1; set $p(x) = ax + b$ where $a,b \in F$. Note $a^{-1} \in F$; setting $\alpha = -a^{-1}b$ we see
    \begin{align*}
        f(\alpha) &= (a\alpha+b)q(\alpha)\\
        &= \left(a\left(-a^{-1}b\right) + b\right)q(\alpha)\\
        &= \left(-aa^{-1}b + b\right)q(\alpha)\\
        &= (-b+b)q(\alpha)\\
        &= 0.
    \end{align*}
    Since $\alpha \in F$, thus $\alpha$ is a zero of $f(x)$ in $F$.
\end{proof}

\myref{thrm-degree-2-or-3-irreducible-iff-has-no-zeroes} is particularly useful for finite fields, since we just have to check all elements of the field for a zero.

\begin{example}
    Consider the opening example: the polynomial $f(x) = x^2 + 1$. In $\Z_3$, we note that
    \begin{itemize}
        \item $f(0) = 0^2 + 1 = 1 \neq 0$;
        \item $f(1) = 1^2 + 1 = 2 \neq 0$; and
        \item $f(2) = 2^2 + 1 = 5 = 2 \neq 0$,
    \end{itemize}
    so $f(x)$ is not reducible (i.e., irreducible) over $\Z_3$ by \myref{thrm-degree-2-or-3-irreducible-iff-has-no-zeroes}. However, note that $f(2) = 5 = 0$ in $\Z_5$, meaning $f(x)$ has a zero in $\Z_5$. Thus $x^2 + 1$ is reducible over $\Z_5$.
\end{example}

\begin{example}
    Consider the polynomial $f(x) = x^2 - 2$. One sees that $f(3) = 3^2 - 2 = 7 = 0$ in $\Z_7$, so $f(x)$ is reducible over the field $\Z_7$. However $f(x)$ is not reducible over $\Q$ since it has no zeroes in $\Q$. But $f(x)$ is reducible over $\R$ since $f(\sqrt2) = 0$.
\end{example}

\begin{example}
    The theorem fails for polynomials with degree of at least 4. For example, $f(x) = x^4 + 2x^2 + 1$ is reducible over $\Q$ since $x^4 + 2x^2 + 1 = (x^2+1)^2$, but $f(x)$ has no zeroes in $\Q$.
\end{example}

\begin{exercise}
    Consider the polynomial $f(x) = 2x^3 + 4x + 9$. For which field(s) listed below is $f(x)$ irreducible?
    \begin{multicols}{4}
        \begin{partquestions}{\alph*}
            \item $\Z_2$
            \item $\Z_3$
            \item $\Z_5$
            \item $\Z_7$
        \end{partquestions}
    \end{multicols}
\end{exercise}

\subsection{Irreducibility Over $\Q$}
Before we introduce the next test, which involves polynomials with integer coefficients, we first introduce some terminology and a lemma to ease the proof of the test.

\begin{definition}
    Let $D$ be an integral domain. The \term{content}\index{polynomial!content} of a non-zero polynomial $f(x) \in D[x]$ is the GCD of its coefficients.
\end{definition}

\begin{definition}
    Let $D$ be an integral domain. A \term{primitive polynomial}\index{polynomial!primitive} is a non-zero polynomial $f(x) \in D[x]$ with a content of 1.
\end{definition}

\begin{lemma}[Gauss]\label{lemma-gauss-for-integer-coefficients}\index{Gauss's Lemma!for integer coefficients}
    The product of two primitive polynomials in $\Z[x]$ is primitive.
\end{lemma}
\begin{proof}[Proof (see {\cite[p.~291]{gallian_2016}})]
    Seeking a contradiction, suppose that $f(x)$ and $g(x)$ are primitive polynomials with a non-primitive product $h(x) = f(x)g(x)$. Therefore the content of $h(x)$ is not 1; in fact it is at least 2, meaning the content of $h(x)$ is either prime or composite. Thus the content of $h(x)$ has a prime divisor, say $p$.

    Let $\bar{f}(x)$, $\bar{g}(x)$, and $\bar{h}(x)$ be polynomials obtained from $f(x)$, $g(x)$, and $h(x)$ by reducing their coefficients modulo $p$. Since the content of $h(x)$ is not 1, thus coefficients of $h(x)$ are multiples of $p$; accordingly when they are reduced modulo $p$, we see $\bar{h}(x) = 0$. Also, we may interpret $\bar{f}(x)$ and $\bar{g}(x)$ as polynomials belonging in $\Z_p[x]$ where $\bar{f}(x)\bar{g}(x) = \bar{h}(x) = 0$. Since $\Z_p[x]$ is an integral domain (\myref{thrm-integral-domain-iff-polynomial-ring-is-integral-domain}), this means that $\bar{f}(x) = 0$ or $\bar{g}(x) = 0$. Without loss of generality we may assume $\bar{f}(x) = 0$. Therefore $p$ divides every coefficient of $f(x)$, meaning that the content of $f(x)$ is at least $p$, i.e. $f(x)$ is not primitive, a contradiction.

    The contradiction means a product of two primitive polynomials is primitive.
\end{proof}

\begin{theorem}\label{thrm-irreducible-over-Z-means-irreducible-over-Q}
    Let $f(x) \in \Z[x]$.
    \begin{itemize}
        \item If $f(x)$ is irreducible over $\Z$ then it is irreducible over $\Q$.
        \item If $f(x)$ is a primitive polynomial and is irreducible over $\Q$ then it is irreducible over $\Z$.
    \end{itemize}
    Furthermore, if $f(x)$ is reducible over $\Z$ with $f(x) = p(x)q(x)$ for some non-unit polynomials $p(x), q(x) \in \Z[x]$, then neither $p(x)$ nor $q(x)$ are constant polynomials.
\end{theorem}
\begin{proof}
    We prove the contrapositive of the first statement. Suppose that $f(x) \in \Z[x]$ is reducible over $\Q$, meaning that $f(x) = p(x)q(x)$ where $p(x), q(x) \in \Q[x]$ are non-constant polynomials. Without loss of generality, assume the content of $f(x)$ is 1, since otherwise we can just divide $p(x)$ and $q(x)$ by the content of $f(x)$ to obtain the same result. Thus $f(x)$ is primitive.

    Let $a$ be the LCM of the denominators of the coefficients of $p(x)$, and likewise let $b$ be the LCM of the denominators of the coefficients of $q(x)$. Then
    \[
        abf(x) = (ap(x))(bq(x)),
    \]
    noting that $ap(x)$ and $bq(x)$ are now polynomials with integer coefficients. Let $C_p$ and $C_q$ denote the contents of $ap(x)$ and $bq(x)$ respectively; write $ap(x) = C_pP(x)$ and $bq(x) = C_qQ(x)$ where $P(x)$ and $Q(x)$ are primitive polynomials. One sees that
    \[
        ab \times f(x) = C_pC_q \times P(x)Q(x).
    \]
    As $f(x)$ is primitive, thus the content of $ab \times f(x)$ is $ab$. As the product of primitive polynomials is primitive (\myref{lemma-gauss-for-integer-coefficients}), so $P(x)Q(x)$ is also primitive; the content of $C_pC_q \times P(x)Q(x)$ is therefore $C_pC_q$. So $ab = C_pC_q$.

    Dividing both sides by $ab = C_pC_q$ leaves $f(x) = P(x)Q(x)$, where both $P(x)$ and $Q(x)$ are polynomials with integer coefficients. Note that
    \begin{align*}
        &\deg p(x)\\
        &= \deg (ap(x)) & (\text{multiplying by constant does not change degree})\\
        &= \deg(C_pP(x)) & (\text{since } ap(x) = C_pP(x))\\
        &= \deg P(x)
    \end{align*}
    and likewise $\deg q(x) = \deg Q(x)$. Since $f(x) = p(x)q(x)$ and $\Q$ is a field, thus we see that $\deg f(x) = \deg p(x) + \deg q(x)$ (\myref{thrm-polynomial-degree-properties}), meaning $\deg p(x) = \deg P(x) < \deg f(x)$, and likewise $\deg Q(x) < \deg f(x)$, So we have expressed $f(x)$, a polynomial with integer coefficients, as a product of two polynomials $P(x)$ and $Q(x)$ with integer coefficients and of smaller degree than $f(x)$, showing that $f(x)$ is reducible over $\Z$.

    Before we prove the second statement, we note that the final assertion made in the theorem is readily proven by noticing that $P(x)$ and $Q(x)$ are non-constant polynomials of degree smaller than $f(x)$.

    We now prove the contrapositive of the second statement. If $f(x)$ is reducible over $\Z$, then $f(x) = p(x)q(x)$ where $p(x)$ and $q(x)$ are non-unit polynomials. We note that $p(x)$ and $q(x)$ cannot be constant polynomials, since otherwise that would mean that $f(x)$ has a content that is not 1. Thus $p(x)$ and $q(x)$ are non-constant polynomials of degree smaller than $f(x)$. As all integers are also rational numbers, these same polynomials appear in $\Q[x]$ and so $f(x)$ is reducible over $\Q$.
\end{proof}

\begin{example}
    Consider the polynomial $f(x) = 6x^2 + 5x - 4$. One sees that $f(x)$ is reducible over $\Q$ since $6x^2 + 5x - 4 = \left(2x + \frac83\right)\left(3x - \frac32\right)$. Set $p(x) = 2x + \frac83$ and $q(x) = 3x - \frac32$. In this case, and using the notation in \myref{thrm-irreducible-over-Z-means-irreducible-over-Q}, we see $a = 3$, $b = 2$, $C_p = 2$, $C_q = 3$. So $P(x) = 3x + 4$ and $Q(x) = 2x - 1$, which means
    \[
        (3\times2)(6x^2 + 5x - 4) = (2\times3)(3x+4)(2x-1),
    \]
    so $6x^2 + 5x - 4 = (3x+4)(2x-1)$, i.e. $f(x)$ is reducible over $\Z$.
\end{example}

\begin{example}
    Consider $f(x) = 144x^4 + 168x^3 + 73x^2 + 14x + 1$. Note that $f(x) = \left(16x^2 + \frac{32}3x + \frac{16}9\right)\left(9x^2 + \frac92x + \frac9{16}\right)$. Set $p(x) = 16x^2 + \frac{32}3x + \frac{16}9$ and $q(x) = 9x^2 + \frac92x + \frac9{16}$ and using the notation in \myref{thrm-irreducible-over-Z-means-irreducible-over-Q} we see $a = 9$, $b = 16$, $C_p = 16$, and $C_q = 9$. So $P(x) = 9x^2+6x+1$ and $Q(x) = 16x^2 + 8x + 9$, which means
    \begin{align*}
        &(9 \times 16)\left(144x^4 + 168x^3 + 73x^2 + 14x + 1\right)\\
        &= (16 \times 9)(9x^2+6x+1)(16x^2 + 8x + 9)
    \end{align*}
    so $144x^4 + 168x^3 + 73x^2 + 14x + 1 = (9x^2+6x+1)(16x^2 + 8x + 9)$, i.e. $f(x)$ is reducible over $\Z$.
\end{example}

\begin{exercise}
    Let $f(x) \in \Z[x]$. Prove or disprove the following statements.
    \begin{partquestions}{\alph*}
        \item If $f(x)$ is reducible over $\Z$ then it is reducible over $\Q$.
        \item If $f(x)$ is reducible over $\Q$ then it is reducible over $\Z$.
    \end{partquestions}
\end{exercise}

\subsection{Mod $p$ Irreducibility Test}
\begin{theorem}[Mod $p$ Irreducibility Test]\label{thrm-mod-p-irreducibility-test}\index{Mod $p$ Irreducibility Test}
    Let $p$ be a prime and suppose $f(x) \in \Z[x]$ is a non-constant polynomial. Let $\bar{f}(x) \in \Z_p[x]$ be the polynomial obtained after reducing all coefficients of $f(x)$ modulo $p$. If $\bar{f}(x)$ is irreducible over $\Z_p$ and $\deg \bar{f}(x) = \deg f(x)$ then $f(x)$ is irreducible over $\Q$.
\end{theorem}
\begin{proof}
    Seeking a contrapositive proof, suppose $f(x)$ is reducible over $\Q$. Then \myref{thrm-irreducible-over-Z-means-irreducible-over-Q} tells us that there exists non-constant polynomials $p(x), q(x) \in \Z[x]$, both with degree smaller than that of $f(x)$, such that $f(x) = p(x)q(x)$. Let $\bar{f}(x), \bar{p}(x), \bar{q}(x) \in \Z_p[x]$ be the polynomials after reducing the coefficients of $f(x)$, $p(x)$, and $q(x)$ modulo $p$ respectively. Since $\deg \bar{f}(x) = \deg f(x)$ by assumption, we have
    \begin{align*}
        \deg \bar{p}(x) &\leq \deg p(x) < \deg f(x),\\
        \deg \bar{q}(x) &\leq \deg q(x) < \deg f(x),
    \end{align*}
    and $\bar{f}(x) = \bar{p}(x)\bar{q}(x)$. Thus $\bar{f}(x)$ is reducible over $\Z_p$.
\end{proof}

\begin{example}
    Consider the polynomial $f(x) = 7x^3 + x^2 + 2x + 5$. Reducing coefficients modulo 2 yields $\bar{f}(x) = x^3 + x^2 + 1$. Note that
    \begin{itemize}
        \item $\bar{f}(0) = 0^3 + 0^2 + 1 = 1 \neq 0$; and
        \item $\bar{f}(1) = 1^3 + 1^2 + 1 = 3 = 1 \neq 0$,
    \end{itemize}
    so $\bar{f}(x)$ has no zeroes in $\Z_2$. Thus $\bar{f}(x)$ is irreducible over $\Z_2$ (\myref{thrm-degree-2-or-3-irreducible-iff-has-no-zeroes}) and so $f(x)$ is irreducible over $\Q$ by Mod 2 Irreducibility Test (\myref{thrm-mod-p-irreducibility-test}).
\end{example}
\begin{example}
    Consider the polynomial $f(x) = 2x^2 + 3x + 4$.

    We note that reducing $f(x)$ modulo 2 yields $3x$, but we cannot use the Mod 2 Irreducibility Test (\myref{thrm-mod-p-irreducibility-test}) on it since $3x$ is of smaller degree than $f(x)$.

    If instead we reduce $f(x)$ modulo 3, we get the polynomial $\bar{f}(x) = 2x^2 + 1$. But one sees clearly that $\bar{f}(1) = 2(1)^2 + 1 = 3 = 0$ in $\Z_3$, so $\bar{f}(x)$ is reducible modulo 3 (\myref{thrm-degree-2-or-3-irreducible-iff-has-no-zeroes}).

    Let's try reducing $f(x)$ modulo 5, yielding a polynomial with the same coefficients in $\Z_5[x]$. Observe that, when evaluating in $\Z_5$, we have
    \begin{itemize}
        \item $f(0) = 2(0)^2 + 3(0) + 4 = 4 \neq 0$;
        \item $f(1) = 2(1)^2 + 3(1) + 4 = 9 = 4 \neq 0$;
        \item $f(2) = 2(2)^2 + 3(2) + 4 = 18 = 3 \neq 0$;
        \item $f(3) = 2(3)^2 + 3(3) + 4 = 31 = 1 \neq 0$; and
        \item $f(4) = 2(4)^2 + 3(4) + 4 = 48 = 3 \neq 0$,
    \end{itemize}
    so $f(x)$ is irreducible modulo 5 (\myref{thrm-degree-2-or-3-irreducible-iff-has-no-zeroes}). Therefore $f(x)$ is irreducible over $\Q$ by Mod 5 Irreducibility Test.
\end{example}

It is important to know that not all primes $p$ used in the Mod $p$ Irreducibility Test (\myref{thrm-mod-p-irreducibility-test}) will result in a polynomial that is irreducible over $\Z_p$. As long as \textit{one} prime $p$ makes $f(x)$ irreducible over $\Z_p$, then the Mod $p$ Irreducibility Test holds and we may conclude that $f(x)$ is irreducible over $\Q$.

\begin{exercise}
    Prove that, for all prime numbers $p$, there exists a non-negative integer $n < p$ such that $n^2 + 2n + 1 \equiv 0 \pmod{p}$.
\end{exercise}

The Mod $p$ Irreducibility Test can also be used to check for irreducibility of polynomials of degree greater than 3, though more thought needs to be executed in order to come up with a rigorous proof.

\begin{example}\label{example-x^4+x+1-is-irreducible-over-Z2}
    We will show that $f(x) = \frac15x^4 + \frac8{15}x^3 + \frac45x^2 + \frac13x + 3$ is irreducible over $\Q$. First, let $F(x) = 15f(x) = 3x^4 + 8x^3 + 12x^2 + 5x + 45$, which is a polynomial with integer coefficients. Reducing coefficients of $F(x)$ modulo 2 yields $\bar{F}(x) = x^4 + x + 1$. One sees that
    \begin{itemize}
        \item $\bar{F}(0) = 0^4 + 0 + 1 = 1 \neq 0$; and
        \item $\bar{F}(1) = 1^4 + 1 + 1 = 3 = 1 \neq 0$
    \end{itemize}
    so $\bar{F}(x)$ has no zeroes in $\Z_2$. Now we \textit{cannot} conclude that $F(x)$ is irreducible at this juncture as $\bar{F}(x)$ is a degree 4 polynomial, and \myref{thrm-degree-2-or-3-irreducible-iff-has-no-zeroes} only applies to degree 2 and degree 3 polynomials. We need to do more work to conclude that $F(x)$ is irreducible.

    Seeking a contradiction, suppose that $\bar{F}(x) = p(x)q(x)$ where both $p(x), q(x) \in \Z_2[x]$ are polynomials with lower degree than $\bar{F}(x)$. Note neither $p(x)$ nor $q(x)$ can be linear, as this would imply that $\bar{F}(x)$ has a zero in $\Z_2$ (which it does not). So both $p(x)$ and $q(x)$ are quadratic factors. There are 4 distinct quadratic polynomials in $\Z_2[x]$, namely
    \begin{enumerate}
        \item $x^2$;
        \item $x^2 + 1$;
        \item $x^2 + x$; and
        \item $x^2 + x + 1$,
    \end{enumerate}
    and clearly options 1, 2, and 3 are not possible factors of $\bar{F}(x)$ since they clearly have zeroes in $\Z_2$. That leaves $x^2 + x + 1$ being the only possible quadratic factor; since $p(x)$ and $q(x)$ are both quadratic this means $p(x) = q(x) = x^2 + x + 1$. But one sees that
    \begin{align*}
        (x^2 + x + 1)^2 &= x^4 + 2x^3 + 3x^2 + 2x + 1\\
        &= x^4 + x^2 + 1
    \end{align*}
    which is not $\bar{F}(x)$, a contradiction. Thus $\bar{F}(x)$ is irreducible over $\Z_2$.

    Since $\bar{F}(x)$ is irreducible, thus $F(x)$ is irreducible over $\Q$ by Mod 2 Irreducibility Test (\myref{thrm-mod-p-irreducibility-test}). Hence $f(x) = \frac1{15} F(x)$ is irreducible over $\Q$.
\end{example}

\subsection{Eisenstein's Criterion}
We now look at another irreducibility test, given by Gotthold Eisenstein in 1850.

\newpage

\begin{theorem}[Eisenstein's Criterion]\label{thrm-eisenstein-criterion}\index{Eisenstein's Criterion}
    For the polynomial
    \[
        f(x) = a_0 + a_1x + \cdots + a_{n-1}x^{n-1} + a_nx^n
    \]
    in $\Z[x]$, if there is a prime number $p$ such that
    \begin{itemize}
        \item $p \nmid a_n$,
        \item $p \vert a_i$ for all $0 \leq i < n$, and
        \item $p^2 \nmid a_0$,
    \end{itemize}
    then $f(x)$ is irreducible over $\Q$.
\end{theorem}
\begin{proof}
    Seeking a contradiction, suppose that $f(x)$ is reducible over $\Q$. Thus $f(x)$ is also reducible over $\Z$, and, in fact, there exist non-constant polynomials $g(x),h(x) \in \Z[x]$ with degree smaller than that of $f(x)$ such that $f(x) = g(x)h(x)$ (\myref{thrm-irreducible-over-Z-means-irreducible-over-Q}).

    Write
    \begin{align*}
        g(x) &= b_0 + b_1x + \cdots + b_rx^r\\
        h(x) &= c_0 + c_1x + \cdots + c_sx^s
    \end{align*}
    where $0 < r, s < n$. Note $a_0 = b_0c_0$. Since $p \vert a_0$ but $p^2 \nmid a_0$ thus $p$ divides exactly one of $b_0$ and $c_0$. Without loss of generality assume $p \vert b_0$ and $p \nmid c_0$. Note also that $a_n = b_rc_s$ and $p \nmid a_n$ by assumption. Thus $p \nmid b_r$. So there must exist a smallest integer $t$, where $0 < t \leq r$, such that $p \nmid b_t$.

    Consider the coefficient of the degree $t$ term in $f(x)$, which is
    \[
        a_t = \sum_{i=0}^tb_ic_{t-i} = b_0c_t + b_1c_{t-1} + \cdots + b_tc_0.
    \]
    By assumption, $p \vert a_t$ since $0 < t \leq r < n$; also $p \vert b_i$ for all $0 \leq i < t$, so every term except for the last is divisible by $p$. This means that
    \[
        p \vert (a_t - (b_0c_t + b_1c_{t-1} + \cdots + b_{t-1}c_1)).
    \]
    However $a_t - (b_0c_t + b_1c_{t-1} + \cdots + b_{t-1}c_1) = b_tc_0$, and neither $b_t$ nor $c_0$ is divisible by $p$, a contradiction.

    Hence $f(x)$ is irreducible over $\Q$.
\end{proof}

\begin{example}
    The polynomial $f(x) = 3x^4 + 15x^2 + 10$ is irreducible over $\Q$ by Eisenstein's Criterion (\myref{thrm-eisenstein-criterion}) since the prime 5 divides 10 and 15 but does not divide 3, and also that $5^2 = 25$ does not divide 10.
\end{example}

\begin{exercise}
    Let $n$ be an arbitrary positive integer. Find a polynomial $f(x) \in \Z[x]$ of degree $n$ that is irreducible over $\Q$.
\end{exercise}

\section{Irreducibility via Transformation}
Sometimes the tests may not be applied directly. In these cases, appropriate substitutions or transformations are needed to convert polynomials into a form that is suitable for testing its irreducibility.

\begin{theorem}[Transformation Rule]\label{thrm-transformation-rule-for-irreducibility}
    Let $D[x]$ be an integral domain and $f(x) \in D[x]$ be a non-zero, non-unit polynomial. Suppose $\phi$ is an automorphism of $D[x]$. Then $f(x)$ is irreducible if and only if $\phi(f(x))$ is irreducible.
\end{theorem}
\begin{proof}
    We prove the reverse direction first, that is, we will prove that if $\phi(f(x))$ is irreducible then $f(x)$ is irreducible, via contrapositive. Let $f(x) \in D[x]$ be reducible, meaning there exist non-unit $p(x), q(x) \in D[x]$ such that $f(x) = p(x)q(x)$. Then
    \[
        \phi(f(x)) = \phi(p(x)q(x)) = \phi(p(x))\phi(q(x)).
    \]
    Note that, for a polynomial $u(x) \in D[x]$, it is a unit if and only if there exists $v(x) \in D[x]$ such that $u(x)v(x) = 1$. This is true if and only if $\phi(u(x))\phi(v(x))  = \phi(u(x)v(x)) = \phi(1) = 1$ because $\phi$ is an automorphism (which is an isomorphism), which means that $\phi(u(x))$ is a unit. To summarise we know $u(x)$ is a unit if and only if $\phi(u(x))$ is a unit. So we know that neither $\phi(p(x))$ nor $\phi(q(x))$ are units. Therefore $\phi(f(x))$ is also reducible.

    We now show the forward direction, again via contrapositive. Suppose $\phi(f(x))$ is reducible. Note $\phi^{-1}$ is also a ring isomorphism (\myref{problem-properties-of-ring-isomorphism}) so it is also a ring automorphism, so use the reverse direction's result to conclude that $f(x)$ is also reducible.
\end{proof}

With this general result proven, we come up with a few rules for substitution.

\begin{corollary}\label{corollary-irreducible-iff-translation-is-irreducible}
    Let $D$ be an integral domain, $k \in D$, and $f(x) \in D[x]$. Then $f(x)$ is irreducible if and only if $f(x + k)$ is irreducible.
\end{corollary}
\begin{proof}
    Consider the map $\phi: D[x] \to D[x]$ defined by
    \[
        \sum_{i=0}^n a_ix^i \mapsto \sum_{i=0}^na_i(x+k)^i.
    \]
    We show that this is a ring automorphism. For brevity let
    \begin{align*}
        f(x) &= a_0 + a_1x + \cdots + a_mx^m\\
        g(x) &= b_0 + b_1x + \cdots + b_nx^n
    \end{align*}
    be polynomials in $D[x]$, where we assume, without loss of generality, that $m \geq n$, and set $b_i = 0$ for $i > n$.
    \begin{itemize}
        \item \textbf{Homomorphism}: One sees that
        \begin{align*}
            &\phi(f(x) + g(x))\\
            &= \phi\left(\sum_{i=0}^m(a_i+b_i)x^i\right)\\
            &= \sum_{i=0}^m(a_i+b_i)(x+k)^i\\
            &= \left(\sum_{i=0}^ma_i(x+k)^i\right) + \left(\sum_{i=0}^mb_i(x+k)^i\right)\\
            &= \left(\sum_{i=0}^ma_i(x+k)^i\right) + \left(\sum_{i=0}^nb_i(x+k)^i\right) & (\text{as } b_i = 0 \text{ for } i > n)\\
            &= \phi(f(x)) + \phi(g(x))
        \end{align*}
        and
        \begin{align*}
            &\phi(f(x)g(x))\\
            &= \phi\left(\sum_{r=0}^{m+n}\left(\sum_{i=0}^ra_ib_{r-i}\right)x^r\right)\\
            &= \sum_{r=0}^{m+n}\left(\sum_{i=0}^ra_ib_{r-i}\right)(x+k)^r\\
            &= \left(\sum_{r=0}^ma_r(x+k)^r\right)\left(\sum_{r=0}^nb_r(x+k)^r\right)\\
            &= \phi(f(x))\phi(g(x))
        \end{align*}
        so $\phi$ is a ring homomorphism.

        \item \textbf{Injective}: Suppose $\phi(f(x)) = \phi(g(x))$. Then
        \[
            \sum_{i=0}^ma_i(x+k)^i = \sum_{i=0}^nb_i(x+k)^i
        \]
        which, by comparing coefficients, we see $a_i = b_i$ for all $0 \leq i \leq m$. Therefore $f(x) = g(x)$ which means $\phi$ is injective.

        \item \textbf{Surjective}: Let $p(x) = c_0 + c_1x + \cdots + c_rx^r \in D[x]$. Note that $q(x) = c_0 + c_1(x-k) + \cdots + c_r(x-k)^r \in D[x]$ also; observe
        \begin{align*}
            \phi(q(x)) &= c_0 + c_1((x-k)+k) + \cdots + c_r((x-k)+k)^r\\
            &= c_0 + c_1x + \cdots + c_rx^r\\
            &= p(x)
        \end{align*}
        so any $p(x) \in D[x]$ has a pre-image under $\phi$.
    \end{itemize}
    Therefore $\phi$ is an isomorphism (and so it is an automorphism), thus $f(x)$ is irreducible if and only if $f(x+k)$ is irreducible by \myref{thrm-transformation-rule-for-irreducibility}.
\end{proof}

\begin{corollary}\label{corollary-irreducible-iff-constant-factor-multiple-is-irreducible}
    Let $D$ be an integral domain, $k \in D$ be a unit, and let $f(x) \in D[x]$. Then $f(x)$ is irreducible if and only if $f(kx)$ is irreducible.
\end{corollary}
\begin{proof}
    Consider the map $\phi: D[x] \to D[x]$ where
    \[
        \sum_{i=0}^n a_ix^i \mapsto \sum_{i=0}^na_i(kx)^i.
    \]
    \myref{exercise-substitution-by-constant-factor-multiple-map} (later) shows that this is a ring isomorphism (and so is an automorphism), so this means that $f(x)$ is irreducible if and only if $f(kx)$ is irreducible.
\end{proof}

\begin{example}
    Since $x^2 + 1$ is irreducible over $\Z$, we know that $(x+2)^2 + 1 = x^2 + 4x + 5$ is also irreducible over $\Z$.
\end{example}

\begin{example}
    We note that $x^2 + x + 2$ is irreducible over $\Z_3$ since
    \begin{align*}
        x^2 + x + 2 &= x^2 + 4x + 5\\
        &= (x^2 + 4x + 4) + 1\\
        &= (x+2)^2 + 1
    \end{align*}
    and $x^2 + 1$ is irreducible over $\Z_3$.
\end{example}

\begin{example}
    We note that $x^3 + x + 1$ has no zeroes in $\Q$ so $x^3 + x + 1$ is irreducible over $\Q$. So we also know that
    \begin{itemize}
        \item $(x+1)^3 + (x+1) 1 = x^3 + 3x^2 + 4x + 3$;
        \item $(\frac12x)^3 + (\frac12x) + 1 = \frac18x^3 + \frac12x + 1$; and
        \item $(\frac12x + 3)^3 + (\frac12x + 3) + 1 = \frac18x^3 + \frac94x^2 + 14x + 31$
    \end{itemize}
    are all irreducible polynomials in $\Q$.
\end{example}

\begin{example}
    We can show that $f(x) = x^2 + x + 2$ is irreducible over $\Q$. Since the coefficient of $x$, which is 1, is not divisible by any prime, we cannot directly apply Eisenstein's Criterion (\myref{thrm-eisenstein-criterion}). However we may use the substitution $x + 3$ and see that $f(x+3) = (x+3)^2 + (x+3) + 2 = x^2 + 7x + 14$ which is irreducible by Eisenstein's Criterion using the prime 7. Therefore $f(x)$ is irreducible over $\Q$.
\end{example}

\begin{exercise}
    Let $f(x) = x^4 + 3$.
    \begin{partquestions}{\roman*}
        \item Show that $f(x)$ is irreducible over $\Q$.
        \item Explain why $x^4 + 4x^3 + 6x^2 + 4x + 4$ has no zeroes in $\Z$.
        \item Show that $x^4 - 8x^3 + 24x^2 - 32x + 19$ is irreducible over $\Q$.
    \end{partquestions}
\end{exercise}

\begin{exercise}\label{exercise-substitution-by-constant-factor-multiple-map}
    Show that the map $\phi$ given in \myref{corollary-irreducible-iff-constant-factor-multiple-is-irreducible} is indeed a ring automorphism.
\end{exercise}

\section{Uses of Irreducible Polynomials}
One might ask why we are so concerned with irreducible polynomials. It turns out that there is a fundamental connection between irreducible polynomials, maximal ideals, and fields.

We reveal the first connection in the following theorem.

\begin{theorem}\label{thrm-irreducible-iff-principal-ideal-is-maximal}
    Suppose $D$ is a PID with a polynomial $f(x) \in D[x]$. Then $p(x)$ is irreducible over $D$ if and only if $\princ{f(x)}$ is maximal.
\end{theorem}
\begin{proof}
    We first prove the forward direction. Assume that $f(x)$ is irreducible over $D$, and let $M = \princ{f(x)}$. Suppose we have an ideal $I$ of $D[x]$ such that $M \subseteq I \subseteq D[x]$. Since $D[x]$ is a PID, we know that $I = \princ{g(x)}$ for some $g(x) \in D[x]$. Note $f(x) \in I$ so $f(x) = g(x)h(x)$ for some polynomial $h(x) \in D[x]$. Since $f(x)$ is irreducible over $D$, thus either $g(x)$ or $h(x)$ is a unit.
    \begin{itemize}
        \item If $g(x)$ is a unit, then $I = D[x]$ by \myref{prop-ideal-contains-unit-iff-ideal-is-whole-ring}.
        \item If $h(x)$ is a unit, say $h(x) = u \in D$, then $I = \princ{uf(x)} = \princ{f(x)} = M$ (\myref{prop-principal-ideals-equal-iff-associates}).
    \end{itemize}
    Thus $M$ is a maximal ideal.

    Now we prove the reverse direction; suppose $M = \princ{f(x)}$ is a maximal ideal in $D[x]$. Suppose $f(x) = p(x)q(x)$ where $p(x), q(x) \in D[x]$. We show that at least one of $p(x)$ or $q(x)$ is a unit.

    Seeking a contradiction, suppose neither $p(x)$ nor $q(x)$ are units. We show that $\princ{f(x)} \subset \princ{p(x)}$. Let $g(x) \in \princ{f(x)}$, meaning there exists $k(x) \in D[x]$ such that $g(x) = f(x)k(x)$. Since $f(x) = p(x)q(x)$ we have $g(x) = (p(x)q(x))k(x) = p(x)(q(x)k(x)) \in \princ{p(x)}$. Therefore $\princ{f(x)} \subseteq \princ{p(x)}$. Now if $p(x) \in \princ{f(x)}$, then there exists an $r(x) \in F[x]$ such that $p(x) = f(x)r(x)$. As $f(x) = p(x)q(x)$ we have
    \[
        f(x) = p(x)q(x) = f(x)r(x)q(x)
    \]
    which thus means $r(x)q(x) = 1$, and so $q(x)$ is a unit, contradicting the original assumption that neither $p(x)$ nor $q(x)$ are units. Thus $p(x) \notin \princ{f(x)}$. But clearly $p(x) \in \princ{p(x)}$, so $\princ{f(x)} \subset \princ{p(x)}$.

    As $M = \princ{f(x)}$ is maximal, and since $M \subset \princ{p(x)} \subseteq D[x]$, thus $\princ{p(x)} = D[x]$. But $p(x)$ is not a unit, so there is not a $k(x) \in \princ{p(x)}$ such that $p(x)k(x) = 1$. In particular, this means no element of $\princ{p(x)}$ are units, and so $\princ{p(x)} \neq D[x]$ (\myref{prop-ideal-contains-unit-iff-ideal-is-whole-ring}), the contradiction we wish to achieve. Therefore, at least one of $p(x)$ or $q(x)$ is a unit, meaning that $f(x)$ is irreducible.
\end{proof}

Knowing that $\princ{f(x)}$ is maximal gives us a way to create new fields.

\begin{corollary}\label{corollary-polynomial-quotient-by-principal-ideal-is-field-iff-polynomial-irreducible}
    Let $D$ be a PID and $f(x) \in D[x]$. Then $D[x]/\princ{f(x)}$ is a field if and only if $f(x)$ is irreducible over $D$.
\end{corollary}
\begin{proof}
    We note
    \begin{align*}
        &D[x]/\princ{f(x)} \text{ is a field}\\
        \iff&\princ{f(x)} \text{ is a maximal ideal} & (\myref{thrm-maximal-ideal-iff-quotient-ring-is-field})\\
        \iff&f(x) \text{ is irreducible} & (\myref{thrm-irreducible-iff-principal-ideal-is-maximal})
    \end{align*}
    which proves this corollary.
\end{proof}

We now look at a polynomial analog of Euclid's lemma (\myref{corollary-euclid}).

\begin{corollary}\label{corollary-irreducible-polynomial-division-rule}
    Let $D$ be a PID and let $p(x), a(x), b(x) \in D[x]$. If $p(x)$ is irreducible over $D$ and $p(x) \vert a(x)b(x)$ then $p(x) \vert a(x)$ or $p(x) \vert b(x)$.
\end{corollary}
\begin{proof}
    See \myref{exercise-irreducible-polynomial-division-rule} (later).
\end{proof}

\begin{exercise}\label{exercise-irreducible-polynomial-division-rule}
    Prove \myref{corollary-irreducible-polynomial-division-rule}.
\end{exercise}

With these results, we are able to construct finite fields with a prime-power number of elements. Suppose we desire a field with $p^k$ elements. If $k = 1$ then we know $\Z_p$ is a field of $p^k = p$ elements. Otherwise, by finding an irreducible polynomial of degree $k$ in $\Z_p[x]$, say $f(x)$, we can see that
\[
    \Z_p[x]/\princ{f(x)} = \left\{a_0 + a_1x + a_2x^2 + \cdots + a_{k-1}x^{k-1} + \princ{f(x)} \vert a_i \in \Z_p\right\}
\]
is a field of degree $p^k$ (\myref{corollary-polynomial-quotient-by-principal-ideal-is-field-iff-polynomial-irreducible}).

\begin{example}
    We construct a field with 8 elements. Writing $8 = 2^3$, we just need to find an irreducible degree 3 polynomial in $\Z_2$. One sees that $x^3 + x + 1$ is irreducible over $\Z_2$ since $x^3 + x + 1$ has no zeroes in $\Z_2$ (\myref{thrm-degree-2-or-3-irreducible-iff-has-no-zeroes}). Thus
    \[
        \Z_2[x]/\princ{x^3+x+1} = \left\{a_0 + a_1x + a_2x^2 + \princ{x^3+x+1} \vert a_i \in \Z_2\right\}
    \]
    is a field with 8 elements.

    Let's do some calculations in $\Z_2[x]/\princ{x^3+x+1}$. Addition is easy since the sum of two polynomials is just the maximum of the degree of the two polynomials. For example,
    \begin{align*}
        &\left(x^2 + x + 1 + \princ{x^3 + x + 1}\right) + \left(x^2 + 1 + \princ{x^3 + x + 1}\right)\\
        &= 2x^2 + x + 2 + \princ{x^3 + x + 1}\\
        &= 0x^2 + x + 0 + \princ{x^3 + x + 1}\\
        &= x + \princ{x^3 + x + 1}
    \end{align*}
    and
    \begin{align*}
        &\left(x^2 + 1 + \princ{x^3 + x + 1}\right) + \left(x + 1 + \princ{x^3 + x + 1}\right)\\
        &= x^2 + x + 2 + \princ{x^3 + x + 1}\\
        &= x^2 + x + \princ{x^3 + x + 1}.
    \end{align*}

    Multiplication is harder since the resultant coset may no longer fit the form of $a_0 + a_1x + a_2x^2 + \princ{x^3+x+1}$. For example,
    \begin{align*}
        &\left(x^2 + x + 1 + \princ{x^3 + x + 1}\right)\left(x^2 + 1 + \princ{x^3 + x + 1}\right)\\
        &= x^4+x^3+x+1 + \princ{x^3 + x + 1}\\
        &= x^4 + \princ{x^3 + x + 1}
    \end{align*}
    so we need to do some work to convert $x^4$ into a form to make it `compatible' with $\princ{x^3 + x + 1}$. We note that, by polynomial long division (\myref{thrm-polynomial-long-division}), we have
    \begin{align*}
        x^4 &= x(x^3+x+1) - x^2 - x\\
        &= x(x^3+x+1) + x^2 + x & (\text{In } \Z_2)
    \end{align*}
    so we see that
    \begin{align*}
        &x^4 + \princ{x^3 + x + 1}\\
        &= (x(x^3+x+1) + x^2 + x) + \princ{x^3 + x + 1}\\
        &= x^2 + x + \princ{x^3 + x + 1}.
    \end{align*}
    Similarly, one sees
    \begin{align*}
        &\left(x^2 + \princ{x^3 + x + 1}\right)\left(x + \princ{x^3 + x + 1}\right)\\
        &= x^3 + \princ{x^3 + x + 1}\\
        &= 1(x^3+x+1) - x - 1 + \princ{x^3 + x + 1}\\
        &= -x - 1 + \princ{x^3 + x + 1}\\
        &= x + 1 + \princ{x^3 + x + 1}.
    \end{align*}
\end{example}

\begin{example}\label{example-Z3/<x^2+1>}
    We construct a field of 9 elements. Writing $9 = 3^2$ we just need to find an irreducible degree 2 polynomial over $\Z_3$. Clearly $x^2 + 1$ is an irreducible polynomial over $\Z_3$ (by observing that $x^2 + 1$ has no zeroes in $\Z_3$) and so
    \[
        \Z_3[x]/\princ{x^2+1} = \left\{ax+b + \princ{x^2+1} \vert a,b \in \Z_3\right\}
    \]
    is a field with 9 elements.

    Let $F = \Z_3[x]/\princ{x^2+1}$. We produce the Cayley table for $(F, +)$ below.
    \begin{table}[H]
        \centering
        \fontsize{7pt}{10pt}\selectfont
        \begin{tabular}{|l|l|l|l|l|l|l|l|l|l|}
            \hline
            \textbf{+} & $\boldsymbol{0}$ & $\boldsymbol{1}$ & $\boldsymbol{2}$ & $\boldsymbol{x}$ & $\boldsymbol{x+1}$ & $\boldsymbol{x+2}$ & $\boldsymbol{2x}$ & $\boldsymbol{2x+1}$ & $\boldsymbol{2x+2}$ \\ \hline
            $\boldsymbol{0}$ & 0 & 1 & 2 & $x$ & $x+1$ & $x+2$ & $2x$ & $2x+1$ & $2x+2$ \\ \hline
            $\boldsymbol{1}$ & 1 & 2 & 0 & $x+1$ & $x+2$ & $x$ & $2x+1$ & $2x+2$ & $2x$ \\ \hline
            $\boldsymbol{2}$ & 2 & 0 & 1 & $x+2$ & $x$ & $x+1$ & $2x+2$ & $2x$ & $2x+1$ \\ \hline
            $\boldsymbol{x}$ & $x$ & $x+1$ & $x+2$ & $2x$ & $2x+1$ & $2x+2$ & 0 & 1 & 2 \\ \hline
            $\boldsymbol{x+1}$ & $x+1$ & $x+2$ & $x$ & $2x+1$ & $2x+2$ & $2x$ & 1 & 2 & 0 \\ \hline
            $\boldsymbol{x+2}$ & $x+2$ & $x$ & $x+1$ & $2x+2$ & $2x$ & $2x+1$ & 2 & 0 & 1 \\ \hline
            $\boldsymbol{2x}$ & $2x$ & $2x+1$ & $2x+2$ & 0 & 1 & 2 & $x$ & $x+1$ & $x+2$ \\ \hline
            $\boldsymbol{2x+1}$ & $2x+1$ & $2x+2$ & $2x$ & 1 & 2 & 0 & $x+1$ & $x+2$ & $x$ \\ \hline
            $\boldsymbol{2x+2}$ & $2x+2$ & $2x$ & $2x+1$ & 2 & 0 & 1 & $x+2$ & $x$ & $x+1$ \\ \hline
        \end{tabular}
    \end{table}

    Let $F^\ast = F \setminus \left\{0 + \princ{x^2+1}\right\}$. We provide a partial multiplication table for $(F^\ast, \times)$. Filling in the missing spots in the table is left for \myref{exercise-fill-in-missing-elements-in-table-of-Z3/<x^2+1>} (later).

    \begin{table}[H]
        \centering
        \fontsize{8pt}{11pt}\selectfont
        \begin{tabular}{|l|l|l|l|l|l|l|l|l|l|}
            \hline
            $\boldsymbol{\times}$ & $\boldsymbol{1}$ & $\boldsymbol{2}$ & $\boldsymbol{x}$ & $\boldsymbol{x+1}$ & $\boldsymbol{x+2}$ & $\boldsymbol{2x}$ & $\boldsymbol{2x+1}$ & $\boldsymbol{2x+2}$ \\ \hline
            $\boldsymbol{1}$ & 1 & 2 & $x$ & $x+1$ & $x+2$ & $2x$ & $2x+1$ & $2x+2$ \\ \hline
            $\boldsymbol{2}$ & 2 & 1 & $2x$ & $2x+2$ & $2x+1$ & $x$ & $x+2$ & $x+1$ \\ \hline
            $\boldsymbol{x}$ & $x$ & $2x$ & \textbf{(a)} & $x+2$ & $2x+2$ & 1 & $x+1$ & $2x+1$ \\ \hline
            $\boldsymbol{x+1}$ & $x+1$ & $2x+2$ & $x+2$ & $2x$ & 1 & $2x+1$ & \textbf{(b)} & $x$ \\ \hline
            $\boldsymbol{x+2}$ & $x+2$ & $2x+1$ & $2x+2$ & 1 & $x$ & $x+1$ & $2x$ & \textbf{(c)} \\ \hline
            $\boldsymbol{2x}$ & $2x$ & $x$ & 1 & $2x+1$ & $x+1$ & 2 & $2x+2$ & $2x$ \\ \hline
            $\boldsymbol{2x+1}$ & $2x+1$ & $x+2$ & $x+1$ & \textbf{(b)} & $2x$ & $2x+2$ & $x$ & \textbf{(d)} \\ \hline
            $\boldsymbol{2x+2}$ & $2x+2$ & $x+1$ & $2x+1$ & $x$ & \textbf{(c)} & $2x$ & \textbf{(d)} & $2x$ \\ \hline
        \end{tabular}
    \end{table}
\end{example}

\begin{exercise}
    Let $p(x) = x^2 + 2$, $f(x) = 2x+3$, and $g(x) = 4x^2+3x+2$ be polynomials in $\Z_5[x]$.
    \begin{partquestions}{\roman*}
        \item Show that $p(x)$ is irreducible over $\Z_5$.
        \item Hence construct a field, $F$, of 25 elements.
        \item Let $I = \princ{p(x)}$ be an ideal of $F$.  Simplify the following cosets.
        \begin{partquestions}{\alph*}
            \item $(f(x) + I) + (g(x) + I)$
            \item $(f(x) + I)(g(x) + I)$
        \end{partquestions}
    \end{partquestions}
\end{exercise}

\begin{exercise}\label{exercise-fill-in-missing-elements-in-table-of-Z3/<x^2+1>}
    Fill in the remaining of the multiplication table in \myref{example-Z3/<x^2+1>}.
\end{exercise}

\newpage

\section{Problems}
\begin{problem}
    Determine which of the following polynomials, if any, are irreducible over $\Q$.
    \begin{multicols}{2}
        \begin{partquestions}{\alph*}
            \item $f_1(x) = x^3 + x^2 + x + 1$
            \item $f_2(x) = 6x^3 + x^2 + x + 1$
            \item $f_3(x) = x^3 + 9$
            \item $f_4(x) = x^4 + 5$
            \item $f_5(x) = x^4 - 2x^3 + x^2 - x + 1$
            \item $f_6(x) = x^4 + 12x^3 + 54x^2 + 108x + 86$
        \end{partquestions}
    \end{multicols}
\end{problem}

\begin{problem}
    By considering the polynomial $x^2 - 2$, prove that $\sqrt2$ is irrational. You may assume $\sqrt2 \in \R$.
\end{problem}

\begin{problem}\label{problem-primitive-degree-1-polynomial-in-Z[x]-is-irreducible}
    Prove that any primitive degree 1 polynomial in $\Z[x]$ is irreducible over $\Z$.
\end{problem}

\begin{problem}
    Let $f(x) = x^3 + 6$.
    \begin{partquestions}{\roman*}
        \item Show that $f(x)$ is reducible over $\Z_7$.
        \item Write $f(x)$ as a product of irreducible polynomials over $\Z_7$.
    \end{partquestions}
\end{problem}

\begin{problem}
    Find all distinct irreducible polynomials of
    \begin{partquestions}{\alph*}
        \item degree 2
        \item degree 3
    \end{partquestions}
    in $\Z_2[x]$.
\end{problem}

\begin{problem}
    Let $p$ be a prime number. Show that the number of reducible polynomials over $\Z_p$ with the form $x^2 + ax + b$, where $a,b \in \Z_p$, is $\frac{p(p+1)}2$.
\end{problem}

\begin{problem}\label{problem-failure-case-of-mod-p-irreducibility-test}
    We show a failure case of the Mod $p$ Irreducibility Test (\myref{thrm-mod-p-irreducibility-test}). Let the polynomial $f(x) = x^4 + 1$.
    \begin{partquestions}{\alph*}
        \item Prove that $f(x)$ is irreducible over $\Q$.
        \item Show that $f(x)$ is reducible over $\Z_2$ via factorization.\newline
        For the rest of this part, assume $p > 2$.
        \begin{partquestions}{\roman*}
            \item Suppose there is an $r \in \Z_p$ satisfying $r^2 = k$ for some $k$ (to be specified). Deduce a factorization of $f(x)$ in the following cases.
            \begin{partquestions}{\alph*}
                \item $k = 2$
                \item $k = -1$
                \item $k = -2$
            \end{partquestions}
            \item Let $\Z_p^\ast = \Z_p \setminus \{0\}$. Explain why $\Z_p^\ast$ is a group under multiplication modulo $p$. In particular, explain why $\Z_p^\ast = \Un{p}$, the group of units modulo $p$.
            \item Explain why $\Un{p}$ is a cyclic group of even order.
            \item Hence, prove that one of the cases in \textbf{(b)(i)} must occur.
        \end{partquestions}
    \end{partquestions}
\end{problem}
