\section{Basics of Rings}
\begin{questions}
    \item We note the following.
    \begin{itemize}
        \item \textbf{Addition-Abelian}: $(\Z, +)$ is an abelian group.
        \item \textbf{Multiplication-Semigroup}: $(\Z, \times)$ is a semigroup since
        \begin{itemize}
            \item multiplying two integers always results in an integer, so $\Z$ is closed under $\times$; and
            \item $\times$ is associative.
        \end{itemize}
        \item \textbf{Distributive}: We know $+$ and $\times$ distribute.
    \end{itemize}
    Hence $(\Z, +, \times)$ is a ring.

    \item Consider $(-a)(-b) + (-ab)$ and note
    \begin{align*}
        &(-a)(-b) + (-ab)\\
        &= (-a)(-b) + (-a)b & (\text{\myref{prop-product-of-element-and-additive-inverse-is-additive-inverse-of-product}})\\
        &= (-a)(-b + b) & (\text{by \textbf{Distributive} axiom})\\
        &= (-a)0\\
        &= 0 & (\myref{prop-multiplying-by-zero-is-zero})
    \end{align*}
    which means $(-a)(-b) = -(-ab) = ab$ as required.

    \item The ring $\Mn{2}{\mathbb{R}}$ indeed has zero divisors, as $\begin{pmatrix}0&1\\0&0\end{pmatrix} \neq \begin{pmatrix}0&0\\0&0\end{pmatrix}$ but $\begin{pmatrix}0&1\\0&0\end{pmatrix}^2 = \begin{pmatrix}0&0\\0&0\end{pmatrix}$ which means that $\begin{pmatrix}0&1\\0&0\end{pmatrix}$ is a zero divisor.

    \item \begin{partquestions}{\alph*}
        \item $\Z$ is not a field. Note that the multiplicative inverse of 2 is $\frac12$ which is not an integer. Hence not all non-zero elements in $\Z$ has a multiplicative inverse, meaning that not all non-zero elements are units.

        \item $\Q$ is a field. Note for any rational number $\frac ab$ (where $b \neq 0$) it has an inverse of $\frac ba$. Thus any non-zero rational number is a unit, which means $\Q$ is a division ring. Since $\Q$ is also a commutative ring, therefore $\Q$ is a field.
    \end{partquestions}

    \item Let $u$ and $v$ be units, meaning that $u^{-1}$ and $v^{-1}$ exist. Then one sees that $(uv)(v^{-1}u^{-1}) = (v^{-1}u^{-1})(uv) = 1$, which means that $uv$ is also a unit.

    \item We first show that $(R, +) \leq (\Mn{2}{\R}, +)$.
    \begin{itemize}
        \item Clearly the identity of $(\Mn{2}{\R}, +)$, the zero matrix $\begin{pmatrix}0&0\\0&0\end{pmatrix}$, is inside $R$.
        \item Consider $\begin{pmatrix}a&a\\a&a\end{pmatrix}, \begin{pmatrix}b&b\\b&b\end{pmatrix} \in R$. The additive inverse of the matrix $\begin{pmatrix}b&b\\b&b\end{pmatrix}$ is the matrix $\begin{pmatrix}-b&-b\\-b&-b\end{pmatrix}$, and so their sum is
        \[
            \begin{pmatrix}a&a\\a&a\end{pmatrix} + \begin{pmatrix}-b&-b\\-b&-b\end{pmatrix} = \begin{pmatrix}a-b&a-b\\a-b&a-b\end{pmatrix} \in R
        \]
        which means $R$ is closed under addition.
    \end{itemize}
    Hence $(R, +) \leq (\Mn{2}{\R}, +)$ by subgroup test.

    We now show that $R$ is closed under multiplication. Some calculation yields that
    \[
        \begin{pmatrix}a&a\\a&a\end{pmatrix}\begin{pmatrix}b&b\\b&b\end{pmatrix} = \begin{pmatrix}2ab&2ab\\2ab&2ab\end{pmatrix}
    \]
    which is clearly in $R$. Therefore $R$ is a subring of $\Mn{2}{\R}$.
\end{questions}
