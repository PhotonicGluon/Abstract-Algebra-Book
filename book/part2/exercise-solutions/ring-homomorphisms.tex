\section{Ring Homomorphisms and Isomorphisms}
\begin{questions}
    \item We show that $\phi$ is not a ring homomorphism. Consider the matrices
    \[
        \begin{pmatrix}1&0\\0&0\end{pmatrix} \text{ and } \begin{pmatrix}0&0\\0&1\end{pmatrix}.
    \]
    Note that $\phi\left(\begin{pmatrix}1&0\\0&0\end{pmatrix}\right) = 1 + 0 = 1$ and $\phi\left(\begin{pmatrix}0&0\\0&1\end{pmatrix}\right) = 0 + 1 = 1$, so
    \[
        \phi\left(\begin{pmatrix}1&0\\0&0\end{pmatrix}\right)\phi\left(\begin{pmatrix}0&0\\0&1\end{pmatrix}\right) = 1 \times 1 = 1.
    \]
    However, note
    \[
        \begin{pmatrix}1&0\\0&0\end{pmatrix}\begin{pmatrix}0&0\\0&1\end{pmatrix} = \begin{pmatrix}0&0\\0&0\end{pmatrix}
    \]
    so $\phi\left(\begin{pmatrix}1&0\\0&0\end{pmatrix}\begin{pmatrix}0&0\\0&1\end{pmatrix}\right) = 0$. Thus $\phi$ is not a homomorphism.

    \item Note
    \[
        \phi(a+b) = 0 = 0 + 0 = \phi(a) + \phi(b)
    \]
    and
    \[
        \phi(ab) = 0 = 0\times0 = \phi(a)\phi(b)
    \]
    so $\phi$ is indeed a ring homomorphism.

    \item Note
    \[
        \id(a+b) = a + b = \id(a) + \id(b)
    \]
    and
    \[
        \id(ab) = ab = \id(a)\id(b)
    \]
    so $\id$ is a ring endomorphism.

    \item We have shown that the identity homomorphism is a homomorphism, so we just need to prove that it is a bijection.
    \begin{itemize}
        \item \textbf{Injective}: Suppose $a, b \in R$ are such that $\phi(a) = \phi(b)$. But since $\phi(x) = x$ thus $a = b$.
        \item \textbf{Surjective}: As $\phi(x) = x$ thus any element is its own pre-image.
    \end{itemize}
    Therefore the identity homomorphism is an isomorphism. As it is also an endomorphism, therefore it is an automorphism.

    \item Note that
    \[
        \phi(0_1) = \phi(0_1 + 0_1) = \phi(0_1) + \phi(0_1)
    \]
    so by 'adding' $-\phi(0_1)$ on both sides we see that $\phi(0_1) = 0_2$.

    \item Note that
    \[
        \phi(1_1) = \phi(1_1 \times 1_1) = \phi(1_1)\phi(1_1).
    \]
    Since $R_1$ and $R_2$ are division rings, we may apply $\phi(1_1)^{-1}$ on both sides to yield $\phi(1_1) = 1_2$.

    \item \begin{partquestions}{\alph*}
        \item Notice that
        \[
            \phi(x + (-x)) = \phi(x) + \phi(-x)
        \]
        and
        \[
            \phi(x + (-x)) = \phi(0_1) = 0_2
        \]
        so subtracting $-\phi(x)$ on both sides yields $\phi(-x) = -\phi(x)$.

        \item Notice that
        \[
            \phi(xx^{-1}) = \phi(x)\phi(x^{-1})
        \]
        and
        \[
            \phi(xx^{-1}) = \phi(1_1) = 1_2
        \]
        so applying $\phi(x)^{-1}$ on the left on both sides $\phi(x^{-1}) = \phi(x)^{-1}$.
    \end{partquestions}

    \item Recall that $\{0_2\}$, the trivial ideal, is an ideal of $R_2$, where $0_2$ is the additive identity of $R_2$. Therefore $\ker\phi = \phi^{-1}(\{0\})$ is an ideal of $R_1$ by \myref{prop-inverse-homomorphism-on-ideal-is-ideal}.

    \item We show that $\phi$ is surjective. Note that for any $k \in \Z_n$, we have $k \leq n$. Thus, $\phi(k) = k$, so $\phi$ is surjective.

    We now find the kernel of $\phi$.
    \begin{align*}
        \ker\phi &= \{m \in \Z \vert \phi(m) = 0\}\\
        &= \{m \in \Z \vert m \cong 0 \pmod n\}\\
        &= \{kn \vert k \in \Z\}\\
        &= n\Z.
    \end{align*}

    The FRIT (\myref{thrm-ring-isomorphism-1}) on $\phi$ tells us that
    \[
        \Z/n\Z \cong \Z_n.
    \]

    \item Note that $\phi(1) = 1$ is given. Now suppose $\phi(k) = k$ for some positive integer $k$. Then
    \[
        \phi(k+1) = \phi(k) + \phi(1) = k + 1
    \]
    by induction hypothesis and by the base case. Thus by mathematical induction we prove the statement.

    \item We borrow some calculation in the case where $\phi(1) = 1$ from \myref{example-endomorphisms-of-Z} to yield $\phi(n) = n$ for all integers $n$. Now note that for any positive integer $n$ we have
    \begin{align*}
        1 = \phi(1) &= \phi\left(\underbrace{\frac1n + \frac1n + \cdots + \frac1n}_{n \text{ times}}\right)\\
        &= \underbrace{\phi\left(\frac1n\right) + \phi\left(\frac1n\right) + \cdots + \phi\left(\frac1n\right)}_{n \text{ times}}\\
        &= n\phi\left(\frac1n\right)
    \end{align*}
    which means $\phi\left(\frac1n\right) = \frac1n$.

    Note that for any positive $\frac mn \in \Q$ with $m$ and $n$ as positive integers, we have
    \[
        \phi\left(\frac mn\right) = \phi(m)\phi\left(\frac1n\right) = m \times \frac1n = \frac mn
    \]
    and
    \[
        \phi\left(-\frac mn\right) = \phi(-m)\phi\left(\frac1n\right) = (-m) \times \frac1n = -\frac mn
    \]
    so $\phi(q) = q$ for any $q \in \Q$.
\end{questions}
