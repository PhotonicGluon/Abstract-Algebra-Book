\section{Integral Domains}
\begin{questions}
    \item Clearly multiplication is commutative and $1 = 1 + 0\sqrt2 \in \Z[\sqrt2]$. All that is needed is to show that there are no zero divisors in $\Z[\sqrt2]$.

    Take $a+b\sqrt2, c+d\sqrt2 \in \Z[\sqrt2]$ such that $a+b\sqrt2 \neq 0$ and $(a+b\sqrt2)(c+d\sqrt2) = 0$. We want to show that the only way this is possible is if $c = d = 0$. Now consider
    \[
        \left((a+b\sqrt2)\underbrace{(a-b\sqrt2)}_{\neq 0}\right)\left((c+d\sqrt2)\underbrace{(c-d\sqrt2)}_{\neq 0}\right) = 0.
    \]
    This simplifies to $(a^2-2b^2)(c^2-2d^2) = 0$. Hence either $a^2-2b^2 = 0$ or $c^2-2d^2 = 0$, implying $a = \sqrt2b$ or $c = \sqrt2d$. Now we cannot have $a = \sqrt2b$ as $\sqrt2$ is not an integer; the only case that is possible is if $c = \sqrt2d$ which finally means that $c = d = 0$.

    \item Let $n = ab$ where $a,b \in Z$ and, without loss of generality, assume $1 < a \leq b < n$ (we exclude 1 and $n$ because we want $a$ and $b$ to be `proper' factors). Now clearly $a, b \in \Z_n$ with them both being non-zero but $ab = n = 0$ in $\Z_n$. Hence $a$ and $b$ are zero divisors in $\Z_n$, meaning $\Z_n$ is not an integral domain.

    \item We show that that $\Z_2[\alpha]$ is indeed a field. We note that multiplication is commutative with identity 1. The multiplication table in $\Z_2[\alpha]$ is provided below.
    \begin{table}[H]
        \centering
        \resizebox{\textwidth}{!}{
            \begin{tabular}{|l|l|l|l|}
                \hline
                $\boldsymbol{\times}$   & \textbf{1} & $\boldsymbol{\alpha}$                 & $\boldsymbol{1+\alpha}$                     \\ \hline
                \textbf{1}          & 1          & $\alpha$                          & $1+\alpha$                              \\ \hline
                $\boldsymbol{\alpha}$   & $\alpha$   & $\alpha^2 = 1+\alpha$             & $\alpha+\alpha^2 = 1+2\alpha = 1$       \\ \hline
                $\boldsymbol{1+\alpha}$ & $1+\alpha$ & $\alpha+\alpha^2 = 1+2\alpha = 1$ & $1+2\alpha+\alpha^2 = 2+3\alpha=\alpha$ \\ \hline
            \end{tabular}
        }
    \end{table}

    What we see from this table is that no non-zero elements multiply together to form zero, meaning that there are no zero divisors in $\Z_2[\alpha]$. Therefore $\Z_2[\alpha]$ is an integral field. Furthermore as $\Z_2[\alpha]$ is finite thus $\Z_2[\alpha]$ is a field by \myref{thrm-finite-integral-domain-is-field}.

    \item The trivial ring $\{0\}$ is not an integral domain. Seeking a contradiction, if $\{0\}$ is indeed an integral domain, then by \myref{prop-zero-of-prime-characteristic-if-integral-domain} it has to have either 0 or prime characteristic. However, one sees clearly that the characteristic of the trivial ring is 1, which is neither 0 nor prime. Therefore $\{0\}$ is not an integral domain.

    \item The additive identity in $\Z_2[\alpha]$ is 1. Clearly $1 + 1 = 0$, so the order of 1 in $(\Z_2[\alpha],+)$ is 2. Now \myref{exercise-Zn2[alpha]} tells us that $\Z_2[\alpha]$ is an integral domain, so by the previous proposition this means that $\Char{\Z_2[\alpha]} = 2$.
\end{questions}
