\chapter{Basics of Rings}
With an intuition and definition of rings out of the way, we are now ready to tackle the basics of rings in this chapter.

\section{Obvious Rings}
Before we introduce some examples of rings, we note the convention on notation used in Ring Theory.
\begin{itemize}
    \item The multiplication symbol $\cdot$ is usually omitted, so $x \cdot y$ is written as $xy$.
    \item The additive identity of $R$ will always be denoted by 0 and the multiplicative identity of $R$ (if it exists) will always be denoted by 1.
    \item The additive inverse of the element $x$ will always be denoted by $-x$ and the multiplicative inverse of $x$ (if it exists) will always be denoted by $x^{-1}$.
    \item $n$ applications of $+$ on an element $x$ will always be denoted $nx$ (and will be denoted $-nx$ if the element is $-x$), while $n$ applications of $\cdot$ on an element $x$ will always be denoted $x^n$ (and will be denoted $x^{-n}$ if the element is $x^{-1}$ and if $x^{-1}$ exists).
    \item $a - b$ means $a + (-b)$ for any $a, b \in R$.
\end{itemize}

Let's look at some examples of rings.
\begin{definition}
    The \term{ring of integers}\index{ring!of integers} is the set $\Z$ together with integer addition and multiplication.
\end{definition}
\begin{remark}
    We denote the ring of integers by $\Z$.
\end{remark}
\begin{proposition}
    $\Z$ is a commutative ring with identity.
\end{proposition}
\begin{proof}
    \myref{exercise-ring-of-integers-is-a-ring} (later) shows that $\Z$ is a ring. In addition, multiplication is commutative (\myref{axiom-multiplication-is-commutative}), and 1 is the multiplicative identity. Thus $\Z$ is a commutative ring with identity.
\end{proof}

\begin{definition}
    Let the integer $n \geq 2$. The \term{ring of integers modulo $n$}\index{ring!of integers!modulo $n$} is $(\Z_n, \oplus_n, \otimes_n)$, where $\oplus_n$ and $\otimes_n$ denote addition and multiplication modulo $n$ respectively.
\end{definition}
\begin{remark}
    We denote the ring of integers modulo $n$ by $\Z_n$.
\end{remark}
\begin{proposition}
    $\Z_n$ is a commutative ring with identity.
\end{proposition}
\begin{proof}
    We first prove the ring axioms before showing that it is commutative with a multiplicative identity.
    \begin{itemize}
        \item \textbf{Addition-Abelian}: We know that $(\Z_n, \oplus_n)$ is an abelian group by \myref{prop-Zn-is-abelian-group}.
        \item \textbf{Multiplication-Semigroup}: We can see that $(\Z_n, \otimes_n)$ is a semigroup as
        \begin{itemize}
            \item $\Z_n$ is closed under $\otimes_n$ because $a \otimes_n b \in \{0, 1, 2, \dots, n-1\} = \Z_n$; and
            \item multiplication is associative (\myref{axiom-multiplication-is-associative}), so multiplication modulo $n$ is associative.
        \end{itemize}
        \item \textbf{Distributive}: Since \myref{axiom-distributivity} tells us that multiplication distributes over addition, thus multiplication modulo $n$ (i.e. $\otimes_n$) distributes over addition modulo $n$ (i.e. $\oplus_n$).
    \end{itemize}
    Hence $(\Z_n, \oplus_n, \otimes_n)$ is a ring.

    Furthermore, multiplication is commutative (\myref{axiom-multiplication-is-commutative}), so $\otimes_n$ is commutative. Also $\otimes_n$ has an identity of 1. Therefore $(\Z_n, \oplus_n, \otimes_n)$ is a commutative ring with identity.
\end{proof}

\begin{definition}
    The \term{ring of rational numbers}\index{ring!of rational numbers} is $(\Q, +, \times)$, where $+$ and $\times$ denote normal addition and multiplication.
\end{definition}
\begin{remark}
    We denote the ring of rational numbers by $\Q$.
\end{remark}
\begin{proposition}
    $\Q$ is a commutative ring with identity.
\end{proposition}
\begin{proof}
    We first show that $\Q$ satisfies the ring axioms.
    \begin{itemize}
        \item \textbf{Addition-Abelian}: We know that $(\Q, +)$ is an abelian group from \myref{problem-Q-is-abelian-group-under-addition}.
        \item \textbf{Multiplication-Semigroup}: We note that $(\Q, \times)$ is a semigroup as
        \begin{itemize}
            \item $\Q$ is closed under $\times$ because multiplying two rational numbers together produce a rational number; and
            \item multiplication is associative (\myref{axiom-multiplication-is-associative}).
        \end{itemize}
        \item \textbf{Distributive}: Multiplication distributes over addition by \myref{axiom-distributivity}.
    \end{itemize}
    Hence $\Q$ is a ring. Furthermore, $\times$ has an identity of 1 and is commutative (\myref{axiom-multiplication-is-commutative}). So $\Q$ is a commutative ring with identity.
\end{proof}

\begin{definition}
    The \term{ring of real numbers}\index{ring!of real numbers} is the ring $(\R, +, \times)$ where $+$ and $\times$ denotes regular addition and multiplication respectively.
\end{definition}
\begin{remark}
    We denote the ring of real numbers by $\R$.
\end{remark}
\begin{proposition}
    $\R$ is a commutative ring with identity.
\end{proposition}
\begin{proof}
    Replace $(\Q, +)$ with $(\R, +)$ and $(\Q, \times)$ with $(\R, \times)$ in the previous proof.
\end{proof}

We end this section by looking at the ring of complex numbers.

\begin{definition}
    Let the set of \term{complex numbers}\index{complex numbers}
    \[
        \C = \{a + bi \vert a, b \in \R\}
    \]
    where $i$ is known as the \term{imaginary unit}\index{imaginary unit}, where $i^2 = -1$. Define complex addition and multiplication by
    \begin{align*}
        (a+bi) + (c+di) &= (a+c) + (b+d)i \text{ and}\\
        (a+bi) \cdot (c+di) &= (ac-bd) + (ad+bc)i.
    \end{align*}
    respectively. Then $\C$ under complex addition and multiplication is the \term{ring of complex numbers}\index{ring!of complex numbers}.
\end{definition}
\begin{remark}
    We denote the ring of complex numbers by $\C$.
\end{remark}
\begin{proposition}
    $\C$ is a commutative ring with identity.
\end{proposition}
\begin{proof}
    We first show that $\C$ satisfies the ring axioms.
    \begin{itemize}
        \item \textbf{Addition-Abelian}: We show that $(\C, +)$ satisfies the group axioms, and then show that $(\C, +)$ is commutative.
        \begin{itemize}
            \item \textbf{Closure}: Clearly for all real numbers $a, b, c, d \in \R$ we have $a + c \in \R$ and $b+d \in \R$. Thus $(a+bi) + (c+di) = (a+c) + (b+d)i \in \C$, meaning $\C$ is closed under complex addition.

            \item \textbf{Associativity}: Let $a+bi, c+di, e+fi \in \C$. Recall that $+$ is associative by \myref{axiom-addition-is-associative}. Then note that
            \begin{align*}
                &(a+bi) + ((c+di) + (e+fi))\\
                &= (a+bi) + ((c+e) + (d+f)i)\\
                &= (a+(c+e)) + (b+(d+f))i\\
                &= ((a+c)+e) + ((b+d)+f)i & (\myref{axiom-addition-is-associative})\\
                &= ((a+c) + (b+d)i) + (e+fi)\\
                &= ((a+bi) + (c+di)) + (e+fi)
            \end{align*}
            so complex addition is associative.

            \item \textbf{Identity}: The identity in $\C$ is $0 + 0i = 0$ since
            \[
                (0+0i) + (a+bi) = (0+a) + (0+b)i = a+bi
            \]
            and complex addition is commutative (to be proved later), so $(\C,+)$ has an additive identity.

            \item \textbf{Inverse}: Let $a+bi \in \C$. Clearly $-a, -b \in \R$ and that
            \[
                (a+bi) + (-a+(-b)i)= (a+(-a)) + (b+(-b))i = 0.
            \]
            Also, complex addition is commutative (to be proved later). So any $a+bi \in \C$ has an additive inverse of $-a-bi \in \C$.

            \item \textbf{Commutative}: Let $a+bi, c+di \in \C$. Recall that $+$ is commutative by \myref{axiom-addition-is-commutative}. Then
            \begin{align*}
                &(a+bi) + (c+di)\\
                &= (a+c) + (b+d)i\\
                &= (c+a) + (d+b)i & (\myref{axiom-addition-is-commutative})\\
                &= (c+di) + (a+bi)
            \end{align*}
            so complex addition is commutative.
        \end{itemize}

        \item \textbf{Multiplication-Semigroup}: We show that $(\C, \times)$ is a semigroup.
        \begin{itemize}
            \item \textbf{Closure}: Clearly for all real numbers $a, b, c, d \in \R$ we have $ac, bd, ad, bc \in \R$, so $ac - bd, ad + bc \in \R$. Therefore
            \[
                (a+bi)(c+di) = (ac-bd) + (ad+bc)i \in \C
            \]
            which means $\C$ is closed under multiplication.

            \item \textbf{Associativity}: Let $a+bi, c+di, e+fi \in \C$. Note that
            \begin{align*}
                &(a+bi)((c+di)(e+fi))\\
                &= (a+bi)((ce-df)+(cf+de)i)\\
                &= (a(ce-df) - b(cf+de)) + (a(cf+de) + b(ce-df))i\\
                &= (ace - adf - bcf - bde) + (acf + ade + bce - bdf)i\\
                &= (ace - bde - adf - bcf) + (acf - bdf + ade + bce)i\\
                &= ((ac-bd)e - (ad+bc)f) + ((ac-bd)f + (ad+bc)e)i\\
                &= ((ac-bd)+(ad+bc)i)(e+fi)\\
                &= ((a+bi)(c+di))(e+fi)
            \end{align*}
            so complex multiplication is associative.
        \end{itemize}

        \item \textbf{Distributive}: We only prove left distributivity because complex multiplication is commutative (to be shown later). Let $a+bi, c+di, e+fi \in \C$. Note that
        \begin{align*}
            &(a+bi)((c+di) + (e+fi))\\
            &= (a+bi)((c+e) + (d+f)i)\\
            &= (a(c+e)-b(d+f)) + (a(d+f) + b(c+e))i\\
            &= (ac+ae-bd-bf) + (ad+af+bc+be)i\\
            &= (ac-bd+ae-bf) + (ad+bc+af+be)i\\
            &= ((ac-bd) + (ad+bc)i) + ((ae - bf) + (af + be)i)\\
            &= (a+bi)(c+di) + (a+bi)(e+fi)
        \end{align*}
        so complex multiplication distributes over complex addition.
    \end{itemize}
    Hence $\C$ is a ring.

    We now show that complex multiplication is commutative. Let $a+bi, c+di \in \C$. Recall that $\times$ is commutative by \myref{axiom-multiplication-is-commutative}. Then we see
    \begin{align*}
        &(a+bi)(c+di)\\
        &= (ac-bd) + (ad+bc)i\\
        &= (ca-db) + (da+cb)i & (\myref{axiom-multiplication-is-commutative})\\
        &= (c+di)(a+bi)
    \end{align*}
    so complex multiplication is commutative.

    Finally we show that complex multiplication has an identity. Consider $1 + 0i \in \C$. Note that
    \[
        (1+0i)(a+bi) = (1a-0b) + (1b+0a)i = a+bi,
    \]
    and since complex multiplication is commutative, therefore $1+0i$ is the multiplicative identity in $\C$.

    Therefore $\C$ is a commutative ring with identity.
\end{proof}

These are just some examples of rings; we explore more later in this chapter.
\begin{exercise}\label{exercise-ring-of-integers-is-a-ring}
    Prove that $\Z$ is a ring under regular addition and multiplication. You do not need to prove the \textbf{Distributive} axiom.
\end{exercise}

\section{General Properties of Rings}
We list some properties of rings here. For each of the propositions, let $R$ be a ring.

\begin{proposition}[Multiplication by Zero]\label{prop-multiplying-by-zero-is-zero}
    $0x = x0 = 0$ for all $x \in R$.
\end{proposition}
\begin{proof}
    We note that
    \begin{align*}
        0x &= (0 + 0)x & (0 \text{ is additive inverse})\\
        &= 0x + 0x & (\text{by \textbf{Distributive} axiom})
    \end{align*}
    so by `subtracting' $0x$ on both sides (i.e., adding $-0x$ on both sides) we see $0 = 0x$.

    Also
    \begin{align*}
        x0 &= x(0 + 0) & (0 \text{ is additive inverse})\\
        &= x0 + x0 & (\text{by \textbf{Distributive} axiom})
    \end{align*}
    so by `subtracting' $x0$ on both sides we see $0 = x0$.

    Therefore $0x = x0 = 0$ for all $x \in R$.
\end{proof}

\begin{proposition}\label{prop-product-of-element-and-additive-inverse-is-additive-inverse-of-product}
    $(-a)b = a(-b) = -(ab)$ for any $a, b \in R$.
\end{proposition}
\begin{proof}
    We show that $(-a)b = -(ab)$ and $a(-b) = -(ab)$ to complete the proof.
    \begin{itemize}
        \item Note $(-a)b + ab = (-a + a)b = 0b = 0$ by \textbf{Distributive} axiom. Hence by subtracting $ab$ on both sides we see $(-a)b = -(ab)$.
        \item Note also $a(-b) + ab = a(-b + b) = a0 = 0$ by \textbf{Distributive} axiom. Hence by subtracting $ab$ on both sides we see $a(-b) = -(ab)$.
    \end{itemize}
    So $(-a)b = a(-b) = -(ab)$ as required.
\end{proof}

\begin{proposition}\label{prop-product-of-additive-inverses}
    $(-a)(-b) = ab$ for any $a, b \in R$.
\end{proposition}
\begin{proof}
    See \myref{exercise-product-of-additive-inverses} (later).
\end{proof}

\begin{proposition}
    If $R$ has an identity, it is unique.
\end{proposition}
\begin{proof}
    Suppose 1 and $1'$ are identities. Consider the sum $1 + 1'$. Then
    \begin{align*}
        1 + 1' &= 1\times(1+1') & (\text{multiplying by identity }1)\\
        &= 1\times1 + 1\times1' & (\text{by \textbf{Distributive} axiom})\\
        &= 1 + 1. & (1 \text{ and } 1' \text{ are identities})
    \end{align*}
    Subtracting 1 on both sides yields $1 = 1'$, so the identity is unique.
\end{proof}

\begin{exercise}\label{exercise-product-of-additive-inverses}
    Prove \myref{prop-product-of-additive-inverses}.
\end{exercise}

\section{Matrix Rings}
The rings that we explored in previous sections can be thought of as the `obvious' rings, since they are number systems. However, there are less obvious rings. We looked at matrices in the context of the General/Special Linear Group of matrices (\myref{section-groups-of-matrices}). Here we see that matrices in fact form rings, known as matrix rings. Before that though, we need to define the operations within that ring.

\begin{definition}[Matrix Addition]\index{matrix addition}
    For any two matrices $\mathbf{A}$ and $\mathbf{B}$ with $n$ rows and columns and with entries in the ring $(R, \oplus, \otimes)$, their sum is the matrix $\mathbf{C} = \mathbf{A} + \mathbf{B}$ with $n$ rows and columns such that
    \[
        c_{i,j} = a_{i,j} \oplus b_{i,j}
    \]
    for all $i,j \in \{1, 2, \dots, n\}$.
\end{definition}
\begin{definition}[Matrix Multiplication]\index{matrix multiplication}
    For any two matrices $\mathbf{A}$ and $\mathbf{B}$ with $n$ rows and columns and with entries in the ring $(R, \oplus, \otimes)$, their product is the matrix $\mathbf{C} = \textbf{AB}$ with $n$ rows and columns such that, for all $i,j \in \{1, 2, \dots, n\}$, we have
    \begin{align*}
        c_{i,j} &= (a_{i,1}\otimes b_{1,j}) \oplus (a_{i,2}\otimes b_{2,j}) \oplus \cdots \oplus (a_{i,n}\otimes b_{n,j})\\
        &= \bigoplus_{k=1}^n (a_{i,k}\otimes b_{k,j}).
    \end{align*}
\end{definition}

We also define two matrices that are useful when we work with matrix rings.
\begin{definition}
    The \term{zero matrix}\index{zero matrix} with $n$ rows and columns is
    \[
        \ZeroM{n} =
        \begin{pmatrix}
            0 & 0 & 0 & \cdots & 0 \\
            0 & 0 & 0 & \cdots & 0 \\
            0 & 0 & 0 & \cdots & 0 \\
            \vdots & \vdots & \vdots & \ddots & \vdots \\
            0 & 0 & 0 & \cdots & 0 \\
        \end{pmatrix}
    \]
    where 0 is the additive identity (i.e. zero) of the ring $(R, \oplus, \otimes)$.
\end{definition}
\begin{definition}
    The \term{identity matrix}\index{identity matrix} with $n$ rows and columns is
    \[
        \IdentityM{n} =
        \begin{pmatrix}
            1 & 0 & 0 & \cdots & 0 \\
            0 & 1 & 0 & \cdots & 0 \\
            0 & 0 & 1 & \cdots & 0 \\
            \vdots & \vdots & \vdots & \ddots & \vdots \\
            0 & 0 & 0 & \cdots & 1 \\
        \end{pmatrix}
    \]
    where 0 and 1 are the additive and multiplicative identities (i.e. zero and one) of the ring $(R, \oplus, \otimes)$ respectively. That is, the identity matrix is the matrix with 1s in the leading diagonal.
\end{definition}

We can now define what is a matrix ring.
\begin{definition}
    Let $(R, \oplus, \otimes)$ be a ring and $n$ be a positive integer. Then $\Mn{n}{R}$ under matrix addition and multiplication is a ring, known as the \term{matrix ring}\index{matrix ring} with elements in $(R, \oplus, \otimes)$.
\end{definition}
\begin{proposition}
    $\Mn{n}{R}$ is a ring with identity.
\end{proposition}
\begin{proof}
    We need to prove that the ring axioms hold.
    \begin{itemize}
        \item \textbf{Addition-Abelian}: We first prove that $(\Mn{n}{R}, +)$ is indeed an abelian group.
        \begin{itemize}
            \item \textbf{Closure}: Clearly the sum of any two matrices in $\Mn{n}{R}$ is also a square matrix with $n$ rows with elements inside $R$, meaning that $\Mn{n}{R}$ is closed under matrix addition.

            \item \textbf{Associativity}: Let $\mathbf{A}, \mathbf{B}, \mathbf{C} \in \Mn{n}{R}$. Let $\mathbf{P} = \mathbf{A} + (\mathbf{B} + \mathbf{C})$ and $\mathbf{Q} = (\mathbf{A} + \mathbf{B}) + \mathbf{C}$. We note that $\mathbf{P} = \mathbf{Q}$ as
            \[
                p_{i,j} = a_{i,j} \oplus (b_{i,j} \oplus c_{i,j}) = (a_{i,j} \oplus b_{i,j}) \oplus c_{i,j} = q_{i,j}
            \]
            by associativity of $\oplus$, which proves that matrix addition is associative.

            \item \textbf{Identity}: We show that $\ZeroM{n}$ is the additive identity in $\Mn{n}{R}$. Let $\mathbf{M} \in \Mn{n}{R}$; let $\mathbf{N} = \mathbf{M} + \ZeroM{n}$. Note that $n_{i,j} = m_{i,j} \oplus 0 = m_{i,j}$ so $\mathbf{M} + \ZeroM{n} = \mathbf{M}$. Therefore $\mathbf{M} + \ZeroM{n} = \mathbf{M}$ for any matrix in $\Mn{n}{R}$.

            \item \textbf{Inverse}: Let $\mathbf{A} \in \Mn{n}{R}$. Define the matrix $\mathbf{B} = -\mathbf{A}$ such that $b_{i,j} = -a_{i,j}$, i.e. $b_{i,j}$ is the additive inverse of $a_{i,j}$ in the ring $R$. Then one sees that $\mathbf{A} + \mathbf{B} = \ZeroM{n}$. (We denote the additive inverse of a matrix $\mathbf{M}$ by $-\mathbf{M}$).

            \item \textbf{Commutative}: Let $\mathbf{A}, \mathbf{B} \in \Mn{n}{R}$. Set $\mathbf{C} = \mathbf{A} + \mathbf{B}$ and $\mathbf{D} = \mathbf{B} + \mathbf{A}$. Consider $c_{i,j} = a_{i,j} \oplus b_{i,j}$. Since $\oplus$ is commutative, thus $a_{i,j} \oplus b_{i,j} = b_{i,j} \oplus a_{i,j}$. But $d_{i,j} = b_{i,j} \oplus a_{i,j}$, so we have $c_{i,j} = d_{i,j}$. Therefore $\mathbf{C} = \mathbf{D}$.
        \end{itemize}

        \item \textbf{Multiplication-Semigroup}: We show that $(\Mn{n}{R}, \cdot)$ is a semigroup.
        \begin{itemize}
            \item \textbf{Closure}: In \myref{subsection-intro-to-matrices} we showed that matrix multiplication produces another $n \times n$ matrix. Furthermore the entries of the new matrix are elements of $R$. Hence $\Mn{n}{R}$ is closed under matrix multiplication.

            \item \textbf{Associativity}: We proved that matrix multiplication is associative in \myref{subsection-GLR-matrix-group}.
        \end{itemize}

        \item \textbf{Distributive}: We prove only $\mathbf{A}(\mathbf{B} + \mathbf{C}) = (\textbf{AB}) + (\textbf{AC})$ as the other case is proven similarly. Let $\mathbf{R} = \mathbf{A}(\mathbf{B} + \mathbf{C})$, $\mathbf{G} = \textbf{AB}$, and $\mathbf{H} = \textbf{AC}$. We note
        \begin{align*}
            r_{i,j} &= \bigoplus_{k=1}^n \left(a_{i,k} \otimes \left(b_{k,j} \oplus c_{k,j}\right)\right)\\
            &= \bigoplus_{k=1}^n \left((a_{i,k} \otimes b_{k,j}) \oplus (a_{i,k} \otimes c_{k,j})\right) & (\textbf{Distributive}\text{ in }R)\\
            &= \left(\bigoplus_{k=1}^n (a_{i,k} \otimes b_{k,j})\right) \oplus \left(\bigoplus_{k=1}^n (a_{i,k} \otimes c_{k,j})\right)\\
            &= g_{i,j}\oplus h_{i,j}
        \end{align*}
        which means $\mathbf{R} = \mathbf{G} + \mathbf{H}$.
    \end{itemize}
    As all the ring axioms are satisfied, thus $\Mn{n}{R}$ is a ring.

    We now show that $\Mn{n}{R}$ has a multiplicative identity, namely the identity matrix $\IdentityM{n}$. Let $\mathbf{A} \in \Mn{n}{R}$ and let $\mathbf{B} = \IdentityM{n}$. Note that $b_{i,j} = 1$ if and only if $i = j$.
    \begin{itemize}
        \item Let $\mathbf{C} = \textbf{AB}$ and we see
        \begin{align*}
            c_{i,j} &= \bigoplus_{k=1}^n(a_{i,k}\otimes b_{k,j})\\
            &= (a_{i,1}\otimes b_{1,j}) \oplus \cdots \oplus (a_{i,j-1}\otimes b_{j-1,j}) \oplus (a_{i,j}\otimes b_{j,j})\\
            &\quad\quad\oplus (a_{i,{j+1}}\otimes b_{j+1,j}) \oplus \cdots \oplus (a_{i,n}\otimes b_{n,j})\\
            &= (a_{i,1}\otimes 0) \oplus \cdots \oplus (a_{i,{j-1}}\otimes 0)\oplus (a_{i,j}\otimes 1)\\
            &\quad\quad\oplus (a_{i,{j+1}}\otimes 0) \oplus \cdots \oplus (a_{i,n}\otimes 0)\\
            &= 0 \oplus \cdots \oplus 0 \oplus a_{i,j} \oplus 0 \oplus \cdots \oplus 0\\
            &= a_{i,j}
        \end{align*}
        so $\mathbf{A}\IdentityM{n} = \mathbf{A}$.

        \item Now let $\mathbf{D} = \textbf{BA}$ and we also see
        \begin{align*}
            d_{i,j} &= \bigoplus_{k=1}^n(b_{i,k}\otimes a_{k,j})\\
            &= (b_{i,1}\otimes a_{1,j}) \oplus \cdots \oplus (b_{i,i-1}\otimes b_{i-1,j}) \oplus (b_{i,i}\otimes a_{i,j})\\
            &\quad\quad\oplus (b_{i,{i+1}}\otimes a_{i+1,j}) \oplus \cdots \oplus (b_{i,n}\otimes a_{n,j})\\
            &= (0 \otimes a_{1,j}) \oplus \cdots \oplus (0\otimes a_{i-1,j})\oplus (1\otimes a_{i,j})\\
            &\quad\quad\oplus (0\otimes a_{i+1,j}) \oplus \cdots \oplus (0\otimes a_{n,j})\\
            &= 0 \oplus \cdots \oplus 0 \oplus a_{i,j} \oplus 0 \oplus \cdots \oplus 0\\
            &= a_{i,j}
        \end{align*}
        so $\IdentityM{n}\mathbf{A} = \mathbf{A}$.
    \end{itemize}
    Therefore the identity matrix $\IdentityM{n}$ is the multiplicative identity.

    Hence $\Mn{n}{R}$ is a ring with identity.
\end{proof}

\section{Zero Divisors and Units}\label{section-rings-zero-divisors-and-units}
\begin{definition}
    Let $R$ be a ring. A non-zero element $a \in R$ is a \term{zero divisor}\index{zero divisor} if and only if there exists a non-zero element $b \in R$ such that $ab = 0$.
\end{definition}
\begin{example}
    Consider the ring $\Z_{12}$. Clearly 4 and 6 are in $\Z_{12}$, and their product is $24 = 2 \times 12 = 0$ in $\Z_{12}$. Hence 4 and 6 are zero divisors in $\Z_{12}$.
\end{example}
\begin{example}
    Let $R$ be the ring of functions with domain and codomain $[0, 1]$. We claim that $R$ has zero divisors. Consider the functions
    \begin{align*}
        f:[0,1]\to[0,1], x &\mapsto x\\
        g:[0,1]\to[0,1], x &\mapsto \begin{cases}
            0 & \text{ if } x \neq 0\\
            1 & \text{ if } x = 0
        \end{cases}
    \end{align*}
    Clearly neither of them are the zero function. Now consider $f(x)g(x)$.
    \begin{itemize}
        \item If $x \neq 0$, then $g(x) = 0$ which means $f(x)g(x) = 0$.
        \item If $x = 0$, then $f(x) = 0$ which means $f(x)g(x) = 0$.
    \end{itemize}
    Hence their product is the zero function, meaning that $R$ has zero divisors $f(x)$ and $g(x)$.
\end{example}
\begin{exercise}
    Does the ring $\Mn{2}{\mathbb{R}}$ have zero divisors?
\end{exercise}

We note one property about zero divisors, which will be used in future chapters.

\begin{proposition}\label{prop-zero-divisors-have-no-inverses}
    Zero divisors do not have inverses.
\end{proposition}
\begin{proof}
    Assume $a \neq 0$ and $b \neq 0$ are zero divisors in the ring $R$, so $ab = 0$. Seeking a contradiction, assume $a$ has an inverse. This means
    \[
        b = (a^{-1}a)b = a^{-1}(ab) = a^{-1}0 = 0
    \]
    which contradicts $b \neq 0$. Hence a zero divisor has no inverse.
\end{proof}

\begin{definition}
    Suppose $R$ is a ring with identity such that $0 \neq 1$. An element $u \in R$ is called a \term{unit}\index{unit} if there exists $v \in R$ such that $uv=vu=1$.

    Equivalently, we say that $u$ is a unit if it has a multiplicative inverse.
\end{definition}
\begin{example}
    Every non-zero element of $\Z_3$ is a unit since $1 \times 1 = 1$ and $2 \times 2 = 4 = 1$.
\end{example}
\begin{example}
    5 is a unit in $\Z_6$ since $5 \times 5 = 25 = 1$ in $\Z_6$.
\end{example}
\begin{example}
    3 and 7 are units in $\Z_{10}$ since $3 \times 7 = 7 \times 3 = 21 = 1$ in $\Z_{10}$.
\end{example}
\begin{example}
    Every non-zero element of $\Q$ is a unit since, for any rational number $\frac pq \in \Q$, we can find $\frac qp \in \Q$ where $\frac pq \times \frac qp = 1$, the multiplicative identity.
\end{example}

\begin{proposition}\label{prop-product-of-units-is-unit}
    The product of two units is a unit.
\end{proposition}
\begin{proof}
    See \myref{exercise-product-of-units-is-unit} (later).
\end{proof}
\begin{example}
    2 and 3 are units in $\Z_{13}$ since $2 \times 7 = 14 = 1$ and $3 \times 9 = 27 = 1$ in $\Z_{13}$. Thus their product, $2 \times 3 = 6$, is a unit too. Indeed, one sees $6 \times 11 = 66 = 1$, so 6 is in fact a unit in $\Z_{13}$.
\end{example}
\begin{example}
    Since 3 and 7 are units in $\Z_{10}$, their product, $3 \times 7 = 21 = 1$ is too.
\end{example}

\begin{definition}
    Suppose $R$ is a ring with identity such that $0 \neq 1$. Then $R$ is a \term{division ring}\index{division ring} if and only if every non-zero element in $R$ is a unit.
\end{definition}

\begin{definition}
    A commutative division ring is called a \term{field}\index{field}.
\end{definition}

\begin{example}
    We have shown in an earlier example that $\R$ is a commutative ring. We now show that $\R$ is actually a field by noting that every non-zero $x \in \R$ has a reciprocal $\frac1x$ that is a real number (\myref{axiom-reciprocal}) such that $x\left(\frac1x\right) = \left(\frac1x\right)x = 1$. Thus every non-zero $x \in \R$ is a unit, meaning that $\R$ is a division ring. Coupled with the fact that $\R$ is a commutative ring means that $\R$ is a field.
\end{example}
\begin{example}
    We also shown earlier that $\C$ is a commutative ring. We note that any non-zero complex number $z = a + bi$ has a multiplicative inverse given by
    \[
        w = \frac{a}{a^2+b^2} - \frac{b}{a^2+b^2}i
    \]
    since
    \begin{align*}
        zw &= (a+bi)\left(\frac{a}{a^2+b^2} - \frac{b}{a^2+b^2}i\right)\\
        &= \frac{(a+bi)(a-bi)}{a^2+b^2}\\
        &= \frac{a^2 - b^2i^2}{a^2+b^2}\\
        &= \frac{a^2+b^2}{a^2+b^2} & (\text{since } i^2 = -1)\\
        &= 1.
    \end{align*}
    Thus any non-zero complex number is a unit, meaning that $\C$ is a division ring. As $\C$ is also a commutative ring, this means that $\C$ is a field.
\end{example}

We will look at fields in more detail in part III.

\begin{exercise}
    Which of the following rings, if any, are fields?
    \begin{partquestions}{\alph*}
        \item $\Z$
        \item $\Q$
    \end{partquestions}
\end{exercise}

\begin{exercise}\label{exercise-product-of-units-is-unit}
    Prove \myref{prop-product-of-units-is-unit}.
\end{exercise}

\section{Subrings}
We end this chapter off with an exploration about subrings.

\begin{definition}
    Let $R$ be a ring and $S$ be a subset of $R$. Then $S$ is a \term{subring}\index{subring} of $R$ if and only if $S$ is also a ring under the same operations of addition and multiplication as $R$.
\end{definition}
\begin{remark}
    Equivalently, a subset $S$ of $R$ is a subring of $R$ if and only if
    \begin{itemize}
        \item $(S, +) \leq (R, +)$, that is, the subset $S$ under addition is a subgroup of $R$ under addition; and
        \item for all $a, b \in S$ we have $ab \in S$, i.e. $S$ is closed under multiplication.
    \end{itemize}
\end{remark}

\begin{example}
    We know that $\Z$ and $\Q$ are rings, and clearly $\Z \subseteq \Q$. Hence $\Z$ is a subring of $\Q$. Similarly, since $\Q \subseteq \R$ and $\R \subseteq \C$, thus $\Q$ is a subring of $\R$ and $\R$ is a subring of $\C$.
\end{example}

\begin{example}
    Consider the set of \term{Gaussian integers}\index{Gaussian integers}
    \[
        \Z[i] = \{a + bi \vert a,b\in\mathbb{Z}\},
    \]
    which is read as ``$\Z$ adjoin $i$''. We first show that $\Z[i]$ is a subring of $\C$.

    Clearly $\Z[i] \subseteq \C$. We show that $(\Z[i], +) \leq (\C, +)$.
    \begin{itemize}
        \item Clearly the identity of $(\C, +)$, which is 0, is inside $(\Z[i], +)$ as $0 = 0 + 0i$.
        \item For any $a + bi, c+di \in \Z[i]$, we have
        \[
            (a+bi) + (-(c+di)) = (a-c) + (b-d)i \in \Z[i].
        \]
    \end{itemize}
    Thus the subgroup test tells us that $(\Z[i], +) \leq (\C, +)$.

    Now we show that $\Z[i]$ is closed under multiplication. Let $a + bi, c+di \in \Z[i]$. Note that $ac, bd, ad, bc \in \Z$; the product of the two Gaussian integers is
    \[
        (a+bi)(c+di) = (ac-bd) + (ad+bc)i \in \Z[i]
    \]
    so $\Z[i]$ is closed under multiplication.

    Therefore $\Z[i]$ is a subring of $\C$.
\end{example}
\begin{exercise}
    Show that
    \[
        R = \left\{\begin{pmatrix}a&a\\a&a\end{pmatrix} \vert a \in \R\right\}
    \]
    is a subring of $\Mn{2}{\mathbb{R}}$.
\end{exercise}

\newpage

\section{Problems}
\begin{problem}
    Let $R$ be a ring. Prove that if $u \in R$ is a unit then so is $-u$.
\end{problem}

\begin{problem}
    Prove that the trivial ring is the unique ring with identity in which $0 = 1$.
\end{problem}

\begin{problem}
    Let $R$ be a set with an operation $\ast$ such that for all $x \ast y \in R$ for all $x, y \in R$. If $(R, \ast, \ast)$ is a ring, describe the elements in $R$.
\end{problem}

\begin{problem}
    Let $R$ be a ring with identity 1, and let $x \in R$.
    \begin{partquestions}{\roman*}
        \item Find four closed forms for the \term{geometric series}\index{geometric series}
        \[
            1 + x + x^2 + x^3 + \cdots + x^n.
        \]
        \item What are the condition(s) such that each of the closed forms are valid?
        \item Factor 112 into primes, and thus evaluate 112 in $\Z_{37}$.
        \item Hence, using the result(s) above, evaluate
        \[
            1 + 2^3 + 2^6 + 2^9 + \cdots + 2^{72}
        \]
        in the ring $\Z_{37}$.
    \end{partquestions}
\end{problem}

\begin{problem}
    Show that
    \[
        \Q[\sqrt2] = \{a + b\sqrt2 \vert a,b \in \Q\}
    \]
    is a ring. Hence show it is a field.
\end{problem}

\begin{problem}
    Let
    \[
        R = \left\{\begin{pmatrix}a&b\\0&0\end{pmatrix} \vert a,b \in \R\right\}
    \]
    be a ring under matrix addition and multiplication.
    \begin{partquestions}{\roman*}
        \item Show that $R$ has no identity.
        \item Show that $R$ contains a non-trivial subring $S$ with identity.
    \end{partquestions}
\end{problem}

\begin{problem}
    A ring $R$ is called a \term{Boolean ring}\index{Boolean ring} if $r^2 = r$ for all $r \in R$.
    \begin{partquestions}{\roman*}
        \item Show that $r = -r$ for all $r \in R$.
        \item Prove that every Boolean ring is commutative.
    \end{partquestions}
\end{problem}

\begin{problem}
    Let $R$ be a commutative ring with identity. We say that an element $x \in R$ is \term{nilpotent}\index{nilpotent} if there exists a positive integer $n$ such that $x^n = 0$.
    \begin{partquestions}{\roman*}
        \item Let $u \in R$ be a unit and $x \in R$ be nilpotent. Show that $ux$ is nilpotent.
        \item Show that $u - x$ is a unit.
    \end{partquestions}
\end{problem}
