\section{Integral Domains}
\subsection*{Exercises}
\begin{questions}
    \item Clearly multiplication is commutative (\myref{axiom-multiplication-is-commutative}) and $1 = 1 + 0\sqrt2 \in \Z[\sqrt2]$. All that is needed is to show that there are no zero divisors in $\Z[\sqrt2]$.

    Take $a+b\sqrt2, c+d\sqrt2 \in \Z[\sqrt2]$ such that $a+b\sqrt2 \neq 0$ and $(a+b\sqrt2)(c+d\sqrt2) = 0$. We want to show that the only way this is possible is if $c = d = 0$. Now consider
    \[
        \left((a+b\sqrt2)\underbrace{(a-b\sqrt2)}_{\neq 0}\right)\left((c+d\sqrt2)\underbrace{(c-d\sqrt2)}_{\neq 0}\right) = 0.
    \]
    This simplifies to $(a^2-2b^2)(c^2-2d^2) = 0$. Hence either $a^2-2b^2 = 0$ or $c^2-2d^2 = 0$, implying $a = \sqrt2b$ or $c = \sqrt2d$. Now we cannot have $a = \sqrt2b$ as $\sqrt2$ is not an integer; the only case that is possible is if $c = \sqrt2d$ which finally means that $c = d = 0$.

    \item Let $n = ab$ where $a,b \in Z$ and, without loss of generality, assume $1 < a \leq b < n$ (we exclude 1 and $n$ because we want $a$ and $b$ to be `proper' factors). Now clearly $a, b \in \Z_n$ with them both being non-zero but $ab = n = 0$ in $\Z_n$. Hence $a$ and $b$ are zero divisors in $\Z_n$, meaning $\Z_n$ is not an integral domain.

    \item We show that that $\Z_2[\alpha]$ is indeed a field. We note that multiplication is commutative with identity 1. The multiplication table in $\Z_2[\alpha]$ is provided below.
    \begin{table}[H]
        \centering
        \resizebox{\textwidth}{!}{
            \begin{tabular}{|l|l|l|l|}
                \hline
                $\boldsymbol{\times}$ & \textbf{1} & $\boldsymbol{\alpha}$ & $\boldsymbol{1+\alpha}$ \\ \hline
                \textbf{1} & 1 & $\alpha$ & $1+\alpha$ \\ \hline
                $\boldsymbol{\alpha}$ & $\alpha$ & $\alpha^2 = 1+\alpha$ & $\alpha+\alpha^2 = 1+2\alpha = 1$ \\ \hline
                $\boldsymbol{1+\alpha}$ & $1+\alpha$ & $\alpha+\alpha^2 = 1+2\alpha = 1$ & $1+2\alpha+\alpha^2 = 2+3\alpha=\alpha$ \\ \hline
            \end{tabular}
        }
    \end{table}

    What we see from this table is that no non-zero elements multiply together to form zero, meaning that there are no zero divisors in $\Z_2[\alpha]$. Therefore $\Z_2[\alpha]$ is an integral field. Furthermore, as $\Z_2[\alpha]$ is finite, thus $\Z_2[\alpha]$ is a field by \myref{thrm-finite-integral-domain-is-field}.

    \item The trivial ring $\{0\}$ is not an integral domain. If $\{0\}$ is indeed an integral domain, then by \myref{prop-zero-or-prime-characteristic-if-integral-domain} it has to have either 0 or prime characteristic. However, one sees clearly that the characteristic of the trivial ring is 1, which is neither 0 nor prime. Therefore $\{0\}$ is not an integral domain.

    \item The additive identity in $\Z_2[\alpha]$ is 1. Clearly $1 + 1 = 0$, so the order of 1 in $(\Z_2[\alpha],+)$ is 2. Now \myref{exercise-Zn2[alpha]} tells us that $\Z_2[\alpha]$ is an integral domain, so by the previous proposition this means that $\Char{\Z_2[\alpha]} = 2$.
\end{questions}

\subsection*{Problems}
\begin{questions}
    \item We are tasked to find $a+bi \in \Z_5[i]$ such that there exist $c+di \in \Z_5[i]$ where $(a+bi)(c+di) = 0$ but both $a+bi$ and $c+di$ are non-zero. Expanding $(a+bi)(c+di)$ yields $(ac-bd)+(ad+bc)i = 0$. Therefore we must have $ac-bd = 0$ and $ad+bc = 0$. For simplicity let's choose $a=c=1$. Using the second equation we have $d+b = 0$ which means $d = -b$. Hence $(1 - b(-b))+(-b + b)i = 1+b^2 = 0$. Therefore choosing $b = 2$ would make it work. Therefore one solution is $a = 1, b = 2, c = 1, d = -2 = 3$; i.e. two zero divisors are $1+2i$ and $1+3i$.

    \item \begin{partquestions}{\alph*}
        \item Note that multiplication is commutative with identity $1 = 1 + 0\sqrt{n} \in R$. We just need to show that there are no zero divisors in $R$.

        Take $a+b\sqrt n, c+d\sqrt n \in R$ such that $a+b\sqrt n \neq 0$ but $(a+b\sqrt n)(c+d\sqrt n) = 0$. We want to show $c = d = 0$. Consider
        \[
            \left((a+b\sqrt n)(\underbrace{a-b\sqrt n}_{\neq 0})\right)\left((c+d\sqrt n)(\underbrace{c-d\sqrt n}_{\neq 0})\right) = 0.
        \]
        This means that $(a^2-nb^2)(c^2-nd^2) = 0$, so either $a^2-nb^2 = 0$ or $c^2-nd^2 = 0$.

        Now if $n < 0$ then clearly we have to have $c = d = 0$. Otherwise we have $a = b\sqrt n$ or $c = d\sqrt n$. But $\sqrt n$ is not an integer, so the only way for equality is if $c = d = 0$. Thus $\Z[\sqrt n]$ has no zero divisors, meaning $\Z[\sqrt n]$ is an integral domain.

        \item Consider $2 + \sqrt 2 \in \Z[\sqrt 2]$. Its multiplicative inverse is
        \begin{align*}
            \frac{1}{2+\sqrt2} &= \frac{2-\sqrt2}{(2+\sqrt2)(2-\sqrt2)}\\
            &= \frac{2-\sqrt2}{4-2}\\
            &= 1 - \frac12\sqrt2 \notin \Z[\sqrt2].
        \end{align*}
        This means that $2+\sqrt2$, a non-zero element in $\Z[\sqrt2]$, does not have an inverse in $\Z[\sqrt2]$. Therefore $\Z[\sqrt2]$ is not a field, meaning $R$ is not a field in the general case.
    \end{partquestions}

    \item For brevity let $\text{O} = \begin{pmatrix}0&0\\0&0\end{pmatrix}$, $\text{I} = \begin{pmatrix}1&0\\0&1\end{pmatrix}$, $\text{A} = \begin{pmatrix}1&1\\1&0\end{pmatrix}$, and $\text{B} = \begin{pmatrix}0&1\\1&1\end{pmatrix}$.

    \begin{partquestions}{\roman*}
        \item Clearly one sees that $R$ is a subset of $\Mn{2}{\Z_2}$.
        \begin{itemize}
            \item We show $(R, +)\leq(\Mn{2}{\Z_2},+)$.
            \begin{table}[H]
                \centering
                \begin{tabular}{|l|l|l|l|l|}
                    \hline
                    \textbf{+} & \textbf{O} & \textbf{I} & \textbf{A} & \textbf{B} \\ \hline
                    \textbf{O} & O          & I          & A          & B          \\ \hline
                    \textbf{I} & I          & O          & B          & A          \\ \hline
                    \textbf{A} & A          & B          & O          & I          \\ \hline
                    \textbf{B} & B          & A          & I          & O          \\ \hline
                \end{tabular}
            \end{table}

            From the Cayley table, one sees that the identity of the ring $\Mn{2}{\Z_2}$ is in $R$ and $R$ is closed under addition. Hence $(R, +)\leq(\Mn{2}{\Z_2},+)$.

            \item We show $R$ is closed under multiplication.
            \begin{table}[H]
                \centering
                \begin{tabular}{|l|l|l|l|l|}
                    \hline
                    $\boldsymbol{\cdot}$ & \textbf{O} & \textbf{I} & \textbf{A} & \textbf{B} \\ \hline
                    \textbf{O}           & O          & O          & O          & O          \\ \hline
                    \textbf{I}           & O          & I          & A          & B          \\ \hline
                    \textbf{A}           & O          & A          & B          & I          \\ \hline
                    \textbf{B}           & O          & B          & I          & A          \\ \hline
                \end{tabular}
            \end{table}

            From the Cayley table, clearly $R$ is closed under multiplication.
        \end{itemize}
        Therefore $R$ is a subring of $\Mn{2}{\Z_2}$.

        \item Since $R$ is a subring of $\Mn{2}{\Z_2}$, it is a ring. Furthermore, by the Cayley table of $(R, \cdot)$, we see that $R$ is commutative with identity I. Finally, one sees that $\mathrm{A}^{-1} = \mathrm{B}$, $\mathrm{B}^{-1} = \mathrm{A}$, and $\mathrm{I}^{-1} = \mathrm{I}$. Therefore all non-zero elements of $R$ have inverses. Hence $R$ is a field.
    \end{partquestions}
\end{questions}
