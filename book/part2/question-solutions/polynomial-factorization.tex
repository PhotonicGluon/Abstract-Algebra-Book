\section{Factorization of Polynomials}
\subsection*{Exercises}
\begin{questions}
    \item \begin{partquestions}{\alph*}
        \item Not irreducible since $3x^2 - 6 = 3(x^2-2)$ and both 3 and $x^2-2$ are non-units in $\Z$.
        \item Irreducible in $\Q$ since we cannot express $3x^2-6$ as a product of two polynomials of smaller degree that are in $\Q[x]$ (by \myref{thrm-irreducible-iff-not-expressable-as-product-of-smaller-polynomials}).
        \item Reducible in $\R$ since $3x^2 - 6 = (3x-3\sqrt2)(x+\sqrt2)$ and both $3x-3\sqrt2, x+\sqrt2 \in \R[x]$.
    \end{partquestions}

    \item For brevity we compute all possible values of $f(x)$ for $x \in \Z_7$ and then check for zeroes.
    \begin{table}[H]
        \centering
        \begin{tabular}{|l|l|l|l|l|l|}
            \hline
            $\boldsymbol{x}$ & $\boldsymbol{f(x)}$ & $\boldsymbol{f(x) \mod2}$ & $\boldsymbol{f(x) \mod3}$ & $\boldsymbol{f(x) \mod5}$ & $\boldsymbol{f(x) \mod7}$ \\ \hline
            \textbf{0} & 9 & 1 & 0 & 4 & 2 \\ \hline
            \textbf{1} & 15 & 1 & 0 & 0 & 1 \\ \hline
            \textbf{2} & 33 &  & 0 & 3 & 5 \\ \hline
            \textbf{3} & 75 &  &  & 0 & 5 \\ \hline
            \textbf{4} & 153 &  &  & 3 & 6 \\ \hline
            \textbf{5} & 279 &  &  &  & 6 \\ \hline
            \textbf{6} & 465 &  &  &  & 3 \\ \hline
        \end{tabular}
    \end{table}
    We therefore see that $f(x)$ has zeroes in $\Z_3$ and $\Z_5$ and no zeroes in $\Z_2$ and $\Z_7$. Thus $f(x)$ is irreducible over $\Z_2$ and $\Z_7$ and reducible in $\Z_3$ and $\Z_5$.

    \item \begin{partquestions}{\alph*}
        \item Disprove. $2(x^2+1)$ is reducible in $\Z$. However $2(x^2+1)$ has no zeroes in $\Q$ and so is irreducible over $\Q$ (\myref{thrm-degree-2-or-3-irreducible-iff-has-no-zeroes}).
        
        \item Prove. Statement is the contrapositive of \myref{thrm-irreducible-over-Z-means-irreducible-over-Q}.
    \end{partquestions}

    \item Clearly $1^2 + 2(1) + 1 = 4 \equiv 0 \pmod2$, so we consider $p > 3$. Consider the polynomial $f(x) = x^2 + 2x + 1$. Clearly $f(-1) = 0$ so $f(x)$ is reducible over $\Q$ (\myref{thrm-degree-2-or-3-irreducible-iff-has-no-zeroes}). Thus the contrapositive of the Mod $p$ Irreducibility Test (\myref{thrm-mod-p-irreducibility-test}) tells us that $f(x)$ is reducible in $\Z_p$ (since reducing the coefficients modulo $p$ yields the same polynomial). Therefore $f(x) \in \Z_p[x]$ must have a zero (\myref{thrm-degree-2-or-3-irreducible-iff-has-no-zeroes}), i.e. there is an integer $n \in \Z_p$ such that $f(n) = n^2 + 2n + 1 = 0$ when working modulo $p$. That same $n$ satisfies $n^2 + 2n + 1 \equiv 0 \pmod{p}$ as required.

    \item We claim that $f(x) = x^n + 2$ is irreducible over $\Q$ for any $n \geq 1$. We note that 2 does not divide 1, 2 divides 2, and $2^2 = 4$ does not divide 2. Thus $f(x)$ is irreducible over $\Q$ by Eisenstein's Criterion (\myref{thrm-eisenstein-criterion}) with the prime 2.

    \item \begin{partquestions}{\roman*}
        \item Choose the prime 3. Clearly 3 does not divide 1, 3 divides 3, but $3^2 = 9$ does not divide 3. Thus Eisenstein's Criterion (\myref{thrm-eisenstein-criterion}) tells us that $f(x)$ is irreducible over $\Q$.

        \item One may use Eisenstein's Criterion (with the prime 2) on the given polynomial to show that it is irreducible over $\Q$. Another method is to observe that
        \[
            x^4 + 4x^3 + 6x^2 + 4x + 4 = (x+1)^4 + 3
        \]
        and since $x^4 + 3$ is irreducible thus the above polynomial is also irreducible over $\Q$. Note that the given polynomial is primitive, so it is also irreducible over $\Z$ (\myref{thrm-irreducible-over-Z-means-irreducible-over-Q}). Hence $x^4 + 4x^3 + 6x^2 + 4x + 4$ has no zeroes in $\Z$ by contrapositive of \myref{thrm-degree-above-1-reducible-if-has-zero}.

        \item We note we cannot use Eisenstein's Criterion directly, since the only prime that works on the constant term is 19, and $19 \nmid 32$. We need to consider a substitution.

        One may see that, after some trial and error, that
        \begin{align*}
            x^4 - 8x^3 + 24x^2 - 32x + 19 &= (x^4 - 8x^3 + 24x^2 - 32x + 16) + 3\\
            &= (x - 2)^4 + 3
        \end{align*}
        and since $x^4 + 3$ is irreducible by \textbf{(i)}, thus $x^4 - 8x^3 + 24x^2 - 32x + 19$ is irreducible by transformation.
    \end{partquestions}

    \item We prove the three requirements for a ring isomorphism.
    \begin{itemize}
        \item \textbf{Homomorphism}: One sees that
        \begin{align*}
            \phi(f(x) + g(x)) &= \phi\left(\sum_{i=0}^m(a_i+b_i)x^i\right)\\
            &= \sum_{i=0}^m(a_i+b_i)(kx)^i\\
            &= \left(\sum_{i=0}^ma_i(kx)^i\right) + \left(\sum_{i=0}^mb_i(kx)^i\right)\\
            &= \left(\sum_{i=0}^ma_i(kx)^i\right) + \left(\sum_{i=0}^nb_i(kx)^i\right) & (\text{as } b_i = 0 \text{ for } i > n)\\
            &= \phi(f(x)) + \phi(g(x))
        \end{align*}
        and
        \begin{align*}
            \phi(f(x)g(x)) &= \phi\left(\sum_{r=0}^{m+n}\left(\sum_{i=0}^ra_ib_{r-i}\right)x^r\right)\\
            &= \sum_{r=0}^{m+n}\left(\sum_{i=0}^ra_ib_{r-i}\right)(kx)^r\\
            &= \left(\sum_{r=0}^ma_r(kx)^r\right)\left(\sum_{r=0}^nb_r(kx)^r\right)\\
            &= \phi(f(x))\phi(g(x))
        \end{align*}
        so $\phi$ is a ring homomorphism.

        \item \textbf{Injective}: Suppose $\phi(f(x)) = \phi(g(x))$. Then
        \begin{align*}
            \sum_{i=0}^m(a_ik^i)x^i &= \sum_{i=0}^ma_i(kx)^i\\
            &= \sum_{i=0}^nb_i(kx)^i\\
            &= \sum_{i=0}^n(b_ik^i)x^i
        \end{align*}
        which, by comparing coefficients, we see $a_i = b_i$ for all $0 \leq i \leq m$. Therefore $f(x) = g(x)$, meaning $\phi$ is injective.

        \item \textbf{Surjective}: Let $p(x) = c_0 + c_1x + \cdots + c_rx^r$ be a polynomial in $D[x]$. Since $k$ is a unit, therefore $k^{-1}$ exists. Note that $q(x) = c_0 + c_1(k^{-1}x) + \cdots + c_r(k^{-1}x)^r$ is also a polynomial in $D[x]$. Observe
        \begin{align*}
            \phi(q(x)) &= c_0 + c_1(k(k^{-1}x)) + \cdots + c_r(k(k^{-1}x))^r\\
            &= c_0 + c_1x + \cdots + c_rx^r\\
            &= p(x)
        \end{align*}
        so any $p(x) \in D[x]$ has a pre-image under $\phi$.
    \end{itemize}
    Therefore $\phi$ is an isomorphism.

    \item Since $p(x)$ is irreducible we know $D[x]/\princ{p(x)}$ is a field (\myref{corollary-polynomial-quotient-by-principal-ideal-is-field-iff-polynomial-irreducible}), which is a integral domain (\myref{prop-field-is-integral-domain}), and so $\princ{p(x)}$ is a prime ideal (\myref{thrm-prime-ideal-iff-quotient-ring-is-integral-domain}). As $p(x) = a(x)b(x)$ therefore $a(x)b(x) \in \princ{p(x)}$. Hence $a(x) \in \princ{p(x)}$ or $b(x) \in \princ{p(x)}$ by definition of prime ideal. So $a(x) = k(x)p(x)$ or $b(x) = k(x)p(x)$ for some $k(x) \in D[x]$, meaning $p(x) \vert a(x)$ or $p(x) \vert b(x)$.

    \item \begin{partquestions}{\roman*}
        \item Let $p(x) = x^2 + 2 \in \Z_5[x]$. One sees that
        \begin{itemize}
            \item $p(0) = 0^2 + 2 = 2 \neq 0$;
            \item $p(1) = 1^2 + 2 = 3 \neq 0$;
            \item $p(2) = 2^2 + 2 = 6 = 1 \neq 0$;
            \item $p(3) = 3^2 + 2 = 11 = 1 \neq 0$; and
            \item $p(4) = 4^2 + 2 = 18 = 3 \neq 0$,
        \end{itemize}
        so $p(x)$ is irreducible over $\Z_5$ by \myref{thrm-degree-2-or-3-irreducible-iff-has-no-zeroes}.

        \item $F = Z_5[x]/\princ{p(x)}$ is a field of 25 elements.

        \item \begin{partquestions}{\alph*}
            \item We see
            \begin{align*}
                \left((2x+3) + I\right) + \left((4x^2+3x+2) + I\right) &= (4x^2+5x+5) + I\\
                &= 4(x^2+2) + 5x - 3 + I\\
                &= 5x - 3 + I\\
                &= 2 + I
            \end{align*}
            so $(f(x) + I) + (g(x) + I) = 2 + I$.

            \item We note
            \begin{align*}
                ((2x+3) + I)((4x^2+3x+2) + I) &= (8 x^3 + 18 x^2 + 13 x + 6) + I\\
                &= \left((8x + 18)(x^2+2) - 3x - 30\right) + I\\
                &= -3x - 30 + I\\
                &= 2x + I
            \end{align*}
            so $(f(x) + I)(g(x) + I) = 2x + I$.
        \end{partquestions}
    \end{partquestions}

    \item \begin{partquestions}{\alph*}
        \item $x \times x = x^2 = 1(x^2+1) - 1 = -1 = 2$.

        \item We note
        \begin{align*}
            (x+1)(2x+1) &= 2x^2 + 3x + 1\\
            &= 2x^2 + 0x + 1\\
            &= 2x^2 + 1\\
            &= 2(2) + 1 & (\text{since } x^2 = 2)\\
            &= 5\\
            &= 2.
        \end{align*}

        \item We see
        \begin{align*}
            (2x+2)(x+2) &= 2\left((x+1)(x+2)\right)\\
            &= 2(1) & (\text{since } (x+1)(x+2) = 1 \text{ from table})\\
            &= 2.
        \end{align*}

        \item Note
        \begin{align*}
            (2x+2)(2x+1) &= 2(x+1)(2x+1)\\
            &= 2(2) & (\text{from }\textbf{(b)})\\
            &= 4\\
            &= 1.
        \end{align*}
    \end{partquestions}
\end{questions}

\subsection*{Problems}
\begin{questions}
    \item \begin{partquestions}{\alph*}
        \item Reducible. Since
        \[
            f_1(-1) = (-1)^3 + (-1)^2 + (-1) + 1 = 0
        \]
        thus $f_1(x)$ has a zero in $\Q$, meaning that it is reducible over $\Q$ (\myref{thrm-degree-above-1-reducible-if-has-zero}).

        \item Reducible. Since
        \[
            f_2\left(-\frac12\right) = 6\left(-\frac12\right)^3 + \left(-\frac12\right)^2 + \left(-\frac12\right) + 1 = 0
        \]
        thus $f_2(x)$ has a zero in $\Q$, meaning that it is reducible over $\Q$ (\myref{thrm-degree-above-1-reducible-if-has-zero}).

        \item Irreducible. Note that reducing the coefficients of $f_3(x)$ modulo 7 results in the polynomial $\bar{f_3}(x) = x^3 + 2$, and that
        \begin{itemize}
            \item $\bar{f_3}(0) = 0^3 + 2 = 2 \neq 0$;
            \item $\bar{f_3}(1) = 1^3 + 2 = 3 \neq 0$;
            \item $\bar{f_3}(2) = 2^3 + 2 = 10 = 3 \neq 0$;
            \item $\bar{f_3}(3) = 3^3 + 2 = 24 = 1 \neq 0$;
            \item $\bar{f_3}(4) = 4^3 + 2 = 66 = 4 \neq 0$;
            \item $\bar{f_3}(5) = 5^3 + 2 = 127 = 1 \neq 0$; and
            \item $\bar{f_3}(6) = 6^3 + 2 = 218 = 1 \neq 0$,
        \end{itemize}
        so $\bar{f_3}(x)$ is irreducible over $\Z_7$ by \myref{thrm-degree-2-or-3-irreducible-iff-has-no-zeroes}, and so $f_3(x)$ is irreducible over $\Q$ by Mod 7 Irreducibility Test (\myref{thrm-mod-p-irreducibility-test}).

        \item Irreducible. Note that 5 does not divide 1, 5 divides 5, and $5^2 = 25$ does not divide 5, so $f_4(x)$ is irreducible over $\Q$ by Eisenstein's Criterion (\myref{thrm-eisenstein-criterion}) with the prime 5.

        \item Reducible. Since
        \[
            f_5(1) = (1)^4 - 2(1)^3 + (1)^2 - (1) + 1 = 0
        \]
        thus $f_5(x)$ has a zero in $\Q$ and so is reducible over $\Q$ (\myref{thrm-degree-above-1-reducible-if-has-zero}).

        \item Irreducible. As
        \begin{align*}
            f_6(x) &= x^4 + 12x^3 + 54x^2 + 108x + 86\\
            &= (x^4 + 12x^3 + 54x^2 + 108x + 81) + 5\\
            &= (x+3)^4 + 5,
        \end{align*}
        and since $x^4 + 5$ is irreducible by \textbf{(d)}, therefore $f_6(x)$ is also irreducible by a corollary of the Transformation Rule (\myref{corollary-irreducible-iff-translation-is-irreducible}).
    \end{partquestions}

    \item Let $f(x) = x^2 - 2$. Reducing coefficients of $f(x)$ modulo 3 yields $\bar{f}(x) = x^2 + 1$. Note that in $\Z_3$ we have
    \begin{itemize}
        \item $\bar{f}(0) = 1 \neq 0$;
        \item $\bar{f}(1) = 1^2 + 1 = 2 \neq 0$; and
        \item $\bar{f}(2) = 2^2 + 1 = 5 = 2 \neq 0$.
    \end{itemize}
    Therefore $\bar{f}(x)$ is irreducible over $\Z_3$ by \myref{thrm-degree-2-or-3-irreducible-iff-has-no-zeroes} and thus $f(x)$ is irreducible over $\Q$ by Mod 3 Irreducibility Test (\myref{thrm-mod-p-irreducibility-test}), which means $f(x)$ has no zeroes in $\Q$ (\myref{thrm-degree-2-or-3-irreducible-iff-has-no-zeroes}). But in $\R$ one sees that $f(x)$ has the zeroes $\pm\sqrt2$. Therefore $\sqrt2 \notin \Q$.

    \item Suppose $f(x)$ is a primitive, degree 1 polynomial in $\Z[x]$. By way of contradiction, suppose $f(x)$ is reducible, meaning $f(x) = p(x)q(x)$ for some non-zero non-unit polynomials $p(x), q(x) \in \Z[x]$. Since $\Z[x]$ is an integral domain, we must have
    \[
        1 = \deg f(x) = \deg(p(x)q(x)) = \deg p(x) + \deg q(x)
    \]
    by \myref{thrm-polynomial-degree-properties}. So exactly one of $p(x)$ or $q(x)$ is constant; without loss of generality assume $p(x)$ is the constant polynomial. Set $p(x) = k \in \Z$ and we see $f(x) = kq(x)$. But $f(x)$ is primitive, meaning that a factorization of $kq(x)$ is only possible if $k = \pm1$, which are both units, a contradiction. Therefore, $f(x)$ is irreducible.

    \item \begin{partquestions}{\roman*}
        \item Note $f(1) = 1^3 + 6 = 7 = 0$ so $f(x)$ has a zero in $\Z_7$.

        \item As 1 is a zero of $f(x)$, thus $x-1 = x+6$ is a factor of $f(x)$ by Factor Theorem (\myref{corollary-factor-theorem}). Performing long division on $x+6$ we see
        \begin{align*}
            x^3 + 6 &= x^2(x+6) - 6x(x+6) + 36(x+6) - 210\\
            &= (x^2-6x+36)(x+6) - 210\\
            &= (x^2+x+1)(x+6) + 0 & (\text{Evaluating in }\Z_7)\\
            &= (x^2+x+1)(x+6).
        \end{align*}
        One can see that 2 is a zero of $x^2 + x + 1$ in $\Z_7$, so $x - 2 = x+5$ is a factor of $x^2 + x + 1$ by Factor Theorem again. Performing division on it yields
        \begin{align*}
            x^2 + x + 1 &= x(x+5) - 4(x+5) + 21\\
            &= (x-4)(x+5) + 21\\
            &= (x+3)(x+5) + 0 & (\text{Evaluating in }\Z_7)\\
            &= (x+3)(x+5)
        \end{align*}
        which means that
        \[
            x^3 + 6 = (x+3)(x+5)(x+6).
        \]
        We note that $x+3$, $x+5$, and $x+6$ are all irreducible polynomials in $\Z_7$, so we have accomplished our goal.
    \end{partquestions}

    \item \begin{partquestions}{\alph*}
        \item We note that there are only 4 distinct degree 2 polynomials in $\Z_2[x]$. We determine the reducibility of each of the polynomials.
        \begin{itemize}
            \item $\boxed{x^2}$ Reducible since 0 is a zero of $x^2$.
            \item $\boxed{x^2 + 1}$ Reducible since 1 is a zero of $x^2 + 1$.
            \item $\boxed{x^2+x}$ Reducible since 0 is a zero of $x^2 + x$.
            \item $\boxed{x^2+x+1}$ Irreducible since neither 0 nor 1 are zeroes of the polynomial (\myref{thrm-degree-2-or-3-irreducible-iff-has-no-zeroes}).
        \end{itemize}
        Thus the only distinct degree 2 polynomial that is irreducible over $\Z_2[x]$ is $x^2+x+1$.

        \item We do the same for degree 3 polynomials. We note that there are 8 such distinct polynomials.
        \begin{itemize}
            \item $\boxed{x^3}$ Reducible since 0 is a zero.
            \item $\boxed{x^3 + 1}$ Reducible since 1 is a zero.
            \item $\boxed{x^3 + x}$ Reducible since 0 is a zero.
            \item $\boxed{x^3 + x + 1}$ Irreducible since Irreducible since neither 0 nor 1 are zeroes of the polynomial.
            \item $\boxed{x^3 + x^2}$ Reducible since 0 is a zero.
            \item $\boxed{x^3 + x^2 + 1}$ Irreducible since neither 0 nor 1 are zeroes of the polynomial.
            \item $\boxed{x^3 + x^2 + x}$ Reducible since 0 is a zero.
            \item $\boxed{x^3 + x^2 + x + 1}$ Reducible since 1 is a zero.
        \end{itemize}
        Therefore the only distinct irreducible degree 3 polynomials in $\Z_2[x]$ are $x^3+x+1$ and $x^3+x^2+1$.
    \end{partquestions}

    \item A reducible polynomial of the required form must have the factorization $(x+\alpha)(x+\beta)$. We split into two cases.
    \begin{itemize}
        \item If $\alpha \neq \beta$, then there are $p$ possibilities for $\alpha$ and $p - 1$ possibilities for $\beta$. However, we need to account for commutativity of the two factors, so we divide the total by 2. This leaves us with a total of $\frac{p(p-1)}{2}$ distinct polynomials for this case.
        \item If instead $\alpha = \beta$, then there are just $p$ choices for $\alpha = \beta$.
    \end{itemize}
    All in all, there are $\frac{p(p-1)}{2} + p = \frac{p(p+1)}{2}$ distinct polynomials of the required form.

    \item \begin{partquestions}{\alph*}
        \item Note that
        \begin{align*}
            (x+1)^4 + 1 &= (x^4 + 4x^3 + 6x^2 + 4x + 1) + 1\\
            &= x^4 + 4x^3 + 6x^2 + 4x + 2.
        \end{align*}
        We see that the prime 2 does not divide the leading coefficient 1, divides all other coefficients, and $2^2 = 4$ does not divide the constant term 2. Therefore, by Eisenstein's Criterion (\myref{thrm-eisenstein-criterion}) with the prime 2, we know $(x+1)^4 + 1$ is irreducible. Hence $x^4 + 1$ is irreducible by a corollary of the Transformation Rule (\myref{corollary-irreducible-iff-translation-is-irreducible}).

        \item Note that in $\Z_2[x]$ we have
        \begin{align*}
            x^4 + 1 &= x^4 + 2x + 1\\
            &= (x^2+1)^2
        \end{align*}
        so $f(x) = x^4+1$ is reducible over $\Z_2$.

        \begin{partquestions}{\roman*}
            \item \begin{partquestions}{\alph*}
                \item If $r^2 = 2$, then note
                \begin{align*}
                    x^4 + 1 &= (x^4 + 2x^2 +1) - 2x^2\\
                    &= (x^2+1)^2 - 2x^2\\
                    &= (x^2+1)^2 - r^2x^2 & (\text{since } r^2 = 2)\\
                    &= (x^2+1+rx)(x^2+1-rx)
                \end{align*}
                so $f(x)$ is reducible.

                \item If instead $r^2 = -1$, then one sees
                \begin{align*}
                    x^4 + 1 &= x^4 - (-1)\\
                    &= x^4 - r^2 & (\text{since } r^2 = -1)\\
                    &= (x^2+r)(x^2-r)
                \end{align*}
                so, once again, $f(x)$ is reducible.

                \item In the third case, if $r^2 = -2$, then
                \begin{align*}
                    x^4 + 1 &= (x^4 - 2x^2 +1) + 2x^2\\
                    &= (x^2-1)^2 - (-2)x^2\\
                    &= (x^2-1)^2 - r^2x^2 & (\text{since } r^2 = -2)\\
                    &= (x^2-1+rx)(x^2-1-rx)
                \end{align*}
                so $f(x)$ is, again, reducible.
            \end{partquestions}

            \item As sets, we see that
            \begin{align*}
                \Z_p^\ast &= \{0, 1, 2, \dots, p-1\} \setminus \{0\}\\
                &= \{1, 2, \dots, p-1\}\\
                &= \left\{m \vert 1 \leq m < p \text{ and } \gcd(m,p) = 1\right\}\\
                &= \Un{p},
            \end{align*}
            where the third line is justified since any prime is coprime to any positive integer that is smaller than it. Now because the group operations on both sets is the same (i.e., multiplication modulo $p$), they must be the same group.

            \item There are $p - 1$ numbers from 1 to $p - 1$ inclusive. As $p > 2$, thus $p$ is odd and therefore $p - 1$ is even, meaning $\Un{p}$ is a group with even order.

            Also, as $p$ is an odd integer, there exists a primitive root modulo $p$ (\myref{axiom-primitive-root-modulo-p}). Therefore $\Un{p}$ is cyclic (\myref{prop-Un-cyclic-only-if-exists-primitive-root}), which therefore means that $\Un{p}$ is a cyclic group with even order.

            \item Let $g$ be the generator of $\Un{p}$. We consider again the three cases.
            \begin{itemize}
                \item If $2 = g^{2k}$ for some positive integer $k$, then the integer $r$ in question is $g^k$. The case in \textbf{(b)(i)(a)} applies.
                \item Otherwise, if $-1 = g^{2k}$ for some positive integer $k$, then the integer $r$ in question is $g^k$. The case in \textbf{(b)(i)(b)} applies.
                \item Otherwise, we must have $2 = g^{2m - 1}$ and $-1 = g^{2n - 1}$ for some positive integers $m$ and $n$. So one sees that
                \begin{align*}
                    -2 &= (2)(-1)\\
                    &= \left(g^{2m-1}\right)\left(g^{2n-1}\right)\\
                    &= g^{2m+2n-2}\\
                    &= g^{2(m+n-1)},
                \end{align*}
                so the integer $r$ in question is $g^{m+n-1}$ and the case in \textbf{(b)(i)(c)} applies.
            \end{itemize}
            In all cases, we obtain a factorization of $f(x)$, which shows that $f(x)$ is reducible over $\Z_p$ for all prime numbers $p$.
        \end{partquestions}
    \end{partquestions}
\end{questions}
