\section{Ring Homomorphisms and Isomorphisms}
\subsection*{Exercises}
\begin{questions}
    \item We show that $\phi$ is not a ring homomorphism. Consider the matrices $\begin{pmatrix}1&0\\0&0\end{pmatrix}$ and $\begin{pmatrix}0&0\\0&1\end{pmatrix}$. Note that $\phi\left(\begin{pmatrix}1&0\\0&0\end{pmatrix}\right) = 1 + 0 = 1$ and $\phi\left(\begin{pmatrix}0&0\\0&1\end{pmatrix}\right) = 0 + 1 = 1$, so
    \[
        \phi\left(\begin{pmatrix}1&0\\0&0\end{pmatrix}\right)\phi\left(\begin{pmatrix}0&0\\0&1\end{pmatrix}\right) = 1 \times 1 = 1.
    \]
    However, note
    \[
        \begin{pmatrix}1&0\\0&0\end{pmatrix}\begin{pmatrix}0&0\\0&1\end{pmatrix} = \begin{pmatrix}0&0\\0&0\end{pmatrix}
    \]
    so $\phi\left(\begin{pmatrix}1&0\\0&0\end{pmatrix}\begin{pmatrix}0&0\\0&1\end{pmatrix}\right) = 0$. Thus $\phi$ is not a homomorphism.

    \item Note for any $a, b \in R$ we have
    \[
        \phi(a+b) = 0 = 0 + 0 = \phi(a) + \phi(b)
    \]
    and
    \[
        \phi(ab) = 0 = 0\times0 = \phi(a)\phi(b)
    \]
    so $\phi$ is indeed a ring homomorphism.

    \item Note for any $a, b \in R$ that
    \[
        \id(a+b) = a + b = \id(a) + \id(b)
    \]
    and
    \[
        \id(ab) = ab = \id(a)\id(b)
    \]
    so $\id$ is a ring endomorphism.

    \item We have shown that the identity homomorphism is a homomorphism, so we just need to prove that it is a bijection.
    \begin{itemize}
        \item \textbf{Injective}: Suppose $a, b \in R$ are such that $\phi(a) = \phi(b)$. But since $\phi(x) = x$ thus $a = b$.
        \item \textbf{Surjective}: As $\phi(x) = x$ thus any element is its own pre-image.
    \end{itemize}
    Therefore the identity homomorphism is an isomorphism. As it is also an endomorphism, therefore it is an automorphism.

    \item \begin{partquestions}{\alph*}
        \item Note that
        \[
            \phi(0_1) = \phi(0_1 + 0_1) = \phi(0_1) + \phi(0_1)
        \]
        so by 'adding' $-\phi(0_1)$ on both sides we see that $\phi(0_1) = 0_2$.

        \item Note that
        \[
            \phi(1_1) = \phi(1_1 \times 1_1) = \phi(1_1)\phi(1_1).
        \]
        Since $R_1$ and $R_2$ are division rings, we may apply $\phi(1_1)^{-1}$ on both sides to yield $\phi(1_1) = 1_2$.
    \end{partquestions}

    \item \begin{partquestions}{\alph*}
        \item Notice that
        \[
            \phi(x + (-x)) = \phi(x) + \phi(-x)
        \]
        and
        \[
            \phi(x + (-x)) = \phi(0_1) = 0_2
        \]
        so subtracting $-\phi(x)$ on both sides yields $\phi(-x) = -\phi(x)$.

        \item Notice that
        \[
            \phi(xx^{-1}) = \phi(x)\phi(x^{-1})
        \]
        and
        \[
            \phi(xx^{-1}) = \phi(1_1) = 1_2
        \]
        so applying $\phi(x)^{-1}$ on the left on both sides $\phi(x^{-1}) = \phi(x)^{-1}$.
    \end{partquestions}

    \item Recall that $\{0_2\}$, the trivial ideal, is an ideal of $R_2$, where $0_2$ is the additive identity of $R_2$. Therefore $\ker\phi = \phi^{-1}(\{0\})$ is an ideal of $R_1$ by \myref{prop-inverse-homomorphism-on-ideal-is-ideal}.

    \item We show that $\phi$ is surjective. Note that for any $k \in \Z_n$, we have $k \leq n$. Thus, $\phi(k) = k$, so $\phi$ is surjective.

    We now find the kernel of $\phi$.
    \begin{align*}
        \ker\phi &= \{m \in \Z \vert \phi(m) = 0\}\\
        &= \{m \in \Z \vert m \equiv 0 \pmod n\}\\
        &= \{kn \vert k \in \Z\}\\
        &= n\Z.
    \end{align*}

    The FRIT (\myref{thrm-ring-isomorphism-1}) on $\phi$ tells us that
    \[
        \Z/n\Z \cong \Z_n.
    \]

    \item Note that $\phi(1) = 1$ is given. Now suppose $\phi(k) = k$ for some positive integer $k$. Then
    \[
        \phi(k+1) = \phi(k) + \phi(1) = k + 1
    \]
    by induction hypothesis and by the base case. Thus by mathematical induction we prove the statement.

    \item We borrow the result for the case where $\phi(1) = 1$ from \myref{example-endomorphisms-of-Z} to yield $\phi(n) = n$ for all integers $n$. Now note that for any positive integer $n$ we have
    \begin{align*}
        1 = \phi(1) &= \phi\left(\underbrace{\frac1n + \frac1n + \cdots + \frac1n}_{n \text{ times}}\right)\\
        &= \underbrace{\phi\left(\frac1n\right) + \phi\left(\frac1n\right) + \cdots + \phi\left(\frac1n\right)}_{n \text{ times}}\\
        &= n\phi\left(\frac1n\right)
    \end{align*}
    which means $\phi\left(\frac1n\right) = \frac1n$.

    Note that for any positive $\frac mn \in \Q$ with $m$ and $n$ as positive integers, we have
    \[
        \phi\left(\frac mn\right) = \phi(m)\phi\left(\frac1n\right) = m \times \frac1n = \frac mn
    \]
    and
    \[
        \phi\left(-\frac mn\right) = \phi(-m)\phi\left(\frac1n\right) = (-m) \times \frac1n = -\frac mn
    \]
    so $\phi(q) = q$ for any $q \in \Q$.
\end{questions}

\subsection*{Problems}
\begin{questions}
    \item \begin{partquestions}{\roman*}
        \item Consider the identity homomorphism $\id$. Clearly $\id$ is surjective. Note that
        \begin{align*}
            \ker\id &= \{r \in R \vert \id(r) = 0\}\\
            &= \{r \in R \vert r = 0\}\\
            &= \{0\}.
        \end{align*}
        By the FRIT (\myref{thrm-ring-isomorphism-1}),
        \[
            R / \{0\} \cong R
        \]
        which is what we wanted to show.

        \item By \myref{thrm-prime-ideal-iff-quotient-ring-is-integral-domain}, $\{0\}$ is prime if and only if $R/\{0\}$ is an integral domain. But since $R/\{0\} \cong R$, thus $\{0\}$ is prime if and only if $R$ is an integral domain.

        \item By \myref{thrm-maximal-ideal-iff-quotient-ring-is-field}, $\{0\}$ is maximal if and only if $R/\{0\} \cong R$ is a field.
    \end{partquestions}

    \item Let $\phi: \Q \to \Q$ be a ring endomorphism. From \myref{prop-homomorphism-on-multiplicative-identity-is-idempotent}, we know that $\phi(1)$ is an idempotent in $\Q$, meaning that $\phi(1) = 0$ or $\phi(1) = 1$.

    Clearly if $\phi(1) = 0$ then
    \[
        \phi(x) = \phi(1x) = \phi(1)\phi(x) = 0\phi(x) = 0
    \]
    so $\phi(x)$ is the trivial homomorphism.

    If instead $\phi(1) = 1$, then from \myref{exercise-homomorphism-over-Q-fixes-elements-of-Q} we know that $\phi(x) = x$ for all $x \in \Q$, i.e. $\phi$ is the identity homomorphism.

    Therefore the only two endomorphisms in $\Q$ are the trivial and identity homomorphisms.

    \item By way of contradiction, suppose $\phi: \Z^2 \to \Q$ is an isomorphism. Then \myref{prop-ring-image-of-additive-identity-is-additive-identity} tells us that $\phi((0,0)) = 0$. Also \myref{prop-homomorphism-on-multiplicative-identity-is-idempotent} tells us that $\phi((1,1))$ is an idempotent in $\Q$, which means that $\phi((1,1)) = 0$ or $\phi((1,1)) = 1$.

    If $\phi((1,1)) = 0$ then we have $\phi((0,0)) = \phi((1,1)) = 0$, so $\phi$ is not injective, which thus means that $\phi$ is not an isomorphism, a contradiction.

    Thus $\phi((1,1)) = 1$. Note that
    \begin{align*}
        \phi((1,1)) &= \phi((0,1)) + \phi((1,0)) = 1,\\
        \phi((0,0)) &= \phi((0,1)) \times \phi((1,0)) = 0
    \end{align*}
    so this means that either $\phi((0,1)) = 0$ or $\phi((1,0)) = 0$. But this means that either $\phi((0,0)) = \phi((0,1)) = 0$ or $\phi((0,0)) = \phi((1,0)) = 0$ which again contradicts the fact that $\phi$ is injective and hence an isomorphism.

    Therefore $\Z^2 \not\cong \Q$.

    \item By way of contradiction, suppose we have an isomorphism $\phi: \Q[\sqrt2] \to \Q[\sqrt3]$.

    Suppose $\phi(\sqrt2) = a + b\sqrt3$ where $a, b \in \Q$. Note that $\phi(2) = 2$ by \myref{exercise-homomorphism-over-Q-fixes-elements-of-Q}, so
    \[
        2 = \phi(2) = \phi\left(\left(\sqrt2\right)^2\right) = \left(\phi(\sqrt2)\right)^2 = a^2 + 2\sqrt3ab + 3b^2.
    \]
    So we see that $2ab = 0$, which means $a = 0$ or $b = 0$.

    If $a = 0$ then $2 = 3b^2$ which means $b = \pm\sqrt{\frac23}$. However $\sqrt{\frac23}$ is not a rational number, contradicting the fact that $b$ is a rational number. Thus $b = 0$, meaning that $\phi(\sqrt2) = a$. However, since $a \in \Q$, we also have $\phi(a) = a$, which means $\phi$ is not injective, contradicting the fact that $\phi$ is an isomorphism.

    Therefore $\Q[\sqrt2] \not\cong \Q[\sqrt3]$.

    \item Let $\phi: \Z^2 \to R, (a,b)\mapsto \begin{pmatrix}a&0\\0&b\end{pmatrix}$. We show that $\phi$ is a bijective ring homomorphism.
    \begin{itemize}
        \item \textbf{Homomorphism}: For any $(a, b), (x, y) \in \Z^2$ we see
        \begin{align*}
            \phi((a,b) + (x,y)) &= \phi((a+x,b+y))\\
            &= \begin{pmatrix}a+x&0\\0&b+y\end{pmatrix}\\
            &= \begin{pmatrix}a&0\\0&b\end{pmatrix} + \begin{pmatrix}x&0\\0&y\end{pmatrix}\\
            &= \phi((a,b)) + \phi((x,y))
        \end{align*}
        and
        \begin{align*}
            \phi((a,b)(x,y)) &= \phi((ax,by))\\
            &= \begin{pmatrix}ax&0\\0&by\end{pmatrix}\\
            &= \begin{pmatrix}a&0\\0&b\end{pmatrix}\begin{pmatrix}x&0\\0&y\end{pmatrix}\\
            &= \phi((a,b))\phi((x,y))
        \end{align*}
        so $\phi$ is a ring homomorphism.

        \item \textbf{Injective}: Let $(a, b), (x, y) \in \Z^2$ such that $\phi((a,b)) = \phi((x,y))$. Therefore $\begin{pmatrix}a&0\\0&b\end{pmatrix} = \begin{pmatrix}x&0\\0&y\end{pmatrix}$. Thus $a = x$ and $b = y$, which means $(a,b) = (x,y)$. Therefore $\phi$ is injective.

        \item \textbf{Surjective}: Let $\begin{pmatrix}a&0\\0&b\end{pmatrix} \in R$. Clearly one sees that $\phi((a, b)) = \begin{pmatrix}a&0\\0&b\end{pmatrix}$ so $\phi$ is surjective.
    \end{itemize}

    Therefore $\phi$ is a ring isomorphism, which means $R \cong \Z^2$.

    \item \begin{partquestions}{\alph*}
        \item We prove the statement. Note that since $\phi$ is bijective, thus $\phi^{-1}$ is bijective. We just need to show that $\phi^{-1}$ is a homomorphism.

        Let $u,v\in R'$. Then there exists $x,y \in R$ such that $\phi(x) = u$ and $\phi(y) = v$. Note
        \begin{align*}
            \phi^{-1}(u + v) &= \phi^{-1}(\phi(x) + \phi(y))\\
            &= \phi^{-1}(\phi(x + y))\\
            &= x + y\\
            &= \phi^{-1}(u) + \phi^{-1}(v)
        \end{align*}
        and
        \begin{align*}
            \phi^{-1}(uv) &= \phi^{-1}(\phi(x)\phi(y))\\
            &= \phi^{-1}(\phi(xy))\\
            &= xy\\
            &= \phi^{-1}(u)\phi^{-1}(v)
        \end{align*}
        so $\phi^{-1}$ is a ring homomorphism. Therefore $\phi^{-1}$ is a ring isomorphism.

        \item We prove the statement. Suppose $S$ is a subring of $R$ with $n$ elements. Consider $\phi(S)$; by \myref{prop-homomorphism-on-subring-is-subring} we know $\phi(S)$ is a subring of $R'$. Also, since $\phi$ is a bijection, thus $S$ and $\phi(S)$ are equinumerous. Therefore $\phi(S)$ is a subring of $R'$ with $n$ elements.

        \item Suppose $I$ is an ideal of $R$. Consider $\phi(I)$; by \myref{prop-homomorphism-on-subring-is-subring} we know $\phi(I)$ is a subring of $R'$. We just need to check that $\phi(I)$ is an ideal of $R'$.

        Let $r' \in R'$ and $i' \in \phi(I)$. Since $\phi$ is surjective, there is an $r \in R$ and an $i \in I$ such that $r' = \phi(r)$ and $i' = \phi(i)$. Note
        \begin{align*}
            r'i' = \phi(r)\phi(i) = \phi(\underbrace{ri}_{\text{In }I}) \in \phi(I)\\
            i'r' = \phi(i)\phi(r) = \phi(\underbrace{ir}_{\text{In }I}) \in \phi(I)
        \end{align*}
        so $\phi(I)$ is an ideal of $R'$.
    \end{partquestions}

    \item Suppose $\phi: \Z_{10}\to\Z_{10}$ is a ring endomorphism. Let $a = \phi(1)$. By \myref{prop-homomorphism-on-multiplicative-identity-is-idempotent} we know that $a^2 = \phi(1)^2 = \phi(1) = a$. Once again, we cannot assume that 0 and 1 are the only idempotents in $\Z_{10}$; by exhaustion we see that
    \begin{itemize}
        \item $0^2 = 0$;
        \item $1^2 = 1$;
        \item $5^2 = 25 = 5$; and
        \item $6^2 = 36 = 6$
    \end{itemize}
    so the idempotents in $\Z_{10}$ are 0, 1, 5, and 6.

    Recall from \myref{example-homomorphisms-from-Z12-to-Z28} that $|\phi(1)|_+$ divides $|1|_+$ (\myref{exercise-order-of-homomorphism-divides-order}) so $|\phi(1)|_+$ divides 10, and $|k|_+ = \frac{n}{\gcd(k,10)}$ (\myref{thrm-order-of-element-in-cyclic-group}). Observe that
    \begin{itemize}
        \item $|0|_+ = 1$ which divides 10;
        \item $|1|_+ = 10$ which divides 10;
        \item $|5|_+ = \frac{10}{\gcd(5,10)} = \frac{10}{5} = 2$ which divides 10; and
        \item $|6|_+ = \frac{10}{\gcd(6,10)} = \frac{10}{2} = 5$ which divides 10.
    \end{itemize}
    Therefore the possible values of $\phi(1)$ are 0, 1, 5, and 6.

    Note that
    \begin{align*}
        \phi(n) &= \phi(\underbrace{1+1+\cdots+1}_{n \text{ times}})\\
        &= \underbrace{\phi(1)+\phi(1)+\cdots+\phi(1)}_{n \text{ times}}\\
        &= \underbrace{a + a + \cdots + a}_{n \text{ times}}\\
        &= na
    \end{align*}
    so the only endomorphisms $\phi: \Z_{10} \to \Z_{10}$ are $\phi(n) = 0$, $\phi(n) = n$, $\phi(n) = 5n$, and $\phi(n) = 6n$.

    Now consider the possibility that $\psi: \Z_{10} \to \Z_{10}$ is an automorphism. We require that $\psi$ is an isomorphism.
    \begin{itemize}
        \item Clearly $\psi(n) = 0$ is not a valid isomorphism since $\psi(0) = \psi(1) = 0$ which means that $\psi$ is not an injective.
        \item $\psi(n) = n$, the identity homomorphism, is an isomorphism by \myref{exercise-identity-homomorphism-is-an-isomorphism}.
        \item $\psi(n) = 5n$ is not possible since $\psi(0) = 0$ and $\psi(2) = 10 = 0$, so $\psi$ is not injective.
        \item $\psi(n) = 6n$ is not possible since $\psi(0) = 0$ and $\psi(5) = 30 = 0$, so $\psi$ is not injective.
    \end{itemize}
    Therefore the only automorphism $\psi:\Z_{10} \to \Z_{10}$ is $\psi(n) = n$.

    \item Let $\phi: \Q[\sqrt3] \to \Q[\sqrt3]$ be an endomorphism. By \myref{prop-homomorphism-on-multiplicative-identity-is-idempotent} we know that $\phi(1)^2 = \phi(1)$, so $\phi(1) = 0$ or $\phi(1) = 1$ (since 0 and 1 are the only idempotents in $\Q[\sqrt3]$).

    If $\phi(1) = 0$ then
    \[
        \phi(a+b\sqrt3) = \phi(1)\phi(a+b\sqrt3) = 0\phi(a+b\sqrt3) = 0
    \]
    so $\phi$ is the trivial homomorphism.

    If instead $\phi(1) = 1$, then $\phi(q) = q$ for all $q \in \Q$ (\myref{exercise-homomorphism-over-Q-fixes-elements-of-Q}). Let $\phi(\sqrt3) = a+b\sqrt3$ where $a, b \in \Q$. Note
    \[
        3 = \phi(3) = \phi((\sqrt3)^3) = \left(\phi(\sqrt3)\right)^2 = a^2+2\sqrt3ab+3b^2
    \]
    which means $2ab = 0$, so $a = 0$ or $b = 0$. If $b = 0$ then $3 = a^2$ which means $a = \pm\sqrt3$, a contradiction since $\sqrt3$ is not a rational number. Therefore $a = 0$, so $3 = 3b^2$ which means $b = \pm1$. Thus $\phi(\sqrt3) = \sqrt3$ or $\phi(\sqrt3) = -\sqrt3$.

    Thus we have 3 possibilities,
    \begin{itemize}
        \item $\phi(a+b\sqrt3) = 0$;
        \item $\phi(a+b\sqrt3) = a+b\sqrt3$; and
        \item $\phi(a+b\sqrt3) = a-b\sqrt3$.
    \end{itemize}
    The first two possibilities are the trivial and identity homomorphism respectively, so we just need to check whether the last possibility is a homomorphism. Let $a + b\sqrt3, x + y\sqrt3 \in \Q[\sqrt3]$. Observe
    \begin{align*}
        \phi((a+b\sqrt3) + (x+y\sqrt3)) &= \phi((a+x)+(b+y)\sqrt3)\\
        &= (a+x)-(b+y)\sqrt3\\
        &= (a-b\sqrt3) + (x-y\sqrt3)\\
        &= \phi(a+b\sqrt3) + \phi(x+y\sqrt3)
    \end{align*}
    and
    \begin{align*}
        \phi((a+b\sqrt3)(x+y\sqrt3)) &= \phi((ax+3by)+(ay+bx)\sqrt3)\\
        &= (ax+3by)-(ay+bx)\sqrt3\\
        &= (a-b\sqrt3)(x-y\sqrt3)\\
        &= \phi(a+b\sqrt3)\phi(x+y\sqrt3)
    \end{align*}
    so $\phi(a+b\sqrt3) = a-b\sqrt3$ is indeed a homomorphism.

    Therefore the 3 possible endomorphisms are $\phi(a+b\sqrt3) = 0$, $\phi(a+b\sqrt3) = a+b\sqrt3$, and $\phi(a+b\sqrt3) = a-b\sqrt3$.

    Now consider the possibility that $\psi: \Q[\sqrt3] \to \Q[\sqrt3]$ is an automorphism, meaning that $\psi$ has to be at least an isomorphism. Clearly the trivial homomorphism is not, while the identity homomorphism is (\myref{exercise-identity-homomorphism-is-an-isomorphism}). We check that case where $\psi(a+b\sqrt3) = a-b\sqrt3$.
    \begin{itemize}
        \item \textbf{Injective}: Suppose $a+b\sqrt3, x+y\sqrt3 \in \Q[\sqrt3]$ such that $\phi(a+b\sqrt3) = \phi(x+y\sqrt3)$. Thus $a - b\sqrt3 = x - y\sqrt3$, which clearly means $a = x$ and $b = y$. Therefore $a+b\sqrt3 = x+y\sqrt3$ which means that $\psi$ is injective.
        \item \textbf{Surjective}: For any $x+y\sqrt3 \in \Q[\sqrt3]$, note that $x-y\sqrt3 \in \Q[\sqrt3]$ and $\psi(x-y\sqrt3) = x+y\sqrt3$, so $\psi$ is surjective.
    \end{itemize}
    Therefore $\psi(a+b\sqrt3) = a-b\sqrt3$ is also an isomorphism. Hence, the two automorphisms $\psi: \Q[\sqrt3] \to \Q[\sqrt3]$ are $\psi(a+b\sqrt3) = a+b\sqrt3$ and $\psi(a+b\sqrt3) = a-b\sqrt3$.

    \item \begin{partquestions}{\roman*}
        \item Let $x \in \phi(\sqrt I)$. Therefore there is an $a \in \sqrt{I}$ such that $\phi(a) = x$. Now by the definition of $\sqrt{I}$, this means that there is a positive integer $n$ such that $a^n \in I$. Note that
        \[
            x^n = \left(\phi(a)\right)^n = \phi(a^n) \in \phi(I)
        \]
        which means $x \in \sqrt{\phi(I)}$. Therefore $\phi(\sqrt I) \subseteq \sqrt{\phi(I)}$.

        \item Let $y \in \sqrt{\phi(I)}$, so there is a positive integer $n$ such that $y^n \in \phi(I)$. Thus there exists an $a \in I$ such that $\phi(a) = y^n$.

        Note $y \in R'$, and since $\phi$ is surjective, therefore there exists $x \in R$ such that $\phi(x) = y$. So
        \[
            \phi(a) = y^n = \left(\phi(x)\right)^n = \phi\left(x^n\right)
        \]
        which thus means $\phi(a-x^n) = 0$. Hence $a-x^n \in \ker\phi \subseteq I$, so $a-x^n \in I$. Since $a \in I$, this means that $x^n \in I$, so $x \in \sqrt{I}$.

        Therefore
        \[
            y = \phi(x) = \phi(\sqrt I)
        \]
        which thus means $\sqrt{\phi(I)} \subseteq \phi(\sqrt I)$. So $\phi(\sqrt I) = \sqrt{\phi(I)}$ since $\phi(\sqrt I) \subseteq \sqrt{\phi(I)}$ is given by \textbf{(i)}.
    \end{partquestions}

    \item \begin{partquestions}{\roman*}
        \item We first show $(S+I, +) \leq (R,+)$.
        \begin{itemize}
            \item $0 = 0 + 0 \in S + I$ so $S + I$ is non-empty.
            \item For any $s_1, s_2 \in S$ and $i_1, i_2 \in I$ we see that
            \[
                (s_1+i_1) - (s_2 + i_2) = (\underbrace{s_1 - s_2}_{\text{In }S}) + (\underbrace{i_1 + i_2}_{\text{In }I}) \in S + I
            \]
        \end{itemize}
        Therefore $(S+I, +) \leq (R,+)$ by the subgroup test.

        We now show $S+I$ is closed under multiplication. Let $s_1, s_2 \in S$ and $i_1, i_2 \in I$. Then
        \[
            (s_1 + i_1)(s_2 + i_2) = s_1s_2 + s_1i_2 + i_1s_2 + i_1i_2.
        \]
        Note $s_1i_2, i_1s_2 \in I$ since $I$ is an ideal, so
        \[
            (s_1 + i_1)(s_2 + i_2) = \underbrace{s_1s_2}_{\text{In }S} + \underbrace{s_1i_2 + i_1s_2 + i_1i_2}_{\text{In }I} \in S + I.
        \]

        Therefore $S+I$ is a subring of $R$.

        \item We consider the test for ideal (\myref{thrm-test-for-ideal}).
        \begin{itemize}
            \item $S \cap I$ is non-empty since $0 \in S$ and $0 \in I$ so $0 \in S \cap I$.
            \item Let $a, b \in S \cap I$. Thus $a, b \in S$ and $a, b \in I$, which means $a - b \in S$ and $a - b \in I$ as $S$ and $I$ are both subrings. Therefore $a - b \in S \cap I$.
            \item Let $s \in S$ and $a \in S \cap I$, which means $a \in S$ and $a \in I$.
            \begin{itemize}
                \item Note $sa \in S$ (because $S$ is a subring) and $sa \in I$ (because $I$ is an ideal), so $sa \in S \cap I$.
                \item Note also $as \in S$ (because $S$ is a subring) and $as \in I$ (because $I$ is an ideal), so $as \in S \cap I$.
            \end{itemize}
        \end{itemize}
        By the test for ideal, $S \cap I$ is an ideal of $S$.

        \item Define the map $\phi: S \to (S+I)/I, s \mapsto s+I$. We show that $\phi$ is a homomorphism and then find its image and kernel.
        \begin{itemize}
            \item \textbf{Homomorphism}: Let $s_1, s_2 \in S$. Then
            \begin{align*}
                \phi(s_1 + s_2) &= (s_1 + s_2) + I\\
                &= (s_1 + I) + (s_2 + I)\\
                &= \phi(s_1) + \phi(s_2)
            \end{align*}
            and
            \begin{align*}
                \phi(s_1s_2) &= (s_1s_2) + I\\
                &= (s_1+I)(s_2+I)\\
                &= \phi(s_1)\phi(s_2)
            \end{align*}
            so $\phi$ is a ring homomorphism.

            \item \textbf{Image}: We show that $\phi$ is surjective. Let $(s+i) + I \in (S+I)/I$. Note that since $i \in I$,
            \[
                (s+i)+I = (s+I) + (i+I) = (s+I) + (0+I) = s+I.
            \]
            Clearly $\phi(s) = s+I = (s+i)+I$, so for any $(s+i) + I \in (S+I)/I$ has a pre-image $s \in S$. Therefore $\im\phi = (S+I)/I$.

            \item \textbf{Kernel}: We find the kernel of $\phi$.
            \begin{align*}
                \ker\phi &= \{s \in S \vert \phi(s) = 0 + I\}\\
                &= \{s \in S \vert s + I = 0 + I\}\\
                &= \{s \in S \vert s \in I\}\\
                &= S \cap I,
            \end{align*}
            where $s + I = 0 + I$ implies $s \in I$ due to Coset Equality (\myref{lemma-coset-equality}).
        \end{itemize}

        By the FRIT (\myref{thrm-ring-isomorphism-1}),
        \[
            S/(S\cap I) \cong (S+I)/I
        \]
        which is what we wanted to prove.
    \end{partquestions}

    \item \begin{partquestions}{\roman*}
        \item We consider the test for ideal (\myref{thrm-test-for-ideal}).
        \begin{itemize}
            \item $J/I$ is non-empty since $0+I \in J/I$.
            \item Let $a+I,b+I \in J/I$. This means that $a,b \in J$, so $a - b \in J$ (since $J$ is a subring), which means $(a+I) - (b+I) = (a-b) + I \in J/I$.
            \item Let $r+I \in R/I$ and $j+I \in J/I$. Clearly $rj, jr \in J$ since $J$ is an ideal. Thus $(r+I)(j+I) = rj + I \in J/I$ and $(j+I)(r+I) = jr + I \in J/I$.
        \end{itemize}
        By the test for ideal, $J/I$ is an ideal of $R/I$.

        \item Consider the map $\phi: R/I \to R/J, r+I\mapsto r+J$. We show that $\phi$ is a well-defined homomorphism and then find its image and kernel.
        \begin{itemize}
            \item \textbf{Well-Defined}: Suppose $r_1 + I = r_2 + I \in R/I$. This means $r_1 - r_2 + I = I$, so $r_1 - r_2 \in I$ by Coset Equality (\myref{lemma-coset-equality}). Since $I \subset J$ thus $r_1 - r_2 \in J$, meaning $r_1 + J = r_2 + J$ by Coset Equality. Therefore
            \[
                \phi(r_1 + I) = r_1 + J = r_2 + J = \phi(r_2 + I)
            \]
            so $\phi$ is well defined.

            \item \textbf{Homomorphism}: Let $r_1 + I, r_2 + I \in R/I$. Note
            \begin{align*}
                \phi(r_1+r_2) &= (r_1+r_2) + I\\
                &= (r_1+I) + (r_2+I)\\
                &= \phi(r_1) + \phi(r_2)
            \end{align*}
            and
            \begin{align*}
                \phi(r_1r_2) = &= (r_1r_2) + I\\
                &= (r_1+I)(r_2+I)\\
                &= \phi(r_1)\phi(r_2)
            \end{align*}
            so $\phi$ is a ring homomorphism.

            \item \textbf{Image}: We show $\phi$ is surjective. Suppose $r + J \in R/J$. Then clearly $\phi(r+I) = r+J$ so any $r+J$ has a pre-image in $R/I$. Thus $\im\phi = R/J$.

            \item \textbf{Kernel}: We find the kernel of $\phi$.
            \begin{align*}
                \ker\phi &= \{r+I \in R/I \vert \phi(r+I) = 0+J\}\\
                &= \{r+I \in R/I \vert r+J = J\}\\
                &= \{r+I \in R/I \vert r \in J\}\\
                &= J/I
            \end{align*}
            where $r+J=J$ implies $r \in J$ due to Coset Equality (\myref{lemma-coset-equality}).
        \end{itemize}
        Finally, by FRIT (\myref{thrm-ring-isomorphism-1}),
        \[
            \frac{R/I}{J/I} \cong R/J.
        \]
    \end{partquestions}
\end{questions}
