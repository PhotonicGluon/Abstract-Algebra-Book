\chapter{Symmetry Groups}
This chapter is central to the relevance and analysis of group theory. The core result of this chapter, Cayley's theorem, links our ideas of symmetry with the idea of groups, and how groups are a form of `generalized symmetry'. It answers why group theory is oft called ``the study of symmetry'', and highlights the importance of bijections in the study of groups.

\section{Permutations}
A bijective function is too abstract an object. Such functions can take many forms. Thus, it is worth asking: what properties must a bijective function satisfy?

A bijective function is a function that maps all elements from one set to another set exactly. There are no leftovers (surjective), and each output has exactly one input that produces it (injective). In a sense, a bijective function rearranges the elements in a set; it renames elements and shuffles them around, without destroying the relative relationships between the elements.

For bijections between finite sets, each set has the same number of elements, so it is reasonable to talk about the rearrangement and enumeration of elements in such sets.
\begin{itemize}
    \item What we mean by ``rearrange'' is to rename elements. We can give elements a new name and place it in the codomain.
    \item What we mean by ``enumerate'' is to assign each element in each finite set a unique `index number'. Each element can have a unique number identifying its original position in the set, and its final position in the destination set.
\end{itemize}

Such bijections between finite sets are called \term{permutations}\index{permutations}, since they simply permute the `index number' of the elements in the sets.

\begin{example}
    Consider the set $S = \{1, 2, 3, 4, 5, 6\}$. A bijection $f: S \to S$ could perform the following mapping,
    \begin{multicols}{2}
        \begin{itemize}
            \item $1 \mapsto 2$;
            \item $2 \mapsto 4$;
            \item $3 \mapsto 3$;
            \item $4 \mapsto 5$;
            \item $5 \mapsto 1$; and
            \item $6 \mapsto 6$.
        \end{itemize}
    \end{multicols}
    In this case, the function $f$ is said to be a permutation because the list $[2, 4, 3, 5, 1, 6]$ is one rearrangement of the items in the set $\{1, 2, 3, 4, 5, 6\}$.
\end{example}

\begin{remark}
    It is certainly confusing that the operation of rearranging the items is also called ``permuting'' the items in the set, and one such rearrangement is called a permutation. For groups, treat a ``permutation'' as a bijective function between finite groups.
\end{remark}

Permutations come in many different forms, but the core thing that they do is to rearrange items. From the above example, one could form a `cycle' of how each item is mapped to another, i.e.
\begin{itemize}
    \item $1 \mapsto 2 \mapsto 4 \mapsto 5 \mapsto 1$; and
    \item $3 \mapsto 3$.
\end{itemize}
We formally describe what a cycle is.
\begin{definition}
    A \term{cycle}\index{cycle} is a permutation that cyclically permutes the integers in the cycle and fixes all other integers. In particular, for distinct $a_1, a_2, \dots, a_n$, the cycle
    \[
        \pi = \begin{pmatrix}a_1&a_2&a_3&\cdots&a_n\end{pmatrix}
    \]
    is a permutation of \term{length}\index{cycle!length} $n$ such that
    \[
        \pi(a_i) = \begin{cases}
            a_{i+1}& \text{if } 1 \leq i \leq n - 1\\
            a_1 & \text{if } i = n
        \end{cases}
    \]
\end{definition}
\begin{example}
    The cycle $f = \begin{pmatrix}1 & 2 & 4 & 5\end{pmatrix}$ is the map where
    \begin{multicols}{3}
        \begin{itemize}
            \item $1 \mapsto 2$;
            \item $2 \mapsto 4$;
            \item $3 \mapsto 3$;
            \item $4 \mapsto 5$;
            \item $5 \mapsto 1$; and
            \item $n \mapsto n$ for $n \geq 6$.
        \end{itemize}
    \end{multicols}
\end{example}

We can describe a permutation $\phi$ based on its cycles; such a description is the \term{cycle decomposition}\index{cycle!decomposition} (or \term{cycle notation}\index{cycle!notation}) of the permutation. For brevity let $S = \{1, 2, 3, \dots, n\}$. The algorithm that produce the cycle decomposition is as follows.
\begin{enumerate}
    \item If all elements of $S$ are present in some cycle, go to step 8.
    \item Otherwise, pick the smallest element of $S$ that has not appeared in a previous cycle; call it $m$. Set $a = m$.
    \item Begin the new cycle with ``$(a$'', placing it to the right of any existing cycle.
    \item Set $b = \phi(a)$.
    \item If $b = m$, end the cycle with a right parenthesis and return to step 1.
    \item Otherwise, write $b$ next to $a$.
    \item Set $a = b$ and go to step 4.
    \item Remove all cycles of length 1.
    \item If there are no cycles left, denote the permutation by $\id$.
\end{enumerate}

\newpage

\begin{example}
    Let $S = \{1,2,3,4,5\}$. Consider the map $f: S \to S$, where
    \begin{itemize}
        \item $1 \mapsto 2$;
        \item $2 \mapsto 4$;
        \item $3 \mapsto 5$;
        \item $4 \mapsto 1$; and
        \item $5 \mapsto 3$.
    \end{itemize}
    We describe how to find the cycle decomposition of $f$.
    \begin{itemize}
        \item Not all elements of $S$ are present in a cycle, so we continue.
        \item The smallest element of $S$ not used is 1; set $a = m = 1$.
        \item We begin the cycle with ``$(1$''.
        \item Set $b = f(1) = 2$.
        \item As $b = 2 \neq m = 1$, we write $b$ next to $a$: ``$(1\quad2$''.
        \item Set $a = 2$; head back to step 4.
        \item Set $b = f(2) = 4$.
        \item As $b = 4 \neq m = 1$, we write $b$ next to $a$: ``$(1\quad2\quad4$''.
        \item Set $a = 4$; head back to step 4.
        \item Set $b = f(4) = 1$.
        \item As $b = 1 = m = 1$, we end the cycle: ``$(1\quad2\quad4)$''. Return to step 1.
        \item There are elements of $S$ that are not present in a cycle, so we continue.
        \item The smallest unused element of $S$ is 3; set $a = m = 3$.
        \item We begin the new cycle with ``$(3$'', so our current decomposition is ``$(1\quad2\quad4)(3$''.
        \item Set $b = f(3) = 5$.
        \item As $b = 5 \neq m = 3$, we write $b$ next to $a$: ``$(1\quad2\quad4)(3\quad5$''.
        \item Set $a = 5$ and head back to step 4.
        \item Set $b = f(5) = 3$.
        \item As $b = 3 = m = 3$, we end the cycle: ``$(1\quad2\quad4)(3\quad5)$''.
        \item All elements of $S$ are present in some cycle.
        \item No cycle contains one element.
    \end{itemize}
    Therefore the cycle decomposition of $f$ is $\begin{pmatrix}1&2&4\end{pmatrix}\begin{pmatrix}3&5\end{pmatrix}$.
\end{example}

\newpage

\begin{example}
    Let the permutation $\alpha$ have cycle decomposition $\begin{pmatrix}1 & 3 & 5 & 2\end{pmatrix}$. Then
    \begin{multicols}{3}
        \begin{itemize}
            \item $\alpha(1) = 3$;
            \item $\alpha(2) = 1$;
            \item $\alpha(3) = 5$;
            \item $\alpha(4) = 4$;
            \item $\alpha(5) = 2$; and
            \item $\alpha(n) = n$ for $n \geq 6$.
        \end{itemize}
    \end{multicols}
\end{example}

\begin{example}
    Let the permutation $\beta$ have cycle decomposition $\begin{pmatrix}1 & 6 & 2 & 9 & 7 & 4\end{pmatrix}$. Then
    \begin{multicols}{3}
        \begin{itemize}
            \item $\beta(1) = 6$;
            \item $\beta(2) = 9$;
            \item $\beta(3) = 3$;
            \item $\beta(4) = 1$;
            \item $\beta(5) = 5$;
            \item $\beta(6) = 2$;
            \item $\beta(7) = 4$;
            \item $\beta(8) = 8$;
            \item $\beta(9) = 7$; and
            \item $\beta(n) = n$ for $n \geq 10$.
        \end{itemize}
    \end{multicols}
\end{example}

\begin{exercise}
    Find the cycle decomposition of the following permutations.
    \begin{partquestions}{\alph*}
        \item $1 \mapsto 2$, $2 \mapsto 3$, $3 \mapsto 1$
        \item $1 \mapsto 3$, $2 \mapsto 2$, $3 \mapsto 1$
        \item $1 \mapsto 3$, $2 \mapsto 4$, $3 \mapsto 1$, $4 \mapsto 5$, $5 \mapsto 2$
    \end{partquestions}
\end{exercise}

We now look at composing permutations\index{permutation!composing}.
\begin{example}
    Let $f$ be a permutation with cycle decomposition $\begin{pmatrix}1 & 3 & 5 & 2\end{pmatrix}$ and let $g$ be a permutation with cycle decomposition $\begin{pmatrix}2 & 4 & 3\end{pmatrix}$. Then $h = fg$ is a permutation where
    \begin{itemize}
        \item $h(1) = f(g(1)) = f(1) = 3$;
        \item $h(2) = f(g(2)) = f(4) = 4$;
        \item $h(3) = f(g(3)) = f(2) = 1$;
        \item $h(4) = f(g(4)) = f(3) = 5$; and
        \item $h(5) = f(g(5)) = f(5) = 2$.
    \end{itemize}

    So $h$ maps 1 to 3, 2 to 4, 3 to 1, 4 to 5, and 5 to 2, meaning
    \[
        h = \underbrace{\begin{pmatrix}1 & 3 & 5 & 2\end{pmatrix}}_{f}\underbrace{\begin{pmatrix}2 & 4 & 3\end{pmatrix}}_{g} = \begin{pmatrix}1 & 3\end{pmatrix}\begin{pmatrix}2 & 4 & 5\end{pmatrix}.
    \]
\end{example}

\begin{example}
    We have $ \begin{pmatrix}2 & 9 & 7 & 4\end{pmatrix}\begin{pmatrix}1 & 6 & 4\end{pmatrix} = \begin{pmatrix}1 & 6 & 2 & 9 & 7 & 4\end{pmatrix}$.
\end{example}

We now describe how to find the inverse of a permutation\index{permutation!inverse}. Given a cycle decomposition for the permutation $f$, simply write the cycle notation backwards, ensuring that the smallest element remains at the front.

\begin{example}
    The inverse of $\begin{pmatrix}1 & 8 & 4 & 2\end{pmatrix}$ is $\begin{pmatrix}2 & 4 & 8 & 1\end{pmatrix}$, which is $\begin{pmatrix}1 & 2 & 4 & 8\end{pmatrix}$.
\end{example}

\begin{example}
    $\begin{pmatrix}1 & 7 & 5 & 3 & 9\end{pmatrix}^{-1} = \begin{pmatrix}1 & 9 & 3 & 5 & 7\end{pmatrix}$.
\end{example}

\begin{exercise}
    Consider the permutation
    \[
        \pi = \begin{pmatrix}1 & 5 & 2\end{pmatrix}\begin{pmatrix}2 & 5 & 3 & 4\end{pmatrix}.
    \]
    Find the inverse of $\pi$.
\end{exercise}

\section{The Symmetric Group of a Set}
With the definition of permutations out of the way, we can finally introduce a very important type of group: the symmetric group of a set $X$.

\begin{definition}
    Let $X$ be a non-empty set. The \term{symmetric group of $X$}\index{symmetric group}\index{symmetric group!of a set $X$} is
    \[
        \Sym{X} = \{f: X \to X \vert f \text{ is a bijection}\}
    \]
    under function composition (which is denoted by $\circ$).
\end{definition}
\begin{remark}
    Elements of $\Sym{X}$ are called permutations.
\end{remark}
\begin{remark}
    We usually suppress the function composition operator when working with permutations in $\Sym{X}$.
\end{remark}

\begin{proposition}
    Let $X$ be a non-empty set. Then $\Sym{X}$ is indeed a group.
\end{proposition}
\begin{proof}
    We prove the 4 group axioms.
    \begin{itemize}
        \item \textbf{Closure}: Let $f, g \in \Sym{X}$, so $f: X\to X$ and $g:X \to X$ are bijective functions. Define $h:X \to X$ where $h = f\circ g$. By \myref{exercise-composition-of-isomorphisms-is-isomorphisms} we know $h$ is bijective. Thus $\Sym{X}$ is closed under $\circ$.

        \item \textbf{Associativity}: Function composition is associative by \myref{axiom-function-composition-associative}.

        \item \textbf{Identity}: Clearly the identity map, $\id$, is in $\Sym{X}$. Also $\id$ is a bijection by \myref{exercise-identity-map-is-isomorphism}. Now we show that $\id$ is indeed the identity in $\Sym{X}$. Let $x \in X$ and $f \in \Sym{X}$. Then
        \[
            (\id \circ f)(x) = \id(f(x)) = f(x)
        \]
        and
        \[
            (f \circ \id)(x) = f(\id(x)) = f(x)
        \]
        so $\id$ is the identity in $\Sym{X}$.

        \item \textbf{Inverse}: For all functions $f \in \Sym{X}$, note that $f^{-1}$ exists since $f$ is a bijection. Furthermore, $f^{-1}$ is a bijection from $X$ to $X$, so $f^{-1} \in \Sym{X}$. Hence, by the definition of $f^{-1}$, we see
        \[
            f \circ f^{-1} = f^{-1} \circ f = \id
        \]
        so $f^{-1}$ is indeed the inverse of $f \in \Sym{X}$.
    \end{itemize}
    Therefore $(\Sym{X}, \circ)$ is a group.
\end{proof}

The most relevant type of symmetric group we encounter when working with finite groups is the symmetric group of degree $n$.

\begin{definition}
    The \term{symmetric group of degree $n$}\index{symmetric group!of degree $n$} (or the \term{symmetric group of $n$ letters}\index{symmetric group!of $n$ letters}) is denoted by $\Sn{n}$ and is the group $\Sym{\{1, 2, 3, \dots, n\}}$.
\end{definition}

\begin{example}\label{example-symmetric-group-of-degree-3}
    Consider the symmetric group of degree 3, $\Sn{3}$. We show all permutations of $\Sn{3}$.

    \begin{figure}[h]
        \centering
        \pdfteximgframed{0.35\textwidth}{part1/images/symmetry-groups/maps-of-s3.pdf_tex}
        \caption{All Permutations of $\Sn{3}$}
    \end{figure}

    Thus $|\Sn{3}| = 6$.
\end{example}
\begin{exercise}\label{exercise-order-of-Sn}
    Explain why $|\Sn{n}| = n!$.
\end{exercise}

It should also be noted that subgroups of $\Sym{X}$ are called \term{permutation groups}\index{permutation groups} because they contain permutations.

To end this section, we prove an important result regarding symmetric groups of equal order.
\begin{theorem}\label{thrm-symmetric-groups-of-same-order-are-isomorphic}
    Let $X_1$ and $X_2$ be two non-empty finite sets with cardinality $n$. Then we have $\Sym{X_1} \cong \Sym{X_2}$.
\end{theorem}
\begin{proof}
    Since the two sets $X_1$ and $X_2$ have the same cardinality, they have the same number of elements. Hence there exists a bijection $f: X_1 \to X_2$.

    Let the map $\phi: \Sym{X_1} \to \Sym{X_2}$ where $\phi(\sigma) = f\sigma f^{-1}$ for all $\sigma \in \Sym{X_1}$. We show that $\phi$ is an isomorphism.
    \begin{itemize}
        \item \textbf{Homomorphism}: Let $\sigma, \pi \in \Sym{X_1}$. Then
        \begin{align*}
            \phi(\sigma\pi) &= f(\sigma\pi)f^{-1}\\
            &= f\sigma(f^{-1}f)\pi f^{-1}\\
            &= (f\sigma f^{-1})(f\pi f^{-1})\\
            &= \phi(\sigma)\phi(\pi)
        \end{align*}
        which shows that $\phi$ is a homomorphism.

        \item \textbf{Injective}: Let $\sigma, \pi \in \Sym{X_1}$ such that $\phi(\sigma) = \phi(\pi)$. Thus $f\sigma f^{-1} = f\pi f^{-1}$ which quickly implies $\sigma = \pi$ by cancellation law. Therefore $\phi$ is injective.

        \item \textbf{Surjective}: Let $\pi \in \Sym{X_2}$. Let the map $\sigma = f^{-1}\pi f$. We note that $\sigma: X_1 \to X_1$ is a bijection, so $\sigma \in \Sym{X_1}$. Clearly
        \[
            \phi(\sigma) = \phi(f^{-1}\pi f) = f(f^{-1}\pi f)f^{-1} = \pi
        \]
        so $\sigma$ is the pre-image of $\pi$. Hence $\sigma$ is the pre-image of $\pi$.
    \end{itemize}
    Therefore $\phi$ is an isomorphism, which means that $\Sym{X_1} \cong \Sym{X_2}$.
\end{proof}

\begin{corollary}\label{corollary-symmetric-group-of-finite-order}
    Let $A$ be a non-empty finite set with cardinality $n$. Then $\Sym{A} \cong \Sn{n}$.
\end{corollary}
\begin{proof}
    Let $X = \{1, 2, 3, \dots, n\}$. We note that $\Sn{n} = \Sym{X}$ by definition, and that $|X| = n$. We also note that $|A| = n$ so $A$ and $X$ are two finite sets with equal cardinality. Result follows from \myref{thrm-symmetric-groups-of-same-order-are-isomorphic}.
\end{proof}

\section{Cayley's Theorem}
We now have sufficient background to state and prove Cayley's theorem.

\begin{theorem}[Cayley]\label{thrm-cayley}\index{Cayley's Theorem}
    Every group is isomorphic to a permutation group.
\end{theorem}

The statement of the theorem, although simple, is the reason why group theorists study group theory: to explore all the ways that a group can be symmetric.

The proof of this theorem is involved and technical, but we'll try and simplify its proof as much as possible. We first give an outline of the proof, before beginning proper.
\begin{proofsketch}
    For any group $G$, define a bijection that involves a group element $g$. Create a new set $H$ consisting of these bijections, enumerating over all $g \in G$. Show that $H$ is actually a permutation group, i.e. a subgroup of $\Sym{G}$. Finally, show that $G$ is actually isomorphic to $H$, proving the theorem.
\end{proofsketch}
\begin{proof}
    Let $G$ be any group. We want to prove that there exists a group of bijective functions from $G$ to $G$ that is isomorphic to $G$ (i.e., a permutation group).

    For any $g \in G$ define the map $\lambda_g: G \to G$ such that $x \mapsto gx$. We claim that $\lambda_g$ is a bijection.
    \begin{itemize}
        \item \textbf{Injective}: Let $x, y \in G$ such that $\lambda_g(x) = \lambda_g(y)$. Then $gx = gy$ by definition of $\lambda_g$, which immediately means $x = y$ by cancellation law. Thus $\lambda_g$ is injective.

        \item \textbf{Surjective}: Let $y \in G$. Note that $g^{-1}y \in G$ (since $G$ is closed), and that $\lambda_g(g^{-1}y) = g(g^{-1}y) = y$. Thus, a preimage of $y$ is $g^{-1}y$ and it exists in the domain $G$, meaning $\lambda_g$ is surjective.
    \end{itemize}
    Since $\lambda_g$ is both injective and surjective it is thus bijective.

    Now let $H = \{\lambda_g \vert g \in G\}$. Since $\lambda_g$ is a bijection from $G$ to $G$ for any $g \in G$, thus $H \subseteq \Sym{G}$. We show that $H \leq \Sym{G}$ via the subgroup test (\myref{thrm-subgroup-test}).

    We first note that $\lambda_e = \id$ since
    \[
        \lambda_e(x) = ex = x = \id(x)
    \]
    for all $x$ in $G$, meaning $\id = \lambda_e \in H$. Also, the inverse of any function $\lambda_g$ is $\lambda_{g^{-1}}$ since
    \[
        (\lambda_g \circ \lambda_{g^{-1}})(x) = gg^{-1}x = x = \lambda_e(x)
    \]
    and
    \[
        (\lambda_{g^{-1}} \circ \lambda_g)(x) = g^{-1}gx = x = \lambda_e(x).
    \]
    Now suppose $\lambda_{g_1}, \lambda_{g_2} \in H$ (where $g_1, g_2 \in G$). Note $g_1g_2^{-1} \in G$ since $G$ is closed. Therefore, for all $x \in G$,
    \begin{align*}
        \left(\lambda_{g_1} \circ \left(\lambda_{g_2}\right)^{-1}\right)(x) &= \lambda_{g_1}\circ\lambda_{g_2^{-1}}(x)\\
        &= g_1g_2^{-1}x\\
        &= \lambda_{g_1g_2^{-1}}(x).
    \end{align*}
    Note $\lambda_{g_1g_2^{-1}}$ is clearly an element of $H$. Thus if $\lambda_{g_1}, \lambda_{g_2} \in H$, then $\left(\lambda_{g_1} \circ \left(\lambda_{g_2}\right)^{-1}\right) \in H$. Therefore by the subgroup test we have $H \leq \Sym{G}$.

    We finally show that $G \cong H$ by considering the map $\phi: G\to H$ where $g \mapsto \lambda_g$. We need to show that $\phi$ is an isomorphism.
    \begin{itemize}
        \item \textbf{Homomorphism}: For any $x \in G$,
            \begin{align*}
                \phi(gh)(x) &= \lambda_{gh}(x)\\
                &= ghx\\
                &= g(hx)\\
                &= \lambda_g\left(\lambda_h(x)\right)\\
                &= \lambda_g\circ\lambda_h(x)\\
                &= (\phi(g)\phi(h))(x).
            \end{align*}
            Thus, $\phi(gh) = \phi(g)\phi(h)$.

        \item \textbf{Injective}: Let $g_1, g_2 \in G$ such that $\phi(g_1) = \phi(g_2)$. Then $\lambda_{g_1} = \lambda_{g_2}$. Therefore, $\lambda_{g_1}(x) = \lambda_{g_2}(x)$ for all $x \in G$, which means that $\lambda_{g_1}(e) = \lambda_{g_2}(e)$. By definition of $\lambda_g$, we have $eg_1 = eg_2$ which means that $g_1=g_2$. Thus $\phi$ is injective.

        \item \textbf{Surjective}: Let $\lambda_g \in H$. Clearly $\phi(g) = \lambda_g$, which means $\lambda_g$ has a pre-image of $g$. So $\phi$ is surjective.
    \end{itemize}
    Therefore we have proven that $\phi$ is an isomorphism, which means that
    \[
        G \cong H \leq \Sym{G},
    \]
    that is, any group $G$ is isomorphic to a subgroup of the symmetric group of $G$. So $G$ is isomorphic to a permutation group.
\end{proof}

We note one corollary of this theorem.
\begin{corollary}
    Let $G$ be a finite group of order $n$. Then there exists a group $H \leq \Sn{n}$ such that $G \cong H$.
\end{corollary}
\begin{proof}
    By Cayley's theorem (\myref{thrm-cayley}), there exists a group $H \leq \Sym{G}$ such that $G \cong H$. Now since $G$ is finite with order $n$, thus $\Sym{G} \cong \Sn{n}$ by \myref{corollary-symmetric-group-of-finite-order}. Hence $H \leq \Sn{n}$ and $G \cong H$.
\end{proof}

One might ask what the use of Cayley's theorem is in group theory. To put it simply, it is a sanity check on the definition of a group. Before anyone had the idea of writing down the axioms for groups, people studied collections of bijections of sets closed under composition and inverses. Cayley's theorem tells us that every abstract group is a type of the above collection, so the axioms of group theory capture the concrete phenomenon that groups were designed to capture.

\newpage

\section{Problems}
\begin{problem}
    Let the permutations
    \begin{align*}
        &\alpha = \begin{pmatrix}1 & 5 & 2 & 3\end{pmatrix},\\
        &\beta  = \begin{pmatrix}1 & 5 & 2\end{pmatrix}\begin{pmatrix}3 & 4\end{pmatrix},\\
        &\gamma = \begin{pmatrix}1 & 2 & 5\end{pmatrix}\begin{pmatrix}3 & 4\end{pmatrix}, \text{ and}\\
        &\delta = \begin{pmatrix}1 & 3 & 2 & 5\end{pmatrix}.
    \end{align*}
    What is $\alpha\beta\gamma\delta$?
\end{problem}

\begin{problem}
    Prove that $\Sn{3}$ is isomorphic to $D_3$, the dihedral group of order 6.
\end{problem}

\begin{problem}
    State the number of elements in $\Sn{4}$.
    \begin{partquestions}{\alph*}
        \item Let $G$ be the cyclic group of order 4. Cayley's theorem (\myref{thrm-cayley}) says that it is isomorphic to a subgroup of $\Sn{4}$. Find one such subgroup of $\Sn{4}$ and prove that it is, indeed, isomorphic to $G$.
        \item Let $G$ be the group with presentation
        \[
            \langle a, b \vert a^2 = b^2 = (ab)^2 = e \rangle.
        \]
        Cayley's theorem says that it is isomorphic to a subgroup of $\Sn{4}$. Find one such subgroup of $\Sn{4}$ and prove that it is, indeed, isomorphic to $G$.
    \end{partquestions}
\end{problem}
