\chapter{Basics of Groups}
With a basic intuition of the definition of groups, we look at the basic properties satisfied by groups and introduce two simple types of groups.

\section{Basic Examples of Groups}
Recall that a group satisfies four axioms.
\begin{itemize}
    \item \textbf{Closure}: For all $a, b \in G$ we have $a \ast b \in G$.
    \item \textbf{Associativity}: For all $a, b, c \in G$ we have $a \ast (b \ast c) = (a \ast b) \ast c$.
    \item \textbf{Identity}: There exists $e \in G$, called the identity element\index{group!identity}, such that for any $x \in G$ we have $e \ast x = x \ast e = x$.
    \item \textbf{Inverse}: For every $x \in G$ there exists $x^{-1} \in G$ such that $x \ast x^{-1} = x^{-1} \ast x = e$.
\end{itemize}

Recall also we write $a \ast b$ as $ab$, and denote the group $(G, \ast)$ by $G$ only.
\begin{remark}
    If the group operation is additive\index{group!additive}, we write $a + b$ instead of $ab$.
\end{remark}

Let's look at some examples of groups.
\begin{example}
    Let $\Z$ be the set of integers and let $+$ denote regular addition. Then $(\Z, +)$ forms a group.
    \begin{itemize}
        \item \textbf{Closure}: For all integers $a$ and $b$, we know $a + b$ is an integer.
        \item \textbf{Associativity}: By \myref{axiom-addition-is-associative} we know for all integers $a$, $b$, and $c$ that $a + (b + c) = (a + b) + c$.
        \item \textbf{Identity}: The identity is 0, since $0 + x = x + 0 = x$.
        \item \textbf{Inverse}: Since $x + (-x) = (-x) + x = 0$, thus the inverse of the integer $x$ is $-x$.
    \end{itemize}
\end{example}

An important thing to note about addition is that it is commutative (by \myref{axiom-addition-is-commutative}). A group with a commutative operation is called a \term{commutative group}\index{group!commutative}. However, it is more often called an \term{abelian group}\index{group!abelian}, named after Norwegian mathematician Niels Henrik Abel.

We now look at $\Z_n$, the set of integers modulo $n$.
\begin{definition}
    Define $\Z_n$ to be the set $\{0, 1, 2, \dots, n-1\}$ where $n$ is a non-negative integer.
\end{definition}
\begin{definition}
    Define $\oplus_n$ to be addition modulo $n$. That is, $a \oplus_n b = (a + b) \mod{n}$ for any integers $a$ and $b$.
\end{definition}
\begin{proposition}\label{prop-Zn-is-abelian-group}
    The set $\Z_n$ with the operation $\oplus_n$ forms an abelian group.
\end{proposition}
\begin{proof}
    We prove this by first showing that the group axioms hold.
    \begin{itemize}
        \item \textbf{Closure}: For any $a, b \in \Z_n$ we know $a \oplus_n b = (a + b) \mod{n}$ is a integer between 0 and $n - 1$. Thus $a \oplus b \in Z_n$.
        \item \textbf{Associativity}: For any $a, b, c \in \Z_n$, since addition is associative (\myref{axiom-addition-is-associative}), thus $a \oplus_n (b \oplus_n c) = (a + (b + c)) \mod{n} = ((a + b) + c) \mod{n} = (a \oplus_n b) \oplus_n c$.
        \item \textbf{Identity}: The identity is $0$ since for every $x \in \Z_n$, $0 \oplus_n x = (0 + x) \mod{n} = (x + 0) \mod{n} = x \oplus_n 0 = x$.
        \item \textbf{Inverse}:
        \begin{itemize}
            \item $0$ is its own inverse since $0 \oplus_n 0 = 0$ which is the identity.
            \item For any other integer $x \in \Z_n$, its inverse is $n - x$. Since $1 \leq x \leq n - 1$, thus $1 \leq n - x \leq n - 1$ so $n - x \in \Z_n$. Also, $x \oplus_n (n - x) = (x + (n - x)) \mod{n} = n \mod{n} = 0$ and $(n - x) \oplus_n x = ((n-x) + x)\mod{n} = n \mod{n} = 0$.
        \end{itemize}
    \end{itemize}
    Since the four group axioms are satisfied, this is a group. Furthermore, as addition is commutative (\myref{axiom-addition-is-commutative}), thus addition modulo $n$ is commutative. Therefore $(\Z_n, \oplus_n)$ is a commutative group.
\end{proof}
\begin{remark}
    Some sources (e.g. {\cite[\S 33]{clark_1984}} and {\cite[Proposition 2.31]{humphreys_1996}}) define $\Z_n$ as a set of congruence classes modulo $n$, i.e. $\Z/n\Z$. However we follow \cite[p.~42]{gallian_2016} and define $\Z_n$ as just the set of integers from 0 to $n - 1$ inclusive. We will show that these two definitions are equivalent in a future chapter.
\end{remark}

One could use a \term{Cayley table}\index{cayley table} (or \term{group table}\index{group table}) to show that a structure is a group.
\begin{example}
    We draw the Cayley table of $(\Z_6, \oplus_6)$ to show that it is a group.
    \begin{table}[h]
        \centering
        \begin{tabular}{|l|l|l|l|l|l|l|}
        \hline
        $\boldsymbol{\oplus_6}$ & $\boldsymbol{0}$ & $\boldsymbol{1}$ & $\boldsymbol{2}$ & $\boldsymbol{3}$ & $\boldsymbol{4}$ & $\boldsymbol{5}$ \\ \hline
        $\boldsymbol{0}$ & 0 & 1 & 2 & 3 & 4 & 5 \\ \hline
        $\boldsymbol{1}$ & 1 & 2 & 3 & 4 & 5 & 0 \\ \hline
        $\boldsymbol{2}$ & 2 & 3 & 4 & 5 & 0 & 1 \\ \hline
        $\boldsymbol{3}$ & 3 & 4 & 5 & 0 & 1 & 2 \\ \hline
        $\boldsymbol{4}$ & 4 & 5 & 0 & 1 & 2 & 3 \\ \hline
        $\boldsymbol{5}$ & 5 & 0 & 1 & 2 & 3 & 4 \\ \hline
        \end{tabular}
    \end{table}

    One observes from the Cayley table that
    \begin{itemize}
        \item for all $x, y \in \Z_6$ we have $x \oplus_6 y \in \Z_6$;
        \item for all $x, y, z \in \Z_6$ we have $x \oplus_6 (y \oplus_6 z) = (x \oplus_6 y) \oplus_6 z$;
        \item 0 is the identity since adding anything to it gives the original number; and
        \item every row has an integer that, when added, gives 0.
    \end{itemize}
    Thus $(\Z_6, \oplus_6)$ is a group.
\end{example}

It should be noted that, in this book, we use the convention of reading the row before the column in a group table. However, since $(\Z_6, \oplus_6)$ is an abelian group, the order does not matter. We will look at Cayley tables of non-abelian groups later.

\begin{definition}
    Let $\otimes_n$ denote multiplication modulo $n$. That is, $a \otimes_n b = (a \times b) \mod{n}$ for any integers $a$ and $b$.
\end{definition}
\begin{example}
    Let the set $A = \{1, 2, 3, 4\}$. We draw the Cayley table of $(A, \otimes_5)$ to show that it is a group.
    \begin{table}[h]
        \centering
        \begin{tabular}{|l|l|l|l|l|l|l|}
        \hline
        $\boldsymbol{\otimes_5}$ & $\boldsymbol{1}$ & $\boldsymbol{2}$ & $\boldsymbol{3}$ & $\boldsymbol{4}$ \\ \hline
        $\boldsymbol{1}$ & 1 & 2 & 3 & 4 \\ \hline
        $\boldsymbol{2}$ & 2 & 4 & 1 & 3 \\ \hline
        $\boldsymbol{3}$ & 3 & 1 & 4 & 2 \\ \hline
        $\boldsymbol{4}$ & 4 & 3 & 2 & 1 \\ \hline
        \end{tabular}
    \end{table}

    One observes from the Cayley table that
    \begin{itemize}
        \item for all $x, y \in A$ we see $x \otimes_5 y \in A$;
        \item for all $x, y, z \in A$ we see $x \otimes_5 (y \otimes_5 z) = (x \otimes_5 y) \otimes_5 z$;
        \item 1 is the identity since multiplying anything to it gives the original number; and
        \item every row has an integer that, when multiplied, gives 1.
    \end{itemize}
    Thus $(A, \otimes_5)$ is a group.
\end{example}

\begin{exercise}
    By using a Cayley table, show that $(\Z_6, \otimes_6)$ does not form a group.
\end{exercise}

\section{General Properties of Groups}
\begin{proposition}
    The identity of a group is unique.
\end{proposition}
\begin{proof}
    Suppose $e_1$ and $e_2$ are identities of a group $G$. Then, for all $x \in G$, we have
    \[
        e_1x = xe_1 = x \text{ and } e_2x = xe_2 = x,
    \]
    since $e_1$ and $e_2$ are identities. Thus,
    \begin{align*}
        e_1 &= e_1e_2 & (e_2 \text{ is an identity, so } xe_2 = x)\\
        &= e_2 & (e_1 \text{ is an identity, so } xe_1 = x)
    \end{align*}
    which means $e_1 = e_2$. Hence the identity of a group is unique.
\end{proof}

\begin{proposition}
    The inverse of any element in a group is unique.
\end{proposition}
\begin{proof}
    Suppose that $a$ and $b$ are inverses of $x$ in a group $G$. Then by definition, $ax = xa = e$ and $bx = xb = e$. So,
    \begin{align*}
        a &= ae & (e \text{ is the identity})\\
        &= a(xb) & (b \text{ is an inverse, so } xb = e)\\
        &= (ax)b & (\textbf{Associativity})\\
        &= eb & (a \text{ is an inverse, so } ax = e)\\
        &= b & (e \text{ is the identity})
    \end{align*}
    Therefore $a = b$. Thus the inverse of $x$ is unique.
\end{proof}

\begin{proposition}\label{prop-inverse-of-identity-is-identity}
    Let $G$ be a group with identity $e$. Then $e^{-1} = e$.
\end{proposition}
\begin{proof}
    Clearly $e = ee^{-1}$ since $xx^{-1} = e$ for all elements $x \in G$, including $x = e$. Thus, by left multiplying $e^{-1}$ on both sides, we see $e = e^{-1}$.
\end{proof}

\begin{proposition}\label{prop-inverse-of-inverse-is-element}
    Let $G$ be a group. Then $\left(x^{-1}\right)^{-1} = x$ for all $x \in G$.
\end{proposition}
\begin{proof}
    See \myref{exercise-inverse-of-inverse-is-element} (later).
\end{proof}

\begin{theorem}[Shoes and Socks]\index{Shoes and Socks Theorem}
    Let $G$ be a group. For all elements $a, b \in G$ we have $(ab)^{-1} = b^{-1}a^{-1}$.
\end{theorem}
\begin{proof}
    Recall that $g^{-1}$ is the inverse of $g$ if and only if $gg^{-1} = g^{-1}g = e$. We show that $(ab)(b^{-1}a^{-1}) = e$ and $(b^{-1}a^{-1})(ab) = e$.
    \begin{align*}
        (ab)(b^{-1}a^{-1}) &= a(bb^{-1})a^{-1} & (\textbf{Associativity})\\
        &= a(e)a^{-1} & (bb^{-1} = e)\\
        &= aa^{-1} & (e \text{ is the identity})\\
        &= e, & (aa^{-1} = e)\\
        (b^{-1}a^{-1})(ab) &= b^{-1}(a^{-1}a)b & (\textbf{Associativity})\\
        &= b^{-1}(e)b & (a^{-1}a = e)\\
        &= b^{-1}b & (e \text{ is the identity})\\
        &= e & ( b^{-1}b = e).
    \end{align*}
    Thus, $b^{-1}a^{-1}$ is the inverse of $ab$, i.e. $(ab)^{-1} = b^{-1}a^{-1}$.
\end{proof}

\begin{exercise}\label{exercise-inverse-of-inverse-is-element}
    Prove \myref{prop-inverse-of-inverse-is-element}.
\end{exercise}

We now prove the cancellation law, an important property of groups.

\begin{proposition}[Cancellation Law]\index{cancellation law!groups}
    Let $G$ be a group, and let $g, x, y \in G$. Then the following statements are equivalent.
    \begin{enumerate}
        \item $x = y$
        \item $gx = gy$
        \item $xg = yg$
    \end{enumerate}
\end{proposition}

\begin{proof}
    We prove the statements in order.
    \begin{itemize}
        \item $\boxed{(1) \implies (2)}$ Given $x = y$, applying $g$ on the left on both sides yields $gx = gy$.

        \item $\boxed{(2) \implies (3)}$ Given $gx = gy$. Applying
        $g^{-1}$ on the left on both sides yields $g^{-1}(gx) = g^{-1}(gy)$, meaning $(g^{-1}g)x = (g^{-1}g)y$ by associativity. Since $g^{-1}g = e$ by definition of $g^{-1}$, so $x = y$. Applying $g$ on the right on both sides yields $xg = yg$.

        \item $\boxed{(3) \implies (1)}$ Given $xg = yg$ we may apply $g^{-1}$ on the right on both sides to obtain $(xg)g^{-1} = (yg)g^{-1}$. By associativity we have $x(gg^{-1}) = y(gg^{-1})$. Now $gg^{-1} = e$ by definition of $g^{-1}$, so $x = y$.
    \end{itemize}
    This completes the proof.
\end{proof}

To end this section, we introduce notation for the repeated application of $\ast$ on a single element $x$ in the group $G$.
\[
    x^n =
    \begin{cases}
        \underbrace{x\ast x\ast \cdots \ast x}_{n \text{ copies of } x} & \text{ if } n > 0\\
        e & \text{ if } n=0 \\
        \underbrace{x^{-1}\ast x^{-1}\ast \cdots \ast x^{-1}}_{|n| \text{ copies of } x^{-1}} & \text{ if } n<0
    \end{cases}
\]
In the case of an additive group operation, $x^n$ is written as $nx$, where $n$ is an integer.

Note that some laws of exponents apply to the above operation.

\begin{proposition}\label{prop-group-laws-of-exponents}
    Let $G$ be a group, $x \in G$, and $m$ and $n$ be non-negative integers. Then
    \begin{enumerate}
        \item $x^m \ast x^n = x^{m+n}$;
        \item $\left(x^m\right)^n = x^{mn}$; and
        \item $\left(x^{-1}\right)^n = \left(x^n\right)^{-1}$.
    \end{enumerate}
\end{proposition}

\begin{proof}
    We prove each statement individually.
    \begin{enumerate}
        \item We consider strong induction on $n$.

        When $n = 0$, one sees
        \begin{align*}
            x^m \ast x^0 &= x^m \ast e & (\text{definition of }x^0)\\
            &= x^m & (\text{definition of }e)\\
            &= x^{m+0}
        \end{align*}
        so the case where $n=0$ is true.

        Now assume that for some non-negative integer $k$, the statement holds for all integers $0 \leq i \leq k$. We are to show that the statement holds for $k+1$ as well.

        Note
        \begin{align*}
            x^m\ast x^{k+1} &= x^m\ast (x^k\ast x) & (\text{base case})\\
            &= (x^m \ast x^k) \ast x & (\textbf{Associativity})\\
            &= x^{m+k}\ast x & (\text{by the }k^{\text{th}}\text{ case})\\
            &= x^{(m+k)+1} & (\text{base case})\\
            &= x^{m+(k+1)}
        \end{align*}
        so the statement also holds for $k+1$.

        Hence, by mathematical induction, we have $x^m \ast x^n = x^{m+n}$ for all non-negative integers $m$ and $n$.

        \item We again use a proof by induction by inducting on $n$.

        When $n = 0$, one sees
        \begin{align*}
            \left(x^m\right)^0 &= e & (\text{definition of }g^0 \text{ for any }g\in G)\\
            &= x^0 & (\text{definition of }x^0)\\
            &= x^{m \times 0}
        \end{align*}
        so the case when $n = 0$ is true.

        Now assume that the statement holds for some non-negative integer $k$, i.e. $\left(x^m\right)^k = x^{mk}$ for all non-negative integers $m$. We are to show that the statement holds for $k+1$, i.e. $\left(x^m\right)^{k+1} = x^{m(k+1)}$.

        We see that
        \begin{align*}
            \left(x^m\right)^{k+1} &= \left(x^m\right)^k\ast x^m & (\text{by statement 1})\\
            &= x^{mk} \ast x^m & (\text{by induction hypothesis})\\
            &= x^{mk+k} & (\text{by statement 1})\\
            &= x^{m(k+1)}
        \end{align*}
        so the statement holds for $k+1$.

        Hence, by mathematical induction, we have $\left(x^m\right)^n = x^{mn}$ for all non-negative integers $m$ and $n$.

        \item Left as \myref{exercise-swap-inverse-with-power} (later).
    \end{enumerate}
    This proves the proposition.
\end{proof}
\begin{exercise}\label{exercise-swap-inverse-with-power}
    Prove that $(x^{-1})^n = (x^n)^{-1}$ for all non-negative integers $n$.
\end{exercise}

\section{Orders}
\begin{definition}
    Let $G$ be a group. The \term{order} of a group\index{order!group}, denoted by $|G|$, is the cardinality of the set $G$.
\end{definition}

If $|G| = n$ where $n$ is finite, we say that $G$ is a \term{finite group}. On the other hand, if $|G| = \infty$, then we say that $G$ is an \term{infinite group}.

\begin{example}
    The group $(\Z_4, \oplus_4)$ has order 4 since it has four elements, namely 0, 1, 2, and 3.
\end{example}

\begin{example}
    The group $(\R, +)$ is an infinite group since $\R$ has an infinite number of elements.
\end{example}

\begin{definition}
    Let $g$ be an element of the group $G$. Then the order of $g$\index{order!element}, denoted by $|g|$, is the least positive integer $n$ such that $g^n = e$.
\end{definition}
Note that if $n$ is infinite, we say that the order of $g$ is infinite (or that $g$ has \term{infinite order}\index{order!infinite}).

\begin{example}
    Consider the group $G = (\Z_4, \oplus_4)$ and its 4 elements 0, 1, 2, and 3.
    \begin{itemize}
        \item The element $0$ has order 1 since $0^1 = 0$, which is the identity. Thus $|0| = 1$ in $G$.
        \item The element $1$ has order 4 since $1 \oplus_4 1 \oplus_4 1 \oplus_4 1 = 0$ and no smaller $n$ than 4 exists. Thus $|1| = 4$ in $G$.
        \item The element $2$ has order 2 since $2 \oplus_4 2 = 0$ and no smaller $n$ than 2 exists. Thus $|2| = 2$ in $G$.
        \item The element $3$ has order 4 since $3 \oplus_4 3 \oplus_4 3 \oplus_4 3 = 0$ and no smaller $n$ than 4 exists. Thus $|3| = 4$ in $G$.
    \end{itemize}
\end{example}

We note a few things about the order of elements in a group.
\begin{itemize}
    \item The identity element has order 1.
    \item A group where every element has finite order is a \term{periodic group}\index{group!periodic}.
    \item A finite group is always periodic since all elements in it has finite order.
    \item The order of any element in a group divides the order of the group.
\end{itemize}
The last point is actually a consequence of Lagrange's theorem (\myref{thrm-lagrange}). We will look into its proof in the next chapter.

\begin{exercise}
    Let $i$ be a number such that $i^2 = -1$. Let $\mathcal{S} = \{1, -1, i, -i\}$.
    \begin{partquestions}{\roman*}
        \item Find the identity of the group $(\mathcal{S}, \times)$ where $\times$ denotes regular multiplication.
        \item Find the orders of the elements in $(\mathcal{S}, \times)$.
    \end{partquestions}
\end{exercise}

\begin{lemma}\label{lemma-order-of-an-element-that-is-equivalent-to-identity}
    Let $G$ be a group with identity $e$ and let $x \in G$ have finite order $n$. Then a positive integer $m$ is a multiple of $n$ if and only if $x^m = e$.
\end{lemma}
\begin{proof}
    See \myref{problem-element-to-power-of-multiple-of-order-is-identity} (later).
\end{proof}

\begin{theorem}\label{thrm-order-of-power-of-element}
    Let $G$ be a finite group and let $x \in G$ have order $n$. Then for any positive integer $k$ we have
    \[
        \left|x^k\right| = \frac{n}{\gcd(n,k)}.
    \]
\end{theorem}
\begin{proof}
    Let $d = \gcd(n,k)$ and $m = |x^k|$. Note that $d$ divides both $n$ and $k$, so $\frac nd$ and $\frac kd$ are integers. It is trivial to see then that $(x^k)^{\frac nd} = \left(x^n\right)^{\frac kd} = e$, which means that $m$ divides $\frac nd$ by \myref{lemma-order-of-an-element-that-is-equivalent-to-identity}.

    Also, one sees that $x^{km} = (x^k)^m = e$ so $n$ divides $km$. Hence we must have $\frac nd$ divides $\frac {km}{d} = m\frac kd$. But one sees that $\frac nd$ and $\frac kd$ are coprime, so by \myref{theorem-n-divides-ab-and-n-coprime-with-a-implies-n-divides-b} we see that $\frac nd$ divides $m$.

    As $m$ divides $\frac nd$ and $\frac nd$ divides $m$ thus $m = \frac nd$ as required.
\end{proof}

\section{Cyclic Groups}\index{cyclic group}
Now that we have gotten some basic terminology and some properties of groups out of the way, let's introduce a very simple type of group: the cyclic groups.

\begin{definition}
    Let $G$ be a group and $g \in G$. If
    \[
        G = \{g, g^2, g^3, \dots, g^n\}
    \]
    for some positive integer $n$, then $G$ is a \term{cyclic group of order $n$}\index{cyclic group!of order $n$} and has a \term{generator $g$}\index{cyclic group!generator}, and is written as $G = \langle g \rangle$.
\end{definition}

\begin{example}
    Let $i$ be the imaginary unit, i.e. $i^2 = -1$. Let the set $\mathcal{S} = \{1, -1, i, -i\}$. Notice the group $(\mathcal{S}, \times)$ is completely generated by the element $i$ since
    \[
    i^1 = i,\; i^2 = -1,\; i^3 = -i, \text{ and } i^4 = 1.
    \]
    Thus, $\mathcal{S} = \{i, i^2, i^3, i^4\} = \langle i \rangle$.
\end{example}

\begin{exercise}
    Find the other generator of the group $(\{1, -1, i, -i\}, \times)$, where $i^2 = -1$.
\end{exercise}

From the above exercise we see that not every element in a cyclic group is a generator, and that a cyclic group may have more than 1 generator.

Cyclic groups may also be of infinite order. Such cyclic groups are called \term{cyclic groups of infinite order} or \term{infinite cyclic groups}\index{cyclic group!infinite}.
\begin{example}
    The group $(\Z, +)$ is an infinite cyclic group with generators 1 and -1.
\end{example}

We now look at two results involving cyclic groups.
\begin{proposition}\label{prop-cyclic-group-is-abelian}
    Every cyclic group is abelian.
\end{proposition}
\begin{proof}
    Let $G$ be a cyclic group with generator $g$. Suppose $x, y \in G$. Then $x = g^m$ and $y = g^n$ for some positive integers $m$ and $n$. Thus,
    \begin{align*}
        xy &= (g^m)(g^n)\\
        &= g^mg^n\\
        &= g^{m+n}\\
        &= g^{n+m} & (\text{+ is commutative, } \myref{axiom-addition-is-commutative})\\
        &= g^ng^m\\
        &= (g^n)(g^m)\\
        &= yx
    \end{align*}
    so $xy = yx$. Therefore $G$ is abelian.
\end{proof}

\begin{theorem}\label{thrm-cyclic-group-has-element-with-same-order}
    A finite group $G$ is cyclic if and only if there exists an element $g \in G$ with the same order as the group.
\end{theorem}
\begin{proof}
    We first prove the forward direction. Suppose $G$ is cyclic and $|G| = n$. Then, by definition, there exists an element $g \in G$ such that
    \[
        G = \langle g \rangle = \{g^k \vert 1 \leq k \leq n, k \in \Z\},
    \]
    i.e. $g$ is a generator of $G$. We just need to show $|g| = n$. Suppose on the contrary there exists an integer $1 \leq m < n$ where $g^m = e$. Then $\langle g \rangle = \{g, g^2, \dots, g^m\}$. Thus $|\langle g \rangle| = m < n = |G|$. But by the hypothesis of the forward direction, $G = \langle g \rangle$ so $n = |G| = |\langle g \rangle| = m$. This is a contradiction, i.e. there does \textbf{not} exist an integer $1 \leq m < n$ where $g^m = e$. Therefore $g^n = e$, i.e. $|g| = n$.

    We now prove the reverse direction: suppose there is an element $g \in G$ with order $n$. We claim that $g, g^2, \dots, g^n$ are all distinct. Suppose on the contrary that there exist integers $i$ and $j$ where $1 \leq i < j \leq n$ such that $g^i = g^j$. Then $g^i = g^ig^{j-i}$, which means $g^{j-i} = e$ by the cancellation law. Note that $1 \leq j - i < n$. Thus, since $g^{j-i} = e$, therefore $|g| = j - i < n$ which contradicts $|g| = n$. Hence, $g, g^2, \dots, g^n$ are all distinct. Therefore, $\langle g \rangle = \{g, g^2, \dots, g^n\}$ contain distinct elements of $G$. But there are only $n$ elements in $G$ and $\langle g \rangle$ contains $n$ distinct elements. Therefore, $G = \langle g \rangle$ which means that $G$ is cyclic with generator $g$.
\end{proof}

\section{Dihedral Groups}
We motivate the definition of the dihedral groups by discussing the symmetries of an equilateral triangle.

\begin{figure}[h]
    \centering
    \pdfteximgframed[12pt]{0.3\textwidth}{part1/images/basics-of-groups/symmetries-of-triangle.pdf_tex}
    \caption{Symmetries of an Equilateral Triangle}
\end{figure}

What actions could we perform in order to maintain symmetry on an equilateral triangle? Well, we could rotate the triangle in $120^\circ$ anti-clockwise increments about the center of the triangle. We denote this action by the symbol $r$. Another thing we could do is reflect the triangle about the line going through one of the vertices and the center, like we discussed in an earlier chapter. This action is denoted by $s$.

Now, suppose we define $r$ to be the $120^\circ$ anti-clockwise rotation about the center and $s$ be the reflection of the triangle about the line going through vertex 1 and the center, like shown in the diagram. How do we obtain a $240^\circ$ anti-clockwise rotation? Well, we apply two $120^\circ$ anticlockwise rotations one after another. In other words, if $\ast$ means ``action composition'', then a $240^\circ$ rotation would be represented by $r^2$. Note that $r^3$, which represents a $360^\circ$ anti-clockwise rotation, is the same as doing nothing. So $r^3 = e$. Similarly, applying the reflection $s$ twice in a row (i.e., $s^2$) is the same as doing nothing, so $s^2 = e$. Thus, we have
\[
    r^3 = s^2 = e
\]
for the case of an equilateral triangle.

There's another relationship governing $r$ and $s$. Consider this: how do we obtain a reflection about the line through vertex 3 and the center? Well, we apply $r$ first, followed by $s$. This means that a reflection about the line through vertex 3 and the center is given by $rs$. Notice that this is the same thing as reflecting first and then applying $r$ twice, i.e. $sr^2$. Thus, we have the second relationship:
\[
    rs = sr^2
\]
for the case of an equilateral triangle.

The group of symmetries of an equilateral triangle is called the \term{dihedral group of order 6} (or the \term{dihedral group of degree 3}) and is denoted by $D_3$. In general, the \term{dihedral group of order $2n$}\index{dihedral group!of order $2n$} (or the \term{dihedral group of degree $n$}\index{dihedral group!of degree $n$}) is denoted by $D_n$ and can be thought of as the symmetries of a regular polygon of $n$ sides (a regular $n$-gon).

\newpage

\begin{example}
    The symmetries of the square is given by the group $D_4$.
\end{example}
\begin{figure}[h]
    \centering
    \pdfteximgframed[12pt]{0.3\textwidth}{part1/images/basics-of-groups/symmetries-of-square.pdf_tex}
    \caption{Symmetries of a Square}
\end{figure}

Thus, in general, the set $D_n$ consists of the following elements.
\[
    D_n = \{e, r, r^2, \dots, r^{n-1}, s, rs, r^2s, \dots, r^{n-1}s\}
\]
with the relationship between $r$ and $s$ given by $r^n = s^2 = e$ and $rs = sr^{n-1}$.

\begin{remark}
    Some authors (e.g. {\cite[p.~25]{dummit_foote_2004}}) write the reflections of $D_n$ with $s$ leading $r$, i.e. $s, sr, sr^2, sr^3, \dots, sr^{n-1}$. The underlying definition, however, remains the same in either case. This fact will become evident with the proof of \myref{prop-Dn-cannonical-form} (later).
\end{remark}

These relationships are succinctly given by the following \term{presentation}\index{presentation}, and is the actual definition of the dihedral group.
\begin{definition}
    The \term{dihedral group of degree $n$}\index{dihedral group!of degree $n$} (or the \term{dihedral group of order $2n$}\index{dihedral group!of order $2n$}) is
    \[
        D_n = \langle r, s \vert r^n = s^2 = e,\;rs = sr^{n-1} \rangle.
    \]
\end{definition}
\begin{remark}
    In the above definition, $r$ and $s$ can be thought of as `generators'\index{presentation!generators} and the conditions\index{presentation!conditions} are given on the right side of the pipe ($|$).
\end{remark}

\begin{example}\label{example-presentation-of-D3}
    The group $D_3$ has presentation
    \[
        D_3 = \langle r, s \vert r^3 = s^2 = e,\;rs = sr^2 \rangle.
    \]
    The Cayley table of $D_3$ is as follows.

    \begin{table}[H]
        \centering
        \begin{tabular}{|l|l|l|l|l|l|l|}
        \hline
        $\boldsymbol{\ast}$ & $\boldsymbol{e}$ & $\boldsymbol{r}$ & $\boldsymbol{r^2}$ & $\boldsymbol{s}$ & $\boldsymbol{rs}$ & $\boldsymbol{r^2s}$ \\ \hline
        $\boldsymbol{e}$ & $e$ & $r$ & $r^2$ & $s$ & $rs$ & $r^2s$ \\ \hline
        $\boldsymbol{r}$ & $r$ & $r^2$ & $e$ & $rs$ & $r^2s$ & $s$ \\ \hline
        $\boldsymbol{r^2}$ & $r^2$ & $e$ & $r$ & $r^2s$ & $s$ & $rs$ \\ \hline
        $\boldsymbol{s}$ & $s$ & $r^2s$ & $rs$ & $e$ & $r^2$ & $r$ \\ \hline
        $\boldsymbol{rs}$ & $rs$ & $s$ & $r^2s$ & $r$ & $e$ & $r^2$ \\ \hline
        $\boldsymbol{r^2s}$ & $r^2s$ & $rs$ & $s$ & $r^2$ & $r$ & $e$ \\ \hline
        \end{tabular}
    \end{table}

    We use the convention of reading the row before the column, so the action $rs \ast r^2$ (usually written as $rsr^2$) is given by the row of $rs$ and the column of $r^2$, which is $r^2s$.
\end{example}

Such a definition for the dihedral groups removes the need to associate the group $D_n$ with any idea of a regular $n$-gon, and focuses solely on the relationship between $r$ and $s$. In fact, the removal of reliance on regular $n$-gons help define the non-intuitive groups $D_1$ and $D_2$.
\begin{example}
    The group $D_1$ has presentation
    \[
        D_1 = \langle r, s \vert r = s^2 = e, rs = s \rangle,
    \]
    which quickly means that $D_1$ has only two elements, $e$ (which is $r$) and $s$. Thus $D_1$ is, in fact, just the cyclic group of order 2. One can verify the fact by drawing the Cayley table of $D_1$.

    \begin{table}[H]
        \centering
        \begin{tabular}{|l|l|l|}
            \hline
            $\boldsymbol{*}$ & $\boldsymbol{e}$ & $\boldsymbol{s}$ \\ \hline
            $\boldsymbol{e}$ & $e$ & $s$ \\ \hline
            $\boldsymbol{s}$ & $s$ & $e$ \\ \hline
        \end{tabular}
    \end{table}
\end{example}

\begin{example}
    The group $D_2$ has presentation
    \[
        D_2 = \langle r, s \vert r^2 = s^2 = e, rs = sr \rangle.
    \]
    So we see that $D_2 = \{e, r, s, rs\}$. Let us look at the Cayley table of $D_2$.

    \begin{table}[H]
        \centering
        \begin{tabular}{|l|l|l|l|l|}
            \hline
            $\boldsymbol{\ast}$ & $\boldsymbol{e}$ & $\boldsymbol{r}$ & $\boldsymbol{s}$ & $\boldsymbol{rs}$ \\ \hline
            $\boldsymbol{e}$ & $e$ & $r$ & $s$ & $rs$ \\ \hline
            $\boldsymbol{r}$ & $r$ & $e$ & $rs$ & $s$ \\ \hline
            $\boldsymbol{s}$ & $s$ & $rs$ & $e$ & $r$ \\ \hline
            $\boldsymbol{rs}$ & $rs$ & $s$ & $r$ & $e$ \\ \hline
        \end{tabular}
    \end{table}

    We can see from the above Cayley table that $D_2$ is actually abelian. This group has some interesting properties, which will be explored in a future chapter.
\end{example}

The \term{canonical form}\index{dihedral group!canonical form} of an element in a dihedral group is $r^ms^n$, where $m$ and $n$ are non-negative integers. So how do we find the canonical form of elements like $sr$, $srs$, or $rsr^3$? We have this useful proposition to help.
\begin{proposition}\label{prop-Dn-cannonical-form}
    In the group $D_n$, we have $r^ms = sr^{n-m}$ for all integers $1 \leq m < n$.
\end{proposition}
\begin{proof}
    We induct on $m$. When $m = 1$, $rs = sr^{n-1}$ by the definition of $D_n$. Assume now that for some integer $1 \leq k < n$, we have $r^ks = sr^{n-k}$. We consider two cases.
    \begin{itemize}
        \item If $k = n - 1$, then $k + 1 = n$. Thus, $r^{k+1}s = r^ns = s$ since $r^n = e$. Note that $sr^{(k+1)-n} = sr^{n-n} = sr^0 = s$. Therefore $r^{k+1}s = sr^{n-(k+1)}$.
        \item The other case is if $1 \leq k \leq n - 2$. Then we have
        \begin{align*}
            r^{k+1}s &= r^k(rs)\\
            &= r^k(sr^{n-1}) & (\text{base case})\\
            &= (r^ks)r^{n-1} & (\textbf{Associativity})\\
            &= (sr^{n-k})r^{n-1} & (\text{by induction hypothesis})\\
            &= sr^{2n - k - 1}\\
            &= sr^nr^{n-k-1}\\
            &= sr^{n-(k+1)} & (\text{since } r^n = e)
        \end{align*}
        which means $r^{k+1}s = sr^{n-(k+1)}$.
    \end{itemize}
    In either case, the statement is true for $k+1$.

    Therefore $r^ms = sr^{n-m}$ for all integers $1 \leq m < n$.
\end{proof}

\begin{exercise}
    Simplify $rsr^4sr^3$ in the group $D_6$.
\end{exercise}

\newpage

\section{Problems}
\begin{problem}
    Draw the Cayley table for $D_4$, the dihedral group of order 8, representing the symmetries of a square.\newline
    By referring to the Cayley table,
    \begin{partquestions}{\alph*}
        \item explain why $D_4$ is not abelian;
        \item simplify $r^3srsr^3sr^3sr^2$.
    \end{partquestions}
\end{problem}

\begin{problem}\label{problem-Q-is-abelian-group-under-addition}
    Prove that $\Q$ under addition forms an abelian group.\newline
    (\textit{Note: addition is assumed to be associative and commutative by \myref{axiom-addition-is-associative} and \myref{axiom-addition-is-commutative} respectively, so you do not need to prove them.})
\end{problem}

\begin{problem}
    Let $G$ be a group where every element in $G$ is its own inverse. Prove that $G$ is abelian.
\end{problem}

\begin{problem}\label{problem-element-to-power-of-multiple-of-order-is-identity}
    Let $G$ be a group with identity $e$. Suppose an element $x \in G$ has finite order $n$. Prove that a positive integer $m$ is a multiple of $n$ if and only if $x^m = e$.\newline
    (\textit{Hint: consider Euclid's division lemma (\myref{lemma-euclid-division}) to prove one direction of the claim.})
\end{problem}

\begin{problem}
    Let $G$ be a group.
    \begin{partquestions}{\alph*}
        \item Suppose $(gh)^2 = g^2h^2$ for all elements $g, h \in G$. Prove that $G$ is abelian.
        \item Suppose $G$ is abelian. Prove that $(gh)^n = g^nh^n$ for all elements $g$ and $h$ in $G$ and for all positive integers $n$.
    \end{partquestions}
\end{problem}

\begin{problem}
    Let the group $G = (\Z_n, \oplus_n)$. Show that $G$ is cyclic with order $n$.
\end{problem}

\begin{problem}
    Let the function $T: \R^2 \to \R^2$ be defined by
    \[
        T(x, y) = (-y, x+y).
    \]
    Define the set $A = \{T^r \vert r \in \Z \text{ and } r \geq 1\}$. Show that $A$ is a group under function composition ($\circ$) and state the order of this group.
\end{problem}
