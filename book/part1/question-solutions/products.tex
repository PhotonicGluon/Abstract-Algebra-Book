\section{Direct Products of Groups}
\subsection*{Exercises}
\begin{questions}
    \item We prove the group axioms.
    \begin{itemize}
        \item \textbf{Closure}: Let $(g_1, h_1), (g_2, h_2) \in G \times H$. So $g_1, g_2 \in G$ and $h_1, h_2 \in H$. Thus $g_1g_2 \in G$ and $h_1h_2 \in H$ by closure of groups, and therefore $(g_1g_2, h_1h_2) \in G\times H$. Note that $(g_1, h_1)(g_2, h_2) = (g_1g_2, h_1h_2)$ by definition of the group operation in the external direct product, so $G \times H$ is closed under component-wise application of the group operations.

        \item \textbf{Associativity}: Let $(g_1, h_1), (g_2, h_2), (g_3, h_3) \in G \times H$. Then we see
        \begin{align*}
            (g_1, h_1)\left((g_2, h_2)(g_3, h_3)\right) &= (g_1, h_1)(g_2g_3, h_2h_3)\\
            &= (g_1(g_2g_3), h_1(h_2h_3))\\
            &= ((g_1g_2)g_3, (h_1h_2)h_3)\\
            &= (g_1g_2, h_1h_2)(g_3,h_3)\\
            &= \left((g_1,h_1)(g_2,h_2)\right)(g_3,h_3)
        \end{align*}
        which proves the associativity of the group operation.

        \item \textbf{Identity}: Let $e_G \in G$ and $e_H \in H$ be the identities of $G$ and $H$ respectively. Then we see for any $(g, h) \in G \times H$ that
        \[
            (e_G, e_H)(g, h) = (e_Gg, e_Hh) = (g, h)
        \]
        and
        \[
            (g, h)(e_G, e_H) = (ge_G, he_H) = (g, h),
        \]
        so $(e_G, e_H)$ is indeed the identity of $G \times H$.

        \item \textbf{Inverse}: Let $(g, h) \in G \times H$. Note that
        \[
            (g, h)(g^{-1}, h^{-1}) = (gg^{-1}, hh^{-1}) = (e_G, e_H)
        \]
        and
        \[
            (g^{-1}, h^{-1})(g, h) = (g^{-1}g, h^{-1}h) = (e_G, e_H)
        \]
        so $\left((g, h)\right)^{-1} = (g^{-1}, h^{-1})$.
    \end{itemize}
    Therefore $G \times H$ is a group.

    \item We work component-wise:
    \begin{align*}
        (s, rs)(r^2s, r^3) &= (sr^2s, rsr^3)\\
        &= (s(r^2s), r(sr^3))\\
        &= (s(sr), r(rs))\\
        &= ((ss)r, (rr)s)\\
        &= (r, r^2s)
    \end{align*}

    \item Note that $180 = 2^2 \times 3^2 \times 5$. By \myref{thrm-Zm-cross-Zn-isomorphic-to-Zmn-condition}, we must have $mn = 180$ and $\gcd(m, n) = 1$. Thus, the possible values for $m$ and $n$ are
    \begin{itemize}
        \item $m = 4$ and $n = 45$;
        \item $m = 5$ and $n = 36$; and
        \item $m = 9$ and $n = 20$.
    \end{itemize}

    \item Note that $5 \otimes_{12} 7 = 11$. Hence $GH = \{1, 5, 7, 11\}$.

    \item From the previous exercise we know that $GH = \mathcal{S}$. Now $G = \langle 5 \rangle \cong \Z_2$ and $H = \langle 7 \rangle \cong \Z_2$. Thus, $\mathcal{S} = GH \cong G \times H \cong \Z_2 \times \Z_2 = (\Z_2)^2$, meaning $n = 2$.
\end{questions}

\subsection*{Problems}
\begin{questions}
    \item Let $g_1, g_2 \in G$ and $h_1, h_2 \in H$. Then for $(g_1, h_1), (g_2, h_2) \in G\times H$ we see that
    \begin{align*}
        (g_1, h_1)(g_2, h_2) &= (g_1g_2, h_1h_2)\\
        &= (g_2g_1, h_2h_1)\\
        &= (g_2,h_2)(g_1,h_1)
    \end{align*}
    which means that $G \times H$ is abelian.

    \item Let the map $\phi: G\times H \to H \times G, (g, h) \mapsto (h, g)$. We prove that $\phi$ is an isomorphism:
    \begin{itemize}
        \item \textbf{Homomorphism}: Let $(g_1, h_1), (g_2, h_2) \in G \times H$. We note that
        \begin{align*}
            \phi((g_1, h_1)(g_2, h_2)) &= \phi((g_1g_2, h_1h_2))\\
            &= (h_1h_2, g_1g_2)\\
            &= (h_1, g_1)(h_2, g_2)\\
            &= \phi((g_1, h_1))\phi((g_2, h_2))
        \end{align*}
        which proves that $\phi$ is a homomorphism.
        \item \textbf{Injective}: Let $(g_1, h_1), (g_2, h_2) \in G \times H$ be such that $\phi((g_1, h_1)) = \phi((g_2, h_2))$. Then by definition of $\phi$ we have $(h_1, g_1) = (h_2, g_2)$. Clearly by comparing component parts of each ordered pair, we have $g_1 = g_2$ and $h_1 = h_2$, meaning $(g_1, h_1) = (g_2, h_2)$. Hence $\phi$ is injective.
        \item \textbf{Surjective}: Let $(h, g) \in H \times G$. Clearly $(g, h) \in G \times H$ and $\phi((g, h)) = (h, g)$, meaning that $(h, g)$ has a pre-image of $(g, h)$. Therefore $\phi$ is surjective.
    \end{itemize}
    Therefore $\phi$ is an isomorphism, meaning $G \times H \cong H \times G$.

    \item We claim that $G$ is the internal direct product of $H$ and $K$. We need to check 3 things.
    \begin{itemize}
        \item $\boxed{G = HK}$ We note that
        \begin{align*}
            HK &= \{h \oplus_6 k \vert h \in H, k \in K\}\\
            &= \{0 \oplus_6 0, 0 \oplus_6 3, 2 \oplus_6 0, 2 \oplus_6 3, 4 \oplus_3 0, 4 \oplus_3 3\}\\
            &= \{0, 3, 2, 5, 4, 1\}\\
            &= \Z_6\\
            &= G
        \end{align*}
        so in fact $G = HK$.

        \item $\boxed{H \cap K = \{e\}}$ Clearly $H \cap K = \{0\}$.

        \item $\boxed{hk = kh}$ Since $\oplus_6$ is commutative, thus $h \oplus_6 k = k \oplus_6$.
    \end{itemize}
    Thus $G$ is the internal direct product of $H$ and $K$.

    \item Define the subgroups $H = \{e, a\}$ and $K = \{e, b\}$. We show the $\mathrm{V}$ is the internal direct product of $H$ and $K$.
    \begin{itemize}
        \item $\boxed{\mathrm{V} = HK}$ Observe that
        \begin{align*}
            HK &= \{hk \vert h \in H, k \in K\}\\
            &= \{ee, eb, ae, ab\}\\
            &= \{e, b, a, ab\}\\
            &= \mathrm{V}
        \end{align*}
        so in fact $\mathrm{V} = HK$.

        \item $\boxed{H \cap K = \{e\}}$ Clearly $H \cap K = \{e\}$.

        \item $\boxed{hk = kh}$ Clearly if one of the elements is the identity then result follows. So assume that $h$ and $k$ are both non-identity elements, so $h = a$ and $k = b$. Note
        \begin{align*}
            kh &= ba\\
            &= (ba)\left((ab)(ab)\right) & (\text{since }(ab)^2 = e)\\
            &= (ba ab)(ab)\\
            &= (bb)(ab) & (\text{since }a^2 = e)\\
            &= ab & (\text{since }b^2 = e)\\
            &= hk
        \end{align*}
        so in fact $hk = kh$ for all $h \in H$, $k \in K$.
    \end{itemize}
    Therefore $\mathrm{V}$ is the internal direct product of $H$ and $K$.

    We note $H = \langle a\rangle \cong \Z_2$ and $K = \langle b \rangle \cong \Z_2$. By direct product equivalence (\myref{thrm-direct-product-equivalence}) we know $\mathrm{V} \cong H \times K \cong \Z_2 \times \Z_2 = (\Z_2)^2$.
\end{questions}
