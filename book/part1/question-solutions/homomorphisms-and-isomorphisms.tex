\section{Homomorphisms and Isomorphisms}
\subsection*{Exercises}
\begin{questions}
    \item \begin{partquestions}{\alph*}
        \item No, since $\phi(x + y) = x + y$ while $\phi(x)\phi(y) = xy \neq x+y$ for $x, y \in G$.
        \item Yes, since $\phi(x + y) = 2^{x + y} = 2^x2^y = \phi(x)\phi(y)$ for all $x, y \in G$.
    \end{partquestions}

    \item One sees that
    \[
        \phi(xy) = e_H = e_He_H = \phi(x)\phi(y)
    \]
    so $\phi$ is indeed a homomorphism.

    \item The codomain of $\phi$ is $G_2$, so $\phi(H_1) \subseteq G_2$. Clearly $e_2 \in \phi(H_1)$ since $e_2 = \phi(e_1)$ and $e_1 \in H_1$. Now suppose $x, y \in \phi(H_1)$, meaning that $\phi(h_x) = x$ and $\phi(h_y) = y$ for some $h_x, h_y \in H$. So $h_xh_y^{-1} \in H$. Furthermore,
    \begin{align*}
        \phi(h_xh_y^{-1}) &= \phi(h_x)\phi(h_y^{-1})\\
        &= \phi(h_x)\left(\phi(h_y)\right)^{-1}\\
        &= xy^{-1},
    \end{align*}
    meaning that $xy^{-1} \in \phi(H_1)$. Therefore $\phi(H_1) \leq G_2$ by the subgroup test (\myref{thrm-subgroup-test}).

    \item Disprove. Let $G_1 = H_1 = \Z$ be the additive group of integers and let $G_2 = D_n$, the dihedral group of order $2n$. Consider the map $\phi: G_1 \to G_2$ where $\phi(m) = s^m$. Clearly, $H_1 \unlhd G_1$. Note that $\phi(H_1) = \{e, s\} = \langle s \rangle$. From \myref{example-normal-subgroups-of-d3}, we know that $\langle s \rangle$ is not a normal subgroup of $D_3 = G_2$, so $\phi(H_1)$ is not a normal subgroup of $G_2$.

    \item Suppose $|a| = n$. Note that
    \[
        \left(\phi(a)\right)^n = \phi\left(a^n\right) = \phi(e_G) = e_H
    \]
    so $|\phi(a)|$ divides $n = |a|$ by \myref{lemma-order-of-an-element-that-is-equivalent-to-identity}.

    \item \begin{partquestions}{\roman*}
        \item We show that $\id$ is both injective and surjective.
        \begin{itemize}
            \item \textbf{Injective}: Suppose $x, y \in S$ such that $\id(x) = \id(y)$. Then clearly $x = y$ by definition of the identity map.
            \item \textbf{Surjective}: Suppose $y \in S$. Clearly $\id(y) = y$, so $y$ is its own pre-image.
        \end{itemize}
        Thus $\id$ is a bijection.

        \item Clearly
        \[
            \id(xy) = xy = \id(x)\id(y)
        \]
        so $\id$ is a homomorphism. Coupled with \textbf{(i)}, this means that $\id$ is an isomorphism.
    \end{partquestions}

    \item \begin{partquestions}{\roman*}
        \item Since $3^0 = 1$, $3^1 = 3$, $3^2 = 9 \equiv 4 \pmod{5}$, and $3^3 = 27 \equiv 2 \pmod{5}$, thus $G = \langle 3 \rangle$. Also since $7^0 = 1$, $7^1 = 7$, $7^2 = 49 \equiv 9 \pmod{10}$, and $7^3 = 343 \equiv 3 \pmod{10}$, thus $H = \langle 7 \rangle$.
        \item We need to prove that it is a homomorphic bijection.
        \begin{itemize}
            \item \textbf{Homomorphism}:
            \begin{align*}
                \phi(3^m3^n) &= \phi(3^{m+n})\\
                &= 7^{m+n}\\
                &= 7^m7^n\\
                &= \phi(3^m)\phi(3^n)
            \end{align*}

            \item \textbf{Bijection}: Note that $1 \mapsto 1$, $3 \mapsto 7$, $4 \mapsto 9$, $2 \mapsto 3$ which clearly shows that $\phi$ is bijective.
        \end{itemize}
        Therefore $\phi$ is an isomorphism, meaning $G \cong H$.
    \end{partquestions}

    \item \begin{partquestions}{\roman*}
        \item We show that $h = g\circ f$ is both injective and surjective.
        \begin{itemize}
            \item \textbf{Injective}: Let $x, y \in A$. Then
            \begin{align*}
                &h(x) = h(y)\\
                \iff&g(f(x)) = g(f(y))\\
                \iff&f(x) = f(y) & (g \text{ is a bijection})\\
                \iff&x = y & (f \text{ is a bijection})
            \end{align*}
            so $h$ is injective.
            \item \textbf{Surjective}: Suppose $c \in C$. Note that the functions $f^{-1}: B \to A$ and $g^{-1}: C \to B$ both exist since both the functions $f$ and $g$ are injective. Let $a = f^{-1}(g^{-1}(c)) \in A$. Then note
            \[
                h(a) = g(f(f^{-1}(g^{-1}(c)))) = c
            \]
            so $c$ has a pre-image of $a \in A$, meaning $h$ is surjective.
        \end{itemize}
        Therefore $h$ is a bijection.

        \item Recall that an isomorphism is also a homomorphism.  Note that
        \begin{align*}
            h(xy) &= g(f(xy))\\
            &= g(f(x)f(y)) & (f \text{ is a homomorphism})\\
            &= g(f(x))g(f(y)) & (g \text{ is a homomorphism})\\
            &= h(x)h(y)
        \end{align*}
        so $h$ is a homomorphism. As $h$ is also a bijection (from \textbf{(i)}), thus $h$ is an isomorphism.
    \end{partquestions}

    \item Suppose $N \unlhd G$ such that $|N| = k$. Then $\phi(N)$ is a subgroup of $H$ with order $k$ by \myref{thrm-isomorphism-consequences}, statement 5. All that remains to prove is that $\phi(N)$ is normal.

    Let $n \in N$ and $\hat{n} \in \phi(N)$ such that $\hat{n} = \phi(n)$. Let $h \in H$ be an arbitrary element. We show that $h\hat{n}h^{-1}$ is in $\phi(N)$.

    Let $g \in G$ such that $\phi(g) = h$. Then
    \begin{align*}
        h\hat{n}h^{-1} &= \phi(g)\phi(n)\phi(g^{-1})\\
        &= \phi(\underbrace{gng^{-1}}_{\text{In } N})\\
        &\in \phi(N)
    \end{align*}
    which proves that $\phi(N)$ is normal. Hence there exists a normal subgroup of order $k$, namely $\phi(N)$.

    \item \begin{partquestions}{\alph*}
        \item We know that $\id: G \to G, g \mapsto g$ is an isomorphism (\myref{exercise-identity-map-is-isomorphism}), so $G \cong G$.
        \item Suppose $G \cong H$. This means that there is an isomorphism $\phi: G \to H$. \myref{thrm-isomorphism-consequences}, statement 2 tells us that $\phi^{-1}: H \to G$ is also an isomorphism. Therefore $H \cong G$.
        \item Suppose $G \cong H$ and $H \cong K$. Then there exist isomorphisms $\phi: G \to H$ and $\psi: H \to K$. We know that $\sigma: G \to K, g \mapsto \psi(\phi(g))$ is an isomorphism by \myref{exercise-composition-of-isomorphisms-is-isomorphisms}, which means $G \cong K$.
    \end{partquestions}

    \item Since $7^1 = 7$, $7^2 = 49 \equiv 9 \pmod{10}$, $7^3 = 343 \equiv 3 \pmod{10}$, and $7^4 = 2401 \equiv 1 \pmod{10}$, thus $G = \langle 7 \rangle$. Note $|7| = 4$ so $G \cong \Z_4$, i.e. $n = 4$.
\end{questions}

\subsection*{Problems}
\begin{questions}
    \item We will prove that $f$ is a homomorphism, is injective, and is surjective.
    \begin{itemize}
        \item \textbf{Homomorphism}: Let $x, y \in G$. Then
        \begin{align*}
            f(xy) &= g(xy)g^{-1}\\
            &= (gxg^{-1})(gyg^{-1})\\
            &= f(x)f(y)
        \end{align*}
        which means that $f$ is a homomorphism.

        \item \textbf{Injective}: Let $x, y \in G$ be such that $f(x) = f(y)$. Then we see $gxg^{-1} = gyg^{-1}$. By cancellation law, we see $x = y$.

        \item \textbf{Surjective}: Suppose $y \in G$. Set $x = g^{-1}yg$. Since $G$ is closed, thus $x \in G$. Note $f(x) = g(g^{-1}yg)g^{-1} = y$, so $y$ has a pre-image of $x = g^{-1}yg \in G$.
    \end{itemize}
    Therefore $f$ is an isomorphism.

    \item \begin{partquestions}{\alph*}
        \item Let $m, n \in \Z$. Then
        \[
            \phi(m + n) = 2(m + n) = 2m + 2n = \phi(m) + \phi(n)
        \]
        which means $\phi$ is a homomorphism.

        \item Suppose $m, n \in G$ such that $\phi(m) = \phi(n)$. Then $2m = 2n$. Clearly this means that $m = n$. Thus $\phi$ is injective.

        \item Suppose on the contrary there exists a homomorphism $\psi: \Z \to \Z$ such that $\psi(\phi(n)) = n$. Then $\psi(2n) = n$ by definition of $\phi$. Note that
        \[
            \psi(2n) = \psi(n + n) = \psi(n) + \psi(n) = 2\psi(n)
        \]
        since $\psi$ is a homomorphism. Hence $2\psi(n) = n$ which implies that $\psi(n) = \frac n2$. But for the case of $n = 1$, $\psi(1) = \frac 12 \notin \Z$. Hence $\psi$ does not exist.
    \end{partquestions}

    \item Suppose on the contrary there exists an isomorphism $\phi: G \to H$. Since $\phi$ is an isomorphism, it is surjective. Hence, there must exists a rational number $r \in G$ such that $\phi(r) = 2$. As $r$ is rational, so is $\frac r2$.

    Now consider $\phi\left(\frac r2 + \frac r2\right)$. On one hand, $\phi\left(\frac r2 + \frac r2\right) = \phi(r) = 2$. On another hand, $\phi(\frac r2 + \frac r2) = \left(\phi\left(\frac r2\right)\right)^2$ since $\phi$ is a homomorphism. Therefore, $\left(\phi\left(\frac r2\right)\right)^2 = 2$ which quickly implies $\phi\left(\frac r2\right) = \sqrt 2$ since $\phi\left(\frac r2\right)$ must be positive. However, $\sqrt 2 \notin H = \Q_{>0}$ while $\phi\left(\frac r2\right) \in H = \Q_{>0}$, a contradiction.

    Hence, $G \not\cong H$.

    \item We prove the forward direction first: assume that $G$ is abelian. Then $f$ is a homomorphism since
    \[
        f(xy) = (xy)^{-1} = y^{-1}x^{-1} = x^{-1}y^{-1} = f(x)f(y)
    \]
    for all $x, y \in G$.

    We now prove the reverse direction: assume that $f$ is a homomorphism, meaning $f(xy) = f(x)f(y) = x^{-1}y^{-1}$. But $f(xy) = (xy)^{-1} = y^{-1}x^{-1}$. Therefore we have $x^{-1}y^{-1} = y^{-1}x^{-1}$ which clearly shows that the group is abelian.

    \item Since $\phi$ is surjective, thus $\im \phi = H$. Let $g_1, g_2 \in G$ and $h_1, h_2 \in H$ such that $\phi(g_1) = h_1$ and $\phi(g_2) = h_2$. Consider $\phi(g_1g_2)$.
    \begin{itemize}
        \item On one hand we see $\phi(g_1g_2) = \phi(g_1)\phi(g_2) = h_1h_2$.
        \item On another hand we see $\phi(g_1g_2) = \phi(g_2g_1) = \phi(g_2)\phi(g_1) = h_2h_1$.
    \end{itemize}
    Hence $h_1h_2 = h_2h_1$ which means that $H$ is abelian.

    \item We first prove $\phi(N)$ is a subgroup of $H$ before proving normality.

    The codomain of $\phi$ is $H$, so $\phi(N) \subseteq H$. Note that $e_H \in \phi(N)$ since $e_G \in N$ and $\phi(e_G) = e_H$. Now let $x, y \in \phi(N)$. As $\phi$ is surjective, we know that there exists $n_x, n_y \in N$ where $\phi(n_x) = x$ and $\phi(n_y) = y$. Note that $\phi(n_y^{-1}) = y^{-1}$ and $n_xn_y^{-1} \in N$. Hence, $xy^{-1} = \phi(n_xn_y^{-1}) \in \phi(N)$. By subgroup test (\myref{thrm-subgroup-test}) we see $\phi(N) \leq H$.

    We now show that $\phi(N)$ is a normal subgroup of $H$. Take $g \in G$, $h \in H$, $n \in N$, and $x \in \phi(N)$, such that $\phi(g) = h$ and $\phi(n) = x$. Note that since $N \unlhd G$, thus $gng^{-1} \in N$. So we note
    \begin{align*}
        hxh^{-1} &= \phi(g)\phi(n)\phi(g^{-1})\\
        &= \phi(\underbrace{gng^{-1}}_{\text{In }N})\\
        &\in \phi(N)
    \end{align*}
    which means that $\phi(N) \unlhd H$.

    \item Consider the map $\phi: G \to H, a \mapsto a + n\Z$. We show that $\phi$ is an isomorphism.
    \begin{itemize}
        \item \textbf{Homomorphism}: Let $a, b \in G$. Then
        \begin{align*}
            \phi(a\oplus_n b) &= (a\oplus_n b) + n\Z\\
            &= \{(a \oplus_n b) + pn \vert p \in \Z\}\\
            &= \{a+b + pn \vert p \in \Z\}\\
            &= \{a+b + pn + qn\vert p, q \in \Z\}\\
            &= a+b+n\Z + n\Z\\
            &= (a+n\Z) + (b + n\Z)\\
            &= \phi(a) + \phi(b).
        \end{align*}
        \item \textbf{Injective}: Let $a, b \in G$ such that $\phi(a) = \phi(b)$. Thus
        \[
            \{a + pn \vert p \in \Z \} = \ \{b + qn \vert q \in \Z \}
        \]
        by definition of $\phi$. Hence $a \equiv b \pmod n$. But since $0 \leq a, b < n$, we must have $a = b$.
        \item \textbf{Surjective}: Let $x + n\Z \in H$. We use Euclid's division lemma (\myref{lemma-euclid-division}) to write $x = qn + r$ where $0 \leq r < n$. Note that
        \begin{align*}
            x + n\Z &= \{x + kn \vert k \in \Z\}\\
            &= \{(qn + r) + kn \vert k \in \Z\}\\
            &= \{r + n(\underbrace{q + k}_{\text{In }\Z}) \vertalt k \in \Z \}\\
            &= r + n\Z
        \end{align*}
        with $0 \leq r < n$, meaning $r \in G$. Now observe that $\phi(r) = r+n\Z = x+n\Z$ which means that there is a pre-image for every element in $H$, proving that $\phi$ is surjective.
    \end{itemize}
    Therefore $\phi$ is an isomorphism, which thus implies that $G \cong H$.

    \item Consider the map $\phi: G \to G/N$ such that $g \mapsto gN$. We note that $\phi$ is a homomorphism because
    \[
        \phi(gh) = (gh)N = (gN)(hN) = \phi(g)\phi(h).
    \]
    We note by \myref{prop-homomorphism-inverse-is-subgroup} that $A = \phi^{-1}(B) \leq G$. Thus
    \begin{align*}
        \phi^{-1}(N) &= \{g \in G \vert \phi(g) = N\}\\
        &= \{g \in G \vert gN = N\}\\
        &= \{g \in G \vert g \in N\}\\
        &= G \cap N\\
        &= N\\
        &\subseteq A
    \end{align*}
    by assumption. Since $N$ is a group, we know $N \leq A$. Furthermore $N \leq A \leq G$ and $N \unlhd G$, meaning $N \unlhd A$ (since $gN = Ng$ for all $g \in G$, including those in $A$). Hence $A/N$ is a group.

    Now clearly $\phi$ is surjective (since for any $gN \in G/N$ we know $\phi(g) = gN$), which means that $\phi(\phi^{-1}(B)) = B$. Since $\phi^{-1}(B) = A$, so $\phi(A) = B$. Finally,
    \begin{align*}
        \phi(A) &= \{\phi(a) \vert a \in A\}\\
        &= \{aN \vert a \in A\}\\
        &= A/N
    \end{align*}
    which means $B = A/N$.
\end{questions}
