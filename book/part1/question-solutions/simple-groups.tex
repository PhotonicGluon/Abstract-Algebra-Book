\section{Simple Groups}
\subsection*{Exercises}
\begin{questions}
    \item We find the number of Sylow 2- and Sylow 3-subgroups (which we denote by $n_2$ and $n_3$ respectively) of the group of order 12 (call it $G$) using the Third Sylow Theorem (\myref{thrm-sylow-3}).
    \begin{itemize}
        \item For $n_2$, note $12 = 2^2 \times 3$. So,
        \begin{itemize}
            \item $3 \vert n_2$ meaning $n_2 \in \{1, 3\}$; and
            \item $n_2 \equiv 1 \pmod 2$ meaning $n_2 \in \{1, 3, 5, \dots\}$.
        \end{itemize}
        Hence $n_2 = 1$ or $n_2 = 3$.

        \item For $n_3$, note $12 = 3 \times 2^2$. So,
        \begin{itemize}
            \item $4 \vert n_3$ meaning $n_3 \in \{1, 2, 4\}$; and
            \item $n_3 \equiv 1 \pmod 3$ meaning $n_3 \in \{1, 4, 7, \dots\}$.
        \end{itemize}
        Hence $n_3 = 1$ or $n_3 = 4$.
    \end{itemize}

    Now by way of contradiction suppose both $n_2$ and $n_3$ are not 1. Thus $n_2 = 3$ and $n_3 = 4$. We consider the number of elements of a certain order.
    \begin{itemize}
        \item Number of elements with order 2 or 4 is $3(4-1) = 9$, since each of the 3 Sylow 2-subgroups has 4 elements, 1 of which is the identity.
        \item Number of elements with order 3 is $4(3-1) = 8$, since each of the 4 Sylow 3-subgroups has 3 elements, 1 of which is the identity.
    \end{itemize}
    Therefore, the number of elements in $G$ must be at least $9 + 8 = 17 > 12$, a contradiction.

    Thus, at least one of $n_2$ and $n_3$ must be 1, meaning that there must exist a normal subgroup of order 4 or 3 (or both) by \myref{corollary-sylow-subgroup-is-normal-if-it-is-unique}.

    \item \begin{partquestions}{\alph*}
        \item Note $15 = 3 \times 5$, so \myref{problem-group-of-order-pq-has-normal-subgroup-of-order-q} means that there exists a unique (and hence normal) subgroup of order 5.
        \item Note $20 = 2^2 \times 5$. Then the Third Sylow Theorem (\myref{thrm-sylow-3}) tells us that
        \begin{itemize}
            \item $2^2 \vert n_5$, so $n_5 \in \{1, 2, 4\}$; and
            \item $n_5 \equiv 1 \pmod 5$, so $n_5 \in \{1, 6, 11, 16, \dots\}$.
        \end{itemize}
        Hence $n_5 = 1$, so there exists a unique (and hence normal) subgroup of order 5.
    \end{partquestions}

    \item \begin{partquestions}{\alph*}
        \item Note that $\sigma = \begin{pmatrix}1&3\end{pmatrix}\begin{pmatrix}2&3\end{pmatrix}\begin{pmatrix}2&4\end{pmatrix}\begin{pmatrix}4&5\end{pmatrix}$, so we see $\sigma$ is an even permutation (\myref{thrm-parity-of-permutation}), and since the highest integer that appears in $\sigma$ is 5, thus $\sigma \in \An5$.

        \item Observe that
        \begin{itemize}
            \item $\sigma^1 = \sigma \neq \id$;
            \item $\sigma^2 = \begin{pmatrix}1&2&5&3&4\end{pmatrix} \neq \id$;
            \item $\sigma^3 = \begin{pmatrix}1&4&3&5&2\end{pmatrix} \neq \id$;
            \item $\sigma^4 = \begin{pmatrix}1&5&4&2&3\end{pmatrix} \neq \id$; and
            \item $\sigma^5 = \id$.
        \end{itemize}
        Hence $|\sigma| = 5$ and thus the order of $\langle \sigma \rangle$ is 5.

        \item Consider the permutation $\pi = \begin{pmatrix}1&2&3&4&5\end{pmatrix}$. We note
        \[
            \pi = \begin{pmatrix}1&2\end{pmatrix}\begin{pmatrix}2&3\end{pmatrix}\begin{pmatrix}3&4\end{pmatrix}\begin{pmatrix}4&5\end{pmatrix} \in \An5
        \]
        Also,
        \begin{itemize}
            \item $\pi^1 = \pi \neq \id$;
            \item $\pi^2 = \begin{pmatrix}1&3&5&2&4\end{pmatrix} \neq \id$;
            \item $\pi^3 = \begin{pmatrix}1&4&2&5&3\end{pmatrix} \neq \id$;
            \item $\pi^4 = \begin{pmatrix}1&5&4&3&2\end{pmatrix} \neq \id$; and
            \item $\pi^5 = \id$,
        \end{itemize}
        meaning $|\langle \pi \rangle| = 5$ with $\langle \pi \rangle \neq \langle \sigma \rangle$, so we have found another subgroup of $\An5$.
    \end{partquestions}

    \item Note that $\sigma \in G$, and
    \begin{align*}
        \sigma\Stab{G}{x}\sigma^{-1} &= \{\underbrace{\sigma\pi\sigma^{-1}}_{\text{Set as }\pi'} \vertalt \pi \in G,\; \pi(x) = x\}\\
        &= \{\pi' \vertalt \underbrace{\sigma^{-1}\pi'\sigma \in G}_{\text{True if } \pi' \in G},\; \sigma^{-1}\pi'\sigma(x) = x\}\\
        &= \{\pi' \vert \pi' \in G,\; \sigma^{-1}\pi'(\sigma(x)) = x\}\\
        &= \{\pi' \vert \pi' \in G,\; \pi'(\sigma(x)) = \sigma(x)\}\\
        &= \{\pi \vert \pi \in G,\; \pi(\sigma(x)) = \sigma(x)\}\\
        &= \Stab{G}{\sigma(x)}
    \end{align*}
    which proves the claim.

    \item Observe that if $\sigma(i) = j$, then we must have
    \[
        \pi\sigma\pi^{-1}(\pi(i)) = \pi\sigma(i) = \pi(j).
    \]
    Thus if the ordered pair $(i, j)$ appears in the cycle decomposition of $\sigma$, then the ordered pair $(\pi(i), \pi(j))$ appears in the cycle decomposition of $\pi\sigma\pi^{-1}$, completing the proof of the claim.
\end{questions}

\subsection*{Problems}
No problems.
