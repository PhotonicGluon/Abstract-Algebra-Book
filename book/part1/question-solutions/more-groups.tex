\section{More Types of Groups}
\subsection*{Exercises}
\begin{questions}
    \item Let $G = \Z_{mn}$ and $H = \{0, n, 2n, \dots, (m-1)n\}$. Clearly $H$ is a subgroup of $G$ of order $m$. By \myref{problem-subgroup-of-cyclic-group-is-cyclic} we know $H$ is normal and cyclic with order $m$ and by \myref{exercise-quotient-group-of-cyclic-group-is-cyclic} we know $G/H$ is cyclic. The order of $G/H$ is $\frac{|G|}{|H|} = \frac{mn}{m} = n$ by Lagrange's theorem (\myref{thrm-lagrange}), meaning that $G/H \cong \Z_n$. Hence, $\Z_{mn}/\Z_m \cong G/H \cong \Z_n$.

    \item Note 0 is the identity in $\Z_n$. By \myref{lemma-order-of-an-element-that-is-equivalent-to-identity} we know that if $12$ is equivalent to the identity in $\Z_n$, then $12 = mn$ for some integer $m$. Since $n > 0$ we restrict $m$ to positive integers. Now $12 = 2^2 \times 3$. Thus the possible values of $n$ are
    \begin{itemize}
        \item $n = 1$ with $m = 12$;
        \item $n = 2$ with $m = 6$;
        \item $n = 3$ with $m = 4$;
        \item $n = 4$ with $m = 3$;
        \item $n = 6$ with $m = 2$; and
        \item $n = 12$ with $m = 1$.
    \end{itemize}

    \item $|10| = \frac{210}{\gcd(10, 210)} = \frac{210}{10} = 21$, $|42| = \frac{210}{\gcd(42, 210)} = \frac{210}{42} = 5$, $|75| = \frac{210}{\gcd(75, 210)} = \frac{210}{15} = 14$, and $|140| = \frac{210}{\gcd(140, 210)} = \frac{210}{70} = 3$.

    \item \begin{partquestions}{\alph*}
        \item Note that $10 = 2 \times 5$. Generators of the group $\Z_{10}$ has to satisfy $\gcd(m, 10) = 1$ by \myref{corollary-element-in-cyclic-group-is-generator-iff-gcd-is-1}. The positive integers that satisfy this requirement (and which are less than 10) are 1, 3, 7, 9. Thus they are the generators of $\Z_{10}$.
        \item Note that 101 is prime. Hence all positive integers from 1 to 100 (inclusive) are generators. (Note that 0 is not a generator of $\Z_{101}$ since 0 is the identity.)
    \end{partquestions}

    \item We show that all subgroups of $\mathrm{Q}$ are, in fact, normal. We consider the first definition of the quaternion group.
    \begin{itemize}
        \item Clearly $\{1\} \lhd \mathrm{Q}$ and $\mathrm{Q} \unlhd \mathrm{Q}$.
        \item The subgroups $\langle i\rangle$, $\langle j\rangle$, and $\langle k\rangle$ have order 4. Therefore, Lagrange's theorem (\myref{thrm-lagrange}) tells us that they have index 2. Hence these subgroups are normal by \myref{problem-subgroup-of-index-2}.
        \item Consider the subgroup $\langle -1 \rangle = \{1, -1\}$. \begin{itemize}
            \item $1\langle -1 \rangle = \langle -1 \rangle1$, since 1 is the identity;
            \item $-1\langle -1 \rangle = \{1, -1\} = \langle -1 \rangle(-1)$;
            \item $i\langle -1 \rangle = \{-i, i\} = \langle -1 \rangle i$;
            \item $-i\langle -1 \rangle = \{i, -i\} = \langle -1 \rangle (-i)$;
            \item $j\langle -1 \rangle = \{-j, j\} = \langle -1 \rangle j$;
            \item $-j\langle -1 \rangle = \{j, -j\} = \langle -1 \rangle (-j)$;
            \item $k\langle -1 \rangle = \{-k, k\} = \langle -1 \rangle k$; and
            \item $-k\langle -1 \rangle = \{k, -k\} = \langle -1 \rangle (-k)$.
        \end{itemize}
        Thus $\langle -1 \rangle$ is normal.
    \end{itemize}
    Hence all subgroups of $\mathrm{Q}$ are normal.

    \item $\begin{pmatrix}2&6\end{pmatrix} = \begin{pmatrix}2&3\end{pmatrix}\begin{pmatrix}3&4\end{pmatrix}\begin{pmatrix}4&5\end{pmatrix}\begin{pmatrix}5&6\end{pmatrix}\begin{pmatrix}4&5\end{pmatrix}\begin{pmatrix}3&4\end{pmatrix}\begin{pmatrix}2&3\end{pmatrix}$.

    \item Note that $\begin{pmatrix}1&3&2&5&4\end{pmatrix} = \begin{pmatrix}1&4\end{pmatrix}\begin{pmatrix}1&5\end{pmatrix}\begin{pmatrix}1&2\end{pmatrix}\begin{pmatrix}1&3\end{pmatrix}$. \myref{thrm-parity-of-permutation} tells us that $\begin{pmatrix}1&3&2&5&4\end{pmatrix}$ is even and thus has a sign of $+1$.

    \item Note that $\An{3}$ has order $\frac{3!}{2} = 3$ so we should expect 3 permutations. Clearly the identity is one such permutation. Looking at \myref{example-symmetric-group-of-degree-3} we can find two more, namely $\begin{pmatrix}1&2&3\end{pmatrix}$ and $\begin{pmatrix}1&3&2\end{pmatrix}$.

    \item $\Un{10} = \{1, 3, 7, 9\}$.

    \item By a corollary of Lagrange's theorem (\myref{corollary-order-of-group-multiple-of-order-of-element}), the order of $a$ dives the order of the group $\Un{n}$. Now since $|\Un{n}| = \totient(n)$, thus the order of $a$ divides $\totient(n)$.

    \item $\begin{pmatrix}2&1&2\\1&0&1\\2&1&2\end{pmatrix}$

    \item We already proved that $\Inn{G} \leq \Aut{G}$ so we only need to prove normality.

    Let $\phi \in \Aut{G}$ and $\iota_g \in \Inn{G}$. For brevity let $f = \phi\iota_g\phi^{-1}$. We note that $f \in \Aut{G}$; we need to prove that $f \in \Inn{G}$.

    Suppose $x \in G$. Since $\phi$ is an isomorphism, there exists $w \in G$ such that $x = \phi(w)$, i.e. $w = \phi^{-1}(x)$. So
    \begin{align*}
        f(x) &= \left(\phi\iota_g\phi^{-1}\right)(x)\\
        &= \phi(\iota_g(\phi^{-1}(x)))\\
        &= \phi(\iota_g(w))\\
        &= \phi(gwg^{-1})\\
        &= \phi(g)\phi(w)\phi(g^{-1})\\
        &= \phi(g)x\left(\phi(g)\right)^{-1}
    \end{align*}
    which shows that $f \in \Inn{G}$. Hence, $\Inn{G} \unlhd \Aut{G}$.
\end{questions}

\subsection*{Problems}
\begin{questions}
    \item We note that the two questions are equivalent to finding the orders of 3774 and 1870 in the group $\Z_{10101}$. We note that
    \begin{align*}
        1870 &= 2 \times 5 \times 11 \times 17,\\
        3774 &= 2 \times 3 \times 17 \times 37, \text{ and}\\
        10101 &= 3 \times 7 \times 13 \times 37.
    \end{align*}
    Therefore, $\gcd(1870, 10101) = 1$ and $\gcd(3774, 10101) = 3 \times 37 = 111$. Hence $|1870| = 10101$ and $|3774| = \frac{10101}{111} = 91$. Therefore, $a = 10101$ and $b = 91$.

    \item We claim that $\An{n}$ is non-abelian for any $n > 3$. Note that both $\pi = \begin{pmatrix}1 & 2 & 3\end{pmatrix}$ and $\sigma = \begin{pmatrix}2 & 3 & 4\end{pmatrix}$ are even permutations, and hence are in $\An{n}$ for any $n > 3$. We note
    \begin{itemize}
        \item $\pi\sigma = \begin{pmatrix}1 & 2 & 3\end{pmatrix}\begin{pmatrix}2 & 3 & 4\end{pmatrix} = \begin{pmatrix}1 & 2\end{pmatrix}\begin{pmatrix}3 & 4\end{pmatrix}$; and
        \item $\sigma\pi = \begin{pmatrix}2 & 3 & 4\end{pmatrix}\begin{pmatrix}1 & 2 & 3\end{pmatrix} = \begin{pmatrix}1 & 3\end{pmatrix}\begin{pmatrix}2 & 4\end{pmatrix}$.
    \end{itemize}
    So $\pi\sigma \neq \sigma\pi$. Thus $\An{n}$ is non-abelian for any $n > 3$.

    We note that
    \begin{itemize}
        \item $\An{2}$ has order 1 so $\An{2}$ is the trivial group, which is abelian (and cyclic); and
        \item $\An{3}$ has order 3 so $\An{3}$ is cyclic and thus abelian.
    \end{itemize}
    Thus the largest integer $n$ for which $\An{n}$ is abelian is $n = 3$. Furthermore $\An{k}$ is cyclic if $k = 2$ or $k = 3$.

    \item We first note that
    \[
        \totient(2p^k) = 2p^k\left(1-\frac12\right)\left(1-\frac1p\right) = p^k\left(1-\frac1p\right) = \totient(p^k).
    \]
    Now we are given that $r$ is an odd primitive root of $p^k$. Since $r \in \Un{p^k}$, thus $\gcd(r, 2p^k) = 1$ because $\gcd(r, p^k) = 1$. Now as $r$ is odd, thus $r \in \Un{2p^k}$. Let $n$ be the order of $r$ in $\Un{2p^k}$. Then by \myref{exercise-order-of-a-divides-phi-a} we know $n$ divides $\totient(2p^k)$. At the same time, because $r$ is a generator in $\Un{p^k} \cong \Z_{\phi(p^k)}$, so $\totient(p^k) = \totient(2p^k)$ divides $n$ by \myref{lemma-order-of-an-element-that-is-equivalent-to-identity}. Since $n$ divides $\totient(2p^k)$ and $\totient(2p^k)$ divides $n$ simultaneously, therefore $n = \totient(2p^k) = |\Un{2p^k}|$ which means that $r$ is a primitive root modulo $2p^k$.

    \item \begin{partquestions}{\roman*}
        \item The forward direction is clearly true since if $f_1 = f_2$, then $f_1(x) = f_2(x)$ for all $x \in G$, including $g \in G$. For the reverse direction, assume $f_1(g) = f_2(g)$. Note that
        \[
            f_1(g^k) = (f_1(g))^k = (f_2(g))^k = f_2(g^k)
        \]
        for any integer $k$. Since $g$ is a generator, thus we have $f_1(x) = f_2(x)$ for all $x \in G$, meaning $f_1 = f_2$.

        \item We note $f(g) \in G$. Since $g$ is a generator, hence $f(g) = g^k$ for some integer $k$. Hence any homomorphism from $G$ to $G$ is of the form $f(g) = g^{m_f}$ where $0 \leq m_f \leq n-1$, which means $m_f \in \Z_n$.

        \item Suppose the map $f_2: G \to G$ is another homomorphism where $f_2(g) = g^{m_f}$. Then
        \[
            f(g) = g^{m_f} = f_2(g)
        \]
        which means $f = f_2$ by \textbf{(i)}. Hence the value of $m_f$ is unique to $f$.

        \item Consider $f_1(f_2(g))$. On one hand,
        \[
            f_1(f_2(g)) = f_1(g^{m_{f_2}}) = (f_1(g))^{m_{f_2}} = g^{m_{f_1}m_{f_2}},
        \]
        while on the other,
        \[
            f_1(f_2(g)) = (f_1 \circ f_2)(g) = g^{m_{f_1\circ f_2}}
        \]
        by definition of $m_f$ as introduced in \textbf{(ii)}. Therefore $m_{f_1\circ f_2} \equiv m_{f_1}m_{f_2} \pmod n$. In other words, $m_{f_1\circ f_2} = m_{f_1} \otimes_n m_{f_2}$.

        \item We prove the forward direction first by assuming that the map $f$ is an automorphism. Hence $f$ is surjective, meaning that there exists $a \in G$ such that $f(a) = g$. Since $a \in G$ thus $a = g^k$ for some $k \in \Z_n$ (we will show $k \in \Un{n}$ later). Observe
        \[
            g = f(a) = f(g^k) = (f(g))^k = g^{m_fk}
        \]
        which means $m_fk \equiv 1 \pmod n$. By \myref{prop-multiplicative-inverse-exists-iff-coprime}, this means that we have $\gcd(m_f, n) = 1$ and $\gcd(k, n) = 1$. Therefore, $m_f$ and $k$ are in $\Un{n}$. Hence $k$ is the multiplicative inverse of $m_f$.

        We now prove the reverse direction. Assume $m_f$ has a multiplicative inverse (say $k$), meaning $m_fk \equiv 1 \pmod n$. As above this means that both $m_f$ and $k$ are in $\Un{n}$. We show that $f$ is a bijection.
        \begin{itemize}
            \item \textbf{Injective}: Suppose $x, y \in G$ such that $f(x) = f(y)$. Since $g$ is a generator we may take $x = g^p$ and $y = g^q$ for some integers $p$ and $q$. Hence we have $g^{m_fp} = g^{m_fq}$. Then
            \[
                \left(g^{m_fp}\right)^k = g^{km_fp} = \left(g^{km_f}\right)^p = g^p
            \]
            and $\left(g^{m_fq}\right)^k = g^q$. Hence this implies $g^p = g^q$ which means $x = y$.
            \item \textbf{Surjective}: Suppose $x \in G$. Since $g$ is a generator we may write $x = g^p$ for some integer $p$. Then $f(g^{kp}) = g^{m_fkp} = g^p = x$.
        \end{itemize}
        Also $f$ is given to be a homomorphism. Hence $f$ is an isomorphism. Since $f: G \to G$, it is thus an automorphism.

        \item We prove that $\phi$ is an isomorphism.
        \begin{itemize}
            \item \textbf{Homomorphism}: Let $f_1, f_2 \in \Aut{G}$. Then
            \begin{align*}
                \phi(f_1\circ f_2) &= m_{f_1\circ f_2} & (\text{definition of } m_f \text{ in }\textbf{(ii)})\\
                &= m_{f_1} \otimes_n m_{f_2} & (\text{by \textbf{(iv)}})\\
                &= \phi(f_1)\otimes_n\phi(f_2),
            \end{align*}
            which means $\phi$ is a homomorphism.

            \item \textbf{Injective}: Suppose we have $f_1, f_2 \in \Aut{G}$ such that $\phi(f_1) = \phi(f_2)$. Thus $m_{f_1} = m_{f_2}$ by definition of $\phi$. However, we know that the value of $m$ uniquely defines a homomorphism from $G$ to $G$ from \textbf{(iii)}. Hence $f_1 = f_2$, which shows that $\phi$ is injective.

            \item \textbf{Surjective}: Suppose $r \in \Un{n}$. Define the map $f: G \to G$ where $f(g) = g^r$. Since $r \in \Un{n}$ it has a multiplicative inverse, which means that $f$ is an automorphism by \textbf{(v)}. Clearly $\phi(f) = r$, so $r$ has a pre-image. So $\phi$ is surjective.
        \end{itemize}
        Hence $\phi$ is an isomorphism, meaning $\Aut{G} \cong \Un{n}$.
    \end{partquestions}
\end{questions}
