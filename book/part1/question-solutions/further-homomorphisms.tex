\section{Further Properties of Homomorphisms}
\subsection*{Exercises}
\begin{questions}
    \item $\phi$ is a homomorphism since
    \begin{align*}
        \phi(a \oplus_3 b) &= 2(a\oplus_3 b)\\
        &= (2a) \oplus_6 (2b)\\
        &= \phi(a) \oplus_6 \phi(b).
    \end{align*}
    The image of $\phi$ is $\{0, 2, 4\}$.

    \item $\phi$ is a homomorphism since
    \begin{align*}
        \phi(a+b) &= i^{a+b}\\
        &=i^ai^b\\
        &=\phi(a)\phi(b).
    \end{align*}
    The kernel is the set of values which map to the identity of $H$, i.e. $\ker\phi = \{n \in \Z \vert \phi(n) = 1\}$. Notice that $H$ is a cyclic group and $|i| = 4$. Thus $i^4 = 1$. Furthermore $i^8 = (i^4)^2 = 1, i^{12} = 1, \dots, i^{4k} = 1$. Thus $\ker\phi = \{4n \vert n \in \Z\} = 4\Z$.

    \item We prove the forward direction first. Suppose that $\phi$ is injective. Clearly $\phi(e_G) = e_H$. Let $x \in \ker\phi$, meaning $\phi(x) = e_H$. Then we see $\phi(x) = \phi(e_G) = e_H$ which means $x = e_G$ by injectivity of $\phi$. Hence the kernel is trivial.

    Now we prove the reverse direction. Suppose the kernel of $\phi$ is trivial, i.e. $\ker \phi = \{e_G\}$. Suppose now there exist elements $x, y \in G$ such that $\phi(x) = \phi(y)$. One sees that
    \begin{align*}
        \phi(x^{-1}) &= \left(\phi(x)\right)^{-1}\\
        &= \left(\phi(y)\right)^{-1}\\
        &= \phi(y^{-1}).
    \end{align*}
    Hence,
    \begin{align*}
        \phi(xy^{-1}) &= \phi(x)\phi(y^{-1})\\
        &= \phi(x)\phi(x^{-1})\\
        &= \phi(xx^{-1})\\
        &= \phi(e_G)\\
        &= e_H.
    \end{align*}
    Now since the kernel is trivial, this must mean that $xy^{-1} = e_G$ which immediately means that $x = y$. Hence $\phi$ is injective.

    \item Note that $G / \ker \phi \cong \im \phi$ by the First Isomorphism Theorem (\myref{thrm-isomorphism-1}). Furthermore, we know that $|G / \ker \phi| = \frac{|G|}{|\ker\phi|}$ by Lagrange's theorem (\myref{thrm-lagrange}). Hence, $\frac{|G|}{|\ker\phi|} = |\im\phi|$ which quickly leads to the result.

    \item The Second Isomorphism Theorem (\myref{thrm-isomorphism-2}), statement 6, tells us that $H / (H\cap N) \cong HN / N$. Taking orders on both sides yields $\frac{|H|}{|H \cap N|} = \frac{|HN|}{|N|}$. Rearranging yields the required result.

    \item \begin{partquestions}{\roman*}
        \item Note $H = x\Z = \{ax \vert a \in \Z\}$ and $N = mx\Z = \{a(mx) \vert a \in \Z\}$, which necessarily means $N \subseteq H$.
        \item Let $G = \Z$. Then both $H$ and $N$ are clearly subgroups of $G$. Now since $G$ is abelian (as addition is commutative), therefore $H$ and $N$ are normal by \myref{prop-subgroup-of-abelian-group-is-normal}.
        \item The Third Isomorphism Theorem (\myref{thrm-isomorphism-3}) tells us
        \[
            (G/N)/(H/N) \cong G/H.
        \]
        We know from \myref{problem-Zn-isomorphic-to-Z-by-nZ} that $|G/H| = |\Z/(x\Z)| = x$ and $|G/N| = |\Z/(y\Z)| = y$. Hence
        \[
            \frac{x}{|H/N|} = y
        \]
        which quickly implies $|H/N| = \frac yx$.
    \end{partquestions}
\end{questions}

\subsection*{Problems}
\begin{questions}
    \item Construct the map $\phi: G \to \{e\}$ where $\phi(g) = e$. Clearly $\phi$ is a homomorphism as
    \[
        \phi(gh) = e = ee = \phi(g)\phi(h).
    \]
    Also, one sees that $\im\phi = \{e\}$ and $\ker\phi = G$. The First Isomorphism Theorem (\myref{thrm-isomorphism-1}) tells us that
    \[
        G / \ker\phi \cong \im\phi
    \]
    which immediately implies $G/G \cong \{e\}$.

    \item Consider $\phi: G \to R$ where $(x, y) \mapsto x\sqrt3 - y\sqrt2$. We show that $\phi$ is a homomorphism, find its image, and find its kernel.
    \begin{itemize}
        \item \textbf{Homomorphism}: Let $(x_1, y_1), (x_2, y_2) \in G$. Then
        \begin{align*}
            &\phi((x_1,y_1)(x_2,y_2))\\
            &= \phi((x_1+x_2,y_1+y_2))\\
            &= (x_1+x_2)\sqrt3 - (y_1+y_2)\sqrt2\\
            &= (x_1\sqrt3 - y_1\sqrt2) + (x_2\sqrt3 - y_2\sqrt2)\\
            &= \phi((x_1, y_1)) + \phi((x_2, y_2)).
        \end{align*}

        \item \textbf{Image}: We show that $\phi$ is surjective to prove hat $\im\phi = R$. For any $r \in R$, we have $\phi\left(\left(\frac{r}{\sqrt3}, 0\right)\right) = \frac{r}{\sqrt3} \times \sqrt3 + 0 = r$ and clearly $\left(\frac{r}{\sqrt3}, 0\right) \in G$, so $\phi$ is surjective.

        \item \textbf{Kernel}: We note that
        \begin{align*}
            \ker\phi &= \{(x, y) \in G \vert \phi((x, y)) = 0\}\\
            &= \{(x, y) \in G \vert x\sqrt3-y\sqrt2 = 0\}\\
            &= \left\{(x, y) \in G \vert y = \frac{\sqrt{3}}{\sqrt{2}}x\right\}\\
            &= \left\{\left(x, \frac{\sqrt{3}}{\sqrt{2}}x\right) \vert x \in \R\right\}\\
            &= \left\{\left(r\sqrt2, \frac{\sqrt{3}}{\sqrt{2}}(r\sqrt2)\right) \vert r \in \R\right\}\\
            &= \{(r\sqrt2, r\sqrt3) \vert r \in \R\}\\
            &= H.
        \end{align*}
    \end{itemize}
    Thus the First Isomorphism Theorem (\myref{thrm-isomorphism-1}) tells us that $G / H \cong R$.

    \item Since $K \subseteq H$ we see
    \begin{align*}
        HK &= \{hk \vert h \in H, k \in K \subseteq H\}\\
        &\subseteq \{hk \vert h \in H, k \in H\}\\
        &= \{h_1h_2 \vert h_1, h_2 \in H\}\\
        &= \{h \vert h \in H\}\\
        &= H.
    \end{align*}
    Therefore $HK \subseteq H$. Also, we know that $H \leq HK$ by the Second Isomorphism Theorem (\myref{thrm-isomorphism-2}), statement 3, so $H \subseteq HK$. Hence we see that $H \subseteq HK \subseteq H$ which means $HK = H$ as required.

    \item \begin{partquestions}{\alph*}
        \item Consider the map $\phi: I \to G, (g, g^{-1}) \mapsto g$. We show that $\phi$ is an isomorphism.
        \begin{itemize}
            \item \textbf{Homomorphism}: Recall that $G$ is abelian, so $gh = hg$ for any $g, h \in G$. Let $(g, g^{-1}), (h, h^{-1}) \in I$. Then
            \begin{align*}
                \phi((g, g^{-1})(h, h^{-1})) &= \phi((gh, g^{-1}h^{-1}))\\
                &= \phi((gh, h^{-1}g^{-1}))\\
                &= \phi((gh, (gh)^{-1}))\\
                &= gh\\
                &= \phi((g, g^{-1}))\phi((h, h^{-1})).
            \end{align*}

            \item \textbf{Injective}: Suppose we have $(g, g^{-1}), (h, h^{-1}) \in I$ such that $\phi((g, g^{-1})) = \phi((h, h^{-1}))$. Then we see $g = h$ by definition of $\phi$ which clearly means $(g, g^{-1}) = (h, h^{-1})$.

            \item \textbf{Surjective}: Suppose $g \in G$. Then we see $(g, g^{-1}) \in I$ and $\phi((g, g^{-1})) = g$. Thus $g \in G$ has a pre-image $(g, g^{-1}) \in I$, so $\phi$ is surjective.
        \end{itemize}
        Hence $\phi$ is an isomorphism, meaning $I \cong G$.

        \item Consider the map $\psi: G^2 \to G, (g_1, g_2) \mapsto g_1g_2$. We show that $\psi$ is a homomorphism, then find its image and kernel.
        \begin{itemize}
            \item \textbf{Homomorphism}: Let $(g_1, g_2), (h_1, h_2) \in G^2$, so
            \begin{align*}
                &\psi((g_1, g_2)(h_1, h_2))\\
                &= \psi((g_1h_1, g_2h_2))\\
                &= g_1h_1g_2h_2\\
                &= g_1g_2h_1h_2 & (\text{since } G \text{ is abelian})\\
                &= (g_1g_2)(h_1h_2)\\
                &= \psi((g_1, g_2))\psi((h_1, h_2))
            \end{align*}
            which means $\psi$ is a homomorphism.

            \item \textbf{Image}: We show that $\psi$ is surjective to show $\im \psi = G$. Consider any $g \in G$. Clearly we have $\psi((g, e)) = ge = g$, so $\psi$ is surjective.

            \item \textbf{Kernel}:
            \begin{align*}
                \ker\psi &= \{(g, h) \in G^2 \vert \psi((g, h)) = e\}\\
                &= \{(g, h) \in G^2 \vert gh = e\}\\
                &= \{(g, h) \in G^2 \vert h = g^{-1}\}\\
                &= \{(g, g^{-1}) \ | g \in G\}\\
                &= I.
            \end{align*}
        \end{itemize}

        Thus we have $G^2 / I \cong G$ by the First Isomorphism Theorem (\myref{thrm-isomorphism-1}).
    \end{partquestions}

    \item We show that $\phi$ is an isomorphism.
    \begin{itemize}
        \item \textbf{Homomorphism} Let $am, bm \in m\Z$. Then
        \begin{align*}
            \phi(am + bm) &= \phi((a+b)m)\\
            &= (a+b) + \frac nm \Z\\
            &= \left(a + \frac nm \Z\right) + \left(b + \frac nm \Z\right)\\
            &= \phi(am) + \phi(bm)
        \end{align*}
        which means $\phi$ is a homomorphism.

        \item \textbf{Image}: Suppose $k + \frac nm \Z \in \Z/(\frac nm \Z)$. Clearly $\phi(k) = k + \frac nm \Z$ which means $\phi$ is surjective. Hence $\im\phi = \Z/(\frac nm \Z)$.

        \item \textbf{Kernel}:
        \begin{align*}
            \ker\phi &= \left\{am \vert \phi(am) = \frac nm \Z\right\}\\
            &= \left\{am \vert a + \frac nm \Z = \frac nm \Z\right\}\\
            &= \left\{am \vert a  = k\left(\frac nm\right),\; k \in \Z\right\}\\
            &= \left\{k\left(\frac nm\right)m \vert k \in \Z\right\}\\
            &= \{kn \vert k \in \Z\}\\
            &= n\Z
        \end{align*}
        so the kernel of $\phi$ is $n\Z$.
    \end{itemize}
    Hence the First Isomorphism Theorem (\myref{thrm-isomorphism-1}) tells us that $G/H \cong \Z/(\frac nm \Z)$. But \myref{problem-Zn-isomorphic-to-Z-by-nZ} tells us that $\Z/(\frac nm \Z) \cong \Z_{\frac nm}$. Hence $G/H \cong \Z_{\frac nm}$.
\end{questions}
