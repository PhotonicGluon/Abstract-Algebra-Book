\section{Basics of Groups}
\subsection*{Exercises}
\begin{questions}
    \item The Cayley table of $(\Z_6, \otimes_6)$ is shown below.
    \begin{table}[H]
        \centering
        \begin{tabular}{|l|l|l|l|l|l|l|}
        \hline
        \textbf{$\otimes_n$} & \textbf{0} & \textbf{1} & \textbf{2} & \textbf{3} & \textbf{4} & \textbf{5} \\ \hline
        \textbf{0} & 0 & 0 & 0 & 0 & 0 & 0 \\ \hline
        \textbf{1} & 0 & 1 & 2 & 3 & 4 & 5 \\ \hline
        \textbf{2} & 0 & 2 & 4 & 0 & 2 & 4 \\ \hline
        \textbf{3} & 0 & 3 & 0 & 3 & 0 & 3 \\ \hline
        \textbf{4} & 0 & 4 & 2 & 0 & 4 & 2 \\ \hline
        \textbf{5} & 0 & 5 & 4 & 3 & 2 & 1 \\ \hline
        \end{tabular}
    \end{table}

    Since the identity is $1$, and the row (and column) of 0 does not have a $1$, thus $0$ does not have an inverse. Therefore $(\Z_6, \otimes_6)$ is not a group.

    \item Note that $(xx^{-1})^{-1} = (x^{-1})^{-1}x^{-1}$ by Shoes and that Socks and $(xx^{-1})^{-1} = e^{-1} = e$. Thus $(x^{-1})^{-1}x^{-1} = e$. Multiplying both sides on the right by $x$ yields $(x^{-1})^{-1} = ex = x$.

    \item We induct on $n$.

    The base case of $n = 0$ clearly holds true since
    \begin{align*}
        (x^{-1})^0 &= e & (\text{definition of }g^0 \text{ for any }g\in G)\\
        &= e^{-1} & (\myref{prop-inverse-of-identity-is-identity})\\
        &= (x^0)^{-1}. & (\text{definition of }x^0)
    \end{align*}

    Now assume that the statement holds for some non-negative integer $k$, that is, $(x^{-1})^k = (x^k)^{-1}$. We show that the statement holds for $k + 1$, i.e. $(x^{-1})^{k+1} = (x^{k+1})^{-1}$.

    Observe that
    \begin{align*}
        (x^{-1})^{k+1} &= (x^{-1})^k \ast x^{-1} & (\text{by statement 1})\\
        &= (x^k)^{-1} \ast x^{-1} & (\text{by induction hypothesis})\\
        &= (x\ast x^k)^{-1} & (\text{by Shoes and Socks})\\
        &= (x^{k+1})^{-1} & (\text{by statement 1})
    \end{align*}
    so the statement is true for $k+1$.

    Thus we have $(x^{-1})^n = (x^n)^{-1}$ for any non-negative integer $n$.

    \item \begin{partquestions}{\roman*}
        \item The identity is $1$ since:
        \begin{itemize}
            \item $1 \times 1 = 1$;
            \item $1 \times (-1) = (-1) \times 1 = -1$;
            \item $1 \times i = i \times 1 = i$; and
            \item $1 \times (-i) = (-i) \times 1 = -i$.
        \end{itemize}
        \item The order of the identity $1$ is 1, so we look at the other elements:
        \begin{itemize}
            \item $|-1| = 2$ since $-1 \neq 1$ and $(-1)^2 = -1 \times -1 = 1$.
            \item $|i| = 4$ since $i \neq 1$, $i^2 = -1 \neq 1$, $i^3 = -i \neq 1$, but $i^4 = 1$.
            \item $|-i| = 4$ since $-i \neq 1$, $(-i)^2 = -1 \neq 1$, $(-i)^3 = i \neq 1$, but $(-i)^4 = 1$.
        \end{itemize}
    \end{partquestions}

    \item $-i$ is the other generator since $(-i)^1 = -i$, $(-i)^2 = -1$, $(-i)^3 = i$, and $(-i)^4 = 1$.

    \item We see that
    \begin{align*}
        rsr^4sr^3 &= r(sr^4)(sr^3)\\
        &= r(r^2s)(r^3s)\\
        &= r^3sr^3s\\
        &= r^3(sr^3)s\\
        &= r^3(r^3s)s\\
        &= r^6s^2\\
        &= e.
    \end{align*}
\end{questions}

\subsection*{Problems}
\begin{questions}
    \item The Cayley table of $D_4$ is as follows.
    \begin{table}[H]
        \centering
        \begin{tabular}{|l|l|l|l|l|l|l|l|l|}
            \hline
            $\ast$ & $\boldsymbol{e}$ & $\boldsymbol{r}$ & $\boldsymbol{r^2}$ & $\boldsymbol{r^3}$ & $\boldsymbol{s}$ & $\boldsymbol{rs}$ & $\boldsymbol{r^2s}$ & $\boldsymbol{r^3s}$ \\ \hline
            $\boldsymbol{e}$ & $e$ & $r$ & $r^2$ & $r^3$ & $s$ & $rs$ & $r^2s$ & $r^3s$ \\ \hline
            $\boldsymbol{r}$ & $r$ & $r^2$ & $r^3$ & $e$ & $rs$ & $r^2s$ & $r^3s$ & $s$ \\ \hline
            $\boldsymbol{r^2}$ & $r^2$ & $r^3$ & $e$ & $r$ & $r^2s$ & $r^3s$ & $s$ & $rs$ \\ \hline
            $\boldsymbol{r^3}$ & $r^3$ & $e$ & $r$ & $r^2$ & $r^3s$ & $s$ & $rs$ & $r^2s$ \\ \hline
            $\boldsymbol{s}$ & $s$ & $r^3s$ & $r^2s$ & $rs$ & $e$ & $r^3$ & $r^2$ & $r$ \\ \hline
            $\boldsymbol{rs}$ & $rs$ & $s$ & $r^3s$ & $r^2s$ & $r$ & $e$ & $r^3$ & $r^2$ \\ \hline
            $\boldsymbol{r^2s}$ & $r^2s$ & $rs$ & $s$ & $r^3s$ & $r^2$ & $r$ & $e$ & $r^3$ \\ \hline
            $\boldsymbol{r^3s}$ & $r^3s$ & $r^2s$ & $rs$ & $s$ & $r^3$ & $r^2$ & $r$ & $e$ \\ \hline
        \end{tabular}
    \end{table}
    \begin{partquestions}{\alph*}
        \item $D_4$ is not abelian because $rs \neq sr = r^3s$.
        \item Note that
        \begin{align*}
            r^3 sr sr^3 sr^3 sr^2 &= r^3srs(r^3s)(r^3s)r^2\\
            &= r^3 srs(e)r^2\\
            &= r^3 sr sr^2\\
            &= r^2(rs rs)r^2\\
            &= r^2(e)r^2\\
            &= r^4\\
            &= e.
        \end{align*}
    \end{partquestions}

    \item We need to prove each of the group axioms in order to prove that $(\Q, +)$ is indeed a group.
    \begin{itemize}
        \item \textbf{Closure}: Let $\frac ab, \frac cd \in \Q$. Note that $\frac{ad+bc}{bd} \in \Q$. Therefore $\Q$ is closed under addition.

        \item \textbf{Associativity}: Addition is associative by \myref{axiom-addition-is-associative}.

        \item \textbf{Identity}: 0 is the identity since
        \[
            0 + \frac ab = \frac ab + 0 = \frac ab
        \]
        for any $\frac ab \in \Q$.

        \item \textbf{Inverse}: For any $\frac ab \in \Q$ its inverse is $-\frac ab$ because
        \[
            \frac ab + \left(-\frac ab\right) = \left(-\frac ab\right) + \frac ab = 0.
        \]
    \end{itemize}
    Furthermore addition is commutative by \myref{axiom-addition-is-commutative}. Therefore $(\Q, +)$ is an abelian group.

    \item If every element in $G$ is its own inverse, then for every element $g \in G$ we know $g^{-1} = g$. Consider $(gh)^{-1}$ where $g, h \in G$. On one hand, by Shoes and Socks, $(gh)^{-1} = h^{-1}g^{-1} = hg$ since each element is its own inverse. On the other hand, because $gh \in G$ thus $(gh)^{-1} = gh$. So $gh = hg$ which means $G$ is abelian.

    \item Recall that $n = |x|$ is the smallest positive integer that satisfies $x^n = e$.

    We prove the forward direction first. Suppose $m$ is a multiple of $n$, say $m = qn$ for some integer $q$. Then
    \[
        x^m = x^{qn} = \left(x^n\right)^q = e^q = e
    \]
    which means $x^m = e$.

    We now prove the reverse direction. Suppose $x^m = e$. Using Euclid's division lemma (\myref{lemma-euclid-division}), we write $m = qn + r$ where $q$ and $r$ are integers with $0 \leq r < n$. Hence
    \[
        x^m = x^{qn + r} = x^{qn}x^r = \left(x^n\right)^qx^r = e^qx^r = x^r.
    \]
    Note that for all integers $k$ where $1 \leq k < n$, we have $x^k \neq e$ since $n$ is the smallest positive integer such that $x^n = e$. Hence, if $x^r = e$, we conclude $r = 0$. Therefore $m = qn$, meaning $m$ is a multiple of $n$.

    \item \begin{partquestions}{\alph*}
        \item Note that $(gh)^2 = ghgh$. Given that $(gh)^2 = g^2h^2 = gghh$, by cancellation law, we see $hg = gh$ which means $G$ is abelian.

        \item Suppose $G$ is abelian. Clearly $(gh)^1 = gh$. Suppose $(gh)^{k} = g^kh^k$ for some positive integer $k$. Then
        \begin{align*}
            (gh)^{k+1} &= (gh)(gh)^k\\
            &= (gh)(g^kh^k) & (\text{by induction hypothesis})\\
            &= ghg^kh^k\\
            &= g(hg^k)h^k\\
            &= g(g^kh)h^k & (\text{since } G \text{ is abelian})\\
            &= gg^khh^k\\
            &= g^{k+1}h^{k+1}
        \end{align*}
        so $(gh)^{k+1} = g^{k+1}h^{k+1}$ assuming $(gh)^k = g^kh^k$. Thus the claim is proven by mathematical induction.
    \end{partquestions}

    \item Note that $|1| = n$ since $1^2 = 1 \oplus_n 1 = 2$, $1^3 = 1 \oplus_n 1 \oplus_n 1 = 3$, $1^4 = 4$, ..., $1^{n-1} = n-1$ and $1^n = 0$ which is the identity. Since the group $(\Z_n, \oplus_n)$ has an element with the same order as the group, it is thus cyclic with order $n$ and generator 1.

    \item We show that $(A, \circ)$ is a group.
    \begin{itemize}
            \item \textbf{Closure}: Function composition is closed by definition.
            \item \textbf{Associativity}: Function composition is associative.
            \item \textbf{Identity}: By performing brute-force computation, we find that $T^6(x, y) = (x, y)$. Hence $T^6$ is the identity of $A$.
            \item \textbf{Inverse}: If $r = 6$ then $T^r$ is its own inverse. Otherwise, $T^{6-r}$ is the inverse of $T^r$.
    \end{itemize}
    Thus $(A, \circ)$ is a group with order 6.
\end{questions}
