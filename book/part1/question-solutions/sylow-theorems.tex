\section{Sylow Theorems}
\subsection*{Exercises}
\begin{questions}
    \item We note that $12 = 2^2 \times 3$. Thus a Sylow 2-subgroup must have order 4. Clearly $|3| = 4$ so $\langle 3 \rangle = \{0, 3, 6, 9\}$ is the Sylow 2-subgroup of $\Z_{12}$.

    \item Recall that $|\Sn{5}| = 120 = 2^3 \times 3 \times 5$. By a corollary of the First Sylow Theorem (\myref{corollary-sylow-p-subgroup-exists}), $\Syl{p}{G} \neq \emptyset$ if $p$ is 2, 3, or 5.

    \item We prove this by constructing the map $\phi: H \to gHg^{-1}$ where $h \mapsto ghg^{-1}$. We note that $\phi$ is an isomorphism.
    \begin{itemize}
        \item \textbf{Homomorphism}: Let $x, y \in H$. Then
        \[
            \phi(xy) = g(xy)g^{-1} = (gxg^{-1})(gyg^{-1}) = \phi(x)\phi(y)
        \]
        which clearly means that $\phi$ is a homomorphism.
        
        \item \textbf{Injective}: Suppose $x, y \in H$ such that $\phi(x) = \phi(y)$. Then $gxg^{-1} = gyg^{-1}$ which quickly implies $x = y$.
        
        \item \textbf{Surjective}: Suppose $ghg^{-1} \in gHg^{-1}$. Clearly we have $\phi(h) = ghg^{-1}$, so any element in $gHg^{-1}$ has a pre-image inside $H$.
    \end{itemize}
    Hence $H \cong gHg^{-1}$.

    \item By \myref{prop-order-of-conjugate-element-equals-order-of-element} we know that $|xyx^{-1}| = |y|$ for all $x, y \in G$. Substituting $x = g$, and $y = hg$ yields
    \[
        |xyx^{-1}| = |g(hg)g^{-1}| = |gh| \text{ and } |y| = |hg|
    \]
    so the result follows.

    \item Clearly $e \in \N{G}{S}$ since $eSe^{-1} = S$. Consider $x, y \in \N{G}{S}$, meaning that $xSx^{-1} = S$ and $ySy^{-1} = S$. Note that $y^{-1} \in \N{G}{S}$ since
    \begin{align*}
        y^{-1}S\left(y^{-1}\right)^{-1} &= y^{-1}Sy\\
        &= y^{-1}\left(ySy^{-1}\right)y & (\text{since } y \in \N{G}{S})\\
        &= (y^{-1}y)S(y^{-1}y)\\
        &= S.
    \end{align*}
    Therefore
    \begin{align*}
        \left(xy^{-1}\right)S\left(xy^{-1}\right)^{-1} &= \left(xy^{-1}\right)S\left(yx^{-1}\right)\\
        &= x\left(y^{-1}Sy\right)x^{-1}\\
        &= xSx^{-1} & (\text{since } y^{-1} \in \N{G}{S})\\
        &= S & (\text{since } x \in \N{G}{S})
    \end{align*}
    which means that $xy^{-1} \in \N{G}{S}$. Hence, by the subgroup test, we have $\N{G}{S} \leq G$.

    \item By the Second Sylow Theorem (\myref{thrm-sylow-2}), we know that $gHg^{-1} = K$. Since $H \cong gHg^{-1}$ by \myref{exercise-conjugate-subgroup-isomorphic-to-subgroup} thus $H \cong gHg^{-1} = K$ as required.

    \item We note $784 = 2^4 \times 7^2$, so $m = 16$, $p = 7$, and $k = 2$. By the Third Sylow Theorem (\myref{thrm-sylow-3}), we know that
    \begin{itemize}
        \item $n_7 = [G : \N{G}{P}] = \frac{|G|}{|\N{G}{P}|}$;
        \item $n_7 \mid 16$, which implies $n_7 \in \{1, 2, 4, 8, 16\}$; and
        \item $n_7 \equiv 1 \pmod 7$, which implies $n_7 \in \{1, 8, 15, 22, \dots\}$.
    \end{itemize}
    Hence $n_7 = 1$ or $n_7 = 8$. But since $P$ is not a normal subgroup of $G$, by \myref{corollary-sylow-subgroup-is-normal-if-it-is-unique}, $P$ cannot be the only Sylow 7-subgroup, meaning $n_7 \neq 1$. Hence $n_7 = 8$, so
    \[
        8 = n_7 = \frac{|G|}{|\N{G}{P}|} = \frac{784}{|\N{G}{P}|}
    \]
    which means that $|\N{G}{P}| = 98$.

    \item Note $130 = 2 \times 5 \times 13$. Consider the number of Sylow 13-subgroups, $n_{13}$. The Third Sylow Theorem (\myref{thrm-sylow-3}) tells us that
    \begin{itemize}
        \item $n_{13} \mid 2 \times 5 = 10$, so $n_{13} \in \{1, 2, 5, 10\}$, and
        \item $n_{13} \equiv 1 \pmod{13}$ so $n_{13} \in \{1, 14, 27, \dots\}$.
    \end{itemize}
    Hence $n_{13} = 1$. But by \myref{corollary-sylow-subgroup-is-normal-if-it-is-unique} this means that the only Sylow 13-subgroup is normal. Hence a group of order 130 is non-simple.
\end{questions}

\subsection*{Problems}
\begin{questions}
    \item Note $200 = 2^3 \times 5^2$. Note that for $p = 5$ we have $m = 8$ and the factors of 8 are 1, 2, 4, and 8. Furthermore by the Third Sylow Theorem (\myref{thrm-sylow-3}) we must have $n_5 \equiv 1 \pmod 5$. Hence $n_5 = 1$. By a corollary of the Second Sylow Theorem (\myref{corollary-sylow-subgroup-is-normal-if-it-is-unique}) this means that the only Sylow 5-subgroup is normal.

    \item Note $33 = 3 \times 11$,
    \begin{itemize}
        \item when $p = 3$ we have $m = 11$ and the factors of 11 are 1 and 11; and
        \item when $p = 11$ we have $m = 3$ and the factors of 3 are 1 and 3.
    \end{itemize}
    The Third Sylow Theorem (\myref{thrm-sylow-3}) tells us that $n_p \equiv 1 \pmod p$. Hence we must have $n_3 = n_{11} = 1$. A corollary of the Second Sylow Theorem (\myref{corollary-sylow-subgroup-is-normal-if-it-is-unique}) tells us that the only Sylow 3-subgroup and Sylow 11-subgroup are normal.

    \item For brevity let $q = 2^p - 1$, and we are given that $q$ is a prime. By the Third Sylow Theorem (\myref{thrm-sylow-3}), $n_q \mid 2^{p-1}$ and $n_q \equiv 1 \pmod p$. The factors of $2^{p-1}$ are $1, 2, 4, 8, \dots, 2^{p-1}$. We note $2^{p-1} < 2^p - 1 = q$ for any prime $p$ since
    \[
        2^{p-1} + 1 < 2^{p-1} + 2^{p-1} = 2(2^{p-1}) = 2^p
    \]
    which result immediately follows by subtracting 1 on both sides. Hence, the only possible value that satisfies both conditions is $n_q = 1$. By a corollary of the Second Sylow Theorem (\myref{corollary-sylow-subgroup-is-normal-if-it-is-unique}) this means that the only Sylow $q$-subgroup is normal, hence showing that a group with an even perfect number order is non-simple.

    \item \begin{partquestions}{\roman*}
        \item The divisors of $p$ are 1 and $p$ itself. By the Third Sylow Theorem (\myref{thrm-sylow-3}), $n_q$ divides $p$ and $n_q \equiv 1 \pmod q$. Since $p < q$ hence $p \not\equiv 1 \pmod q$ meaning that $n_q = 1$. By a corollary of the Second Sylow Theorem (\myref{corollary-sylow-subgroup-is-normal-if-it-is-unique}) the only Sylow $q$-subgroup is normal.

        \item The divisors of $q$ are 1 and $q$ itself. By the Third Sylow Theorem (\myref{thrm-sylow-3}), $n_p$ divides $q$ and $n_p \equiv 1 \pmod p$. Since $q \not\equiv 1 \pmod p$ by assumption, we must have $n_p = 1$.

        Recall that the order of an element in a group of order $pq$ must divide $pq$ (\myref{corollary-order-of-group-multiple-of-order-of-element}). Hence the possible orders of an element in such a group are 1, $p$, $q$, or $pq$.
        \begin{itemize}
            \item There is only one element of order 1, the identity.
            \item There are $p - 1$ elements of order $p$, all belonging in the single Sylow $p$-subgroup. Note that we subtract 1 because one element in the Sylow $p$-subgroup is the identity.
            \item There are $q - 1$ elements of order $q$, all in the single Sylow $q$-subgroup.
        \end{itemize}
        Hence, since the total number of elements in a group of order $pq$ is $pq$, the number of elements of order $pq$ is
        \begin{align*}
            pq - \left((p-1)+(q-1)+1\right) &= pq - (p+q - 1)\\
            &= pq - p - q + 1\\
            &> 2q - 2 - q + 1\\
            &= 2q - q - 1\\
            &= q - 1\\
            &> 0
        \end{align*}
        which means that there is at least one element of order $pq$. By \myref{thrm-cyclic-group-has-element-with-same-order} this means that such a group is cyclic.
    \end{partquestions}

    \item \begin{partquestions}{\roman*}
        \item Let $P$ be a Sylow $p$-subgroup of $N$. Lagrange's Theorem (\myref{thrm-lagrange}) tells us that $|G| = [G:N]|N|$. Since $p$ does not divide $[G:N]$ we must have $|N| = p^ka$ where $a$ divides $m$. Hence $|P| = p^k$ as $P$ is a Sylow $p$-subgroup of $N$. Since $P$ has order $p^k$ and $P \leq N \leq G$, thus $P$ is also a Sylow $p$-subgroup of $G$.
        \item Let $Q$ be a Sylow $p$-subgroup of $G$. The Second Sylow Theorem (\myref{thrm-sylow-2}) tells us there exist  a $g \in G$ such that $Q = gPg^{-1}$. Recall by definition of normality that $gNg^{-1} = N$ for any $g \in G$. Note also that $P \leq N$. Hence,
        \[
            Q = gPg^{-1} \leq gNg^{-1} = N
        \]
        which means that $Q$ is also a Sylow $p$-subgroup of $N$.
    \end{partquestions}

    \item We note $3325 = 5^2 \times 7 \times 19$. Let the group of order 3325 be $G$. We know that
    \begin{itemize}
        \item for $p = 5$ we have $m = 7 \times 19 = 133$ and so the possible divisors of $m$ are $\{1, 7, 19, 133\}$;
        \item for $p = 7$ we have $m = 5^2 \times 19 = 475$ and so the possible divisors of $m$ are $\{1, 5, 19, 25, 95, 475\}$; and
        \item for $p = 19$ we have $m = 5^2 \times 7 = 175$ and so the possible divisors of $m$ are $\{1, 5, 7, 25, 35, 175\}$.
    \end{itemize}
    The Third Sylow Theorem (\myref{thrm-sylow-3}) tells us that $n_p \equiv 1 \pmod p$. Thus $n_5 = n_7 = n_{19} = 1$. Let $P$, $Q$, and $R$ be the Sylow 5-subgroup, the Sylow 7-subgroup, and the Sylow 19-subgroup respectively. We note that $P$, $Q$, and $R$ are all normal subgroups of $G$ by \myref{corollary-sylow-subgroup-is-normal-if-it-is-unique}.

    Denote the group $QR$ by $H$. Since $Q$ and $R$ are of prime order, their intersection is the identity (\myref{problem-intersection-of-coprime-subgroups}). Furthermore, as $Q$ and $R$ are normal subgroups of $G$, thus they commute by \myref{problem-intersection-of-coprime-subgroups}. Therefore $H$ is the internal direct product of $Q$ and $R$, meaning $H \cong Q \times R$ by \myref{thrm-direct-product-equivalence}. Hence $|H| = |Q||R| = 7 \times 19 = 133$. Now because as $Q$ and $R$ are of prime order, thus $Q$ and $R$ are abelian and so is $H$. Hence $H$ is an abelian group of order 133.

    Now consider the group $PH$. Since 5 and 133 are coprime, thus $P \cap H = \{e\}$. In addition, since $P \lhd G$ thus $PH \leq G$ by Diamond Isomorphism Theorem (\myref{thrm-isomorphism-2}), statement 3. Also,
    \[
        |PH| = \frac{|P||H|}{|P \cap H|} = |P||H| = 5^2 \times 133 = 3325 = |G|
    \]
    which means that $G = PH$. Since $P \lhd G$, thus $ph = hp$ for any element $h \in H$, meaning elements in $P$ and $H$ commute. Hence, $G$ is the internal direct product of $P$ and $H$, meaning $G \cong P \times H$. As the external direct product of two abelian groups is also abelian (\myref{problem-external-direct-product-of-abelian-groups-is-abelian}) thus $G$ is abelian.

    \item Let $P$ be a Sylow $p$-subgroup of $G$. We note that $|G/P| = \frac{p^km}{p^k} = m$. Let $G$ act on the set of cosets $G/P$ by left multiplication, meaning $g\cdot (xP) = (gx)P$. We know by \myref{thrm-group-action-definition-equivalence} that this induces a homomorphism $\phi: G \to \Sn{m}$ where $\phi(g) = \sigma_g$ such that $\sigma_g(xP) = g\cdot (xP) = (gx)P$. By \myref{example-using-kernel-to-show-non-simple}, $\ker\phi = \bigcap_{x \in G}xPx^{-1}$.

    We note $\ker\phi \neq \{e\}$ since otherwise it would imply that $\phi$ is injective (\myref{exercise-trivial-kernel-means-injective}), which is impossible as that would mean $p^km = |G| \leq |\Sn{m}| = m!$ which is a contradiction. Also $\ker\phi \neq G$ as otherwise
    \[
        p^km = |G| = |\ker\phi| = \left|\bigcap_{x \in G} xPx^{-1}\right| \leq |xPx^{-1}| = |P| = p^k,
    \]
    which would mean $m = 1$, a contradiction. Hence $\ker\phi$ is a non-trivial proper subgroup of $G$. We note that $\ker\phi \lhd G$, so we have found a non-trivial proper normal subgroup of $G$, meaning that $G$ is non-simple.

    \item Let $G$ be a group of order 30. Note $30 = 2 \times 3 \times 5$, and consider $n_5$. The Third Sylow Theorem (\myref{thrm-sylow-3}) tells us that
    \begin{itemize}
        \item $6 \vert n_5$, so $n_5 \in \{1, 2, 3, 6\}$; and
        \item $n_5 \equiv 1 \pmod 5$, so $n_5 \in \{1, 6, 11, 16, \dots\}$.
    \end{itemize}
    Hence $n_5$ is 1 or 6. Seeking a contradiction, assume $n_5 = 6$, and let $P_5$ be a Sylow 5-subgroup.

    Since $|P_5| = 5$, which is prime, each non-identity element of $P_5$ is a generator. Hence, no two Sylow 5-subgroups can share any non-identity elements (otherwise they will be the same group), thereby meaning any two Sylow 5-subgroups intersect in the identity only. Thus there exists $6(5-1) = 24$ elements of order 5, meaning there must be 6 elements of order not equal to 5.

    Now consider $n_3$. Note by the Third Sylow Theorem again,
    \begin{itemize}
        \item $10 \vert n_3$, so $n_3 \in \{1, 2, 5, 10\}$; and
        \item $n_3 \equiv 1 \pmod 3$, so $n_3 \in \{1, 4, 7, 10, 13, \dots\}$.
    \end{itemize}
    Thus $n_3$ is 1 or 10. Now if $n_3 = 10$ then there must be $10(3-1) = 20$ elements of order 3, a contradiction to the fact there exists only 6 elements with order not 5. Hence $n_3 = 1$, meaning the only Sylow 3-subgroup (call it $P_3$) is normal in $G$.

    As $P_5 \leq G$ and $P_3 \lhd G$, by the Diamond Isomorphism Theorem (\myref{thrm-isomorphism-2}), statement 3, we have $P_5P_3 \leq G$. Note $P_5 \cap P_3 = \{e\}$ by \myref{problem-intersection-of-coprime-subgroups}. So \myref{exercise-order-of-subgroup-product} tells us
    \[
        |P_5P_3| = \frac{|P_5||P_3|}{|P_5 \cap P_3|} = \frac{5\times3}{1} = 15.
    \]
    One sees that $[G:P_5P_3] = \frac{30}{15} = 2$, so $P_5P_3 \lhd G$ by \myref{problem-subgroup-of-index-2}.

    Now \myref{problem-group-of-order-pq-has-normal-subgroup-of-order-q} tells us there exists a unique $H \lhd P_5P_3$ with $|H| = 5$. But since $P_5P_3 \lhd G$, \myref{problem-normal-subgroup-of-G-contains-all-sylow-p-subgroups} tells us that $P_5P_3$ contains all Sylow 5-subgroups of $G$, meaning $G$ has only 1 Sylow 5-subgroup, i.e. $n_5 = 1$, a contradiction to our assumption that $n_5 = 6$.

    Hence $n_5 = 1$. Therefore the unique Sylow 5-subgroup is a normal subgroup of $G$ by \myref{corollary-sylow-subgroup-is-normal-if-it-is-unique}.

    \item We prove that $G$ has a normal subgroup of order $p$, $q$, or $r$. By \myref{corollary-sylow-subgroup-is-normal-if-it-is-unique}, subgroups of order $p$, $q$, or $r$ are normal if they are unique. By way of contradiction, assume that they are not unique, meaning $n_p, n_q, n_r > 1$.

    By the Third Sylow Theorem (\myref{thrm-sylow-3}), $n_r \equiv 1 \pmod r$ and $n_r \mid pq$. The divisors of $pq$ are 1, $p$, $q$, and $pq$. We note that since both $p$ and $q$ are less than $r$, thus $p \not\equiv 1 \pmod r$ and $q \not\equiv 1 \pmod r$. The only possibility that is left is $n_r = pq$ as we assume $n_r \neq 1$. Similarly, $n_q \equiv 1 \pmod q$ and $n_q \mid pr$. The divisors of $pr$ are 1, $p$, $r$, and $pr$. Since $p < q$ thus $p \not\equiv 1 \pmod q$. Hence $n_q \geq r$ as we assume $n_q \neq 1$. Similarly, $n_p \geq q$.

    We now consider the number of elements with order $p$, $q$, and $r$.
    \begin{itemize}
        \item $\boxed{p}$ With $n_p \geq q$, there are at least $q(p-1)$ elements of order $p$. We minus 1 because one of the elements in a Sylow $p$-subgroup is the identity with order 1.
        \item $\boxed{q}$ With $n_q \geq r$, there are at least $r(q-1)$ elements of order $q$.
        \item $\boxed{r}$ We know $n_r = pq$ so there are exactly $pq(r-1)$ elements of order $r$.
    \end{itemize}
    Since the total number of elements, $pqr$, must be at least the sum of the numbers of these elements, thus
    \begin{align*}
        pqr &\geq q(p-1) + r(q-1) + pq(r-1)\\
        &= pq - q + qr - r + pqr - pq\\
        &= pqr + qr - q - r
    \end{align*}
    which means $qr - q - r \leq 0$. Rearranging, we see
    \[
        q \leq \frac{r}{r-1} = 1 + \frac{1}{r-1}.
    \]
    Since $p < q$ and they are both primes, we must have $q \geq 3$. Hence one sees
    \[
        3 \leq q \leq 1 + \frac{1}{r-1} \leq 2
    \]
    which is a clear contradiction. Hence, at least one of $n_p$, $n_q$, or $n_r$ is 1, meaning that there exists a non-trivial proper normal subgroup in $G$ by \myref{corollary-sylow-subgroup-is-normal-if-it-is-unique}. Therefore $G$ is non-simple.
\end{questions}
