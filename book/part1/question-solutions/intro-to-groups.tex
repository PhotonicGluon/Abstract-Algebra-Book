\section{Introduction to Groups}
\subsection*{Exercises}
\begin{questions}
    \item There are 6! = 720 possible permutations of 6 points, so there are 720 symmetries in the group given. That is, the order of the symmetric group of degree 6 is 720.
\end{questions}

\subsection*{Problems}
\begin{questions}
    \item \begin{partquestions}{\alph*}
        \item This is a group. Addition is clearly closed and associative. The identity is 0. The inverse of any element $x$ is $-x$.
        \item This is not a group. Inverses do not exist. For example, the element $2$ does not have an inverse under multiplication.
        \item This is a group. Multiplication is clearly closed and associative. The identity is $1$. The inverse of any element $x$ is $\frac1x$.
        \item This is a group. Multiplication is clearly closed and associative. The identity is $0$ and the inverse is $0$.
        \item This is not a group. Addition is not closed: $1 + 1 = 2$ which is not in the group.
        \item This is a group. Multiplication is clearly closed and associative. The identity is $1$ and the inverse is $1$.
    \end{partquestions}

    \item We show that the trivial group is indeed a group by showing that the four group axioms hold.
    \begin{itemize}
        \item \textbf{Closure}: The only element in the underlying set is $e$, and $e \ast e = e \in \{e\}$. Thus the structure is closed under $\ast$.
        \item \textbf{Associativity}: Clearly $e \ast (e \ast e) = e \ast e = e$ and $(e \ast e) \ast e = e \ast e = e$ which means that $\ast$ is associative.
        \item \textbf{Identity}: The identity is clearly $e$.
        \item \textbf{Inverse}: The only element is $e$, and because $e \ast e = e$ hence $e$ is its own inverse.
    \end{itemize}
    Therefore $(\{e,\}, \ast)$ is a group.
\end{questions}
