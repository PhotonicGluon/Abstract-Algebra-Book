\section{Abelian Groups}
\subsection*{Exercises}
\begin{questions}
    \item Note $100 = 2^2 \times 5^2$, so the only 4 possibilities are
    \begin{itemize}
        \item $\Cn{2} \times \Cn{2} \times \Cn{5} \times \Cn{5}$;
        \item $\Cn{2} \times \Cn{2} \times \Cn{25}$;
        \item $\Cn{4} \times \Cn{5} \times \Cn{5}$; and
        \item $\Cn{4} \times \Cn{25}$,
    \end{itemize}
    by the Fundamental Theorem of Finite Abelian Groups (\myref{thrm-fundamental-theorem-of-finite-abelian-groups}).

    \item Let $G$ be a finite abelian group of prime-power order; in particular let the order of $G$ be $p^n$ where $p$ is prime and $n$ is a non-negative integer. We use strong induction on $n$.

    When $n = 0$ then $G$ is just the trivial group, which is itself cyclic.

    Assume that the lemma holds for all finite abelian groups of order $p^r$ where $0 \leq r \leq k$ for some positive integer $k$. We show that a finite abelian $p$-group of order $p^{k+1}$ is also an internal direct product of cyclic groups.

    Let $G$ be a finite abelian $p$-group of order $p^{k+1}$. Let $g$ be an element of maximal order in $G$. If $\langle g \rangle = G$ then we are done; otherwise \myref{lemma-fundamental-theorem-of-finite-abelian-groups-2} tells us that $G \cong \langle g \rangle \times H$ for some subgroup $H$ in $G$. Note that $|H| < |G|$ as otherwise $g = e$ which clearly does not have maximal order in $G$. Therefore we may use the induction hypothesis on $H$ to write it as an internal direct product of cyclic groups; thus $G$ itself is an internal direct product of cyclic groups.

    Therefore the lemma is proven by mathematical induction.

    \item \begin{partquestions}{\roman*}
        \item Note that $e \in G^n$ since $e = e^n$.

        Now suppose $a, b \in G^n$, which means $a = x^n$ and $b = y^n$ for some $x, y \in G$. Then note
        \begin{align*}
            ab^{-1} &= (x^n)(y^n)^{-1}\\
            &= x^n(y^{-1})^n\\
            &= (xy^{-1})^n & (\text{rewriting possible as }G \text{ is abelian})\\
            &\in G^n.
        \end{align*}

        Thus $G^n \leq G$ by subgroup test.

        \item Cauchy's theorem (\myref{thrm-cauchy}) tells us that an element of order $p$ must exist within $G$. Let this element be $a$. If $G = G^p$ then $\phi$ must be at least injective. But note $\phi(a) = a^p = e = \phi(e)$ and $a \neq e$, so $\phi$ is not injective and hence $G \neq G^p$. Therefore $G^p < G$.
    \end{partquestions}
\end{questions}

\subsection*{Problems}
\begin{questions}
    \item $\Cn{p_1}\times\Cn{p_2}\times\cdots\times\Cn{p_n}$ is the only distinct abelian group.

    \item The smallest $n$ is 4, since anything smaller is just a cyclic group of prime order (or the trivial group). The two groups required are $\Cn{4}$ and $\Cn{2} \times \Cn{2}$ (which is actually isomorphic to $D_2$, the dihedral group of order 4).

    \item First note that we can write 4 as a sum in 5 distinct ways, namely
    \begin{itemize}
        \item 4;
        \item 3 + 1;
        \item 2 + 2;
        \item 2 + 1 + 1; and
        \item 1 + 1 + 1 + 1.
    \end{itemize}
    Consequently the distinct isomorphism classes for an abelian group of order $p^4$ are
    \begin{itemize}
        \item $\Cn{p^4}$;
        \item $\Cn{p^3}\times\Cn{p}$;
        \item $\Cn{p^2}\times\Cn{p^2}$;
        \item $\Cn{p^2}\times\Cn{p}\times\Cn{p}$; and
        \item $\Cn{p}\times\Cn{p}\times\Cn{p}\times\Cn{p}$.
    \end{itemize}

    \item Note $a \neq b$, since otherwise $a^2 = b^2$. As $|a| = 4$ and $|b| = 4$, we know that $G$ contains two distinct subgroups of order 4, say $A$ and $B$. So ``$\Cn{4} \times \Cn{4}$'' must be a part of the isomorphism class of $G$. But $\Cn{4} \times \Cn{4}$ has an order of 16. Therefore the isomorphism class of $G$ must be $\Cn{4} \times \Cn{4}$.

    \item Since $x^4 = e$, therefore there cannot be an element of order 8 or order 16 inside the group. So the maximum order of a `component' of the isomorphism class is 4. Consequently, the only 3 choices are
    \begin{itemize}
        \item $\Cn{4}\times\Cn{4}$;
        \item $\Cn{4}\times\Cn{2}\times\Cn{2}$; and
        \item $\Cn{2}\times\Cn{2}\times\Cn{2}\times\Cn{2}$.
    \end{itemize}
\end{questions}
