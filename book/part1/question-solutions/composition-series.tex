\section{Composition Series}
\subsection*{Exercises}
\begin{questions}
    \item \begin{partquestions}{\roman*}
        \item One sees clearly that $\{0, 2\}$ is the only non-trivial proper normal subgroup of $G$, so the subnormal series of length 2 is $1 \lhd \{0, 2\} \lhd G$.
        \item There are 2 factor groups of the above subnormal series. The first is $\{0, 2\} / 1 \cong \Z_2$ and the second is
        \begin{align*}
            G / \{0, 2\} &= \{g \oplus_4 \{0, 2\} \vert g \in G\}\\
            &= \{\{0, 2\}, \{1, 3\}, \{2, 0\}, \{3, 1\}\}\\
            &= \{\{0, 2\}, \{1, 3\}\}\\
            &= \langle \{1, 3\} \rangle\\
            &\cong \Z_2.
        \end{align*}
        \item Since $1 \lhd G$ and $\{0, 2\} \lhd G$ thus the subnormal series in \textbf{(i)} is also a normal series of $G$.
    \end{partquestions}

    \item By Lagrange's theorem (\myref{thrm-lagrange}) we know that the order of a subgroup must divide the order of the group. Furthermore $\Z_{120}$ is abelian, so any subgroup of it is normal. Now the subgroup $N = \{0, 2, 4, \dots, 118\}$ has 60 elements which is the maximum possible guaranteed by Lagrange. Hence $N$ is the maximal normal subgroup of $\Z_{120}$, which has order 60.

    \item $\Cn{6}$ has these two composition series up to isomorphism
    \begin{align*}
        &1 \lhd \Cn{2} \lhd \Cn{6} \text{ and }\\
        &1 \lhd \Cn{3} \lhd \Cn{6}.
    \end{align*}
    In both cases, their composition length is 2. Their respective composition factors are:
    \begin{itemize}
        \item $\Cn{2} / 1 \cong \Cn{2}$ and $\Cn{6} / \Cn{2} \cong \Cn{3}$ by \myref{exercise-Zmn-mod-Zn-cong-Zn}; and
        \item $\Cn{3} / 1 \cong \Cn{3}$ and $\Cn{6} / \Cn{3} \cong \Cn{2}$ by \myref{exercise-Zmn-mod-Zn-cong-Zn},
    \end{itemize}
    up to isomorphism.

    \item Let the group in question be $G$. We know by Cauchy's theorem (\myref{thrm-cauchy}) and \myref{exercise-group-of-order-multiple-of-prime-has-subgroup-of-prime-order}, and by writing $p^2$ as $p \times p$, that $G$ has a subgroup of order $p$ (call this $H$).

    Lagrange's theorem (\myref{thrm-lagrange}) tells us that the possible orders of the subgroups of $G$ are 1, $p$, and $p^2$. These subgroups are $\{e\}$, $H$, and $G$ respectively. Furthermore, by \myref{problem-group-of-order-prime-squared-is-abelian}, $G$ must be abelian, thereby its subgroups are all normal (\myref{prop-subgroup-of-abelian-group-is-normal}). Finally, a corollary of Lagrange's theorem (\myref{corollary-group-with-prime-order-subgroups}) says that the only subgroups of $H$ are the trivial group and the group itself. Hence, $G$ has only one composition series, namely $1 \lhd H \lhd G$.
\end{questions}

\subsection*{Problems}
\begin{questions}
    \item \begin{partquestions}{\roman*}
        \item We note $\mathrm{V}$ has order 4. By writing 4 as $2 \times 2$ we know that $\mathrm{V}$ has a subgroup of order 2 (which is cyclic) by Cauchy's Theorem (\myref{thrm-cauchy}). Now $\mathrm{V}$ is abelian (\myref{problem-group-of-order-prime-squared-is-abelian}) which means that the subgroup of order 2 is normal in $\mathrm{V}$ (\myref{prop-subgroup-of-abelian-group-is-normal}). Finally, the only possible order for a non-trivial proper subgroup of $\mathrm{V}$ is 2 by Lagrange's theorem (\myref{thrm-lagrange}). Hence, the only composition series for $\mathrm{V}$ is $1 \lhd \Cn{2} \lhd \mathrm{V}$ up to isomorphism.\newline
        (Note that this analysis applies for \textit{any} group of order 4.)

        \item Recall that $\mathrm{Q} = \langle \alpha, \beta \vert \alpha^4 = e, \alpha^2 = \beta^2, \text{ and } \beta\alpha = \alpha^3\beta \rangle$. From the solution of \myref{exercise-normal-subgroups-of-quarternion-group}, the maximal subgroups of $\mathrm{Q}$ are $G_1 = \langle \alpha \rangle$, $G_2 = \langle \beta \rangle$, and $G_3 = \langle \alpha\beta \rangle$ (by setting $\alpha = i$ and $\beta = j$). We note the following.
        \begin{itemize}
            \item $G_1 = \{e, \alpha, \alpha^2, \alpha^3\} \cong \Cn{4}$.
            \item $G_2 = \{e, \beta, \beta^2, \beta^3\} = \{e, \beta, \alpha^2, \alpha^2\beta\} \cong \mathrm{V}$ where $a = \alpha^2$ and $b = \beta$.
            \item $G_3 = \{e, \alpha\beta, (\alpha\beta)^2, (\alpha\beta)^3\} = \{e, \alpha\beta, \alpha^2, \alpha^3\beta\} \cong \mathrm{V}$ with $a = \alpha\beta$ and $b = \alpha^2$.
        \end{itemize}
        Also, note that $\Cn{2} \cong \langle \alpha^2 \rangle \lhd G_1$, $\Cn{2} \cong \langle \beta^2 \rangle \lhd G_2$, $\Cn{2} \cong \langle (\alpha\beta)^2 \rangle \lhd G_3$. Hence, the two series up to isomorphism are
        \begin{align*}
            1 \lhd \Cn{2} \lhd \Cn{4} \lhd \mathrm{Q} & \text{ and }\\
            1 \lhd \Cn{2} \lhd \mathrm{V} \lhd \mathrm{Q}
        \end{align*}

        \item By the Jordan-H\"older theorem (\myref{thrm-jordan-holder}), the composition factors are isomorphic to each other. We note
        \begin{itemize}
            \item $\Cn{2} / 1 \cong \Cn{2}$;
            \item $\Cn{4} / \Cn{2} \cong \Cn{2}$ by \myref{exercise-Zmn-mod-Zn-cong-Zn}; and
            \item $\mathrm{V} / \Cn{2} \cong (\Cn{2})^2 / \Cn{2} \cong \Cn{2}$ by \myref{problem-cartesian-product-of-group-by-group-isomorphic-to-group}.
        \end{itemize}
        The only unaccounted set of factors is $\mathrm{Q}/\mathrm{V}$ and $\mathrm{Q}/\Cn{4}$. So, either $\mathrm{Q}/\mathrm{V} \cong \Cn{2}$ and $\mathrm{Q}/\Cn{4} \cong \Cn{2}$, or $\mathrm{Q}/\mathrm{V} \cong \mathrm{Q}/\Cn{4}$. Hence $\mathrm{Q}/H \cong \mathrm{Q}/K$.
    \end{partquestions}

    \item We know that $\An{4} \lhd \Sn{4}$ by \myref{prop-An-normal-subgroup-of-Sn}. Note $\An{4}$ is a maximal normal subgroup since $|\An{4}| = \frac{4!}2 = 12$ by \myref{prop-order-of-An}, and a subgroup's order must divide the order of the group by Lagrange's theorem (\myref{thrm-lagrange}).

    Now applying that theorem on $\An{4}$, we see that the possible orders of a subgroup of $\An{4}$ are 6, 4, 3, 2, and 1. We claim that a subgroup of order 6 does not exist. Note that $\An{4}$ contains
    \begin{itemize}
        \item 1 element of order 1;
        \item 3 elements of order 2; and
        \item 8 elements of order 3.
    \end{itemize}
    If a subgroup of order 6 exists (say, $H$), then its index would be $\frac{12}{6} = 2$ (Lagrange), meaning $H$ contains all odd order elements (\myref{problem-subgroup-of-index-2}). However, there are $1 + 8 = 9$ odd order elements, meaning that $H$ has an order of at least 9, a contradiction. Hence a subgroup of $\An{4}$ of order 6 is impossible.

    Now we note that a subgroup of order $4 = 2^2$ exists by a corollary of the First Sylow Theorem (\myref{corollary-sylow-p-subgroup-exists}) as it is a Sylow 2-subgroup. The Third Sylow Theorem (\myref{thrm-sylow-3}) tells us how many Sylow 2-subgroups there are:
    \begin{itemize}
        \item $n_2 \vert 3$, so $n_2$ is 1 or 3; and
        \item $n_2 \equiv 1 \pmod2$, so $n_2 \in \{1, 3, 5, \dots\}$.
    \end{itemize}
    Hence $n_2$ is either 1 or 3. Now if $n_2 = 3$, then the number of elements of order of 1, 2, or 4 is
    \[
        3 \times (4 - 1) + 1 = 10
    \]
    (where the 3 is $n_2$, the $4-1$ is the number of non-identity elements in each Sylow 2-subgroup, and the $+1$ is to add the identity element). However, as noted above, there are only 4 elements of order 1, 2, or 4, a contradiction. Hence $n_2 = 1$, meaning the Sylow 2-subgroup (which is a subgroup of order 4) is normal (\myref{corollary-sylow-subgroup-is-normal-if-it-is-unique}). Therefore the subgroup of order 4 is the maximal normal subgroup of $\An{4}$.

    We note that the subgroup of order 4 of $\An{4}$ is not $\Cn{4}$ (as this would imply that $\An{4}$ has an element of order 4, which it does not). Hence, from \myref{problem-smallest-nonabelian-group}, the subgroup of order 4 must be isomorphic to the Klein-4 group, $\mathrm{V}$.

    Note that a group of order 4 has a subgroup of order 2 by Cauchy's theorem (\myref{thrm-cauchy}). Clearly such a subgroup is cyclic (since 2 is prime), and has index $\frac42 = 2$, meaning that it is normal in the group of order 4. Furthermore the trivial group is always a subgroup of any group.

    Hence, the composition series for $\Sn{4}$, up to isomorphism, is
    \[
        1 \lhd \Cn{2} \lhd \mathrm{V} \lhd \An{4} \lhd \Sn{4}.
    \]
    \begin{remark}
        We list the actual subgroups that are isomorphic to the above terms in the composition series here.
        \begin{itemize}
            \item $\Cn{2}$: $\{e, \begin{pmatrix}1&2\end{pmatrix}\begin{pmatrix}3&4\end{pmatrix}\}$
            \item V: $\{e, \begin{pmatrix}1&2\end{pmatrix}\begin{pmatrix}3&4\end{pmatrix}, \begin{pmatrix}1&3\end{pmatrix}\begin{pmatrix}2&4\end{pmatrix}, \begin{pmatrix}1&4\end{pmatrix}\begin{pmatrix}2&3\end{pmatrix}\}$
            \item $\An{4}$ is an actual subgroup of $\Sn{4}$
        \end{itemize}
    \end{remark}
\end{questions}
