\chapter{Sylow Theorems}
The Sylow theorems are a collection of theorems named after the Norwegian mathematician Peter Ludwig Sylow. They give detailed information about the number of subgroups of fixed order that a given finite group contains. The Sylow theorems form a fundamental part of finite group theory and have very important applications in the classification of finite simple groups.

\section{The First Sylow Theorem}
The First Sylow Theorem is a follow-up to Cauchy's theorem.

\begin{theorem}[Sylow I]\label{thrm-sylow-1}\index{Sylow Theorem!First}
    Let $G$ be a finite group. Let $p$ be a prime and $n$ be a non-negative integer. If $p^n$ divides $|G|$ then there exists a subgroup of order $p^n$.
\end{theorem}
\begin{proof}[Proof (see {\cite[Theorem 15.4]{judson_beezer_2022}})]
    We use strong induction on the order of $|G|$.

    For the base case, if $|G| = 1$ and $p^n$ divides $|G|$ then it must be the case that $n = 0$. So a subgroup of order $p^n = p^0 = 1$ exists (which is the trivial subgroup). Also, if $|G| = p$ then we are done -- the group itself is a subgroup of order $p$.

    Now assume that the theorem holds for all groups with order less than $k$, where $k > p$ and $p$ divides $k$. We show that if $p^r$ divides $|G| = k$ then $G$ contains a subgroup of order $p^r$.

    Consider the class equation (\myref{thrm-class-equation})
    \[
        |G| = |\CenterGrp{G}| + \sum_{i=1}^l [G:\Centralizer{G}{g_i}]
    \]
    where $g_1, g_2, \dots, g_l$ are representatives of the $l$ distinct conjugacy classes with more than one element. We consider two cases.

    The first case is if $p$ does not divide $[G:\Centralizer{G}{g_i}]$ for some $i$. Since $|G| = [G:\Centralizer{G}{g_i}]|\Centralizer{G}{g_i}|$ by Lagrange's theorem (\myref{thrm-lagrange}) and $p$ divides $|G|$, thus it must be the case that $p$ divides $|\Centralizer{G}{g_i}|$. In fact since $p^r$ divides $|G|$ thus $p^r$ also divides $|\Centralizer{G}{g_i}|$. Using the induction hypothesis on $\Centralizer{G}{g_i}$ results in $\Centralizer{G}{g_i}$ having a subgroup of order $p^r$; as $\Centralizer{G}{g_i} \leq G$ thus $G$ itself has a subgroup of order $p^r$.

    The second case is if $p$ divides every $[G:\Centralizer{G}{g_i}]$. Since $p$ divides $G$, the class equation tells us that $p$ must divide $|\CenterGrp{G}|$ too. By Cauchy's theorem (\myref{thrm-cauchy}) we hence know that $\CenterGrp{G}$ has an element of order $p$, say $g$. Let $N = \langle g \rangle$. Clearly $N \lhd \CenterGrp{G}$ since $\CenterGrp{G}$ is abelian (\myref{prop-subgroup-of-abelian-group-is-normal}); therefore $N \lhd G$ since every element of $\CenterGrp{G}$ commutes with every element of $G$, resulting in $xN = Nx$ for all $x \in G$. Now consider $G/N$, which is a group since $N \lhd G$. Note $|G/N| = \frac{|G|}{|N|} = \frac{|G|}{p}$; since $p^r$ divides $|G|$ thus $p^{r-1}$ divides $\frac{|G|}{p} = |G/N|$. The induction hypothesis therefore gives us a subgroup $H/N$ of $G/N$ with order $p^{r-1}$ (note the form of a subgroup of $G/N$ is of the form $H/N$ as proven in \myref{problem-subgroup-of-quotient-group-is-quotient-group}). In particular one sees $|H/N| = \frac{|H|}{|N|} = \frac{|H|}{p} = p^{r-1}$, thereby meaning that $|H| = p^r$, showing that $G$ has a subgroup of order $p^r$.
\end{proof}

We note a special case of the First Sylow Theorem. Recall that a group with order $p^k$ for some $k \geq 0$ is called a $p$-group.

\begin{definition}\label{definition-sylow-p-subgroup}
    Let $G$ be a finite group. Write the order of the group $G$ as $p^k m$ where $p$ is prime, $k \geq 0$, and $p \nmid m$. Then a subgroup $H$ with order $p^k$ is called a \term{Sylow $p$-subgroup}\index{Sylow $p$-subgroup} of $G$.

    The set of all Sylow $p$-subgroups of the group $G$ for a given prime $p$ is denoted by $\Syl{p}{G}$.
\end{definition}

\begin{corollary}\label{corollary-sylow-p-subgroup-exists}
    Let $G$ be a finite group with order $p^k m$ where $p$ is prime, $k \geq 0$, and $p \nmid m$. Then there exists a Sylow $p$-subgroup.
\end{corollary}
\begin{remark}
    This corollary is the result most often cited as the First Sylow Theorem.
\end{remark}
\begin{proof}
    By \myref{thrm-sylow-1} any group of order $p^km$ has a subgroup of $p^k$. This subgroup is exactly a Sylow $p$-subgroup of $G$ by definition.
\end{proof}

\begin{exercise}
    Find a Sylow 2-subgroup of $\Z_{12}$.
\end{exercise}

\begin{exercise}
    Find all primes $p$ such that $\Syl{p}{\Sn{5}}$ is non-empty.
\end{exercise}

\section{The Conjugate Subgroup}
Before we look at the next Sylow theorem, we need to introduce two more things. The first of which is the conjugate subgroup.

\begin{definition}
    Let $G$ be a group, $H \leq G$, and $g \in G$. Then the \term{conjugate subgroup of $H$ by $g$}\index{conjugate subgroup} is
    \[
        gHg^{-1} = \{ghg^{-1} \vert h \in H\}
    \]
    under the group operation of $G$.
\end{definition}
We proved that $gHg^{-1}$ is a subgroup of $G$ in \myref{exercise-conjugate-subgroup}.

\begin{exercise}\label{exercise-conjugate-subgroup-isomorphic-to-subgroup}
    Let $G$ be a group and $H \leq G$. Prove that $gHg^{-1} \cong H$ for any $g \in G$.
\end{exercise}

We prove some results regarding the conjugate subgroup here.
\begin{proposition}\label{prop-power-of-conjugate-equals-conjugate-of-power}
    Let $G$ be a group, and let $a, x \in G$. Then $(axa^{-1})^n = ax^na^{-1}$ for all integers $n$.
\end{proposition}
\begin{proof}
    Trivially, we have $\left(axa^{-1}\right)^0 = e = x^0 = ax^0a^{-1}$ for the case when $n = 0$.

    We consider a proof by induction for positive integers $n$ and then prove the case when $n$ is negative.

    When $n = 1$, $\left(axa^{-1}\right)^1 = axa^{-1}$ is given. Now assume $n = k$ holds true for some positive integer $k$, meaning that $\left(axa^{-1}\right)^k = ax^ka^{-1}$. Then
    \begin{align*}
        \left(axa^{-1}\right)^{k+1} &= \left(axa^{-1}\right)\left(axa^{-1}\right)^k\\
        &= \left(axa^{-1}\right)\left(ax^ka^{-1}\right) & (\text{by induction hypothesis})\\
        &= axa^{-1}ax^ka^{-1}\\
        &= axx^ka^{-1}\\
        &= ax^{k+1}a^{-1}
    \end{align*}
    which completes the induction for positive integers $n$.

    Now suppose $n$ is a non-negative integer. Then
    \begin{align*}
        \left(axa^{-1}\right)^{-n} &= \left(\left(axa^{-1}\right)^n\right)^{-1}\\
        &= \left(ax^na^{-1}\right)^{-1} & (\text{by above result})\\
        &= a\left(x^n\right)^{-1}a^{-1}\\
        &= ax^{-n}a^{-1}
    \end{align*}
    which completes the proof for all integers.
\end{proof}

\begin{proposition}\label{prop-order-of-conjugate-element-equals-order-of-element}
    Let $G$ be a group. Then for all $g, x \in G$ we have $|gxg^{-1}| = |x|$.
\end{proposition}
\begin{proof}
    The proposition trivially holds true for $g = e$ so we can safely assume $g \neq e$.

    Suppose $|x| = n$, meaning $x^n = e$ and $x^k \neq e$ for all $1 \leq k < n$. Note we have
    \begin{align*}
        \left(gxg^{-1}\right)^n  &= gx^ng^{-1} & (\myref{prop-power-of-conjugate-equals-conjugate-of-power})\\
        &= geg^{-1} & (\text{since } |x| = n)\\
        &= e
    \end{align*}
    which means that $|gxg^{-1}| \leq n = |x|$.

    Now let $k < n$. We will show that if $x^k \neq e$ then $(gxg^{-1})^k \neq e$ using contrapositive proof. Suppose $(gxg^{-1})^k = e$. Thus $gx^kg^{-1} = e$ by \myref{prop-power-of-conjugate-equals-conjugate-of-power}. This implies that $gx^k = g$ which quickly means $x^k = e$. Therefore if $x^k \neq e$ then $(gxg^{-1})^k \neq e$, which shows that $|x| \leq |gxg^{-1}|$.

    As $|gxg^{-1}| \leq |x|$ and $|x| \leq |gxg^{-1}|$, thus $|gxg^{-1}| = |x|$.
\end{proof}

\begin{exercise}
    Let $G$ be a group and $g, h \in G$. Prove that $|gh| = |hg|$.
\end{exercise}



We note an important result with regards to the conjugate subgroup.

\begin{theorem}\label{thrm-unique-subgroup-of-given-order-is-normal}
    Let $G$ be a group. Suppose $H$ is the only subgroup of $G$ with a given order. Then $H \unlhd G$.
\end{theorem}
\begin{proof}
    Suppose $g \in G$. By \myref{exercise-conjugate-subgroup-isomorphic-to-subgroup} we know $gHg^{-1} \cong H$ which means that $|gHg^{-1}| = |H|$. Furthermore, by \myref{exercise-conjugate-subgroup} we know that $gHg^{-1} \leq G$. Since $H$ is the only subgroup of that order (by assumption), we conclude that $gHg^{-1} = H$, which quickly means that $H \unlhd G$ by definition of a normal subgroup.
\end{proof}

\section{The Normalizer}
We now look at the definition of the normalizer.
\begin{definition}
    Let $G$ be a group and $S$ be a non-empty subset of $G$. The \term{normalizer of $S$ in $G$}\index{normalizer} is
    \[
        \N{G}{S} = \{g \in G \vert gS = Sg \}.
    \]
    Equivalently, $\N{G}{S} = \{g \in G \vert gSg^{-1} = S \}$.
\end{definition}
\begin{exercise}\label{exercise-normalizer-is-subgroup-of-main-group}
    Let $G$ be a group and $S$ be a non-empty subset of $G$. Prove that $\N{G}{S}$ is a subgroup of $G$.
\end{exercise}

We prove some properties of the normalizer here.
\begin{proposition}\label{prop-subgroup-is-a-normal-subgroup-of-normalizer}
    Let $G$ be a group and $H \leq G$. Then $H \unlhd \N{G}{H}$.
\end{proposition}
\begin{proof}
    We first prove $H \leq \N{G}{H}$ and then prove normality.

    We know that both $H$ and $\N{G}{H}$ are subgroups of $G$, so both are groups. We just need to check that $H \subseteq \N{G}{H}$ to prove that $H \leq \N{G}{H}$.

    Consider any $h \in H$. We note that $hH = H$ and $Hh^{-1} = H$. Thus, if $h \in H$, then
    \[
        hHh^{-1} = h(Hh^{-1}) = hH = H
    \]
    which means that $h \in \N{G}{H}$ by definition of the normalizer. Hence any element in $H$ is also an element of $\N{G}{H}$, meaning $H \subseteq \N{G}{H}$. It follows then that $H \leq \N{G}{H}$ since $H$ is a group.

    Now we prove normality. Consider any $n \in \N{G}{H}$, which means that $nHn^{-1} = H$. This quickly implies that $H \unlhd \N{G}{H}$.
\end{proof}
\begin{remark}
    Combining the results from both \myref{exercise-normalizer-is-subgroup-of-main-group} and \myref{prop-subgroup-is-a-normal-subgroup-of-normalizer} yields $H \unlhd \N{G}{H} \leq G$.
\end{remark}

\begin{proposition}\label{prop-normalizer-of-subgroup-is-largest-subgroup-containing-that-subgroup-as-a-normal-subgroup}
    Let $G$ be a group, and $H \leq G$. Then $\N{G}{H}$ is the largest subgroup of $G$ containing $H$ as a normal subgroup.
\end{proposition}
\begin{remark}
    What we mean by ``largest'' here is that if there was another subgroup of $G$, say $K$, such that $H \unlhd K$, then $K \subseteq \N{G}{H}$.
\end{remark}
\begin{proof}
    By \myref{prop-subgroup-is-a-normal-subgroup-of-normalizer} we know that $H \unlhd \N{G}{H}$. We just need to prove that any subgroup of $G$ in which $H$ is normal inside it must be a subset of $\N{G}{H}$.

    Consider any subgroup $N \leq G$ such that $H \unlhd N \leq G$. Then for any $n \in N$ we have $nHn^{-1} = H$ by definition of normality, which immediately means that $n \in \N{G}{H}$ by definition of the normalizer of $H$ in $G$. Hence any element in $N$ also belongs in $\N{G}{H}$, meaning $N \subseteq \N{G}{H}$.

    This completes the proof that $\N{G}{H}$ is the largest subgroup of $G$ that contains $H$ as a normal subgroup.
\end{proof}

\begin{proposition}\label{prop-normalizer-of-sylow-p-subgroup}
    Let $G$ be a finite group. Let $P$ be a Sylow $p$-subgroup of $G$, and $Q$ be a $p$-subgroup of $\N{G}{P}$. Then $Q \subseteq P$. In particular, if $Q$ is a Sylow $p$-subgroup of $\N{G}{P}$, then $P = Q$.
\end{proposition}
\begin{proof}[Proof (see {\cite[Proposition 11.9]{humphreys_1996}})]
    For brevity, write the order of $G$ as $xp^n$ where $p \nmid x$, $|P| = p^n$, and $|Q| = p^m$.

    By the Second Isomorphism Theorem (\myref{thrm-isomorphism-2}), statement 3, we see that $PQ \leq \N{G}{P}$ since $P \unlhd \N{G}{P}$ by \myref{prop-subgroup-is-a-normal-subgroup-of-normalizer}. As $\N{G}{P} \leq G$ thus $PQ \leq G$. In addition, by \myref{exercise-order-of-subgroup-product},
    \[
        |PQ| = \frac{|P||Q|}{|P \cap Q|} = p^{n+m-s}
    \]
    where we let $|P \cap Q| = p^s$. By Lagrange's theorem (\myref{thrm-lagrange}) we know that $|G| = a|PQ|$ for some integer $a$, so $p^n = ap^{n+m-s}$ which implies $1 = ap^{m-s}$. Thus $p^{m-s} = \frac 1a \leq 1$ which means $m \leq s$.

    We note by \myref{problem-intersection-of-subgroups} that $P \cap Q \leq Q$. Therefore $|Q| = b|P\cap Q|$ for some positive integer $b$ by Lagrange's theorem. Thus $p^m = bp^s$ which implies $p^{m-s} = b \geq 1$ so $m \geq s$.

    As $m \leq s$ and $m \geq s$, thus $m = s$ which means $|P \cap Q| = |Q|$, i.e. $P \cap Q = Q$. Therefore $Q \subseteq P$ which proves the first part of the proposition.

    Now suppose $Q$ is, in particular, a Sylow $p$-subgroup of $\N{G}{P}$. Since $\N{G}{P} \leq G$ by \myref{exercise-normalizer-is-subgroup-of-main-group}, one concludes that $|G| = y|\N{G}{P}|$ for some positive integer $y$ by Lagrange's Theorem. Hence $xp^n = y|\N{G}{P}|$ which implies $|\N{G}{P}| = \frac{x}{y} p^n$. Set $z = \frac xy$ and note that $p \nmid z$. Therefore, if $Q$ is a Sylow $p$-subgroup, then $|Q| = p^n$. But since $|P| = p^n$ and $Q \subseteq P$, thus $P = Q$, which proves the second part of the proposition.
\end{proof}

\section{The Second Sylow Theorem}
We can now look at the Second Sylow Theorem.

\begin{theorem}[Sylow II]\label{thrm-sylow-2}\index{Sylow Theorem!Second}
    Let $G$ be a finite group and $p$ be a prime number. Suppose $H$ and $K$ are both Sylow $p$-subgroups of $G$. Then there exists an element $g \in G$ such that $gHg^{-1} = K$.
\end{theorem}
\begin{proofsketch}
    Consider the set of conjugates of $H$, i.e. $\mathcal{X} = \{gHg^{-1} \vert g \in G\}$, and let $H$ act on $\mathcal{X}$ by conjugation (which requires proving that this is, indeed, a group action). We then prove that $H$ is the only element of $\mathcal{X}$ with an orbit of 1. Furthermore, by using the Orbit-Stabilizer theorem (\myref{thrm-orbit-stabilizer}), show that any $g \notin H$ results in the orbit of $gHg^{-1}$ being a multiple of $p$. Conclude that the cardinality of $\mathcal{X}$ is 1 more than a multiple of $p$.

    Let $K$ also act on $\mathcal{X}$ by conjugation, and the above conclusion means that there is at least one orbit of length 1. Thus there is a $g \in G$ such that $k(gHg^{-1})k^{-1} = gHg^{-1}$ for all $k \in K$. Deduce that $g^{-1}Kg \subseteq \N{G}{H}$. As $|g^{-1}Kg| = |K|$ which is a Sylow $p$-subgroup, so $g^{-1}Kg$ is a $p$-subgroup, which means $g^{-1}Kg \subseteq H$ by \myref{prop-normalizer-of-sylow-p-subgroup}. Result follows from observing that $|H| = |K|$, meaning $g^{-1}Kg = H$, and trivial manipulation.
\end{proofsketch}
\begin{proof}[Proof (see {\cite[Theorem 11.10]{humphreys_1996}})]
    Define the set
    \[
        \mathcal{X} = \{gHg^{-1} \vert g \in G\}
    \]
    and denote an element from $\mathcal{X}$ by $X$.

    Let the Sylow $p$-subgroup $H$ act on $\mathcal{X}$ by conjugation, meaning $h \cdot X = hXh^{-1}$. We prove that this is a group action.

    \begin{itemize}
        \item \textbf{Closure}: We have to prove closure as it is not implicit in the action's definition.

        Let $X = gHg^{-1}$ for some $g \in G$. Then
        \[
            h\cdot X = hXh^{-1} = h(gHg^{-1})h^{-1} = (hg)H(hg)^{-1}.
        \]
        Since $h \in H \subseteq G$, then $hg \in G$, which thus means that $h \cdot X = (hg)H(hg)^{-1} \in \mathcal{X}$.

        \item \textbf{Identity}: $e \cdot X = eXe^{-1} = X$.

        \item \textbf{Compatibility}: Let $h_1, h_2 \in H$. Then
        \begin{align*}
            h_1 \cdot (h_2 \cdot X) &= h_1 \cdot (h_2Xh_2^{-1})\\
            &= h_1h_2Xh_2^{-1}h_1^{-1}\\
            &= (h_1h_2)X(h_1h_2)^{-1}\\
            &= (h_1h_2) \cdot X.
        \end{align*}
        Thus this is indeed a group action.
    \end{itemize}

    We consider orbits of this group action. In particular, we find element(s) with orbit(s) of only one element. Suppose $X \in \mathcal{X}$ has one element in its orbit, so for all $h \in H$ we have
    \[
        h\cdot X = hXh^{-1} = X.
    \]
    Since $X = gHg^{-1}$ for some $g \in G$, thus
    \begin{align*}
        &h(gHg^{-1})h^{-1} = gHg^{-1}\\
        \iff&hgHg^{-1} = gHg^{-1}h\\
        \iff&hgH = gHg^{-1}hg\\
        \iff&(g^{-1}hg)H = H(g^{-1}hg)\\
        \iff&g^{-1}hg \in \N{G}{H}\\
        \iff&g^{-1}Hg \subseteq \N{G}{H}.
    \end{align*}
    Since $|g^{-1}hg| = |h|$ by \myref{prop-order-of-conjugate-element-equals-order-of-element}, and because $H$ is a $p$-subgroup, therefore $|g^{-1}hg| = p^r$ where $r$ is some positive integer. Note that $g^{-1}Hg$ is a (sub)group by \myref{exercise-conjugate-subgroup}. Therefore $g^{-1}Hg$ is a $p$-subgroup of $\N{G}{H}$.

    Furthermore, we know $g^{-1}Hg \cong H$ by \myref{exercise-conjugate-subgroup-isomorphic-to-subgroup}, so $|H| = |g^{-1}Hg|$. Therefore $g^{-1}Hg$ is a Sylow $p$-subgroup of $\N{G}{H}$. Hence, by \myref{prop-normalizer-of-sylow-p-subgroup}, we see $H = g^{-1}Hg$, implying $gHg^{-1} = H$. But because $X = gHg^{-1}$, so $X = H$.

    Hence $H$ is the only element of $\mathcal{X}$ with $|\Orb{H}{X}| = 1$. Therefore for any $g \notin H$, $|\Orb{H}{gHg^{-1}}| > 1$. In fact, the Orbit-Stabilizer Theorem (\myref{thrm-orbit-stabilizer}) tells us that
    \[
        |\Stab{H}{gHg^{-1}}| = \frac{|H|}{|\Orb{H}{gHg^{-1}}|}
    \]
    and since $|H|$ is a power of $p$ and $|\Stab{H}{gHg^{-1}}|$ is an integer, thus $|\Orb{H}{gHg^{-1}}|$ has to be a power of $p$, meaning $|\Orb{H}{gHg^{-1}}| \equiv 0 \pmod p$. As distinct orbits partition the set $\mathcal{X}$ (\myref{exercise-distinct-orbits-partition-set}), thus $|\mathcal{X}| \equiv 1 \pmod p$.

    Now let the Sylow $p$-subgroup $K$ act on $\mathcal{X}$ by conjugation, meaning $k \star X = kXk^{-1}$. This is a group action as proven before, and the above conclusion means that there is at least one orbit of length 1. Hence there exists a $g \in G$ such that for all $k \in K$, we have $k(gHg^{-1})k^{-1} = gHg^{-1}$. Thus $g^{-1}kg \in \N{G}{H}$ which means $g^{-1}Kg \subseteq \N{G}{H}$. Note that as $|g^{-1}Kg| = |K|$ and $K$ is a Sylow $p$-subgroup of $G$, we see that $g^{-1}Kg$ is a $p$-subgroup of $G$. Therefore by \myref{prop-normalizer-of-sylow-p-subgroup}, we conclude $g^{-1}Kg \subseteq H$, meaning $K \subseteq gHg^{-1}$. But because $|H| = |K|$ as they are both Sylow $p$-subgroups, therefore $K = gHg^{-1}$.
\end{proof}

We note one important corollary of the Second Sylow Theorem.
\begin{corollary}\label{corollary-sylow-subgroup-is-normal-if-it-is-unique}
    Let $G$ be a finite group and $P$ be a Sylow $p$-subgroup for some prime $p$. Then $P$ is a normal subgroup of $G$ if and only if $P$ is the only Sylow $p$-subgroup of $G$.
\end{corollary}
\begin{proof}
    The reverse direction is easy to prove. Since $P$ is the only Sylow $p$-subgroup of $G$, this means that $P$ is the only subgroup of order $p^k$. By \myref{thrm-unique-subgroup-of-given-order-is-normal}, this means that $P \unlhd G$.

    We work on the forward direction now and suppose $P$ is a normal subgroup of $G$. Let $\hat{P}$ be a normal Sylow $p$-subgroup. By the Second Sylow Theorem (\myref{thrm-sylow-2}), there exists $g \in G$ such that $g\hat{P}g^{-1} = P$. But since $\hat{P}$ is normal, thus $g\hat{P}g^{-1} = \hat{P}$ by definition of normality. Hence, $P = \hat{P}$, meaning that there is only one Sylow $p$-subgroup.
\end{proof}

\begin{exercise}
    Let $G$ be a finite group, $p$ be a prime number, and $H$ and $K$ be distinct Sylow $p$-subgroups of $G$. Prove that $H \cong K$.
\end{exercise}

\section{The Third Sylow Theorem}
The Third Sylow Theorem gives us a way to determine how many Sylow $p$-subgroups exist within a group. This theorem is very useful in determining the structure of finite groups, and can help us reason our way through the properties that a finite group of a fixed order must have.

\begin{theorem}[Sylow III]\label{thrm-sylow-3}\index{Sylow Theorem!Third}
    Let $G$ be a finite group with order $p^k m$ where $p$ is prime, $k \geq 1$, and $p \nmid m$. Let $n_p$ denote the number of Sylow $p$-subgroups in $G$, i.e. $n_p = |\Syl{p}{G}|$. Then,
    \begin{enumerate}
        \item $n_p = [G:\N{G}{P}]$, where $P$ is a Sylow $p$-subgroup of $G$;
        \item $n_p$ divides $m$; and
        \item $n_p \equiv 1 \pmod p$.
    \end{enumerate}
\end{theorem}

\begin{proof}[Proof (see \cite{wielandt_1959})]
    We prove the three statements in order.
    \begin{enumerate}
        \item Let $G$ act on $\Syl{p}{G}$ by conjugation, meaning that for any $g \in G$ and $P \in \Syl{p}{G}$, $g\cdot P = gPg^{-1}$. By the Second Sylow Theorem (\myref{thrm-sylow-2}), all Sylow $p$-subgroups are conjugates of each other, so the orbit of any $P \in \Syl{p}{G}$ is the set of all Sylow $p$-subgroups, meaning $|\Orb{G}{P}| = |\Syl{p}{G}| = n_p$. Now consider $\Stab{G}{P}$; note $\Stab{G}{P} = \{g \in G \vert gPg^{-1} = P\} = \N{G}{P}$ by definition of the normalizer. Therefore
        \[
            n_p = |\Orb{G}{P}| = \frac{|G|}{|\Stab{G}{P}|} = \frac{|G|}{|\N{G}{P}|} = [G : \N{G}{P}],
        \]
        by the Orbit-Stabilizer Theorem (\myref{thrm-orbit-stabilizer}), proving the first statement.

        \item Let $P$ be a Sylow $p$-subgroup. We recall that $P \unlhd \N{G}{P} \leq G$. Note that by Lagrange's theorem (\myref{thrm-lagrange}) we know $|\N{G}{P}| = [\N{G}{P} : P]|P| = ap^k$ where $a \leq m$ (since $\N{G}{P} \leq G$). Furthermore
        \[
            mp^k = |G| = [G: \N{G}{P}]|\N{G}{P}| = n_p \times ap^k
        \]
        which means $m = an_p$. In other words, $n_p$ divides $m$.

        \item Let $H$ be a Sylow $p$-subgroup and let it act on $\Syl{p}{G}$ by conjugation, meaning that for any $h \in H$ and $P \in \Syl{p}{G}$, $h \cdot P = hPh^{-1}$. Let $\Omega$ denote the set of fixed points of $\Syl{p}{G}$ under this action.

        Suppose $Q \in \Omega$, which means that $hQh^{-1} = Q$ for all $h \in H$. Thus $H \subseteq \N{G}{Q}$ as $\N{G}{Q} = \{g \in G \vert gQg^{-1} = Q\}$. In fact, since $H$ is a Sylow $p$-subgroup, $H \leq \N{G}{Q}$. We note $Q \unlhd \N{G}{Q}$ by \myref{prop-subgroup-is-a-normal-subgroup-of-normalizer}. Hence $H$ and $Q$ are Sylow $p$-subgroups of $\N{G}{Q}$, which means there exists $n \in \N{G}{Q}$ such that $nQn^{-1} = H$ by the Second Sylow Theorem (\myref{thrm-sylow-2}). Furthermore $nQn^{-1} = Q$ since $Q \unlhd \N{G}{Q}$. Hence $Q = H$, which means that the only element in $\Omega$ is $H$.

        By a similar argument posed in the proof of the Second Sylow Theorem, for any $Q \in \Syl{p}{G}$ where $Q \neq H$ we have $|\Orb{H}{Q}| \equiv 0 \pmod p$. Note $|\Orb{H}{H}| = 1$. As distinct orbits partition $\Syl{p}{G}$ we must have $n_p = |\Syl{p}{G}| \equiv 1 \pmod p$.
    \end{enumerate}
    This completes the proof.
\end{proof}

\begin{example}
    We show that any group with order 4225 is abelian.

    Note that $4225 = 5^2 \times 13^2$. Let $G$ be a group of order 4225. By the Third Sylow Theorem (\myref{thrm-sylow-3}), we know that
    \begin{itemize}
        \item $n_5 \mid 13^2 = 169$ and $n_{13} \mid 5^2 = 25$, which means $n_5 \in \{1, 13, 169\}$ and $n_{13} \in \{1, 5, 25\}$; and
        \item $n_5 \equiv 1 \pmod 5$ and $n_{13} \equiv 1 \pmod{13}$.
    \end{itemize}
    Hence, $n_5 = 1$ and $n_{13} = 1$, meaning that there is only one Sylow 5-subgroup and one Sylow 13-subgroup.

    Let $P$ be the Sylow 5-subgroup and $Q$ be the Sylow 13-subgroup. By \myref{corollary-sylow-subgroup-is-normal-if-it-is-unique}, $P \lhd G$ and $Q \lhd G$, so $pq = qp$ for any $p \in P$ and $q \in Q$. Furthermore, by \myref{problem-intersection-of-coprime-subgroups}, $P \cap Q = \{e\}$. Finally, notice that
    \[
        |PQ| = \frac{|P||Q|}{|P\cap Q|} = |P||Q| = 5^2\times13^2 = |G|
    \]
    which means that $PQ$ and $G$ have the same number of elements. As $PQ \leq G$ by the Second Isomorphism Theorem (\myref{thrm-isomorphism-2}), statement 3, thus $G = PQ$. Hence, $G$ is the internal direct product of $P$ and $Q$. We note that
    \[
        G = PQ \cong P\times Q
    \]
    by direct product equivalence (\myref{thrm-direct-product-equivalence}). In addition, since $P$ and $Q$ are groups of prime-squared order, they are abelian (\myref{problem-group-of-order-prime-squared-is-abelian}), meaning that their external direct product $P\times Q$ is also abelian (\myref{problem-external-direct-product-of-abelian-groups-is-abelian}). Hence $G$ is abelian.
\end{example}

\begin{exercise}
    Let $G$ be a group of order 784, and let $P$ be a Sylow 7-subgroup that is not a normal subgroup of $G$. Find the order of $\N{G}{P}$.
\end{exercise}

\section{Testing the Non-simplicity of Groups}
To end this chapter, we look at the idea of simple groups and describe ways to prove the non-simplicity of groups.
\begin{definition}
    Let $G$ be a finite group with order of at least 2. Then $G$ is \term{simple}\index{group!simple} if the only normal subgroups of $G$ are the trivial subgroup and $G$ itself. Equivalently, $G$ is simple if $G$ has no non-trivial proper normal subgroups.
\end{definition}

We use part of the Third Sylow Theorem to create a test for non-simplicity.

\begin{theorem}[Sylow's Test]\index{Sylow's Test}
    Let $n$ be a non-prime integer and let $p$ be a prime divisor of $n$. If 1 is the only divisor of $n$ that is congruent 1 modulo $p$, then there does not exist a simple group of order $n$.
\end{theorem}
\begin{proof}
    We consider two cases, namely the case where $n=p^k$ where $k>1$ and the case where $n \neq p^k$.

    We first suppose $n = p^k$. Let $G$ be a group of order $n$. We consider two subcases.
    \begin{itemize}
        \item Suppose $G$ is abelian. By writing $p^k$ as $p \times p^{k-1}$, Cauchy's theorem (\myref{thrm-cauchy}) tells us that there exists a subgroup of order $p$. Since every subgroup of an abelian group is normal (\myref{prop-subgroup-of-abelian-group-is-normal}), thus there exists a non-trivial proper normal subgroup of $G$, meaning $G$ is non-simple.
        \item Now suppose $G$ is non-abelian. By \myref{example-group-with-prime-power-order-has-non-trivial-center}, we know $G$ has a non-trivial center. Furthermore, $G \neq \CenterGrp{G}$ because $G$ is non-abelian (\myref{problem-center-of-G}). Since the center is a normal subgroup of $G$, it is thus a non-trivial proper normal subgroup of $G$, meaning $G$ is non-simple.
    \end{itemize}
    Hence any group of order $p^k$ where $k > 1$ is non-simple.

    Now suppose $n$ is not a prime power. By the Third Sylow Theorem (\myref{thrm-sylow-3}), the number of Sylow $p$-subgroups, $n_p$, is congruent to 1 modulo $p$ and divides $n$. Since 1 is the only such number by our assumption, thus $n_p = 1$, meaning that there is only one Sylow $p$-subgroup. By \myref{corollary-sylow-subgroup-is-normal-if-it-is-unique} this means that that Sylow $p$-subgroup (which is non-trivial and proper) is normal, which thus means that any group of order $n$ is non-simple.
\end{proof}
\begin{example}
    Consider a group with order 15. Note $15 = 3 \times 5$, and consider $p = 5$. The divisors of 15 are 1, 3, 5, and 15, and clearly only 1 is congruent to 1 modulo 5. Hence by Sylow's Test we know that a group of order 15 cannot be simple.
\end{example}

For some groups of orders that do not satisfy the condition in Sylow's Test, we can still can prove that they are never simple.

\begin{example}
    We will show that a group of order 2552 is non-simple.

    We note first that $2552 = 2^3 \times 11 \times 29$. By the Third Sylow Theorem (\myref{thrm-sylow-3}), we know $n_p \mid m$ and $n_p \equiv 1 \pmod p$.

    The divisors of $m$ given the following primes are listed below.
    \begin{itemize}
        \item $p = 2$: $m = 319$ and so divisors are $\{1, 11, 29, 319\}$.
        \item $p = 11$: $m = 232$; divisors are $\{1, 2, 4, 8, 29, 58, 116, 232\}$.
        \item $p = 29$: $m = 88$; divisors are $\{1, 2, 4, 8, 11, 22, 44, 88\}$.
    \end{itemize}
    Thus, since $n_p \equiv 1 \pmod p$, we must have $n_2 \in \{1, 11, 29, 319\}$, $n_{11} \in \{1, 232\}$, and $n_{29} \in \{1, 88\}$.

    Seeking a contradiction, suppose that $n_2$, $n_{11}$, and $n_{29}$ are all not 1. Then $n_{11} = 232$ and $n_{29} = 88$. Recall that one element in a Sylow $p$-subgroup has order 1 (i.e., the identity). Thus, the number of elements of order 11 is $232 \times (11 - 1) = 2320$ and the number of elements of order 29 is $88 \times (29 - 1) = 2464$. Hence, the total number of elements in the group of order 2552 must be at least $2320 + 2464 = 4764$, a contradiction.

    Hence, we conclude that at least one of $n_2, n_{11}, n_{29}$ must be 1, meaning that there is a proper normal subgroup, which therefore means that any group of order 2552 is non-simple.
\end{example}

\begin{example}\label{example-using-kernel-to-show-non-simple}
    We show that any group of order 36 is non-simple by considering the kernel of a homomorphism.

    We note $36 = 2^2 \times 3^2$. Let $G$ be a group of order 36 and let $P$ be a Sylow 3-subgroup of the group of order 36. Let $G$ act on the set of cosets $G/P$ by left multiplication, meaning $g \cdot xP = (gx)P$. Then by \myref{thrm-group-action-definition-equivalence} this induces a homomorphism $\phi: G \to \Sn{4}$ since there are 4 cosets in $G/P$. We note $\phi(g) = \sigma_g$ where $\sigma_g(xP) = g\cdot xP = (gx)P$.

    We consider the kernel of $\phi$.
    \begin{align*}
        \ker\phi &= \{g \in G \vert \phi(g) = \id\}\\
        &= \{g \in G \vert \sigma_g = \id\}\\
        &= \{g \in G \vert \sigma_g(xP) = xP \text{ for all } x \in G\}\\
        &= \{g \in G \vert (gx)P = xP \text{ for all } x \in G\}\\
        &= \{g \in G \vert x^{-1}gx \in P \text{ for all } x \in G\}\\
        &= \{g \in G \vert g \in xPx^{-1} \text{ for all } x \in G\}\\
        &= \bigcap_{x \in G} xPx^{-1}.
    \end{align*}
    We note that $\ker\phi \neq \{e\}$ since that would imply that $\phi$ is injective (\myref{exercise-trivial-kernel-means-injective}), which would then mean $36 = |G| \leq |\Sn{4}| = 4! = 24$, a contradiction. We also note $\ker\phi \neq G$, otherwise
    \[
        36 = |G| = |\ker\phi| = \left|\bigcap_{x \in G} xPx^{-1}\right| \leq |xPx^{-1}| = |P| = 9,
    \]
    a contradiction. Hence $\ker\phi$ is a non-trivial proper subgroup of $G$. We note that $\ker\phi \lhd G$, so we have found a non-trivial proper normal subgroup of $G$, meaning that $G$ is non-simple.
\end{example}

\begin{exercise}
    Show that any group of order $130$ is non-simple.
\end{exercise}



\section{Problems}
\begin{problem}
    Show that a group of order 200 has a normal Sylow 5-subgroup.
\end{problem}

\begin{problem}
    Show that any Sylow $p$-subgroup of a group of order 33 must be normal.
\end{problem}

\begin{problem}
    A \term{perfect number}\index{perfect number} is a positive integer that is equal to the sum of its positive divisors, excluding the number itself. All even perfect numbers are of the form $2^{p-1}\left(2^p-1\right)$ where both $p$ and $2^p-1$ are primes. Prove that any group with an even perfect number order is not simple.
\end{problem}

\begin{problem}\label{problem-group-of-order-pq-has-normal-subgroup-of-order-q}
    Let $p$ and $q$ be primes such that $p < q$. Let $G$ be a group of order $pq$.
    \begin{partquestions}{\roman*}
        \item Prove that there is only one subgroup $H$ of $G$ of order $q$. Deduce that $H \lhd G$.
        \item Prove also that if $q \not\equiv 1 \pmod p$ then $G$ is cyclic.
    \end{partquestions}
\end{problem}

\begin{problem}\label{problem-normal-subgroup-of-G-contains-all-sylow-p-subgroups}
    Let $G$ be a finite group, and write the order of $G$ as $p^km$ where $k \geq 0$, $m \geq 1$, and $p$ is a prime such that $p \nmid m$. Let $N \lhd G$ such that $p$ does not divide the index of $N$ in $G$. Prove that $N$ contains all Sylow $p$-subgroups of $G$ and vice versa. That is,
    \begin{partquestions}{\roman*}
        \item prove that any Sylow $p$-subgroup of $N$ is also in $G$; and
        \item prove that any Sylow $p$-subgroup of $G$ is also in $N$.
    \end{partquestions}
\end{problem}

\begin{problem}
    Show that any group of order 3325 is abelian.
\end{problem}

\begin{problem}\label{problem-if-m!<|G|-then-G-is-simple}
    Let $G$ be a finite group such that $|G| = p^km$ where $k \geq 1$, $m > 1$, and $p$ a prime such that $p \nmid m$. Prove that if $m! < |G|$ then $G$ is non-simple.
\end{problem}

\begin{problem}\label{problem-group-of-order-30-has-normal-subgroup-of-order-5}
    Prove that a group of order 30 has a normal subgroup of order 5.
\end{problem}

\begin{problem}\label{problem-group-of-order-pqr-is-non-simple}
    Let $p, q$, and $r$ be distinct primes such that $p < q < r$. Let $G$ be a group of order $pqr$. Prove that $G$ is non-simple.\newline
    (\textit{Hint: show that a normal subgroup of order $p$, $q$, or $r$ must exist.})
\end{problem}
