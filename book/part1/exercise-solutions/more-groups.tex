\section{More Types of Groups}
\begin{questions}
    \item Let $G = \Z_{mn}$ and $H = \{0, n, 2n, \dots, (m-1)n\}$. Clearly $H$ is a subgroup of $G$ of order $m$. By \myref{problem-subgroup-of-cyclic-group-is-cyclic} we know $H$ is normal and cyclic with order $m$ and by \myref{exercise-quotient-group-of-cyclic-group-is-cyclic} we know $G/H$ is cyclic. The order of $G/H$ is $\frac{|G|}{|H|} = \frac{mn}{m} = n$ by Lagrange's Theorem (\myref{thrm-lagrange}), meaning that $G/H \cong \Z_n$. Hence, $\Z_{mn}/\Z_m \cong G/H \cong \Z_n$.

    \item Note 0 is the identity in $\Z_n$. By \myref{lemma-order-of-an-element-that-is-equivalent-to-identity}, we know that if $12$ is equivalent to the identity in $\Z_n$, then $12 = mn$ for some integer $m$. Since $n > 0$ we restrict $m$ to positive integers. Now $12 = 2^2 \times 3$. Thus the possible cases are:
    \begin{itemize}
        \item $n = 1$ with $m = 12$;
        \item $n = 2$ with $m = 6$;
        \item $n = 3$ with $m = 4$;
        \item $n = 4$ with $m = 3$;
        \item $n = 6$ with $m = 2$; and
        \item $n = 12$ with $m = 1$.
    \end{itemize}

    \item $|10| = \frac{210}{\gcd(10, 210)} = \frac{210}{10} = 21$, $|42| = \frac{210}{\gcd(42, 210)} = \frac{210}{42} = 5$, $|75| = \frac{210}{\gcd(75, 210)} = \frac{210}{15} = 14$, and $|140| = \frac{210}{\gcd(140, 210)} = \frac{210}{70} = 3$.

    \item \begin{partquestions}{\alph*}
        \item Note that $10 = 2 \times 5$. Generators of the group $\Z_n$ (which has order 10) has to satisfy $\gcd(m,n) = 1$ by \myref{corollary-element-in-cyclic-group-is-generator-iff-gcd-is-1}. The positive integers that satisfy this requirement (and which are less than 10) are 1, 3, 7, 9. Thus they are the generators of $\Z_{10}$.
        \item Note that 101 is prime. Hence all positive integers from 1 to 100 (inclusive) are generators. (Note that 0 is not as 0 is the identity.)
    \end{partquestions}

    \item We show that all subgroups of $\mathrm{Q}$ are, in fact, normal. We consider the first definition of the quaternion group.
    \begin{itemize}
        \item Clearly $\{1\} \lhd \mathrm{Q}$ and $\mathrm{Q} \unlhd \mathrm{Q}$.
        \item The subgroups $\langle i\rangle$, $\langle j\rangle$, and $\langle k\rangle$ have order 4. Therefore, Lagrange's Theorem (\myref{thrm-lagrange}) tells us that they have index 2. Hence these subgroups are normal by \myref{problem-subgroup-of-index-2}.
        \item Consider the subgroup $\langle -1 \rangle = \{1, -1\}$. \begin{itemize}
            \item $1\langle -1 \rangle = \langle -1 \rangle1$, since 1 is the identity;
            \item $-1\langle -1 \rangle = \{1, -1\} = \langle -1 \rangle(-1)$;
            \item $i\langle -1 \rangle = \{-i, i\} = \langle -1 \rangle i$;
            \item $-i\langle -1 \rangle = \{i, -i\} = \langle -1 \rangle (-i)$;
            \item $j\langle -1 \rangle = \{-j, j\} = \langle -1 \rangle j$;
            \item $-j\langle -1 \rangle = \{j, -j\} = \langle -1 \rangle (-j)$;
            \item $k\langle -1 \rangle = \{-k, k\} = \langle -1 \rangle k$; and
            \item $-k\langle -1 \rangle = \{k, -k\} = \langle -1 \rangle (-k)$.
        \end{itemize}
        Thus $\langle -1 \rangle$ is normal.
    \end{itemize}
    Hence all subgroups of $\mathrm{Q}$ are normal.

    \item (2 6) = (2 3)(3 4)(4 5)(5 6)(4 5)(3 4)(2 3).

    \item Note that (1 3 2 5 4) = (1 4)(1 5)(1 2)(1 3). Thus by \myref{thrm-parity-of-permutation}, (1 3 2 5 4) is even and thus has a sign of $+1$.

    \item Note that $\An{3}$ has order $\frac{3!}{2} = 3$ so we should expect 3 permutations. Clearly the identity is one such permutation. Looking at \myref{example-symmetric-group-of-degree-3} we can find two more: (1 2 3) and (1 3 2).

    \item $\Un{10} = \{1, 3, 7, 9\}$.

    \item By a corollary of Lagrange's Theorem (\myref{corollary-order-of-group-multiple-of-order-of-element}), the order of $a$ dives the order of the group $\Un{n}$. Now the order of $\Un{n} = \totient(n)$. Thus the order of $a$ divides $\totient(n)$.

    \item The matrix product should be $\begin{pmatrix}2&1&2\\1&0&1\\2&1&2\end{pmatrix}$.

    \item We already proved that $\Inn{G} \leq \Aut{G}$ so we only need to prove normality.

    Let $\phi \in \Aut{G}$ and $\iota_g \in \Inn{G}$. For brevity let $f = \phi\iota_g\phi^{-1}$. We note that $f \in \Aut{G}$; we need to prove that $f \in \Inn{G}$.

    Suppose $x \in G$ such that $w = \phi^{-1}(x)$ (as $\phi$ is an isomorphism, there exists a $w \in G$). Then
    \begin{align*}
        f(x) &= \left(\phi\iota_g\phi^{-1}\right)(x)\\
        &= \phi(\iota_g(\phi^{-1}(x)))\\
        &= \phi(\iota_g(w))\\
        &= \phi(gwg^{-1})\\
        &= \phi(g)\phi(w)\phi(g^{-1})\\
        &= \phi(g)x\left(\phi(g)\right)^{-1}
    \end{align*}
    which shows that $f \in \Inn{G}$. Hence, $\Inn{G} \unlhd \Aut{G}$.
\end{questions}
