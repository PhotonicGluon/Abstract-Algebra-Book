\section{Finite Fields}
\begin{questions}
    \item Note for any $\alpha \in S$, because $f(\alpha) = \alpha^{p^n} - \alpha = 0$, thus $\alpha^{p^n} = \alpha$. This fact will be useful later.
    \begin{itemize}
        \item Clearly $f(1) = 1^{p^n} - 1 = 0$ so $1 \in S^\ast$.

        \item Now let $\alpha,\beta \in S$. Then we see
        \begin{align*}
            f(\alpha - \beta) &= (\alpha-\beta)^{p^n} - (\alpha - \beta)\\
            &= \alpha^{p^n} + (-1)^{p^n}\beta^{p^n} - \alpha + \beta & (\myref{prop-freshman-dream})\\
            &= (\alpha^{p^n} - \alpha) + ((-1)^{p^n}\beta^{p^n} + \beta)\\
            &= 0 + (-1)^{p^n}\beta^{p^n} + \beta & (\text{as } \alpha \in S)\\
            &= (-1)^{p^n}\beta^{p^n} + \beta.
        \end{align*}
        We further split into two cases.
        \begin{itemize}
            \item If $p$ is odd then $(-1)^{p^n}\beta^{p^n} + \beta = -(\beta^{p^n} - \beta) = 0$.
            \item If $p$ is even then $p = 2$. So $(-1)^{p^n}\beta^{p^n} + \beta = \beta^{2^n} + \beta = 2\beta = 0$ since $\Char{F} = 2$.
        \end{itemize}
        In either case, we see that $f(\alpha-\beta) = 0$, meaning $\alpha - \beta \in S$.

        \item Finally, for $\alpha \in S$ and $\beta \in S^\ast$ we see
        \begin{align*}
            f(\alpha\beta^{-1}) &= (\alpha\beta^{-1})^{p^n} - (\alpha\beta^{-1})\\
            &= \alpha^{p^n}\left(\beta^{-1}\right)^{p^n} - \alpha\beta^{-1}\\
            &= \alpha^{p^n}\left(\beta^{p^n}\right)^{-1} - \alpha\beta^{-1}\\
            &= \alpha\beta^{-1} - \alpha\beta^{-1}\\
            &= 0
        \end{align*}
        and so $\alpha\beta^{-1} \in S$.
    \end{itemize}
    Therefore by the subfield test (\myref{thrm-subfield-test}) we see that $S$, the set of zeroes of $f(x)$ in $F$, is a subfield of $F$.

    \item After some work, we notice that
    \begin{align*}
        f(\alpha^2) &= (\alpha^2)^4 + (\alpha^2) + 1\\
        &= \alpha^8 + \alpha^2 + 1\\
        &= (\alpha^2 + 1) + \alpha^2 + 1\\
        &= 2\alpha^2 + 2\\
        &= 0\\
        f(\alpha^4) &= (\alpha^4)^4 + (\alpha^4) + 1\\
        &= \alpha^{16} + \alpha^4 + 1\\
        &= \alpha + (\alpha + 1) + 1\\
        &= 2\alpha + 2\\
        &= 0\\
        f(\alpha^8) &= (\alpha^8)^4 + (\alpha^8) + 1\\
        &= \alpha^{32} + \alpha^8 + 1\\
        &= \alpha^2 + (\alpha^2 + 1) + 1\\
        &= 2\alpha^2 + 2\\
        &= 0
    \end{align*}
    so the other 3 zeroes of $f(x)$ in $F$ are $\alpha^2$, $\alpha^4$, and $\alpha^8$.

    \item Clearly $1 \in K$ since $1^{p^m} = 1$, so $K^\ast \neq \emptyset$. Let $x, y \in K$, meaning $x^{p^m} = x$ and $y^{p^m} = y$. Then
    \begin{align*}
        (x-y)^{p^m} &= x^{p^m} - y^{p^m} & (\myref{prop-freshman-dream})\\
        &= x - y\\
        &\in K
    \end{align*}
    and if $y \neq 0$ then
    \begin{align*}
        (xy^{-1})^{p^m} &= x^{p^m}\left(y^{-1}\right)^{p^m}\\
        &= x^{p^m}\left(y^{p^m}\right)^{-1}\\
        &= xy^{-1}\\
        &\in K
    \end{align*}
    and so by the subfield test (\myref{thrm-subfield-test}) we have shown that $K$ is a subfield of $\GF{p^n}$.

    \item By \myref{thrm-subfields-of-finite-field} we know that $\GF{p^n}$ indeed has a subfield isomorphic to $\GF{p^m}$. So one sees by Tower Law (\myref{thrm-tower-law}) that
    \[
        [\GF{p^n}:\GF{p}] = [\GF{p^n}:\GF{p^m}][\GF{p^m}:\GF{p}]
    \]
    which means
    \[
        n = [\GF{p^n}:\GF{p^m}]m
    \]
    by \myref{corollary-degree-of-finite-field-to-prime-power}. Therefore $[\GF{p^n}:\GF{p^m}] = \frac nm$ as required.

    \item Note that $8 = 2^3$ so the only subfields of $F$ must have order $2^1 = 2$ and $2^3 = 8$. In particular, the subfield of order 2 is $\{0, 1\}$ and the subfield of order 8 is the whole field $F$.
\end{questions}
