\chapter{Finite Fields}
Finite fields appear in many applications of algebra (like in the NTRU cryptosystem). We already found out that $\Z_p$ is a field, and that we may construct fields of prime-power order via the use of irreducible polynomials. In this chapter, we discover one of the most surprising and important results relating to these fields.

\section{Classification and Structure of Finite Fields}
\'Evariste Galois first introduced finite fields in his proof of the insolvability of the quintic in 1830. Today, finite fields are immensely useful in many aspects of algebra and computing.

The most surprising and important result regarding finite fields is that there is a unique finite field (up to isomorphism) of order $p^n$ where $p$ is a prime. The existence of such a field was given by Galois and Gauss in the early 19th century, but it took until Dedekind in 1857 and Jordan in 1870 to produce a rigorous proof of the existence of such fields. The uniqueness of such a field was proved by Eliakim Hastings Moore in 1893. We state his theorem below.

\begin{theorem}\label{thrm-finite-field-is-unique}
    For each prime $p$ and every positive integer $n$, there exists a finite field of order $p^n$. Furthermore any field of order $p^n$ is isomorphic to the splitting field of $x^{p^n} - x$ over $\Z_p$.
\end{theorem}
\begin{proof}[Proof (see {\cite[Theorem 22.1]{gallian_2016}} and {\cite[Theorem 22.6]{judson_beezer_2022}})]
    Let the polynomial $f(x) = x^{p^n} - x$ and let $F$ be the splitting field of $f(x)$ over $\Z_p$. We show that $|F| = p^n$. Since $f(x)$ has degree $p^n$ and splits over $F$, thus there must be $p^n$ zeroes of $f(x)$ in $F$. Also we know that each of these $p^n$ zeroes are simple (\myref{problem-(x^p^n-x)-only-has-simple-zeroes}), meaning that $f(x)$ has $p^n$ distinct zeroes in $F$. We note that \myref{exercise-zeroes-of-polynomial-form-subfield} (later) shows that the set of zeroes of $f(x)$ in $F$, which we will denote by $S$, forms a subfield of $F$. Note $|S| = p^n$. But as $S$ contains all zeroes of $f(x)$ in $F$, we know that $f(x)$ splits over $S$. As $S \subseteq F$, thus $S = F$, meaning $|F| = p^n$.

    We now show uniqueness. Suppose $K$ is another field of order $p^n$. To show that $K \cong F$, we need to show that every element of $K$ is a zero of $f(x)$. Certainly $0 \in K$ is a zero of $f(x)$. Let $\alpha \in K$ be a non-zero element. Since the multiplicative group of $K$ has order $p^n - 1$, thus we must have $\alpha^{p^n-1} = 1$, which means $\alpha^{p^n} = \alpha$ and so $\alpha^{p^n}-\alpha = 0$. Thus any non-zero $\alpha \in K$ is a zero of $f(x)$. Since $|K| = p^n$, thus $K$ must be a splitting field of $f(x)$ over $\Z_p$. But by \myref{corollary-splitting-field-unique-up-to-isomorphism}, splitting fields are unique up to isomorphism, which shows that $K \cong F$.
\end{proof}

\begin{exercise}\label{exercise-zeroes-of-polynomial-form-subfield}
    In \myref{thrm-finite-field-is-unique}, prove that $S$, the set of zeroes of $f(x)$ in $F$, is a subfield of $F$.
\end{exercise}

As there is only one field (up to isomorphism) that is of order $p^n$, we may unambiguously call such a field \textit{the} field of order $p^n$.

\begin{definition}
    Let $p$ be a prime number and $n$ be a positive integer. The field of order $p^n$ is called the \term{Galois field of order $p^n$}\index{Galois field} and is denoted by $\GF{p^n}$.
\end{definition}
\begin{remark}
    We call $\GF{p^n}$ the Galois field of order $p^n$ in honour of Galois who first introduced finite fields in 1830.
\end{remark}
\begin{remark}
    $\GF{p}$ is the field $\Z_p$.
\end{remark}

Since there is only one finite field of a given order (up to isomorphism), it is easy to characterise the additive group and multiplicative group of such a field.

\begin{theorem}\label{thrm-structure-of-finite-field}
    Let $p$ be a prime and $n$ be a non-negative integer. The additive group of $\GF{p^n}$ is isomorphic to $\Cn{p}^n$ and the multiplicative group of $\GF{p^n}$ is isomorphic to $\Cn{p^n-1}$ (and is, therefore, cyclic).
\end{theorem}
\begin{proof}
    Since $\GF{p^n}$ has characteristic $p$, the additive order of any element in $\GF{p^n}$ is at most $p$. Thus the cyclic subgroup generated by an arbitrary element in the additive group of $\GF{p^n}$ has at most $p$ elements; in fact there can only be subgroups of $p$ elements or 1 element (the trivial subgroup). Therefore the Fundamental Theorem of Finite Abelian Groups (\myref{thrm-fundamental-theorem-of-finite-abelian-groups}) tells us that
    \[
        \GF{p^n} \cong \underbrace{\Cn{p} \times \Cn{p} \times \cdots \times \Cn{p}}_{n\text{ times}} = \Cn{p}^n
    \]
    when viewed as a group under addition.

    Now note that by the Fundamental Theorem of Finite Abelian Groups again we see that the multiplicative group of $\GF{p^n}$, i.e. $\GF{p^n}^\ast$, is isomorphic to $\Cn{n_1}\times\Cn{n_2}\cdots\times\Cn{n_m}$ where $n_1, n_2, \dots, n_m$ are positive integers with a product of $p^n - 1$. We show that these integers are pairwise coprime via contradiction; suppose for some $n_i$ and $n_j$ in that list that there is a $d > 1$ that divides both. In particular, a prime $q$ must divide both $n_i$ and $n_j$. By Cauchy's theorem (\myref{thrm-cauchy}) there must be a subgroup of order $q$ in both $\Cn{n_i}$ and $\Cn{n_j}$, which means that $\GF{p^n}^\ast$ has two distinct subgroups of order $q$. Call these subgroups $H$ and $K$. Let us focus on $H$ for now. Since $H$ has order $q$, we must have $x^q = 1$ for all $x \in H$, meaning that $x^q - 1 = 0$ for all $x \in H$. We also see that $x^q - 1 = 0$ for all $x \in K$. Since both $H$ and $K$ are of fields of order $q$, thus $x^q - 1$ has more than $q$ zeroes in $\GF{p^n}$, contradicting the fact that a polynomial of degree $q$ has at most $q$ zeroes over a field (\myref{thrm-polynomial-of-degree-n-has-at-most-n-zeroes}). Therefore each of the positive integers $n_1, n_2, \dots, n_m$ are pairwise coprime. By \myref{thrm-Zm-cross-Zn-isomorphic-to-Zmn-condition} we therefore see that
    \[
        \Cn{n_1}\times\Cn{n_2}\cdots\times\Cn{n_m} \cong \Cn{n_1n_2\cdots n_m} = \Cn{p^n-1}
    \]
    and so $\GF{p^n}^\ast \cong \Cn{p^n-1}$, proving the second part of the theorem.
\end{proof}

\begin{corollary}\label{corollary-degree-of-finite-field-to-prime-power}
    $[\GF{p^n}: \GF{p}] = n$ for any prime $p$ and any positive integer $n$.
\end{corollary}
\begin{proof}
    Note that the additive group of $\GF{p^n}$ is isomorphic to $\Z_p^n$ by \myref{thrm-structure-of-finite-field}, and likewise the additive group of $\GF{p}$ is isomorphic to $\Z_p$. One sees that $\Z_p^n$ is a vector space over $\Z_p$ with the standard basis
    \[
        \left\{(\underbrace{1, 0, 0, \dots, 0, 0}_{n \text{ elements}}), (\underbrace{0, 1, 0, \dots, 0, 0}_{n \text{ elements}}), (\underbrace{0, 0, 1, \dots, 0, 0}_{n \text{ elements}}), \dots, (\underbrace{0, 0, 0, \dots, 1}_{n \text{ elements}})\right\}
    \]
    and so $\dim{\GF{p^n}} = n$ over $\GF{p}$, i.e. $[\GF{p^n}: \GF{p}] = n$.
\end{proof}

\begin{corollary}\label{corollary-generator-of-finite-field-is-algebraic-over-subfield}
    Let $\alpha$ be a generator of $\GF{p^n}^\ast$. Then $\alpha$ is algebraic over $\GF{p}$. Furthermore $\alpha$ has degree $n$ over $\GF{p}$.
\end{corollary}
\begin{proof}
    Since $\alpha$ is a generator of $\GF{p^n}^\ast$, thus the extension field $\GF{p}(\alpha)$ is simply $\GF{p^n}$. Therefore one sees
    \[
        [\GF{p}(\alpha): \GF{p}] = [\GF{p^n}:\GF{p}] = n
    \]
    by \myref{corollary-degree-of-finite-field-to-prime-power} and so $\GF{p}(\alpha)$ is a finite extension over $\GF{p}$. Hence $\GF{p}(\alpha)$ is an algebraic extension over $\GF{p}$ (\myref{thrm-finite-extension-is-algebraic}) and so $\alpha$ is algebraic over $\GF{p}$ with degree $n$.
\end{proof}

Let us examine some finite fields in detail.
\begin{example}\label{example-GF16-analysis}
    Let us examine $\GF{16}$ in detail. In part of \myref{example-x^4+x+1-is-irreducible-over-Z2} we have shown that $x^4 + x + 1$ is irreducible over $\Z_2$. Therefore we know that $\princ{x^4 + x + 1}$ is a maximal ideal in $\Z_2[x]$ (\myref{thrm-irreducible-iff-principal-ideal-is-maximal}) and therefore $\Z_2/\princ{x^4+x+1}$ is a field (\myref{thrm-maximal-ideal-iff-quotient-ring-is-field}). In fact this is a field of order 16, which means that
    \[
        \GF{16} \cong \Z_2/\princ{x^4+x+1}.
    \]
    So we may think of $\GF{16}$ as the set
    \[
        F = \{a_0 + a_1\alpha + a_2\alpha^2 + a_3\alpha^3 \vert a_i \in \Z_2 \text{ and } \alpha^4 + \alpha + 1 = 0\},
    \]
    remembering that $\alpha^4 + \alpha + 1 = 0$. We add and multiply elements of $\GF{16}$ exactly as we add and multiply polynomials. We note the following conversions from the `multiplicative' form to the `additive form'.
    \begin{multicols}{2}
        \begin{itemize}
            \item $\alpha^1 = \alpha$
            \item $\alpha^2 = \alpha^2$
            \item $\alpha^3 = \alpha^3$
            \item $\alpha^4 = \alpha + 1$
            \item $\alpha^5 = \alpha^2 + \alpha$
            \item $\alpha^6 = \alpha^3 + \alpha^2$
            \item $\alpha^7 = \alpha^3 + \alpha + 1$
            \item $\alpha^8 = \alpha^2 + 1$
            \item $\alpha^9 = \alpha^3 + \alpha$
            \item $\alpha^{10} = \alpha^2 + \alpha + 1$
            \item $\alpha^{11} = \alpha^3 + \alpha^2 + \alpha$
            \item $\alpha^{12} = \alpha^3 + \alpha^2 + \alpha + 1$
            \item $\alpha^{13} = \alpha^3 + \alpha^2 + 1$
            \item $\alpha^{14} = \alpha^3 + 1$
            \item $\alpha^{15} = 1$
        \end{itemize}
    \end{multicols}
    From this, we infer that $\alpha$ is actually a generator of $F^\ast$, since $|F^\ast| = 15 = |\alpha|$. In fact, $\alpha$ is a zero of the polynomial $x^4 + x + 1$. However, do not be fooled into thinking that a zero of any irreducible polynomial is a generator of the multiplicative group of $\GF{p^n}$ in general. We will examine this more closely in \myref{problem-zero-of-irreducible-in-finite-field-is-not-generator} (later).

    Let us try our hand at doing some calculations in $\GF{16}$. For example, we see that
    \begin{align*}
        \alpha^{10} + \alpha^7 &= (\alpha^2 + \alpha + 1) + (\alpha^3 + \alpha + 1)\\
        &= \alpha^3 + \alpha^2 + 2\alpha + 2\\
        &= \alpha^3 + \alpha^2 & (\text{Evaluate coefficients in }\Z_2)\\
        &= \alpha^6.
    \end{align*}
    Also, we see
    \begin{align*}
        (\alpha^3 + \alpha^2 + 1)(\alpha^3 + \alpha^2 + \alpha + 1) &= \alpha^{13}\alpha^{12}\\
        &= \alpha^{25}\\
        &= \alpha^{15}\alpha^{10}\\
        &= \alpha^{10} & (\text{since }\alpha^{15} = 1)\\
        &= \alpha^2 + \alpha + 1.
    \end{align*}
\end{example}

\begin{example}\label{example-GF8-analysis}
    One sees clearly that $x^3 + x^2 + 1$ has no zeroes in $\Z_2$ and so $x^3 + x^2 + 1$ is irreducible over $\Z_2$ (\myref{thrm-degree-2-or-3-irreducible-iff-has-no-zeroes}). Therefore one sees $K = \Z_2/\princ{x^3+x^2+1}$ is a field of 8 elements, i.e. $\GF{8} \cong K$. Note that $f(x)$ has a zero in $K$ by \myref{thrm-fundamental-theorem-of-field-theory}, say $\alpha$. Note that $\GF{8}^\ast$ has order 7, a prime. Since $\alpha \neq 1$, therefore $|\alpha|_\times = 7$ (since every non-identity element in a group of prime order $p$ must have order $p$). Hence we may think of $\GF{8}$ as being the set
    \begin{align*}
        F &= \{0, 1, \alpha, \alpha^2, \alpha^3, \alpha^4, \alpha^5, \alpha^6\}\\
        &= \{0, 1, \alpha, \alpha^2, \alpha^2 + 1, \alpha^2 + \alpha + 1, \alpha + 1, \alpha^2 + \alpha\}\\
        &= \{0, 1, \alpha, \alpha + 1, \alpha^2, \alpha^2 + 1, \alpha^2 + \alpha, \alpha^2 + \alpha + 1\}
    \end{align*}
    where
    \begin{multicols}{2}
        \begin{itemize}
            \item $\alpha^1 = \alpha$;
            \item $\alpha^2 = \alpha^2$;
            \item $\alpha^3 = \alpha^2 + 1$;
            \item $\alpha^4 = \alpha^2 + \alpha + 1$;
            \item $\alpha^5 = \alpha + 1$;
            \item $\alpha^6 = \alpha^2 + \alpha$; and
            \item $\alpha^7 = 1$.
        \end{itemize}
    \end{multicols}

    Let us find the other zeroes of $f(x)$ in $F$. We see that
    \begin{align*}
        f(\alpha^2) &= (\alpha^2)^3 + (\alpha^2)^2 + 1\\
        &= \alpha^6 + \alpha^4 + 1\\
        &= (\alpha^2 + \alpha) + (\alpha^2 + \alpha + 1) + 1\\
        &= 2\alpha^2 + 2\alpha + 2\\
        &= 0
    \end{align*}
    so $\alpha^2$ is another zero of $f(x)$. Next, we try $\alpha^3$ and we observe
    \begin{align*}
        f(\alpha^3) &= (\alpha^3)^3 + (\alpha^3)^2 + 1\\
        &= \alpha^9 + \alpha^6 + 1\\
        &= \alpha^2 + \alpha^6 + 1\\
        &= \alpha^2 + (\alpha^2 + \alpha) + 1\\
        &= 2\alpha^2 + \alpha + 1\\
        &\neq 0
    \end{align*}
    so $\alpha^3$ is not a zero of $f(x)$. Now we check $\alpha^4$ and note
    \begin{align*}
        f(\alpha^4) &= (\alpha^4)^3 + (\alpha^4)^2 + 1\\
        &= \alpha^{12} + \alpha^8 + 1\\
        &= \alpha^5 + \alpha + 1\\
        &= (\alpha + 1) + \alpha + 1\\
        &= 2\alpha + 2\\
        &= 0
    \end{align*}
    so $\alpha^4$ is another zero of $f(x)$ in $F$. Since $f(x)$ has degree 3, this means that $f(x)$ has at most 3 zeroes over $F$. Thus we have found all zeroes of $f(x)$.

    Therefore we may write
    \[
        f(x) = (x - \alpha)(x - \alpha^2)(x - \alpha^4) = (x + \alpha)(x + \alpha^2)(x + \alpha^4).
    \]
\end{example}

\begin{exercise}
    Find all the zeroes of $f(x) = x^4 + x + 1$ over the field $F$ described in \myref{example-GF16-analysis}, using $\alpha$ as one possible zero of $f(x)$ in $F$.
\end{exercise}

\section{Subfields of Finite Fields}
\myref{thrm-finite-field-is-unique} completely describes of all finite fields of prime-power order. What about the subfields of a finite field? The following theorem describes the subfields of a given finite field.

\begin{theorem}\label{thrm-subfields-of-finite-field}
    For each divisor $m$ of $n$, the finite field $\GF{p^n}$ has a unique subfield of order $p^m$. Furthermore these are the only subfields of $\GF{p^n}$.
\end{theorem}
\begin{proof}[Proof (see {\cite[Theorem 22.3]{gallian_2016}})]
    To show the existence part of the theorem, suppose $m$ divides $n$. Then because
    \[
        p^n - 1 = (p^m-1)(p^{n-m} + p^{n-2m} + \cdots + p^m + 1)
    \]
    thus $p^m - 1$ divides $p^n - 1$, say $p^n - 1 = t(p^m - 1)$ for some positive integer $t$. Let $K = \{x \in \GF{p^n} \vert x^{p^m} = x\}$. Then \myref{exercise-prime-power-elements-idempotent-is-subfield} (later) shows that $K$ is a subfield of $\GF{p^n}$. Since $x^{p^m} - x$ has at most $p^m$ zeroes in $\GF{p^n}$ we must have $|K| \leq p^m$. Also, earlier in \myref{thrm-structure-of-finite-field}, we have shown that $\GF{p^n}^\ast$ is a cyclic group. So let $a \in \GF{p^n}^\ast$ be such that $\princ{a} = \GF{p^n}^\ast$. Then, in $\GF{p^n}^\ast$, we see
    \begin{align*}
        |a^t| &= \frac{|a|}{\gcd(|a|, t)} & (\myref{thrm-order-of-power-of-element})\\
        &= \frac{p^n - 1}{t}\\
        &= p^m - 1.
    \end{align*}
    Thus $(a^t)^{p^m-1} = 1$ and so $(a^t)^{p^m} = a^t$ which means $a^t \in K$. So $|K| \geq p^m - 1$. But as the order of a subgroup must divide the order of the group, thus the order of additive group of $K$ must divide the order of the additive group of $\GF{p^n}$; thus $|K|$ must be a prime power and therefore $|K| = p^m$. Hence $K$ is a subfield of $\GF{p^n}$ of order $p^m$.

    The uniqueness part of the theorem follows quickly from the observation that if $\GF{p^n}$ has more than one distinct subfield of order $p^m$, then the polynomial $x^{p^m} - x$ would have more than $p^m$ zeroes in $\GF{p^n}$, contradicting the fact that a polynomial of degree $p^m$ has exactly $p^m$ zeroes in $\GF{p^n}$ (\myref{thrm-polynomial-of-degree-n-has-at-most-n-zeroes}).

    Finally, suppose that $F$ is a subfield of $\GF{p^n}$. Then $F$ has prime-power order and, therefore, $F \cong \GF{p^m}$ by \myref{thrm-finite-field-is-unique} for some integer $m$. Hence by \myref{thrm-tower-law} we see
    \begin{align*}
        n &= [\GF{p^n}:\GF{p}]\\
        &= [\GF{p^n}:\GF{p^m}][\GF{p^m}:\GF{p}]\\
        &= [\GF{p^n}:\GF{p^m}]m
    \end{align*}
    which means $m$ divides $n$.
\end{proof}

\begin{corollary}\label{corollary-divisibility-of-finite-field-degree}
    If $m$ divides $n$ then $[\GF{p^n}: \GF{p^m}] = \frac nm$ for any prime $p$.
\end{corollary}
\begin{proof}
    See \myref{exercise-divisibility-of-finite-field-degree} (later).
\end{proof}

\begin{exercise}\label{exercise-prime-power-elements-idempotent-is-subfield}
    Let $m$ be a positive integer. Prove that
    \[
        K = \{x \in \GF{p^n} \vert x^{p^m} = x\}
    \]
    is a subfield of $\GF{p^n}$.
\end{exercise}

\begin{exercise}\label{exercise-divisibility-of-finite-field-degree}
    Prove \myref{corollary-divisibility-of-finite-field-degree}.
\end{exercise}

With \myref{thrm-subfields-of-finite-field} finding all the subfields of a given finite field is a relatively painless process.

\begin{example}
    Let $F$ be the field of order 16 constructed in \myref{example-GF16-analysis}. Note that $16 = 2^4$ and 4 has exactly 3 divisors, namely 1, 2, and 4. Therefore the only subfields of $F$ must have orders of $2^1 = 2$, $2^2 = 4$, and $2^4 = 16$. Clearly the subfield of order 2 is the subfield $\{0, 1\}$, and the subfield of order 16 is the whole field itself. The challenge is to identify the subfield of order 4.

    To find this, one notes that the multiplicative group of this subfield must have order 3, which is cyclic. One also knows that we may find a cyclic subfield of order 3 within $F^\ast$, which is generated by $\alpha$. In particular, the multiplicative group of the subfield must be $\{1, \alpha^5, \alpha^{10}\}$, which means that the subfield of order 4 of $F$ is
    \[
        \{0, 1, \alpha^5, \alpha^{10}\} = \{0, 1, \alpha^2 + \alpha, \alpha^2 + \alpha + 1\}.
    \]
\end{example}

\begin{example}
    We build a lattice of subfields of $\GF{p^{24}}$. Note that $24 = 2^3 \times 3$, and so its factors are 1, 2, 3, 4, 6, 8, 12, and 24.

    \begin{figure}[H]
        \centering
        \pdfteximg{0.25\textwidth}{part3/images/finite-fields/GFp24.pdf_tex}
        \caption{Subfield lattice for $\GF{p^{24}}$}
    \end{figure}
\end{example}

\begin{example}
    If $F$ is a field of order $3^6 = 729$ and $\alpha$ is a generator of $F^\ast$, then the possible subfields of $F$ are of order $3^1 = 3$, $3^2 = 9$, $3^3 = 27$, and $3^6 = 729$. In particular,
    \begin{itemize}
        \item the subfield of order 3 is $\{0\} \cup \left\{\alpha^{\frac{729-1}{3-1}}\right\} = \{0\} \cup \{\alpha^{364}\} = \{0,1,2\}$;
        \item the subfield of order 9 is $\{0\} \cup \left\{\alpha^{\frac{729-1}{9-1}}\right\} = \{0\} \cup \{\alpha^{91}\}$;
        \item the subfield of order 27 is $\{0\} \cup \left\{\alpha^{\frac{729-1}{27-1}}\right\} = \{0\} \cup \{\alpha^{28}\}$; and
        \item the subfield of order 729 is $\{0\} \cup \left\{\alpha^{\frac{729-1}{729-1}}\right\} = \{0\} \cup \{\alpha\} = F$.
    \end{itemize}
\end{example}

\begin{exercise}
    Consider the finite field $F$ in \myref{example-GF8-analysis}. What are its subfields?
\end{exercise}

\newpage

\section{Problems}
\begin{problem}
    Is
    \[
        \Z_2[x]/\princ{x^3+x+1} \cong \Z_2[x]/\princ{x^3+x^2+1}?
    \]
    Justify your answer using a concise argument.
\end{problem}

\begin{problem}
    Let $\alpha$ be a zero of $x^3 + x^2 + 1$ in some extension field of $\Z_2$.
    \begin{partquestions}{\roman*}
        \item Find $(\alpha^2 + \alpha + 1)^{-1}$ in $\Z_2(\alpha)$.
        \item Hence solve the equation
        \[
            (\alpha^2 + \alpha + 1)x + \alpha^2 = \alpha
        \]
        for $x$ over $\Z_2(\alpha)$.
    \end{partquestions}
    (\textit{Hint: refer to the conversion formulae given in \myref{example-GF8-analysis}.})
\end{problem}

\begin{problem}
    Explain why every element of a finite field of characteristic $p$ can be written in the form $a^p$, where $a$ is an element of the same finite field.\newline
    (\textit{Hint: Consider the proof of \myref{thrm-finite-field-is-perfect}.})
\end{problem}

\begin{problem}
    Consider the irreducible polynomial $f(x) = x^2 + 2x + 2$ over $\Z_3$. Let $\alpha$ be a zero of $f(x)$ in some extension field of $\Z_3$. Find the other zero(es) of $f(x)$ in $\Z_3(\alpha)$.
\end{problem}

\begin{problem}\label{problem-zero-of-irreducible-in-finite-field-is-not-generator}
    Let $p(x) = x^3 + 2x + 2 \in \Z_3[x]$.
    \begin{partquestions}{\roman*}
        \item Show that $p(x)$ is irreducible over $\Z_3$.
        \item Let $\alpha$ be a zero of $p(x)$ in the field $F = \Z_3[x]/\princ{p(x)}$. Show that $\alpha$ is not a generator of $F^\ast$.
        \item Find a generator of $F^\ast$, and prove that your answer is indeed a generator of $F^\ast$. Express your answer in terms of $\alpha$.
    \end{partquestions}
\end{problem}

\begin{problem}
    Prove that any irreducible factor of $x^{32} - x$ over $\Z_2$ has a degree of 1 or 5.
\end{problem}

\begin{problem}
    Prove that every finite extension of a finite field is simple.
\end{problem}

\begin{problem}
    Prove that no finite field is algebraically closed.
\end{problem}

\begin{problem}
    Let $F$ be a field. We say that $a \in F$ is a \textit{square} if and only if there exists a $b \in F$ such that $a = b^2$. If $a \in F$ is not a square it is called a \textit{non-square}.

    Now let $p$ be an odd prime. Show that if any non-square in $\GF{p}$ is still a non-square in $\GF{p^n}$, then $n$ is an odd positive integer.\newline
    (\textit{Hint: consider the polynomial $f(x) = x^2-a$, where $a$ is a non-square.})
\end{problem}
