\section{Algebraic Extensions}
\subsection*{Exercises}
\begin{questions}
    \item It is not transcendental, i.e. it is algebraic. Let $\alpha = \sqrt{2-\sqrt{i}}$; one sees $\alpha^2 = 2 - \sqrt{i}$. Thus $\alpha^2 - 2 = -\sqrt{i}$ and so $(\alpha^2-2)^2 = i$. Note that $(\alpha^2 - 2)^2 = \alpha^4 - 4\alpha^2 + 4$ and so $\alpha^4 - 4\alpha^2 + 4 = i$. Squaring one more time we see
    \[
        x^8 - 8x^6 + 24x^4 - 32x^2 + 16 = i^2 = -1
    \]
    and so $\alpha$ is a zero of the polynomial $x^8 - 8x^6 + 24x^4 - 32x^2 + 17 \in \Q$. Therefore $\alpha$ is algebraic.

    \item We know by Tower Law (\myref{thrm-tower-law}) that
    \begin{align*}
        [\Q(\sqrt2, \sqrt[3]3): \Q] &= [\Q(\sqrt2, \sqrt[3]3): \Q(\sqrt2)][\Q(\sqrt2):\Q]\\
        &= [\Q(\sqrt2, \sqrt[3]3): \Q(\sqrt[3]3)][\Q(\sqrt[3]3):\Q].
    \end{align*}
    Since $[\Q(\sqrt2):\Q] = 2$ and $[\Q(\sqrt[3]3):\Q] = 3$, thus $[\Q(\sqrt2, \sqrt[3]3): \Q]$ is a multiple of $\lcm(2, 3) = 6$. On the other hand, note that $x^3 - 3 \in \Q(\sqrt2)[x]$ has a zero of $\sqrt[3]3$, so $[\Q(\sqrt2, \sqrt[3]3): \Q(\sqrt2)]$ is at most 3. Hence $[\Q(\sqrt2, \sqrt[3]3): \Q] = [\Q(\sqrt2, \sqrt[3]3): \Q(\sqrt2)][\Q(\sqrt2):\Q] \leq 3 \times 2 = 6$. Hence $[\Q(\sqrt2, \sqrt[3]3): \Q] = 6$.

    Now, note that $\sqrt[6]{72} = \sqrt2 \times \sqrt[3]3$, so $\sqrt[6]{72} \in \Q(\sqrt2, \sqrt[3]3)$. Consequently, we see that $\Q(\sqrt2, \sqrt[3]3, \sqrt[6]{72}) = \Q(\sqrt2, \sqrt[3]3)$ and so $[\Q(\sqrt2, \sqrt[3]3, \sqrt[6]{72}):\Q] = [\Q(\sqrt2, \sqrt[3]3): \Q] = 6$.

    \item Suppose otherwise, that $\sqrt2 \in \Q(\alpha)$. So $\Q \subset \Q(\sqrt2) \subseteq \Q(\alpha)$. Thus we see
    \[
        [\Q(\alpha): \Q] = [\Q(\alpha):\Q(sqrt2)][\Q(\sqrt2):\Q].
    \]
    But $[\Q(\alpha): \Q] = 5$ (since $\alpha$ is a zero of the irreducible polynomial $p(x)$) and $[\Q(\sqrt2):\Q] = 2$, so we see that 2 divides 5, a contradiction. Therefore $\sqrt2 \notin \Q(\alpha)$.

    \item \begin{partquestions}{\roman*}
        \item One clearly sees $\Q(\sqrt3 + \sqrt5) \subseteq \Q(\sqrt3, \sqrt5)$ as $\sqrt3 + \sqrt5 \in \Q(\sqrt3, \sqrt5)$. Now observe
        \begin{align*}
            \left(\sqrt3+\sqrt5\right)^{-1} &= \frac{\sqrt3-\sqrt5}{\left(\sqrt3+\sqrt5\right)\left(\sqrt3-\sqrt5\right)}\\
            &= -\frac12\left(\sqrt3-\sqrt5\right)\\
        \end{align*}
        so $\sqrt3-\sqrt5 \in \Q(\sqrt3 + \sqrt5)$. It follows then $\frac12\left(\sqrt3 + \sqrt5\right) + \frac12\left(\sqrt3 - \sqrt5\right) = \sqrt3$ and $\frac12\left(\sqrt3 + \sqrt5\right) - \frac12\left(\sqrt3 - \sqrt5\right) = \sqrt5$ and so $\sqrt3,\sqrt5\in\Q(\sqrt3+\sqrt5)$. Hence $\Q(\sqrt3,\sqrt5)\subseteq\Q(\sqrt3+\sqrt5)$, and therefore $\Q(\sqrt3, \sqrt5) = \Q(\sqrt3 + \sqrt5)$.

        \item We see
        \[
            [\Q(\sqrt3, \sqrt5): \Q] = \underbrace{[\Q(\sqrt3, \sqrt5): \Q(\sqrt3)]}_2\underbrace{[\Q(\sqrt3):\Q]}_2 = 4
        \]
        by Tower Law (\myref{thrm-tower-law}), which means that the minimal polynomial of $\sqrt3+\sqrt5$ has degree 4.
    \end{partquestions}

    \item For the forward direction, suppose $F$ is an algebraically closed field and let $E$ be an algebraic extension of $F$, which means $F \subseteq E$. Now let $\alpha \in E$. Since $F$ is algebraically closed, by \myref{corollary-field-is-algebraically-closed-iff-irreducible-polynomials-are-of-degree-1}, every irreducible polynomial is of degree 1. Consequently the minimal polynomial of $\alpha$ in $F$ is $x - \alpha$, which clearly means $\alpha \in F$. So $E \subseteq F$, meaning $F = E$.

    For the reverse direction, suppose a field $F$ has no proper algebraic extension. Suppose $f(x) \in F[x]$ is an irreducible polynomial. Then \myref{corollary-polynomial-quotient-by-principal-ideal-is-field-iff-polynomial-irreducible} tells us that $E = F[x]/\princ{f(x)}$ is an extension field of $F$ where $[E:F] = \deg f(x)$, which is finite and thus algebraic (\myref{thrm-finite-extension-is-algebraic}). However since $F$ has no proper algebraic extension, thus $E = F$, meaning $[E:F] = 1$, and thus $\deg f(x) = 1$. The proof is complete by \myref{corollary-field-is-algebraically-closed-iff-irreducible-polynomials-are-of-degree-1}.
\end{questions}

\subsection*{Problems}
\begin{questions}
    \item Note that each of the $2^{\frac1n}$ is algebraic since they are a zero of the polynomial $x^n - 2$ over $\Q$. Therefore $E$ is an algebraic extension.

    However $E$ is not finite, since by Tower Law (\myref{thrm-tower-law}) we have
    \begin{align*}
        [E: F] &= \cdots[\Q(2^{\frac12},2^{\frac13},2^{\frac14}):\Q(2^{\frac12},2^{\frac13})]
        [\Q(2^{\frac12},2^{\frac13}):\Q(2^{\frac12})][\Q(2^{\frac12}):\Q]\\
        &= \cdots \times 4 \times 3 \times 2
    \end{align*}
    which is clearly not finite.

    \item Earlier in \myref{exercise-Q-sqrt3-sqrt5} we have shown $\Q(\sqrt3 + \sqrt5) = \Q(\sqrt3,\sqrt5)$. In \myref{example-Q-sqrt3-sqrt5} we have also shown that $[\Q(\sqrt3,\sqrt5):\Q] = 4$. Furthermore we see $[\Q(\sqrt{15}):\Q] = 2$ and
    \[
        [\Q(\sqrt3,\sqrt5):\Q] = [\Q(\sqrt3,\sqrt5):\Q(\sqrt{15})][\Q(\sqrt{15}):\Q]
    \]
    by Tower Law (\myref{thrm-tower-law}). Therefore
    \[
        4 = [\Q(\sqrt3,\sqrt5):\Q(\sqrt{15})]\times2
    \]
    and so $[\Q(\sqrt3,\sqrt5):\Q(\sqrt{15})] = 2$.

    \begin{partquestions}{\alph*}
        \item Let $\alpha = \sqrt3 + \sqrt5$. Then $\alpha^2 = 8 + 2\sqrt{15}$, so we see that the minimal polynomial of $\sqrt3+\sqrt5$ over $\Q(\sqrt{15})$ is $x^2 - (8 + 2\sqrt{15})$.

        \item A basis for $\Q(\sqrt3 + \sqrt5)$ over $\Q(\sqrt{15})$ is $\{1, \sqrt3 + \sqrt5\}$, since we have shown that $(\sqrt3 + \sqrt5)^2 \in \Q(\sqrt{15})$ in \textbf{(a)}.
    \end{partquestions}

    \item We see $\sqrt[6]2 = (\sqrt2)(\sqrt[3]2)^{-1} \in \Q(\sqrt2, \sqrt[3]2)$ so $\Q(\sqrt[6]2) \subseteq \Q(\sqrt2,\sqrt[3]2)$. But also $\sqrt[3]2 = \left(\sqrt[6]2\right)^2$ and $\sqrt2 = \left(\sqrt[6]2\right)^3$, so $\sqrt2,\sqrt[3]2 \in \Q(\sqrt[6]2)$ and thus $\Q(\sqrt2,\sqrt[3]2) \subseteq \Q(\sqrt[6]2)$. Therefore $\Q(\sqrt2,\sqrt[3]2) = \Q(\sqrt[6]2)$.

    \item Note clearly $\Q(\sqrt a + \sqrt b) \subseteq \Q(\sqrt a, \sqrt b)$ since $\sqrt a + \sqrt b \in \Q(\sqrt a, \sqrt b)$.

    Note also
    \[
        (\sqrt a + \sqrt b)^{-1} = \frac{\sqrt a - \sqrt b}{a^2 - b^2} = \frac1{a^2-b^2}\left(\sqrt a - \sqrt b\right)
    \]
    so $\sqrt a - \sqrt b \in \Q(\sqrt a + \sqrt b)$. Consequently we see
    \begin{align*}
        \sqrt a &= \frac12(\sqrt a + \sqrt b) + \frac12(\sqrt a - \sqrt b) \in \Q(\sqrt a + \sqrt b)\\
        \sqrt b &= \frac12(\sqrt a + \sqrt b) - \frac12(\sqrt a - \sqrt b) \in \Q(\sqrt a + \sqrt b)
    \end{align*}
    and therefore $\Q(\sqrt a, \sqrt b) \subseteq \Q(\sqrt a + \sqrt b)$.

    Hence because $\Q(\sqrt a + \sqrt b) \subseteq \Q(\sqrt a, \sqrt b)$ and $\Q(\sqrt a, \sqrt b) \subseteq \Q(\sqrt a + \sqrt b)$ we have $\Q(\sqrt a, \sqrt b) = \Q(\sqrt a + \sqrt b)$ as required.

    \item Note that $F \subseteq F(\alpha) \subseteq E$, so by Tower Law (\myref{thrm-tower-law}) we have
    \[
        p = [E:F] = [E:F(\alpha)][F(\alpha):F].
    \]
    Since $p$ is a prime we must have either $[E:F(\alpha)] = 1$ or $[F(\alpha):F] = 1$, meaning $E \cong F(\alpha)$ or $F(\alpha) \cong F$ by \myref{prop-finite-extension-of-degree-1-means-extension-equals-base-field}. But as $F \subseteq F(\alpha) \subseteq E$ we see that $F(\alpha) = E$ or $F(\alpha) = F$.

    \item Seeking a contradiction, suppose $\alpha^n$ is algebraic. Thus $\alpha^n$ is a zero of some polynomial $f(x) \in F[x]$, i.e. $f(\alpha^n) = 0$. Consequently $\alpha$ is a zero of the polynomial $g(x) \in F[x]$ where $g(x) = f(x^n)$, which implies that $\alpha$ is algebraic over $F$, a contradiction. Therefore $\alpha^n$ is transcendental for all positive integers $n$.

    \item \begin{partquestions}{\roman*}
        \item Let $\alpha \in E$ be a zero of $f(x)$. We know that $F(a_0, a_1, \dots, a_n)$ is a finite extension over $F$. Furthermore we know that $\alpha$ is algebraic over $F(a_0, a_1, \dots, a_n)$ since $\alpha$ is a zero of $f(x) \in F(a_0, a_1, \dots, a_n)[x]$. Therefore we see
        \[
            [F(a_0, \dots, a_n, \alpha):F] = \underbrace{[F(a_0, \dots, a_n, \alpha):F(a_0, \dots, a_n)]}_{\leq n}\underbrace{[F(a_0, \dots, a_n):F]}_{\text{Finite}}
        \]
        by Tower Law (\myref{thrm-tower-law}), and so $[F(a_0, a_1, \dots, a_n, \alpha):F]$ is an algebraic extension over $F$ (\myref{thrm-finite-extension-is-algebraic}). Hence $\alpha$ is algebraic over $F$.

        \item Suppose not, that both $\alpha + \beta$ and $\alpha\beta$ are algebraic. Then note that
        \[
            x^2 - (\alpha+\beta)x + \alpha\beta
        \]
        is a polynomial of algebraic coefficients over $F$ with zeroes of $\alpha$ and $\beta$, which means that $\alpha$ and $\beta$ are algebraic over $F$, a contradiction. Therefore at least one of $\alpha + \beta$ and $\alpha\beta$ is transcendental over $F$.
    \end{partquestions}

    \item Clearly $\Q(\alpha) \subseteq \Q(\sqrt\alpha)$ since $\alpha = (\sqrt\alpha)^2 \in \Q(\sqrt\alpha)$.

    For brevity let $\omega = \sqrt\alpha$. Note that $\omega$ is a zero of the polynomial $x^4+x^2+1$. But we may factor that polynomial since
    \begin{align*}
        x^4 + x^2 + 1 &= (x^4 + 2x^2 + 1) - x^2\\
        &= (x^2+1)^2 - x^2\\
        &= (x^2+x+1)(x^2-x+1)
    \end{align*}
    and so we see that
    \[
        (\omega^2+\omega+1)(\omega^2-\omega+1) = 0.
    \]
    THerefore we have $\omega^2+\omega+1 = 0$ or $\omega^2-\omega+1 = 0$, which one may rearrange appropriately to yield $\omega = -\omega^2-1$ or $\omega = \omega^2 + 1$, i.e. $\sqrt\alpha = \pm(\alpha+1)$. Therefore we see $\sqrt\alpha \in \Q(\alpha)$, meaning $\Q(\sqrt\alpha) \subseteq \Q(\alpha)$.

    Since $\Q(\alpha) \subseteq \Q(\sqrt\alpha)$ and $\Q(\sqrt\alpha) \subseteq \Q(\alpha)$ thus $\Q(\alpha) = \Q(\sqrt\alpha)$.

    \item From \myref{thrm-characterisation-of-extensions} we know that an element of $\Q(\pi)$ takes the form
    \[
        \frac{a_0 + a_1\pi + a_2\pi^2 + \cdots + a_m\pi^m}{b_0 + b_1\pi + b_2\pi^2 + \cdots + b_n\pi^n}
    \]
    where each $a_i, b_j \in \Q$.

    By way of contradiction suppose $\sqrt2 \in \Q(\pi)$, meaning
    \[
        \sqrt2 = \frac{a_0 + a_1\pi + a_2\pi^2 + \cdots + a_m\pi^m}{b_0 + b_1\pi + b_2\pi^2 + \cdots + b_n\pi^n}
    \]
    for some $a_i, b_j \in \Q$. Rearranging we see
    \[
        \left(a_0 + a_1\pi + a_2\pi^2 + \cdots + a_m\pi^m\right)^2 = 2\left(b_0 + b_1\pi + b_2\pi^2 + \cdots + b_n\pi^n\right)^2.
    \]
    Equating the leading term on both sides we see $a_m^2\pi^{2m} = 2b_n^2\pi^{2n}$. Thus $m = n$ and $a_m^2 = 2b_n^2$ which means $\sqrt2 = \frac{a_m}{b_n}$, a rational number, which is a contradiction. Therefore $\sqrt2 \notin \Q(\pi)$.

    \item \begin{partquestions}{\roman*}
        \item Note that ${a+b\choose a}$ is a positive integer. Also we have ${a+b\choose a} = \frac{(a+b)!}{a!b!}$ by definition of the binomial coefficient. Therefore we see that $a!b!$ divides $(a+b)!$ for all non-negative integers $a$ and $b$.

        \item We use strong induction on $n$.

        When $n = 1$, then the splitting field of $f(x)$ over $F$ is just $F$, so $[E:F] = [F:F] = 1$ which clearly divides $1! = 1$.

        Assume that the claim holds for all polynomials $g(x) \in F[x]$ of degree of at most $k$ over all fields $F'$, for some positive integer $k$. We show that the claim holds for a polynomial of degree $k+1$ over $F$.

        Let $f(x)$ be such a polynomial. We split into two cases.
        \begin{itemize}
            \item The first case is if $f(x)$ is irreducible over $F$. Let $E$ be the splitting field of $f(x)$ over $F$. Let $\alpha \in E$ be a zero of $f(x)$. Then by Factor Theorem (\myref{corollary-factor-theorem}) we may write $f(x) = (x-\alpha)g(x)$ where $g(x) \in F(\alpha)[x]$ has degree $k$. Note $E$ is a splitting field of $g(x)$ over $F(\alpha)$, so by Induction Hypothesis we know that $[E:F(\alpha)]$ divides $k!$. As $[F(\alpha):F] = k+1$ (since $f(x)$ is an irreducible polynomial of degree $k+1$ with a zero of $\alpha$), it follows from Tower Law (\myref{thrm-tower-law}) that $[E:F]$ divides $(k+1)k! = (k+1)!$.

            \item The second case is if $f(x)$ is reducible over $F$. Write $f(x) = g(x)h(x)$ where $g(x), h(x) \in F[x]$. Let $a = \deg g(x)$ and $b = \deg h(x)$. Let $K \subseteq E$ be the splitting field of $g(x)$ over $F$. Then $[K:F]$ divides $a!$ by Induction Hypothesis. Also we must have $E$ being the splitting field of $h(x) \in K[x]$, meaning $[E:K]$ divides $b!$. By Tower Law we see that $[E:F] = [E:K][K:F] = a!b!$ which divides $(a+b)! = (k+1)!$.
        \end{itemize}
        Therefore the statement holds for any polynomial of degree $k+1$ over $F$, proving this claim by induction.
    \end{partquestions}
\end{questions}
