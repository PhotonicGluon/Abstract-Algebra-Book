\section{Extension Fields and Splitting Fields}
\subsection*{Exercises}
\begin{questions}
    \item We need to prove that $\phi$ is a well-defined isomorphism.
    \begin{itemize}
        \item \textbf{Well-defined}: Suppose $ax + b + \princ{x^2+1}, cx + d + \princ{x^2+1} \in F$ are such that $ax + b + \princ{x^2+1} = cx + d + \princ{x^2+1}$. Then $(a - c)x + (b - d) + \princ{x^2 + 1} = 0 + \princ{x^2+1}$ by Coset Equality, which therefore means that $(a-c)x + (b-d) \in \princ{x^2+1}$. Now elements of $\princ{x^2+1}$ have degree of at least 2, or is the zero polynomial. Since $(a-c)x + (b-d)$ has degree less than 2, we must have $(a-c)x + (b-d) = 0$ which means $a - c = 0$ and $b - d = 0$. Hence $a = c$ and $b = d$, and so
        \begin{align*}
            \phi\left(ax + b + \princ{x^2+1}\right) &= b + ai\\
            &= d + ci\\
            &= \phi\left(cx + d + \princ{x^2+1}\right)
        \end{align*}
        which proves that $\phi$ is well-defined.

        \item \textbf{Homomorphism}. Let $ax + b + \princ{x^2+1}, cx + d + \princ{x^2+1} \in F$. Then
        \begin{align*}
            &\phi\left(\left(ax + b + \princ{x^2+1}\right) + \left(cx + d + \princ{x^2+1}\right)\right)\\
            &= \phi\left((a+c)x + (b+d) + \princ{x^2+1}\right)\\
            &= (b+d) + (a+c)i\\
            &= (b+ai) + (d+ci)\\
            &= \phi\left(ax + b + \princ{x^2+1}\right) + \phi\left(cx + d + \princ{x^2+1}\right)
        \end{align*}
        and
        \begin{align*}
            &\phi\left(\left(ax + b + \princ{x^2+1}\right)\left(cx + d + \princ{x^2+1}\right)\right)\\
            &= \phi\left((acx^2 + (ad + bc)x + bd) + \princ{x^2+1}\right)\\
            &= \phi\left(((ad + bc)x + (bd - ac)) + \princ{x^2+1}\right) & (\text{since } x^2 = -1 \text{ in } F)\\
            &= (bd - ac) + (ad + bc)i\\
            &= (b + ai)(d + ci)\\
            &= \phi\left(ax + b + \princ{x^2+1}\right)\phi\left(cx + d + \princ{x^2+1}\right)
        \end{align*}
        which shows that $\phi$ is a ring homomorphism.

        \item \textbf{Injective}: Suppose $ax + b + \princ{x^2+1}, cx + d + \princ{x^2+1} \in F$ such that
        \[
            \phi\left(ax + b + \princ{x^2+1}\right) = \phi\left(cx + d + \princ{x^2+1}\right).
        \]
        This means that $b + ai = d + ci$. Hence $b = d$ and $a = c$, which shows that $ax + b + \princ{x^2+1} = cx + d + \princ{x^2+1}$. Therefore $\phi$ is injective.

        \item \textbf{Surjective}: Suppose $u + vi \in \C$. Then clearly
        \[
            \phi\left(vx + u + \princ{x^2+1}\right) = u + vi
        \]
        which means that $\phi$ is surjective.
    \end{itemize}
    Therefore $\phi$ is a well-defined isomorphism, proving that $F \cong \C$.

    \item By way of contradiction, suppose a ring $R$ has an subring that is isomorphic to $\Z_4$, and that $f(x)$ has a zero in such a ring. Let this zero be denoted $\alpha$, which means $2\alpha + 1 = 0$ in $R$. Note that this means
    \begin{align*}
        0 &= 2\times0\\
        &= 2(2\alpha + 1)\\
        &= 2(2\alpha) + 2\\
        &= (2 \times 2)\alpha + 2\\
        &= 0\alpha + 2\\
        &= 2
    \end{align*}
    but $0 \neq 2$ in $\Z_4$.

    \item We note that $x^4 + 3x^2 + 1 = (x^2+1)(x^2+2)$, and that $x^2+1$ and $x^2 + 2$ are irreducible over $\Q$. By the Fundamental Theorem of Field Theory (\myref{thrm-fundamental-theorem-of-field-theory}) we may choose
    \[
        \Q[x]/\princ{x^2+1} \quad\text{or}\quad \Q[x]/\princ{x^2+2}
    \]
    as the possible extension fields.

    \item For brevity let $K = F(a_1, a_2, \dots, a_{n-1})(a_n)$. Note $K$ contains $F(a_1, a_2, \dots, a_{n-1})$ and the element $a_n$, and that $F(a_1, a_2, \dots, a_{n-1})$ contains the field $F$ and the elements $a_1, a_2, \dots, a_{n-1}$. Thus $K$ contains $F$ and the elements $a_1, a_2, \dots, a_{n-1}, a_n$, which means that $K$ is included in the intersection that generates $F(a_1, a_2, \dots, a_n)$, so $K \subseteq F(a_1, a_2, \dots, a_n)$. But $F(a_1, a_2, \dots, a_n)$ is the smallest subfield of $E$ that contains $F$ and the elements $a_1, a_2, \dots, a_n$. Therefore $K = F(a_1, a_2, \dots, a_n)$ as required.

    \item \begin{partquestions}{\roman*}
        \item One sees that
        \[
            f(x) = (x-\sqrt2)(x+\sqrt2)(x-\sqrt3)(x+\sqrt3)
        \]
        and so a splitting field of $f(x)$ over $\Q$ is
        \begin{align*}
            &\Q(\sqrt2, -\sqrt2, \sqrt3, -\sqrt3)\\
            &=\Q(\sqrt2, \sqrt3)\\
            &=\Q(\sqrt2)(\sqrt3).
        \end{align*}

        \item One sees that
        \begin{align*}
            &p(\sqrt2 + \sqrt3)\\
            &= (\sqrt2 + \sqrt3)^4 - 10(\sqrt2 + \sqrt3)^2 + 1\\
            &= \left((\sqrt2)^4 + 4(\sqrt2)^3(\sqrt3) + 6(\sqrt2)^2(\sqrt3)^2 + 4(\sqrt2)(\sqrt3)^3 + (\sqrt3)^4\right)\\
            &\quad\quad- 10\left((\sqrt2)^2 + 2\sqrt2\sqrt3 + (\sqrt3)^2\right) + 1\\
            &=\left(49 + 20\sqrt6\right) - 10\left(5 + 2\sqrt6\right) + 1\\
            &= 0
        \end{align*}
        so $\sqrt2 + \sqrt3$ is indeed a zero of $p(x)$.

        \item We work backwards. Since $p(x) = x^4 - 10x + 1$ is an irreducible polynomial of degree 4, therefore $\Q(\sqrt2 + \sqrt3)$ has 4 basis vectors. Therefore
        \begin{align*}
            &\Q(\sqrt2 + \sqrt3)\\
            &= \{s + t(\sqrt2 + \sqrt3) + u(\sqrt2 + \sqrt3)^2 + v(\sqrt2 + \sqrt3)^3 \vert s,t,u,v \in \Q\}\\
            &= \{(s + 5u) + (t + 11v)\sqrt2 + (t + 9v)\sqrt3 + 2u\sqrt6 \vert s, t, u, v \in \Q\}\\
            &= \{a + b\sqrt2 + c\sqrt3 + d\sqrt6 \vert a, b, c, d \in \Q\}\\
            &=\{(a+b\sqrt2) + (c+d\sqrt2)\sqrt3 \vert a, b, c, d \in \Q\}\\
            &=\{u + v\sqrt3 \vert u, v \in \Q(\sqrt2)\}\\
            &= \Q(\sqrt2)(\sqrt3)
        \end{align*}
        and so $\Q(\sqrt2 + \sqrt3)$ is a splitting field of $f(x)$ over $\Q$.
    \end{partquestions}

    \item Let $f(x) = a_0 + a_1x + \cdots + a_mx^m$ and $g(x) = b_0 + b_1x + \cdots + b_nx^n$. Without loss of generality assume $m \geq n$, and set $b_i = 0$ for all $i > n$. Thus
    \begin{align*}
        (f(x) + g(x))' &= \left(\sum_{i=0}^m(a_i + b_i)x^i\right)'\\
        &= \sum_{i=1}^m i(a_i+b_i)x^{i-1}\\
        &= \sum_{i=1}^m ia_ix^{i-1} + \sum_{i=1}^m ib_ix^{i-1}\\
        &= \sum_{i=1}^m ia_ix^{i-1} + \sum_{i=1}^n ib_ix^{i-1} & (\text{since } b_i = 0 \text{ for } i > n)\\
        &= f'(x) + g'(x)
    \end{align*}
    as required.

    \item We induct on $n$.

    When $n = 1$ clearly $((f(x))^1)' = (f(x))' = f'(x) = 1(f(x))^{1-1}f'(x)$.

    Assume $((f(x))^k)' = k(f(x))^{k-1}f'(x)$ for some positive integer $k$. We show that the $k+1$ case also works.

    Note
    \begin{align*}
        ((f(x))^{k+1})' &= ((f(x))(f(x))^k)'\\
        &= f(x)((f(x))^k)' + f'(x)(f(x))^k\\
        &= f(x)(k(f(x))^{k-1}f'(x)) + f'(x)(f(x))^k & (\text{Induction Hypothesis})\\
        &= k(f(x))^kf'(x) + (f(x))^kf'(x)\\
        &= (k+1)(f(x))^kf'(x)
    \end{align*}
    so the $k+1$ case also holds.

    Therefore by mathematical induction we have $((f(x))^n)' = n(f(x))^{n-1}f'(x)$ for all positive integers $n$.

    \item If $f'(x) = 0$, we see
    \[
        a_1 + 2a_2x + 3a_3x^2 + \cdots + na_nx^{n-1} = 0.
    \]
    This means that all coefficients of $f'(x)$ must be zero; hence $ra_r = 0$ for all $r \in \{1, 2, \dots, n\}$.

    \item We prove this by induction on $n$. The base case of $n = 1$ was already proven. Assume that the statement holds for some positive integer $k$, i.e. $(x+y)^{p^k} = x^{p^k} + y^{p^k}$. Note that
    \begin{align*}
        (x+y)^{p^{k+1}} &= \left((x+y)^{p^k}\right)^p\\
        &= (x^{p^k}+y^{p^k})^p & (\text{By Induction Hypothesis})\\
        &= \left(x^{p^k}\right)^p + \left(y^{p^k}\right)^p & (\text{By Freshman's Dream}, \myref{prop-freshman-dream})\\
        &= x^{p^{k+1}} + y^{p^{k+1}}
    \end{align*}
    so the statement holds for $k + 1$. Thus by mathematical induction we have proven that $(x+y)^{p^n} = x^{p^n} + y^{p^n}$.
\end{questions}

\subsection*{Problems}
\begin{questions}
    \item Note that since $p, q \in F$, thus $pa + q \in F(a)$. Hence $F(pa+q) \subseteq F(a)$. On the other hand, note that because $p \neq 0$ thus $p^{-1}$ exists, so
    \[
        a = p^{-1}(pa+q) - p^{-1}q
    \]
    and thus $a \in F(pa+q)$. Hence $F(a) \subseteq F(pa+q)$ also. Therefore $F(pa+q) = F(a)$.

    \item \begin{partquestions}{\alph*}
        \item One sees that $x^3-3x+2 = (x-1)^2(x+2)$, so $x^3-3x+2$ splits in $\Q$. Hence $\Q$ is the splitting field of $x^3-3x+2$ over $\Q$.

        \item We see $x^3 - 1 = (x-1)(x^2+x+1)$. Note $x^2 + x + 1$ has zeroes of $\frac{1\pm\sqrt{-3}}{2} = \frac12 \pm \frac12\sqrt{-3}$ by the quadratic formula. Thus the splitting field of $x^3 - 1$ over $\Q$ is
        \begin{align*}
            &\Q\left(1, \frac12 + \frac12\sqrt{-3}, \frac12 - \frac12\sqrt{-3}\right)\\
            &=\Q(1)\left(\frac12 + \frac12\sqrt{-3}\right)\left(\frac12 - \frac12\sqrt{-3}\right)\\
            &=\Q(\sqrt{-3})(\sqrt{-3}) & (\myref{problem-simple-extension-absorbs-field-elements})\\
            &=\Q(\sqrt{-3}).
        \end{align*}

        \item Given $x^4 + x^2 + 1 = (x^2 - x + 1)(x^2 + x + 1)$. The zeroes of $x^2 + x + 1$ are $\frac12 \pm \frac12\sqrt{-3}$; the zeroes of $x^2 - x + 1$ are $-\frac12 \pm \frac12\sqrt{-3}$. Therefore the splitting field of $x^4 + x^2 + 1$ over $\Q$ is
        \begin{align*}
            &\Q\left(\frac12 + \frac12\sqrt{-3}, \frac12 - \frac12\sqrt{-3}, -\frac12 + \frac12\sqrt{-3}, -\frac12 - \frac12\sqrt{-3}\right)\\
            &=\Q\left(\frac12 + \frac12\sqrt{-3}\right)\left(\frac12 - \frac12\sqrt{-3}\right)\left(-\frac12 + \frac12\sqrt{-3}\right)\left(-\frac12 - \frac12\sqrt{-3}\right)\\
            &=\Q(\sqrt{-3})(\sqrt{-3})(\sqrt{-3})(\sqrt{-3})\\
            &= \Q(\sqrt{-3}).
        \end{align*}

        \item The zeroes of $x^2 + 2\sqrt2x + 3$ are $\sqrt2 \pm i$ by quadratic formula. Hence the splitting field of $x^2 + 2\sqrt2x + 3$ over $\Q(\sqrt2)$ is
        \begin{align*}
            \Q(\sqrt2)(\sqrt2+i, \sqrt2-i) &= \Q(\sqrt2)(\sqrt2 + i)(\sqrt2 - i)\\
            &= Q(\sqrt2, i).
        \end{align*}
    \end{partquestions}

    \item \begin{partquestions}{\alph*}
        \item Let $f(x) = x^4 + x + 7$. Note that $f(0) = 7 = 1 \neq 0$ and $f(1) = 1 + 1 + 7 = 9 = 1 \neq 0$, so $f(x)$ does not have a zero in $\Z_2$. Hence $f(x)$ does not have a multiple zero.

        \item Let $f(x) = x^{19} + x^8 + 1$, so $f'(x) = 19x^{18} + 8x^7$. In $\Z_3[x]$, $f'(x) = x^{18} + 2x^7$. Note that $f(1) = 1 + 1 + 1 = 3 = 0$ and $f'(1) = 1 + 2 = 3 = 0$, so both $f(x)$ and $f'(x)$ share a common factor of positive degree of $x-1$ by Factor Theorem (\myref{corollary-factor-theorem}). Hence 1 is a multiple zero by \myref{thrm-criterion-for-multiple-zeroes}.

        \item Let $f(x) = 2x^6 + x^4 + 2x^3 + 2$, so $f'(x) = 12x^5 + 4x^3 + 6x^2$. In $\Z_3[x]$, $f'(x) = x^3$. Note that $f(0) = 2 \neq 0$, $f(1) = 2 + 1 + 2 + 2 = 7 = 1 \neq 0$, but $f(2) = 2(2)^6 + 2^4 + 2(2)^3 + 2 = 162 = 0$, so the only zero of $f(x)$ is 2. But $f'(2) = 2^3 = 8 = 2 \neq 0$. Hence $f(x)$ and $f'(x)$ do not share a common factor of positive degree, meaning that $f(x)$ has no multiple zeroes (\myref{thrm-criterion-for-multiple-zeroes}).

        \item Let $f(x) = x^8 + 3x^5 + x^3 + 5$, so $f'(x) = 8x^7 + 15x^4 + 3x^2$. In $\Z_7[x]$, $f'(x) = x^7 + x^4 + 3x^2$. Note $f(4) = 68677 = 0$ and $f'(4) = 16688 = 0$, so $f(x)$ and $f'(x)$ share a common factor of positive degree of $x-4$ by Factor Theorem (\myref{corollary-factor-theorem}). Hence 4 is a multiple zero by \myref{thrm-criterion-for-multiple-zeroes}.
    \end{partquestions}

    \item \begin{partquestions}{\roman*}
        \item Prove. Since $p(x)$ is irreducible, thus $\princ{p(x)}$ is maximal (\myref{thrm-irreducible-iff-principal-ideal-is-maximal}). Therefore $\Z_2[x]/\princ{p(x)}$ is a field by \myref{thrm-maximal-ideal-iff-quotient-ring-is-field}.

        \item Prove. Since $\Z_2[x]/\princ{p(x)}$ is a field, thus $\left(\Z_2[x]/\princ{p(x)}\right)^\ast$ is in fact the multiplicative group of the field.

        \item Prove. Note that $\left|\Z_2[x]/\princ{p(x)}\right| = 2^7 = 128$, so $\left|\left(\Z_2[x]/\princ{p(x)}\right)^\ast\right| = 2^7 - 1 = 127$, since the multiplicative group omits the additive identity. One sees that 127 is prime. Hence every non-identity element of $\left(\Z_2[x]/\princ{p(x)}\right)^\ast$ is a generator (\myref{exercise-prime-order-element}).
    \end{partquestions}

    \item Suppose $\alpha_1, \alpha_2, \dots, \alpha_n$ are the zeroes of $f(x)$. Then $\alpha_1 - a, \alpha_2 - a, \dots, \alpha_n - a$ are the zeroes of $f(x+a)$. Thus the splitting field of $f(x+a)$ over $F$ is
    \begin{align*}
        F(\alpha_1 - a, \alpha_2 - a, \dots, \alpha_n - a) &= F(\alpha_1 - a)(\alpha_2 - a)\dots(\alpha_n - a)\\
        &= F(\alpha_1)(\alpha_2)\dots(\alpha_n) & (\myref{problem-simple-extension-absorbs-field-elements})\\
        &= F(\alpha_1, \alpha_2, \dots, \alpha_n)
    \end{align*}
    which is the splitting field of $f(x)$ over $F$.

    \item By definition of a simple extension, $\Q(\sqrt2)$ contains $\Q$. But $\Q$ is a prime field (\myref{thrm-Q-is-prime-field}), so it contains no smaller subfields. Therefore the only subfields of $\Q(\sqrt2)$ are $\Q$ and $\Q(\sqrt2)$.

    \item Let $f(x) = x^{p^n} - x$. Thus $f'(x) = p^nx^{p^n-1}-1$. Note that since $\Char{F} = p$, therefore $p^nx^{p^n-1} = 0$. Hence $f'(x) = -1$; clearly $f'(x)$ shares no common factor of positive degree with $f(x)$. Hence, there are no multiple zeroes of $f(x)$ (\myref{thrm-criterion-for-multiple-zeroes}), meaning that any zero of $f(x)$ must be simple.

    \item We consider two cases.

    If $f(x)$ has a zero in $F$, say $\alpha$, then that must mean $\alpha^p - a = 0$ by definition of a zero. Hence $a = \alpha^p$, meaning
    \[
        f(x) = x^p - \alpha^p = (x-\alpha)^p
    \]
    by the Freshman's Dream (\myref{prop-freshman-dream}). This clearly means that $f(x)$ splits over $F$.

    Now suppose $f(x)$ has no zeroes in $F$. Let $E/F$ be the splitting field of $f(x)$, and suppose $\beta \in E$ is a zero of $f(x)$. Using the above working we know that $f(x) = (x-\beta)^p$.

    Now, seeking a contradiction, suppose $f(x)$ is reducible over $F$, meaning that there exist non-constant polynomials $g(x), h(x) \in F[x]$ such that $f(x) = g(x)h(x)$. In particular, $g(x) = (x-\beta)^n$ for some positive integer $n$ where $1 \leq n \leq p$. Expanding $g(x)$ using the Binomial Theorem (\myref{thrm-binomial}) yields
    \[
        g(x) = x^n + n\beta x^{n-1} + {n\choose2}\beta^2x^{n-2} + \cdots + n\beta^{n-1}x + \beta^n \in F[x],
    \]
    and in particular this means that $\beta^n \in F$. Also, since $f(x) = (x-\beta)^p$, we draw a similar conclusion that $\beta^p \in F$.

    Now since $1 \leq n \leq p$ thus $n$ and $p$ are coprime, i.e. $\gcd(n,p) = 1$. By B\'ezout's lemma (\myref{lemma-bezout}) we are able to find integers $s$ and $t$ such that $sn + tp = 1$. In particular, by the closure of fields, we know $\beta^{sn} = \left(\beta^n\right)^s \in F$ and $\beta^{tp} = \left(\beta^p\right)^t \in F$, so
    \[
        \beta = \beta^1 = \beta^{sn+tp} = \beta^{sn}\beta^{tp} \in F.
    \]
    Thus $f(x)$ has a zero, namely $\beta$, in $F$. But this contradicts the fact that $f(x)$ has no zeroes in $F$. Therefore, $f(x)$ is irreducible over $F$.
\end{questions}
