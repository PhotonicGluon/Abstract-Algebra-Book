\setpartpreamble[u][\textwidth]{
    \quoteattr{
        I became convinced that studying the algebraic relationship of numbers is most conveniently based on a concept that is directly connected with the simplest arithmetic properties. I had originally used the term ``rational domain'', which I later changed to ``field''.
    }
    {
        Richard Dedekind, 1871
    }
    {
        \cite[p.~66]{kleiner_2007}
    }

    Fields are rings with more properties attached to them. Most high-school algebra operates under fields, so one would be very familiar with the absolute basics of working in fields. However, fields themselves have some unique properties, which we will explore in depth in this part.

    We start with the basics of fields, recapitulating the definition of a field and the properties that fields inherit from rings. We then further examine the analogous definition of subgroups and subrings for fields, which are subfields. We end by looking at the smallest possible subfield inside a field, called the prime subfield.

    Thereafter we look at a concept from linear algebra -- vector spaces. We look at what they are, spanning sets, linear independence, and bases. Then we move on to field extensions, the core of field theory. Unlike the other sections of abstract algebra, we can generate fields that are \textit{larger} than the original field. These extension fields are important as they allow us to create new zeroes of polynomials that may not exist within the base field. Splitting fields are a special kind of extension field, where polynomials can be fully factored into linear terms.

    Next, we look at algebraic extensions, an important kind of field extension where every element inside it is algebraic over the original base field. We look at algebraic (and transcendental) elements within an extension field and explore the properties of the degree of extensions. We then move on to an important category of fields, the finite fields. We explore how there is only one finite field of any given order (up to isomorphism) and look at the structure of finite fields. We also show how the subfields of finite fields are very restricted.

    To end this part, we look at geometric constructions and how field theory was used to prove the impossibility of the construction of certain geometric figures using a straightedge and compass alone. This final chapter will involve the introduction of basic geometric definitions and proofs, as well as trigonometry, but the payoff is well worth it.
}
\part{Fields}
