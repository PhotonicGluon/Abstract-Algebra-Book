\setpartpreamble[u][\textwidth]{
    \quoteattr{
        I became convinced that studying the algebraic relationship of numbers is most conveniently based on a concept that is directly connected with the simplest arithmetic properties. I had originally used the term ``rational domain'', which I later changed to ``field''.
    }
    {
        Richard Dedekind, 1871
    }
    {
        \cite[p.~66]{kleiner_2007}
    }

    Fields are rings with more properties attached to them. Most high-school algebra operates under fields, so one would be familiar with the basics of working in fields. However, fields have some unique properties, which we will explore in depth.

    We start with the basics of fields, recapitulating the definition of a field and the properties that fields inherit from rings. We then further examine the analogous definition of subgroups and subrings for fields, which are subfields. We look at the smallest possible subfield inside a field, called the prime subfield.

    After that, we look at a concept from linear algebra -- vector spaces. We look at some simple properties of vector spaces and then explore spanning sets, linear independence, and bases. Then, we move on to field extensions, the core of field theory. Unlike the other sections of abstract algebra, we can generate fields that are larger than the original field. These extension fields are essential as they allow us to create new zeroes of polynomials that may not exist within the base field. Splitting fields are a particular category of extension fields where one can fully factor polynomials into linear terms.

    Next, we look at algebraic extensions, an essential field extension where every element inside is algebraic over the original base field. We look at algebraic (and transcendental) elements within an extension field and explore the properties of the degree of extensions. We then move on to an essential category of fields -- the finite fields. We explore how there is only one finite field of any given order (up to isomorphism) and look at the structure of finite fields.

    To end this part, we look at geometric constructions and how field theory was used to prove the impossibility of constructing specific geometric figures using a straightedge and compass alone. This final chapter will introduce basic geometric definitions, proofs, and trigonometry, but the payoff is worth it.
}
\part{Fields}
