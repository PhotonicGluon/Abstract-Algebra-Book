\documentclass[
  a5paper,
  pagesize,
  10pt,
  bibtotoc,
  normalheadings,
  twoside,
  openany,
  chapterprefix,
  DIV=9
]{scrbook}

\usepackage[utf8]{inputenc}
\usepackage{tocloft}
\usepackage{mathtools}
\usepackage{amsfonts}
\usepackage{enumitem}
\usepackage{hyperref}
\usepackage{amsmath}
\usepackage{amsthm}
\usepackage{amssymb}
\usepackage[hmargin=2cm, vmargin=2.5cm]{geometry}
\usepackage{graphicx}
\usepackage{wrapfig}
\usepackage{parskip}
\usepackage{framed}
\usepackage{fancyhdr}
\usepackage{emptypage}

\usepackage[
    backend=bibtex,
    style=alphabetic,
    sorting=ynt
]{biblatex}

%=========== Path to images ==============
\graphicspath{{./images/}}

%============== Resources ================
\addbibresource{../AbstractAlgebra.bib}

%============ Redefinitions ==============
\let\oldemptyset\emptyset
\let\emptyset\varnothing

\let\totient\varphi

\renewcommand{\vert}{ \ | \ }

%======== Theorem-Like Things ============
\newtheoremstyle{exercise-style}
    {-5pt}       % Space above
    {\topsep}    % Space below
    {}           % Font to use in exercise
    {0pt}        % Measure of space to indent
    {\bfseries}  % Name of the head font
    {.}          % Punctuation between head and body
    { }          % Space after theorem head; " " = normal inter-word space
    {\thmname{#1}\thmnumber{ #2}\textnormal{\thmnote{ (#3)}}}

\newtheorem{theorem}{Theorem}[section]
\renewcommand{\thetheorem}{\Roman{part}.\arabic{chapter}.\arabic{section}.\arabic{theorem}}

\newtheorem{conjecture}[theorem]{Conjecture}
\newtheorem{proposition}[theorem]{Proposition}
\newtheorem{definition}[theorem]{Definition}
\newtheorem{lemma}[theorem]{Lemma}
\newtheorem{lemma-thrm}{Lemma}[theorem]
\newtheorem{corollary}[theorem]{Corollary}
\newtheorem{corollary-thrm}{Corollary}[theorem]
\theoremstyle{definition}\newtheorem*{remark}{Remark}
\theoremstyle{definition}\newtheorem{example}[theorem]{Example}

\theoremstyle{exercise-style}\newtheorem{exercisehidden}{Exercise}[chapter]
\renewcommand{\theexercisehidden}{\Roman{part}.\arabic{chapter}.\arabic{exercisehidden}}

\theoremstyle{definition}\newtheorem{problem}{Problem}[chapter]
\renewcommand{\theproblem}{\Roman{part}.\arabic{chapter}.\arabic{problem}}

%============ Environments ===============
\newenvironment{exercise}
{\begin{framed}\noindent\begin{exercisehidden}}
{\end{exercisehidden}\end{framed}}

%=========== Custom Commands =============
\newcommand{\code}[1]{\texttt{#1}}  % Code block
\makeatletter\newcommand*{\rom}[1]{\Ifstr{#1}{0}{0}{\expandafter\@slowromancap\romannumeral #1@}}\makeatother  % Roman numeral

\newcommand{\lcm}{\mathrm{lcm}}  % Lowest common multiple function
\newcommand{\sgn}{\mathrm{sgn}}  % Signum function

\newcommand{\im}{\mathrm{im}\;}  % Image of a function
\newcommand{\id}{\mathrm{id}}    % Identity function

\newcommand{\An}[1]{\mathrm{A}_{#1}}                  % Alternating group of degree n
\newcommand{\Aut}[1]{\mathrm{Aut}(#1)}                % Group of automorphisms of G
\newcommand{\C}[2]{\mathrm{C}_{#1}(#2)}               % Centralizer of an element in G
\newcommand{\Cl}[1]{\mathrm{Cl}(#1)}                  % Conjugacy class of the element x
\newcommand{\Cn}[1]{\mathrm{C}_{#1}}                  % Cyclic group of order n
\newcommand{\GL}[2]{\mathrm{GL}_{#1}\left(#2\right)}  % General Linear Group of degree n
\newcommand{\Inn}[1]{\mathrm{Inn}(#1)}                % Group of inner automorphisms of G
\newcommand{\N}[2]{\mathrm{N}_{#1}(#2)}               % Normalizer of S in G
\newcommand{\Out}[1]{\mathrm{Out}(#1)}                % Group of outer automorphisms of G
\newcommand{\SL}[2]{\mathrm{SL}_{#1}\left(#2\right)}  % Special Linear Group of degree n
\newcommand{\Sn}[1]{\mathrm{S}_{#1}}                  % Symmetric group of degree n
\newcommand{\Syl}[2]{\mathrm{Syl}_{#1}(#2)}           % Set of Sylow p-groups of G
\newcommand{\Sym}[1]{\mathrm{Sym}(#1)}                % Symmetric group of a set
\newcommand{\Un}[1]{\mathcal{U}_{#1}}                 % Group of units modulo n
\newcommand{\Z}[1]{\mathrm{Z}(#1)}                    % Center of a group G

\newcommand{\Stab}[2]{\mathrm{Stab}_{#1}(#2)}  % Stabilzer of x by G
\newcommand{\Fix}[2]{\mathrm{Fix}_{#1}(#2)}    % Set of all elements in X which is fixed by g
\newcommand{\Orb}[2]{\mathrm{Orb}_{#1}(#2)}    % Orbit of x under G

%============ Custom Header ==============
\fancypagestyle{plain}{\fancyhf{}\renewcommand{\headrulewidth}{0pt}} % To clear page numbers from footer, and header line at the start of every chapter

\pagestyle{fancy}
\fancyhf{}% Clear header/footer

\fancyhead[LE,RO]{\thepage}
\fancyhead[LO,RE]{\textit{\nouppercase\leftmark}}

%======== Custom Chapter Styling =========
\makeatletter
\renewcommand*{\chapterformat}{
  \MakeUppercase{\chapapp\nobreakspace\thechapter}
}

\renewcommand*{\chapterlineswithprefixformat}[3]{
    \Ifstr{#1}{chapter}{
        \vspace{-60px}
        \Ifstr{#2}{\empty}{\vspace{40px}}{\raggedleft#2}
        \vspace{-15px}
        \rule{\linewidth}{1pt}\par\nobreak
        \centering{#3}
        \vspace{-10px}
        \rule{\linewidth}{1pt}\par\nobreak
        \vspace{-10px}
    }{#2#3}
}
\makeatother

%== Customise Table of Contents Heading ==
\makeatletter
\def\createtoc{
    \renewcommand\tableofcontents{
        \chapter*{\contentsname}
        \@starttoc{toc}
    }
    \tableofcontents
}
\makeatother

%========= Front Matter Pages ============
\def\volumetitle{Volume \rom{\volumenumber}: \volumename}

\def\frontmatterpages{
    \frontmatter  % Use lowercase roman numerals for page numbers

    % Title page
    \begin{titlepage}
        \centering{
            \selectfont
            \Huge
            \textbf{Abstract Algebra}\\
            \vspace{-0.2cm}
            
            \Large
            \textbf{A Simple Introduction}\\
            \vspace{0.5cm}
            
            \LARGE
            \volumetitle
        }\\
        \vspace{2cm}
        \centering{\Large{Overwrite}}\\
        \vspace{\fill}
        \centering \small{\textit{Version \version}}
    \end{titlepage}

    \newpage{}

    % Edition notice
    \clearpage\null\vfill
    \thispagestyle{empty}
    \begin{minipage}[b]{0.9\textwidth}
        \footnotesize\raggedright
        \setlength{\parskip}{0.5\baselineskip}

        Published by Kan Onn Kit\\
        Singapore
        \vspace{7cm}

        \textbf{Abstract Algebra: A Simple Introduction -- \volumetitle}\par
        Version \version
        \vspace{0.35cm}

        Copyright \copyright \ 2022 -- \the\year\ by Kan Onn Kit\par
        This work is licensed under a
        Creative Commons Attribution-NonCommercial-ShareAlike 4.0 International Licence.\par
        \includegraphics[width=2.5cm]{../Images/CC BY-NC-SA 4.0.png}\\  % With reference to the volumes' folders
        The full licence text is available at \url{http://creativecommons.org/licenses/by-nc-sa/4.0/}.\par    
        The source files for the project are available \href{https://github.com/PhotonicGluon/Abstract-Algebra-Book}{here}.
        \vspace{0.35cm}

        Typeset in 10pt Computer Modern Roman using PDF\LaTeX.
    \end{minipage}

    \vspace*{2\baselineskip}
    \cleardoublepage

    % "Quote" page
    \thispagestyle{empty}
    \vspace*{2cm}

    \begin{center}
        \Large{\parbox{10cm}{
            \begin{raggedright}
                \Large
                \quotepagetext
                \vspace{0.3cm}
                
                \hfill
                --- \quotepageattribution\\
                \vspace{-0.25cm}
                
                \hfill
                \normalsize
                (\quotepagecitation)
            \end{raggedright}
        }
    }
    \end{center}

    \newpage

    % Table of contents
    \createtoc
    \setcounter{part}{\volumenumber}

    % Preface
    \chapter{Preface}
    Although algebra has a long history, it has undergone some quite striking changes in the past few decades. Abstract algebra is widely recognised as an essential element of higher mathematical education. The results and theorems that it showcases, however, are oft hard to grasp and understand without prerequisite knowledge or with a heavy background in mathematics. Most books on this subject are crafted for undergraduates at universities. They are not for a general mathematics enthusiast or one who seeks to understand more about the inner structure of algebra that many mathematicians encounter frequently.

    It is thus the goal of this series of books to provide a step-by-step explanation of core results from abstract algebra; to demystify the core steps that many textbooks skip over when writing proofs. I aim to ensure that the results from such an essential field of study are as accessible, as approachable, and as understandable for as many people as possible.

    \prefacevolumetext

    \hfill{\textit{27 January, 2023}}

    \mainmatter  % Now use arabic numerals for page numbers
}

\usepackage{xr}

%=========== Global Variables ============
\newcommand{\version}{0.1}
\newcommand{\volumenumber}{2}
\newcommand{\volumename}{Rings}
\newcommand{\volumeimage}{cover/Integers.png}

% Quote page variables
\newcommand{\quotepagetext}{
    [Some] of the major discoveries in ring theory have helped shape the course of development of modern abstract algebra... A course in ring theory is an indispensable part of the education of any fledgling algebraist.
}
\newcommand{\quotepageattribution}{Tsit-Yuen Lam, 2001}
\newcommand{\quotepagecitation}{\cite{lam_2001}}

% Preface variables
\newcommand{\prefacevolumetext}{
    This volume covers the basics of ring theory. %TODO: Add
}

%============= Formatting ================
\linespread{1.05}

%============== Resources ================
\externaldocument{../Volume 0/AbstractAlgebraVol0}

%=========== Custom Commands =============
\newcommand{\Mn}[2]{\mathcal{M}_{#1}(#2)}  % The ring of n by n matrices with entries in the ring R

\newcommand{\Q}{\mathbb{Q}}     % The ring rational numbers
\newcommand{\Z}{\mathbb{Z}}     % The ring of integers
\newcommand{\Zn}[1]{\mathbb{Z}_{#1}}  % The ring of integers modulo n

%=========================================
\begin{document}
\frontmatterpages

%=========================================
\chapter{Introduction to Rings}
\section{Basic Algebraic Structures}
In Volume I, we looked exclusively at groups and their operations. We discussed how groups are a generalisation of symmetry and looked at results related to groups. In this volume, we look at rings.

Before we introduce rings, we look at `simpler' algebraic structures and build our way up to them.

\begin{definition}
    A \textbf{magma}\index{magma} is a set $M$ together with a binary operation $\ast$ which is closed. That is, if $a$ and $b$ are in $M$, then $a \ast b \in M$. Such a magma is denoted $(M, \ast)$.
\end{definition}
\begin{example}
    Consider the set $M = \{1, 2, 3, 4\}$ with the operation $\ast$ such that $(M, \ast)$ has the Cayley table as shown below.
    \begin{table}[h]
        \centering
        \begin{tabular}{|l|l|l|l|l|}
            \hline
            $\ast$     & \textbf{1} & \textbf{2} & \textbf{3} & \textbf{4} \\ \hline
            \textbf{1} & 1          & 2          & 1          & 2          \\ \hline
            \textbf{2} & 2          & 3          & 4          & 1          \\ \hline
            \textbf{3} & 1          & 3          & 4          & 2          \\ \hline
            \textbf{4} & 2          & 1          & 2          & 1          \\ \hline
        \end{tabular}
    \end{table}
    
    Clearly for any $a, b\in M$ we have $a \ast b \in M$, so $(M, \ast)$ is a magma.
\end{example}

\begin{definition}
    A \textbf{semigroup}\index{semigroup} is a magma $(\mathcal{S}, \ast)$ where the operation $\ast$ is associative. That is, $a\ast(b\ast c) = (a\ast b)\ast c$.
\end{definition}
\begin{example}
    Consider the set $S = \{1, 2, 3, 4\}$ with the operation $\ast$ such that $(S, \ast)$ has the Cayley table as shown below.
    \begin{table}[h]
        \centering
        \begin{tabular}{|l|l|l|l|l|}
            \hline
            $\ast$     & \textbf{1} & \textbf{2} & \textbf{3} & \textbf{4} \\ \hline
            \textbf{1} & 1          & 1          & 1          & 1          \\ \hline
            \textbf{2} & 2          & 2          & 2          & 2          \\ \hline
            \textbf{3} & 3          & 3          & 3          & 3          \\ \hline
            \textbf{4} & 4          & 4          & 4          & 4          \\ \hline
        \end{tabular}
    \end{table}
    
    One sees that $(S, \ast)$ is closed under $\ast$. In addition, $\ast$ is associative. Hence $(S, \ast)$ is a semigroup.
\end{example}

\begin{definition}
    A \textbf{monoid}\index{monoid} is a semigroup $(M, \ast)$ with an element $e$, called the \textbf{identity}, such that
    \[
        e \ast m = m \ast e = m
    \]
    for all $m \in M$.
\end{definition}
\begin{example}
    Let $X$ be a set. Let the set
    \[
        S = \{f \vert  f: X \to X\}.
    \]
    and let $\circ$ denote function composition. Then $(S, \circ)$ forms a monoid:
    \begin{itemize}
        \item \textbf{Closure}: If $f, g \in S$ then $f\circ g \in S$ since function composition is closed.
        \item \textbf{Associative}: Function composition is associative.
        \item \textbf{Identity}: The identity function $\id: X \to X, x\mapsto x$ is inside $S$ and
        \[
            \id \circ f = f \circ \id = f
        \]
        for all $f \in S$.
    \end{itemize}
    Thus $(S, \circ)$ forms a monoid.
\end{example}

\begin{definition}
    A \textbf{group}\index{group} is a monoid $(G, \ast)$ where every element has an inverse. That is to say, for every $g \in G$, there exists $g^{-1} \in G$ such that
    \[
        g \ast g^{-1} = g^{-1} \ast g = e
    \]
    where $e$ is the identity in $G$.
\end{definition}

\section{Definition of a Ring}
With all that setup, we are ready to define what a ring is. Note that we follow \cite[p.~223]{dummit_foote_2004}, \cite[p.~115, Definition 1.1]{hungerford_1980}, and \cite{proofwiki_ringdefinition} for the definition of a ring.
\begin{definition}
    A \textbf{ring}\index{ring} is a set $R$ with two binary operations $\oplus$ and $\otimes$ satisfying the following axioms.
    \begin{itemize}
        \item \textbf{Addition-Abelian}: $(R, \oplus)$ is an abelian group.
        \item \textbf{Multiplication-Semigroup}: $(R, \otimes)$ is a semigroup.
        \item \textbf{Distribution}: $\otimes$ is distributive over $\oplus$. That is,
        \begin{itemize}
            \item $a \otimes (b \oplus c) = (a \otimes b) \oplus (b \otimes c)$; and
            \item $(a \oplus b) \otimes c = (a \otimes c) \oplus (b \otimes c)$.
        \end{itemize}
    \end{itemize}
    One may denote such a ring by $(R, \oplus, \otimes)$.
\end{definition}
\begin{remark}
    Other authors (e.g. \cite[p.~136]{cohn_1982}, \cite[pp.~145--146]{clark_1984}) may require that $(R, \otimes)$ is a monoid. In this book, any ring that satisfies the above condition is called a \textbf{ring with identity}\index{ring!with identity}.
\end{remark}
\begin{remark}
    A ring where $a \otimes b = b \otimes a$ for all $a$ and $b$ in $R$ is called a \textbf{commutative ring}\index{ring!commutative}.
\end{remark}

We end this chapter by introducing the \textbf{trivial ring}.
\begin{definition}
    The \textbf{trivial ring}\index{trivial ring} (or \textbf{zero ring}\index{zero ring}), denoted $\textbf{0}$, is the ring $(\{0\}, +, \times)$ where
    \[
        0 + 0 = 0 \text{ and } 0 \times 0 = 0.    
    \]
\end{definition}
\begin{exercise}
    Prove that the trivial ring is a commutative ring with identity.
\end{exercise}

%=========================================
\chapter{Basics of Rings}
\section{Basic Examples of Rings}
Before we introduce some examples of rings, we make some remarks for the notation that is used in Ring Theory.
\begin{itemize}
    \item We often write $\oplus$ using the standard addition symbol $+$, so $x \oplus y$ is written as $x + y$.
    \item The multiplication symbol $\otimes$ is usually omitted, so $x \otimes y$ is written as $xy$.
    \item The additive identity of $R$ will always be denoted by 0.
    \item The additive inverse of the element $x$ will be denoted by $-x$.
    \item $n$ applications of $\oplus$ on an element $x$ will be denoted $nx$ (and will be denoted $-nx$ if the element is $-x$).
    \item $n$ applications of $\otimes$ on an element $x$ will be denoted $x^n$ (and will be denoted $x^{-n}$ if the element is $x^{-1}$ and if it exists).
\end{itemize}

Let's look at some examples of rings.
\begin{example}
    We show that $(\Zn{n}, \oplus_n, \otimes_n)$, where $\oplus_n$ and $\otimes_n$ denote addition and multiplication modulo $n$ respectively, form a ring.
    \begin{itemize}
        \item \textbf{Addition-Abelian}: From Volume I we know that $(\Zn, \oplus_n)$ is an abelian group.
        \item \textbf{Multiplication-Semigroup}: We can see that $(\Zn, \otimes_n)$ is a semigroup as
        \begin{itemize}
            \item $\Zn{n}$ is closed under $\otimes_n$ because $a \otimes_n b \in \{0, 1, 2, \dots, n-1\} = \Zn{n}$; and
            \item multiplication is associative, so multiplication modulo $n$ is associative.
        \end{itemize}
        \item \textbf{Distribution}: It is clear that $\oplus_n$ and $\otimes_n$ distribute.
    \end{itemize}
    Hence $(\Zn{n}, \oplus_n, \otimes_n)$ is a ring. Furthermore, $\otimes_n$ has an identity of 1 and is commutative, so in fact $(\Zn, \otimes_n)$ is a commutative monoid. Therefore $(\Zn{n}, \oplus_n, \otimes_n)$ is a commutative ring with identity.

    Note that, in this volume, $\Zn{n}$ refers to the ring $(\Zn{n}, \oplus_n, \otimes_n)$.
\end{example}

\begin{example}
    We show that $(\Q, +, \times)$, where $+$ and $\times$ denote normal addition and multiplication, form a ring.
    \begin{itemize}
        \item \textbf{Addition-Abelian}: From Volume I we know that $(\Q, +)$ is an abelian group.
        \item \textbf{Multiplication-Semigroup}: We note that $(\Q, \times)$ is a semigroup as
        \begin{itemize}
            \item $Q$ is closed under $\times$ because multiplying two rational numbers together produce a rational number; and
            \item multiplication is associative.
        \end{itemize}
        \item \textbf{Distribution}: It is clear that $+$ and $\times$ distribute.
    \end{itemize}
    Hence $(\Q, +, \times)$ is a ring. Furthermore, $\times$ has an identity of 1 and is commutative. So, just like $\Zn{n}$, we see that $(\Q, +, \times)$ is a commutative ring with identity.

    Note that, in this volume, $\Q$ refers to the ring $(\Q, +, \times)$.
\end{example}

\begin{exercise}
    Prove that $\Z$ is a ring under regular addition and multiplication.\newline
    (\textit{You do \textbf{not} need to prove the \textbf{Distributive} axiom.})
\end{exercise}
Note that, in this volume, $\Z$ refers to the ring $(\Z, +, \times)$.

We look at one more ring: the ring of $n \times n$ matrices.

Let $\Mn{n}{R}$ denote the set of $n\times n$ matrices with entries in the ring $(R, \oplus, \otimes)$ (or simply just $R$).
We need to define addition and multiplication within this set.
\begin{itemize}
    \item We define \textbf{matrix addition}\index{matrix addition} within this set. Suppose we have $\textbf{A}, \textbf{B} \in \Mn{n}{R}$, and let their sum be $\textbf{C} = \textbf{A} + \textbf{B}$. Then
    \[
        c_{i,j} = a_{i,j} \oplus b_{i,j}    
    \]
    where $\oplus$ is the addition operation in the ring $R$.
    
    \item We define \textbf{matrix multiplication}\index{matrix multiplication} within this set. Suppose $\textbf{A}, \textbf{B} \in \Mn{n}{R}$, and let their product be $\textbf{C} = \textbf{AB}$. Then
    \begin{align*}
        c_{i,j} &= (a_{i,1}\otimes b_{1,j}) \oplus (a_{i,2}\otimes b_{2,j}) \oplus \cdots \oplus (a_{i,n}\otimes b_{n,j})\\
        &= \sum_{k=1}^n (a_{i,k}\otimes b_{i,k})
    \end{align*}
    where $\oplus$ and $\otimes$ are the addition and multiplication operations in the ring $R$ respectively.
\end{itemize}

\begin{proposition}
    $\Mn{n}{R}$ is a ring under matrix addition and matrix multiplication.
\end{proposition}
\begin{proof}
    We need to prove that the ring axioms hold. For cl
    \begin{itemize}
        \item \textbf{Addition-Abelian}: We first prove that $(\Mn{n}{R}, +)$ is indeed an abelian group.
        \begin{itemize}
            \item \textbf{Closure}: Clearly the sum of any two matrices in $\Mn{n}{R}$ is also a square matrix with $n$ rows with elements inside $R$, meaning that $\Mn{n}{R}$ is closed under matrix addition.

            \item \textbf{Associative}: Let the matrices $\textbf{A}$, $\textbf{B}$, and $\textbf{C}$ belong inside $\Mn{n}{R}$. Let $\textbf{P} = \textbf{A} + (\textbf{B} + \textbf{C})$ and $\textbf{Q} = (\textbf{A} + \textbf{B}) + \textbf{C}$. We note that $\textbf{P} = \textbf{Q}$ as
            \begin{align*}
                p_{i,j} &= a_{i,j} \otimes (b_{i,j} \otimes c_{i,j})\\
                &= (a_{i,j} \otimes b_{i,j}) \otimes c_{i,j} & (\otimes\text{ is associative})\\
                &= q_{i,j}
            \end{align*}
            which proves that matrix addition is associative.
    
            \item \textbf{Identity}: Denote the $n \times n$ matrix with contains only the zero in $R$ by $\textbf{0}_n$. One sees clearly that this is the identity in $\Mn{n}{R}$ as $\textbf{M} + \textbf{0}_n = \textbf{M}$ for any matrix in $\Mn{n}{R}$.
            
            \item \textbf{Inverse}: Let $\textbf{A} \in \Mn{n}{R}$. Define the matrix $\textbf{B} = -\textbf{A}$ where
            \[
                b_{i,j} = -a_{i,j},    
            \]
            that is, $b_{i,j}$ contains the additive inverse of $a_{i,j}$ in the ring $R$. Then one sees that $\textbf{A} + \textbf{B} = \textbf{0}_n$.\newline
            (Note that we denote the additive inverse of a matrix $\textbf{M}$ by $-\textbf{M}$.)

            \item \textbf{Commutative}: Let $\textbf{A}, \textbf{B} \in \Mn{n}{R}$. Set $\textbf{C} = \textbf{A} + \textbf{B}$ and $\textbf{D} = \textbf{B} + \textbf{C}$. Then
            \begin{align*}
                c_{i,j} &= a_{i,j} \oplus b_{i,j}\\
                &= b_{i,j} \oplus a_{i,j} & (\oplus \text{ is commutative})\\
                &= d_{i,j}
            \end{align*}
            which means $\textbf{C} = \textbf{D}$.
        \end{itemize}

        \item \textbf{Multiplication-Semigroup}: We now need to prove that $(\Mn{n}{R}, \cdot)$ is a semigroup.
        \begin{itemize}
            \item \textbf{Closure}: In Volume I we showed that matrix multiplication produces another $n \times n$ matrix. Furthermore the entries of the new matrix are elements of $R$. Hence $\Mn{n}{R}$ is closed under matrix multiplication.
        
            \item \textbf{Associative}: We proved matrix multiplication is associative in Volume I.
        \end{itemize}
        
        \item \textbf{Distribution}: We prove only $\textbf{A}(\textbf{B} + \textbf{C}) = (\textbf{AB}) + (\textbf{AC})$ as the other case is proven similarly.
        
        Let $\textbf{R} = \textbf{A}(\textbf{B} + \textbf{C})$, $\textbf{G} = \textbf{AB}$, and $\textbf{H} = \textbf{AC}$. We note
        \begin{align*}
            r_{i,j} &= \sum_{k=1}^n \left(a_{i,k} \otimes \left(b_{k,j} \oplus c_{k,j}\right)\right)\\
            &= \sum_{k=1}^n \left((a_{i,k} \otimes b_{k,j}) \oplus (a_{i,k} \otimes c_{k,j})\right)\\
            &= \left(\sum_{k=1}^n (a_{i,k} \otimes b_{k,j})\right) \oplus \left(\sum_{k=1}^n (a_{i,k} \otimes c_{k,j})\right)\\
            &= g_{i,j}\otimes h_{i,j}
        \end{align*}
        which means $\textbf{R} = \textbf{G} + \textbf{H}$.
    \end{itemize}
    As all the ring axioms are satisfied, thus $\Mn{n}{R}$ is a ring.
\end{proof}
For brevity, we let $\Mn{n}{R}$ denote the ring under matrix addition and multiplication.

\section{Types of Rings}
We now look at more specific types of rings.
\begin{definition}
    An \textbf{integral domain}\index{integral domain} $R$ is a commutative ring with identity such that for elements $a$ and $b$ in $R$,
    \[
        ab = 0 \text{ if and only if } a = 0 \text{ or } b = 0
    \]
\end{definition}

\begin{definition}
    A \textbf{division ring}\index{division ring} $R$ is a non-trivial ring such that every non-zero element $x$ in $R$ has a multiplicative inverse $x^{-1}$ where
    \[
        xx^{-1} = x^{-1}x = 1.    
    \]
\end{definition}

\begin{definition}
    A \textbf{field}\index{field} is a commutative division ring.
\end{definition}
We look at fields in more detail in Volume III.

% TODO: Add

%=========================================
\appendix
\chapter{Exercise Solutions}
\section{Introduction to Rings}
\begin{enumerate}
    \item We note the following.
    \begin{itemize}
        \item \textbf{Addition-Abelian}: $(\{0\}, +)$ is an abelian group by Volume I.
        \item \textbf{Multiplication-Semigroup}: $(\{0\}, \times)$ is an abelian group by Volume I.
        \item \textbf{Distribution}: We know $+$ and $\times$ distribute.
    \end{itemize}
    Hence $(\{0\}, +, \times)$ is a commutative ring with identity.
\end{enumerate}

\section{Basics of Rings}
\begin{enumerate}
    \item We note the following.
    \begin{itemize}
        \item \textbf{Addition-Abelian}: $(\Z, +)$ is an abelian group by Volume I.
        \item \textbf{Multiplication-Semigroup}: $(\Z, \times)$ is a semigroup since
        \begin{itemize}
            \item multiplying two integers always results in an integer, so $\Z$ is closed under $\times$; and
            \item $\times$ is associative.
        \end{itemize}
        \item \textbf{Distribution}: We know $+$ and $\times$ distribute.
    \end{itemize}
    Hence $(\Z, +, \times)$ is a ring.
\end{enumerate}

%=========================================
\chapter{Problem Solutions}
\section{Introduction to Rings}
No problems.

\section{Basics of Rings}
%TODO: Add

%=========================================
\chapter{Image Acknowledgements}
Unless otherwise stated, all images are the author's own work, and are released under the same licence as this book.

\section{Cover Page}
Image of the set of integers is the \LaTeX\, symbol for the set of integers (``$\mathbb{Z}$'').

%=========================================
\printbibliography[heading=bibintoc, title={References and Bibliography}]
\printindex

\end{document}
