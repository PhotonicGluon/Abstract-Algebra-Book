\section{Sets}
\begin{questions}
    \item \begin{partquestions}{\alph*}
        \item True, as both 1 and 2 appear in the set $\{1, 2, 3, 4\}$.
        \item False, 3 does not appear in $\{1, 2, 4\}$.
        \item True. Any set is a subset of itself, including the empty set.
        \item False, the set $S$ does not contain any element that is not in $S$. That is, $S \subseteq S$ but not $S \subset S$.
        \item True. $S$ is indeed an element of $\{S, \emptyset\}$.
        \item True. The set containing S is not an element of $\{S, \emptyset\}$.
        \item False, the set $S$ is not a subset of the set $\{S, \emptyset\}$.
        \item True. The set containing $S$ is a subset of the set containing $S$ and the empty set.
    \end{partquestions}

    \item \begin{partquestions}{\alph*}
        \item True. The set of elements that are in either $S$ or $R$ is indeed $\{1, 2, 3, 4, 5\}$.
        \item False, $S \cup U = \{1, 2, 3, 4, (2, 2), (3, 3), (5, 5)\}$.
        \item True. The set of elements that are in both $S$ and $T$ is indeed $\{2, 3\}$.
        \item True. $T$ and $U$ share no elements in common, so their intersection is empty.
        \item True. The elements that are in $S$ but not in $T$ are indeed 1 and 4.
        \item False, $S \setminus \{1, 4\} = \{2, 3\}$, not $T = \{2, 3, 5\}$.
        \item True.
        \item True. $(S \cup T)^2 = \{(1,1), (2,2), (3,3), (4,4), (5,5)\}$, so
        \[
            U = \{(2,2), (3,3), (5,5)\} \subset (S \cup T)^2.
        \]
    \end{partquestions}

    \item We note $S$ are all the non-positive rational numbers, and $T = \{-2, 0, 2, \dots, 8, 10\}$. Hence $S \cap T$ has only two elements, namely $-2$ and $0$.
\end{questions}
