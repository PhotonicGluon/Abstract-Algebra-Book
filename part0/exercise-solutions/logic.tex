\section{Mathematical Logic}
\begin{questions}
    \item We work from the inner-most bracket outwards. We note $P$ is true, $Q$ is false, and $R$ is true.
    \begin{itemize}
        \item $P \lor Q$ is ``1 is a positive number \textbf{or} $-1 > 0$'', which is true since $P$ is true.
        \item $(P \lor Q) \land R$ is ``(1 is a positive number or $-1 > 0$) \textbf{and} 1 is an odd number'', which is true since $P \lor Q$ is true and 1 is, indeed, an odd number.
        \item $\lnot((P \lor Q) \land R)$ is false, since $(P \lor Q) \land R$ is true.
    \end{itemize}
    Hence the statement ``$\lnot((P \lor Q) \land R)$ is false'' is a true statement.
    
    \item The truth table for $P \land (\lnot Q)$ is given below:
    \begin{table}[h]
        \centering
        \begin{tabular}{|l|l||l|}
            \hline
            $P$ & $Q$ & $P\land (\lnot Q)$ \\ \hline
            F   & F   & F                  \\ \hline
            F   & T   & F                  \\ \hline
            T   & F   & T                  \\ \hline
            T   & T   & F                  \\ \hline
        \end{tabular}
    \end{table}
    
    \item \begin{partquestions}{\roman*}
        \item $R$: $n$ is a multiple of 5 if and only if the last digit of $n$ is 0 or 5.
        \item If $n$ is a multiple of 5, then its last digit necessarily has to be 5 or 0, hence $P \implies Q$. If the last digit is 5 or 0, then the number $n$ is a multiple of 5, hence $Q \implies P$. Therefore $P \iff Q$.
    \end{partquestions}
    
    \item For brevity, let $R = (P \implies (\lnot Q))$ and $S = ((\lnot Q) \implies P)$. So we want to show that $((\lnot P) \iff Q) \equiv R \land S$.
    \begin{table}[h]
        \centering
        \begin{tabular}{|l|l||l|l|l|l||l|l|}
            \hline
            $P$ & $Q$ & $\lnot P$ & $\lnot Q$ & $R$ & $S$ & $R \land S$ & $(\lnot P) \iff Q$ \\ \hline
            F   & F   & T         & T         & T   & F   & F           & F                  \\ \hline
            F   & T   & T         & F         & T   & T   & T           & T                  \\ \hline
            T   & F   & F         & T         & T   & T   & T           & T                  \\ \hline
            T   & T   & F         & F         & F   & T   & F           & F                  \\ \hline
        \end{tabular}
    \end{table}

    From inspection, the truth tables of $(\lnot P) \iff Q$ and $R \land S$ are the same, proving our required result.
    
    \item For brevity, denote $X = P \lor \lnot Q$, $Y = \lnot R$, and $Z = (P \lor R) \land (P \lor \lnot R)$. Thus the original statement is something like $(X \land Y) \lor (X \land Z)$, which by distributive rules is equal to $X \land (Y \lor Z)$. Note $Z \equiv P \lor (R \land \lnot R)$ by distributive rules, which is equal to $P$ since $R \land \lnot R$ is always false. Hence $X \land (Y \lor Z) \equiv (P \lor \lnot Q) \land (\lnot R \lor P)$. Commutativity of $\lor$ means that $\lnot R \lor P \equiv P \lor \lnot R$, so $(P \lor \lnot Q) \land (\lnot R \lor P) \equiv (P \lor \lnot Q) \land (P \lor \lnot R)$. Now by distributive rules $(P \lor \lnot Q) \land (P \lor \lnot R) \equiv P \lor (\lnot Q \land \lnot R)$. Finally, by De Morgan's Law, $\lnot Q \land \lnot R \equiv \lnot(Q \lor R)$, so the original statement is equal to $P \lor \lnot(Q \lor R)$.
    
    \item $\left[(n \in \mathbb{Z}) \land (n > 2)\right] \implies \left[\exists a, b \in \mathbb{Z} \text{ s.t. } a^3 + b^4 = n^5\right]$
    
    \item \begin{partquestions}{\roman*}
        \item Let
        \begin{align*}
            P:&\ (x \in \mathbb{R}) \land (x \neq 0)\\
            Q:&\ \exists y \in \mathbb{R} \text{ s.t. } xy = 1
        \end{align*}
        then the given statement is $P \implies Q$ as required.
        \item $(x \in \mathbb{R}) \land (x \neq 0) \land (\forall y \in \mathbb{R}, xy \neq 1)$
    \end{partquestions}
\end{questions}