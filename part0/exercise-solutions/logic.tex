\section{Mathematical Logic}
\begin{questions}
    \item We work from the inner-most bracket outwards. We note $p$ is true, $q$ is false, and $r$ is true.
    \begin{itemize}
        \item $p \lor q$ is ``1 is a positive number \textbf{or} $-1 > 0$'', which is true since $p$ is true.
        \item $(p \lor q) \land r$ is ``(1 is a positive number or $-1 > 0$) \textbf{and} 1 is an odd number'', which is true since $p \lor q$ is true and 1 is, indeed, an odd number.
        \item $\lnot((p \lor q) \land r)$ is false, since $(p \lor q) \land r$ is true.
    \end{itemize}
    Hence the statement ``$\lnot((p \lor q) \land r)$ is false'' is a true statement.
    
    \item The truth table for $p \land \lnot q$ is given below.
    \begin{table}[H]
        \centering
        \begin{tabular}{|l|l||l|}
            \hline
            $\boldsymbol{p}$ & $\boldsymbol{q}$ & $\boldsymbol{p\land \lnot q}$ \\ \hline
            F   & F   & F                  \\ \hline
            F   & T   & F                  \\ \hline
            T   & F   & T                  \\ \hline
            T   & T   & F                  \\ \hline
        \end{tabular}
    \end{table}
    
    We will see that, in fact, this is the negation of $p \implies q$ later.
    
    \item \begin{partquestions}{\roman*}
        \item $r$: $n$ is a multiple of 5 if and only if the last digit of $n$ is 0 or 5.
        \item If $n$ is a multiple of 5, then its last digit necessarily has to be 5 or 0, hence $p \implies q$. If the last digit is 5 or 0, then the number $n$ is a multiple of 5, hence $q \implies p$. Therefore $p \iff q$.
    \end{partquestions}
    
    \item For brevity, let $r = p \implies \lnot q$ and $s = \lnot q \implies p$. So we want to show that $(\lnot p \iff q) \equiv r \land s$.
    \begin{table}[H]
        \centering
        \begin{tabular}{|l|l||l|l|l|l||l|l|}
            \hline
            $\boldsymbol{p}$ & $\boldsymbol{q}$ & $\boldsymbol{\lnot p}$ & $\boldsymbol{\lnot q}$ & $\boldsymbol{r}$ & $\boldsymbol{s}$ & $\boldsymbol{r \land s}$ & $\boldsymbol{\lnot p \iff q}$ \\ \hline
            F   & F   & T         & T         & T   & F   & F           & F                  \\ \hline
            F   & T   & T         & F         & T   & T   & T           & T                  \\ \hline
            T   & F   & F         & T         & T   & T   & T           & T                  \\ \hline
            T   & T   & F         & F         & F   & T   & F           & F                  \\ \hline
        \end{tabular}
    \end{table}

    From inspection, the truth tables of $\lnot p \iff q$ and $r \land s$ are the same, proving our required result.
    
    \item We work slowly.
    \begin{align*}
        &((p \lor \lnot q) \land \lnot r) \lor ((p \lor \lnot q) \land (p \lor r) \land (p \lor \lnot r))\\
        &\equiv ((p \lor \lnot q) \land \lnot r) \lor ((p \lor \lnot q) \land ((p \lor r) \land (p \lor \lnot r))) & (\text{Associativity})\\
        &\equiv (p \lor \lnot q) \land (\lnot r \lor ((p \lor r) \land (p \lor \lnot r))) & (\text{Distributivity})\\
        &\equiv (p \lor \lnot q) \land (\lnot r \lor (p \lor (r \land \lnot r))) & (\text{Distributivity})\\
        &\equiv (p \lor \lnot q) \land (\lnot r \lor (p \lor \textbf{false}))\\
        &\equiv (p \lor \lnot q) \land (\lnot r \lor p)\\
        &\equiv (p \lor \lnot q) \land (p \lor \lnot r) & (\text{Commutativity})\\
        &\equiv p \lor (\lnot q \land \lnot r) & (\text{Distributivity})\\
        &\equiv p \lor \lnot(q \lor r) & (\text{De Morgan's Law})
    \end{align*}

    \item $\left[(n \in \mathbb{Z}) \land (n > 2)\right] \implies \left[\exists a, b \in \mathbb{Z} \text{ s.t. } a^3 + b^4 = n^5\right]$
    
    \item \begin{partquestions}{\roman*}
        \item Let
        \begin{align*}
            P(x):&\ (x \in \mathbb{R}) \land (x \neq 0)\\
            Q(x):&\ \exists y \in \mathbb{R} \text{ s.t. } xy = 1
        \end{align*}
        then the given statement is $\forall x, (P(x) \implies Q(x))$ as required.

        \item $\exists x \text{ s.t. } (x \in \mathbb{R}) \land (x \neq 0) \land (\forall y \in \mathbb{R}, xy \neq 1)$
    \end{partquestions}
\end{questions}
