\section{Proof Writing}
\begin{questions}
    \item \begin{proof}
        Suppose $0 < x < 1$. Then $-1 < -x < 0$, meaning $0 < 1 - x < 1$. Therefore $x > 0$ and $1-x > 0$, so their product $x(1-x) > 0$.
    \end{proof}

    \item \begin{proof}
        Suppose that $m$ and $n$ have the same parity. We split into two cases.
        \begin{itemize}
            \item If both $m$ and $n$ are even, then we may write $m = 2a$ and $n = 2b$ where $a$ and $b$ are integers. Hence
            \begin{align*}
                m + n &= (2a) + (2b) \\
                &= 2(a+b)
            \end{align*}
            which clearly means that $m + n$ is even.
            \item If both $m$ and $n$ are odd, then we may write $m = 2a + 1$ and $n = 2b + 1$ where $a$ and $b$ are integers. Hence
            \begin{align*}
                m + n &= (2a + 1) + (2b + 1)\\
                &= 2a + 2b + 2\\
                &= 2(a + b + 1)
            \end{align*}
            which clearly means that $m+n$ is even.
        \end{itemize}
    Hence, in both cases, $m + n$ is even.
    \end{proof}
    
    \item We consider a proof by contrapositive; the statement that we want to prove is ''if \textbf{not} ($a$ is even or $b$ is odd) then $a(b^2+5)$ is \textbf{not} even''. That is, ''if $a$ is \textbf{not} even \textbf{and} $b$ is \textbf{not} odd then $a(b^2+5)$ is \textbf{not} even'', meaning ''if $a$ is odd and $b$ is even then $a(b^2+5)$ is odd''.
    
    \begin{proof}
        Suppose that $a$ is odd and $b$ is even. Then we may write $a = 2m + 1$ and $b = 2n$ where $m$ and $n$ are integers. Hence
        \begin{align*}
            a(b^2+5) &= (2m+1)\left((2n)^2 + 5\right)\\
            &= (2m+1)(4n^2 + 5)\\
            &= 8mn^2 + 10m + 4n^2 + 5\\
            &= 8mn^2 + 10m + 4n^2 + 4 + 1\\
            &= 2(4mn^2 + 5m + 2n^2 + 2) + 1
        \end{align*}
    which clearly means that $a(b^2+5)$ is odd.
    \end{proof}
    
    \item \begin{proof}
        By way of contradiction assume there exist integers $a$ and $b$ such that $2a + 4b = 1$. Then dividing both sides by 2 leads to $a + 2b = \frac12$. Note the left hand side is clearly an integer, while the right hand side is not an integer, a contradiction.    
    \end{proof}
    
    \item \begin{proof}
        By way of contradiction assume that $a$ and $b$ are positive real numbers, and $\frac{a+b}{2} < \sqrt{ab}$. This means $a+b<2\sqrt ab$. Squaring both sides yields $(a+b)^2 < 4ab$. Note
        \[
            (a+b)^2 = a^2 + 2ab + b^2 < 4ab    
        \]
        which implies $a^2 + b^2 < 2ab$, leading to $a^2 - 2ab + b^2 < 0$. However $a^2 - 2ab + b^2 = (a-b)^2 \geq 0$ for all positive real numbers $a$ and $b$. Hence we have $(a-b)^2 < 0$ and $(a-b)^2 \geq 0$, a contradiction.
    \end{proof}
    
    \item We note that a positive odd number is of the form $2n - 1$ where $n$ is a positive integer.
    
    \begin{proof}
        Set $a = 2n - 1$; we induct on $n$.
    
        When $n = 1$, we have $a^2 - 1 = (2(1) - 1)^2 - 1 = 1 - 1 = 0$ which is clearly a multiple of 8.
        
        Assume that the statement holds for some positive integer $k$, i.e. $(2k-1)^2 - 1 = 8m$ for some integer $m$. We show that the statement holds for $k + 1$.
        
        We note that $(2k-1)^2 - 1 = 4k^2 - 4k$. Observe
        \begin{align*}
            (2(k+1)-1)^2 - 1 &= (2k+1)^2 - 1\\
            &= 4k^2 + 4k + 1 - 1\\
            &= (4k^2 - 4k) + 8k\\
            &= 8m + 8k & (\text{by hypothesis})\\
            &= 8(m+k)
        \end{align*}
        which means that $(2(k+1)-1)^2 - 1$ is a multiple of 8, proving that the statement holds for $k+1$. By mathematical induction, $a^2 - 1$ is a multiple of 8 for all positive odd integers $a$.
    \end{proof}
    
    \item \begin{proof}
        We use strong induction on $n$.

        When $n = 2$ the statement is true since 2 is prime.

        Now assume that for some positive integer $k \geq 2$, every integer $m$ satisfying $2 \leq m \leq k$ results in the statement being true, i.e. $m$ is either prime or can be expressed as a product of primes. We are to show that the statement is true for $k + 1$, i.e. $k+1$ is prime or can be expressed as a product of primes.

        Now if $k + 1$ is prime we are done. Otherwise $k + 1$ is composite, meaning that $k + 1 = ab$ for some integers $2 \leq a,b \leq k$. Applying the induction hypothesis on $a$ and $b$ means that $a$ and $b$ are primes or product of primes. Thus $k + 1$ is a product of primes.

        Therefore by mathematical induction, every integer $n \geq 2$ is either prime or can be expressed as a product of primes.
    \end{proof}
    
    \item \begin{proof}
        We prove the forward direction using direct proof. Assume $n$ is one more than a multiple of 5. Then we may write $n = 5a + 1$ where $a$ is an integer. Note $5a + 1 = 5a + (5 - 4) = (5a + 5) - 4 = 5(a+1) - 4$. Setting $k = a+1$ yields required result.
    
        We now prove the reverse direction, using direct proof as well. Assume $n = 5k - 4$. Observe $5k - 4 = 5k - 5 + 1 = 5(k-1) + 1$, meaning $n$ is one more than a multiple of 5.
    \end{proof}
    
    \item \begin{proof}
        The number 7 satisfies this as $7 = 2^3 - 1$ and $7 = 3^2 - 2$.
    \end{proof}
    
    \item \begin{partquestions}{\roman*}
        \item We use a proof by contradiction to prove this claim.
        
        \begin{proof}
            Seeking a contradiction, assume $y$ is rational. Write $y = \frac pq$ where $p$ and $q$ are integers. Note $2^y = 2^{\frac pq}$ and $2^y = 2^{2\log_2{3}} = 9$. Hence $2^{\frac pq} = 9$ meaning $2^p = 9^q$. However $2^p$ is always even and $9^q$ is always odd, a contradiction.
        \end{proof}

        \item \begin{proof}
            Note
            \[
                x^y = (\sqrt2)^{2\log_2{3}} = \left(\sqrt{2}^2\right)^{\log_2{3}} = 2^{\log_2{3}} = 3        
            \]
            which is rational.
        \end{proof}
    \end{partquestions}
\end{questions}
