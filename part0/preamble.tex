\setpartpreamble[u][\textwidth]{
    \quoteattr{
        Aus dem Paradies, das Cantor uns geschaffen, soll uns niemand vertreiben k\"{o}nnen.\\
        \textnormal{(No one shall expel us from the Paradise that Cantor has created.)}
    }
    {
        David Hilbert, 1926
    }
    {
        \cite[p.~170]{hilbert_1926}
    }

    A firm foundation is required to build the core of abstract algebra. Without such a foundation, the claims put forth in future parts would be unwarranted and unjustified. The proofs of those claims would also seem baffling to the uninitiated. This part provides these tools and techniques to understand the subject matter in later parts.

    We start with the absolute fundamentals: sets, logic, and proof writing. Sets are a fundamental object in mathematics and will be used countless times in abstract algebra. Then, in logic, introduce first-order logic notation and its meaning as an overview of how statements are formed in later parts. We also explore the properties of such statements and discuss numerous ways to prove them. We spend a lot of time on proof writing as we will employ various types of proofs in abstract algebra, and it is critical to understand why they are valid proofs and how they work.

    Although this book assumes an understanding of high-school algebra, it is helpful to recapitulate the symbols, terminology, and results used there. Granted, the use of such algebraic skills will be limited in group theory, but they will be critical when we move on to later parts where polynomials and algebraic manipulation are involved.

    Afterwards, we look at relations and functions. Although we do not go into much detail about relations, we highlight the important properties of equivalence relations and link functions to relations. We note functions and maps are so widespread in abstract algebra that they deserve their own chapter.

    Elementary number theory and modular arithmetic will also often come up, as these fields are highly integrated within abstract algebra. Readers are advised to look at the notation of divisibility, Euclid's division lemma, and congruence modulo $n$ in particular. Other results are important but are not used as often.

    Unlike the other parts, not all results will have a proof. This is because most of these results are outside the scope of abstract algebra, and their proofs can be easily found online. We only prove some results there, especially if the result is hard to prove or hard to find a proof online. Other parts will have full proofs for all results.
}
\part{Preliminaries}
