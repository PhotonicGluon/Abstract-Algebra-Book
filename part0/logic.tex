\chapter{Mathematical Logic}
The heart of mathematics is in its logical statements. These statements are used to produce a series of logical steps to prove a claim. We explore the basics of logic here in order to understand what we claim.

\section{Statements}
\begin{definition}
    A \textbf{statement}\index{statement} (or \textbf{proposition}\index{proposition}) is a sentence that is definitely true or definitely false, but not both.
\end{definition}
\begin{remark}
    A statement can be written in English, or using mathematical notation.
\end{remark}
\begin{example}
    ``A square is a quadrilateral'' is a true statement.
\end{example}
\begin{example}
    ``A circle is a polygon'' is a false statement.
\end{example}
\begin{example}
    ``$12 \in \mathbb{Z}$'' is a true statement.
\end{example}
\begin{example}
    ``$\sqrt2 \in \mathbb{Z}$'' is a false statement.
\end{example}

We may name statements using lowercase variables like $p$, $q$, $r$, etc.
\begin{example}
    If
    \begin{align*}
        p: &\ \text{Every odd number is one more than an even number}\\
        q: &\ \text{Every triangle has sides of equal length}\\
        r: &\ \frac12 \in \mathbb{Q}
    \end{align*}
    then $p$ is true, $q$ is false, and $r$ is true.
\end{example}

There are a few operations\index{logical operation} that may be carried out on statements. For the following, assume $p$ and $q$ are statements.
\begin{itemize}
    \item \textbf{Logical NOT}\index{logical operation!NOT}/\textbf{Negation}\index{logical operation!negation} ($\lnot$): The statement $\lnot p$ is read as ``not $p$''.
    \item \textbf{Logical AND}\index{logical operation!AND}/\textbf{Conjunction}\index{logical operation!conjunction} ($\land$): The statement $p\land q$ is read as ``$p$ and $q$''.
    \item \textbf{Logical OR}\index{logical operation!OR}/\textbf{Disjunction}\index{logical operation!disjunction} ($\lor$): The statement $p\lor q$ is read as ``$p$ or $q$''.
    \item \textbf{Conditional}\index{logical operation!conditional}/\textbf{Implication}\index{logical operation!implication} ($\implies$): The statement $p \implies q$ can be read many different ways. We list a few:
    \begin{itemize}
        \item $p$ implies $q$;
        \item if $p$ then $q$;
        \item $q$ if $p$;
        \item $p$ only if $q$;
        \item $p$ is a sufficient condition for $q$; and
        \item $q$ is a necessary condition for $p$.
    \end{itemize}
\end{itemize}

\begin{example}
    If $p$ is the statement ``3 is an odd number'' and $q$ is the statement ``4 is an odd number'', then
    \begin{itemize}
        \item $p\land q$ is ``3 is an odd number \textbf{and} 4 is an odd number'', which is false;
        \item $p\lor q$ is ``3 is an odd number \textbf{or} 4 is an odd number'', which is true; and
        \item $\lnot q$ is ``4 is \textbf{not} an odd number'', which is true.
    \end{itemize}
\end{example}

When statement variables are combined using the above operators, the resulting statement is said to be in statement form.
\begin{definition}
    A \textbf{statement form}\index{statement form}(or \textbf{propositional form}\index{propositional form}) is an expression made up of statement variables (e.g. $p$, $q$, $r$) and logical operators that becomes a statement when actual statements are substituted for the component statement variables.
\end{definition}
\begin{example}
    The expressions $\lnot p$, $p \land q$, and $\lnot(p \land q) \lor r$ are all in statement form.
\end{example}

\begin{exercise}
    Let the statements
    \begin{align*}
        p: &\ \text{1 is a positive number}\\
        q: &\ -1 > 0\\
        r: &\ \text{1 is an odd number}
    \end{align*}
    Is the statement ``$\lnot((p\lor q)\land r)$ is false'' true?
\end{exercise}

We end this section by exploring conjoined statements that are always true or false.

\begin{definition}
    A \textbf{tautology}\index{tautology} is a statement form that is always true, and we denote it by \textbf{true}.
\end{definition}
\begin{definition}
    A \textbf{contradiction}\index{contradiction} is a statement form that is always false, and we denote it by \textbf{false}.
\end{definition}
\begin{example}
    The statement form $p \lor \lnot p$ is always true regardless of whatever statements are substituted for $p$. Thus $p \lor \lnot p$ is a tautology.
\end{example}
\begin{example}
    The statement form $p \land \lnot p$ is always false regardless of whatever statements are substituted for $p$. Thus $p \land \lnot p$ is a contradiction.
\end{example}

\section{Truth Tables}
We use \textbf{truth tables}\index{truth table} to explore the relationships between statements and operators. They list all possibilities of the truth or falsity of the statements $P$ and $Q$, and then write the truth for each of the combinations with operations. In a truth table, we denote true statements by ``T'' and false statements by ``F''. For example, the truth table for the logical AND operator is:
\begin{table}[h]
    \centering
    \begin{tabular}{|l|l||l|}
        \hline
        $\boldsymbol{p}$ & $\boldsymbol{q}$ & $\boldsymbol{p \land q}$ \\ \hline
        F   & F   & F          \\ \hline
        F   & T   & F          \\ \hline
        T   & F   & F          \\ \hline
        T   & T   & T          \\ \hline
    \end{tabular}
\end{table}

The truth table for the logical OR operator is:
\begin{table}[h]
    \centering
    \begin{tabular}{|l|l||l|}
        \hline
        $\boldsymbol{p}$ & $\boldsymbol{q}$ & $\boldsymbol{p \lor q}$ \\ \hline
        F   & F   & F         \\ \hline
        F   & T   & T         \\ \hline
        T   & F   & T         \\ \hline
        T   & T   & T         \\ \hline
    \end{tabular}
\end{table}

The truth table for the logical NOT operator is:
\begin{table}[h]
    \centering
    \begin{tabular}{|l||l|}
        \hline
        $\boldsymbol{p}$ & $\boldsymbol{\lnot p}$ \\ \hline
        F   & T         \\ \hline
        T   & F         \\ \hline
    \end{tabular}
\end{table}

We motivate the truth table for the conditional by considering the statements ``you pass the exam'' and ``you pass the course'', which we will denote by $p$ and $q$ respectively. So $p \implies q$ would be ``if you pass the exam then you pass the course''.
\begin{itemize}
    \item If $p$ and $q$ are true, then that means that you passed the exam and passed the course. Hence, ``if you pass the exam then you pass the course'' is true, meaning $p \implies q$ is true.
    \item If $p$ is true and $q$ is false, then that means that you passed the exam but failed the course. Hence, the promise that ``if you pass the exam then you pass the course'' is broken, meaning $p \implies q$ is false.
    \item Now consider the third case when $p$ is false but $q$ is true. This means that you failed the exam but passed the course. This does not mean that the promise was broken; you could have passed the course through other means. The only promise was that if you pass the exam then you pass the course; the promise was not that passing the exam was the \textit{only} way of passing the course. Since the promise was not broken, thus the promise was kept, so $p \implies q$ is true.
    \item Finally we consider the case when $p$ and $q$ are both false: you failed the exam and failed the course. The promise certainly was not broken in this case, so $p \implies q$ is true.
\end{itemize}
\begin{remark}
    A conditional statement that is true by the virtue of the fact that $p$ is false is called \textbf{vacuously true}\index{vacuously true}.
\end{remark}

In summary, the truth table for the conditional is:
\begin{table}[h]
    \centering
    \begin{tabular}{|l|l||l|}
        \hline
        $\boldsymbol{p}$ & $\boldsymbol{q}$ & $\boldsymbol{p\implies q}$ \\ \hline
        F   & F   & T             \\ \hline
        F   & T   & T             \\ \hline
        T   & F   & F             \\ \hline
        T   & T   & T             \\ \hline
    \end{tabular}
\end{table}

\begin{exercise}\label{exercise-negation-of-implication}
    Let $p$ and $q$ be statements. Draw the truth table for $p \land (\lnot q)$.
\end{exercise}

We now introduce the idea of the \textbf{biconditional}\index{logical operation!biconditional}.
\begin{definition}
    Let $p$ and $q$ be mathematical statements. If both $p \implies q$ and $q \implies p$ are true, then we write $p \iff q$. In other words,
    \[
        p \iff q \equiv (p \implies q) \land (q \implies p).
    \]
\end{definition}
\begin{remark}
    The statement $p \iff q$ can be written in several ways in English:
    \begin{itemize}
        \item $p$ if and only if (iff) $q$;
        \item $p$ is a necessary and sufficient condition for $q$; or
        \item $p$ is equivalent to $q$.
    \end{itemize}
\end{remark}

The truth table for the biconditional is:
\begin{table}[h]
    \centering
    \begin{tabular}{|l|l||l|}
        \hline
        $\boldsymbol{p}$ & $\boldsymbol{q}$ & $\boldsymbol{p \iff q}$ \\ \hline
        F   & F   & T         \\ \hline
        F   & T   & F         \\ \hline
        T   & F   & F         \\ \hline
        T   & T   & T         \\ \hline
    \end{tabular}
\end{table}

\newpage

\begin{exercise}
    Let $n$ be an integer. Let $p$ be the statement ``$n$ is a multiple of 5'' and $q$ be the statement ``the last digit of $n$ is 0 or 5''. Let the statement $r = p \iff q$.
    \begin{partquestions}{\roman*}
        \item Write the statement $r$ in English.
        \item Is the statement $r$ true? Justify your answer.
    \end{partquestions}
\end{exercise}

We end this section by noting the order of operations of the logical operators.
\begin{itemize}
    \item $\lnot$ is performed first.
    \item $\land$ and $\lor$ are performed second, and are coequal in their order of operation.
    \item $\implies$ and $\iff$ are performed last, and are coequal in their order of operation.
\end{itemize}
To disambiguate the order of operations, we use parentheses (i.e., brackets).

\section{Logical Equivalence and Properties of Logical Operators}
\begin{definition}
    Two statements $p$ and $q$ are \textbf{logically equivalent}\index{logically equivalent} if and only if they have identical truth values for each possible statement that $p$ and $q$ may take. If $p$ and $q$ are logically equivalent, we write $p \equiv q$.
\end{definition}
\begin{example}\label{example-implication-law}
    We show that $(p \implies q) \equiv \lnot p \lor q$ by considering the truth table for $\lnot p \lor q$.
    \begin{table}[h]
        \centering
        \begin{tabular}{|l|l||l||l|}
            \hline
            $\boldsymbol{p}$ & $\boldsymbol{q}$ & $\boldsymbol{\lnot p}$ & $\boldsymbol{\lnot P \lor Q}$ \\ \hline
            F   & F   & T         & T                  \\ \hline
            F   & T   & T         & T                  \\ \hline
            T   & F   & F         & F                  \\ \hline
            T   & T   & F         & T                  \\ \hline
        \end{tabular}
    \end{table}

    By inspection, we see that $\lnot p \lor q$ has the same truth table as $p \implies q$. Thus $(p \implies q) \equiv \lnot p \lor q$.
\end{example}
\begin{remark}
    We separate the intermediate value(s) (e.g. $\lnot p$ in the above example) from the rest by drawing a double line to the sides of the intermediate values.
\end{remark}

\newpage

\begin{example}
    We show that $(p \iff q) \equiv (p \land q) \lor (\lnot p \land \lnot q)$. For brevity, let $r = \lnot p \land \lnot q$.
    \begin{table}[h]
        \centering
        \begin{tabular}{|l|l||l|l|l|l||l|}
            \hline
            $\boldsymbol{p}$ & $\boldsymbol{q}$ & $\boldsymbol{\lnot p}$ & $\boldsymbol{\lnot q}$ & $\boldsymbol{p \land q}$ & $\boldsymbol{r}$ & $\boldsymbol{(p \land q) \lor r}$ \\ \hline
            F   & F   & T         & T         & F           & T   & T                    \\ \hline
            F   & T   & T         & F         & F           & F   & F                    \\ \hline
            T   & F   & F         & T         & F           & F   & F                    \\ \hline
            T   & T   & F         & F         & T           & F   & T                    \\ \hline
        \end{tabular}
    \end{table}

    By inspection of the truth table we establish the required result.
\end{example}

\begin{exercise}
    Show that
    \[
        ((\lnot p) \iff q) \equiv (p \implies (\lnot q)) \land ((\lnot q) \implies p)
    \]
    by drawing a truth table.
\end{exercise}

We note some important properties of logical operators.
\begin{itemize}
    \item \textbf{Contrapositive}\index{contrapositive}: $(p \implies q) \equiv (\lnot q \implies \lnot p)$
    \item \textbf{De Morgan's Laws}: \begin{itemize}
        \item $\lnot (p \land q) \equiv \lnot p \lor \lnot q$
        \item $\lnot (p \lor q) \equiv \lnot p \land \lnot q$
    \end{itemize}
    \item \textbf{Commutativity of AND and OR}: \begin{itemize}
        \item $p \land q \equiv q \land p$
        \item $p \lor q \equiv q \lor p$
    \end{itemize}
    \item \textbf{Associativity of AND and OR}: \begin{itemize}
        \item $p \land (q \land r) \equiv (p \land q) \land r$
        \item $p \lor (q \lor r) \equiv (p \lor q) \lor r$
    \end{itemize}
    \item \textbf{Distributive Rules}: \begin{itemize}
        \item $p \land (q \lor r) \equiv (p \land q) \lor (p \land r)$
        \item $p \lor (q \land r) \equiv (p \lor q) \land (p \lor r)$
    \end{itemize}
\end{itemize}
\begin{exercise}
    Simplify the statement
    \[
        ((p \lor \lnot q) \land \lnot r) \lor ((p \lor \lnot q) \land (p \lor r) \land (p \lor \lnot r))
    \]
    into a statement that uses only \textbf{three} operators in total.
\end{exercise}

\section{Predicates and Quantifiers}
\begin{definition}
    A \textbf{predicate}\index{predicate} is a sentence that contains a finite number of variables and becomes a statement when specific values from the \textbf{domain}\index{predicate!domain} are substituted for the variables.
\end{definition}
\begin{remark}
    We denote predicates with uppercase letters.
\end{remark}
\begin{example}
    Let $P(n)$ be the predicate ``$n$ is a multiple of 3'', with the domain being the positive integers. Then $P(1)$ is a false statement, $P(2)$ is a false statement, $P(3)$ is a true statement, and so on.
\end{example}
\begin{example}
    Let $Q(n)$ be the predicate ``$n$ is a factor of 9''.
    \begin{itemize}
        \item If the domain of $n$ is the positive integers, then $Q(1)$, $Q(3)$, and $Q(9)$ are true statements, and every other $Q(n)$ is false.
        \item If the domain of $n$ is the integers, then $Q(1)$, $Q(3)$, $Q(9)$, $Q(-1)$, $Q(-3)$, and $Q(-9)$ are all true statements.
    \end{itemize}
\end{example}

There are other ways to convert predicates into statements. One way is to use quantifiers. Quantifiers are words that refer to quantities such as ``some'' or ``all'' to tell people for how many elements make a predicate true.
\begin{definition}
    The \textbf{universal quantifier}\index{quantifier!universal} is $\forall$ and is read as ``for all''. A statement $Q$ of the form $\forall x \in D, P(x)$, where $P(x)$ is the predicate and $D$ is the domain, is called a \textbf{universal statement}\index{statement!universal}.
    \begin{itemize}
        \item $Q$ is true if and only if $P(x)$ is true for every $x$ in $D$.
        \item $Q$ is false if and only if $P(x)$ is false for at least one $x$ in $D$.
    \end{itemize}
    Values $x \in D$ for which $P(x)$ is false is called a \textbf{counterexample}\index{counterexample}.
\end{definition}
\begin{example}
    The true statement ``for every integer $n$, the integer $2n$ is an even number'' can be written as ``$\forall n \in \mathbb{Z}, 2n$ is even''.
\end{example}

\begin{definition}
    The \textbf{existential quantifier}\index{quantifier!existential} is $\exists$ and is read as ``there exists''. A statement $Q$ of the form $\exists x \in D \textrm{ such that } P(x)$, where $P(x)$ is the predicate and $D$ is the domain, is called a \textbf{existential statement}\index{statement!existential}.
    \begin{itemize}
        \item $Q$ is true if and only if $P(x)$ is true for at least one $x$ in $D$.
        \item $Q$ is false if and only if $P(x)$ is false for all $x$ in $D$.
    \end{itemize}
\end{definition}
\begin{example}
    The statement ``there exists a subset of $\mathbb{Z}$ which has 10 elements'' is equivalent to ``$\exists S \subseteq \mathbb{Z} \text{ such that } |S| = 10$''.
\end{example}
\begin{remark}
    We usually shorten the ``such that'' as ``s.t.'', so the above statement is written symbolically as ``$\exists S \subseteq \mathbb{Z} \text{ s.t. } |S| = 10$''.
\end{remark}

We can also combine quantifiers and logical operators together.
\begin{example}
    The statement ``every $\varepsilon > 0$ has a $\delta > 0$ such that $|x^2 - 4| < \varepsilon$ if $0 < |x - 2| < \delta$'' can be written as ``$\forall \varepsilon > 0,\;\exists \delta > 0 \text{ s.t. } (0 < |x - 2| < \delta \implies |x^2 - 4| < \varepsilon)$''.
\end{example}
\begin{example}
    Fermat's Last Theorem, which states that for all integers $n\geq 3$ the equation $x^n + y^n = z^n$ has no solution for $x, y, z \in \mathbb{N}$, can be written as
    \[
        \left((n \in \mathbb{Z}) \land (n \geq 3)\right) \implies \left(\forall x, y, z \in \mathbb{N},\; x^n + y^n \neq z^n\right).
    \]
    It should be noted that when translating from English to \textit{symbolic writing}, we may need to change some things around.
\end{example}

\begin{exercise}
    Convert the statement ``for all positive integers $n > 2$ there exist integers $a$ and $b$ such that $a^3 + b^4 = n^5$'' into symbolic notation using quantifiers and logical operators.
\end{exercise}

We now look at negating quantifiers\index{quantifier!negation}. Note that
\begin{itemize}
    \item $\lnot(\forall x, P(x)) \equiv \exists x \text{ s.t. } \lnot P(x)$; and
    \item $\lnot(\exists x \text{ s.t. } P(x)) \equiv \forall x, \lnot P(x)$.
\end{itemize}
\begin{example}
    Consider the statement ``for every integer $n$, the integer $2n$ is even''. One may write that using quantifiers as
    \[
        \forall n \in\mathbb{Z}, 2n \text{ is even}.
    \]
    Negating the above statement would make it
    \[
        \exists n \in \mathbb{Z}, 2n \text{ is not even},
    \]
    i.e., ``there exists an integer $n$ such that $2n$ is odd''.
\end{example}

We also note for statements $p$ and $q$ that $\lnot(p \implies q) \equiv p \land \lnot q$ by \myref{exercise-negation-of-implication}. This result is useful when negating conditional statements within quantified statements.

\begin{example}
    Consider the statement ``$n^3$ is odd if $n$ is odd for all integers $n$''. We may write this as
    \[
        \forall n \in \mathbb{Z}, (n \text{ is odd}) \implies (n^3 \text{ is odd}).
    \]
    The negation of such a statement would hence be
    \[
        \exists n \in \mathbb{Z} \text{ s.t. }((n \text{ is odd}) \land \lnot(n^3 \text{ is odd})),
    \]
    which in english would be ``there exists an integer $n$ such that $n$ is odd and $n^3$ is even''.
\end{example}

\begin{exercise}
    Consider the statement ``there exists a real number $y$ such that $xy = 1$ for all non-zero real numbers $x$''.
    \begin{partquestions}{\roman*}
        \item Write down two predicates $P(x)$ and $Q(x)$ in symbols such that the above statement is $\forall x,(P(x) \implies Q(x))$.
        \item Negate the above statement, writing your answer in symbolic notation.
    \end{partquestions}
\end{exercise}

\section{Hierarchy of Mathematical Results}
Mathematical statements often have a certain `tier' attached to them. We look at the hierarchy of some of these `tiers'.
\begin{itemize}
    \item \textbf{Axiom}\index{axiom}: A statement that is assumed to be true to act as a starting point for further reasoning.
    \item \textbf{Proposition}\index{proposition}: A proposition can be thought of as a general proven result. Equivalent names for a proposition are ``\textbf{Claim}'' or ``\textbf{Observation}''.
    \item \textbf{Lemma}\index{lemma}: A lemma can be thought of as a `small' proven result that can help build other mathematical results. For example, Euclid's lemma that ``if a prime $p$ divides the product $ab$ of two integers $a$ and $b$, then $p$ must divide at least one of those integers $a$ or $b$'' is used in the proof of the Fundamental Theorem of Arithmetic.
    \item \textbf{Theorem}\index{theorem}: A theorem can be thought of as a `big' proven result. For example, the Fundamental Theorem of Arithmetic is core to arithmetic as it describes how integers can be uniquely decomposed into its prime factors.
    \item \textbf{Corollary}\index{corollary}: A corollary is said to be a `follow-up result' from a theorem. These results usually follow `very quickly' from a theorem or a proposition. For example, the AM-GM inequality is a corollary of the Pythagorean theorem.
\end{itemize}

\newpage

\section{Problems}
\begin{problem}
    Which of the following are statements? Which are predicates?
    \begin{partquestions}{\alph*}
        \item 4 equals 5.
        \item $n \geq 4$.
        \item $8 = 2 \times 4$.
        \item $x^2 - 1 = (x-1)(x+1)$.
        \item There is a prime number $p$ that is one less than a cube and two less than a square.
        \item $P(x) \equiv P(x)$.
        \item $P(x) \iff P(x)$ for all $x$.
    \end{partquestions}
\end{problem}

\begin{problem}
    Assume $p$, $q$, and $r$ are statements. By considering a truth table,
    \begin{partquestions}{\alph*}
        \item is $(p \implies (q \implies r)) \equiv ((p \implies q) \implies r)$?
        \item is $(p \iff (q \iff r)) \equiv ((p \iff q) \iff r)$?
    \end{partquestions}
\end{problem}

\begin{problem}
    For the following, assume $p$ and $q$ are statements. Which of the following are true statements?
    \begin{multicols}{2}
        \begin{partquestions}{\alph*}
            \item $\forall p\; \forall q\; (p \equiv q)$.
            \item $\forall p\; \exists q\; (p \equiv q)$.
            \item $\exists p\; \forall q\; (p \equiv q)$.
            \item $\exists p\; \exists q\; (p \equiv q)$.
            \item $\forall p\; \forall q\; (p \not\equiv q)$.
            \item $\forall p\; \exists q\; (p \not\equiv q)$.
            \item $\exists p\; \forall q\; (p \not\equiv q)$.
            \item $\exists p\; \exists q\; (p \not\equiv q)$.
        \end{partquestions}
    \end{multicols}
\end{problem}

\begin{problem}
    Simplify $\lnot p \land (\lnot p \implies (p \land q))$.
\end{problem}
