\chapter{Sets}
Sets may be used to describe all of mathematics. All different kinds of mathematical structures may be described and explained using the notion of sets.

\section{What Are Sets?}
\begin{definition}
    A \textbf{set}\index{set} is a collection of things called \textbf{elements}\index{element} of the set.
\end{definition}
If $x$ is an element of the set $S$, we write $x \in S$. This is read as ``$x$ is an element of the set $S$'' or ``$x$ is in $S$''. Otherwise, we write $x \notin S$. For convenience, if $x \in S$ and $y \in S$, we may write $x, y \in S$. If $z \in S$ also, we may write $x, y, z \in S$. The same applies for ``$\notin$''.

A set is often described by listing elements separated by commas, or by a characterizing property of its elements, within braces $\{ \ \}$. 
\begin{example}
    The collection $\{2, 3, 4, 5\}$ is a set with 4 elements, namely 2, 3, 4, and 5.
\end{example}
\begin{example}
    The collection $\{\{1, 2, 3\}, \{4, 5\}, \{6, \{7\}\}\}$ is a set with 3 elements, which are also sets. Namely, it contains the set $\{1, 2, 3\}$, containing the elements 1, 2, and 3, the set $\{4, 5\}$, containing the elements 4 and 5, and the set $\{6, \{7\}\}$, containing the number 6 and the set $\{7\}$ which contains a single element 7.
\end{example}

Some sets may have infinitely many elements.
\begin{example}
    The integers $\{\dots, -3, -2, -1, 0, 1, 2, 3, \dots\}$ is a set with infinitely many elements. The dots indicate that the pattern of integers goes on forever in both directions.
\end{example}

\begin{definition}
    A set with a finite number of elements is called a \textbf{finite set}\index{set!finite}. A set with an infinite number of elements is called an \textbf{infinite set}\index{set!infinte}.
\end{definition}

\begin{definition}\index{set!equality}
    Two sets are equal if and only if they contain the same elements.
\end{definition}
\begin{example}
    The sets $A = \{1, 2, 3, 4\}$, $B = \{4, 3, 2, 1\}$, and $C = \{3, 4, 1, 2\}$ are equal to each other even though their elements are listed in a different order.
\end{example}
\begin{example}
    $\{1, 2, 3, 4\} \neq \{1, 2, 3, 5\}$ since the elements in the two sets differ.
\end{example}

What if a set has no elements?
\begin{definition}
    The \textbf{empty set}\index{set!empty} is the set $\{\}$ and is denoted by $\emptyset$. That is, $\emptyset = \{\}$.
\end{definition}

\newpage

We introduce the idea of subsets.
\begin{definition}
    Let $A$ and $B$ be sets.
    \begin{itemize}
        \item $A$ is a \textbf{subset}\index{subset} of $B$ if and only if all elements of $A$ are elements of $B$. This is denoted by $A \subseteq B$.
        \item $A$ is a \textbf{proper subset}\index{subset!proper} if and only if $A$ is a subset of $B$ and $B$ contains at least one element not in $A$. In this case, we write $A \subset B$.
    \end{itemize}
\end{definition}

\begin{example}
    Let $A = \{1, 2\}$, $B = \{1, 2, 3\}$, $C = \{1, 4\}$, and $S = \{1, 2, 3\}$. Then $A \subseteq S$ since the elements of $A$, namely 1 and 2, also appear in $S$. Also $B \subseteq S$ since all elements of $B$ appear in $S$. However $C \not\subseteq S$ since 4 is not an element of $S$.

    We note that $B \not\subset S$ since $S$ does not contain an element that is not in $B$. But $A \subset S$ since 3 is not in $A$.
\end{example}

\begin{exercise}
    Let $S$ be a non-empty set. Determine whether the following statements are true or false.

    \begin{multicols}{2}
        \begin{partquestions}{\alph*}
            \item $\{1, 2\} \subset \{1, 2, 3, 4\}$
            \item $\{1, 2, 3\} \subseteq \{1, 2, 4\}$
            \item $\emptyset \subseteq \emptyset$
            \item $S \subset S$
            \item $S \in \{S, \emptyset\}$
            \item $\{S\} \notin \{S, \emptyset\}$
            \item $S \subseteq \{S, \emptyset\}$
            \item $\{S\} \subseteq \{S, \emptyset\}$
        \end{partquestions}
    \end{multicols}
\end{exercise}

There are some special sets that are so common that we have given special names and symbols.
\begin{definition}
    The set of positive integers\index{set!of positive integers} is denoted $\mathbb{N}$.
\end{definition}
\begin{remark}
    Some authors denote the set containing the positive integers and 0 by $\mathbb{N}$. We use $\mathbb{N}$ to exclusively denote the positive integers here.
\end{remark}
\begin{definition}
    The set of integers \index{set!of integers} is denoted $\Z$.
\end{definition}
\begin{remark}
    The use of the blackboard ``Z'' (i.e., $\Z$) to denote the set of integers comes from the German ``Z\"{a}hlen'' which means ``numbers'' and is attributed to David Hilbert.
\end{remark}
\begin{definition}
    The set of rational numbers\index{set!of rational numbers} is denoted $\Q$.
\end{definition}
\begin{definition}
    The set of real numbers\index{set!of real numbers} is denoted $\R$.
\end{definition}

\newpage

\section{Set Operations}
Certain operations could be performed on sets. We list the most commonly used ones here.

\begin{definition}
    The \textbf{union}\index{set!union} of two sets is the set of all objects that are a member of $A$, or $B$, or both. It is denoted $A \cup B$.
\end{definition}
\begin{example}
    $\{1, 2, 3\} \cup \{2, 3, 4\} = \{1, 2, 3, 4\}$.
\end{example}

\begin{definition}
    The \textbf{intersection}\index{set!intersection} of two sets is the set of all objects that are a member of \textit{both} the sets $A$ and $B$. It is denoted $A \cap B$.
\end{definition}
\begin{example}
    $\{1, 2, 3\} \cap \{2, 3, 4\} = \{2, 3\}$
\end{example}

\begin{definition}
    The \textbf{set difference}\index{set!difference} of $S$ and $A$, denoted $S \setminus A$, is the set of all members of $S$ that are not members of $A$.
\end{definition}
\begin{remark}
    Some authors will use the minus sign to denote the set difference, i.e. $A - B$ denotes the difference of $A$ and $B$ and is the same as $A \setminus B$ in this book.
\end{remark}
\begin{example}
    Let $A = \{1, 2, 3\}$ and $B = \{2, 3, 4\}$. Then $A \setminus B = \{1\}$ and $B \setminus A = \{4\}$.
\end{example}

\begin{definition}
    The \textbf{cartesian product}\index{cartesian product} of $A$ and $B$, denoted $A \times B$, is the set whose members are all possible ordered pairs $(a, b)$, where $a$ is an element of $A$ and $b$ is an element of $B$.
\end{definition}
\begin{example}
    Let $A = \{1, 2, 3\}$ and $B = \{4, 5\}$. Then
    \[
        A \times B = \{(1, 4), (1, 5), (2, 4), (2, 5), (3, 4), (3, 5)\}.
    \]
\end{example}
\begin{remark}
    In particular, if $A$ is a set, the cartesian product $A \times A = A^2$, $A\times A \times A = A^3$, and so on.
\end{remark}

\begin{exercise}
    Let $S = \{1, 2, 3, 4\}$, $T = \{2, 3, 5\}$, $U = \{(2, 2), (3, 3), (5, 5)\}$. Determine whether the following statements are true or false.
    \begin{multicols}{2}
        \begin{partquestions}{\alph*}
            \item $S \cup T = \{1, 2, 3, 4, 5\}$
            \item $S \cup U = \{1, 2, 3, (5, 5)\}$
            \item $S \cap T = \{2, 3\}$
            \item $T \cap U = \emptyset$
            \item $S \setminus T = \{1, 4\}$
            \item $S \setminus \{1, 4\} = T$
            \item $T^2 = U$
            \item $U \subset (S \cup T)^2$
        \end{partquestions}
    \end{multicols}
\end{exercise}

\newpage

\section{Set-Builder Notation}
Sometimes, some sets are too big or complex to list between the braces. In these cases, we use set-builder notation to describe the sets.
\begin{definition}
    A set $S$ with \textbf{set-builder notation}\index{set!builder notation} has the syntax
    \[
        S = \{\mathrm{expression} \ | \ \mathrm{rule}\},
    \]
    where the elements of $S$ are all values of the expression that satisfy the rule.
\end{definition}

\begin{example}
    Consider the set of even integers, $E = \{\dots, -6, -4, -2, 0, 2, 4, 6, \dots\}$. It is often written as
    \[
        E = \{2n \vert n \in \Z\}.
    \]
    In this case, we may read the above expression as ``$E$ is the set of all things of form $2n$, where $n$ is an element of $\Z$''.

    There are equivalent forms of $E$:
    \[
        E = \{n \vert n \textrm{ is an even integer}\} = \{n \vert n = 2k, k \in \Z\}.
    \]

    Another common way of writing $E$ is
    \[
        E = \{n \in \Z \vert n \textrm{ is even}\}
    \]
    where it could be read as ``$E$ is the set of all $n$ in $\Z$ such that $n$ is even''.
\end{example}
\begin{remark}
    Some authors use the colon instead of a vertical line, for example they may write $S = \{\mathrm{expression} \ : \ \mathrm{rule}\}$.
\end{remark}

\begin{example}
    The set $S = \{x \in \R \ | \ x \geq 0 \}$ is the set of all non-negative real numbers.
\end{example}

We introduce the notation for intervals.
\begin{definition}
    An \textbf{interval}\index{interval} is a subset of the real numbers. In particular, given real numbers $a$ and $b$ where $a \leq b$:
    \begin{align*}
        (a,b) &= \{x \in \R \vert a < x < b\} & [a,b] &= \{x \in \R \vert a \leq x \leq b\}\\
        (a,b] &= \{x \in \R \vert a < x \leq b\} & [a,b) &= \{x \in \R \vert a \leq x < b\}.
    \end{align*}

    Infinity ($\infty$) can also be used as one of the bounds to denote an \textbf{unbounded interval}. In particular, for any real number $r$:
    \begin{align*}
        (r, \infty) &= \{x \in \R \vert x > r\} & [r, \infty) &= \{x \in \R \vert x \geq r\}\\
        (-\infty,r) &= \{x \in \R \vert x < r\} & (-\infty,r] &= \{x \in \R \vert x \leq r\}.
    \end{align*}
\end{definition}
\begin{example}
    The interval $I = [2, 5)$ is the set $\{x \in \R \vert 2 \leq x < 5\}$. Note $1 \notin I$, $2 \in I$, $3 \in I$, $4 \in I$, $5 \notin I$, and $6 \notin I$.
\end{example}
\begin{example}
    The interval $I = (2, \infty)$ is the set $\{x \in \R \vert x > 2\}$. Note $1 \notin I$, $2 \notin I$, $3 \in I$, and $\pi \in I$.
\end{example}

\section{Cardinality}
\begin{definition}
    The \textbf{cardinality}\index{cardinality} of a set $S$, denoted by $|S|$, is a measure of the number of elements of the set.
    \begin{itemize}
        \item If $S$ is finite, then $|S|$ is the number of elements in $S$.
        \item If $S$ is infinite, then we write $|S| = \infty$.
    \end{itemize}
\end{definition}
\begin{remark}
    Of course, the notion that $|S| = \infty$ is poorly defined in other contexts. However, for this book, this definition would be sufficient for most of the things we wish to accomplish.
\end{remark}
\begin{example}
    The set $A = \{1, 2, 3\}$ has cardinality 3, i.e. $|A| = 3$.
\end{example}
\begin{example}
    The empty set $\emptyset$ has cardinality 0 since it has no elements, i.e. $|\emptyset| = 0$.
\end{example}
\begin{example}
    The set of integers has infinite elements, so we write $|\Z| = \infty$.
\end{example}

\begin{exercise}
    Let the sets
    \begin{align*}
        S &= \{x \in \Q \vert x \in (-\infty, 0]\}\\
        T &= \{y \in \Z \vert y \in [-2, 10] \text{ and } y \text{ is an even number} \}
    \end{align*}
    List the elements in the set $S \cap T$.
\end{exercise}

\newpage

\section{Problems}
\begin{problem}
    Let $A$ and $B$ be finite sets with cardinality $n$. For what value(s) of $n$ can we be sure that $A = B$?
\end{problem}

\begin{problem}
    Let the sets
    \begin{align*}
        A &= \{x \in \R \vert x^2 - x - 2 \leq 0\},\\
        B &= \{x \in [0, \infty) \vert 12 - x - x^2 > 0\}.
    \end{align*}
    \begin{partquestions}{\alph*}
        \item Express $A \cap B$ in interval notation.
        \item Express $A \cup B$ in interval notation.
        \item Express $B \setminus A$ in interval notation.
        \item Is $A \setminus B \subset [-1, 0)$? Explain.
    \end{partquestions}
\end{problem}

\begin{problem}
    Let the sets
    \begin{align*}
        A &= \{(x,y) \in \Z^2 \vert 5x+2y+3=0\},\\
        B &= \{(x,y) \in [0,1]^2 \vert 2x^2+5x+2y+1=0\}.
    \end{align*}
    Find the cardinality of the following sets. If the cardinality is finite, list all elements of the set.
    \begin{partquestions}{\alph*}
        \item $A \cap B$
        \item $A \cup B$
    \end{partquestions}
\end{problem}

\begin{problem}
    Show that $A \cap (B \setminus C) = (A \cap B) \setminus C$ for any sets $A$, $B$, and $C$.
\end{problem}
