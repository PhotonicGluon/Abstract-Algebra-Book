\chapter{Algebra}
Competent algebraic manipulation in the real numbers is assumed. We collect a few important and useful concepts and results when we manipulate algebraic objects.

\section{Operations in the Real Numbers}
Operations in the real numbers are essential when we want to perform algebraic manipulations. We formally state some of their properties here. We note that these properties are really hard to prove directly since they are so ingrained in calculation; although a proof exists, it is too hard to reproduce them here. Thus we leave these properties as axioms: statements that are assumed to be true.

We first look at two important properties of addition over the real numbers.
\begin{axiom}\label{axiom-addition-is-commutative}\index{axiom!arithmetic!commutative addition}
    Addition is commutative. That is, for any real numbers $x$ and $y$ we have $x + y = y + x$.
\end{axiom}
\begin{example}
    Clearly adding 3 to 4 (i.e., $3 + 4$) is the same as adding 4 to 3 (i.e., $4 + 3$) through our everyday intuition of adding numbers together.
\end{example}

\begin{axiom}\label{axiom-addition-is-associative}\index{axiom!arithmetic!associative addition}
    Addition is associative. That is, for any real numbers $x$, $y$, and $z$ we have $x+(y+z) = (x+y)+z$.
\end{axiom}
\begin{example}
    We may evaluate the expression $5+(6+7)$ in two ways.
    \begin{itemize}
        \item We may first evaluate $6+7 = 13$, and then sum 5 to it to get $5 + 13 = 18$.
        \item Alternatively, we may use \myref{axiom-addition-is-associative} to rewrite $5+(6+7)$ to be $(5+6)+7$. Then we first evaluate $5+6 = 11$ and then add 7 to it, yielding $11 + 7 = 18$.
    \end{itemize}
\end{example}

Combining the two axioms for addition gives us multiple ways to evaluate expressions.
\begin{example}
    Consider the expression $(3 + 4) + (1 + 2)$.
    \begin{itemize}
        \item We could directly compute the values $3 + 4 = 7$, $1 + 2 = 3$, and then taking their sum to be $7 + 3 = 10$.
        \item Alternatively, via commutativity of addition (\myref{axiom-addition-is-commutative}) we can rewrite $(3 + 4) + (1 + 2) = (1+2)+(3+4)$ and then use associativity (\myref{axiom-addition-is-associative}) to further see that $(1+2)+(3+4) = ((1+2)+3)+4$. We can then compute the sum inside out, $1 + 2 = 3$, $3 + 3 = 6$, and finally $6 + 4 = 10$.
    \end{itemize}
\end{example}
\begin{example}
    Consider the expression $(x^4 + x^2) + (x^3 + x + 1)$. The two axioms above allow us to rewrite this as a nicer expression of $x^4 + x^3 + x^2 + x + 1$.
\end{example}

We now look at two properties of multiplication over the real numbers.
\begin{axiom}\label{axiom-multiplication-is-commutative}\index{axiom!arithmetic!commutative multiplication}
    Multiplication is commutative. That is, for any real numbers $x$ and $y$ we have $x \times y = y \times x$.
\end{axiom}
\begin{remark}
    This also applies to subsets of the real numbers, such as the integers.
\end{remark}
\begin{remark}
    We may also denote multiplication using the centred dot ($\cdot$), or suppress the operator. That is, $x\times y = x\cdot y = xy$.
\end{remark}
\begin{example}
    Clearly multiplying 3 by 4 (i.e., $3 \times 4$) is the same as multiplying 4 by 3 (i.e., $4 \times 3$) through our everyday intuition of adding numbers together.
\end{example}

\begin{axiom}\label{axiom-multiplication-is-associative}\index{axiom!arithmetic!associative multiplication}
    Multiplication is associative. That is, for any real numbers $x$, $y$, and $z$ we have $x\times(y\times z) = (x\times y)\times z$.
\end{axiom}
\begin{example}
    We may evaluate the expression $5\times(6\times7)$ in two ways.
    \begin{itemize}
        \item We may first evaluate $6\times7 = 42$, and then multiply 5 to it to get $5 \times 42 = 210$.
        \item Alternatively, we may use \myref{axiom-multiplication-is-associative} to rewrite $5\times(6\times7)$ to be $(5\times6)\times7$. Then we first evaluate $5\times6 = 30$ and then multiply 7 to it, yielding $30\times7 = 210$.
    \end{itemize}
\end{example}

Again, using the two axioms for multiplication gives us several ways to evaluate expressions.
\begin{example}
    Consider the expression $(3\times4)(1\times2)$.
    \begin{itemize}
        \item We could directly compute the values $3 \times 4 = 12$, $1 \times 2 = 2$, and then taking their product to be $12\times2 = 24$.
        \item Alternatively, via commutativity of multiplication (\myref{axiom-multiplication-is-commutative}) we can rewrite $(3\times4)(1\times2)=(1\times2)\times(3\times4)$ and then use associativity (\myref{axiom-multiplication-is-associative}) to further see that $(1\times2)\times(3\times4) = ((1\times2)\times3)\times4$. We can then compute the sum inside out, $1 \times 2 = 2$, $2 \times 3 = 6$, and finally $6 \times 4 = 24$.
    \end{itemize}
\end{example}
\begin{example}
    Consider the expression $(x^4\times x^2)(x^3 \times x \times 1)$. The two axioms above allow us to rewrite this as a nicer expression of $x^4 \times x^3 \times x^2 \times x \times 1 = x^{10}$.
\end{example}

We also note one property of combining addition and multiplication together.

\begin{axiom}\label{axiom-distributivity}\index{axiom!arithmetic!distributivity}
    Multiplication distributes over addition. That is, for any real numbers $x$, $y$, and $z$, we have
    \begin{itemize}
        \item $x\times(y+z) = (x\times y) + (x\times z)$; and
        \item $(x+y)\times z = (x\times z) + (y\times z)$.
    \end{itemize}
\end{axiom}

\begin{example}
    Consider the expression $4\times(5+6)$.
    \begin{itemize}
        \item We could first evaluate $5+6=11$ and then multiply 4 by that to yield $4\times11=44$.
        \item Alternatively, we can use \myref{axiom-distributivity} to write $4\times(5+6) = (4\times5)+(4\times6)$. Then we can evaluate each part separately, seeing that $4\times5 = 20$, $4\times6 = 24$, so $(4\times5)+(4\times6) = 20+24=44$.
    \end{itemize}
\end{example}
\begin{example}
    Consider the expression $(x+1)(x+2)$. We use the axioms to expand the expression.
    \begin{align*}
        (x+1)(x+2) &= ((x+1)x) + ((x+1)\times2) & (\myref{axiom-distributivity})\\
        &= ((x\times x) + (1\times x)) + ((x\times2) + (1\times2)) & (\myref{axiom-distributivity})\\
        &= (x^2+x) + (2x+2)\\
        &= x^2+(x + (2x+2)) & (\myref{axiom-addition-is-associative})\\
        &= x^2+((x + 2x)+2) & (\myref{axiom-addition-is-associative})\\
        &= x^2 + (3x+2)\\
        &= x^2 + 3x + 2.
    \end{align*}
\end{example}

We end this section by noting an axiom regarding the multiplicative inverse of a non-zero real number.
\begin{axiom}\label{axiom-reciprocal}\index{axiom!arithmetic!reciprocal}
    Every non-zero real number $x$ has a multiplicative inverse, called the \textbf{reciprocal}\index{reciprocal} of $x$ and denoted $\frac1x$, in the real numbers such that $x\left(\frac1x\right) = \left(\frac1x\right)x = 1$.
\end{axiom}

\section{Summation}
When we are summing many similar terms, we use the capital Greek letter sigma ($\sum$) to represent the sum of similar terms. 
\begin{definition}\index{summation}
    Define
    \[
        \sum_{i=m}^{n}a_i = a_m + a_{m+1} + a_{m+2} + \cdots + a_{n-1} + a_n,
    \]
    where $i$ is the \textbf{index of summation}\index{summation!index}, $a_i$ represents each \textbf{summand}\index{summation!summand} of the sum, $m$ is the \textbf{lower bound of summation}\index{summation!lower bound}, and $n$ is the \textbf{upper bound of summation}\index{summation!upper bound}.
\end{definition}
\begin{remark}
    $\displaystyle \sum_{i=m}^{n}a_i$ is read as ``the sum of $a_i$ from $i=m$ to $n$''.
\end{remark}
\begin{remark}
    The ``$i = m$'' at the bottom of the summation means that the index of summation starts at $m$, increments by 1 for each successive term, and continues until $i = n$.
\end{remark}

\begin{example}
    The sum of the first 10 positive integers can be written as
    \[
        \sum_{i=1}^{10}i = 1 + 2 + \cdots + 9 + 10
    \]
    and is evaluated to 55.
\end{example}

\begin{example}
    The sum of the squares from 3 to 9 can be written as
    \[
        \sum_{i=3}^{9}i^2 = 3^2 + 4^2 + \cdots + 8^2 + 9^2.
    \]
    There is no reason to just use ``$i$'' for the index; we may also use other letters, such as $j$:
    \[
        \sum_{j=3}^{9}j^2 = 3^2 + 4^2 + \cdots + 8^2 + 9^2.
    \]
\end{example}

\begin{example}
    The case where the summation has one summand is just equal to the summand itself. That is,
    \[
        \sum_{i=k}^{k}a_i = a_k.
    \]
    For instance,
    \[
        \sum_{i=4}^4(i+2)^2 = (4+2)^2 = 36.
    \]
\end{example}

There is one degenerate case for summation, which we note as a definition.
\begin{definition}
    Let $k$ and $n$ be integers such that $k > n$. Define
    \[
        \sum_{i=k}^{n}a_i = 0.
    \]
\end{definition}
\begin{example}
    We see
    \[
        \sum_{i=5}^{4}i = \sum_{i=3}^{2}1 = \sum_{i=12}^{3}(i^4 + 2i^3 + 3i^2 + 4i + 5) = 0
    \]
    by virtue of the above definition.
\end{example}
\begin{exercise}
    Evaluate the following sums.
    \begin{multicols}{3}
        \begin{partquestions}{\alph*}
            \item $\displaystyle \sum_{i=3}^{5}(7ix+11)$
            \item $\displaystyle \sum_{x=3}^{5}(7ix+11)$
            \item $\displaystyle \sum_{i=-3}^{-5}(-7ix-11)$
            \item $\displaystyle \sum_{i=4}^{8}ijk$
            \item $\displaystyle \sum_{j=4}^{8}ijk$
            \item $\displaystyle \sum_{n=4}^{8}ijk$
            \item $\displaystyle \sum_{i=3}^{7}13$
            \item $\displaystyle \sum_{i=1}^{3}i^2 + \sum_{j=4}^{6}j^2$
            \item $\displaystyle \sum_{i=1}^{3}\left(\sum_{j=5}^{7}(i+j)\right)$
        \end{partquestions}
    \end{multicols}
\end{exercise}

\newpage

We note some properties of summation.
\begin{itemize}
    \item $\displaystyle \sum_{i=m}^na_i = \sum_{j=m}^na_j$\hfill(Index renaming)
    \item $\displaystyle \sum_{i=m}^n(Ca_i) = C\sum_{i=m}^na_i$\hfill(Factor constants out of a sum)
    \item $\displaystyle \left(\sum_{i=m}^na_i\right) \pm \left(\sum_{i=m}^nb_i\right) = \sum_{i=m}^n(a_i \pm b_i)$\hfill(Break sum across sum or difference)
    \item $\displaystyle \sum_{i=m}^ka_i = \left(\sum_{i=m}^na_i\right) + \left(\sum_{j={n+1}}^ka_j\right)$\hfill(Splitting the sum)
    \item $\displaystyle \sum_{i=m}^na_i = \sum_{i=m+k}^{n+k}a_{i-k}$\hfill(Index shift)
\end{itemize}

\begin{exercise}
    Given $\displaystyle \sum_{i=1}^5a_i^2 = 100$, $\displaystyle \sum_{i=2}^5a_{i-1} = 10$, and $\displaystyle \sum_{i=1}^5(a_i+2)^2 = 200$, what is $a_5$?
\end{exercise}

We also note some examples for indexing by elements in a subset of the real numbers.
\begin{example}
    Let $A = \{1, 3, 5, 7\}$. Then
    \[
        \sum_{x \in A} x = 1 + 3 + 5 + 7 = 16.
    \]
    We also see that
    \[
        \sum_{x \in A}(3x + 5) = (3\times1 + 5) + (3\times3 + 5) + (3\times5 + 5) + (3\times7 + 5) = 68
    \]
    and
    \[
        \sum_{x\in A}x^2 = 1^2 + 3^2 + 5^2 + 7^2 = 84.
    \]
\end{example}
\begin{example}
    Let $S = \{1, \frac23, \frac45, 6\}$. Then
    \[
        \sum_{n \in S}n = 1 + \frac23 + \frac45 + 6 = \frac{127}{15}
    \]
    and
    \[
        \sum_{k \in S}k^2 = 1^2 + \left(\frac23\right)^2 + \left(\frac45\right)^2 + 6^2 = \frac{8569}{225}.
    \]
\end{example}

\newpage

\section{Common Expressions}
We look at some common expressions that appear in algebra.

\subsection{Factorial}
\begin{definition}
    The \textbf{factorial}\index{factorial} of a non-negative integer $n$ is
    \[
        n! = \begin{cases}
            1 & \text{if } n = 0\\
            n \times (n-1)! & \text{if } n > 0
        \end{cases}
    \]
\end{definition}

\begin{example}
    Clearly 0 factorial is 1, i.e. $0! = 1$.
\end{example}

\begin{example}
    We find 3 factorial.
    \begin{align*}
        3! &= 3 \times 2!\\
        &= 3 \times 2 \times 1!\\
        &= 3 \times 2 \times 1 \times 0!\\
        &= 3 \times 2  \times 1  \times 1\\
        &= 6.
    \end{align*}
\end{example}

\begin{example}
    We find 5 factorial.
    \begin{align*}
        5! &= 5 \times 4!\\
        &= 5 \times 4 \times 3!\\
        &= 5 \times 4 \times 3 \times 2!\\
        &= 5 \times 4 \times 3 \times 2 \times 1!\\
        &= 5 \times 4 \times 3 \times 2 \times 1 \times 0!\\
        &= 5 \times 4 \times 3 \times 2  \times 1  \times 1\\
        &= 120.
    \end{align*}
\end{example}

\subsection{Absolute Value}
\begin{definition}
    The \textbf{absolute value}\index{absolute value} of a real number $x$ is
    \[
        |x| = \begin{cases}
            -x & \text{if } x < 0\\
            x & \text{if } x \geq 0
        \end{cases}
    \]
\end{definition}

\begin{example}
    We see that $|5| = 5$, $|-5| = -(-5) = 5$, and $|0| = 0$.
\end{example}

\begin{example}
    We solve the equation $|x-3| = 4$.

    Note that $|x-3| = 4$ means that $x-3 = \pm 4$. Thus $x = 3 \pm 4$, meaning that $x = 7$ or $x = -1$.
\end{example}

We note some properties of the absolute value.
\begin{proposition}
    $|x| \geq 0$ for all real values $x$.
\end{proposition}
\begin{proof}
    Let $x$ be a real number. We split into two cases.
    \begin{itemize}
        \item If $x \geq 0$, then $|x| = x \geq 0$.
        \item If $x < 0$, then $-x > 0$. Thus $|x| = -x > 0$.
    \end{itemize}
    In any case, $|x| \geq 0$.
\end{proof}

\begin{proposition}
    $|x|^2 = x^2$ for all real values $x$.
\end{proposition}
\begin{proof}
    Supposing $x$ is a real value, we split into two cases.
    \begin{itemize}
        \item If $x \geq 0$, then $|x| = x$. Clearly one sees that $|x|^2 = (x)^2 = x^2$.
        \item If $x < 0$, then $|x| = -x$. Thus $|x|^2 = (-x)^2 = x^2$.
    \end{itemize}
    In any case, $|x|^2 = x^2$.
\end{proof}

\begin{proposition}
    $|x_1x_2\cdots x_n| = |x_1||x_2|\cdots|x_n|$ for all real values $x_1, x_2, \dots, x_n$.
\end{proposition}
\begin{proof}
    See \myref{problem-absolute-value-is-multiplicative-map} (later).
\end{proof}

\begin{example}
    We solve the equation $x^2 - 2|x| - 3 = 0$. We replace $x^2$ with $|x|^2$ to obtain $|x|^2 - 2|x| - 3 = 0$, i.e. $(|x|+1)(|x|-3) = 0$. Thus $|x| = -1$ or $|x| = 3$. Now $|x| = -1$ is impossible, so $|x| = 3$, meaning $x = 3$ or $x = -3$.
\end{example}

\begin{exercise}
    Solve the inequality $x^2 + 15 < 8|x|$.
\end{exercise}

\section{The Binomial Theorem}
Sometimes we are forced to expand expressions such as $(x-1)^5$, $(3x^2 + 5)^4$, and $(7x - 3)^9$. These expressions can be readily expanded using the \textbf{binomial theorem}\index{Bionomial Theorem}.
\begin{theorem}[Binomial Theorem]\label{thrm-binomial}
    Let $n$ be a non-negative integer. Then
    \[
        (x+y)^n = \sum_{k=0}^n {n \choose k}x^ky^{n-k} = \sum_{k=0}^n {n \choose k}x^{n-k}y^k
    \]
    where
    \[
        {n \choose k} = \frac{n!}{k!(n-k)!}.
    \]
\end{theorem}
We note that ${n \choose k}$ is read as ``$n$ choose $k$'' and is known as the \textbf{binomial coefficient}\index{binomial coefficient}.

\begin{example}
    $(x+1)^4 = x^4 + 4x^3 + 6x^2 + 4x + 1$.
\end{example}
\begin{example}
    $(x-1)^5 = x^5 - 5x^4 + 10x^3 - 10x^2 + 5x - 1$.
\end{example}
\begin{example}
    $(3x^2 + 5)^4 = 81x^8 + 540x^6 + 1350x^4 + 1500x^2 + 625$.
\end{example}

\begin{exercise}
    Find the coefficient of $x^6$ in $(7x-3)^9$.
\end{exercise}

We note two facts about the binomial coefficient here.
\begin{proposition}
    ${n\choose k} = {n\choose {k-n}}$ for all integers $0 \leq k \leq n$.
\end{proposition}
\begin{proof}
    One sees clearly that
    \begin{align*}
        {n\choose k} &= \frac{n!}{k!(n-k)!}\\
        &= \frac{n!}{(n-k)!k!}\\
        &= \frac{n!}{(n-k)!(n-(n-k))!}\\
        &= {n\choose {k-n}}
    \end{align*}
    which proves this proposition.
\end{proof}

\begin{proposition}\label{prop-binomial-coefficient-multiple-of-p}
    If $p$ is a prime, then $p\choose k$ is a multiple of $p$ for all integers $1 \leq k \leq p - 1$.
\end{proposition}
\begin{proof}
    Note that $p \choose k$ is an integer, and that
    \[
        {p \choose k} = \frac{p!}{k!(p-k)!} = \frac{p(p-1)(p-2)\cdots(p-k+1)}{k!}.
    \]
    Let $a = {p \choose k} \in \Z$, $b = p(p-1)(p-2)\cdots(p-k+1)$, and $c = k!$. Clearly $b$ is a multiple of $p$, and $c$ is not a multiple of $p$ (since $k < p$ thus $c = k!$ is not a multiple of $p$). Now if a prime divides a product $xy$, it must either divide $x$ or $y$. As $b = ac$ is a multiple of $p$ and $c$ is not a multiple of $p$, thus $a = {p \choose k}$ has to be a multiple of $p$.
\end{proof}

\newpage

\section{Problems}
\begin{problem}
    Suppose that commutativity of multiplication (\myref{axiom-multiplication-is-commutative}) is no longer true. Expand $(x+y)^3$.
\end{problem}

\begin{problem}
    By using the substitution $u = x^2 - 2x - 2$, find all $x \in \R$ such that
    \[
        x^6 + 6x^4 + 16x^3 + 19 = 6x^5 + 12x^2 + 24x.
    \]
\end{problem}

\begin{problem}
    Let $x \in [0,3]$.
    \begin{partquestions}{\roman*}
        \item Expand $(x-2)(x^2-3x+1)$.
        \item Hence find the range of values of $x$ such that
        \[
            \frac{4x^3 - 11x^2 - 47x + 106}{(x-2)(x^2-18)} \geq 3.
        \]
    \end{partquestions}
\end{problem}

\begin{problem}
    Solve the inequality
    \[
        \frac{3|x-3|}{x^2-6x+5} > 1
    \]
    given $x \geq 0$.
\end{problem}

\begin{problem}
    Let $N$ be a positive integer.
    \begin{partquestions}{\roman*}
        \item Express $\frac{7x+4}{x^3+3x^2+2x}$ in the form $\frac{A}{x} + \frac{B}{x+1} + \frac{C}{x+2}$, where $A$, $B$, and $C$ are real constants to be determined.
        \item Hence simplify
        \[
            \frac{7N+9}{N^2+3N+2} + \sum_{r=1}^N \frac{7r+4}{r^3+3r^2+2r}.
        \]
    \end{partquestions}
\end{problem}

\begin{problem}
    Prove that
    \[
        \sum_{r=0}^n (r\times r!) = (n+1)! - 1
    \]
    for all non-negative integers $n$.
\end{problem}

\begin{problem}\label{problem-absolute-value-is-multiplicative-map}
    Let $x_1, x_2, \dots, x_n$ be real numbers.
    \begin{partquestions}{\roman*}
        \item Prove that $|xy| = |x||y|$ for all real numbers $x$ and $y$.
        \item Prove that $|x_1x_2\cdots x_n| = |x_1||x_2|\cdots|x_n|$.
    \end{partquestions}
\end{problem}
