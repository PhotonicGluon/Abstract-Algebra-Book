\section{Modular Arithmetic}
\begin{questions}
    \item Note
    \begin{align*}
        5^{2n+3} &= 5^3 \times 5^{2n}\\
        &= 125 \times 25^n \\
        &\equiv (5) \times (1)^n & (\text{since } 25 \equiv 1 \pmod8 \text{ and } 125 \equiv 8 \pmod8)\\
        &= 5 \pmod8
    \end{align*}
    so $5^{2n+3} \equiv 5 \pmod8$ for all positive integers $n$.

    \item We induct on $n$.
    
    When $n = 1$, clearly $7^{3^1} = 7^3 = 343$ which is a multiple of 9.

    Now assume that $7^{3^k} \equiv 1 \pmod9$ for some positive integer $k$; we show that $7^{3^{k+1}} \equiv 1 \pmod9$ for $k + 1$.

    Note
    \begin{align*}
        7^{3^{k+1}} &= 7^{3\times3^k}\\
        &= \left(7^{3^k}\right)^3\\
        &\equiv 1^3 & (\text{by induction hypothesis})\\
        &= 1 \pmod9
    \end{align*}
    so the statement holds for $k+1$.

    Therefore by mathematical induction we see $7^{3^n} \equiv 1 \pmod9$ for all positive integers $n$.

    \item Consider any integer $x$; it has to have a remainder of 0, 1, 2, or 3 when divided by 4. Thus we consider 4 cases.
    \begin{itemize}
        \item If $x \equiv 0 \pmod4$, then $x^2 \equiv 0 \pmod4$.
        \item If $x \equiv 1 \pmod4$, then $x^2 \equiv 1 \pmod4$.
        \item If $x \equiv 2 \pmod4$, then $x^2 \equiv 4 \equiv 0 \pmod4$.
        \item If $x \equiv 3 \pmod4$, then $x^2 \equiv 9 \equiv 1 \pmod4$.
    \end{itemize}
    In any case, the remainder of any perfect square modulo 4 is either 0 or 1. If it is 1, then $x$ is odd; if it is 0 then $x$ is even. Thus if $x$ is an even perfect square then $x^2$ necessarily has to have a remainder 0 modulo 4, i.e. $x^2$ is a multiple of 4. Therefore no such even perfect square exists.

    \item We work modulo 1000 to find the last 3 digits of the number. We note that $57^1 = 57$, $57^2 = 3249 \equiv 249 \pmod{1000}$, $57^3 = 185193 \equiv 193 \pmod{1000}$, and $57^4 = 10556001 \equiv 1 \pmod{1000}$. We note that powers of 57 cycle between 57, 249, 193, and 1 when taking modulo 1000.
    
    Note $2023 = 4\times505+3$, so we see
    \begin{align*}
        57^{2023} &= 57^{4\times505+3}\\
        &= 57^3 \times \left(57^4\right)^{505}\\
        &\equiv 193 \times 1^{505}\\
        &= 193 \pmod{1000}
    \end{align*}
    which means $57^{2023}$ ends in 193.
\end{questions}
