\section{Elementary Number Theory}
\begin{questions}
    \item We know that $50a = 5 \times 50 = 750$ by \myref{prop-product-of-gcd-and-lcm}. Thus $a = 15$.
    
    \item Given that $\lcm(a,b) = a^2$, we use \myref{prop-product-of-gcd-and-lcm} to see that
    \[
        ab = \gcd(a,b)\lcm(a,b) = \gcd(a,b)a^2.
    \]
    Let $d = \gcd(a,b)$, so $ab = da^2$ which means $b = da$. Note $\lcm(a,b) = \lcm(a, da) = da$, so $a^2 = da$ which means $d = a$. Therefore $b = a^2$.

    \item Note
    \begin{align*}
        5^{2n+3} &= 5^3 \times 5^{2n}\\
        &= 125 \times 25^n \\
        &\equiv (5) \times (1)^n & (\text{since } 25 \equiv 1 \pmod8 \text{ and } 125 \equiv 8 \pmod8)\\
        &= 5 \pmod8
    \end{align*}
    so $5^{2n+3} \equiv 5 \pmod8$ for all positive integers $n$.
    
    \item Note that
    \begin{align*}
        (1+n)^n - 1 &= \left(1 + {n \choose 1}n + {n \choose 2}n^2 + \cdots + {n \choose n}n^n\right) - 1\\
        &= {n \choose 1}n + {n \choose 2}n^2 + \cdots + {n \choose n}n^n.
    \end{align*}
    From the second term onwards we see that they are all multiples of $n^2$; the first term is exactly $n^2$. Therefore $n^2 \vert (1+n)^n - 1$ for all positive integers $n$.

    \item We induct on $n$.
    
    When $n = 1$, clearly $7^{3^1} = 7^3 = 343$ which is a multiple of 9.

    Now assume that $7^{3^k} \equiv 1 \pmod9$ for some positive integer $k$; we show that $7^{3^{k+1}} \equiv 1 \pmod9$ for $k + 1$.

    Note
    \begin{align*}
        7^{3^{k+1}} &= 7^{3\times3^k}\\
        &= \left(7^{3^k}\right)^3\\
        &\equiv 1^3 & (\text{by induction hypothesis})\\
        &= 1 \pmod9
    \end{align*}
    so the statement holds for $k+1$.

    Therefore by mathematical induction we see $7^{3^n} \equiv 1 \pmod9$ for all positive integers $n$.

    \item Consider any integer $x$; it has to have a remainder of 0, 1, 2, or 3 when divided by 4. Thus we consider 4 cases.
    \begin{itemize}
        \item If $x \equiv 0 \pmod4$, then $x^2 \equiv 0 \pmod4$.
        \item If $x \equiv 1 \pmod4$, then $x^2 \equiv 1 \pmod4$.
        \item If $x \equiv 2 \pmod4$, then $x^2 \equiv 4 \equiv 0 \pmod4$.
        \item If $x \equiv 3 \pmod4$, then $x^2 \equiv 9 \equiv 1 \pmod4$.
    \end{itemize}
    In any case, the remainder of any perfect square modulo 4 is either 0 or 1. If it is 1, then $x$ is odd; if it is 0 then $x$ is even. Thus if $x$ is an even perfect square then $x^2$ necessarily has to have a remainder 0 modulo 4, i.e. $x^2$ is a multiple of 4. Therefore no such even perfect square exists.

    \item We work modulo 1000 to find the last 3 digits of the number. We note that $57^1 = 57$, $57^2 = 3249 \equiv 249 \pmod{1000}$, $57^3 = 185193 \equiv 193 \pmod{1000}$, and $57^4 = 10556001 \equiv 1 \pmod{1000}$. We note that powers of 57 cycle between 57, 249, 193, and 1 when taking modulo 1000.
    
    Note $2023 = 4\times505+3$, so we see
    \begin{align*}
        57^{2023} &= 57^{4\times505+3}\\
        &= 57^3 \times \left(57^4\right)^{505}\\
        &\equiv 193 \times 1^{505}\\
        &= 193 \pmod{1000}
    \end{align*}
    which means $57^{2023}$ ends in 193.

    \item \begin{partquestions}{\roman*}
        \item We induct on $n$.
        
        When $n = 1$ we see that $6 \vert 2(1)^3 + 3(1)^2 + (1) = 6$ so the statement holds for $n = 1$.

        Assume that the statement is true for some positive integer $k$, i.e. $6 \vert 2k^3 + 3k^2 + k$, meaning $2k^3 + 3k^2 + k = 6a$ for some integer $a$. We show that the statement also holds for $k + 1$, i.e. $6 \vert 2(k+1)^3 + 3(k+1)^2 + (k+1)$, meaning $2(k+1)^3 + 3(k+1)^2 + (k+1) = 6b$ for some integer $b$.

        Note
        \begin{align*}
            &2(k+1)^3 + 3(k+1)^2 + (k+1)\\
            &= 2(k^3+3k^2+3k+1) + 3(k^2+2k+1) + (k+1)\\
            &= 2k^3 + 9k^2 + 13k + 6\\
            &= (2k^3 + 3k^2 + k) + (12k + 6)\\
            &= 6a + 6(2k + 1)\\
            &= 6(a + 2k + 1)
        \end{align*}
        so $6 \vert 2(k+1)^3 + 3(k+1)^2 + (k+1)$.

        Therefore by mathematical induction we see that $6 \vert 2n^3 + 3n^2 + n$ for all positive integers $n$.
        
        \item We induct on positive integer values of $n$.
    
        When $n = 1$ clearly $12 \vert 1^4 - 1^2 = 0$.

        Assume that for some positive integer $k$ we have $12 \vert k^4 - k^2$, i.e. $k^4 - k^2 = 12a$ for some integer $a$. We show that the statement holds for $k+1$, i.e. $12 \vert (k+1)^4 - (k+1)^2$, meaning $(k+1)^4 - (k+1)^2 = 12b$ for some integer $b$.

        Note
        \begin{align*}
            &(k+1)^4 - (k+1)^2\\
            &= (k^4 + 4k^3 + 6k^2 + 4k + 1) - (k^2 + 2k + 1)\\
            &= k^4 + 4k^3 + 5k^2 + 2k\\
            &= (k^4 - k^2) + (4k^3 + 6k^2 + 2k)\\
            &= 12a + 2(2k^3 + 3k^2 + 1k) & (\text{by induction hypothesis})\\
            &= 12a + 2(6m) & (\text{by \textbf{(i)}})\\
            &= 12a + 12m\\
            &= 12(a+m)
        \end{align*}
        so $12 \vert (k+1)^4 - (k+1)^2$.

        Therefore by mathematical induction we see that $12 \vert n^4 - n^2$ for all positive values of $n$.

        Now $12 \vert 0^4 - 0^2 = 0$, so it also works for $n = 0$.

        Finally, since $n^4 - n^2 = (-n)^4 - (-n)^2$, this also holds for negative values of $n$.
    \end{partquestions}

    \item Since $n$ and $a$ are coprime, therefore there exist integers $\lambda$ and $\mu$ such that
    \[
        \lambda n + \mu a = 1
    \]
    by B\'ezout's lemma (\myref{lemma-bezout}). Multiplying both sides by $b$ yields
    \[
        \lambda nb + \mu ab = b.
    \]
    Since $n \vert ab$ by assumption, thus $ab = kn$ for some integer $k$. Therefore
    \[
        \lambda nb + \mu kn = n(\lambda b + \mu k) = b.
    \]
    which means that $b$ is divisible by $n$ as required.
    
    \item \begin{partquestions}{\roman*}
        \item By B\'{e}zout's Lemma (\myref{lemma-bezout}) we can find integers $x$ and $y$ such that $ax + by = \gcd(a, b)$. For brevity let $d = \gcd(a,b)$, so $ax+by=d$.
    
        We take the non-obvious step of considering $d^3$. We see
        \[
            d^3 = (ax+by)^3 = a^3x^3 + 3a^2bx^2y + 3ab^2xy^2 + b^3y^3.
        \]
        By definition of the GCD, we know $d$ divides both $a$ and $b$, so $\frac ad$ and $\frac bd$ are integers. So by dividing both sides of the above equality by $d$ we see
        \begin{align*}
            d^2 &= \frac{a^3x^3}{d} + \frac{3a^2bx^2y}{d} + \frac{3ab^2xy^2}{d} + \frac{b^3y^3}{d}\\
            &= a^2\underbrace{\left(\frac{a}{d}x^3 + 3\frac{b}{d}x^2y\right)}_{\text{an integer}} + b^2\underbrace{\left(3\frac{a}{d}xy^2 + \frac{b}{d}y^3\right)}_{\text{an integer}}.
        \end{align*}
        Note B\'{e}zout's Lemma tells us that integers of the form $a^2\lambda + b^2\mu$ (where $\lambda$ and $\mu$ are integers) are multiples of $\gcd(a^2, b^2)$. Therefore $d^2$ is a multiple of $\gcd(a^2, b^2)$, i.e. $\gcd(a^2, b^2) \vert d^2$.

        Also as $d$ divides both $a$ and $b$, thus $d^2$ divides both $a^2$ and $b^2$. Therefore we see that $d^2 \vert \gcd(a^2, b^2)$ by \myref{prop-gcd-divides-common-divisor}.

        Therefore, as $\gcd(a^2, b^2) \vert d^2$ and $d^2 \vert \gcd(a^2, b^2)$, one obtains the fact that $d^2 = (\gcd(a,b))^2 = \gcd(a^2, b^2)$.

        \item We see that
        \begin{align*}
            \lcm(a^2,b^2) &= \frac{a^2b^2}{\gcd(a^2,b^2)} & (\text{by } \myref{prop-product-of-gcd-and-lcm})\\
            &= \frac{a^2b^2}{(\gcd(a,b))^2}\\
            &= \frac{ab}{\gcd(a,b)} \times \frac{ab}{\gcd(a,b)}\\
            &= \lcm(a,b) \times \lcm(a,b) & (\text{by } \myref{prop-product-of-gcd-and-lcm})\\
            &= (\lcm(a,b))^2
        \end{align*}
        so $\lcm(a^2,b^2) = (\lcm(a,b))^2$.
    \end{partquestions}
\end{questions}
